%
% API Documentation for API Documentation
% Module xformslib.flformbrowser
%
% Generated by epydoc 3.0.1
% [Fri May 21 15:38:49 2010]
%

%%%%%%%%%%%%%%%%%%%%%%%%%%%%%%%%%%%%%%%%%%%%%%%%%%%%%%%%%%%%%%%%%%%%%%%%%%%
%%                          Module Description                           %%
%%%%%%%%%%%%%%%%%%%%%%%%%%%%%%%%%%%%%%%%%%%%%%%%%%%%%%%%%%%%%%%%%%%%%%%%%%%

    \index{xformslib \textit{(package)}!xformslib.flformbrowser \textit{(module)}|(}
\section{Module xformslib.flformbrowser}

    \label{xformslib:flformbrowser}

xforms-python's functions to manage formbrowser objects.

Copyright (C) 2009, 2010  Luca Lazzaroni ``LukenShiro''
e-mail: <\href{mailto:lukenshiro@ngi.it}{lukenshiro@ngi.it}>

This program is free software: you can redistribute it and/or modify
it under the terms of the GNU Lesser General Public License as
published by the Free Software Foundation, version 2.1 of the License.

This program is distributed in the hope that it will be useful,
but WITHOUT ANY WARRANTY; without even the implied warranty of
MERCHANTABILITY or FITNESS FOR A PARTICULAR PURPOSE. See the
GNU Lesser General Public License for more details.

You should have received a copy of the GNU LGPL along with this
program. If not, see <\href{http://www.gnu.org/licenses/}{http://www.gnu.org/licenses/}>.

See CREDITS file to read acknowledgements and thanks to XForms,
ctypes and other developers.

%%%%%%%%%%%%%%%%%%%%%%%%%%%%%%%%%%%%%%%%%%%%%%%%%%%%%%%%%%%%%%%%%%%%%%%%%%%
%%                               Functions                               %%
%%%%%%%%%%%%%%%%%%%%%%%%%%%%%%%%%%%%%%%%%%%%%%%%%%%%%%%%%%%%%%%%%%%%%%%%%%%

  \subsection{Functions}

    \label{xformslib:flformbrowser:fl_addto_formbrowser}
    \index{xformslib \textit{(package)}!xformslib.flformbrowser \textit{(module)}!xformslib.flformbrowser.fl\_addto\_formbrowser \textit{(function)}}

    \vspace{0.5ex}

\hspace{.8\funcindent}\begin{boxedminipage}{\funcwidth}

    \raggedright \textbf{fl\_addto\_formbrowser}(\textit{pFlObject}, \textit{pFlForm})

    \vspace{-1.5ex}

    \rule{\textwidth}{0.5\fboxrule}
\setlength{\parskip}{2ex}

Populates a formbrowser. The form so added is appended to the list of
forms that are already in the formbrowser. Form should be valid for the
duration of the formbrowser and the application program should not destroy
a form that is added to a formbrowser before deleting the form from the
formbrowser first.

-{}-
\setlength{\parskip}{1ex}
      \textbf{Parameters}
      \vspace{-1ex}

      \begin{quote}
        \begin{Ventry}{xxxxxxxxx}

          \item[pFlObject]


formbrowser object
            {\it (type=pointer to xfdata.FL\_OBJECT)}

          \item[pFlForm]


form
            {\it (type=pointer to xfdata.FL\_FORM)}

        \end{Ventry}

      \end{quote}

      \textbf{Return Value}
    \vspace{-1ex}

      \begin{quote}

total number of forms in the formbrowser
      {\it (type=int)}

      \end{quote}

\textbf{Note:} 
e.g. ntotfrms = fl\_addto\_formbrowser(frmbrobj, newform)


\textbf{Status:} 
Tested + Doc + NoDemo = OK


    \end{boxedminipage}

    \label{xformslib:flformbrowser:fl_delete_formbrowser_bynumber}
    \index{xformslib \textit{(package)}!xformslib.flformbrowser \textit{(module)}!xformslib.flformbrowser.fl\_delete\_formbrowser\_bynumber \textit{(function)}}

    \vspace{0.5ex}

\hspace{.8\funcindent}\begin{boxedminipage}{\funcwidth}

    \raggedright \textbf{fl\_delete\_formbrowser\_bynumber}(\textit{pFlObject}, \textit{seqnum})

    \vspace{-1.5ex}

    \rule{\textwidth}{0.5\fboxrule}
\setlength{\parskip}{2ex}

Removes a form from the formbrowser using a sequence number, an
integer between 1 and the number of forms in the browser. After a
form is removed, the sequence numbers are re-adjusted so they are
always consecutive.

-{}-
\setlength{\parskip}{1ex}
      \textbf{Parameters}
      \vspace{-1ex}

      \begin{quote}
        \begin{Ventry}{xxxxxxxxx}

          \item[pFlObject]


formbrowser object
            {\it (type=pointer to xfdata.FL\_OBJECT)}

          \item[seqnum]


sequence number of form to be removed
            {\it (type=int)}

        \end{Ventry}

      \end{quote}

      \textbf{Return Value}
    \vspace{-1ex}

      \begin{quote}

removed form (pFlForm) or None (on failure)
      {\it (type=pointer to xfdata.FL\_FORM)}

      \end{quote}

\textbf{Note:} 
e.g. delfrm = fl\_delete\_formbrowser\_bynumber(frmbrobj, 2)


\textbf{Status:} 
Tested + Doc + NoDemo = OK


    \end{boxedminipage}

    \label{xformslib:flformbrowser:fl_delete_formbrowser}
    \index{xformslib \textit{(package)}!xformslib.flformbrowser \textit{(module)}!xformslib.flformbrowser.fl\_delete\_formbrowser \textit{(function)}}

    \vspace{0.5ex}

\hspace{.8\funcindent}\begin{boxedminipage}{\funcwidth}

    \raggedright \textbf{fl\_delete\_formbrowser}(\textit{pFlObject}, \textit{pFlForm})

    \vspace{-1.5ex}

    \rule{\textwidth}{0.5\fboxrule}
\setlength{\parskip}{2ex}

Removes a specified form from the formbrowser.

-{}-
\setlength{\parskip}{1ex}
      \textbf{Parameters}
      \vspace{-1ex}

      \begin{quote}
        \begin{Ventry}{xxxxxxxxx}

          \item[pFlObject]


formbrowser object
            {\it (type=pointer to xfdata.FL\_OBJECT)}

          \item[pFlForm]


form candidate to deletion
            {\it (type=pointer to xfdata.FL\_FORM)}

        \end{Ventry}

      \end{quote}

      \textbf{Return Value}
    \vspace{-1ex}

      \begin{quote}

current (after deletion) number of forms in the formbrowser
or -1 (on failure)
      {\it (type=int)}

      \end{quote}

\textbf{Note:} 
e.g. num = fl\_delete\_formbrowser(frmbrobj, p2ndform)


\textbf{Status:} 
Tested + Doc + NoDemo = OK


    \end{boxedminipage}

    \label{xformslib:flformbrowser:fl_replace_formbrowser}
    \index{xformslib \textit{(package)}!xformslib.flformbrowser \textit{(module)}!xformslib.flformbrowser.fl\_replace\_formbrowser \textit{(function)}}

    \vspace{0.5ex}

\hspace{.8\funcindent}\begin{boxedminipage}{\funcwidth}

    \raggedright \textbf{fl\_replace\_formbrowser}(\textit{pFlObject}, \textit{seqnum}, \textit{pFlForm})

    \vspace{-1.5ex}

    \rule{\textwidth}{0.5\fboxrule}
\setlength{\parskip}{2ex}

Replaces a form in formbrowser specified by a sequence number

-{}-
\setlength{\parskip}{1ex}
      \textbf{Parameters}
      \vspace{-1ex}

      \begin{quote}
        \begin{Ventry}{xxxxxxxxx}

          \item[pFlObject]


formbrowser object
            {\it (type=pointer to xfdata.FL\_OBJECT)}

          \item[seqnum]


sequence number of form to be replaced
            {\it (type=int)}

          \item[pFlForm]


form used as replacement
            {\it (type=pointer to xfdata.FL\_FORM)}

        \end{Ventry}

      \end{quote}

      \textbf{Return Value}
    \vspace{-1ex}

      \begin{quote}

form that has been replaced (pFlForm), or None (on failure)
      {\it (type=pointer to xfdata.FL\_FORM)}

      \end{quote}

\textbf{Note:} 
e.g. replfrm = fl\_replace\_formbrowser(frmbrobj, 4, newreplfrm)


\textbf{Status:} 
Tested + Doc + NoDemo = OK


    \end{boxedminipage}

    \label{xformslib:flformbrowser:fl_insert_formbrowser}
    \index{xformslib \textit{(package)}!xformslib.flformbrowser \textit{(module)}!xformslib.flformbrowser.fl\_insert\_formbrowser \textit{(function)}}

    \vspace{0.5ex}

\hspace{.8\funcindent}\begin{boxedminipage}{\funcwidth}

    \raggedright \textbf{fl\_insert\_formbrowser}(\textit{pFlObject}, \textit{seqnum}, \textit{pFlForm})

    \vspace{-1.5ex}

    \rule{\textwidth}{0.5\fboxrule}
\setlength{\parskip}{2ex}

Inserts a form into a formbrowser at arbitrary location.

-{}-
\setlength{\parskip}{1ex}
      \textbf{Parameters}
      \vspace{-1ex}

      \begin{quote}
        \begin{Ventry}{xxxxxxxxx}

          \item[pFlObject]


formbrowser object
            {\it (type=pointer to xfdata.FL\_OBJECT)}

          \item[seqnum]


the sequence number before which the new form form is to be inserted
into the formbrowser
            {\it (type=int)}

          \item[pFlForm]


new form to insert
            {\it (type=pointer to xfdata.FL\_FORM)}

        \end{Ventry}

      \end{quote}

      \textbf{Return Value}
    \vspace{-1ex}

      \begin{quote}

number of forms in the formbrowser, or -1 (on failure)
      {\it (type=int)}

      \end{quote}

\textbf{Note:} 
e.g. frmsnum = fl\_insert\_formbrowser(frmbrobj, 5, pform)


\textbf{Status:} 
Tested + Doc + NoDemo = OK


    \end{boxedminipage}

    \label{xformslib:flformbrowser:fl_get_formbrowser_area}
    \index{xformslib \textit{(package)}!xformslib.flformbrowser \textit{(module)}!xformslib.flformbrowser.fl\_get\_formbrowser\_area \textit{(function)}}

    \vspace{0.5ex}

\hspace{.8\funcindent}\begin{boxedminipage}{\funcwidth}

    \raggedright \textbf{fl\_get\_formbrowser\_area}(\textit{pFlObject})

    \vspace{-1.5ex}

    \rule{\textwidth}{0.5\fboxrule}
\setlength{\parskip}{2ex}

Obtains the actual size of the forms area. The area occupied by the
formbrowser contains the space for the scrollbars.

-{}-
\setlength{\parskip}{1ex}
      \textbf{Parameters}
      \vspace{-1ex}

      \begin{quote}
        \begin{Ventry}{xxxxxxxxx}

          \item[pFlObject]


formbrowser object
            {\it (type=pointer to xfdata.FL\_OBJECT)}

        \end{Ventry}

      \end{quote}

      \textbf{Return Value}
    \vspace{-1ex}

      \begin{quote}

1 or 0 (on failure), horizontal (x), vertical position (y),
width (w) and height (h)
      {\it (type=int, int, int, int)}

      \end{quote}

\textbf{Note:} 
e.g. exval, x, y, w, h = fl\_get\_formbrowser\_area(frmbrobj)


\textbf{Attention:} 
API change from XForms - upstream was
fl\_get\_formbrowser\_area(pFlObject, x, y, w, h)


\textbf{Status:} 
Tested + Doc + NoDemo = OK


    \end{boxedminipage}

    \label{xformslib:flformbrowser:fl_set_formbrowser_scroll}
    \index{xformslib \textit{(package)}!xformslib.flformbrowser \textit{(module)}!xformslib.flformbrowser.fl\_set\_formbrowser\_scroll \textit{(function)}}

    \vspace{0.5ex}

\hspace{.8\funcindent}\begin{boxedminipage}{\funcwidth}

    \raggedright \textbf{fl\_set\_formbrowser\_scroll}(\textit{pFlObject}, \textit{how})

    \vspace{-1.5ex}

    \rule{\textwidth}{0.5\fboxrule}
\setlength{\parskip}{2ex}

Changes the vertical scrollbar so each action of the scrollbar scrolls
to the next forms. By default it scrolls a fixed number of pixels.

-{}-
\setlength{\parskip}{1ex}
      \textbf{Parameters}
      \vspace{-1ex}

      \begin{quote}
        \begin{Ventry}{xxxxxxxxx}

          \item[pFlObject]


formbrowser object
            {\it (type=pointer to xfdata.FL\_OBJECT)}

          \item[how]


How it scrolls. Values (from xfdata module) FL\_SMOOTH\_SCROLL (default)
or FL\_JUMP\_SCROLL
            {\it (type=int)}

        \end{Ventry}

      \end{quote}

\textbf{Note:} 
e.g. fl\_set\_formbrowser\_scroll(frmbrobj, xfdata.FL\_JUMP\_SCROLL)


\textbf{Status:} 
Tested + Doc + NoDemo = OK


    \end{boxedminipage}

    \label{xformslib:flformbrowser:fl_set_formbrowser_hscrollbar}
    \index{xformslib \textit{(package)}!xformslib.flformbrowser \textit{(module)}!xformslib.flformbrowser.fl\_set\_formbrowser\_hscrollbar \textit{(function)}}

    \vspace{0.5ex}

\hspace{.8\funcindent}\begin{boxedminipage}{\funcwidth}

    \raggedright \textbf{fl\_set\_formbrowser\_hscrollbar}(\textit{pFlObject}, \textit{how})

    \vspace{-1.5ex}

    \rule{\textwidth}{0.5\fboxrule}
\setlength{\parskip}{2ex}

Controls the presence of horizontal scrollbar. By default, if the size
of the forms exceeds the size of the formbrowser, scrollbars are added
automatically.

-{}-
\setlength{\parskip}{1ex}
      \textbf{Parameters}
      \vspace{-1ex}

      \begin{quote}
        \begin{Ventry}{xxxxxxxxx}

          \item[pFlObject]


formbrowser object
            {\it (type=pointer to xfdata.FL\_OBJECT)}

          \item[how]


if scrollbar is added or not. Values (from xfdata module) FL\_ON,
FL\_OFF, FL\_AUTO
            {\it (type=int)}

        \end{Ventry}

      \end{quote}

\textbf{Note:} 
e.g. fl\_set\_formbrowser\_hscrollbar(frmbrobj, FL\_OFF)


\textbf{Status:} 
Tested + Doc + NoDemo = OK


    \end{boxedminipage}

    \label{xformslib:flformbrowser:fl_set_formbrowser_vscrollbar}
    \index{xformslib \textit{(package)}!xformslib.flformbrowser \textit{(module)}!xformslib.flformbrowser.fl\_set\_formbrowser\_vscrollbar \textit{(function)}}

    \vspace{0.5ex}

\hspace{.8\funcindent}\begin{boxedminipage}{\funcwidth}

    \raggedright \textbf{fl\_set\_formbrowser\_vscrollbar}(\textit{pFlObject}, \textit{how})

    \vspace{-1.5ex}

    \rule{\textwidth}{0.5\fboxrule}
\setlength{\parskip}{2ex}

Controls the presence of vertical scrollbar. By default, if the size of
the forms exceeds the size of the formbrowser, scrollbars are added
automatically

-{}-
\setlength{\parskip}{1ex}
      \textbf{Parameters}
      \vspace{-1ex}

      \begin{quote}
        \begin{Ventry}{xxxxxxxxx}

          \item[pFlObject]


formbrowser object
            {\it (type=pointer to xfdata.FL\_OBJECT)}

          \item[how]


if scrollbar is added or not. Values (from xfdata module) FL\_ON,
FL\_OFF, FL\_AUTO
            {\it (type=int)}

        \end{Ventry}

      \end{quote}

\textbf{Note:} 
e.g. fl\_set\_formbrowser\_vscrollbar(frmbrobj, FL\_OFF)


\textbf{Status:} 
Tested + Doc + NoDemo = OK


    \end{boxedminipage}

    \label{xformslib:flformbrowser:fl_get_formbrowser_topform}
    \index{xformslib \textit{(package)}!xformslib.flformbrowser \textit{(module)}!xformslib.flformbrowser.fl\_get\_formbrowser\_topform \textit{(function)}}

    \vspace{0.5ex}

\hspace{.8\funcindent}\begin{boxedminipage}{\funcwidth}

    \raggedright \textbf{fl\_get\_formbrowser\_topform}(\textit{pFlObject})

    \vspace{-1.5ex}

    \rule{\textwidth}{0.5\fboxrule}
\setlength{\parskip}{2ex}

Obtains the form that is currently the first form in the formbrowser
visible to the user.

-{}-
\setlength{\parskip}{1ex}
      \textbf{Parameters}
      \vspace{-1ex}

      \begin{quote}
        \begin{Ventry}{xxxxxxxxx}

          \item[pFlObject]


formbrowser object
            {\it (type=pointer to xfdata.FL\_OBJECT)}

        \end{Ventry}

      \end{quote}

      \textbf{Return Value}
    \vspace{-1ex}

      \begin{quote}

first visible form (pFlForm)
      {\it (type=pointer to xfdata.FL\_FORM)}

      \end{quote}

\textbf{Note:} 
e.g. pform = fl\_get\_formbrowser\_topform(frmbrobj)


\textbf{Status:} 
Tested + Doc + NoDemo = OK


    \end{boxedminipage}

    \label{xformslib:flformbrowser:fl_set_formbrowser_topform}
    \index{xformslib \textit{(package)}!xformslib.flformbrowser \textit{(module)}!xformslib.flformbrowser.fl\_set\_formbrowser\_topform \textit{(function)}}

    \vspace{0.5ex}

\hspace{.8\funcindent}\begin{boxedminipage}{\funcwidth}

    \raggedright \textbf{fl\_set\_formbrowser\_topform}(\textit{pFlObject}, \textit{pFlForm})

    \vspace{-1.5ex}

    \rule{\textwidth}{0.5\fboxrule}
\setlength{\parskip}{2ex}

Sets which form to show by setting the top form.

-{}-
\setlength{\parskip}{1ex}
      \textbf{Parameters}
      \vspace{-1ex}

      \begin{quote}
        \begin{Ventry}{xxxxxxxxx}

          \item[pFlObject]


formbrowser object
            {\it (type=pointer to xfdata.FL\_OBJECT)}

          \item[pFlForm]


form
            {\it (type=pointer to xfdata.FL\_FORM)}

        \end{Ventry}

      \end{quote}

      \textbf{Return Value}
    \vspace{-1ex}

      \begin{quote}

sequence number of the form (seqnum)
      {\it (type=int)}

      \end{quote}

\textbf{Note:} 
e.g. frmid = fl\_set\_formbrowser\_topform(frmbrobj, pform)


\textbf{Status:} 
Tested + Doc + NoDemo = OK


    \end{boxedminipage}

    \label{xformslib:flformbrowser:fl_set_formbrowser_topform_bynumber}
    \index{xformslib \textit{(package)}!xformslib.flformbrowser \textit{(module)}!xformslib.flformbrowser.fl\_set\_formbrowser\_topform\_bynumber \textit{(function)}}

    \vspace{0.5ex}

\hspace{.8\funcindent}\begin{boxedminipage}{\funcwidth}

    \raggedright \textbf{fl\_set\_formbrowser\_topform\_bynumber}(\textit{pFlObject}, \textit{seqnum})

    \vspace{-1.5ex}

    \rule{\textwidth}{0.5\fboxrule}
\setlength{\parskip}{2ex}

Sets which form to show by setting the top form, using a sequence
number.

-{}-
\setlength{\parskip}{1ex}
      \textbf{Parameters}
      \vspace{-1ex}

      \begin{quote}
        \begin{Ventry}{xxxxxxxxx}

          \item[pFlObject]


formbrowser object
            {\it (type=pointer to xfdata.FL\_OBJECT)}

          \item[seqnum]


sequence number of form
            {\it (type=int)}

        \end{Ventry}

      \end{quote}

      \textbf{Return Value}
    \vspace{-1ex}

      \begin{quote}

new top form (pFlForm)
      {\it (type=pointer to xfdata.FL\_FORM)}

      \end{quote}

\textbf{Note:} 
e.g. pform = fl\_set\_formbrowser\_topform\_bynumber(frmbrobj, 2)


\textbf{Status:} 
Tested + Doc + NoDemo = OK


    \end{boxedminipage}

    \label{xformslib:flformbrowser:fl_set_formbrowser_xoffset}
    \index{xformslib \textit{(package)}!xformslib.flformbrowser \textit{(module)}!xformslib.flformbrowser.fl\_set\_formbrowser\_xoffset \textit{(function)}}

    \vspace{0.5ex}

\hspace{.8\funcindent}\begin{boxedminipage}{\funcwidth}

    \raggedright \textbf{fl\_set\_formbrowser\_xoffset}(\textit{pFlObject}, \textit{offset})

    \vspace{-1.5ex}

    \rule{\textwidth}{0.5\fboxrule}
\setlength{\parskip}{2ex}

Scrolls within a formbrowser in horizontal direction.

-{}-
\setlength{\parskip}{1ex}
      \textbf{Parameters}
      \vspace{-1ex}

      \begin{quote}
        \begin{Ventry}{xxxxxxxxx}

          \item[pFlObject]


formbrowser object
            {\it (type=pointer to xfdata.FL\_OBJECT)}

          \item[offset]


positive number, measuring in pixels the offset from the natural
position from the left. 0 indicates the natural position of the
content within the formbrowser.
            {\it (type=int)}

        \end{Ventry}

      \end{quote}

      \textbf{Return Value}
    \vspace{-1ex}

      \begin{quote}

num.
      {\it (type=int)}

      \end{quote}

\textbf{Note:} 
e.g. num = fl\_set\_formbrowser\_xoffset(frmbrobj, 15)


\textbf{Status:} 
Tested + Doc + NoDemo = OK


    \end{boxedminipage}

    \label{xformslib:flformbrowser:fl_set_formbrowser_yoffset}
    \index{xformslib \textit{(package)}!xformslib.flformbrowser \textit{(module)}!xformslib.flformbrowser.fl\_set\_formbrowser\_yoffset \textit{(function)}}

    \vspace{0.5ex}

\hspace{.8\funcindent}\begin{boxedminipage}{\funcwidth}

    \raggedright \textbf{fl\_set\_formbrowser\_yoffset}(\textit{pFlObject}, \textit{offset})

    \vspace{-1.5ex}

    \rule{\textwidth}{0.5\fboxrule}
\setlength{\parskip}{2ex}

Scrolls within a formbrowser in vertical direction.

-{}-
\setlength{\parskip}{1ex}
      \textbf{Parameters}
      \vspace{-1ex}

      \begin{quote}
        \begin{Ventry}{xxxxxxxxx}

          \item[pFlObject]


formbrowser object
            {\it (type=pointer to xfdata.FL\_OBJECT)}

          \item[offset]


positive number, measuring in pixels the offset from the natural
position from the top. 0 indicates the natural position of the
content within the formbrowser
            {\it (type=int)}

        \end{Ventry}

      \end{quote}

      \textbf{Return Value}
    \vspace{-1ex}

      \begin{quote}

num.
      {\it (type=int)}

      \end{quote}

\textbf{Note:} 
e.g. num = fl\_set\_formbrowser\_yoffset(frmbrobj, 15)


\textbf{Status:} 
Tested + Doc + NoDemo = OK


    \end{boxedminipage}

    \label{xformslib:flformbrowser:fl_get_formbrowser_xoffset}
    \index{xformslib \textit{(package)}!xformslib.flformbrowser \textit{(module)}!xformslib.flformbrowser.fl\_get\_formbrowser\_xoffset \textit{(function)}}

    \vspace{0.5ex}

\hspace{.8\funcindent}\begin{boxedminipage}{\funcwidth}

    \raggedright \textbf{fl\_get\_formbrowser\_xoffset}(\textit{pFlObject})

    \vspace{-1.5ex}

    \rule{\textwidth}{0.5\fboxrule}
\setlength{\parskip}{2ex}

Returns the current horizontal offset from left in pixel of a
formbrowser.

-{}-
\setlength{\parskip}{1ex}
      \textbf{Parameters}
      \vspace{-1ex}

      \begin{quote}
        \begin{Ventry}{xxxxxxxxx}

          \item[pFlObject]


formbrowser object
            {\it (type=pointer to xfdata.FL\_OBJECT)}

        \end{Ventry}

      \end{quote}

      \textbf{Return Value}
    \vspace{-1ex}

      \begin{quote}

horizontal offset
      {\it (type=int)}

      \end{quote}

\textbf{Note:} 
e.g. xoffset = fl\_get\_formbrowser\_xoffset(frmbrobj)


\textbf{Status:} 
Tested + Doc + NoDemo = OK


    \end{boxedminipage}

    \label{xformslib:flformbrowser:fl_get_formbrowser_yoffset}
    \index{xformslib \textit{(package)}!xformslib.flformbrowser \textit{(module)}!xformslib.flformbrowser.fl\_get\_formbrowser\_yoffset \textit{(function)}}

    \vspace{0.5ex}

\hspace{.8\funcindent}\begin{boxedminipage}{\funcwidth}

    \raggedright \textbf{fl\_get\_formbrowser\_yoffset}(\textit{pFlObject})

    \vspace{-1.5ex}

    \rule{\textwidth}{0.5\fboxrule}
\setlength{\parskip}{2ex}

Returns the current vertical offset from top in pixel of a
formbrowser.

-{}-
\setlength{\parskip}{1ex}
      \textbf{Parameters}
      \vspace{-1ex}

      \begin{quote}
        \begin{Ventry}{xxxxxxxxx}

          \item[pFlObject]


formbrowser object
            {\it (type=pointer to xfdata.FL\_OBJECT)}

        \end{Ventry}

      \end{quote}

      \textbf{Return Value}
    \vspace{-1ex}

      \begin{quote}

vertical offset
      {\it (type=int)}

      \end{quote}

\textbf{Note:} 
e.g. yoffset = fl\_get\_formbrowser\_yoffset(frmbrobj)


\textbf{Status:} 
Tested + Doc + NoDemo = OK


    \end{boxedminipage}

    \label{xformslib:flformbrowser:fl_find_formbrowser_form_number}
    \index{xformslib \textit{(package)}!xformslib.flformbrowser \textit{(module)}!xformslib.flformbrowser.fl\_find\_formbrowser\_form\_number \textit{(function)}}

    \vspace{0.5ex}

\hspace{.8\funcindent}\begin{boxedminipage}{\funcwidth}

    \raggedright \textbf{fl\_find\_formbrowser\_form\_number}(\textit{pFlObject}, \textit{pFlForm})

    \vspace{-1.5ex}

    \rule{\textwidth}{0.5\fboxrule}
\setlength{\parskip}{2ex}

Finds out the sequence number of a particular form.

-{}-
\setlength{\parskip}{1ex}
      \textbf{Parameters}
      \vspace{-1ex}

      \begin{quote}
        \begin{Ventry}{xxxxxxxxx}

          \item[pFlObject]


formbrowser object
            {\it (type=pointer to xfdata.FL\_OBJECT)}

          \item[pFlForm]


form candidate to be found
            {\it (type=pointer to xfdata.FL\_FORM)}

        \end{Ventry}

      \end{quote}

      \textbf{Return Value}
    \vspace{-1ex}

      \begin{quote}

sequence number of form (seqnum)
      {\it (type=int)}

      \end{quote}

\textbf{Note:} 
e.g. frmid = fl\_find\_formbrowser\_form\_number(frmbrobj, pform)


\textbf{Status:} 
Tested + Doc + NoDemo = OK


    \end{boxedminipage}

    \label{xformslib:flformbrowser:fl_add_formbrowser}
    \index{xformslib \textit{(package)}!xformslib.flformbrowser \textit{(module)}!xformslib.flformbrowser.fl\_add\_formbrowser \textit{(function)}}

    \vspace{0.5ex}

\hspace{.8\funcindent}\begin{boxedminipage}{\funcwidth}

    \raggedright \textbf{fl\_add\_formbrowser}(\textit{frmbrwstype}, \textit{x}, \textit{y}, \textit{w}, \textit{h}, \textit{label})

    \vspace{-1.5ex}

    \rule{\textwidth}{0.5\fboxrule}
\setlength{\parskip}{2ex}

Adds a formbrowser object.

-{}-
\setlength{\parskip}{1ex}
      \textbf{Parameters}
      \vspace{-1ex}

      \begin{quote}
        \begin{Ventry}{xxxxxxxxxxx}

          \item[frmbrwstype]


type of formbrowser to be added. Values (from xfdata.py)
FL\_NORMAL\_FORMBROWSER
            {\it (type=int)}

          \item[x]


horizontal position (upper-left corner)
            {\it (type=int)}

          \item[y]


vertical position (upper-left corner)
            {\it (type=int)}

          \item[w]


width in coord units
            {\it (type=int)}

          \item[h]


height in coord units
            {\it (type=int)}

          \item[label]


text label of formbrowser
            {\it (type=str)}

        \end{Ventry}

      \end{quote}

      \textbf{Return Value}
    \vspace{-1ex}

      \begin{quote}

formbrowser object added (pFlObject)
      {\it (type=pointer to xfdata.FL\_OBJECT)}

      \end{quote}

\textbf{Note:} 
e.g. frmbrobj = fl\_add\_formbrowser(xfdata.FL\_NORMAL\_FORMBROWSER,
110, 60, 550, 750, ``My Formbrowser)


\textbf{Status:} 
Tested + Doc + NoDemo = OK


    \end{boxedminipage}

    \label{xformslib:flformbrowser:fl_get_formbrowser_numforms}
    \index{xformslib \textit{(package)}!xformslib.flformbrowser \textit{(module)}!xformslib.flformbrowser.fl\_get\_formbrowser\_numforms \textit{(function)}}

    \vspace{0.5ex}

\hspace{.8\funcindent}\begin{boxedminipage}{\funcwidth}

    \raggedright \textbf{fl\_get\_formbrowser\_numforms}(\textit{pFlObject})

    \vspace{-1.5ex}

    \rule{\textwidth}{0.5\fboxrule}
\setlength{\parskip}{2ex}

Obtains the total number of forms in a formbrowser object.

-{}-
\setlength{\parskip}{1ex}
      \textbf{Parameters}
      \vspace{-1ex}

      \begin{quote}
        \begin{Ventry}{xxxxxxxxx}

          \item[pFlObject]


formbrowser object
            {\it (type=pointer to xfdata.FL\_OBJECT)}

        \end{Ventry}

      \end{quote}

      \textbf{Return Value}
    \vspace{-1ex}

      \begin{quote}

number of forms in formbrowser
      {\it (type=int)}

      \end{quote}

\textbf{Note:} 
e.g. frmsnum = fl\_get\_formbrowser\_numforms(frmbrobj)


\textbf{Status:} 
Tested + Doc + NoDemo = OK


    \end{boxedminipage}

    \label{xformslib:flformbrowser:fl_get_formbrowser_form}
    \index{xformslib \textit{(package)}!xformslib.flformbrowser \textit{(module)}!xformslib.flformbrowser.fl\_get\_formbrowser\_form \textit{(function)}}

    \vspace{0.5ex}

\hspace{.8\funcindent}\begin{boxedminipage}{\funcwidth}

    \raggedright \textbf{fl\_get\_formbrowser\_form}(\textit{pFlObject}, \textit{seqnum})

    \vspace{-1.5ex}

    \rule{\textwidth}{0.5\fboxrule}
\setlength{\parskip}{2ex}

Obtains the form handle from the sequence number.

-{}-
\setlength{\parskip}{1ex}
      \textbf{Parameters}
      \vspace{-1ex}

      \begin{quote}
        \begin{Ventry}{xxxxxxxxx}

          \item[pFlObject]


formbrowser object
            {\it (type=pointer to xfdata.FL\_OBJECT)}

          \item[seqnum]


sequence number of the form
            {\it (type=int)}

        \end{Ventry}

      \end{quote}

      \textbf{Return Value}
    \vspace{-1ex}

      \begin{quote}

form (pFlForm)
      {\it (type=pointer to xfdata.FL\_FORM)}

      \end{quote}

\textbf{Note:} 
e.g. pform = fl\_get\_formbrowser\_form(frmbrobj, 2)


\textbf{Status:} 
Tested + Doc + NoDemo = OK


    \end{boxedminipage}

    \index{xformslib \textit{(package)}!xformslib.flformbrowser \textit{(module)}|)}
