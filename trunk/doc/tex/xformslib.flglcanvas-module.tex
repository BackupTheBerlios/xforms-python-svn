%
% API Documentation for API Documentation
% Module xformslib.flglcanvas
%
% Generated by epydoc 3.0.1
% [Thu May 20 23:16:42 2010]
%

%%%%%%%%%%%%%%%%%%%%%%%%%%%%%%%%%%%%%%%%%%%%%%%%%%%%%%%%%%%%%%%%%%%%%%%%%%%
%%                          Module Description                           %%
%%%%%%%%%%%%%%%%%%%%%%%%%%%%%%%%%%%%%%%%%%%%%%%%%%%%%%%%%%%%%%%%%%%%%%%%%%%

    \index{xformslib \textit{(package)}!xformslib.flglcanvas \textit{(module)}|(}
\section{Module xformslib.flglcanvas}

    \label{xformslib:flglcanvas}

xforms-python's functions to manage GLcanvas objects.

Copyright (C) 2009, 2010  Luca Lazzaroni ``LukenShiro''
e-mail: <\href{mailto:lukenshiro@ngi.it}{lukenshiro@ngi.it}>

This program is free software: you can redistribute it and/or modify
it under the terms of the GNU Lesser General Public License as
published by the Free Software Foundation, version 2.1 of the License.

This program is distributed in the hope that it will be useful,
but WITHOUT ANY WARRANTY; without even the implied warranty of
MERCHANTABILITY or FITNESS FOR A PARTICULAR PURPOSE. See the
GNU Lesser General Public License for more details.

You should have received a copy of the GNU LGPL along with this
program. If not, see <\href{http://www.gnu.org/licenses/}{http://www.gnu.org/licenses/}>.

See CREDITS file to read acknowledgements and thanks to XForms,
ctypes and other developers.

%%%%%%%%%%%%%%%%%%%%%%%%%%%%%%%%%%%%%%%%%%%%%%%%%%%%%%%%%%%%%%%%%%%%%%%%%%%
%%                               Functions                               %%
%%%%%%%%%%%%%%%%%%%%%%%%%%%%%%%%%%%%%%%%%%%%%%%%%%%%%%%%%%%%%%%%%%%%%%%%%%%

  \subsection{Functions}

    \label{xformslib:flglcanvas:fl_add_glcanvas}
    \index{xformslib \textit{(package)}!xformslib.flglcanvas \textit{(module)}!xformslib.flglcanvas.fl\_add\_glcanvas \textit{(function)}}

    \vspace{0.5ex}

\hspace{.8\funcindent}\begin{boxedminipage}{\funcwidth}

    \raggedright \textbf{fl\_add\_glcanvas}(\textit{canvastype}, \textit{x}, \textit{y}, \textit{w}, \textit{h}, \textit{label})

    \vspace{-1.5ex}

    \rule{\textwidth}{0.5\fboxrule}
\setlength{\parskip}{2ex}

Adds a glcanvas object to the form.

-{}-
\setlength{\parskip}{1ex}
      \textbf{Parameters}
      \vspace{-1ex}

      \begin{quote}
        \begin{Ventry}{xxxxxxxxxx}

          \item[canvastype]


type of glcanvas to be added. Values (from xfdata.py) FL\_NORMAL\_CANVAS,
FL\_SCROLLED\_CANVAS (not enabled)
            {\it (type=int)}

          \item[x]


horizontal position (upper-left corner)
            {\it (type=int)}

          \item[y]


vertical position (upper-left corner)
            {\it (type=int)}

          \item[w]


width in coord units
            {\it (type=int)}

          \item[h]


height in coord units
            {\it (type=int)}

          \item[label]


text label of glcanvas
            {\it (type=str)}

        \end{Ventry}

      \end{quote}

      \textbf{Return Value}
    \vspace{-1ex}

      \begin{quote}

glcanvas object added (pFlObject)
      {\it (type=pointer to xfdata.FL\_OBJECT)}

      \end{quote}

\textbf{Note:} 
e.g. fl\_add\_glcanvas(xfdata.FL\_NORMAL\_CANVAS, 14, 21, 654, 457,
``My Gl Canvas'')


\textbf{Status:} 
Tested + Doc + NoDemo = OK


    \end{boxedminipage}

    \label{xformslib:flglcanvas:fl_set_glcanvas_defaults}
    \index{xformslib \textit{(package)}!xformslib.flglcanvas \textit{(module)}!xformslib.flglcanvas.fl\_set\_glcanvas\_defaults \textit{(function)}}

    \vspace{0.5ex}

\hspace{.8\funcindent}\begin{boxedminipage}{\funcwidth}

    \raggedright \textbf{fl\_set\_glcanvas\_defaults}(\textit{config})

    \vspace{-1.5ex}

    \rule{\textwidth}{0.5\fboxrule}
\setlength{\parskip}{2ex}

Modifies the global default attributes for glcanvas, before the
creation of glcanvases.

-{}-
\setlength{\parskip}{1ex}
      \textbf{Parameters}
      \vspace{-1ex}

      \begin{quote}
        \begin{Ventry}{xxxxxx}

          \item[config]


configuration settings. Attributes are those defined in OpenGL
glXChooseVisual() function \emph{todo}
            {\it (type=int)}

        \end{Ventry}

      \end{quote}

\textbf{Note:} 
e.g. fl\_set\_glcanvas\_defaults(?)


\textbf{Status:} 
Untested + NoDoc + NoDemo = NOT OK


    \end{boxedminipage}

    \label{xformslib:flglcanvas:fl_get_glcanvas_defaults}
    \index{xformslib \textit{(package)}!xformslib.flglcanvas \textit{(module)}!xformslib.flglcanvas.fl\_get\_glcanvas\_defaults \textit{(function)}}

    \vspace{0.5ex}

\hspace{.8\funcindent}\begin{boxedminipage}{\funcwidth}

    \raggedright \textbf{fl\_get\_glcanvas\_defaults}()

    \vspace{-1.5ex}

    \rule{\textwidth}{0.5\fboxrule}
\setlength{\parskip}{2ex}

Returns the global defaults attributes for glcanvas.

-{}-
\setlength{\parskip}{1ex}
      \textbf{Return Value}
    \vspace{-1ex}

      \begin{quote}

configuration settings
      {\it (type=int)}

      \end{quote}

\textbf{Note:} 
e.g. cnfset = fl\_get\_glcanvas\_defaults()


\textbf{Attention:} 
API change from XForms - upstream was
fl\_get\_glcanvas\_defaults(config)


\textbf{Status:} 
Tested + Doc + NoDemo = OK


    \end{boxedminipage}

    \label{xformslib:flglcanvas:fl_set_glcanvas_attributes}
    \index{xformslib \textit{(package)}!xformslib.flglcanvas \textit{(module)}!xformslib.flglcanvas.fl\_set\_glcanvas\_attributes \textit{(function)}}

    \vspace{0.5ex}

\hspace{.8\funcindent}\begin{boxedminipage}{\funcwidth}

    \raggedright \textbf{fl\_set\_glcanvas\_attributes}(\textit{pFlObject}, \textit{config})

    \vspace{-1.5ex}

    \rule{\textwidth}{0.5\fboxrule}
\setlength{\parskip}{2ex}

Modifies the default configuration of a particular glcanvas
object. You can change a glcanvas attribute on the fly even if
the canvas is already visible and active.

-{}-
\setlength{\parskip}{1ex}
      \textbf{Parameters}
      \vspace{-1ex}

      \begin{quote}
        \begin{Ventry}{xxxxxxxxx}

          \item[pFlObject]


glcanvas object
            {\it (type=pointer to xfdata.FL\_OBJECT)}

          \item[config]


configuration settings to be set. Attributes are those defined in
OpenGL glXChooseVisual() function \emph{todo}
            {\it (type=int)}

        \end{Ventry}

      \end{quote}

\textbf{Note:} 
e.g. fl\_set\_glcanvas\_attributes(glcanobj, ?)


\textbf{Status:} 
Untested + NoDoc + NoDemo = NOT OK


    \end{boxedminipage}

    \label{xformslib:flglcanvas:fl_get_glcanvas_attributes}
    \index{xformslib \textit{(package)}!xformslib.flglcanvas \textit{(module)}!xformslib.flglcanvas.fl\_get\_glcanvas\_attributes \textit{(function)}}

    \vspace{0.5ex}

\hspace{.8\funcindent}\begin{boxedminipage}{\funcwidth}

    \raggedright \textbf{fl\_get\_glcanvas\_attributes}(\textit{pFlObject})

    \vspace{-1.5ex}

    \rule{\textwidth}{0.5\fboxrule}
\setlength{\parskip}{2ex}

Returns the attributes of a glcanvas object.

-{}-
\setlength{\parskip}{1ex}
      \textbf{Parameters}
      \vspace{-1ex}

      \begin{quote}
        \begin{Ventry}{xxxxxxxxx}

          \item[pFlObject]


glcanvas object
            {\it (type=pointer to xfdata.FL\_OBJECT)}

        \end{Ventry}

      \end{quote}

      \textbf{Return Value}
    \vspace{-1ex}

      \begin{quote}

glcanvas attributes
      {\it (type=int)}

      \end{quote}

\textbf{Note:} 
e.g. fl\_get\_glcanvas\_attributes(glcanobj)


\textbf{Attention:} 
API change from XForms - upstream was
fl\_get\_glcanvas\_attributes(pFlObject, attributes)


\textbf{Status:} 
Tested + Doc + NoDemo = OK


    \end{boxedminipage}

    \label{xformslib:flglcanvas:fl_set_glcanvas_direct}
    \index{xformslib \textit{(package)}!xformslib.flglcanvas \textit{(module)}!xformslib.flglcanvas.fl\_set\_glcanvas\_direct \textit{(function)}}

    \vspace{0.5ex}

\hspace{.8\funcindent}\begin{boxedminipage}{\funcwidth}

    \raggedright \textbf{fl\_set\_glcanvas\_direct}(\textit{pFlObject}, \textit{yesno})

    \vspace{-1.5ex}

    \rule{\textwidth}{0.5\fboxrule}
\setlength{\parskip}{2ex}

Changes the rendering context created by a glcanvas. By default it
uses direct rendering (i.e. by-passing the Xserver).

-{}-
\setlength{\parskip}{1ex}
      \textbf{Parameters}
      \vspace{-1ex}

      \begin{quote}
        \begin{Ventry}{xxxxxxxxx}

          \item[pFlObject]


glcanvas object
            {\it (type=pointer to xfdata.FL\_OBJECT)}

          \item[yesno]


flag to use direct or through-Xserver rendering. Values 0 (to use
Xserver rendering) or 1 (to use direct rendering)
            {\it (type=int)}

        \end{Ventry}

      \end{quote}

\textbf{Note:} 
e.g. fl\_set\_glcanvas\_direct(glcanobj, 0)


\textbf{Status:} 
Tested + Doc + NoDemo = OK


    \end{boxedminipage}

    \label{xformslib:flglcanvas:fl_activate_glcanvas}
    \index{xformslib \textit{(package)}!xformslib.flglcanvas \textit{(module)}!xformslib.flglcanvas.fl\_activate\_glcanvas \textit{(function)}}

    \vspace{0.5ex}

\hspace{.8\funcindent}\begin{boxedminipage}{\funcwidth}

    \raggedright \textbf{fl\_activate\_glcanvas}(\textit{pFlObject})

    \vspace{-1.5ex}

    \rule{\textwidth}{0.5\fboxrule}
\setlength{\parskip}{2ex}

Activates a glcanvas object before drawing into glcanvas object. OpenGL
drawing routines always draw into the window the current context is bound
to. For application with a single canvas, this is not a problem. In case
of multiple canvases, the canvas driver takes care of setting the proper
context before invoking the expose handler. In some cases, the
application may want to draw into canvases actively. In this case, use
this function for explicit drawing context switching.

-{}-
\setlength{\parskip}{1ex}
      \textbf{Parameters}
      \vspace{-1ex}

      \begin{quote}
        \begin{Ventry}{xxxxxxxxx}

          \item[pFlObject]


glcanvas object
            {\it (type=pointer to xfdata.FL\_OBJECT)}

        \end{Ventry}

      \end{quote}

\textbf{Note:} 
e.g. fl\_activate\_glcanvas(glcanobj)


\textbf{Status:} 
Tested + Doc + NoDemo = OK


    \end{boxedminipage}

    \label{xformslib:flglcanvas:fl_get_glcanvas_xvisualinfo}
    \index{xformslib \textit{(package)}!xformslib.flglcanvas \textit{(module)}!xformslib.flglcanvas.fl\_get\_glcanvas\_xvisualinfo \textit{(function)}}

    \vspace{0.5ex}

\hspace{.8\funcindent}\begin{boxedminipage}{\funcwidth}

    \raggedright \textbf{fl\_get\_glcanvas\_xvisualinfo}(\textit{pFlObject})

    \vspace{-1.5ex}

    \rule{\textwidth}{0.5\fboxrule}
\setlength{\parskip}{2ex}

Obtains the XVisual information that is used to create the context of
a glcanvas object.

-{}-
\setlength{\parskip}{1ex}
      \textbf{Parameters}
      \vspace{-1ex}

      \begin{quote}
        \begin{Ventry}{xxxxxxxxx}

          \item[pFlObject]


glcanvas object
            {\it (type=pointer to xfdata.FL\_OBJECT)}

        \end{Ventry}

      \end{quote}

      \textbf{Return Value}
    \vspace{-1ex}

      \begin{quote}

XVisualInfo instance class
      {\it (type=pointer to xfdata.XVisualInfo)}

      \end{quote}

\textbf{Note:} 
e.g. pxviscls = fl\_get\_glcanvas\_xvisualinfo(glcanobj)


\textbf{Status:} 
Tested + Doc + NoDemo = OK


    \end{boxedminipage}

    \label{xformslib:flglcanvas:fl_get_glcanvas_context}
    \index{xformslib \textit{(package)}!xformslib.flglcanvas \textit{(module)}!xformslib.flglcanvas.fl\_get\_glcanvas\_context \textit{(function)}}

    \vspace{0.5ex}

\hspace{.8\funcindent}\begin{boxedminipage}{\funcwidth}

    \raggedright \textbf{fl\_get\_glcanvas\_context}(\textit{pFlObject})

    \vspace{-1.5ex}

    \rule{\textwidth}{0.5\fboxrule}
\setlength{\parskip}{2ex}

Returns GLXContext of a glcanvas object.

-{}-
\setlength{\parskip}{1ex}
      \textbf{Parameters}
      \vspace{-1ex}

      \begin{quote}
        \begin{Ventry}{xxxxxxxxx}

          \item[pFlObject]


glcanvas object
            {\it (type=pointer to xfdata.FL\_OBJECT)}

        \end{Ventry}

      \end{quote}

      \textbf{Return Value}
    \vspace{-1ex}

      \begin{quote}

glxcontext class instance
      {\it (type=pointer to xfdata.GLXContext)}

      \end{quote}

\textbf{Note:} 
e.g. glxcont = fl\_get\_glcanvas\_context(glcanobj)


\textbf{Status:} 
Tested + Doc + NoDemo = OK


    \end{boxedminipage}

    \label{xformslib:flglcanvas:fl_glwincreate}
    \index{xformslib \textit{(package)}!xformslib.flglcanvas \textit{(module)}!xformslib.flglcanvas.fl\_glwincreate \textit{(function)}}

    \vspace{0.5ex}

\hspace{.8\funcindent}\begin{boxedminipage}{\funcwidth}

    \raggedright \textbf{fl\_glwincreate}(\textit{config}, \textit{glxcontext}, \textit{w}, \textit{h})

    \vspace{-1.5ex}

    \rule{\textwidth}{0.5\fboxrule}
\setlength{\parskip}{2ex}

Creates a toplevel OpenGl window.

-{}-
\setlength{\parskip}{1ex}
      \textbf{Parameters}
      \vspace{-1ex}

      \begin{quote}
        \begin{Ventry}{xxxxxxxxxx}

          \item[config]


GL configuration settings
            {\it (type=int)}

          \item[glxcontext]


glxcontext class instance
            {\it (type=xfdata.GLXContext instance)}

          \item[w]


width of GL window in coord units
            {\it (type=int)}

          \item[h]


height of GL window in coord units
            {\it (type=int)}

        \end{Ventry}

      \end{quote}

      \textbf{Return Value}
    \vspace{-1ex}

      \begin{quote}

window created (win)
      {\it (type=long\_pos)}

      \end{quote}

\textbf{Note:} 
e.g. gwin0 = fl\_glwincreate(?, ?, 650, 560)


\textbf{Status:} 
Tested + Doc + NoDemo = OK


    \end{boxedminipage}

    \label{xformslib:flglcanvas:fl_glwinopen}
    \index{xformslib \textit{(package)}!xformslib.flglcanvas \textit{(module)}!xformslib.flglcanvas.fl\_glwinopen \textit{(function)}}

    \vspace{0.5ex}

\hspace{.8\funcindent}\begin{boxedminipage}{\funcwidth}

    \raggedright \textbf{fl\_glwinopen}(\textit{config}, \textit{glxcontext}, \textit{w}, \textit{h})

    \vspace{-1.5ex}

    \rule{\textwidth}{0.5\fboxrule}
\setlength{\parskip}{2ex}

Opens a toplevel OpenGL window.

-{}-
\setlength{\parskip}{1ex}
      \textbf{Parameters}
      \vspace{-1ex}

      \begin{quote}
        \begin{Ventry}{xxxxxxxxxx}

          \item[config]


GL configuration settings
            {\it (type=int)}

          \item[glxcontext]


glxcontext class instance
            {\it (type=pointer to xfdata.GLXContext)}

          \item[w]


width of GL window in coord units
            {\it (type=int)}

          \item[h]


height of GL window in coord units
            {\it (type=int)}

        \end{Ventry}

      \end{quote}

      \textbf{Return Value}
    \vspace{-1ex}

      \begin{quote}

window opened (win)
      {\it (type=long\_pos)}

      \end{quote}

\textbf{Note:} 
e.g. gwin0 = fl\_glwinopen(?, ?, 650, 560)


\textbf{Status:} 
Untested + Doc + NoDemo = NOT OK


    \end{boxedminipage}


%%%%%%%%%%%%%%%%%%%%%%%%%%%%%%%%%%%%%%%%%%%%%%%%%%%%%%%%%%%%%%%%%%%%%%%%%%%
%%                               Variables                               %%
%%%%%%%%%%%%%%%%%%%%%%%%%%%%%%%%%%%%%%%%%%%%%%%%%%%%%%%%%%%%%%%%%%%%%%%%%%%

  \subsection{Variables}

    \vspace{-1cm}
\hspace{\varindent}\begin{longtable}{|p{\varnamewidth}|p{\vardescrwidth}|l}
\cline{1-2}
\cline{1-2} \centering \textbf{Name} & \centering \textbf{Description}& \\
\cline{1-2}
\endhead\cline{1-2}\multicolumn{3}{r}{\small\textit{continued on next page}}\\\endfoot\cline{1-2}
\endlastfoot\raggedright \_\-\_\-p\-a\-c\-k\-a\-g\-e\-\_\-\_\- & \raggedright \textbf{Value:} 
{\tt \texttt{'}\texttt{xformslib}\texttt{'}}&\\
\cline{1-2}
\end{longtable}

    \index{xformslib \textit{(package)}!xformslib.flglcanvas \textit{(module)}|)}
