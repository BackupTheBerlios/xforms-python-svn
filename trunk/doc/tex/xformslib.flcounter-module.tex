%
% API Documentation for API Documentation
% Module xformslib.flcounter
%
% Generated by epydoc 3.0.1
% [Fri May 14 18:28:31 2010]
%

%%%%%%%%%%%%%%%%%%%%%%%%%%%%%%%%%%%%%%%%%%%%%%%%%%%%%%%%%%%%%%%%%%%%%%%%%%%
%%                          Module Description                           %%
%%%%%%%%%%%%%%%%%%%%%%%%%%%%%%%%%%%%%%%%%%%%%%%%%%%%%%%%%%%%%%%%%%%%%%%%%%%

    \index{xformslib \textit{(package)}!xformslib.flcounter \textit{(module)}|(}
\section{Module xformslib.flcounter}

    \label{xformslib:flcounter}

flcounters.py - xforms-python's functions to manage counters.

Copyright (C) 2009, 2010  Luca Lazzaroni ``LukenShiro''
e-mail: <\href{mailto:lukenshiro@ngi.it}{lukenshiro@ngi.it}>

This program is free software: you can redistribute it and/or modify
it under the terms of the GNU Lesser General Public License as
published by the Free Software Foundation, version 2.1 of the License.

This program is distributed in the hope that it will be useful,
but WITHOUT ANY WARRANTY; without even the implied warranty of
MERCHANTABILITY or FITNESS FOR A PARTICULAR PURPOSE. See the
GNU Lesser General Public License for more details.

You should have received a copy of the GNU LGPL along with this
program. If not, see <\href{http://www.gnu.org/licenses/}{http://www.gnu.org/licenses/}>.

See CREDITS file to read acknowledgements and thanks to XForms,
ctypes and other developers.

%%%%%%%%%%%%%%%%%%%%%%%%%%%%%%%%%%%%%%%%%%%%%%%%%%%%%%%%%%%%%%%%%%%%%%%%%%%
%%                               Functions                               %%
%%%%%%%%%%%%%%%%%%%%%%%%%%%%%%%%%%%%%%%%%%%%%%%%%%%%%%%%%%%%%%%%%%%%%%%%%%%

  \subsection{Functions}

    \label{xformslib:flcounter:fl_add_counter}
    \index{xformslib \textit{(package)}!xformslib.flcounter \textit{(module)}!xformslib.flcounter.fl\_add\_counter \textit{(function)}}

    \vspace{0.5ex}

\hspace{.8\funcindent}\begin{boxedminipage}{\funcwidth}

    \raggedright \textbf{fl\_add\_counter}(\textit{countertype}, \textit{x}, \textit{y}, \textit{w}, \textit{h}, \textit{label})

    \vspace{-1.5ex}

    \rule{\textwidth}{0.5\fboxrule}
\setlength{\parskip}{2ex}

Adds a counter object.

-{}-
\setlength{\parskip}{1ex}
      \textbf{Parameters}
      \vspace{-1ex}

      \begin{quote}
        \begin{Ventry}{xxxxxxxxxxx}

          \item[countertype]


type of counter to be added. Values (from xfdata.py) FL\_NORMAL\_COUNTER,
FL\_SIMPLE\_COUNTER
            {\it (type=int)}

          \item[x]


horizontal position (upper-left corner)
            {\it (type=int)}

          \item[y]


vertical position (upper-left corner)
            {\it (type=int)}

          \item[w]


width in coord units
            {\it (type=int)}

          \item[h]


height in coord units
            {\it (type=int)}

          \item[label]


text label of counter
            {\it (type=str)}

        \end{Ventry}

      \end{quote}

      \textbf{Return Value}
    \vspace{-1ex}

      \begin{quote}

counter object added (pFlObject)
      {\it (type=pointer to xfdata.FL\_OBJECT)}

      \end{quote}

\textbf{Note:} 
e.g. ctrobj = fl\_add\_counter(FL\_NORMAL\_COUNTER, 142, 230, 142,
100, ``My Counter'')


\textbf{Status:} 
Tested + Doc + Demo = OK


    \end{boxedminipage}

    \label{xformslib:flcounter:fl_set_counter_value}
    \index{xformslib \textit{(package)}!xformslib.flcounter \textit{(module)}!xformslib.flcounter.fl\_set\_counter\_value \textit{(function)}}

    \vspace{0.5ex}

\hspace{.8\funcindent}\begin{boxedminipage}{\funcwidth}

    \raggedright \textbf{fl\_set\_counter\_value}(\textit{pFlObject}, \textit{val})

    \vspace{-1.5ex}

    \rule{\textwidth}{0.5\fboxrule}
\setlength{\parskip}{2ex}

Sets the value of the counter.

-{}-
\setlength{\parskip}{1ex}
      \textbf{Parameters}
      \vspace{-1ex}

      \begin{quote}
        \begin{Ventry}{xxxxxxxxx}

          \item[pFlObject]


counter object
            {\it (type=pointer to xfdata.FL\_OBJECT)}

          \item[val]


value to be set. By default it is 0.
            {\it (type=float)}

        \end{Ventry}

      \end{quote}

\textbf{Note:} 
e.g. fl\_set\_counter\_value(ctrobj, 42.0)


\textbf{Status:} 
Tested + Doc + Demo = OK


    \end{boxedminipage}

    \label{xformslib:flcounter:fl_set_counter_bounds}
    \index{xformslib \textit{(package)}!xformslib.flcounter \textit{(module)}!xformslib.flcounter.fl\_set\_counter\_bounds \textit{(function)}}

    \vspace{0.5ex}

\hspace{.8\funcindent}\begin{boxedminipage}{\funcwidth}

    \raggedright \textbf{fl\_set\_counter\_bounds}(\textit{pFlObject}, \textit{minbound}, \textit{maxbound})

    \vspace{-1.5ex}

    \rule{\textwidth}{0.5\fboxrule}
\setlength{\parskip}{2ex}

Sets the minimum and maximum values that the counter will take. For
conficting settings bound take precedence over value.

-{}-
\setlength{\parskip}{1ex}
      \textbf{Parameters}
      \vspace{-1ex}

      \begin{quote}
        \begin{Ventry}{xxxxxxxxx}

          \item[pFlObject]


counter object
            {\it (type=pointer to xfdata.FL\_OBJECT)}

          \item[minbound]


minimum value to be set. By default it is -1000000.
            {\it (type=float)}

          \item[maxbound]


maximum value to be set. By default it is 1000000.
            {\it (type=float)}

        \end{Ventry}

      \end{quote}

\textbf{Note:} 
e.g. fl\_set\_counter\_bounds(ctrobj, -100.0, 100.0)


\textbf{Status:} 
Tested + Doc + Demo = OK


    \end{boxedminipage}

    \label{xformslib:flcounter:fl_set_counter_step}
    \index{xformslib \textit{(package)}!xformslib.flcounter \textit{(module)}!xformslib.flcounter.fl\_set\_counter\_step \textit{(function)}}

    \vspace{0.5ex}

\hspace{.8\funcindent}\begin{boxedminipage}{\funcwidth}

    \raggedright \textbf{fl\_set\_counter\_step}(\textit{pFlObject}, \textit{small}, \textit{large})

    \vspace{-1.5ex}

    \rule{\textwidth}{0.5\fboxrule}
\setlength{\parskip}{2ex}

Sets the sizes of the small and large steps of a counter. For simple
counters only the small step is used.

-{}-
\setlength{\parskip}{1ex}
      \textbf{Parameters}
      \vspace{-1ex}

      \begin{quote}
        \begin{Ventry}{xxxxxxxxx}

          \item[pFlObject]


counter object
            {\it (type=pointer to xfdata.FL\_OBJECT)}

          \item[small]


small step's size. By default it is 0.1.
            {\it (type=float)}

          \item[large]


large step's size. By default it is 1.
            {\it (type=float)}

        \end{Ventry}

      \end{quote}

\textbf{Note:} 
e.g. fl\_set\_counter\_step(ctrobj, 0.2, 2)


\textbf{Status:} 
Tested + Doc + Demo = OK


    \end{boxedminipage}

    \label{xformslib:flcounter:fl_set_counter_precision}
    \index{xformslib \textit{(package)}!xformslib.flcounter \textit{(module)}!xformslib.flcounter.fl\_set\_counter\_precision \textit{(function)}}

    \vspace{0.5ex}

\hspace{.8\funcindent}\begin{boxedminipage}{\funcwidth}

    \raggedright \textbf{fl\_set\_counter\_precision}(\textit{pFlObject}, \textit{prec})

    \vspace{-1.5ex}

    \rule{\textwidth}{0.5\fboxrule}
\setlength{\parskip}{2ex}

Sets the precision (number of digits after the dot) with which
the counter value is displayed.

-{}-
\setlength{\parskip}{1ex}
      \textbf{Parameters}
      \vspace{-1ex}

      \begin{quote}
        \begin{Ventry}{xxxxxxxxx}

          \item[pFlObject]


counter object
            {\it (type=pointer to xfdata.FL\_OBJECT)}

          \item[prec]


precision to be set
            {\it (type=int)}

        \end{Ventry}

      \end{quote}

\textbf{Note:} 
e.g. fl\_set\_counter\_precision(ctrobj, 2)


\textbf{Status:} 
Tested + Doc + Demo = OK


    \end{boxedminipage}

    \label{xformslib:flcounter:fl_get_counter_precision}
    \index{xformslib \textit{(package)}!xformslib.flcounter \textit{(module)}!xformslib.flcounter.fl\_get\_counter\_precision \textit{(function)}}

    \vspace{0.5ex}

\hspace{.8\funcindent}\begin{boxedminipage}{\funcwidth}

    \raggedright \textbf{fl\_get\_counter\_precision}(\textit{pFlObject})

    \vspace{-1.5ex}

    \rule{\textwidth}{0.5\fboxrule}
\setlength{\parskip}{2ex}

Determines the current value of the precision (number of digits
after the dot) of the counter.

-{}-
\setlength{\parskip}{1ex}
      \textbf{Parameters}
      \vspace{-1ex}

      \begin{quote}
        \begin{Ventry}{xxxxxxxxx}

          \item[pFlObject]


counter object
            {\it (type=pointer to xfdata.FL\_OBJECT)}

        \end{Ventry}

      \end{quote}

      \textbf{Return Value}
    \vspace{-1ex}

      \begin{quote}

number of digits after the dot
      {\it (type=int)}

      \end{quote}

\textbf{Note:} 
e.g. currprec = fl\_get\_counter\_precision(ctrobj)


\textbf{Status:} 
Tested + Doc + NoDemo = OK


    \end{boxedminipage}

    \label{xformslib:flcounter:fl_get_counter_value}
    \index{xformslib \textit{(package)}!xformslib.flcounter \textit{(module)}!xformslib.flcounter.fl\_get\_counter\_value \textit{(function)}}

    \vspace{0.5ex}

\hspace{.8\funcindent}\begin{boxedminipage}{\funcwidth}

    \raggedright \textbf{fl\_get\_counter\_value}(\textit{pFlObject})

    \vspace{-1.5ex}

    \rule{\textwidth}{0.5\fboxrule}
\setlength{\parskip}{2ex}

Obtains the current value of the counter.

-{}-
\setlength{\parskip}{1ex}
      \textbf{Parameters}
      \vspace{-1ex}

      \begin{quote}
        \begin{Ventry}{xxxxxxxxx}

          \item[pFlObject]


counter object
            {\it (type=pointer to xfdata.FL\_OBJECT)}

        \end{Ventry}

      \end{quote}

      \textbf{Return Value}
    \vspace{-1ex}

      \begin{quote}

current value
      {\it (type=float)}

      \end{quote}

\textbf{Note:} 
e.g. currvalue = fl\_get\_counter\_value(ctrobj)


\textbf{Status:} 
Tested + Doc + Demo = OK


    \end{boxedminipage}

    \label{xformslib:flcounter:fl_get_counter_bounds}
    \index{xformslib \textit{(package)}!xformslib.flcounter \textit{(module)}!xformslib.flcounter.fl\_get\_counter\_bounds \textit{(function)}}

    \vspace{0.5ex}

\hspace{.8\funcindent}\begin{boxedminipage}{\funcwidth}

    \raggedright \textbf{fl\_get\_counter\_bounds}(\textit{pFlObject})

    \vspace{-1.5ex}

    \rule{\textwidth}{0.5\fboxrule}
\setlength{\parskip}{2ex}

Obtains the current minimum and maximum bounds of the counter.

-{}-
\setlength{\parskip}{1ex}
      \textbf{Parameters}
      \vspace{-1ex}

      \begin{quote}
        \begin{Ventry}{xxxxxxxxx}

          \item[pFlObject]


counter object
            {\it (type=pointer to xfdata.FL\_OBJECT)}

        \end{Ventry}

      \end{quote}

      \textbf{Return Value}
    \vspace{-1ex}

      \begin{quote}

minimum bound, maximum bound
      {\it (type=float, float)}

      \end{quote}

\textbf{Note:} 
e.g. minb, maxb = fl\_get\_counter\_bounds(ctrobj)


\textbf{Attention:} 
API change from XForms - upstream was
fl\_get\_counter\_bounds(pFlObject, minbound, maxbound)


\textbf{Status:} 
Tested + Doc + NoDemo = OK


    \end{boxedminipage}

    \label{xformslib:flcounter:fl_get_counter_step}
    \index{xformslib \textit{(package)}!xformslib.flcounter \textit{(module)}!xformslib.flcounter.fl\_get\_counter\_step \textit{(function)}}

    \vspace{0.5ex}

\hspace{.8\funcindent}\begin{boxedminipage}{\funcwidth}

    \raggedright \textbf{fl\_get\_counter\_step}(\textit{pFlObject})

    \vspace{-1.5ex}

    \rule{\textwidth}{0.5\fboxrule}
\setlength{\parskip}{2ex}

Obtains the current small and large step's sizes of a counter.

-{}-
\setlength{\parskip}{1ex}
      \textbf{Parameters}
      \vspace{-1ex}

      \begin{quote}
        \begin{Ventry}{xxxxxxxxx}

          \item[pFlObject]


counter object
            {\it (type=pointer to xfdata.FL\_OBJECT)}

        \end{Ventry}

      \end{quote}

      \textbf{Return Value}
    \vspace{-1ex}

      \begin{quote}

small step size, large step size
      {\it (type=float, float)}

      \end{quote}

\textbf{Note:} 
e.g. minb, maxb = fl\_get\_counter\_step(ctrobj)


\textbf{Attention:} 
API change from XForms - upstream was
fl\_get\_counter\_step(pFlObject, sml, lrg)


\textbf{Status:} 
Tested + Doc + NoDemo = OK


    \end{boxedminipage}

    \label{xformslib:flcounter:fl_set_counter_filter}
    \index{xformslib \textit{(package)}!xformslib.flcounter \textit{(module)}!xformslib.flcounter.fl\_set\_counter\_filter \textit{(function)}}

    \vspace{0.5ex}

\hspace{.8\funcindent}\begin{boxedminipage}{\funcwidth}

    \raggedright \textbf{fl\_set\_counter\_filter}(\textit{pFlObject}, \textit{py\_ValFilter})

    \vspace{-1.5ex}

    \rule{\textwidth}{0.5\fboxrule}
\setlength{\parskip}{2ex}

Overrides the format and value shown by the counter. By default the
value is shown in floating point format.

-{}-
\setlength{\parskip}{1ex}
      \textbf{Parameters}
      \vspace{-1ex}

      \begin{quote}
        \begin{Ventry}{xxxxxxxxxxxx}

          \item[pFlObject]


counter object
            {\it (type=pointer to xfdata.FL\_OBJECT)}

          \item[py\_ValFilter]


name referring to function(pObject, valfloat, intprec) -> str
            {\it (type=python callback function, returning value)}

        \end{Ventry}

      \end{quote}

\textbf{Notes:}
\begin{quote}
  \begin{itemize}

  \item
    \setlength{\parskip}{0.6ex}

e.g. def ctrvalfilt(pobj, fvalue, prec): > ... ; return string


  \item 
e.g. fl\_set\_counter\_filter(ctrobj, ctrvalfilt)


\end{itemize}

\end{quote}

\textbf{Status:} 
Tested + Doc + NoDemo = OK


    \end{boxedminipage}

    \label{xformslib:flcounter:fl_get_counter_repeat}
    \index{xformslib \textit{(package)}!xformslib.flcounter \textit{(module)}!xformslib.flcounter.fl\_get\_counter\_repeat \textit{(function)}}

    \vspace{0.5ex}

\hspace{.8\funcindent}\begin{boxedminipage}{\funcwidth}

    \raggedright \textbf{fl\_get\_counter\_repeat}(\textit{pFlObject})

    \vspace{-1.5ex}

    \rule{\textwidth}{0.5\fboxrule}
\setlength{\parskip}{2ex}

Returns the initial delay of the counter.

-{}-
\setlength{\parskip}{1ex}
      \textbf{Parameters}
      \vspace{-1ex}

      \begin{quote}
        \begin{Ventry}{xxxxxxxxx}

          \item[pFlObject]


counter object
            {\it (type=pointer to xfdata.FL\_OBJECT)}

        \end{Ventry}

      \end{quote}

      \textbf{Return Value}
    \vspace{-1ex}

      \begin{quote}

initial delay in milliseconds
      {\it (type=int)}

      \end{quote}

\textbf{Note:} 
e.g. intdly = fl\_get\_counter\_repeat(ctrobj)


\textbf{Status:} 
Tested + Doc + NoDemo = OK


    \end{boxedminipage}

    \label{xformslib:flcounter:fl_set_counter_repeat}
    \index{xformslib \textit{(package)}!xformslib.flcounter \textit{(module)}!xformslib.flcounter.fl\_set\_counter\_repeat \textit{(function)}}

    \vspace{0.5ex}

\hspace{.8\funcindent}\begin{boxedminipage}{\funcwidth}

    \raggedright \textbf{fl\_set\_counter\_repeat}(\textit{pFlObject}, \textit{msec})

    \vspace{-1.5ex}

    \rule{\textwidth}{0.5\fboxrule}
\setlength{\parskip}{2ex}

Sets the initial delay of a counter. By default the counter value
changes first slowly and the rate of change then accelerates until a
final speed is reached. The default delay between the value changing
is 600 ms at the start.

-{}-
\setlength{\parskip}{1ex}
      \textbf{Parameters}
      \vspace{-1ex}

      \begin{quote}
        \begin{Ventry}{xxxxxxxxx}

          \item[pFlObject]


counter object
            {\it (type=pointer to xfdata.FL\_OBJECT)}

          \item[msec]


initial delay in milliseconds
            {\it (type=int)}

        \end{Ventry}

      \end{quote}

\textbf{Note:} 
e.g. fl\_set\_counter\_repeat(ctrobj, 200)


\textbf{Status:} 
Tested + Doc + NoDemo = OK


    \end{boxedminipage}

    \label{xformslib:flcounter:fl_get_counter_min_repeat}
    \index{xformslib \textit{(package)}!xformslib.flcounter \textit{(module)}!xformslib.flcounter.fl\_get\_counter\_min\_repeat \textit{(function)}}

    \vspace{0.5ex}

\hspace{.8\funcindent}\begin{boxedminipage}{\funcwidth}

    \raggedright \textbf{fl\_get\_counter\_min\_repeat}(\textit{pFlObject})

    \vspace{-1.5ex}

    \rule{\textwidth}{0.5\fboxrule}
\setlength{\parskip}{2ex}

Returns the final delay of a counter object.

-{}-
\setlength{\parskip}{1ex}
      \textbf{Parameters}
      \vspace{-1ex}

      \begin{quote}
        \begin{Ventry}{xxxxxxxxx}

          \item[pFlObject]


counter object
            {\it (type=pointer to xfdata.FL\_OBJECT)}

        \end{Ventry}

      \end{quote}

      \textbf{Return Value}
    \vspace{-1ex}

      \begin{quote}

final delay in milliseconds
      {\it (type=int)}

      \end{quote}

\textbf{Note:} 
e.g. fnldly = fl\_get\_counter\_min\_repeat(ctrobj)


\textbf{Status:} 
Tested + Doc + NoDemo = OK


    \end{boxedminipage}

    \label{xformslib:flcounter:fl_set_counter_min_repeat}
    \index{xformslib \textit{(package)}!xformslib.flcounter \textit{(module)}!xformslib.flcounter.fl\_set\_counter\_min\_repeat \textit{(function)}}

    \vspace{0.5ex}

\hspace{.8\funcindent}\begin{boxedminipage}{\funcwidth}

    \raggedright \textbf{fl\_set\_counter\_min\_repeat}(\textit{pFlObject}, \textit{msec})

    \vspace{-1.5ex}

    \rule{\textwidth}{0.5\fboxrule}
\setlength{\parskip}{2ex}

Sets the final delay of a counter. By default the counter value
changes first slowly and the rate of change then accelerates until
a final speed is reached. The default the final delay is 50 ms.

-{}-
\setlength{\parskip}{1ex}
      \textbf{Parameters}
      \vspace{-1ex}

      \begin{quote}
        \begin{Ventry}{xxxxxxxxx}

          \item[pFlObject]


counter object
            {\it (type=pointer to xfdata.FL\_OBJECT)}

          \item[msec]


final delay in milliseconds
            {\it (type=int)}

        \end{Ventry}

      \end{quote}

\textbf{Note:} 
e.g. fl\_set\_counter\_min\_repeat(ctrobj, 100)


\textbf{Status:} 
Tested + Doc + NoDemo = OK


    \end{boxedminipage}

    \label{xformslib:flcounter:fl_get_counter_speedjump}
    \index{xformslib \textit{(package)}!xformslib.flcounter \textit{(module)}!xformslib.flcounter.fl\_get\_counter\_speedjump \textit{(function)}}

    \vspace{0.5ex}

\hspace{.8\funcindent}\begin{boxedminipage}{\funcwidth}

    \raggedright \textbf{fl\_get\_counter\_speedjump}(\textit{pFlObject})

    \vspace{-1.5ex}

    \rule{\textwidth}{0.5\fboxrule}
\setlength{\parskip}{2ex}

Determines the setting for speedjumping.

-{}-
\setlength{\parskip}{1ex}
      \textbf{Parameters}
      \vspace{-1ex}

      \begin{quote}
        \begin{Ventry}{xxxxxxxxx}

          \item[pFlObject]


counter object
            {\it (type=pointer to xfdata.FL\_OBJECT)}

        \end{Ventry}

      \end{quote}

      \textbf{Return Value}
    \vspace{-1ex}

      \begin{quote}

setting flag of speedjump (1 if set, or 0 if unset)
      {\it (type=int)}

      \end{quote}

\textbf{Note:} 
e.g. isspdjmp = fl\_get\_counter\_speedjump(ctrobj)


\textbf{Status:} 
Tested + Doc + NoDemo = OK


    \end{boxedminipage}

    \label{xformslib:flcounter:fl_set_counter_speedjump}
    \index{xformslib \textit{(package)}!xformslib.flcounter \textit{(module)}!xformslib.flcounter.fl\_set\_counter\_speedjump \textit{(function)}}

    \vspace{0.5ex}

\hspace{.8\funcindent}\begin{boxedminipage}{\funcwidth}

    \raggedright \textbf{fl\_set\_counter\_speedjump}(\textit{pFlObject}, \textit{yesno})

    \vspace{-1.5ex}

    \rule{\textwidth}{0.5\fboxrule}
\setlength{\parskip}{2ex}

Makes only the first change of the counter has a different delay
from all the following ones. The delay for the first change of the
counter value will then be the one set by fl\_set\_counter\_repeat()
and the following delays last as long as set by
fl\_set\_counter\_min\_repeat().

-{}-
\setlength{\parskip}{1ex}
      \textbf{Parameters}
      \vspace{-1ex}

      \begin{quote}
        \begin{Ventry}{xxxxxxxxx}

          \item[pFlObject]


counter object
            {\it (type=pointer to xfdata.FL\_OBJECT)}

          \item[yesno]


flag. Values 1 (to set speedjump) or 0 (to unset speedjump)
            {\it (type=int)}

        \end{Ventry}

      \end{quote}

\textbf{Note:} 
e.g. fl\_set\_counter\_speedjump(ctrobj, 1)


\textbf{Status:} 
Tested + Doc + NoDemo = OK


    \end{boxedminipage}


%%%%%%%%%%%%%%%%%%%%%%%%%%%%%%%%%%%%%%%%%%%%%%%%%%%%%%%%%%%%%%%%%%%%%%%%%%%
%%                               Variables                               %%
%%%%%%%%%%%%%%%%%%%%%%%%%%%%%%%%%%%%%%%%%%%%%%%%%%%%%%%%%%%%%%%%%%%%%%%%%%%

  \subsection{Variables}

    \vspace{-1cm}
\hspace{\varindent}\begin{longtable}{|p{\varnamewidth}|p{\vardescrwidth}|l}
\cline{1-2}
\cline{1-2} \centering \textbf{Name} & \centering \textbf{Description}& \\
\cline{1-2}
\endhead\cline{1-2}\multicolumn{3}{r}{\small\textit{continued on next page}}\\\endfoot\cline{1-2}
\endlastfoot\raggedright \_\-\_\-p\-a\-c\-k\-a\-g\-e\-\_\-\_\- & \raggedright \textbf{Value:} 
{\tt \texttt{'}\texttt{xformslib}\texttt{'}}&\\
\cline{1-2}
\end{longtable}

    \index{xformslib \textit{(package)}!xformslib.flcounter \textit{(module)}|)}
