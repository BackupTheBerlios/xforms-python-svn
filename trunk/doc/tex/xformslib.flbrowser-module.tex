%
% API Documentation for API Documentation
% Module xformslib.flbrowser
%
% Generated by epydoc 3.0.1
% [Tue May 18 00:48:11 2010]
%

%%%%%%%%%%%%%%%%%%%%%%%%%%%%%%%%%%%%%%%%%%%%%%%%%%%%%%%%%%%%%%%%%%%%%%%%%%%
%%                          Module Description                           %%
%%%%%%%%%%%%%%%%%%%%%%%%%%%%%%%%%%%%%%%%%%%%%%%%%%%%%%%%%%%%%%%%%%%%%%%%%%%

    \index{xformslib \textit{(package)}!xformslib.flbrowser \textit{(module)}|(}
\section{Module xformslib.flbrowser}

    \label{xformslib:flbrowser}

xforms-python's functions to manage browser objects.

Copyright (C) 2009, 2010  Luca Lazzaroni ``LukenShiro''
e-mail:  <\href{mailto:lukenshiro@ngi.it}{lukenshiro@ngi.it}>

This program is free software: you can redistribute it and/or modify
it under the terms of the GNU Lesser General Public License as
published by the Free Software Foundation, version 2.1 of the License.

This program is distributed in the hope that it will be useful,
but WITHOUT ANY WARRANTY; without even the implied warranty of
MERCHANTABILITY or FITNESS FOR A PARTICULAR PURPOSE. See the
GNU Lesser General Public License for more details.

You should have received a copy of the GNU LGPL along with this
program. If not, see <\href{http://www.gnu.org/licenses/}{http://www.gnu.org/licenses/}>.

See CREDITS file to read acknowledgements and thanks to XForms,
ctypes and other developers.

%%%%%%%%%%%%%%%%%%%%%%%%%%%%%%%%%%%%%%%%%%%%%%%%%%%%%%%%%%%%%%%%%%%%%%%%%%%
%%                               Functions                               %%
%%%%%%%%%%%%%%%%%%%%%%%%%%%%%%%%%%%%%%%%%%%%%%%%%%%%%%%%%%%%%%%%%%%%%%%%%%%

  \subsection{Functions}

    \label{xformslib:flbrowser:fl_add_browser}
    \index{xformslib \textit{(package)}!xformslib.flbrowser \textit{(module)}!xformslib.flbrowser.fl\_add\_browser \textit{(function)}}

    \vspace{0.5ex}

\hspace{.8\funcindent}\begin{boxedminipage}{\funcwidth}

    \raggedright \textbf{fl\_add\_browser}(\textit{browsertype}, \textit{x}, \textit{y}, \textit{w}, \textit{h}, \textit{label})

    \vspace{-1.5ex}

    \rule{\textwidth}{0.5\fboxrule}
\setlength{\parskip}{2ex}

Adds a browser object. The label is placed below the box by default.

-{}-
\setlength{\parskip}{1ex}
      \textbf{Parameters}
      \vspace{-1ex}

      \begin{quote}
        \begin{Ventry}{xxxxxxxxxxx}

          \item[browsertype]


type of the browser to be added. Values (from xfdata.py)
FL\_NORMAL\_BROWSER, FL\_SELECT\_BROWSER, FL\_HOLD\_BROWSER, FL\_MULTI\_BROWSER
            {\it (type=int)}

          \item[x]


horizontal position (upper-left corner)
            {\it (type=int)}

          \item[y]


vertical position (upper-left corner)
            {\it (type=int)}

          \item[w]


width in coord units
            {\it (type=int)}

          \item[h]


height in coord units
            {\it (type=int)}

          \item[label]


text label of browser
            {\it (type=str)}

        \end{Ventry}

      \end{quote}

      \textbf{Return Value}
    \vspace{-1ex}

      \begin{quote}

browser object created (pFlObject)
      {\it (type=pointer to xfdata.FL\_OBJECT)}

      \end{quote}

\textbf{Note:} 
e.g. brobj = fl\_add\_browser(FL\_SELECT\_BROWSER, 200, 250, 200, 200,
``BrowserList'')


\textbf{Status:} 
Tested + Doc + Demo = OK


    \end{boxedminipage}

    \label{xformslib:flbrowser:fl_clear_browser}
    \index{xformslib \textit{(package)}!xformslib.flbrowser \textit{(module)}!xformslib.flbrowser.fl\_clear\_browser \textit{(function)}}

    \vspace{0.5ex}

\hspace{.8\funcindent}\begin{boxedminipage}{\funcwidth}

    \raggedright \textbf{fl\_clear\_browser}(\textit{pFlObject})

    \vspace{-1.5ex}

    \rule{\textwidth}{0.5\fboxrule}
\setlength{\parskip}{2ex}

Clears contents of a browser object.

-{}-
\setlength{\parskip}{1ex}
      \textbf{Parameters}
      \vspace{-1ex}

      \begin{quote}
        \begin{Ventry}{xxxxxxxxx}

          \item[pFlObject]


browser object
            {\it (type=pointer to xfdata.FL\_OBJECT)}

        \end{Ventry}

      \end{quote}

\textbf{Note:} 
e.g. fl\_clear\_browser(brobj)


\textbf{Status:} 
Tested + Doc + Demo = OK


    \end{boxedminipage}

    \label{xformslib:flbrowser:fl_add_browser_line}
    \index{xformslib \textit{(package)}!xformslib.flbrowser \textit{(module)}!xformslib.flbrowser.fl\_add\_browser\_line \textit{(function)}}

    \vspace{0.5ex}

\hspace{.8\funcindent}\begin{boxedminipage}{\funcwidth}

    \raggedright \textbf{fl\_add\_browser\_line}(\textit{pFlObject}, \textit{newtext})

    \vspace{-1.5ex}

    \rule{\textwidth}{0.5\fboxrule}
\setlength{\parskip}{2ex}

Add a line to a browser object. The line may contain embedded newline
characters: these will result in the text being split up into several
lines, separated at the newline characters. It is possible to change the
appearance of individual lines in the browser. Whenever a line starts
with the symbol '@' the next letter indicates the special characteristics
associated with this line. The following possibilities exist at the moment:
f (Fixed width font), n (Normal, Helvetica font), t (Times-Roman like
font), b (Boldface modifier), i (Italics modifier), l (Large, new size is
xfdata.FL\_LARGE\_SIZE), m (Medium, new size is xfdata.FL\_MEDIUM\_SIZE),
s (Small, new size is xfdata.FL\_SMALL\_SIZE), L (Large, new size = current
size + 6), M (Medium, new size = current size + 4), S (Small, new size =
current size - 2), c (Centered), r (Right aligned), \_ (Draw underlined
text, - (An engraved separator. Text following '-' is ignored), C (The
next number indicates the color index for this line).,N (Non-selectable
line, in selectable browsers), @@, double @ (Regular '@' character). The
modifiers (bold and italic) work by adding xfdata.FL\_BOLD\_STYLE and
xfdata.FL\_ITALIC\_STYLE to the current active font index to look up the
font in the font table (you can modify the table using fl\_set\_font\_name()).
More than one option can be used by putting them next to each other. For
example, ``@C1@l@f@b@cTitle'' will give you the red, large, bold fixed font,
centered word ``Title''. As you can see the font change requests accumulate
and the order is important, i.e., ``@f@b@i'' gives you a fixed bold italic
font while ``@b@i@f'' gives you a (plain) fixed font. Depending on the font
size and style lines may have different heights.

-{}-
\setlength{\parskip}{1ex}
      \textbf{Parameters}
      \vspace{-1ex}

      \begin{quote}
        \begin{Ventry}{xxxxxxxxx}

          \item[pFlObject]


browser object
            {\it (type=pointer to xfdata.FL\_OBJECT)}

          \item[newtext]


line of text to be added
            {\it (type=str)}

        \end{Ventry}

      \end{quote}

\textbf{Note:} 
e.g. fl\_add\_browser\_line(brobj, ``My new line text'')


\textbf{Status:} 
Tested + Doc + Demo = OK


    \end{boxedminipage}

    \label{xformslib:flbrowser:fl_addto_browser}
    \index{xformslib \textit{(package)}!xformslib.flbrowser \textit{(module)}!xformslib.flbrowser.fl\_addto\_browser \textit{(function)}}

    \vspace{0.5ex}

\hspace{.8\funcindent}\begin{boxedminipage}{\funcwidth}

    \raggedright \textbf{fl\_addto\_browser}(\textit{pFlObject}, \textit{newtext})

    \vspace{-1.5ex}

    \rule{\textwidth}{0.5\fboxrule}
\setlength{\parskip}{2ex}

Adds text to browser. The browser will be shifted such that the newly
appended line is visible. This is useful when e.g. using the browser to
display messages. The text may contain embedded newline characters. See
fl\_add\_browser\_line() for appearance change.

-{}-
\setlength{\parskip}{1ex}
      \textbf{Parameters}
      \vspace{-1ex}

      \begin{quote}
        \begin{Ventry}{xxxxxxxxx}

          \item[pFlObject]


browser object
            {\it (type=pointer to xfdata.FL\_OBJECT)}

          \item[newtext]


text to be added
            {\it (type=str)}

        \end{Ventry}

      \end{quote}

\textbf{Note:} 
e.g. fl\_addto\_browser(brobj, ``blablablablablublublu'')


\textbf{Status:} 
Tested + Doc + Demo = OK


    \end{boxedminipage}

    \label{xformslib:flbrowser:fl_addto_browser_chars}
    \index{xformslib \textit{(package)}!xformslib.flbrowser \textit{(module)}!xformslib.flbrowser.fl\_addto\_browser\_chars \textit{(function)}}

    \vspace{0.5ex}

\hspace{.8\funcindent}\begin{boxedminipage}{\funcwidth}

    \raggedright \textbf{fl\_addto\_browser\_chars}(\textit{pFlObject}, \textit{addedtext})

    \vspace{-1.5ex}

    \rule{\textwidth}{0.5\fboxrule}
\setlength{\parskip}{2ex}

Appends text to the last line in the browser without advancing the
line counter. The text may contain embedded newline characters. In that
case, the text before the first embedded newline is appended to the last
line, and everything aferwards is put onto new lines. As in the case of
fl\_addto\_browser() the last added line will be visible in the browser.
See fl\_add\_browser\_line() for appearance change.

-{}-
\setlength{\parskip}{1ex}
      \textbf{Parameters}
      \vspace{-1ex}

      \begin{quote}
        \begin{Ventry}{xxxxxxxxx}

          \item[pFlObject]


browser object
            {\it (type=pointer to xfdata.FL\_OBJECT)}

          \item[addedtext]


text to be added
            {\it (type=str)}

        \end{Ventry}

      \end{quote}

\textbf{Note:} 
e.g. fl\_addto\_browser\_chars(brobj, ``some chars'')


\textbf{Status:} 
Tested + Doc + NoDemo = OK


    \end{boxedminipage}

    \label{xformslib:flbrowser:fl_addto_browser_chars}
    \index{xformslib \textit{(package)}!xformslib.flbrowser \textit{(module)}!xformslib.flbrowser.fl\_addto\_browser\_chars \textit{(function)}}

    \vspace{0.5ex}

\hspace{.8\funcindent}\begin{boxedminipage}{\funcwidth}

    \raggedright \textbf{fl\_append\_browser}(\textit{pFlObject}, \textit{addedtext})

    \vspace{-1.5ex}

    \rule{\textwidth}{0.5\fboxrule}
\setlength{\parskip}{2ex}

Appends text to the last line in the browser without advancing the
line counter. The text may contain embedded newline characters. In that
case, the text before the first embedded newline is appended to the last
line, and everything aferwards is put onto new lines. As in the case of
fl\_addto\_browser() the last added line will be visible in the browser.
See fl\_add\_browser\_line() for appearance change.

-{}-
\setlength{\parskip}{1ex}
      \textbf{Parameters}
      \vspace{-1ex}

      \begin{quote}
        \begin{Ventry}{xxxxxxxxx}

          \item[pFlObject]


browser object
            {\it (type=pointer to xfdata.FL\_OBJECT)}

          \item[addedtext]


text to be added
            {\it (type=str)}

        \end{Ventry}

      \end{quote}

\textbf{Note:} 
e.g. fl\_addto\_browser\_chars(brobj, ``some chars'')


\textbf{Status:} 
Tested + Doc + NoDemo = OK


    \end{boxedminipage}

    \label{xformslib:flbrowser:fl_insert_browser_line}
    \index{xformslib \textit{(package)}!xformslib.flbrowser \textit{(module)}!xformslib.flbrowser.fl\_insert\_browser\_line \textit{(function)}}

    \vspace{0.5ex}

\hspace{.8\funcindent}\begin{boxedminipage}{\funcwidth}

    \raggedright \textbf{fl\_insert\_browser\_line}(\textit{pFlObject}, \textit{linenum}, \textit{newtext})

    \vspace{-1.5ex}

    \rule{\textwidth}{0.5\fboxrule}
\setlength{\parskip}{2ex}

Inserts a line in front of a given line in browser. All lines after it
will be shifted. Embedded newline characters don't result in the line
being split up as it's done in the previous functions. Instead they will
rather likely appear as strange looking characters in the text shown. The
only exception is when inserting into an empty browser or after the last
line, then this function works exactly as if you had called
fl\_add\_browser\_line(). See fl\_add\_browser\_line() for appearance change.

-{}-
\setlength{\parskip}{1ex}
      \textbf{Parameters}
      \vspace{-1ex}

      \begin{quote}
        \begin{Ventry}{xxxxxxxxx}

          \item[pFlObject]


browser object
            {\it (type=pointer to xfdata.FL\_OBJECT)}

          \item[linenum]


line number after which insert it (top line is numbered 1, not 0)
            {\it (type=int)}

          \item[newtext]


text to be inserted
            {\it (type=str)}

        \end{Ventry}

      \end{quote}

\textbf{Note:} 
e.g. fl\_insert\_browser\_line(brobj, 1, ``blblabla'')


\textbf{Status:} 
Tested + Doc + Demo = OK


    \end{boxedminipage}

    \label{xformslib:flbrowser:fl_delete_browser_line}
    \index{xformslib \textit{(package)}!xformslib.flbrowser \textit{(module)}!xformslib.flbrowser.fl\_delete\_browser\_line \textit{(function)}}

    \vspace{0.5ex}

\hspace{.8\funcindent}\begin{boxedminipage}{\funcwidth}

    \raggedright \textbf{fl\_delete\_browser\_line}(\textit{pFlObject}, \textit{linenum})

    \vspace{-1.5ex}

    \rule{\textwidth}{0.5\fboxrule}
\setlength{\parskip}{2ex}

Deletes a line (shifting the following lines)

-{}-
\setlength{\parskip}{1ex}
      \textbf{Parameters}
      \vspace{-1ex}

      \begin{quote}
        \begin{Ventry}{xxxxxxxxx}

          \item[pFlObject]


browser object
            {\it (type=pointer to xfdata.FL\_OBJECT)}

          \item[linenum]


line number to delete
            {\it (type=int)}

        \end{Ventry}

      \end{quote}

\textbf{Note:} 
e.g. fl\_delete\_browser\_line(brobj, 3)


\textbf{Status:} 
Tested + Doc + Demo = OK


    \end{boxedminipage}

    \label{xformslib:flbrowser:fl_replace_browser_line}
    \index{xformslib \textit{(package)}!xformslib.flbrowser \textit{(module)}!xformslib.flbrowser.fl\_replace\_browser\_line \textit{(function)}}

    \vspace{0.5ex}

\hspace{.8\funcindent}\begin{boxedminipage}{\funcwidth}

    \raggedright \textbf{fl\_replace\_browser\_line}(\textit{pFlObject}, \textit{linenum}, \textit{newtext})

    \vspace{-1.5ex}

    \rule{\textwidth}{0.5\fboxrule}
\setlength{\parskip}{2ex}

Replaces a line in the browser. As in the case of
fl\_insert\_browser\_line() newline characters embedded into the replacement
text don't have any special meaning, i.e. they don't result in replacement
of more than a single line. See fl\_add\_browser\_line() for appearance
change.

-{}-
\setlength{\parskip}{1ex}
      \textbf{Parameters}
      \vspace{-1ex}

      \begin{quote}
        \begin{Ventry}{xxxxxxxxx}

          \item[pFlObject]


browser object
            {\it (type=pointer to xfdata.FL\_OBJECT)}

          \item[linenum]


line number to replace
            {\it (type=int)}

          \item[newtext]


text line used as replacement
            {\it (type=str)}

        \end{Ventry}

      \end{quote}

\textbf{Note:} 
e.g. fl\_replace\_browser\_line(brobj, 5, ``newblabla'')


\textbf{Status:} 
Tested + Doc + Demo = OK


    \end{boxedminipage}

    \label{xformslib:flbrowser:fl_get_browser_line}
    \index{xformslib \textit{(package)}!xformslib.flbrowser \textit{(module)}!xformslib.flbrowser.fl\_get\_browser\_line \textit{(function)}}

    \vspace{0.5ex}

\hspace{.8\funcindent}\begin{boxedminipage}{\funcwidth}

    \raggedright \textbf{fl\_get\_browser\_line}(\textit{pFlObject}, \textit{linenum})

    \vspace{-1.5ex}

    \rule{\textwidth}{0.5\fboxrule}
\setlength{\parskip}{2ex}

Obtains the contents of a particular line in the browser.

-{}-
\setlength{\parskip}{1ex}
      \textbf{Parameters}
      \vspace{-1ex}

      \begin{quote}
        \begin{Ventry}{xxxxxxxxx}

          \item[pFlObject]


browser object
            {\it (type=pointer to xfdata.FL\_OBJECT)}

          \item[linenum]


line number to return
            {\it (type=int)}

        \end{Ventry}

      \end{quote}

      \textbf{Return Value}
    \vspace{-1ex}

      \begin{quote}

line text
      {\it (type=str)}

      \end{quote}

\textbf{Note:} 
e.g. txt4thline = fl\_get\_browser\_line(brobj, 4)


\textbf{Status:} 
Tested + Doc + Demo = OK


    \end{boxedminipage}

    \label{xformslib:flbrowser:fl_load_browser}
    \index{xformslib \textit{(package)}!xformslib.flbrowser \textit{(module)}!xformslib.flbrowser.fl\_load\_browser \textit{(function)}}

    \vspace{0.5ex}

\hspace{.8\funcindent}\begin{boxedminipage}{\funcwidth}

    \raggedright \textbf{fl\_load\_browser}(\textit{pFlObject}, \textit{filename})

    \vspace{-1.5ex}

    \rule{\textwidth}{0.5\fboxrule}
\setlength{\parskip}{2ex}

Loads an entire file into a browser. An empty string (or the file
couldn't be opened for reading) the browser is just cleared. This
routine is particularly useful when using the browser for a help
facility. You can create different help files and load the needed one
depending on context.

-{}-
\setlength{\parskip}{1ex}
      \textbf{Parameters}
      \vspace{-1ex}

      \begin{quote}
        \begin{Ventry}{xxxxxxxxx}

          \item[pFlObject]


browser object
            {\it (type=pointer to xfdata.FL\_OBJECT)}

          \item[filename]


name of the file
            {\it (type=str)}

        \end{Ventry}

      \end{quote}

      \textbf{Return Value}
    \vspace{-1ex}

      \begin{quote}

1 (if file is successfully loaded), otherwise 0
      {\it (type=int)}

      \end{quote}

\textbf{Note:} 
e.g. ival = fl\_load\_browser(brobj, ``somefile.txt'')


\textbf{Status:} 
Tested + Doc + Demo = OK


    \end{boxedminipage}

    \label{xformslib:flbrowser:fl_select_browser_line}
    \index{xformslib \textit{(package)}!xformslib.flbrowser \textit{(module)}!xformslib.flbrowser.fl\_select\_browser\_line \textit{(function)}}

    \vspace{0.5ex}

\hspace{.8\funcindent}\begin{boxedminipage}{\funcwidth}

    \raggedright \textbf{fl\_select\_browser\_line}(\textit{pFlObject}, \textit{linenum})

    \vspace{-1.5ex}

    \rule{\textwidth}{0.5\fboxrule}
\setlength{\parskip}{2ex}

Selects a line in the browser.

-{}-
\setlength{\parskip}{1ex}
      \textbf{Parameters}
      \vspace{-1ex}

      \begin{quote}
        \begin{Ventry}{xxxxxxxxx}

          \item[pFlObject]


browser object
            {\it (type=pointer to xfdata.FL\_OBJECT)}

          \item[linenum]


line number to select
            {\it (type=int)}

        \end{Ventry}

      \end{quote}

\textbf{Note:} 
e.g. fl\_select\_browser\_line(brobj, 4)


\textbf{Status:} 
Tested + Doc + NoDemo = OK


    \end{boxedminipage}

    \label{xformslib:flbrowser:fl_deselect_browser_line}
    \index{xformslib \textit{(package)}!xformslib.flbrowser \textit{(module)}!xformslib.flbrowser.fl\_deselect\_browser\_line \textit{(function)}}

    \vspace{0.5ex}

\hspace{.8\funcindent}\begin{boxedminipage}{\funcwidth}

    \raggedright \textbf{fl\_deselect\_browser\_line}(\textit{pFlObject}, \textit{linenum})

    \vspace{-1.5ex}

    \rule{\textwidth}{0.5\fboxrule}
\setlength{\parskip}{2ex}

Deselects a line in the browser.

-{}-
\setlength{\parskip}{1ex}
      \textbf{Parameters}
      \vspace{-1ex}

      \begin{quote}
        \begin{Ventry}{xxxxxxxxx}

          \item[pFlObject]


browser object
            {\it (type=pointer to xfdata.FL\_OBJECT)}

          \item[linenum]


line id to deselect
            {\it (type=int)}

        \end{Ventry}

      \end{quote}

\textbf{Note:} 
e.g. fl\_deselect\_browser\_line(brobj, 4)


\textbf{Status:} 
Tested + Doc + Demo = OK


    \end{boxedminipage}

    \label{xformslib:flbrowser:fl_deselect_browser}
    \index{xformslib \textit{(package)}!xformslib.flbrowser \textit{(module)}!xformslib.flbrowser.fl\_deselect\_browser \textit{(function)}}

    \vspace{0.5ex}

\hspace{.8\funcindent}\begin{boxedminipage}{\funcwidth}

    \raggedright \textbf{fl\_deselect\_browser}(\textit{pFlObject})

    \vspace{-1.5ex}

    \rule{\textwidth}{0.5\fboxrule}
\setlength{\parskip}{2ex}

Deselects all lines in the browser.

-{}-
\setlength{\parskip}{1ex}
      \textbf{Parameters}
      \vspace{-1ex}

      \begin{quote}
        \begin{Ventry}{xxxxxxxxx}

          \item[pFlObject]


browser object
            {\it (type=pointer to xfdata.FL\_OBJECT)}

        \end{Ventry}

      \end{quote}

\textbf{Note:} 
e.g. fl\_deselect\_browser(brobj)


\textbf{Status:} 
Tested + Doc + Demo = OK


    \end{boxedminipage}

    \label{xformslib:flbrowser:fl_isselected_browser_line}
    \index{xformslib \textit{(package)}!xformslib.flbrowser \textit{(module)}!xformslib.flbrowser.fl\_isselected\_browser\_line \textit{(function)}}

    \vspace{0.5ex}

\hspace{.8\funcindent}\begin{boxedminipage}{\funcwidth}

    \raggedright \textbf{fl\_isselected\_browser\_line}(\textit{pFlObject}, \textit{linenum})

    \vspace{-1.5ex}

    \rule{\textwidth}{0.5\fboxrule}
\setlength{\parskip}{2ex}

Checks whether a line is selected or not.

-{}-
\setlength{\parskip}{1ex}
      \textbf{Parameters}
      \vspace{-1ex}

      \begin{quote}
        \begin{Ventry}{xxxxxxxxx}

          \item[pFlObject]


browser object
            {\it (type=pointer to xfdata.FL\_OBJECT)}

          \item[linenum]


line id to evaluate
            {\it (type=int)}

        \end{Ventry}

      \end{quote}

      \textbf{Return Value}
    \vspace{-1ex}

      \begin{quote}

num.
      {\it (type=int)}

      \end{quote}

\textbf{Note:} 
e.g. if fl\_isselected\_browser\_line(brobj, 2): ...


\textbf{Status:} 
Tested + Doc + NoDemo = OK


    \end{boxedminipage}

    \label{xformslib:flbrowser:fl_get_browser_topline}
    \index{xformslib \textit{(package)}!xformslib.flbrowser \textit{(module)}!xformslib.flbrowser.fl\_get\_browser\_topline \textit{(function)}}

    \vspace{0.5ex}

\hspace{.8\funcindent}\begin{boxedminipage}{\funcwidth}

    \raggedright \textbf{fl\_get\_browser\_topline}(\textit{pFlObject})

    \vspace{-1.5ex}

    \rule{\textwidth}{0.5\fboxrule}
\setlength{\parskip}{2ex}

Finds out the (un-obscured) line that is currently shown at the top
of the browser. The index of the top line is 1, not 0.

-{}-
\setlength{\parskip}{1ex}
      \textbf{Parameters}
      \vspace{-1ex}

      \begin{quote}
        \begin{Ventry}{xxxxxxxxx}

          \item[pFlObject]


browser object
            {\it (type=pointer to xfdata.FL\_OBJECT)}

        \end{Ventry}

      \end{quote}

      \textbf{Return Value}
    \vspace{-1ex}

      \begin{quote}

line id number
      {\it (type=int)}

      \end{quote}

\textbf{Note:} 
e.g. topl = fl\_get\_browser\_topline(brobj)


\textbf{Status:} 
Tested + Doc + NoDemo = OK


    \end{boxedminipage}

    \label{xformslib:flbrowser:fl_get_browser}
    \index{xformslib \textit{(package)}!xformslib.flbrowser \textit{(module)}!xformslib.flbrowser.fl\_get\_browser \textit{(function)}}

    \vspace{0.5ex}

\hspace{.8\funcindent}\begin{boxedminipage}{\funcwidth}

    \raggedright \textbf{fl\_get\_browser}(\textit{pFlObject})

    \vspace{-1.5ex}

    \rule{\textwidth}{0.5\fboxrule}
\setlength{\parskip}{2ex}

Obtains the last selection made by the user, e.g. when the browser is
returned.

-{}-
\setlength{\parskip}{1ex}
      \textbf{Parameters}
      \vspace{-1ex}

      \begin{quote}
        \begin{Ventry}{xxxxxxxxx}

          \item[pFlObject]


browser object
            {\it (type=pointer to xfdata.FL\_OBJECT)}

        \end{Ventry}

      \end{quote}

      \textbf{Return Value}
    \vspace{-1ex}

      \begin{quote}

line num. of the last selection, or 0 (if no selection was made).
For FL\_MULTI\_BROWSER only: the negative of deselected line num. (when
the last action was a deselection)
      {\it (type=int)}

      \end{quote}

\textbf{Note:} 
e.g. lastsel = fl\_get\_browser(pFlObject)


\textbf{Status:} 
Tested + Doc + Demo = OK


    \end{boxedminipage}

    \label{xformslib:flbrowser:fl_get_browser_maxline}
    \index{xformslib \textit{(package)}!xformslib.flbrowser \textit{(module)}!xformslib.flbrowser.fl\_get\_browser\_maxline \textit{(function)}}

    \vspace{0.5ex}

\hspace{.8\funcindent}\begin{boxedminipage}{\funcwidth}

    \raggedright \textbf{fl\_get\_browser\_maxline}(\textit{pFlObject})

    \vspace{-1.5ex}

    \rule{\textwidth}{0.5\fboxrule}
\setlength{\parskip}{2ex}

Returns the number of lines in the browser.

-{}-
\setlength{\parskip}{1ex}
      \textbf{Parameters}
      \vspace{-1ex}

      \begin{quote}
        \begin{Ventry}{xxxxxxxxx}

          \item[pFlObject]


browser object
            {\it (type=pointer to xfdata.FL\_OBJECT)}

        \end{Ventry}

      \end{quote}

      \textbf{Return Value}
    \vspace{-1ex}

      \begin{quote}

number of lines
      {\it (type=int)}

      \end{quote}

\textbf{Note:} 
e.g. maxlin = fl\_get\_browser\_maxline(brobj)


\textbf{Status:} 
Tested + Doc + NoDemo = OK


    \end{boxedminipage}

    \label{xformslib:flbrowser:fl_get_browser_screenlines}
    \index{xformslib \textit{(package)}!xformslib.flbrowser \textit{(module)}!xformslib.flbrowser.fl\_get\_browser\_screenlines \textit{(function)}}

    \vspace{0.5ex}

\hspace{.8\funcindent}\begin{boxedminipage}{\funcwidth}

    \raggedright \textbf{fl\_get\_browser\_screenlines}(\textit{pFlObject})

    \vspace{-1.5ex}

    \rule{\textwidth}{0.5\fboxrule}
\setlength{\parskip}{2ex}

Returns an approximation of the number of lines shown in the browser.
This count only includes lines that are shown completely in the browser,
partially obscured ones aren't counted in.

-{}-
\setlength{\parskip}{1ex}
      \textbf{Parameters}
      \vspace{-1ex}

      \begin{quote}
        \begin{Ventry}{xxxxxxxxx}

          \item[pFlObject]


browser object
            {\it (type=pointer to xfdata.FL\_OBJECT)}

        \end{Ventry}

      \end{quote}

      \textbf{Return Value}
    \vspace{-1ex}

      \begin{quote}

number of lines
      {\it (type=int)}

      \end{quote}

\textbf{Note:} 
e.g. visiblines = fl\_get\_browser\_screenlines(brobj)


\textbf{Status:} 
Tested + Doc + NoDemo = OK


    \end{boxedminipage}

    \label{xformslib:flbrowser:fl_set_browser_topline}
    \index{xformslib \textit{(package)}!xformslib.flbrowser \textit{(module)}!xformslib.flbrowser.fl\_set\_browser\_topline \textit{(function)}}

    \vspace{0.5ex}

\hspace{.8\funcindent}\begin{boxedminipage}{\funcwidth}

    \raggedright \textbf{fl\_set\_browser\_topline}(\textit{pFlObject}, \textit{linenum})

    \vspace{-1.5ex}

    \rule{\textwidth}{0.5\fboxrule}
\setlength{\parskip}{2ex}

Moves a text line to the top of the browser. Line numbers start with 1.

-{}-
\setlength{\parskip}{1ex}
      \textbf{Parameters}
      \vspace{-1ex}

      \begin{quote}
        \begin{Ventry}{xxxxxxxxx}

          \item[pFlObject]


browser object
            {\it (type=pointer to xfdata.FL\_OBJECT)}

          \item[linenum]


line id to be moved to top
            {\it (type=int)}

        \end{Ventry}

      \end{quote}

\textbf{Note:} 
e.g. fl\_set\_browser\_topline(brobj, 5)


\textbf{Status:} 
Tested + Doc + NoDemo = OK


    \end{boxedminipage}

    \label{xformslib:flbrowser:fl_set_browser_bottomline}
    \index{xformslib \textit{(package)}!xformslib.flbrowser \textit{(module)}!xformslib.flbrowser.fl\_set\_browser\_bottomline \textit{(function)}}

    \vspace{0.5ex}

\hspace{.8\funcindent}\begin{boxedminipage}{\funcwidth}

    \raggedright \textbf{fl\_set\_browser\_bottomline}(\textit{pFlObject}, \textit{linenum})

    \vspace{-1.5ex}

    \rule{\textwidth}{0.5\fboxrule}
\setlength{\parskip}{2ex}

Moves a text line to the bottom of the browser. Line numbers start with
1.

-{}-
\setlength{\parskip}{1ex}
      \textbf{Parameters}
      \vspace{-1ex}

      \begin{quote}
        \begin{Ventry}{xxxxxxxxx}

          \item[pFlObject]


browser object
            {\it (type=pointer to xfdata.FL\_OBJECT)}

          \item[linenum]


line id to be moved to bottom
            {\it (type=int)}

        \end{Ventry}

      \end{quote}

\textbf{Note:} 
e.g. fl\_set\_browser\_bottomline(brobj, 2)


\textbf{Status:} 
Tested + Doc + NoDemo = OK


    \end{boxedminipage}

    \label{xformslib:flbrowser:fl_set_browser_fontsize}
    \index{xformslib \textit{(package)}!xformslib.flbrowser \textit{(module)}!xformslib.flbrowser.fl\_set\_browser\_fontsize \textit{(function)}}

    \vspace{0.5ex}

\hspace{.8\funcindent}\begin{boxedminipage}{\funcwidth}

    \raggedright \textbf{fl\_set\_browser\_fontsize}(\textit{pFlObject}, \textit{size})

    \vspace{-1.5ex}

    \rule{\textwidth}{0.5\fboxrule}
\setlength{\parskip}{2ex}

Sets the font size used inside the browser.

-{}-
\setlength{\parskip}{1ex}
      \textbf{Parameters}
      \vspace{-1ex}

      \begin{quote}
        \begin{Ventry}{xxxxxxxxx}

          \item[pFlObject]


browser object
            {\it (type=pointer to xfdata.FL\_OBJECT)}

          \item[size]


font size to be set. Values (from xfdata.py) FL\_TINY\_SIZE,
FL\_SMALL\_SIZE, FL\_NORMAL\_SIZE, FL\_MEDIUM\_SIZE, FL\_LARGE\_SIZE,
FL\_HUGE\_SIZE, FL\_DEFAULT\_SIZE
            {\it (type=int)}

        \end{Ventry}

      \end{quote}

\textbf{Note:} 
e.g. fl\_set\_browser\_fontsize(brobj, xfdata.FL\_NORMAL\_SIZE)


\textbf{Status:} 
Tested + Doc + Demo = OK


    \end{boxedminipage}

    \label{xformslib:flbrowser:fl_set_browser_fontstyle}
    \index{xformslib \textit{(package)}!xformslib.flbrowser \textit{(module)}!xformslib.flbrowser.fl\_set\_browser\_fontstyle \textit{(function)}}

    \vspace{0.5ex}

\hspace{.8\funcindent}\begin{boxedminipage}{\funcwidth}

    \raggedright \textbf{fl\_set\_browser\_fontstyle}(\textit{pFlObject}, \textit{style})

    \vspace{-1.5ex}

    \rule{\textwidth}{0.5\fboxrule}
\setlength{\parskip}{2ex}

Sets the font style of a browser object.

-{}-
\setlength{\parskip}{1ex}
      \textbf{Parameters}
      \vspace{-1ex}

      \begin{quote}
        \begin{Ventry}{xxxxxxxxx}

          \item[pFlObject]


browser object
            {\it (type=pointer to xfdata.FL\_OBJECT)}

          \item[style]


font style to be set. Values (from xfdata.py) FL\_NORMAL\_STYLE,
FL\_BOLD\_STYLE, FL\_ITALIC\_STYLE, FL\_BOLDITALIC\_STYLE, FL\_FIXED\_STYLE,
FL\_FIXEDBOLD\_STYLE, FL\_FIXEDITALIC\_STYLE, FL\_FIXEDBOLDITALIC\_STYLE,
FL\_TIMES\_STYLE, FL\_TIMESBOLD\_STYLE, FL\_TIMESITALIC\_STYLE,
FL\_TIMESBOLDITALIC\_STYLE, FL\_MISC\_STYLE, FL\_MISCBOLD\_STYLE,
FL\_MISCITALIC\_STYLE, FL\_SYMBOL\_STYLE, FL\_SHADOW\_STYLE,
FL\_ENGRAVED\_STYLE, FL\_EMBOSSED\_STYLE
            {\it (type=int)}

        \end{Ventry}

      \end{quote}

\textbf{Note:} 
e.g. fl\_set\_browser\_fontstyle(brobj, xfdata.FL\_EMBOSSED\_STYLE)


\textbf{Status:} 
Tested + Doc + NoDemo = OK


    \end{boxedminipage}

    \label{xformslib:flbrowser:fl_set_browser_specialkey}
    \index{xformslib \textit{(package)}!xformslib.flbrowser \textit{(module)}!xformslib.flbrowser.fl\_set\_browser\_specialkey \textit{(function)}}

    \vspace{0.5ex}

\hspace{.8\funcindent}\begin{boxedminipage}{\funcwidth}

    \raggedright \textbf{fl\_set\_browser\_specialkey}(\textit{pFlObject}, \textit{specialkey})

    \vspace{-1.5ex}

    \rule{\textwidth}{0.5\fboxrule}
\setlength{\parskip}{2ex}

Changes the special character (used to change appearance, see
fl\_add\_browser\_line()) to something other than '@'. In some cases the
character '@' might need to be placed at the beginning of the lines
without introducing the special meaning mentioned above. In this case you
can use ``@@''.

-{}-
\setlength{\parskip}{1ex}
      \textbf{Parameters}
      \vspace{-1ex}

      \begin{quote}
        \begin{Ventry}{xxxxxxxxxx}

          \item[pFlObject]


browser object
            {\it (type=pointer to xfdata.FL\_OBJECT)}

          \item[specialkey]


escape key to be set
            {\it (type=int or char)}

        \end{Ventry}

      \end{quote}

\textbf{Note:} 
e.g. fl\_set\_browser\_specialkey(brobj, ``|'')


\textbf{Status:} 
Tested + Doc + NoDemo = OK


    \end{boxedminipage}

    \label{xformslib:flbrowser:fl_set_browser_vscrollbar}
    \index{xformslib \textit{(package)}!xformslib.flbrowser \textit{(module)}!xformslib.flbrowser.fl\_set\_browser\_vscrollbar \textit{(function)}}

    \vspace{0.5ex}

\hspace{.8\funcindent}\begin{boxedminipage}{\funcwidth}

    \raggedright \textbf{fl\_set\_browser\_vscrollbar}(\textit{pFlObject}, \textit{how})

    \vspace{-1.5ex}

    \rule{\textwidth}{0.5\fboxrule}
\setlength{\parskip}{2ex}

Turns the vertical scrollbar on or off. When you switch the scrollbars
off the text can't be scrolled by the user anymore at all (i.e. also not
using methods that don't use scrollbars, e.g. using the cursor keys).

-{}-
\setlength{\parskip}{1ex}
      \textbf{Parameters}
      \vspace{-1ex}

      \begin{quote}
        \begin{Ventry}{xxxxxxxxx}

          \item[pFlObject]


browser object
            {\it (type=pointer to xfdata.FL\_OBJECT)}

          \item[how]


how bar is turned. Values (from xfdata.py) FL\_ON, FL\_OFF, FL\_AUTO
(default)
            {\it (type=int)}

        \end{Ventry}

      \end{quote}

\textbf{Note:} 
fl\_set\_browser\_vscrollbar(brobj, xfdata.FL\_OFF)


\textbf{Status:} 
Tested + Doc + Demo = OK


    \end{boxedminipage}

    \label{xformslib:flbrowser:fl_set_browser_hscrollbar}
    \index{xformslib \textit{(package)}!xformslib.flbrowser \textit{(module)}!xformslib.flbrowser.fl\_set\_browser\_hscrollbar \textit{(function)}}

    \vspace{0.5ex}

\hspace{.8\funcindent}\begin{boxedminipage}{\funcwidth}

    \raggedright \textbf{fl\_set\_browser\_hscrollbar}(\textit{pFlObject}, \textit{how})

    \vspace{-1.5ex}

    \rule{\textwidth}{0.5\fboxrule}
\setlength{\parskip}{2ex}

Turns the horizontal scrollbar on or off. When you switch the
scrollbars off the text can't be scrolled by the user anymore at all (i.e.
also not using methods that don't use scrollbars, e.g. using the cursor
keys).

-{}-
\setlength{\parskip}{1ex}
      \textbf{Parameters}
      \vspace{-1ex}

      \begin{quote}
        \begin{Ventry}{xxxxxxxxx}

          \item[pFlObject]


browser object
            {\it (type=pointer to xfdata.FL\_OBJECT)}

          \item[how]


how bar is turned. Values (from xfdata.py) FL\_ON, FL\_OFF, FL\_AUTO
(default)
            {\it (type=int)}

        \end{Ventry}

      \end{quote}

\textbf{Note:} 
fl\_set\_browser\_hscrollbar(brobj, xfdata.FL\_OFF)


\textbf{Status:} 
Tested + Doc + NoDemo = OK


    \end{boxedminipage}

    \label{xformslib:flbrowser:fl_set_browser_line_selectable}
    \index{xformslib \textit{(package)}!xformslib.flbrowser \textit{(module)}!xformslib.flbrowser.fl\_set\_browser\_line\_selectable \textit{(function)}}

    \vspace{0.5ex}

\hspace{.8\funcindent}\begin{boxedminipage}{\funcwidth}

    \raggedright \textbf{fl\_set\_browser\_line\_selectable}(\textit{pFlObject}, \textit{linenum}, \textit{yesno})

    \vspace{-1.5ex}

    \rule{\textwidth}{0.5\fboxrule}
\setlength{\parskip}{2ex}

Sets if a line of browser object is selectable or not.

-{}-
\setlength{\parskip}{1ex}
      \textbf{Parameters}
      \vspace{-1ex}

      \begin{quote}
        \begin{Ventry}{xxxxxxxxx}

          \item[pFlObject]


browser object
            {\it (type=pointer to xfdata.FL\_OBJECT)}

          \item[linenum]


line id to set
            {\it (type=int)}

          \item[yesno]


selectable state. Values 1 (selectable) or 0 (not selectable)
            {\it (type=int)}

        \end{Ventry}

      \end{quote}

\textbf{Status:} 
Untested + NoDoc + NoDemo = NOT OK


    \end{boxedminipage}

    \label{xformslib:flbrowser:fl_get_browser_dimension}
    \index{xformslib \textit{(package)}!xformslib.flbrowser \textit{(module)}!xformslib.flbrowser.fl\_get\_browser\_dimension \textit{(function)}}

    \vspace{0.5ex}

\hspace{.8\funcindent}\begin{boxedminipage}{\funcwidth}

    \raggedright \textbf{fl\_get\_browser\_dimension}(\textit{pFlObject})

    \vspace{-1.5ex}

    \rule{\textwidth}{0.5\fboxrule}
\setlength{\parskip}{2ex}

Obtains the browser size in pixels for the text area.

-{}-
\setlength{\parskip}{1ex}
      \textbf{Parameters}
      \vspace{-1ex}

      \begin{quote}
        \begin{Ventry}{xxxxxxxxx}

          \item[pFlObject]


browser object
            {\it (type=pointer to xfdata.FL\_OBJECT)}

        \end{Ventry}

      \end{quote}

      \textbf{Return Value}
    \vspace{-1ex}

      \begin{quote}

horizontal (x) and vertical position (y), width (w), height (h)
      {\it (type=int, int, int, int)}

      \end{quote}

\textbf{Note:} 
e.g. x, y, wid, hei = fl\_get\_browser\_dimension(brobj)


\textbf{Attention:} 
API change from XForms - upstream was
fl\_get\_browser\_dimension(pFlObject, x, y, w, h)


\textbf{Status:} 
Tested + Doc + NoDemo = OK


    \end{boxedminipage}

    \label{xformslib:flbrowser:fl_set_browser_dblclick_callback}
    \index{xformslib \textit{(package)}!xformslib.flbrowser \textit{(module)}!xformslib.flbrowser.fl\_set\_browser\_dblclick\_callback \textit{(function)}}

    \vspace{0.5ex}

\hspace{.8\funcindent}\begin{boxedminipage}{\funcwidth}

    \raggedright \textbf{fl\_set\_browser\_dblclick\_callback}(\textit{pFlObject}, \textit{py\_CallbackPtr}, \textit{data})

    \vspace{-1.5ex}

    \rule{\textwidth}{0.5\fboxrule}
\setlength{\parskip}{2ex}

Registers a callback function that gets called when a line is
double-clicked on. Double-click callbacks make most sense for
xfdata.FL\_HOLD\_BROWSERs.

-{}-
\setlength{\parskip}{1ex}
      \textbf{Parameters}
      \vspace{-1ex}

      \begin{quote}
        \begin{Ventry}{xxxxxxxxxxxxxx}

          \item[pFlObject]


browser object
            {\it (type=pointer to xfdata.FL\_OBJECT)}

          \item[py\_CallbackPtr]


name referring to function(pObject, longdata)
            {\it (type=python function callback, no return)}

          \item[data]


user data to be passed to function
            {\it (type=long)}

        \end{Ventry}

      \end{quote}

\textbf{Notes:}
\begin{quote}
  \begin{itemize}

  \item
    \setlength{\parskip}{0.6ex}

e.g. def browsercb(pobj, data): > ...


  \item 
e.g. fl\_set\_browser\_dblclick\_callback(brobj, browsercb, data)


\end{itemize}

\end{quote}

\textbf{Status:} 
Tested + Doc + NoDemo = OK


    \end{boxedminipage}

    \label{xformslib:flbrowser:fl_get_browser_xoffset}
    \index{xformslib \textit{(package)}!xformslib.flbrowser \textit{(module)}!xformslib.flbrowser.fl\_get\_browser\_xoffset \textit{(function)}}

    \vspace{0.5ex}

\hspace{.8\funcindent}\begin{boxedminipage}{\funcwidth}

    \raggedright \textbf{fl\_get\_browser\_xoffset}(\textit{pFlObject})

    \vspace{-1.5ex}

    \rule{\textwidth}{0.5\fboxrule}
\setlength{\parskip}{2ex}

Obtains the amount of text that is scrolled in horizontal direction.

-{}-
\setlength{\parskip}{1ex}
      \textbf{Parameters}
      \vspace{-1ex}

      \begin{quote}
        \begin{Ventry}{xxxxxxxxx}

          \item[pFlObject]


browser object
            {\it (type=pointer to xfdata.FL\_OBJECT)}

        \end{Ventry}

      \end{quote}

      \textbf{Return Value}
    \vspace{-1ex}

      \begin{quote}

coord num.
      {\it (type=int)}

      \end{quote}

\textbf{Note:} 
e.g. \emph{todo}


\textbf{Status:} 
Untested + Doc + NoDemo = NOT OK


    \end{boxedminipage}

    \label{xformslib:flbrowser:fl_get_browser_rel_xoffset}
    \index{xformslib \textit{(package)}!xformslib.flbrowser \textit{(module)}!xformslib.flbrowser.fl\_get\_browser\_rel\_xoffset \textit{(function)}}

    \vspace{0.5ex}

\hspace{.8\funcindent}\begin{boxedminipage}{\funcwidth}

    \raggedright \textbf{fl\_get\_browser\_rel\_xoffset}(\textit{pFlObject})

    \vspace{-1.5ex}

    \rule{\textwidth}{0.5\fboxrule}
\setlength{\parskip}{2ex}

Obtains the relative amount of text that is scrolled in horizontal
direction.

-{}-
\setlength{\parskip}{1ex}
      \textbf{Parameters}
      \vspace{-1ex}

      \begin{quote}
        \begin{Ventry}{xxxxxxxxx}

          \item[pFlObject]


browser object
            {\it (type=pointer to xfdata.FL\_OBJECT)}

        \end{Ventry}

      \end{quote}

      \textbf{Return Value}
    \vspace{-1ex}

      \begin{quote}

relative offset
      {\it (type=float)}

      \end{quote}

\textbf{Note:} 
e.g. \emph{todo}


\textbf{Status:} 
Untested + Doc + NoDemo = NOT OK


    \end{boxedminipage}

    \label{xformslib:flbrowser:fl_set_browser_xoffset}
    \index{xformslib \textit{(package)}!xformslib.flbrowser \textit{(module)}!xformslib.flbrowser.fl\_set\_browser\_xoffset \textit{(function)}}

    \vspace{0.5ex}

\hspace{.8\funcindent}\begin{boxedminipage}{\funcwidth}

    \raggedright \textbf{fl\_set\_browser\_xoffset}(\textit{pFlObject}, \textit{npixels})

    \vspace{-1.5ex}

    \rule{\textwidth}{0.5\fboxrule}
\setlength{\parskip}{2ex}

Sets the amount of text that is scrolled in horizontal direction.

-{}-
\setlength{\parskip}{1ex}
      \textbf{Parameters}
      \vspace{-1ex}

      \begin{quote}
        \begin{Ventry}{xxxxxxxxx}

          \item[pFlObject]


browser object
            {\it (type=pointer to xfdata.FL\_OBJECT)}

          \item[npixels]


amount of text to be scrolled
            {\it (type=int)}

        \end{Ventry}

      \end{quote}

\textbf{Note:} 
e.g. \emph{todo}


\textbf{Status:} 
Untested + Doc + NoDemo = NOT OK


    \end{boxedminipage}

    \label{xformslib:flbrowser:fl_set_browser_rel_xoffset}
    \index{xformslib \textit{(package)}!xformslib.flbrowser \textit{(module)}!xformslib.flbrowser.fl\_set\_browser\_rel\_xoffset \textit{(function)}}

    \vspace{0.5ex}

\hspace{.8\funcindent}\begin{boxedminipage}{\funcwidth}

    \raggedright \textbf{fl\_set\_browser\_rel\_xoffset}(\textit{pFlObject}, \textit{val})

    \vspace{-1.5ex}

    \rule{\textwidth}{0.5\fboxrule}
\setlength{\parskip}{2ex}

Sets the relative amount of text that is scrolled in horizontal
direction.

-{}-
\setlength{\parskip}{1ex}
      \textbf{Parameters}
      \vspace{-1ex}

      \begin{quote}
        \begin{Ventry}{xxxxxxxxx}

          \item[pFlObject]


browser object
            {\it (type=pointer to xfdata.FL\_OBJECT)}

          \item[val]


relative amount to be scrolled
            {\it (type=float)}

        \end{Ventry}

      \end{quote}

\textbf{Note:} 
e.g. \emph{todo}


\textbf{Status:} 
Untested + NoDoc + NoDemo = NOT OK


    \end{boxedminipage}

    \label{xformslib:flbrowser:fl_get_browser_yoffset}
    \index{xformslib \textit{(package)}!xformslib.flbrowser \textit{(module)}!xformslib.flbrowser.fl\_get\_browser\_yoffset \textit{(function)}}

    \vspace{0.5ex}

\hspace{.8\funcindent}\begin{boxedminipage}{\funcwidth}

    \raggedright \textbf{fl\_get\_browser\_yoffset}(\textit{pFlObject})

    \vspace{-1.5ex}

    \rule{\textwidth}{0.5\fboxrule}
\setlength{\parskip}{2ex}

Obtains the amount of text that is scrolled in horizontal direction.

-{}-
\setlength{\parskip}{1ex}
      \textbf{Parameters}
      \vspace{-1ex}

      \begin{quote}
        \begin{Ventry}{xxxxxxxxx}

          \item[pFlObject]


browser object
            {\it (type=pointer to xfdata.FL\_OBJECT)}

        \end{Ventry}

      \end{quote}

      \textbf{Return Value}
    \vspace{-1ex}

      \begin{quote}

coord. num.
      {\it (type=int)}

      \end{quote}

\textbf{Note:} 
e.g. \emph{todo}


\textbf{Status:} 
Tested + Doc + Demo = OK


    \end{boxedminipage}

    \label{xformslib:flbrowser:fl_get_browser_rel_yoffset}
    \index{xformslib \textit{(package)}!xformslib.flbrowser \textit{(module)}!xformslib.flbrowser.fl\_get\_browser\_rel\_yoffset \textit{(function)}}

    \vspace{0.5ex}

\hspace{.8\funcindent}\begin{boxedminipage}{\funcwidth}

    \raggedright \textbf{fl\_get\_browser\_rel\_yoffset}(\textit{pFlObject})

    \vspace{-1.5ex}

    \rule{\textwidth}{0.5\fboxrule}
\setlength{\parskip}{2ex}

Obtains the relative amount of text that is scrolled in horizontal
direction.

-{}-
\setlength{\parskip}{1ex}
      \textbf{Parameters}
      \vspace{-1ex}

      \begin{quote}
        \begin{Ventry}{xxxxxxxxx}

          \item[pFlObject]


browser object
            {\it (type=pointer to xfdata.FL\_OBJECT)}

        \end{Ventry}

      \end{quote}

      \textbf{Return Value}
    \vspace{-1ex}

      \begin{quote}

relative offset
      {\it (type=float)}

      \end{quote}

\textbf{Note:} 
e.g. \emph{todo}


\textbf{Status:} 
Untested + Doc + NoDemo = NOT OK


    \end{boxedminipage}

    \label{xformslib:flbrowser:fl_set_browser_yoffset}
    \index{xformslib \textit{(package)}!xformslib.flbrowser \textit{(module)}!xformslib.flbrowser.fl\_set\_browser\_yoffset \textit{(function)}}

    \vspace{0.5ex}

\hspace{.8\funcindent}\begin{boxedminipage}{\funcwidth}

    \raggedright \textbf{fl\_set\_browser\_yoffset}(\textit{pFlObject}, \textit{npixels})

    \vspace{-1.5ex}

    \rule{\textwidth}{0.5\fboxrule}
\setlength{\parskip}{2ex}

Sets the amount of text that is scrolled in horizontal direction.

-{}-
\setlength{\parskip}{1ex}
      \textbf{Parameters}
      \vspace{-1ex}

      \begin{quote}
        \begin{Ventry}{xxxxxxxxx}

          \item[pFlObject]


browser object
            {\it (type=pointer to xfdata.FL\_OBJECT)}

          \item[npixels]


amount of text to be scrolled
            {\it (type=int)}

        \end{Ventry}

      \end{quote}

\textbf{Note:} 
e.g. \emph{todo}


\textbf{Status:} 
Tested + NoDoc + Demo = OK


    \end{boxedminipage}

    \label{xformslib:flbrowser:fl_set_browser_rel_yoffset}
    \index{xformslib \textit{(package)}!xformslib.flbrowser \textit{(module)}!xformslib.flbrowser.fl\_set\_browser\_rel\_yoffset \textit{(function)}}

    \vspace{0.5ex}

\hspace{.8\funcindent}\begin{boxedminipage}{\funcwidth}

    \raggedright \textbf{fl\_set\_browser\_rel\_yoffset}(\textit{pFlObject}, \textit{val})

    \vspace{-1.5ex}

    \rule{\textwidth}{0.5\fboxrule}
\setlength{\parskip}{2ex}

Sets the relative amount of text that is scrolled in horizontal
direction.

-{}-
\setlength{\parskip}{1ex}
      \textbf{Parameters}
      \vspace{-1ex}

      \begin{quote}
        \begin{Ventry}{xxxxxxxxx}

          \item[pFlObject]


browser object
            {\it (type=pointer to xfdata.FL\_OBJECT)}

          \item[val]


relative amount of text to be scrolled
            {\it (type=float)}

        \end{Ventry}

      \end{quote}

\textbf{Note:} 
e.g. \emph{todo}


\textbf{Status:} 
Untested + Doc + NoDemo = NOT OK


    \end{boxedminipage}

    \label{xformslib:flbrowser:fl_set_browser_scrollbarsize}
    \index{xformslib \textit{(package)}!xformslib.flbrowser \textit{(module)}!xformslib.flbrowser.fl\_set\_browser\_scrollbarsize \textit{(function)}}

    \vspace{0.5ex}

\hspace{.8\funcindent}\begin{boxedminipage}{\funcwidth}

    \raggedright \textbf{fl\_set\_browser\_scrollbarsize}(\textit{pFlObject}, \textit{hh}, \textit{vw})

    \vspace{-1.5ex}

    \rule{\textwidth}{0.5\fboxrule}
\setlength{\parskip}{2ex}

Sets the scrollbar size of the browser. By default, the scrollbar size
is based on the relation between the size of the browser and the size of
the text.

-{}-
\setlength{\parskip}{1ex}
      \textbf{Parameters}
      \vspace{-1ex}

      \begin{quote}
        \begin{Ventry}{xxxxxxxxx}

          \item[pFlObject]


browser object
            {\it (type=pointer to xfdata.FL\_OBJECT)}

          \item[hh]


horizontal scrollbar height (0 for the default)
            {\it (type=int)}

          \item[vw]


vertical scrollbar width (0 for the default)
            {\it (type=int)}

        \end{Ventry}

      \end{quote}

\textbf{Note:} 
e.g. fl\_set\_browser\_scrollbarsize(brobj, 10, 10)


\textbf{Status:} 
Tested + Doc + NoDemo = OK


    \end{boxedminipage}

    \label{xformslib:flbrowser:fl_show_browser_line}
    \index{xformslib \textit{(package)}!xformslib.flbrowser \textit{(module)}!xformslib.flbrowser.fl\_show\_browser\_line \textit{(function)}}

    \vspace{0.5ex}

\hspace{.8\funcindent}\begin{boxedminipage}{\funcwidth}

    \raggedright \textbf{fl\_show\_browser\_line}(\textit{pFlObject}, \textit{line})

    \vspace{-1.5ex}

    \rule{\textwidth}{0.5\fboxrule}
\setlength{\parskip}{2ex}

Brings a browser line into view.

-{}-
\setlength{\parskip}{1ex}
      \textbf{Parameters}
      \vspace{-1ex}

      \begin{quote}
        \begin{Ventry}{xxxxxxxxx}

          \item[pFlObject]


browser object
            {\it (type=pointer to xfdata.FL\_OBJECT)}

          \item[line]


line number to show
            {\it (type=int)}

        \end{Ventry}

      \end{quote}

\textbf{Note:} 
e.g. fl\_show\_browser\_line(brobj, 12)


\textbf{Status:} 
Tested + Doc + NoDemo = OK


    \end{boxedminipage}

    \label{xformslib:flbrowser:fl_set_browser_hscroll_callback}
    \index{xformslib \textit{(package)}!xformslib.flbrowser \textit{(module)}!xformslib.flbrowser.fl\_set\_browser\_hscroll\_callback \textit{(function)}}

    \vspace{0.5ex}

\hspace{.8\funcindent}\begin{boxedminipage}{\funcwidth}

    \raggedright \textbf{fl\_set\_browser\_hscroll\_callback}(\textit{pFlObject}, \textit{py\_BrowserScrollCallback}, \textit{vdata})

    \vspace{-1.5ex}

    \rule{\textwidth}{0.5\fboxrule}
\setlength{\parskip}{2ex}

Sets the callback function to be invoked whenever the horizontal
scrollbar changes position.

-{}-
\setlength{\parskip}{1ex}
      \textbf{Parameters}
      \vspace{-1ex}

      \begin{quote}
        \begin{Ventry}{xxxxxxxxxxxxxxxxxxxxxxxx}

          \item[pFlObject]


browser object
            {\it (type=pointer to xfdata.FL\_OBJECT)}

          \item[py\_BrowserScrollCallback]


name referring to function(pFlObject, num, vdata)
            {\it (type=python function callback, no return)}

          \item[vdata]


user data to be passed to function; callback has to take care of
type check
            {\it (type=any type (e.g. 'None', int, str, etc..))}

        \end{Ventry}

      \end{quote}

\textbf{Note:} 
e.g. \emph{todo}


\textbf{Status:} 
Untested + Doc + NoDemo = NOT OK


    \end{boxedminipage}

    \label{xformslib:flbrowser:fl_set_browser_vscroll_callback}
    \index{xformslib \textit{(package)}!xformslib.flbrowser \textit{(module)}!xformslib.flbrowser.fl\_set\_browser\_vscroll\_callback \textit{(function)}}

    \vspace{0.5ex}

\hspace{.8\funcindent}\begin{boxedminipage}{\funcwidth}

    \raggedright \textbf{fl\_set\_browser\_vscroll\_callback}(\textit{pFlObject}, \textit{py\_BrowserScrollCallback}, \textit{vdata})

    \vspace{-1.5ex}

    \rule{\textwidth}{0.5\fboxrule}
\setlength{\parskip}{2ex}

Sets the callback function to be invoked whenever the vertical
scrollbar changes position.

-{}-
\setlength{\parskip}{1ex}
      \textbf{Parameters}
      \vspace{-1ex}

      \begin{quote}
        \begin{Ventry}{xxxxxxxxxxxxxxxxxxxxxxxx}

          \item[pFlObject]


browser object
            {\it (type=pointer to xfdata.FL\_OBJECT)}

          \item[py\_BrowserScrollCallback]


name referring to function(pFlObject, num, vdata)
            {\it (type=python function callback, no return)}

          \item[vdata]


user data to be passed to function; callback has to take care of
type check
            {\it (type=any type (e.g. 'None', int, str, etc..))}

        \end{Ventry}

      \end{quote}

\textbf{Note:} 
e.g. \emph{todo}


\textbf{Status:} 
Tested + NoDoc + Demo = OK


    \end{boxedminipage}

    \label{xformslib:flbrowser:fl_get_browser_line_yoffset}
    \index{xformslib \textit{(package)}!xformslib.flbrowser \textit{(module)}!xformslib.flbrowser.fl\_get\_browser\_line\_yoffset \textit{(function)}}

    \vspace{0.5ex}

\hspace{.8\funcindent}\begin{boxedminipage}{\funcwidth}

    \raggedright \textbf{fl\_get\_browser\_line\_yoffset}(\textit{pFlObject}, \textit{line})

    \vspace{-1.5ex}

    \rule{\textwidth}{0.5\fboxrule}
\setlength{\parskip}{2ex}

Returns the y-offset for a line.

-{}-
\setlength{\parskip}{1ex}
      \textbf{Parameters}
      \vspace{-1ex}

      \begin{quote}
        \begin{Ventry}{xxxxxxxxx}

          \item[pFlObject]


browser object
            {\it (type=pointer to xfdata.FL\_OBJECT)}

          \item[line]


line number
            {\it (type=int)}

        \end{Ventry}

      \end{quote}

      \textbf{Return Value}
    \vspace{-1ex}

      \begin{quote}

offset of the line, or -1 (if the line does not exist)
      {\it (type=int)}

      \end{quote}

\textbf{Status:} 
Untested + Doc + NoDemo = NOT OK


    \end{boxedminipage}

    \label{xformslib:flbrowser:fl_get_browser_hscroll_callback}
    \index{xformslib \textit{(package)}!xformslib.flbrowser \textit{(module)}!xformslib.flbrowser.fl\_get\_browser\_hscroll\_callback \textit{(function)}}

    \vspace{0.5ex}

\hspace{.8\funcindent}\begin{boxedminipage}{\funcwidth}

    \raggedright \textbf{fl\_get\_browser\_hscroll\_callback}(\textit{pFlObject})

    \vspace{-1.5ex}

    \rule{\textwidth}{0.5\fboxrule}
\setlength{\parskip}{2ex}

Obtains the callback function created for horizontal scrollbar position
change.

-{}-
\setlength{\parskip}{1ex}
      \textbf{Parameters}
      \vspace{-1ex}

      \begin{quote}
        \begin{Ventry}{xxxxxxxxx}

          \item[pFlObject]


browser object
            {\it (type=pointer to xfdata.FL\_OBJECT)}

        \end{Ventry}

      \end{quote}

      \textbf{Return Value}
    \vspace{-1ex}

      \begin{quote}

xfdata.FL\_BROWSER\_SCROLL\_CALLBACK function
      \end{quote}

\textbf{Note:} 
e.g. \emph{todo}


\textbf{Status:} 
Untested + Doc + NoDemo = NOT OK


    \end{boxedminipage}

    \label{xformslib:flbrowser:fl_get_browser_vscroll_callback}
    \index{xformslib \textit{(package)}!xformslib.flbrowser \textit{(module)}!xformslib.flbrowser.fl\_get\_browser\_vscroll\_callback \textit{(function)}}

    \vspace{0.5ex}

\hspace{.8\funcindent}\begin{boxedminipage}{\funcwidth}

    \raggedright \textbf{fl\_get\_browser\_vscroll\_callback}(\textit{pFlObject})

    \vspace{-1.5ex}

    \rule{\textwidth}{0.5\fboxrule}
\setlength{\parskip}{2ex}

Obtains the callback function created for horizontal scrollbar position
change.

-{}-
\setlength{\parskip}{1ex}
      \textbf{Parameters}
      \vspace{-1ex}

      \begin{quote}
        \begin{Ventry}{xxxxxxxxx}

          \item[pFlObject]


browser object
            {\it (type=pointer to xfdata.FL\_OBJECT)}

        \end{Ventry}

      \end{quote}

      \textbf{Return Value}
    \vspace{-1ex}

      \begin{quote}

xfdata.FL\_BROWSER\_SCROLL\_CALLBACK function
      \end{quote}

\textbf{Note:} 
e.g. \emph{todo}


\textbf{Status:} 
Untested + Doc + NoDemo = NOT OK


    \end{boxedminipage}


%%%%%%%%%%%%%%%%%%%%%%%%%%%%%%%%%%%%%%%%%%%%%%%%%%%%%%%%%%%%%%%%%%%%%%%%%%%
%%                               Variables                               %%
%%%%%%%%%%%%%%%%%%%%%%%%%%%%%%%%%%%%%%%%%%%%%%%%%%%%%%%%%%%%%%%%%%%%%%%%%%%

  \subsection{Variables}

    \vspace{-1cm}
\hspace{\varindent}\begin{longtable}{|p{\varnamewidth}|p{\vardescrwidth}|l}
\cline{1-2}
\cline{1-2} \centering \textbf{Name} & \centering \textbf{Description}& \\
\cline{1-2}
\endhead\cline{1-2}\multicolumn{3}{r}{\small\textit{continued on next page}}\\\endfoot\cline{1-2}
\endlastfoot\raggedright \_\-\_\-p\-a\-c\-k\-a\-g\-e\-\_\-\_\- & \raggedright \textbf{Value:} 
{\tt \texttt{'}\texttt{xformslib}\texttt{'}}&\\
\cline{1-2}
\end{longtable}

    \index{xformslib \textit{(package)}!xformslib.flbrowser \textit{(module)}|)}
