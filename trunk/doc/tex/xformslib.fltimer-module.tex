%
% API Documentation for API Documentation
% Module xformslib.fltimer
%
% Generated by epydoc 3.0.1
% [Fri May 21 15:38:50 2010]
%

%%%%%%%%%%%%%%%%%%%%%%%%%%%%%%%%%%%%%%%%%%%%%%%%%%%%%%%%%%%%%%%%%%%%%%%%%%%
%%                          Module Description                           %%
%%%%%%%%%%%%%%%%%%%%%%%%%%%%%%%%%%%%%%%%%%%%%%%%%%%%%%%%%%%%%%%%%%%%%%%%%%%

    \index{xformslib \textit{(package)}!xformslib.fltimer \textit{(module)}|(}
\section{Module xformslib.fltimer}

    \label{xformslib:fltimer}

xforms-python's functions to manage timer objects.

Copyright (C) 2009, 2010  Luca Lazzaroni ``LukenShiro''
e-mail: <\href{mailto:lukenshiro@ngi.it}{lukenshiro@ngi.it}>

This program is free software: you can redistribute it and/or modify
it under the terms of the GNU Lesser General Public License as
published by the Free Software Foundation, version 2.1 of the License.

This program is distributed in the hope that it will be useful,
but WITHOUT ANY WARRANTY; without even the implied warranty of
MERCHANTABILITY or FITNESS FOR A PARTICULAR PURPOSE. See the
GNU Lesser General Public License for more details.

You should have received a copy of the GNU LGPL along with this
program. If not, see <\href{http://www.gnu.org/licenses/}{http://www.gnu.org/licenses/}>.

See CREDITS file to read acknowledgements and thanks to XForms,
ctypes and other developers.

%%%%%%%%%%%%%%%%%%%%%%%%%%%%%%%%%%%%%%%%%%%%%%%%%%%%%%%%%%%%%%%%%%%%%%%%%%%
%%                               Functions                               %%
%%%%%%%%%%%%%%%%%%%%%%%%%%%%%%%%%%%%%%%%%%%%%%%%%%%%%%%%%%%%%%%%%%%%%%%%%%%

  \subsection{Functions}

    \label{xformslib:fltimer:fl_add_timer}
    \index{xformslib \textit{(package)}!xformslib.fltimer \textit{(module)}!xformslib.fltimer.fl\_add\_timer \textit{(function)}}

    \vspace{0.5ex}

\hspace{.8\funcindent}\begin{boxedminipage}{\funcwidth}

    \raggedright \textbf{fl\_add\_timer}(\textit{timertype}, \textit{x}, \textit{y}, \textit{w}, \textit{h}, \textit{label})

    \vspace{-1.5ex}

    \rule{\textwidth}{0.5\fboxrule}
\setlength{\parskip}{2ex}

Adds a timer object.

-{}-
\setlength{\parskip}{1ex}
      \textbf{Parameters}
      \vspace{-1ex}

      \begin{quote}
        \begin{Ventry}{xxxxxxxxx}

          \item[timertype]


type of timer to be added. Values (from xfdata.py) FL\_NORMAL\_TIMER,
FL\_VALUE\_TIMER, FL\_HIDDEN\_TIMER
            {\it (type=int)}

          \item[x]


horizontal position (upper-left corner)
            {\it (type=int)}

          \item[y]


vertical position (upper-left corner)
            {\it (type=int)}

          \item[w]


width in coord units
            {\it (type=int)}

          \item[h]


height in coord units
            {\it (type=int)}

          \item[label]


text label of timer
            {\it (type=str)}

        \end{Ventry}

      \end{quote}

      \textbf{Return Value}
    \vspace{-1ex}

      \begin{quote}

timer object added (pFlObject)
      {\it (type=pointer to xfdata.FL\_OBJECT)}

      \end{quote}

\textbf{Note:} 
e.g. ptimerobj = fl\_add\_timer(xfdata.FL\_NORMAL\_TIMER, 120, 120,
210, 210, ``My Timer'')


\textbf{Status:} 
Tested + NoDoc + Demo = OK


    \end{boxedminipage}

    \label{xformslib:fltimer:fl_set_timer}
    \index{xformslib \textit{(package)}!xformslib.fltimer \textit{(module)}!xformslib.fltimer.fl\_set\_timer \textit{(function)}}

    \vspace{0.5ex}

\hspace{.8\funcindent}\begin{boxedminipage}{\funcwidth}

    \raggedright \textbf{fl\_set\_timer}(\textit{pFlObject}, \textit{delay})

    \vspace{-1.5ex}

    \rule{\textwidth}{0.5\fboxrule}
\setlength{\parskip}{2ex}

Sets the timer to a particular value.

-{}-
\setlength{\parskip}{1ex}
      \textbf{Parameters}
      \vspace{-1ex}

      \begin{quote}
        \begin{Ventry}{xxxxxxxxx}

          \item[pFlObject]


timer object
            {\it (type=pointer to xfdata.FL\_OBJECT)}

          \item[delay]


number of seconds the timer should run. If it's 0.0, resets/de-blinks
the timer.
            {\it (type=float)}

        \end{Ventry}

      \end{quote}

\textbf{Note:} 
e.g. fl\_set\_timer(ptimerobj, 20)


\textbf{Status:} 
Tested + NoDoc + Demo = OK


    \end{boxedminipage}

    \label{xformslib:fltimer:fl_get_timer}
    \index{xformslib \textit{(package)}!xformslib.fltimer \textit{(module)}!xformslib.fltimer.fl\_get\_timer \textit{(function)}}

    \vspace{0.5ex}

\hspace{.8\funcindent}\begin{boxedminipage}{\funcwidth}

    \raggedright \textbf{fl\_get\_timer}(\textit{pFlObject})

    \vspace{-1.5ex}

    \rule{\textwidth}{0.5\fboxrule}
\setlength{\parskip}{2ex}

Obtains the time left in the timer.

-{}-
\setlength{\parskip}{1ex}
      \textbf{Parameters}
      \vspace{-1ex}

      \begin{quote}
        \begin{Ventry}{xxxxxxxxx}

          \item[pFlObject]


timer object
            {\it (type=pointer to xfdata.FL\_OBJECT)}

        \end{Ventry}

      \end{quote}

      \textbf{Return Value}
    \vspace{-1ex}

      \begin{quote}

time left
      {\it (type=float)}

      \end{quote}

\textbf{Note:} 
e.g. lefttim = fl\_get\_timer(ptimerobj)


\textbf{Status:} 
Untested + NoDoc + NoDemo = NOT OK


    \end{boxedminipage}

    \label{xformslib:fltimer:fl_set_timer_countup}
    \index{xformslib \textit{(package)}!xformslib.fltimer \textit{(module)}!xformslib.fltimer.fl\_set\_timer\_countup \textit{(function)}}

    \vspace{0.5ex}

\hspace{.8\funcindent}\begin{boxedminipage}{\funcwidth}

    \raggedright \textbf{fl\_set\_timer\_countup}(\textit{pFlObject}, \textit{yesno})

    \vspace{-1.5ex}

    \rule{\textwidth}{0.5\fboxrule}
\setlength{\parskip}{2ex}

Changes timer behavior so the timer counts up and shows elapsed time.
By default, a timer counts down toward zero and the value shown (for
xfdata.FL\_VALUE\_TIMERs) is the time left until the timer expires.

-{}-
\setlength{\parskip}{1ex}
      \textbf{Parameters}
      \vspace{-1ex}

      \begin{quote}
        \begin{Ventry}{xxxxxxxxx}

          \item[pFlObject]


timer object
            {\it (type=pointer to xfdata.FL\_OBJECT)}

          \item[yesno]


flag to set count up or down. Values 0 (counts down and shows time
left) or 1 (counts up and shows elapsed time)
            {\it (type=int)}

        \end{Ventry}

      \end{quote}

\textbf{Note:} 
e.g. fl\_set\_timer\_countup(ptimerobj, 1)


\textbf{Status:} 
Tested + NoDoc + Demo = OK


    \end{boxedminipage}

    \label{xformslib:fltimer:fl_set_timer_filter}
    \index{xformslib \textit{(package)}!xformslib.fltimer \textit{(module)}!xformslib.fltimer.fl\_set\_timer\_filter \textit{(function)}}

    \vspace{0.5ex}

\hspace{.8\funcindent}\begin{boxedminipage}{\funcwidth}

    \raggedright \textbf{fl\_set\_timer\_filter}(\textit{pFlObject}, \textit{py\_TimerFilter})

    \vspace{-1.5ex}

    \rule{\textwidth}{0.5\fboxrule}
\setlength{\parskip}{2ex}

Sets a function to change the way the time is presented in
xfdata.FL\_VALUE\_TIMER. By default, it returns the time in a
hour:minutes:seconds.fraction format

-{}-
\setlength{\parskip}{1ex}
      \textbf{Parameters}
      \vspace{-1ex}

      \begin{quote}
        \begin{Ventry}{xxxxxxxxxxxxxx}

          \item[pFlObject]


timer object
            {\it (type=pointer to xfdata.FL\_OBJECT)}

          \item[py\_TimerFilter]


name referring to function(pFlObject, float secs) -> str
Parameter secs is time left for count-down timers and the elapsed
time for up-counting timers (in units of seconds). Returns string
representation of time.
            {\it (type=python callback function, returning value)}

        \end{Ventry}

      \end{quote}

      \textbf{Return Value}
    \vspace{-1ex}

      \begin{quote}

old timer filter function
      {\it (type=pointer to xfdata.FL\_TIMER\_FILTER)}

      \end{quote}

\textbf{Notes:}
\begin{quote}
  \begin{itemize}

  \item
    \setlength{\parskip}{0.6ex}

e.g. def timefilt(pobj, elapsedsecs): > ... ; return newstr


  \item 
e.g. oldtimerfunc = fl\_set\_timer\_filter(ptimerobj, timefilt)


\end{itemize}

\end{quote}

\textbf{Status:} 
Untested + NoDoc + NoDemo = NOT OK


    \end{boxedminipage}

    \label{xformslib:fltimer:fl_suspend_timer}
    \index{xformslib \textit{(package)}!xformslib.fltimer \textit{(module)}!xformslib.fltimer.fl\_suspend\_timer \textit{(function)}}

    \vspace{0.5ex}

\hspace{.8\funcindent}\begin{boxedminipage}{\funcwidth}

    \raggedright \textbf{fl\_suspend\_timer}(\textit{pFlObject})

    \vspace{-1.5ex}

    \rule{\textwidth}{0.5\fboxrule}
\setlength{\parskip}{2ex}

Suspends timer, pausing time.

-{}-
\setlength{\parskip}{1ex}
      \textbf{Parameters}
      \vspace{-1ex}

      \begin{quote}
        \begin{Ventry}{xxxxxxxxx}

          \item[pFlObject]


timer object
            {\it (type=pointer to xfdata.FL\_OBJECT)}

        \end{Ventry}

      \end{quote}

\textbf{Note:} 
e.g. fl\_suspend\_timer(ptimerobj)


\textbf{Status:} 
Tested + Doc + Demo = OK


    \end{boxedminipage}

    \label{xformslib:fltimer:fl_resume_timer}
    \index{xformslib \textit{(package)}!xformslib.fltimer \textit{(module)}!xformslib.fltimer.fl\_resume\_timer \textit{(function)}}

    \vspace{0.5ex}

\hspace{.8\funcindent}\begin{boxedminipage}{\funcwidth}

    \raggedright \textbf{fl\_resume\_timer}(\textit{pFlObject})

    \vspace{-1.5ex}

    \rule{\textwidth}{0.5\fboxrule}
\setlength{\parskip}{2ex}

Resumes timer previously paused (with fl\_suspend\_timer). Unlike
fl\_set\_timer() a suspended timer keeps its internal state (total delay,
time left etc.), so when it is resumed, it starts from where it was
suspended.

-{}-
\setlength{\parskip}{1ex}
      \textbf{Parameters}
      \vspace{-1ex}

      \begin{quote}
        \begin{Ventry}{xxxxxxxxx}

          \item[pFlObject]


timer object
            {\it (type=pointer to xfdata.FL\_OBJECT)}

        \end{Ventry}

      \end{quote}

\textbf{Note:} 
e.g. fl\_resume\_timer(ptimobj)


\textbf{Status:} 
Tested + Doc + Demo = OK


    \end{boxedminipage}

    \index{xformslib \textit{(package)}!xformslib.fltimer \textit{(module)}|)}
