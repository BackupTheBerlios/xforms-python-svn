%
% API Documentation for API Documentation
% Module xformslib.flpositioner
%
% Generated by epydoc 3.0.1
% [Fri May 21 19:29:32 2010]
%

%%%%%%%%%%%%%%%%%%%%%%%%%%%%%%%%%%%%%%%%%%%%%%%%%%%%%%%%%%%%%%%%%%%%%%%%%%%
%%                          Module Description                           %%
%%%%%%%%%%%%%%%%%%%%%%%%%%%%%%%%%%%%%%%%%%%%%%%%%%%%%%%%%%%%%%%%%%%%%%%%%%%

    \index{xformslib \textit{(package)}!xformslib.flpositioner \textit{(module)}|(}
\section{Module xformslib.flpositioner}

    \label{xformslib:flpositioner}

xforms-python's functions to manage positioner objects.

Copyright (C) 2009, 2010  Luca Lazzaroni ``LukenShiro''
e-mail: <\href{mailto:lukenshiro@ngi.it}{lukenshiro@ngi.it}>

This program is free software: you can redistribute it and/or modify
it under the terms of the GNU Lesser General Public License as
published by the Free Software Foundation, version 2.1 of the License.

This program is distributed in the hope that it will be useful,
but WITHOUT ANY WARRANTY; without even the implied warranty of
MERCHANTABILITY or FITNESS FOR A PARTICULAR PURPOSE. See the
GNU Lesser General Public License for more details.

You should have received a copy of the GNU LGPL along with this
program. If not, see <\href{http://www.gnu.org/licenses/}{http://www.gnu.org/licenses/}>.

See CREDITS file to read acknowledgements and thanks to XForms,
ctypes and other developers.

%%%%%%%%%%%%%%%%%%%%%%%%%%%%%%%%%%%%%%%%%%%%%%%%%%%%%%%%%%%%%%%%%%%%%%%%%%%
%%                               Functions                               %%
%%%%%%%%%%%%%%%%%%%%%%%%%%%%%%%%%%%%%%%%%%%%%%%%%%%%%%%%%%%%%%%%%%%%%%%%%%%

  \subsection{Functions}

    \label{xformslib:flpositioner:fl_add_positioner}
    \index{xformslib \textit{(package)}!xformslib.flpositioner \textit{(module)}!xformslib.flpositioner.fl\_add\_positioner \textit{(function)}}

    \vspace{0.5ex}

\hspace{.8\funcindent}\begin{boxedminipage}{\funcwidth}

    \raggedright \textbf{fl\_add\_positioner}(\textit{postype}, \textit{x}, \textit{y}, \textit{w}, \textit{h}, \textit{label})

    \vspace{-1.5ex}

    \rule{\textwidth}{0.5\fboxrule}
\setlength{\parskip}{2ex}

Adds a positioner object.

-{}-
\setlength{\parskip}{1ex}
      \textbf{Parameters}
      \vspace{-1ex}

      \begin{quote}
        \begin{Ventry}{xxxxxxx}

          \item[postype]


type of positioner to be added. Values (from xfdata.py)
FL\_NORMAL\_POSITIONER, FL\_OVERLAY\_POSITIONER, FL\_INVISIBLE\_POSITIONER
            {\it (type=int)}

          \item[x]


horizontal position (upper-left corner)
            {\it (type=int)}

          \item[y]


vertical position (upper-left corner)
            {\it (type=int)}

          \item[w]


width in coord units
            {\it (type=int)}

          \item[h]


height in coord units
            {\it (type=int)}

          \item[label]


text label of positioner. By default the label is placed below the box.
            {\it (type=str)}

        \end{Ventry}

      \end{quote}

      \textbf{Return Value}
    \vspace{-1ex}

      \begin{quote}

positioner object added (pFlObject).
      {\it (type=pointer to xfdata.FL\_OBJECT)}

      \end{quote}

\textbf{Note:} 
e.g. \emph{todo}


\textbf{Status:} 
Tested + NoDoc + Demo = OK


    \end{boxedminipage}

    \label{xformslib:flpositioner:fl_set_positioner_xvalue}
    \index{xformslib \textit{(package)}!xformslib.flpositioner \textit{(module)}!xformslib.flpositioner.fl\_set\_positioner\_xvalue \textit{(function)}}

    \vspace{0.5ex}

\hspace{.8\funcindent}\begin{boxedminipage}{\funcwidth}

    \raggedright \textbf{fl\_set\_positioner\_xvalue}(\textit{pFlObject}, \textit{val})

    \vspace{-1.5ex}

    \rule{\textwidth}{0.5\fboxrule}
\setlength{\parskip}{2ex}

Sets the actual value of positioner object in horizontal direction.

-{}-
\setlength{\parskip}{1ex}
      \textbf{Parameters}
      \vspace{-1ex}

      \begin{quote}
        \begin{Ventry}{xxxxxxxxx}

          \item[pFlObject]


positioner object
            {\it (type=pointer to xfdata.FL\_OBJECT)}

          \item[val]


value to be set. By default it is 0.5.
            {\it (type=float)}

        \end{Ventry}

      \end{quote}

\textbf{Note:} 
e.g. \emph{todo}


\textbf{Status:} 
Tested + NoDoc + Demo = OK


    \end{boxedminipage}

    \label{xformslib:flpositioner:fl_get_positioner_xvalue}
    \index{xformslib \textit{(package)}!xformslib.flpositioner \textit{(module)}!xformslib.flpositioner.fl\_get\_positioner\_xvalue \textit{(function)}}

    \vspace{0.5ex}

\hspace{.8\funcindent}\begin{boxedminipage}{\funcwidth}

    \raggedright \textbf{fl\_get\_positioner\_xvalue}(\textit{pFlObject})

    \vspace{-1.5ex}

    \rule{\textwidth}{0.5\fboxrule}
\setlength{\parskip}{2ex}

Obtains value of positioner object in horizontal direction.

-{}-
\setlength{\parskip}{1ex}
      \textbf{Parameters}
      \vspace{-1ex}

      \begin{quote}
        \begin{Ventry}{xxxxxxxxx}

          \item[pFlObject]


positioner object
            {\it (type=pointer to xfdata.FL\_OBJECT)}

        \end{Ventry}

      \end{quote}

      \textbf{Return Value}
    \vspace{-1ex}

      \begin{quote}

value in horizontal direction
      {\it (type=float)}

      \end{quote}

\textbf{Note:} 
e.g. \emph{todo}


\textbf{Status:} 
Tested + NoDoc + Demo = OK


    \end{boxedminipage}

    \label{xformslib:flpositioner:fl_set_positioner_xbounds}
    \index{xformslib \textit{(package)}!xformslib.flpositioner \textit{(module)}!xformslib.flpositioner.fl\_set\_positioner\_xbounds \textit{(function)}}

    \vspace{0.5ex}

\hspace{.8\funcindent}\begin{boxedminipage}{\funcwidth}

    \raggedright \textbf{fl\_set\_positioner\_xbounds}(\textit{pFlObject}, \textit{minbound}, \textit{maxbound})

    \vspace{-1.5ex}

    \rule{\textwidth}{0.5\fboxrule}
\setlength{\parskip}{2ex}

Sets minimum and maximum bounds/limits of a positioner in horizontal
direction.

-{}-
\setlength{\parskip}{1ex}
      \textbf{Parameters}
      \vspace{-1ex}

      \begin{quote}
        \begin{Ventry}{xxxxxxxxx}

          \item[pFlObject]


positioner object
            {\it (type=pointer to xfdata.FL\_OBJECT)}

          \item[minbound]


minimum bound to be set. By default the minimum value is 0.0.
            {\it (type=float)}

          \item[maxbound]


maximum bound to be set. By default the the maximum value is 1.0.
            {\it (type=float)}

        \end{Ventry}

      \end{quote}

\textbf{Note:} 
e.g. \emph{todo}


\textbf{Status:} 
Tested + NoDoc + Demo = OK


    \end{boxedminipage}

    \label{xformslib:flpositioner:fl_get_positioner_xbounds}
    \index{xformslib \textit{(package)}!xformslib.flpositioner \textit{(module)}!xformslib.flpositioner.fl\_get\_positioner\_xbounds \textit{(function)}}

    \vspace{0.5ex}

\hspace{.8\funcindent}\begin{boxedminipage}{\funcwidth}

    \raggedright \textbf{fl\_get\_positioner\_xbounds}(\textit{pFlObject})

    \vspace{-1.5ex}

    \rule{\textwidth}{0.5\fboxrule}
\setlength{\parskip}{2ex}

Obtain minumum and maximum bounds/limits of a positioner in horizontal
direction.

-{}-
\setlength{\parskip}{1ex}
      \textbf{Parameters}
      \vspace{-1ex}

      \begin{quote}
        \begin{Ventry}{xxxxxxxxx}

          \item[pFlObject]


positioner object
            {\it (type=pointer to xfdata.FL\_OBJECT)}

        \end{Ventry}

      \end{quote}

      \textbf{Return Value}
    \vspace{-1ex}

      \begin{quote}

minimum  bound, maximum bound
      {\it (type=float, float)}

      \end{quote}

\textbf{Note:} 
e.g. \emph{todo}


\textbf{Attention:} 
API change from XForms - upstream was
fl\_get\_positioner\_xbounds(pFlObject, minbound, maxbound)


\textbf{Status:} 
Untested + NoDoc + NoDemo = NOT OK


    \end{boxedminipage}

    \label{xformslib:flpositioner:fl_set_positioner_yvalue}
    \index{xformslib \textit{(package)}!xformslib.flpositioner \textit{(module)}!xformslib.flpositioner.fl\_set\_positioner\_yvalue \textit{(function)}}

    \vspace{0.5ex}

\hspace{.8\funcindent}\begin{boxedminipage}{\funcwidth}

    \raggedright \textbf{fl\_set\_positioner\_yvalue}(\textit{pFlObject}, \textit{val})

    \vspace{-1.5ex}

    \rule{\textwidth}{0.5\fboxrule}
\setlength{\parskip}{2ex}

Sets the actual value of positioner object in vertical direction.

-{}-
\setlength{\parskip}{1ex}
      \textbf{Parameters}
      \vspace{-1ex}

      \begin{quote}
        \begin{Ventry}{xxxxxxxxx}

          \item[pFlObject]


positioner object
            {\it (type=pointer to xfdata.FL\_OBJECT)}

          \item[val]


value to be set. By default it is 0.5.
            {\it (type=float)}

        \end{Ventry}

      \end{quote}

\textbf{Note:} 
e.g. \emph{todo}


\textbf{Status:} 
Tested + NoDoc + Demo = OK


    \end{boxedminipage}

    \label{xformslib:flpositioner:fl_get_positioner_yvalue}
    \index{xformslib \textit{(package)}!xformslib.flpositioner \textit{(module)}!xformslib.flpositioner.fl\_get\_positioner\_yvalue \textit{(function)}}

    \vspace{0.5ex}

\hspace{.8\funcindent}\begin{boxedminipage}{\funcwidth}

    \raggedright \textbf{fl\_get\_positioner\_yvalue}(\textit{pFlObject})

    \vspace{-1.5ex}

    \rule{\textwidth}{0.5\fboxrule}
\setlength{\parskip}{2ex}

Obtains value of positioner object in vertical direction.

-{}-
\setlength{\parskip}{1ex}
      \textbf{Parameters}
      \vspace{-1ex}

      \begin{quote}
        \begin{Ventry}{xxxxxxxxx}

          \item[pFlObject]


positioner object
            {\it (type=pointer to xfdata.FL\_OBJECT)}

        \end{Ventry}

      \end{quote}

      \textbf{Return Value}
    \vspace{-1ex}

      \begin{quote}

value in vertical direction
      {\it (type=float)}

      \end{quote}

\textbf{Note:} 
e.g. \emph{todo}


\textbf{Status:} 
Untested + NoDoc + NoDemo = NOT OK


    \end{boxedminipage}

    \label{xformslib:flpositioner:fl_set_positioner_ybounds}
    \index{xformslib \textit{(package)}!xformslib.flpositioner \textit{(module)}!xformslib.flpositioner.fl\_set\_positioner\_ybounds \textit{(function)}}

    \vspace{0.5ex}

\hspace{.8\funcindent}\begin{boxedminipage}{\funcwidth}

    \raggedright \textbf{fl\_set\_positioner\_ybounds}(\textit{pFlObject}, \textit{minbound}, \textit{maxbound})

    \vspace{-1.5ex}

    \rule{\textwidth}{0.5\fboxrule}
\setlength{\parskip}{2ex}

Sets minimum and maximum bounds/limits of a positioner in vertical
direction.

-{}-
\setlength{\parskip}{1ex}
      \textbf{Parameters}
      \vspace{-1ex}

      \begin{quote}
        \begin{Ventry}{xxxxxxxxx}

          \item[pFlObject]


positioner object
            {\it (type=pointer to xfdata.FL\_OBJECT)}

          \item[minbound]


minimum bound to be set. By default the minimum value is 0.0.
            {\it (type=float)}

          \item[maxbound]


maximum bound to be set. By default the the maximum value is 1.0.
            {\it (type=float)}

        \end{Ventry}

      \end{quote}

\textbf{Note:} 
e.g. \emph{todo}


\textbf{Status:} 
Tested + NoDoc + Demo = OK


    \end{boxedminipage}

    \label{xformslib:flpositioner:fl_get_positioner_ybounds}
    \index{xformslib \textit{(package)}!xformslib.flpositioner \textit{(module)}!xformslib.flpositioner.fl\_get\_positioner\_ybounds \textit{(function)}}

    \vspace{0.5ex}

\hspace{.8\funcindent}\begin{boxedminipage}{\funcwidth}

    \raggedright \textbf{fl\_get\_positioner\_ybounds}(\textit{pFlObject})

    \vspace{-1.5ex}

    \rule{\textwidth}{0.5\fboxrule}
\setlength{\parskip}{2ex}

Obtain minimum and maximum bounds/limits of a positioner in vertical
direction.

-{}-
\setlength{\parskip}{1ex}
      \textbf{Parameters}
      \vspace{-1ex}

      \begin{quote}
        \begin{Ventry}{xxxxxxxxx}

          \item[pFlObject]


positioner object
            {\it (type=pointer to xfdata.FL\_OBJECT)}

        \end{Ventry}

      \end{quote}

      \textbf{Return Value}
    \vspace{-1ex}

      \begin{quote}

minimum bound, maximum bound
      {\it (type=float, float)}

      \end{quote}

\textbf{Note:} 
e.g. \emph{todo}


\textbf{Attention:} 
API change from XForms - upstream was
fl\_get\_positioner\_ybounds(pFlObject, minbound, maxbound)


\textbf{Status:} 
Untested + NoDoc + NoDemo = NOT OK


    \end{boxedminipage}

    \label{xformslib:flpositioner:fl_set_positioner_xstep}
    \index{xformslib \textit{(package)}!xformslib.flpositioner \textit{(module)}!xformslib.flpositioner.fl\_set\_positioner\_xstep \textit{(function)}}

    \vspace{0.5ex}

\hspace{.8\funcindent}\begin{boxedminipage}{\funcwidth}

    \raggedright \textbf{fl\_set\_positioner\_xstep}(\textit{pFlObject}, \textit{step})

    \vspace{-1.5ex}

    \rule{\textwidth}{0.5\fboxrule}
\setlength{\parskip}{2ex}

Handles positioner values in horizontal direction to be rounded to some
values (multiples of step), e.g. to integer values.

-{}-
\setlength{\parskip}{1ex}
      \textbf{Parameters}
      \vspace{-1ex}

      \begin{quote}
        \begin{Ventry}{xxxxxxxxx}

          \item[pFlObject]


positioner object
            {\it (type=pointer to xfdata.FL\_OBJECT)}

          \item[step]


rounded value. If it's 0.0, switch off rounding.
            {\it (type=float)}

        \end{Ventry}

      \end{quote}

\textbf{Note:} 
e.g. \emph{todo}


\textbf{Status:} 
Untested + NoDoc + NoDemo = NOT OK


    \end{boxedminipage}

    \label{xformslib:flpositioner:fl_set_positioner_ystep}
    \index{xformslib \textit{(package)}!xformslib.flpositioner \textit{(module)}!xformslib.flpositioner.fl\_set\_positioner\_ystep \textit{(function)}}

    \vspace{0.5ex}

\hspace{.8\funcindent}\begin{boxedminipage}{\funcwidth}

    \raggedright \textbf{fl\_set\_positioner\_ystep}(\textit{pFlObject}, \textit{step})

    \vspace{-1.5ex}

    \rule{\textwidth}{0.5\fboxrule}
\setlength{\parskip}{2ex}

Handles positioner values in vertical direction to be rounded to some
values (multiples of step), e.g. to integer values.

-{}-
\setlength{\parskip}{1ex}
      \textbf{Parameters}
      \vspace{-1ex}

      \begin{quote}
        \begin{Ventry}{xxxxxxxxx}

          \item[pFlObject]


positioner object
            {\it (type=pointer to xfdata.FL\_OBJECT)}

          \item[step]


rounded value. If it's 0.0, switch off rounding.
            {\it (type=float)}

        \end{Ventry}

      \end{quote}

\textbf{Note:} 
e.g. \emph{todo}


\textbf{Status:} 
Untested + NoDoc + NoDemo = NOT OK


    \end{boxedminipage}

    \index{xformslib \textit{(package)}!xformslib.flpositioner \textit{(module)}|)}
