%
% API Documentation for API Documentation
% Module xformslib.flflimage
%
% Generated by epydoc 3.0.1
% [Fri May 14 18:28:31 2010]
%

%%%%%%%%%%%%%%%%%%%%%%%%%%%%%%%%%%%%%%%%%%%%%%%%%%%%%%%%%%%%%%%%%%%%%%%%%%%
%%                          Module Description                           %%
%%%%%%%%%%%%%%%%%%%%%%%%%%%%%%%%%%%%%%%%%%%%%%%%%%%%%%%%%%%%%%%%%%%%%%%%%%%

    \index{xformslib \textit{(package)}!xformslib.flflimage \textit{(module)}|(}
\section{Module xformslib.flflimage}

    \label{xformslib:flflimage}

flflimage.py - xforms-python's functions to manage image objects.

Copyright (C) 2009, 2010  Luca Lazzaroni ``LukenShiro''
e-mail: <\href{mailto:lukenshiro@ngi.it}{lukenshiro@ngi.it}>

This program is free software: you can redistribute it and/or modify
it under the terms of the GNU Lesser General Public License as
published by the Free Software Foundation, version 2.1 of the License.

This program is distributed in the hope that it will be useful,
but WITHOUT ANY WARRANTY; without even the implied warranty of
MERCHANTABILITY or FITNESS FOR A PARTICULAR PURPOSE. See the
GNU Lesser General Public License for more details.

You should have received a copy of the GNU LGPL along with this
program. If not, see <\href{http://www.gnu.org/licenses/}{http://www.gnu.org/licenses/}>.

See CREDITS file to read acknowledgements and thanks to XForms,
ctypes and other developers.

%%%%%%%%%%%%%%%%%%%%%%%%%%%%%%%%%%%%%%%%%%%%%%%%%%%%%%%%%%%%%%%%%%%%%%%%%%%
%%                               Functions                               %%
%%%%%%%%%%%%%%%%%%%%%%%%%%%%%%%%%%%%%%%%%%%%%%%%%%%%%%%%%%%%%%%%%%%%%%%%%%%

  \subsection{Functions}

    \label{xformslib:flflimage:FL_RGB2GRAY}
    \index{xformslib \textit{(package)}!xformslib.flflimage \textit{(module)}!xformslib.flflimage.FL\_RGB2GRAY \textit{(function)}}

    \vspace{0.5ex}

\hspace{.8\funcindent}\begin{boxedminipage}{\funcwidth}

    \raggedright \textbf{FL\_RGB2GRAY}(\textit{r}, \textit{g}, \textit{b})

\setlength{\parskip}{2ex}
\setlength{\parskip}{1ex}
    \end{boxedminipage}

    \label{xformslib:flflimage:FL_IsRGB}
    \index{xformslib \textit{(package)}!xformslib.flflimage \textit{(module)}!xformslib.flflimage.FL\_IsRGB \textit{(function)}}

    \vspace{0.5ex}

\hspace{.8\funcindent}\begin{boxedminipage}{\funcwidth}

    \raggedright \textbf{FL\_IsRGB}(\textit{pImage})

\setlength{\parskip}{2ex}
\setlength{\parskip}{1ex}
    \end{boxedminipage}

    \label{xformslib:flflimage:FL_IsPacked}
    \index{xformslib \textit{(package)}!xformslib.flflimage \textit{(module)}!xformslib.flflimage.FL\_IsPacked \textit{(function)}}

    \vspace{0.5ex}

\hspace{.8\funcindent}\begin{boxedminipage}{\funcwidth}

    \raggedright \textbf{FL\_IsPacked}(\textit{pImage})

\setlength{\parskip}{2ex}
\setlength{\parskip}{1ex}
    \end{boxedminipage}

    \label{xformslib:flflimage:flimage_setup}
    \index{xformslib \textit{(package)}!xformslib.flflimage \textit{(module)}!xformslib.flflimage.flimage\_setup \textit{(function)}}

    \vspace{0.5ex}

\hspace{.8\funcindent}\begin{boxedminipage}{\funcwidth}

    \raggedright \textbf{flimage\_setup}(\textit{pImageSetup})

    \vspace{-1.5ex}

    \rule{\textwidth}{0.5\fboxrule}
\setlength{\parskip}{2ex}

Sets up and configures image objects support and initializes
xfdata.FLIMAGE\_SETUP class instance.

-{}-
\setlength{\parskip}{1ex}
      \textbf{Parameters}
      \vspace{-1ex}

      \begin{quote}
        \begin{Ventry}{xxxxxxxxxxx}

          \item[pImageSetup]


imagesetup class instance
            {\it (type=pointer to xfdata.FLIMAGE\_SETUP)}

        \end{Ventry}

      \end{quote}

\textbf{Status:} 
Untested + Doc + NoDemo = NOT OK


    \end{boxedminipage}

    \label{xformslib:flflimage:flimage_load}
    \index{xformslib \textit{(package)}!xformslib.flflimage \textit{(module)}!xformslib.flflimage.flimage\_load \textit{(function)}}

    \vspace{0.5ex}

\hspace{.8\funcindent}\begin{boxedminipage}{\funcwidth}

    \raggedright \textbf{flimage\_load}(\textit{fname})

    \vspace{-1.5ex}

    \rule{\textwidth}{0.5\fboxrule}
\setlength{\parskip}{2ex}

Reads an image file.

-{}-
\setlength{\parskip}{1ex}
      \textbf{Parameters}
      \vspace{-1ex}

      \begin{quote}
        \begin{Ventry}{xxxxx}

          \item[fname]


name of image file to load
            {\it (type=str)}

        \end{Ventry}

      \end{quote}

      \textbf{Return Value}
    \vspace{-1ex}

      \begin{quote}

an image class instance (pImage), or None (on failure)
      {\it (type=pointer to xfdata.FL\_IMAGE)}

      \end{quote}

\textbf{Note:} 
e.g. \emph{todo}


\textbf{Status:} 
Untested + Doc + NoDemo = NOT OK


    \end{boxedminipage}

    \label{xformslib:flflimage:flimage_read}
    \index{xformslib \textit{(package)}!xformslib.flflimage \textit{(module)}!xformslib.flflimage.flimage\_read \textit{(function)}}

    \vspace{0.5ex}

\hspace{.8\funcindent}\begin{boxedminipage}{\funcwidth}

    \raggedright \textbf{flimage\_read}(\textit{pImage})

    \vspace{-1.5ex}

    \rule{\textwidth}{0.5\fboxrule}
\setlength{\parskip}{2ex}

Takes a xfdata.FL\_IMAGE class instance returned by flimage\_open() and
fills the image structure.

-{}-
\setlength{\parskip}{1ex}
      \textbf{Parameters}
      \vspace{-1ex}

      \begin{quote}
        \begin{Ventry}{xxxxxx}

          \item[pImage]


image class instance
            {\it (type=pointer to xfdata.FL\_IMAGE)}

        \end{Ventry}

      \end{quote}

      \textbf{Return Value}
    \vspace{-1ex}

      \begin{quote}

an image class instance (pImage), or None (on failure)
      {\it (type=pointer to xfdata.FL\_IMAGE)}

      \end{quote}

\textbf{Note:} 
e.g. \emph{todo}


\textbf{Status:} 
Untested + Doc + NoDemo = NOT OK


    \end{boxedminipage}

    \label{xformslib:flflimage:flimage_dump}
    \index{xformslib \textit{(package)}!xformslib.flflimage \textit{(module)}!xformslib.flflimage.flimage\_dump \textit{(function)}}

    \vspace{0.5ex}

\hspace{.8\funcindent}\begin{boxedminipage}{\funcwidth}

    \raggedright \textbf{flimage\_dump}(\textit{pImage}, \textit{fname}, \textit{fmt})

    \vspace{-1.5ex}

    \rule{\textwidth}{0.5\fboxrule}
\setlength{\parskip}{2ex}

Takes an image, either returned by flimage\_load() (possibly after some
processing) or created on the fly by the application, attempts to create
a file to store the image.

-{}-
\setlength{\parskip}{1ex}
      \textbf{Parameters}
      \vspace{-1ex}

      \begin{quote}
        \begin{Ventry}{xxxxxx}

          \item[pImage]


image class instance
            {\it (type=pointer to xfdata.FL\_IMAGE)}

          \item[fname]


name of file to be saved
            {\it (type=str)}

          \item[fmt]


formal name or short name of a supported image format. Values: jpeg,
ppm, gif, bmp, etc... or some other formats the application knows
how to write. If it is 'None', the original format the image was in
is used.
            {\it (type=str)}

        \end{Ventry}

      \end{quote}

      \textbf{Return Value}
    \vspace{-1ex}

      \begin{quote}

non-negative, or negative num. (on failure)
      {\it (type=int)}

      \end{quote}

\textbf{Note:} 
e.g. \emph{todo}


\textbf{Status:} 
Untested + Doc + NoDemo = NOT OK


    \end{boxedminipage}

    \label{xformslib:flflimage:flimage_close}
    \index{xformslib \textit{(package)}!xformslib.flflimage \textit{(module)}!xformslib.flflimage.flimage\_close \textit{(function)}}

    \vspace{0.5ex}

\hspace{.8\funcindent}\begin{boxedminipage}{\funcwidth}

    \raggedright \textbf{flimage\_close}(\textit{pImage})

    \vspace{-1.5ex}

    \rule{\textwidth}{0.5\fboxrule}
\setlength{\parskip}{2ex}

Closes all file streams used to create the image.

-{}-
\setlength{\parskip}{1ex}
      \textbf{Parameters}
      \vspace{-1ex}

      \begin{quote}
        \begin{Ventry}{xxxxxx}

          \item[pImage]


image to be closed
            {\it (type=pointer to xfdata.FL\_IMAGE)}

        \end{Ventry}

      \end{quote}

      \textbf{Return Value}
    \vspace{-1ex}

      \begin{quote}

0?, or -1 (on failure)
      {\it (type=int)}

      \end{quote}

\textbf{Note:} 
e.g. flimage\_close(pimg)


\textbf{Status:} 
Untested + Doc + NoDemo = NOT OK


    \end{boxedminipage}

    \label{xformslib:flflimage:flimage_alloc}
    \index{xformslib \textit{(package)}!xformslib.flflimage \textit{(module)}!xformslib.flflimage.flimage\_alloc \textit{(function)}}

    \vspace{0.5ex}

\hspace{.8\funcindent}\begin{boxedminipage}{\funcwidth}

    \raggedright \textbf{flimage\_alloc}()

    \vspace{-1.5ex}

    \rule{\textwidth}{0.5\fboxrule}
\setlength{\parskip}{2ex}

Creates an image structure whose dynamically allocated memory is
properly initialized, and returning it.

-{}-
\setlength{\parskip}{1ex}
      \textbf{Return Value}
    \vspace{-1ex}

      \begin{quote}

image class instance (pImage)
      {\it (type=pointer to xfdata.FL\_IMAGE)}

      \end{quote}

\textbf{Note:} 
e.g. \emph{todo}


\textbf{Status:} 
Untested + Doc + NoDemo = NOT OK


    \end{boxedminipage}

    \label{xformslib:flflimage:flimage_getmem}
    \index{xformslib \textit{(package)}!xformslib.flflimage \textit{(module)}!xformslib.flflimage.flimage\_getmem \textit{(function)}}

    \vspace{0.5ex}

\hspace{.8\funcindent}\begin{boxedminipage}{\funcwidth}

    \raggedright \textbf{flimage\_getmem}(\textit{pImage})

    \vspace{-1.5ex}

    \rule{\textwidth}{0.5\fboxrule}
\setlength{\parskip}{2ex}

Allocates the proper amount of memory appropriate for the image type,
including colormaps when needed.

-{}-
\setlength{\parskip}{1ex}
      \textbf{Parameters}
      \vspace{-1ex}

      \begin{quote}
        \begin{Ventry}{xxxxxx}

          \item[pImage]


image
            {\it (type=pointer to xfdata.FL\_IMAGE)}

        \end{Ventry}

      \end{quote}

      \textbf{Return Value}
    \vspace{-1ex}

      \begin{quote}

num.
      {\it (type=int)}

      \end{quote}

\textbf{Note:} 
e.g. \emph{todo}


\textbf{Status:} 
Untested + Doc + NoDemo = NOT OK


    \end{boxedminipage}

    \label{xformslib:flflimage:flimage_is_supported}
    \index{xformslib \textit{(package)}!xformslib.flflimage \textit{(module)}!xformslib.flflimage.flimage\_is\_supported \textit{(function)}}

    \vspace{0.5ex}

\hspace{.8\funcindent}\begin{boxedminipage}{\funcwidth}

    \raggedright \textbf{flimage\_is\_supported}(\textit{fname})

    \vspace{-1.5ex}

    \rule{\textwidth}{0.5\fboxrule}
\setlength{\parskip}{2ex}

Finds out if a specific file is a known image file or not.

-{}-
\setlength{\parskip}{1ex}
      \textbf{Parameters}
      \vspace{-1ex}

      \begin{quote}
        \begin{Ventry}{xxxxx}

          \item[fname]


name of file to be evaluated
            {\it (type=str)}

        \end{Ventry}

      \end{quote}

      \textbf{Return Value}
    \vspace{-1ex}

      \begin{quote}

1 (if it's a known image file), or 0 (on failure)
      {\it (type=int)}

      \end{quote}

\textbf{Note:} 
e.g. \emph{todo}


\textbf{Status:} 
Untested + Doc + NoDemo = NOT OK


    \end{boxedminipage}

    \label{xformslib:flflimage:flimage_description_via_filter}
    \index{xformslib \textit{(package)}!xformslib.flflimage \textit{(module)}!xformslib.flflimage.flimage\_description\_via\_filter \textit{(function)}}

    \vspace{0.5ex}

\hspace{.8\funcindent}\begin{boxedminipage}{\funcwidth}

    \raggedright \textbf{flimage\_description\_via\_filter}(\textit{pImage}, \textit{cmds}, \textit{what}, \textit{verbose})

    \vspace{-1.5ex}

    \rule{\textwidth}{0.5\fboxrule}
\setlength{\parskip}{2ex}

Adds a description to be used with flimage\_add\_format() to add image
formats via an external filter's command.

-{}-
\setlength{\parskip}{1ex}
      \textbf{Parameters}
      \vspace{-1ex}

      \begin{quote}
        \begin{Ventry}{xxxxxxx}

          \item[pImage]


image
            {\it (type=pointer to xfdata.FL\_IMAGE)}

          \item[cmds]


a list of shell commands (filters) that convert the format in
question into one of the supported formats.
            {\it (type=str?)}

          \item[what]


text for reporting purpose
            {\it (type=str)}

          \item[verbose]


controls if some information and error messages should be printed
(mainly for debugging purpose). Values 0 (to disable) or 1 (to enable)
            {\it (type=int)}

        \end{Ventry}

      \end{quote}

      \textbf{Return Value}
    \vspace{-1ex}

      \begin{quote}

num.
      {\it (type=int)}

      \end{quote}

\textbf{Note:} 
e.g. \emph{todo}


\textbf{Status:} 
Untested + Doc + NoDemo = NOT OK


    \end{boxedminipage}

    \label{xformslib:flflimage:flimage_write_via_filter}
    \index{xformslib \textit{(package)}!xformslib.flflimage \textit{(module)}!xformslib.flflimage.flimage\_write\_via\_filter \textit{(function)}}

    \vspace{0.5ex}

\hspace{.8\funcindent}\begin{boxedminipage}{\funcwidth}

    \raggedright \textbf{flimage\_write\_via\_filter}(\textit{pImage}, \textit{cmds}, \textit{formats}, \textit{verbose})

    \vspace{-1.5ex}

    \rule{\textwidth}{0.5\fboxrule}
\setlength{\parskip}{2ex}

Uses external filters to add image formats, in order to convert
an unsupported format into one that is. pbmplus or netpbm are excellent
packages for this purpose.

-{}-
\setlength{\parskip}{1ex}
      \textbf{Parameters}
      \vspace{-1ex}

      \begin{quote}
        \begin{Ventry}{xxxxxxx}

          \item[pImage]


image
            {\it (type=pointer to xfdata.FL\_IMAGE)}

          \item[cmds]


a list of shell commands (filters) that convert the format in
question into one of the supported formats.
            {\it (type=str?)}

          \item[formats]


list of strings. Values ppm, pgm, pbm, .. etc..
            {\it (type=str?)}

          \item[verbose]


controls if some information and error messages should be printed
(mainly for debugging purpose). Values 0 (to disable) or 1 (to enable)
            {\it (type=int)}

        \end{Ventry}

      \end{quote}

      \textbf{Return Value}
    \vspace{-1ex}

      \begin{quote}

num.
      {\it (type=int)}

      \end{quote}

\textbf{Note:} 
e.g. \emph{todo}


\textbf{Status:} 
Untested + Doc + NoDemo = NOT OK


    \end{boxedminipage}

    \label{xformslib:flflimage:flimage_free}
    \index{xformslib \textit{(package)}!xformslib.flflimage \textit{(module)}!xformslib.flflimage.flimage\_free \textit{(function)}}

    \vspace{0.5ex}

\hspace{.8\funcindent}\begin{boxedminipage}{\funcwidth}

    \raggedright \textbf{flimage\_free}(\textit{pImage})

    \vspace{-1.5ex}

    \rule{\textwidth}{0.5\fboxrule}
\setlength{\parskip}{2ex}

Frees all memory allocated for the image, then the image structure
itself. After the function returns, the image should not be referenced.

-{}-
\setlength{\parskip}{1ex}
      \textbf{Parameters}
      \vspace{-1ex}

      \begin{quote}
        \begin{Ventry}{xxxxxx}

          \item[pImage]


image
            {\it (type=pointer to xfdata.FL\_IMAGE)}

        \end{Ventry}

      \end{quote}

\textbf{Note:} 
e.g. \emph{todo}


\textbf{Status:} 
Untested + Doc + NoDemo = NOT OK


    \end{boxedminipage}

    \label{xformslib:flflimage:flimage_display}
    \index{xformslib \textit{(package)}!xformslib.flflimage \textit{(module)}!xformslib.flflimage.flimage\_display \textit{(function)}}

    \vspace{0.5ex}

\hspace{.8\funcindent}\begin{boxedminipage}{\funcwidth}

    \raggedright \textbf{flimage\_display}(\textit{pImage}, \textit{win})

    \vspace{-1.5ex}

    \rule{\textwidth}{0.5\fboxrule}
\setlength{\parskip}{2ex}

Displays a single or multiple images in a window.

-{}-
\setlength{\parskip}{1ex}
      \textbf{Parameters}
      \vspace{-1ex}

      \begin{quote}
        \begin{Ventry}{xxxxxx}

          \item[pImage]


image
            {\it (type=pointer to xfdata.FL\_IMAGE)}

          \item[win]


window
            {\it (type=long\_pos)}

        \end{Ventry}

      \end{quote}

      \textbf{Return Value}
    \vspace{-1ex}

      \begin{quote}

non-negative num., or negative num. (on failure)
      {\it (type=int)}

      \end{quote}

\textbf{Note:} 
e.g. \emph{todo}


\textbf{Status:} 
Untested + Doc + NoDemo = NOT OK


    \end{boxedminipage}

    \label{xformslib:flflimage:flimage_sdisplay}
    \index{xformslib \textit{(package)}!xformslib.flflimage \textit{(module)}!xformslib.flflimage.flimage\_sdisplay \textit{(function)}}

    \vspace{0.5ex}

\hspace{.8\funcindent}\begin{boxedminipage}{\funcwidth}

    \raggedright \textbf{flimage\_sdisplay}(\textit{pImage}, \textit{win})

    \vspace{-1.5ex}

    \rule{\textwidth}{0.5\fboxrule}
\setlength{\parskip}{2ex}

Displays a single image in a window.

-{}-
\setlength{\parskip}{1ex}
      \textbf{Parameters}
      \vspace{-1ex}

      \begin{quote}
        \begin{Ventry}{xxxxxx}

          \item[pImage]


image
            {\it (type=pointer to xfdata.FL\_IMAGE)}

          \item[win]


window
            {\it (type=long\_pos)}

        \end{Ventry}

      \end{quote}

      \textbf{Return Value}
    \vspace{-1ex}

      \begin{quote}

non-negative num., or negative num. (on failure)
      {\it (type=int)}

      \end{quote}

\textbf{Note:} 
e.g. \emph{todo}


\textbf{Status:} 
Untested + Doc + NoDemo = NOT OK


    \end{boxedminipage}

    \label{xformslib:flflimage:flimage_convert}
    \index{xformslib \textit{(package)}!xformslib.flflimage \textit{(module)}!xformslib.flflimage.flimage\_convert \textit{(function)}}

    \vspace{0.5ex}

\hspace{.8\funcindent}\begin{boxedminipage}{\funcwidth}

    \raggedright \textbf{flimage\_convert}(\textit{pImage}, \textit{newtype}, \textit{ncolors})

    \vspace{-1.5ex}

    \rule{\textwidth}{0.5\fboxrule}
\setlength{\parskip}{2ex}

Convert an image to a new type. Depending on which quantization
function is used, the number of quantized colors may not be more than 256.

-{}-
\setlength{\parskip}{1ex}
      \textbf{Parameters}
      \vspace{-1ex}

      \begin{quote}
        \begin{Ventry}{xxxxxxx}

          \item[pImage]


image
            {\it (type=pointer to xfdata.FL\_IMAGE)}

          \item[newtype]


one of supported image type to convert to. Values (from xfdata.py)
FL\_IMAGE\_NONE, FL\_IMAGE\_MONO, FL\_IMAGE\_GRAY, FL\_IMAGE\_CI, FL\_IMAGE\_RGB,
FL\_IMAGE\_PACKED, FL\_IMAGE\_GRAY16, FL\_IMAGE\_RGB16, FL\_IMAGE\_FLEX
            {\it (type=int)}

          \item[ncolors]


number of colors to generate. It makes sense only when newtype is
xfdata.FL\_IMAGE\_CI.
            {\it (type=int)}

        \end{Ventry}

      \end{quote}

      \textbf{Return Value}
    \vspace{-1ex}

      \begin{quote}

non-negative num., or negative num. (on failure)
      {\it (type=int)}

      \end{quote}

\textbf{Note:} 
e.g. \emph{todo}


\textbf{Status:} 
Untested + Doc + NoDemo = NOT OK


    \end{boxedminipage}

    \label{xformslib:flflimage:flimage_type_name}
    \index{xformslib \textit{(package)}!xformslib.flflimage \textit{(module)}!xformslib.flflimage.flimage\_type\_name \textit{(function)}}

    \vspace{0.5ex}

\hspace{.8\funcindent}\begin{boxedminipage}{\funcwidth}

    \raggedright \textbf{flimage\_type\_name}(\textit{imagetype})

    \vspace{-1.5ex}

    \rule{\textwidth}{0.5\fboxrule}
\setlength{\parskip}{2ex}

Obtains the image type name in string format, e.g., for reporting
purposes.

-{}-
\setlength{\parskip}{1ex}
      \textbf{Parameters}
      \vspace{-1ex}

      \begin{quote}
        \begin{Ventry}{xxxxxxxxx}

          \item[imagetype]


type of image
            {\it (type=int)}

        \end{Ventry}

      \end{quote}

      \textbf{Return Value}
    \vspace{-1ex}

      \begin{quote}

name string
      {\it (type=str)}

      \end{quote}

\textbf{Note:} 
e.g. \emph{todo}


\textbf{Status:} 
Untested + Doc + NoDemo = NOT OK


    \end{boxedminipage}

    \label{xformslib:flflimage:flimage_add_text}
    \index{xformslib \textit{(package)}!xformslib.flflimage \textit{(module)}!xformslib.flflimage.flimage\_add\_text \textit{(function)}}

    \vspace{0.5ex}

\hspace{.8\funcindent}\begin{boxedminipage}{\funcwidth}

    \raggedright \textbf{flimage\_add\_text}(\textit{pImage}, \textit{text}, \textit{length}, \textit{style}, \textit{size}, \textit{txtcolr}, \textit{bgcolr}, \textit{nobk}, \textit{tx}, \textit{ty}, \textit{rot})

    \vspace{-1.5ex}

    \rule{\textwidth}{0.5\fboxrule}
\setlength{\parskip}{2ex}

Place text into the image, passing parameters individually. If text
starts with character '@' a symbol is drawn.

-{}-
\setlength{\parskip}{1ex}
      \textbf{Parameters}
      \vspace{-1ex}

      \begin{quote}
        \begin{Ventry}{xxxxxxx}

          \item[pImage]


image
            {\it (type=pointer to xfdata.FL\_IMAGE)}

          \item[text]


text string to be placed in image
            {\it (type=str)}

          \item[length]


length of text
            {\it (type=int)}

          \item[style]


label style. Values (from xfdata.py) FL\_NORMAL\_STYLE, FL\_BOLD\_STYLE,
FL\_ITALIC\_STYLE, FL\_BOLDITALIC\_STYLE, FL\_FIXED\_STYLE,
FL\_FIXEDBOLD\_STYLE, FL\_FIXEDITALIC\_STYLE, FL\_FIXEDBOLDITALIC\_STYLE,
FL\_TIMES\_STYLE, FL\_TIMESBOLD\_STYLE, FL\_TIMESITALIC\_STYLE,
FL\_TIMESBOLDITALIC\_STYLE, FL\_MISC\_STYLE, FL\_MISCBOLD\_STYLE,
FL\_MISCITALIC\_STYLE, FL\_SYMBOL\_STYLE, FL\_SHADOW\_STYLE,
FL\_ENGRAVED\_STYLE, FL\_EMBOSSED\_STYLE
            {\it (type=int)}

          \item[size]


label size. Values (from xfdata.py) FL\_TINY\_SIZE, FL\_SMALL\_SIZE,
FL\_NORMAL\_SIZE, FL\_MEDIUM\_SIZE, FL\_LARGE\_SIZE, FL\_HUGE\_SIZE,
FL\_DEFAULT\_SIZE
            {\it (type=int)}

          \item[txtcolr]


color to use for text
            {\it (type=int\_pos)}

          \item[bgcolr]


color to use for background (only if nobk is 0)
            {\it (type=int\_pos)}

          \item[nobk]


flag to enable/disable background. Values 0 (drawn with background)
or 1 (text is drawn without a background)
            {\it (type=int)}

          \item[tx]


horizontal location of the text relative to the image origin. The
location specified is the lower-right corner of the text.
            {\it (type=float)}

          \item[ty]


vertical location of the text relative to the image origin. The
location specified is the lower-right corner of the text.
            {\it (type=float)}

          \item[rot]


rotation
            {\it (type=int)}

        \end{Ventry}

      \end{quote}

      \textbf{Return Value}
    \vspace{-1ex}

      \begin{quote}

current number of strings for the image
      {\it (type=int)}

      \end{quote}

\textbf{Note:} 
e.g. \emph{todo}


\textbf{Status:} 
Untested + Doc + NoDemo = NOT OK


    \end{boxedminipage}

    \label{xformslib:flflimage:flimage_add_text_struct}
    \index{xformslib \textit{(package)}!xformslib.flflimage \textit{(module)}!xformslib.flflimage.flimage\_add\_text\_struct \textit{(function)}}

    \vspace{0.5ex}

\hspace{.8\funcindent}\begin{boxedminipage}{\funcwidth}

    \raggedright \textbf{flimage\_add\_text\_struct}(\textit{pImage}, \textit{pImageText})

    \vspace{-1.5ex}

    \rule{\textwidth}{0.5\fboxrule}
\setlength{\parskip}{2ex}

Places text into the image, using xfdata.FLIMAGE\_TEXT class instance.
If text starts with character '@' a symbol is drawn.
-{}-
\setlength{\parskip}{1ex}
      \textbf{Parameters}
      \vspace{-1ex}

      \begin{quote}
        \begin{Ventry}{xxxxxxxxxx}

          \item[pImage]


image
            {\it (type=pointer to xfdata.FL\_IMAGE)}

          \item[pImageText]


flimagetext class instance
            {\it (type=pointer to xfdata.FLIMAGE\_TEXT)}

        \end{Ventry}

      \end{quote}

      \textbf{Return Value}
    \vspace{-1ex}

      \begin{quote}

current number of strings for the image
      {\it (type=int)}

      \end{quote}

\textbf{Note:} 
e.g. \emph{todo}


\textbf{Status:} 
Untested + Doc + NoDemo = NOT OK


    \end{boxedminipage}

    \label{xformslib:flflimage:flimage_delete_all_text}
    \index{xformslib \textit{(package)}!xformslib.flflimage \textit{(module)}!xformslib.flflimage.flimage\_delete\_all\_text \textit{(function)}}

    \vspace{0.5ex}

\hspace{.8\funcindent}\begin{boxedminipage}{\funcwidth}

    \raggedright \textbf{flimage\_delete\_all\_text}(\textit{pImage})

    \vspace{-1.5ex}

    \rule{\textwidth}{0.5\fboxrule}
\setlength{\parskip}{2ex}

Deletes all the texts you added to an image.

-{}-
\setlength{\parskip}{1ex}
      \textbf{Parameters}
      \vspace{-1ex}

      \begin{quote}
        \begin{Ventry}{xxxxxx}

          \item[pImage]


image
            {\it (type=pointer to xfdata.FL\_IMAGE)}

        \end{Ventry}

      \end{quote}

\textbf{Note:} 
e.g. \emph{todo}


\textbf{Status:} 
Untested + Doc + NoDemo = NOT OK


    \end{boxedminipage}

    \label{xformslib:flflimage:flimage_add_marker}
    \index{xformslib \textit{(package)}!xformslib.flflimage \textit{(module)}!xformslib.flflimage.flimage\_add\_marker \textit{(function)}}

    \vspace{0.5ex}

\hspace{.8\funcindent}\begin{boxedminipage}{\funcwidth}

    \raggedright \textbf{flimage\_add\_marker}(\textit{pImage}, \textit{name}, \textit{x}, \textit{y}, \textit{w}, \textit{h}, \textit{style}, \textit{fill}, \textit{rot}, \textit{colr}, \textit{bcolr})

    \vspace{-1.5ex}

    \rule{\textwidth}{0.5\fboxrule}
\setlength{\parskip}{2ex}

Adds simple markers (arrows, circles etc) to an image, passing
parameters individually.

-{}-
\setlength{\parskip}{1ex}
      \textbf{Parameters}
      \vspace{-1ex}

      \begin{quote}
        \begin{Ventry}{xxxxxx}

          \item[pImage]


image
            {\it (type=pointer to xfdata.FL\_IMAGE)}

          \item[name]


marker name \emph{todo}
            {\it (type=str)}

          \item[x]


horizontal position of the center of the marker in physical
coordinates relative to the origin of the image
            {\it (type=float)}

          \item[y]


vertical position of the center of the marker in physical
coordinates relative to the origin of the image
            {\it (type=float)}

          \item[w]


width of the bounding box of the marker in physical coordinates
            {\it (type=float)}

          \item[h]


height of the bounding box of the marker in physical coordinates
            {\it (type=float)}

          \item[style]


style of the line to draw. Values (from xfdata.py) FL\_SOLID,
FL\_USERDASH, FL\_USERDOUBLEDASH, FL\_DOT, FL\_DOTDASH, FL\_DASH,
FL\_LONGDASH
            {\it (type=int)}

          \item[fill]


flag if the marker should be filled or not. Values 1 (filled) or 0
(not filled)
            {\it (type=int)}

          \item[rot]


angle of rotation in tenth of degree
            {\it (type=int)}

          \item[colr]


color of the marker (in packed RGB format)
            {\it (type=long\_pos)}

          \item[bcolr]


currently unused
            {\it (type=long\_pos)}

        \end{Ventry}

      \end{quote}

      \textbf{Return Value}
    \vspace{-1ex}

      \begin{quote}

num.
      {\it (type=int)}

      \end{quote}

\textbf{Note:} 
e.g. \emph{todo}


\textbf{Status:} 
Untested + NoDoc + NoDemo = NOT OK


    \end{boxedminipage}

    \label{xformslib:flflimage:flimage_add_marker_struct}
    \index{xformslib \textit{(package)}!xformslib.flflimage \textit{(module)}!xformslib.flflimage.flimage\_add\_marker\_struct \textit{(function)}}

    \vspace{0.5ex}

\hspace{.8\funcindent}\begin{boxedminipage}{\funcwidth}

    \raggedright \textbf{flimage\_add\_marker\_struct}(\textit{pImage}, \textit{pImageMarker})

    \vspace{-1.5ex}

    \rule{\textwidth}{0.5\fboxrule}
\setlength{\parskip}{2ex}

Adds simple markers (arrows, circles etc) to an image, using
xfdata.FLIMAGE\_MARKER class instance.

-{}-
\setlength{\parskip}{1ex}
      \textbf{Parameters}
      \vspace{-1ex}

      \begin{quote}
        \begin{Ventry}{xxxxxxxxxxxx}

          \item[pImage]


image
            {\it (type=pointer to xfdata.FL\_IMAGE)}

          \item[pImageMarker]


flimagemarker class instance
            {\it (type=pointer to xfdata.FLIMAGE\_MARKER)}

        \end{Ventry}

      \end{quote}

      \textbf{Return Value}
    \vspace{-1ex}

      \begin{quote}

num.
      {\it (type=int)}

      \end{quote}

\textbf{Note:} 
e.g. \emph{todo}


\textbf{Status:} 
Untested + Doc + NoDemo = NOT OK


    \end{boxedminipage}

    \label{xformslib:flflimage:flimage_define_marker}
    \index{xformslib \textit{(package)}!xformslib.flflimage \textit{(module)}!xformslib.flflimage.flimage\_define\_marker \textit{(function)}}

    \vspace{0.5ex}

\hspace{.8\funcindent}\begin{boxedminipage}{\funcwidth}

    \raggedright \textbf{flimage\_define\_marker}(\textit{mkname}, \textit{py\_FlimageMarkerDraw}, \textit{psdraw})

    \vspace{-1.5ex}

    \rule{\textwidth}{0.5\fboxrule}
\setlength{\parskip}{2ex}

Defines a custom marker, using a specific function for drawing it.

-{}-
\setlength{\parskip}{1ex}
      \textbf{Parameters}
      \vspace{-1ex}

      \begin{quote}
        \begin{Ventry}{xxxxxxxxxxxxxxxxxxxx}

          \item[mkname]


name of the marker \emph{todo}
            {\it (type=str)}

          \item[py\_FlimageMarkerDraw]


name referring to function(pImageMarker)
            {\it (type=python function to draw marker, no return?)}

          \item[psdraw]


string that draws a marker in a square with the corner coordinates
(-1, -1), (-1, 1), (1, 1) and (1, -1) in PostScript. e.g. the rectangle
marker has the following psdraw string: ``-1 -1 moveto -1  1 lineto
1  1 lineto  1 -1 lineto  closepath''
            {\it (type=str)}

        \end{Ventry}

      \end{quote}

      \textbf{Return Value}
    \vspace{-1ex}

      \begin{quote}

num.
      {\it (type=int)}

      \end{quote}

\textbf{Note:} 
e.g. \emph{todo}


\textbf{Status:} 
Untested + Doc + NoDemo = NOT OK


    \end{boxedminipage}

    \label{xformslib:flflimage:flimage_delete_all_markers}
    \index{xformslib \textit{(package)}!xformslib.flflimage \textit{(module)}!xformslib.flflimage.flimage\_delete\_all\_markers \textit{(function)}}

    \vspace{0.5ex}

\hspace{.8\funcindent}\begin{boxedminipage}{\funcwidth}

    \raggedright \textbf{flimage\_delete\_all\_markers}(\textit{pImage})

    \vspace{-1.5ex}

    \rule{\textwidth}{0.5\fboxrule}
\setlength{\parskip}{2ex}

Deletes all markers added to an image

-{}-
\setlength{\parskip}{1ex}
      \textbf{Parameters}
      \vspace{-1ex}

      \begin{quote}
        \begin{Ventry}{xxxxxx}

          \item[pImage]


image
            {\it (type=pointer to xfdata.FL\_IMAGE)}

        \end{Ventry}

      \end{quote}

\textbf{Note:} 
e.g. \emph{todo}


\textbf{Status:} 
Untested + Doc + NoDemo = NOT OK


    \end{boxedminipage}

    \label{xformslib:flflimage:flimage_render_annotation}
    \index{xformslib \textit{(package)}!xformslib.flflimage \textit{(module)}!xformslib.flflimage.flimage\_render\_annotation \textit{(function)}}

    \vspace{0.5ex}

\hspace{.8\funcindent}\begin{boxedminipage}{\funcwidth}

    \raggedright \textbf{flimage\_render\_annotation}(\textit{pImage}, \textit{win})

    \vspace{-1.5ex}

    \rule{\textwidth}{0.5\fboxrule}
\setlength{\parskip}{2ex}

Makes the annotations a part of the image pixel. By default annotations
placed on the image are kept seperate from the image pixels themselves, as
keeping the annotation seperate makes it possible to later edit the
annotations, and typically the screen has a lower resolutions than other
output devices (by keeping the annotations separate from the pixels makes
it possible to obtain better image qualities when the annotations are
rendered on higher-resolution devices, e.g. a PostScript printer). Note
that during rendering the image type may change depending on the
capabilities of win. Annotations that were kept separately are deleted.
The image must have been displayed at least once prior to calling this
function for it to work correctly.

-{}-
\setlength{\parskip}{1ex}
      \textbf{Parameters}
      \vspace{-1ex}

      \begin{quote}
        \begin{Ventry}{xxxxxx}

          \item[pImage]


image
            {\it (type=pointer to xfdata.FL\_IMAGE)}

          \item[win]


window
            {\it (type=long\_pos)}

        \end{Ventry}

      \end{quote}

      \textbf{Return Value}
    \vspace{-1ex}

      \begin{quote}

num., or -1 (on failure)
      {\it (type=int)}

      \end{quote}

\textbf{Note:} 
e.g. \emph{todo}


\textbf{Status:} 
Untested + Doc + NoDemo = NOT OK


    \end{boxedminipage}

    \label{xformslib:flflimage:flimage_error}
    \index{xformslib \textit{(package)}!xformslib.flflimage \textit{(module)}!xformslib.flflimage.flimage\_error \textit{(function)}}

    \vspace{0.5ex}

\hspace{.8\funcindent}\begin{boxedminipage}{\funcwidth}

    \raggedright \textbf{flimage\_error}(\textit{pImage}, \textit{text})

    \vspace{-1.5ex}

    \rule{\textwidth}{0.5\fboxrule}
\setlength{\parskip}{2ex}

Calls the error message handler for an image.

-{}-
\setlength{\parskip}{1ex}
      \textbf{Parameters}
      \vspace{-1ex}

      \begin{quote}
        \begin{Ventry}{xxxxxx}

          \item[pImage]


image to be worked on
            {\it (type=pointer to xfdata.FL\_IMAGE)}

          \item[text]


a brief message, such as ``memory allocation failed'' etc..
            {\it (type=str)}

        \end{Ventry}

      \end{quote}

\textbf{Note:} 
e.g. \emph{todo}


\textbf{Status:} 
Untested + Doc + NoDemo = NOT OK


    \end{boxedminipage}

    \label{xformslib:flflimage:flimage_enable_pnm}
    \index{xformslib \textit{(package)}!xformslib.flflimage \textit{(module)}!xformslib.flflimage.flimage\_enable\_pnm \textit{(function)}}

    \vspace{0.5ex}

\hspace{.8\funcindent}\begin{boxedminipage}{\funcwidth}

    \raggedright \textbf{flimage\_enable\_pnm}()

    \vspace{-1.5ex}

    \rule{\textwidth}{0.5\fboxrule}
\setlength{\parskip}{2ex}

Enables use of PNM (Portable anymap) image format

-{}-
\setlength{\parskip}{1ex}
\textbf{Note:} 
e.g. flimage\_enable\_pnm()


\textbf{Status:} 
Tested + Doc + NoDemo = OK


    \end{boxedminipage}

    \label{xformslib:flflimage:flimage_set_fits_bits}
    \index{xformslib \textit{(package)}!xformslib.flflimage \textit{(module)}!xformslib.flflimage.flimage\_set\_fits\_bits \textit{(function)}}

    \vspace{0.5ex}

\hspace{.8\funcindent}\begin{boxedminipage}{\funcwidth}

    \raggedright \textbf{flimage\_set\_fits\_bits}(\textit{nbits})

    \vspace{-1.5ex}

    \rule{\textwidth}{0.5\fboxrule}
\setlength{\parskip}{2ex}

Sets the number of bit of a FITS image.

-{}-
\setlength{\parskip}{1ex}
      \textbf{Parameters}
      \vspace{-1ex}

      \begin{quote}
        \begin{Ventry}{xxxxx}

          \item[nbits]


number of bit to be set
            {\it (type=int)}

        \end{Ventry}

      \end{quote}

      \textbf{Return Value}
    \vspace{-1ex}

      \begin{quote}

old number of bit, or negative number (on failure)
      {\it (type=int)}

      \end{quote}

\textbf{Note:} 
e.g. flimage\_set\_fits\_bits(16)


\textbf{Status:} 
Tested + Doc + NoDemo = OK


    \end{boxedminipage}

    \label{xformslib:flflimage:flimage_jpeg_output_options}
    \index{xformslib \textit{(package)}!xformslib.flflimage \textit{(module)}!xformslib.flflimage.flimage\_jpeg\_output\_options \textit{(function)}}

    \vspace{0.5ex}

\hspace{.8\funcindent}\begin{boxedminipage}{\funcwidth}

    \raggedright \textbf{flimage\_jpeg\_output\_options}(\textit{pImageJpegOption})

    \vspace{-1.5ex}

    \rule{\textwidth}{0.5\fboxrule}
\setlength{\parskip}{2ex}

Sets quality and smoothing options of a JPEG image, using
xfdata.FLIMAGE\_JPEG\_OPTION. The default quality factor for JPEG output
is 75. In general, the higher the quality factor rhe better the image
is, but the file size gets larger. The default smoothing factor is 0.

-{}-
\setlength{\parskip}{1ex}
      \textbf{Parameters}
      \vspace{-1ex}

      \begin{quote}
        \begin{Ventry}{xxxxxxxxxxxxxxxx}

          \item[pImageJpegOption]


flimage jpeg option class instance
            {\it (type=pointer to xfdata.FLIMAGE\_JPEG\_OPTION)}

        \end{Ventry}

      \end{quote}

\textbf{Note:} 
e.g. \emph{todo}


\textbf{Status:} 
Untested + Doc + NoDemo = NOT OK


    \end{boxedminipage}

    \label{xformslib:flflimage:flimage_pnm_output_options}
    \index{xformslib \textit{(package)}!xformslib.flflimage \textit{(module)}!xformslib.flflimage.flimage\_pnm\_output\_options \textit{(function)}}

    \vspace{0.5ex}

\hspace{.8\funcindent}\begin{boxedminipage}{\funcwidth}

    \raggedright \textbf{flimage\_pnm\_output\_options}(\textit{rawformat})

    \vspace{-1.5ex}

    \rule{\textwidth}{0.5\fboxrule}
\setlength{\parskip}{2ex}

Sets variant options for PNM (ppm, pgm and pbm) images.

-{}-
\setlength{\parskip}{1ex}
      \textbf{Parameters}
      \vspace{-1ex}

      \begin{quote}
        \begin{Ventry}{xxxxxxxxx}

          \item[rawformat]


flag of supported variants. Values 1 (binary raw format, default)
or 0 (ASCII format). If the output image is of type
xfdata.FL\_IMAGE\_GRAY16, it is always ASCII format
            {\it (type=int)}

        \end{Ventry}

      \end{quote}

\textbf{Note:} 
e.g. \emph{todo}


\textbf{Status:} 
Untested + Doc + NoDemo = NOT OK


    \end{boxedminipage}

    \label{xformslib:flflimage:flimage_gif_output_options}
    \index{xformslib \textit{(package)}!xformslib.flflimage \textit{(module)}!xformslib.flflimage.flimage\_gif\_output\_options \textit{(function)}}

    \vspace{0.5ex}

\hspace{.8\funcindent}\begin{boxedminipage}{\funcwidth}

    \raggedright \textbf{flimage\_gif\_output\_options}(\textit{interlace})

    \vspace{-1.5ex}

    \rule{\textwidth}{0.5\fboxrule}
\setlength{\parskip}{2ex}

Sets options of GIF images.

-{}-
\setlength{\parskip}{1ex}
      \textbf{Parameters}
      \vspace{-1ex}

      \begin{quote}
        \begin{Ventry}{xxxxxxxxx}

          \item[interlace]


flag if interlace is enabled/disabled. Values 1 (interlaced) or 0
(not interlaced).
            {\it (type=int)}

        \end{Ventry}

      \end{quote}

\textbf{Note:} 
e.g. \emph{todo}


\textbf{Status:} 
Untested + Doc + NoDemo = NOT OK


    \end{boxedminipage}

    \label{xformslib:flflimage:flimage_ps_options}
    \index{xformslib \textit{(package)}!xformslib.flflimage \textit{(module)}!xformslib.flflimage.flimage\_ps\_options \textit{(function)}}

    \vspace{0.5ex}

\hspace{.8\funcindent}\begin{boxedminipage}{\funcwidth}

    \raggedright \textbf{flimage\_ps\_options}()

    \vspace{-1.5ex}

    \rule{\textwidth}{0.5\fboxrule}
\setlength{\parskip}{2ex}

Sets reading and writing options for PostScript.

-{}-
\setlength{\parskip}{1ex}
      \textbf{Return Value}
    \vspace{-1ex}

      \begin{quote}

flpscontrol class instance (pFlpsControl)
      {\it (type=pointer to xfdata.FLPS\_CONTROL)}

      \end{quote}

\textbf{Note:} 
e.g. \emph{todo}


\textbf{Status:} 
Untested + Doc + NoDemo = NOT OK


    \end{boxedminipage}

    \label{xformslib:flflimage:flimage_jpeg_output_options}
    \index{xformslib \textit{(package)}!xformslib.flflimage \textit{(module)}!xformslib.flflimage.flimage\_jpeg\_output\_options \textit{(function)}}

    \vspace{0.5ex}

\hspace{.8\funcindent}\begin{boxedminipage}{\funcwidth}

    \raggedright \textbf{flimage\_jpeg\_options}(\textit{pImageJpegOption})

    \vspace{-1.5ex}

    \rule{\textwidth}{0.5\fboxrule}
\setlength{\parskip}{2ex}

Sets quality and smoothing options of a JPEG image, using
xfdata.FLIMAGE\_JPEG\_OPTION. The default quality factor for JPEG output
is 75. In general, the higher the quality factor rhe better the image
is, but the file size gets larger. The default smoothing factor is 0.

-{}-
\setlength{\parskip}{1ex}
      \textbf{Parameters}
      \vspace{-1ex}

      \begin{quote}
        \begin{Ventry}{xxxxxxxxxxxxxxxx}

          \item[pImageJpegOption]


flimage jpeg option class instance
            {\it (type=pointer to xfdata.FLIMAGE\_JPEG\_OPTION)}

        \end{Ventry}

      \end{quote}

\textbf{Note:} 
e.g. \emph{todo}


\textbf{Status:} 
Untested + Doc + NoDemo = NOT OK


    \end{boxedminipage}

    \label{xformslib:flflimage:flimage_pnm_output_options}
    \index{xformslib \textit{(package)}!xformslib.flflimage \textit{(module)}!xformslib.flflimage.flimage\_pnm\_output\_options \textit{(function)}}

    \vspace{0.5ex}

\hspace{.8\funcindent}\begin{boxedminipage}{\funcwidth}

    \raggedright \textbf{flimage\_pnm\_options}(\textit{rawformat})

    \vspace{-1.5ex}

    \rule{\textwidth}{0.5\fboxrule}
\setlength{\parskip}{2ex}

Sets variant options for PNM (ppm, pgm and pbm) images.

-{}-
\setlength{\parskip}{1ex}
      \textbf{Parameters}
      \vspace{-1ex}

      \begin{quote}
        \begin{Ventry}{xxxxxxxxx}

          \item[rawformat]


flag of supported variants. Values 1 (binary raw format, default)
or 0 (ASCII format). If the output image is of type
xfdata.FL\_IMAGE\_GRAY16, it is always ASCII format
            {\it (type=int)}

        \end{Ventry}

      \end{quote}

\textbf{Note:} 
e.g. \emph{todo}


\textbf{Status:} 
Untested + Doc + NoDemo = NOT OK


    \end{boxedminipage}

    \label{xformslib:flflimage:flimage_gif_output_options}
    \index{xformslib \textit{(package)}!xformslib.flflimage \textit{(module)}!xformslib.flflimage.flimage\_gif\_output\_options \textit{(function)}}

    \vspace{0.5ex}

\hspace{.8\funcindent}\begin{boxedminipage}{\funcwidth}

    \raggedright \textbf{flimage\_gif\_options}(\textit{interlace})

    \vspace{-1.5ex}

    \rule{\textwidth}{0.5\fboxrule}
\setlength{\parskip}{2ex}

Sets options of GIF images.

-{}-
\setlength{\parskip}{1ex}
      \textbf{Parameters}
      \vspace{-1ex}

      \begin{quote}
        \begin{Ventry}{xxxxxxxxx}

          \item[interlace]


flag if interlace is enabled/disabled. Values 1 (interlaced) or 0
(not interlaced).
            {\it (type=int)}

        \end{Ventry}

      \end{quote}

\textbf{Note:} 
e.g. \emph{todo}


\textbf{Status:} 
Untested + Doc + NoDemo = NOT OK


    \end{boxedminipage}

    \label{xformslib:flflimage:flimage_get_number_of_formats}
    \index{xformslib \textit{(package)}!xformslib.flflimage \textit{(module)}!xformslib.flflimage.flimage\_get\_number\_of\_formats \textit{(function)}}

    \vspace{0.5ex}

\hspace{.8\funcindent}\begin{boxedminipage}{\funcwidth}

    \raggedright \textbf{flimage\_get\_number\_of\_formats}()

    \vspace{-1.5ex}

    \rule{\textwidth}{0.5\fboxrule}
\setlength{\parskip}{2ex}

Obtains the number of currently supported image format.

-{}-
\setlength{\parskip}{1ex}
      \textbf{Return Value}
    \vspace{-1ex}

      \begin{quote}

number of formats supported, for reading or writing or both
      {\it (type=int)}

      \end{quote}

\textbf{Note:} 
e.g. \emph{todo}


\textbf{Status:} 
Untested + Doc + NoDemo = NOT OK


    \end{boxedminipage}

    \label{xformslib:flflimage:flimage_get_format_info}
    \index{xformslib \textit{(package)}!xformslib.flflimage \textit{(module)}!xformslib.flflimage.flimage\_get\_format\_info \textit{(function)}}

    \vspace{0.5ex}

\hspace{.8\funcindent}\begin{boxedminipage}{\funcwidth}

    \raggedright \textbf{flimage\_get\_format\_info}(\textit{nformat})

    \vspace{-1.5ex}

    \rule{\textwidth}{0.5\fboxrule}
\setlength{\parskip}{2ex}

Obtains detailed information for each image format.

-{}-
\setlength{\parskip}{1ex}
      \textbf{Parameters}
      \vspace{-1ex}

      \begin{quote}
        \begin{Ventry}{xxxxxxx}

          \item[nformat]


number between 1 and the return value of
flimage\_get\_number\_of\_formats()
            {\it (type=int)}

        \end{Ventry}

      \end{quote}

      \textbf{Return Value}
    \vspace{-1ex}

      \begin{quote}

ImageFormatInfo class instance
      {\it (type=pointer to xfdata.FLIMAGE\_FORMAT\_INFO)}

      \end{quote}

\textbf{Note:} 
e.g. \emph{todo}


\textbf{Status:} 
Untested + Doc + NoDemo = NOT OK


    \end{boxedminipage}

    \label{xformslib:flflimage:fl_get_matrix}
    \index{xformslib \textit{(package)}!xformslib.flflimage \textit{(module)}!xformslib.flflimage.fl\_get\_matrix \textit{(function)}}

    \vspace{0.5ex}

\hspace{.8\funcindent}\begin{boxedminipage}{\funcwidth}

    \raggedright \textbf{fl\_get\_matrix}(\textit{nrows}, \textit{ncols}, \textit{elemsize})

    \vspace{-1.5ex}

    \rule{\textwidth}{0.5\fboxrule}
\setlength{\parskip}{2ex}

Creates a 2-dimensional array of entities of size elemsize. The array
is of nrows by ncols in size.

-{}-
\setlength{\parskip}{1ex}
      \textbf{Parameters}
      \vspace{-1ex}

      \begin{quote}
        \begin{Ventry}{xxxxxxxx}

          \item[nrows]


number of rows
            {\it (type=int)}

          \item[ncols]


number of columns
            {\it (type=int)}

          \item[elemsize]


size of matrix in bytes
            {\it (type=int\_pos)}

        \end{Ventry}

      \end{quote}

      \textbf{Return Value}
    \vspace{-1ex}

      \begin{quote}

a matrix?
      {\it (type=\emph{todo})}

      \end{quote}

\textbf{Note:} 
e.g. \emph{todo}


\textbf{Status:} 
Untested + NoDoc + NoDemo = NOT OK


    \end{boxedminipage}

    \label{xformslib:flflimage:fl_make_matrix}
    \index{xformslib \textit{(package)}!xformslib.flflimage \textit{(module)}!xformslib.flflimage.fl\_make\_matrix \textit{(function)}}

    \vspace{0.5ex}

\hspace{.8\funcindent}\begin{boxedminipage}{\funcwidth}

    \raggedright \textbf{fl\_make\_matrix}(\textit{nrows}, \textit{ncols}, \textit{elemsize}, \textit{mem})

    \vspace{-1.5ex}

    \rule{\textwidth}{0.5\fboxrule}
\setlength{\parskip}{2ex}

Makes a matrix out of a given piece of memory.

-{}-
\setlength{\parskip}{1ex}
      \textbf{Parameters}
      \vspace{-1ex}

      \begin{quote}
        \begin{Ventry}{xxxxxxxx}

          \item[nrows]


number of rows
            {\it (type=int)}

          \item[ncols]


number of columns
            {\it (type=int)}

          \item[elemsize]


size of matrix in bytes
            {\it (type=int\_pos)}

          \item[mem]


memory
            {\it (type=\emph{todo})}

        \end{Ventry}

      \end{quote}

      \textbf{Return Value}
    \vspace{-1ex}

      \begin{quote}

\emph{todo}
      {\it (type=\emph{todo})}

      \end{quote}

\textbf{Note:} 
e.g. \emph{todo}


\textbf{Status:} 
Untested + NoDoc + NoDemo = NOT OK


    \end{boxedminipage}

    \label{xformslib:flflimage:fl_free_matrix}
    \index{xformslib \textit{(package)}!xformslib.flflimage \textit{(module)}!xformslib.flflimage.fl\_free\_matrix \textit{(function)}}

    \vspace{0.5ex}

\hspace{.8\funcindent}\begin{boxedminipage}{\funcwidth}

    \raggedright \textbf{fl\_free\_matrix}(\textit{mtrx})

    \vspace{-1.5ex}

    \rule{\textwidth}{0.5\fboxrule}
\setlength{\parskip}{2ex}

Frees a matrix allocated using fl\_get\_matrix() or fl\_make\_matrix().

-{}-
\setlength{\parskip}{1ex}
      \textbf{Parameters}
      \vspace{-1ex}

      \begin{quote}
        \begin{Ventry}{xxxx}

          \item[mtrx]


\emph{todo}
            {\it (type=\emph{todo})}

        \end{Ventry}

      \end{quote}

\textbf{Note:} 
e.g. \emph{todo}


\textbf{Status:} 
Untested + NoDoc + NoDemo = NOT OK


    \end{boxedminipage}

    \label{xformslib:flflimage:fl_lookup_RGBcolor}
    \index{xformslib \textit{(package)}!xformslib.flflimage \textit{(module)}!xformslib.flflimage.fl\_lookup\_RGBcolor \textit{(function)}}

    \vspace{0.5ex}

\hspace{.8\funcindent}\begin{boxedminipage}{\funcwidth}

    \raggedright \textbf{fl\_lookup\_RGBcolor}(\textit{colrname})

    \vspace{-1.5ex}

    \rule{\textwidth}{0.5\fboxrule}
\setlength{\parskip}{2ex}

\emph{todo}

-{}-
\setlength{\parskip}{1ex}
      \textbf{Parameters}
      \vspace{-1ex}

      \begin{quote}
        \begin{Ventry}{xxxxxxxx}

          \item[colrname]


text of color name
            {\it (type=str)}

        \end{Ventry}

      \end{quote}

      \textbf{Return Value}
    \vspace{-1ex}

      \begin{quote}

o or -1 (on failure), red value, green value, blue value
      {\it (type=int, int, int, int)}

      \end{quote}

\textbf{Note:} 
e.g. \emph{todo}


\textbf{Attention:} 
API change from XForms - upstream was
fl\_lookup\_RGBcolor(text, r, g, b)


\textbf{Status:} 
Untested + NoDoc + NoDemo = NOT OK


    \end{boxedminipage}

    \label{xformslib:flflimage:flimage_add_format}
    \index{xformslib \textit{(package)}!xformslib.flflimage \textit{(module)}!xformslib.flflimage.flimage\_add\_format \textit{(function)}}

    \vspace{0.5ex}

\hspace{.8\funcindent}\begin{boxedminipage}{\funcwidth}

    \raggedright \textbf{flimage\_add\_format}(\textit{formalname}, \textit{shortname}, \textit{extension}, \textit{imagetype}, \textit{py\_ImageIdentify}, \textit{py\_ImageDescription}, \textit{py\_ImageReadPixels}, \textit{py\_ImageWriteImage})

    \vspace{-1.5ex}

    \rule{\textwidth}{0.5\fboxrule}
\setlength{\parskip}{2ex}

Adds the newly specified image format to a recognized image format
pool in the library.

-{}-
\setlength{\parskip}{1ex}
      \textbf{Parameters}
      \vspace{-1ex}

      \begin{quote}
        \begin{Ventry}{xxxxxxxxxxxxxxxxxxx}

          \item[formalname]


the formal name of image format
            {\it (type=str)}

          \item[shortname]


an abbreviated name for the image format
            {\it (type=str)}

          \item[extension]


file extension. If it is None, shortname will be substituted
            {\it (type=str)}

          \item[imagetype]


The image type, generally one of the supported image types (e.g.
xfdata.FL\_IMAGE\_RGB), but it does not have to. For image file formats
that are capable of holding more than one type of images, this field
can be set to indicate this by ORing the supported types together
(e.g., xfdata.FL\_IMAGE\_RGB|FL\_IMAGE\_GRAY). However, when description
returns, the image type should be set to the actual type in the file.
            {\it (type=int)}

          \item[py\_ImageIdentify]


name referring to function(pFile) -> num.
This function should return 1 if the file pointed to by the file
pointer passed in is the expected image format (e.g by checking
signature). It should return a negative number if the file is not
recognized. The decision if the file pointer should be rewound or not
is between this function and the description function.
            {\it (type=function to identify format, returning value)}

          \item[py\_ImageDescription]


name referring to function(pImage) -> num.
This function in general should set the image dimension and type
fields (and colormap length for color index images) if successful, so
the driver can allocate the necessary memory for read pixel. Of
course, if read\_pixels elects to allocate memory itself, the
description function does not have to set any fields. However, if
reading should continue, the function should return 1 otherwise a
negative number.
            {\it (type=function to set description, returning value)}

          \item[py\_ImageReadPixels]


name referring to function(pImage) -> num.
This function reads the pixels from the file and fills one of the
pixel matrix in the image structure depending on the type. If reading
is successful, a non-negative number should be returned otherwise a
negative number should be returned. Upon entry,
pImage.contents.completed is set to zero. The function should not
close the file.
            {\it (type=python function to read pixels, returning value)}

          \item[py\_ImageWriteImage]


name referring to function(pImage) -> num.
This function takes an image structure and should write the image out
in a format it knows. Prior to calling this routine, the driver will
have already converted the image type to the type it wants. The
function should return 1 on success and a negative number otherwise.
If only reading of the image format is supported this parameter can
be set to None. The function should write to file stream
pImage.contents.fpout.
            {\it (type=python function to write image, returning value)}

        \end{Ventry}

      \end{quote}

      \textbf{Return Value}
    \vspace{-1ex}

      \begin{quote}

num.
      {\it (type=int)}

      \end{quote}

\textbf{Note:} 
e.g. \emph{todo}


\textbf{Status:} 
Untested + NoDoc + NoDemo = NOT OK


    \end{boxedminipage}

    \label{xformslib:flflimage:flimage_set_annotation_support}
    \index{xformslib \textit{(package)}!xformslib.flflimage \textit{(module)}!xformslib.flflimage.flimage\_set\_annotation\_support \textit{(function)}}

    \vspace{0.5ex}

\hspace{.8\funcindent}\begin{boxedminipage}{\funcwidth}

    \raggedright \textbf{flimage\_set\_annotation\_support}(\textit{in\_}, \textit{yesno})

    \vspace{-1.5ex}

    \rule{\textwidth}{0.5\fboxrule}
\setlength{\parskip}{2ex}

Set support for annotations.

-{}-
\setlength{\parskip}{1ex}
      \textbf{Parameters}
      \vspace{-1ex}

      \begin{quote}
        \begin{Ventry}{xxxxx}

          \item[in\_]


image number?. Values between 0 and number of images?
            {\it (type=int)}

          \item[yesno]


flag to enable/disable support. Values 1 (to enable) or 0 (to disable)
            {\it (type=int)}

        \end{Ventry}

      \end{quote}

\textbf{Note:} 
e.g. \emph{todo}


\textbf{Status:} 
Untested + NoDoc + NoDemo = NOT OK


    \end{boxedminipage}

    \label{xformslib:flflimage:flimage_getcolormap}
    \index{xformslib \textit{(package)}!xformslib.flflimage \textit{(module)}!xformslib.flflimage.flimage\_getcolormap \textit{(function)}}

    \vspace{0.5ex}

\hspace{.8\funcindent}\begin{boxedminipage}{\funcwidth}

    \raggedright \textbf{flimage\_getcolormap}(\textit{pImage})

    \vspace{-1.5ex}

    \rule{\textwidth}{0.5\fboxrule}
\setlength{\parskip}{2ex}

Obtains color map for an image

-{}-
\setlength{\parskip}{1ex}
      \textbf{Parameters}
      \vspace{-1ex}

      \begin{quote}
        \begin{Ventry}{xxxxxx}

          \item[pImage]


image
            {\it (type=pointer to xfdata.FL\_IMAGE)}

        \end{Ventry}

      \end{quote}

      \textbf{Return Value}
    \vspace{-1ex}

      \begin{quote}

0, or -1 (on failure)
      {\it (type=int)}

      \end{quote}

\textbf{Note:} 
e.g. \emph{todo}


\textbf{Status:} 
Untested + NoDoc + NoDemo = NOT OK


    \end{boxedminipage}

    \label{xformslib:flflimage:fl_select_mediancut_quantizer}
    \index{xformslib \textit{(package)}!xformslib.flflimage \textit{(module)}!xformslib.flflimage.fl\_select\_mediancut\_quantizer \textit{(function)}}

    \vspace{0.5ex}

\hspace{.8\funcindent}\begin{boxedminipage}{\funcwidth}

    \raggedright \textbf{fl\_select\_mediancut\_quantizer}()

    \vspace{-1.5ex}

    \rule{\textwidth}{0.5\fboxrule}
\setlength{\parskip}{2ex}

Selects median cut quantizer, who uses Heckbert's median cut algorithm
followed by Floyd-Steinberg dithering after which the pixels are mapped
to the colors selected. This tends to give better images because of the
dithering step. However, in this particular implementation, the number
of quantized colors is limited to 256. Color quantization is one way of
reduce the number of colors in the original image to display a RGB image
on a color-mapped device of limited depth.

-{}-
\setlength{\parskip}{1ex}
\textbf{Note:} 
e.g. flimage\_select\_mediancut\_quantizer()


\textbf{Status:} 
Tested + Doc + NoDemo = NOT OK


    \end{boxedminipage}

    \label{xformslib:flflimage:flimage_convolve}
    \index{xformslib \textit{(package)}!xformslib.flflimage \textit{(module)}!xformslib.flflimage.flimage\_convolve \textit{(function)}}

    \vspace{0.5ex}

\hspace{.8\funcindent}\begin{boxedminipage}{\funcwidth}

    \raggedright \textbf{flimage\_convolve}(\textit{pImage}, \textit{kernel}, \textit{krows}, \textit{kcols})

    \vspace{-1.5ex}

    \rule{\textwidth}{0.5\fboxrule}
\setlength{\parskip}{2ex}

Takes a convolution kernel of krows by kcols and convolves it with
the image. The result replaces the input image.

-{}-
\setlength{\parskip}{1ex}
      \textbf{Parameters}
      \vspace{-1ex}

      \begin{quote}
        \begin{Ventry}{xxxxxx}

          \item[pImage]


image
            {\it (type=pointer to xfdata.FL\_IMAGE)}

          \item[kernel]


The kernel size should be odd, and should be allocated by
fl\_get\_matrix(). \emph{todo}
            {\it (type=\emph{todo})}

          \item[krows]


number of kernel rows
            {\it (type=int)}

          \item[kcols]


number of kernel cols
            {\it (type=int)}

        \end{Ventry}

      \end{quote}

      \textbf{Return Value}
    \vspace{-1ex}

      \begin{quote}

positive num., or negative num (on failure)
      {\it (type=int)}

      \end{quote}

\textbf{Note:} 
e.g. \emph{todo}


\textbf{Status:} 
Untested + NoDoc + NoDemo = NOT OK


    \end{boxedminipage}

    \label{xformslib:flflimage:flimage_convolvea}
    \index{xformslib \textit{(package)}!xformslib.flflimage \textit{(module)}!xformslib.flflimage.flimage\_convolvea \textit{(function)}}

    \vspace{0.5ex}

\hspace{.8\funcindent}\begin{boxedminipage}{\funcwidth}

    \raggedright \textbf{flimage\_convolvea}(\textit{pImage}, \textit{kernel}, \textit{krow}, \textit{kcol})

    \vspace{-1.5ex}

    \rule{\textwidth}{0.5\fboxrule}
\setlength{\parskip}{2ex}

Takes a convolution kernel of krow by kcol and convolves it with
the image. The result replaces the input image. It uses a kernel that's a
C 2-dimensional array (cast to a pointer to int).

-{}-
\setlength{\parskip}{1ex}
      \textbf{Parameters}
      \vspace{-1ex}

      \begin{quote}
        \begin{Ventry}{xxxxxx}

          \item[pImage]


image
            {\it (type=pointer to xfdata.FL\_IMAGE)}

          \item[kernel]


C 2-dimensional array \emph{todo}
            {\it (type=\emph{todo})}

          \item[krow]


number of kernel rows
            {\it (type=int)}

          \item[kcol]


number of kernel columns
            {\it (type=int)}

        \end{Ventry}

      \end{quote}

      \textbf{Return Value}
    \vspace{-1ex}

      \begin{quote}

positive num., or negative num (on failure)
      {\it (type=int)}

      \end{quote}

\textbf{Note:} 
e.g. \emph{todo}


\textbf{Status:} 
Untested + NoDoc + NoDemo = NOT OK


    \end{boxedminipage}

    \label{xformslib:flflimage:flimage_tint}
    \index{xformslib \textit{(package)}!xformslib.flflimage \textit{(module)}!xformslib.flflimage.flimage\_tint \textit{(function)}}

    \vspace{0.5ex}

\hspace{.8\funcindent}\begin{boxedminipage}{\funcwidth}

    \raggedright \textbf{flimage\_tint}(\textit{pImage}, \textit{packed}, \textit{opacity})

    \vspace{-1.5ex}

    \rule{\textwidth}{0.5\fboxrule}
\setlength{\parskip}{2ex}

Emulates the effect of looking at an image through a piece of colored
glass. Tint is most useful in cases where you want to put some annotations
on the image, but do not want to use a uniform and opaque background that
completely obscures the image behind. By using tint, you can have a
background that provides some contrast to the text, yet not obscures the
image beneath completely. Tint operation uses the subimage settings.

-{}-
\setlength{\parskip}{1ex}
      \textbf{Parameters}
      \vspace{-1ex}

      \begin{quote}
        \begin{Ventry}{xxxxxxx}

          \item[pImage]


image
            {\it (type=pointer to xfdata.FL\_IMAGE)}

          \item[packed]


packed RGB color, specifying the color of the glass.
            {\it (type=int\_pos)}

          \item[opacity]


how much the color of the image is absorbed by the glass. Values
between 0 (the glass is totally transparent, i.e. the glass has no
effect) and 1.0 (total opaqueness, i.e. all you see is the color of
the glass. Any value between these two extremes results in a color
that is a combination of the pixel color and the glass color.
            {\it (type=float)}

        \end{Ventry}

      \end{quote}

      \textbf{Return Value}
    \vspace{-1ex}

      \begin{quote}

num.
      {\it (type=int)}

      \end{quote}

\textbf{Notes:}
\begin{quote}
  \begin{itemize}

  \item
    \setlength{\parskip}{0.6ex}

e.g. pkdcolr = flxbasic.FL\_PACK3(20, 30, 40)


  \item 
e.g. exval = flimage\_tint(pimg, pkdcolr, 0.5)


\end{itemize}

\end{quote}

\textbf{Status:} 
Tested + Doc + NoDemo = OK


    \end{boxedminipage}

    \label{xformslib:flflimage:flimage_rotate}
    \index{xformslib \textit{(package)}!xformslib.flflimage \textit{(module)}!xformslib.flflimage.flimage\_rotate \textit{(function)}}

    \vspace{0.5ex}

\hspace{.8\funcindent}\begin{boxedminipage}{\funcwidth}

    \raggedright \textbf{flimage\_rotate}(\textit{pImage}, \textit{angle}, \textit{subpixel})

    \vspace{-1.5ex}

    \rule{\textwidth}{0.5\fboxrule}
\setlength{\parskip}{2ex}

Does an image rotation. Repeated rotations should be avoided if
possible. If you have to call it more than once it's a good idea to crop
after rotations in order to get rid of the regions that contain only fill
color.

-{}-
\setlength{\parskip}{1ex}
      \textbf{Parameters}
      \vspace{-1ex}

      \begin{quote}
        \begin{Ventry}{xxxxxxxx}

          \item[pImage]


image
            {\it (type=pointer to xfdata.FL\_IMAGE)}

          \item[angle]


the angle in one-tenth of a degree (i.e., a 45 degree rotation should
be specified as 450) with a positive sign for counter-clock rotation.
            {\it (type=int)}

          \item[subpixel]


if subpixel sampling should be enabled. Values (from xfdata.py)
FLIMAGE\_NOSUBPIXEL or FLIMAGE\_SUBPIXEL. If enabled, the resulting image
pixels are interpolated from the original pixels; this usually has an
``anti-aliasing'' effect that leads to less severe jagged edges and
similar artifacts commonly encountered in rotations. However, it also
means that a color indexed image gets converted to a RGB image. If
preserving the pixel value is important, you should not turn subpixel
sampling on.
            {\it (type=int)}

        \end{Ventry}

      \end{quote}

      \textbf{Return Value}
    \vspace{-1ex}

      \begin{quote}

num., or negative num. (on failure)
      {\it (type=int)}

      \end{quote}

\textbf{Note:} 
e.g. \emph{todo}


\textbf{Status:} 
Untested + Doc + NoDemo = NOT OK


    \end{boxedminipage}

    \label{xformslib:flflimage:flimage_flip}
    \index{xformslib \textit{(package)}!xformslib.flflimage \textit{(module)}!xformslib.flflimage.flimage\_flip \textit{(function)}}

    \vspace{0.5ex}

\hspace{.8\funcindent}\begin{boxedminipage}{\funcwidth}

    \raggedright \textbf{flimage\_flip}(\textit{pImage}, \textit{what})

    \vspace{-1.5ex}

    \rule{\textwidth}{0.5\fboxrule}
\setlength{\parskip}{2ex}

Does the mirror operation in x- or y-direction at the center. For
example, to flip the columns of an image, the left and right of the
image are flipped (just like having a vertical mirror in the center of
the image) thus the first pixel on any given row becomes the last, and
the last pixel becomes the first etc.

-{}-
\setlength{\parskip}{1ex}
      \textbf{Parameters}
      \vspace{-1ex}

      \begin{quote}
        \begin{Ventry}{xxxxxx}

          \item[pImage]


image
            {\it (type=pointer to xfdata.FL\_IMAGE)}

          \item[what]


desired direction of flipping. Values 'c' (column, horizontal flipping)
or 'r' (row, vertical flipping)
            {\it (type=int or char)}

        \end{Ventry}

      \end{quote}

      \textbf{Return Value}
    \vspace{-1ex}

      \begin{quote}

num.
      {\it (type=int)}

      \end{quote}

\textbf{Note:} 
e.g. \emph{todo}


\textbf{Status:} 
Untested + NoDoc + NoDemo = NOT OK


    \end{boxedminipage}

    \label{xformslib:flflimage:flimage_scale}
    \index{xformslib \textit{(package)}!xformslib.flflimage \textit{(module)}!xformslib.flflimage.flimage\_scale \textit{(function)}}

    \vspace{0.5ex}

\hspace{.8\funcindent}\begin{boxedminipage}{\funcwidth}

    \raggedright \textbf{flimage\_scale}(\textit{pImage}, \textit{newwidth}, \textit{newheight}, \textit{option})

    \vspace{-1.5ex}

    \rule{\textwidth}{0.5\fboxrule}
\setlength{\parskip}{2ex}

Scales an image to any desired size with or without subpixel sampling.
Without subpixel sampling simple pixel replication is used, otherwise a
box average algorithm is employed that yields an anti-aliased image with
much less artifacts.

-{}-
\setlength{\parskip}{1ex}
      \textbf{Parameters}
      \vspace{-1ex}

      \begin{quote}
        \begin{Ventry}{xxxxxxxxx}

          \item[pImage]


image
            {\it (type=pointer to xfdata.FL\_IMAGE)}

          \item[newwidth]


desired image width
            {\it (type=int)}

          \item[newheight]


desired image height
            {\it (type=int)}

          \item[option]


option to scale the image to the desired size but keeping the aspect
ratio of the image the same by filling the part of the image that
would otherwise be empty. Values (from xfdata.py) FLIMAGE\_NOSUBPIXEL,
FLIMAGE\_SUBPIXEL, FLIMAGE\_ASPECT, FLIMAGE\_CENTER, FLIMAGE\_NOCENTER.
Any value can be single or bitwise-ORed.
            {\it (type=int)}

        \end{Ventry}

      \end{quote}

      \textbf{Return Value}
    \vspace{-1ex}

      \begin{quote}

num.
      {\it (type=int)}

      \end{quote}

\textbf{Note:} 
e.g. \emph{todo}


\textbf{Status:} 
Untested + Doc + NoDemo = NOT OK


    \end{boxedminipage}

    \label{xformslib:flflimage:flimage_warp}
    \index{xformslib \textit{(package)}!xformslib.flflimage \textit{(module)}!xformslib.flflimage.flimage\_warp \textit{(function)}}

    \vspace{0.5ex}

\hspace{.8\funcindent}\begin{boxedminipage}{\funcwidth}

    \raggedright \textbf{flimage\_warp}(\textit{pImage}, \textit{mtrx}, \textit{newwidth}, \textit{newheight}, \textit{subpixel})

    \vspace{-1.5ex}

    \rule{\textwidth}{0.5\fboxrule}
\setlength{\parskip}{2ex}

Does transformation of pixel coordinates. Rotation, scaling, shearing
etc. are examples of (linear and non-perspective) image warping. User can
specify whatever size he/she wants and the warp function will fill the
empty grid location with the fill color. This is how the aspect ratio
preserving scaling is implemented. The image is transformed in place.

-{}-
\setlength{\parskip}{1ex}
      \textbf{Parameters}
      \vspace{-1ex}

      \begin{quote}
        \begin{Ventry}{xxxxxxxxx}

          \item[pImage]


image
            {\it (type=pointer to xfdata.FL\_IMAGE)}

          \item[mtrx]


the warp matrix
            {\it (type=\emph{todo})}

          \item[newwidth]


desired image width
            {\it (type=int)}

          \item[newheight]


desired image height
            {\it (type=int)}

          \item[subpixel]


if subpixel sampling should be used. Although subpixel sampling adds
processing time, it generally improves image quality significantly.
Values (from xfdata.py) any logical OR of FLIMAGE\_NOSUBPIXEL,
FLIMAGE\_SUBPIXEL and FLIMAGE\_NOCENTER (only useful if you specify an
image dimension that is larger than the warped image, and in that
case the warped image is flushed top-left within the image grid,
otherwise it is centered).
            {\it (type=int)}

        \end{Ventry}

      \end{quote}

      \textbf{Return Value}
    \vspace{-1ex}

      \begin{quote}

num.
      {\it (type=int)}

      \end{quote}

\textbf{Note:} 
e.g. \emph{todo}


\textbf{Status:} 
Untested + NoDoc + NoDemo = NOT OK


    \end{boxedminipage}

    \label{xformslib:flflimage:flimage_autocrop}
    \index{xformslib \textit{(package)}!xformslib.flflimage \textit{(module)}!xformslib.flflimage.flimage\_autocrop \textit{(function)}}

    \vspace{0.5ex}

\hspace{.8\funcindent}\begin{boxedminipage}{\funcwidth}

    \raggedright \textbf{flimage\_autocrop}(\textit{pImage}, \textit{bgcolr})

    \vspace{-1.5ex}

    \rule{\textwidth}{0.5\fboxrule}
\setlength{\parskip}{2ex}

Automatically crops an image using the background as the color to crop,
by searching the image from all four sides and removing all contiguous
regions of the uniform background from the sides. The image is modified in
place.

-{}-
\setlength{\parskip}{1ex}
      \textbf{Parameters}
      \vspace{-1ex}

      \begin{quote}
        \begin{Ventry}{xxxxxx}

          \item[pImage]


image
            {\it (type=pointer to xfdata.FL\_IMAGE)}

          \item[bgcolr]


background color to crop. If it's xfdata.FLIMAGE\_AUTOCOLOR, the
background is chosen as the first pixel of the image.
            {\it (type=int\_pos)}

        \end{Ventry}

      \end{quote}

      \textbf{Return Value}
    \vspace{-1ex}

      \begin{quote}

non-negative num., or -1 (on failure)
      {\it (type=int)}

      \end{quote}

\textbf{Note:} 
e.g. \emph{todo}


\textbf{Status:} 
Untested + Doc + NoDemo = NOT OK


    \end{boxedminipage}

    \label{xformslib:flflimage:flimage_get_autocrop}
    \index{xformslib \textit{(package)}!xformslib.flflimage \textit{(module)}!xformslib.flflimage.flimage\_get\_autocrop \textit{(function)}}

    \vspace{0.5ex}

\hspace{.8\funcindent}\begin{boxedminipage}{\funcwidth}

    \raggedright \textbf{flimage\_get\_autocrop}(\textit{pImage}, \textit{bgcolr})

    \vspace{-1.5ex}

    \rule{\textwidth}{0.5\fboxrule}
\setlength{\parskip}{2ex}

Obtains the auto-cropping offsets (from left, right, top and bottom
sides) of an image.

-{}-
\setlength{\parskip}{1ex}
      \textbf{Parameters}
      \vspace{-1ex}

      \begin{quote}
        \begin{Ventry}{xxxxxx}

          \item[pImage]


image
            {\it (type=pointer to xfdata.FL\_IMAGE)}

          \item[bgcolr]


background color to crop. If it's xfdata.FLIMAGE\_AUTOCOLOR, the
background is chosen as the first pixel of the image.
            {\it (type=int\_pos)}

        \end{Ventry}

      \end{quote}

      \textbf{Return Value}
    \vspace{-1ex}

      \begin{quote}

num., from left side (xl), from top side (yt), from right side
(xr), from bottom side (yb) offsets
      {\it (type=int, int, int, int, int)}

      \end{quote}

\textbf{Note:} 
e.g. \emph{todo}


\textbf{Attention:} 
API change from XForms - upstream was
flimage\_get\_autocrop(pImage, bk, xl, yt, xr, yb)


\textbf{Status:} 
Untested + Doc + NoDemo = NOT OK


    \end{boxedminipage}

    \label{xformslib:flflimage:flimage_crop}
    \index{xformslib \textit{(package)}!xformslib.flflimage \textit{(module)}!xformslib.flflimage.flimage\_crop \textit{(function)}}

    \vspace{0.5ex}

\hspace{.8\funcindent}\begin{boxedminipage}{\funcwidth}

    \raggedright \textbf{flimage\_crop}(\textit{pImage}, \textit{xl}, \textit{yt}, \textit{xr}, \textit{yb})

    \vspace{-1.5ex}

    \rule{\textwidth}{0.5\fboxrule}
\setlength{\parskip}{2ex}

Crops an image, using offsets supplied by the user.

-{}-
\setlength{\parskip}{1ex}
      \textbf{Parameters}
      \vspace{-1ex}

      \begin{quote}
        \begin{Ventry}{xxxxxx}

          \item[pImage]


image
            {\it (type=pointer to xfdata.FL\_IMAGE)}

          \item[xl]


offset from left side. If it is negative, it indicates enlargement of
the image. (e.g. if it is 1 the cropping removes the first column from
the image).
            {\it (type=int)}

          \item[yt]


offset from top side. If it is negative, it indicates enlargement of
the image. (e.g. if it is 1 the cropping removes the first row from
the image).
            {\it (type=int)}

          \item[xr]


offset from right side. If it is negative, it indicates enlargement
of the image. (e.g. if it is 1 the cropping removes the last column
from the image).
            {\it (type=int)}

          \item[yb]


offset from bottom side. If it is negative, it indicates enlargement
of the image. (e.g. if it is 1 the cropping removes the last row from
the image).
            {\it (type=int)}

        \end{Ventry}

      \end{quote}

      \textbf{Return Value}
    \vspace{-1ex}

      \begin{quote}

num.
      {\it (type=int)}

      \end{quote}

\textbf{Note:} 
e.g. \emph{todo}


\textbf{Status:} 
Untested + NoDoc + NoDemo = NOT OK


    \end{boxedminipage}

    \label{xformslib:flflimage:flimage_replace_pixel}
    \index{xformslib \textit{(package)}!xformslib.flflimage \textit{(module)}!xformslib.flflimage.flimage\_replace\_pixel \textit{(function)}}

    \vspace{0.5ex}

\hspace{.8\funcindent}\begin{boxedminipage}{\funcwidth}

    \raggedright \textbf{flimage\_replace\_pixel}(\textit{pImage}, \textit{targetcolr}, \textit{newcolr})

    \vspace{-1.5ex}

    \rule{\textwidth}{0.5\fboxrule}
\setlength{\parskip}{2ex}

Replaces all color targetcolr with the new desired color.

-{}-
\setlength{\parskip}{1ex}
      \textbf{Parameters}
      \vspace{-1ex}

      \begin{quote}
        \begin{Ventry}{xxxxxxxxxx}

          \item[pImage]


image
            {\it (type=pointer to xfdata.FL\_IMAGE)}

          \item[targetcolr]


color to be replaced
            {\it (type=int\_pos)}

          \item[newcolr]


new color to be used
            {\it (type=int\_pos)}

        \end{Ventry}

      \end{quote}

      \textbf{Return Value}
    \vspace{-1ex}

      \begin{quote}

0, or -1 (on failure)
      {\it (type=int)}

      \end{quote}

\textbf{Note:} 
e.g. \emph{todo}


\textbf{Status:} 
Untested + Doc + NoDemo = NOT OK


    \end{boxedminipage}

    \label{xformslib:flflimage:flimage_transform_pixels}
    \index{xformslib \textit{(package)}!xformslib.flflimage \textit{(module)}!xformslib.flflimage.flimage\_transform\_pixels \textit{(function)}}

    \vspace{0.5ex}

\hspace{.8\funcindent}\begin{boxedminipage}{\funcwidth}

    \raggedright \textbf{flimage\_transform\_pixels}(\textit{pImage}, \textit{red}, \textit{green}, \textit{blue})

    \vspace{-1.5ex}

    \rule{\textwidth}{0.5\fboxrule}
\setlength{\parskip}{2ex}

Processes an image in place with RGB transformation and replaces it.

-{}-
\setlength{\parskip}{1ex}
      \textbf{Parameters}
      \vspace{-1ex}

      \begin{quote}
        \begin{Ventry}{xxxxxx}

          \item[pImage]


image
            {\it (type=pointer to xfdata.FL\_IMAGE)}

          \item[red]


lookup tables for red color of a length of at least FL\_PCMAX + 1 (256).
            {\it (type=int)}

          \item[green]


lookup tables for green color of a length of at least FL\_PCMAX + 1
(256).
            {\it (type=int)}

          \item[blue]


lookup tables for blue color of a length of at least FL\_PCMAX + 1
(256).
            {\it (type=int)}

        \end{Ventry}

      \end{quote}

      \textbf{Return Value}
    \vspace{-1ex}

      \begin{quote}

positive num., or -1 (on failure)
      {\it (type=int)}

      \end{quote}

\textbf{Note:} 
e.g. \emph{todo}


\textbf{Status:} 
Untested + NoDoc + NoDemo = NOT OK


    \end{boxedminipage}

    \label{xformslib:flflimage:flimage_windowlevel}
    \index{xformslib \textit{(package)}!xformslib.flflimage \textit{(module)}!xformslib.flflimage.flimage\_windowlevel \textit{(function)}}

    \vspace{0.5ex}

\hspace{.8\funcindent}\begin{boxedminipage}{\funcwidth}

    \raggedright \textbf{flimage\_windowlevel}(\textit{pImage}, \textit{winlvl}, \textit{width})

    \vspace{-1.5ex}

    \rule{\textwidth}{0.5\fboxrule}
\setlength{\parskip}{2ex}

Sets the window level for an image. It it points to a multiple image,
window level parameters are changed for all images.

-{}-
\setlength{\parskip}{1ex}
      \textbf{Parameters}
      \vspace{-1ex}

      \begin{quote}
        \begin{Ventry}{xxxxxx}

          \item[pImage]


image
            {\it (type=pointer to xfdata.FL\_IMAGE)}

          \item[winlvl]


new level of window
            {\it (type=int)}

          \item[width]


width. If zero disables window leveling.
            {\it (type=int)}

        \end{Ventry}

      \end{quote}

      \textbf{Return Value}
    \vspace{-1ex}

      \begin{quote}

1 (if window level parameters are modified), otherwise 0 is
returned.
      {\it (type=int)}

      \end{quote}

\textbf{Note:} 
e.g. \emph{todo}


\textbf{Status:} 
Untested + NoDoc + NoDemo = NOT OK


    \end{boxedminipage}

    \label{xformslib:flflimage:flimage_enhance}
    \index{xformslib \textit{(package)}!xformslib.flflimage \textit{(module)}!xformslib.flflimage.flimage\_enhance \textit{(function)}}

    \vspace{0.5ex}

\hspace{.8\funcindent}\begin{boxedminipage}{\funcwidth}

    \raggedright \textbf{flimage\_enhance}(\textit{pImage}, \textit{delta})

    \vspace{-1.5ex}

    \rule{\textwidth}{0.5\fboxrule}
\setlength{\parskip}{2ex}

\emph{todo}

-{}-
\setlength{\parskip}{1ex}
      \textbf{Parameters}
      \vspace{-1ex}

      \begin{quote}
        \begin{Ventry}{xxxxxx}

          \item[pImage]


image
            {\it (type=pointer to xfdata.FL\_IMAGE)}

          \item[delta]


unused.
            {\it (type=int)}

        \end{Ventry}

      \end{quote}

      \textbf{Return Value}
    \vspace{-1ex}

      \begin{quote}

0
      {\it (type=int)}

      \end{quote}

\textbf{Note:} 
e.g. \emph{todo}


\textbf{Status:} 
Untested + NoDoc + NoDemo = NOT OK


    \end{boxedminipage}

    \label{xformslib:flflimage:flimage_from_pixmap}
    \index{xformslib \textit{(package)}!xformslib.flflimage \textit{(module)}!xformslib.flflimage.flimage\_from\_pixmap \textit{(function)}}

    \vspace{0.5ex}

\hspace{.8\funcindent}\begin{boxedminipage}{\funcwidth}

    \raggedright \textbf{flimage\_from\_pixmap}(\textit{pImage}, \textit{pixmap})

    \vspace{-1.5ex}

    \rule{\textwidth}{0.5\fboxrule}
\setlength{\parskip}{2ex}

Converts a Pixmap to an image. \emph{todo}

-{}-
\setlength{\parskip}{1ex}
      \textbf{Parameters}
      \vspace{-1ex}

      \begin{quote}
        \begin{Ventry}{xxxxxx}

          \item[pImage]


image
            {\it (type=pointer to xfdata.FL\_IMAGE)}

          \item[pixmap]


pixmap id
            {\it (type=long\_pos)}

        \end{Ventry}

      \end{quote}

      \textbf{Return Value}
    \vspace{-1ex}

      \begin{quote}

num.
      {\it (type=int)}

      \end{quote}

\textbf{Note:} 
e.g. \emph{todo}


\textbf{Status:} 
Untested + NoDoc + NoDemo = NOT OK


    \end{boxedminipage}

    \label{xformslib:flflimage:flimage_to_pixmap}
    \index{xformslib \textit{(package)}!xformslib.flflimage \textit{(module)}!xformslib.flflimage.flimage\_to\_pixmap \textit{(function)}}

    \vspace{0.5ex}

\hspace{.8\funcindent}\begin{boxedminipage}{\funcwidth}

    \raggedright \textbf{flimage\_to\_pixmap}(\textit{pImage}, \textit{win})

    \vspace{-1.5ex}

    \rule{\textwidth}{0.5\fboxrule}
\setlength{\parskip}{2ex}

Converts an image into a Pixmap (a server side resource) that can be
used in the pixmap object.

-{}-
\setlength{\parskip}{1ex}
      \textbf{Parameters}
      \vspace{-1ex}

      \begin{quote}
        \begin{Ventry}{xxxxxx}

          \item[pImage]


image
            {\it (type=pointer to xfdata.FL\_IMAGE)}

          \item[win]


window id
            {\it (type=long\_pos)}

        \end{Ventry}

      \end{quote}

      \textbf{Return Value}
    \vspace{-1ex}

      \begin{quote}

pixmap id
      {\it (type=long\_pos)}

      \end{quote}

\textbf{Note:} 
e.g. \emph{todo}


\textbf{Status:} 
Untested + Doc + NoDemo = NOT OK


    \end{boxedminipage}

    \label{xformslib:flflimage:flimage_dup}
    \index{xformslib \textit{(package)}!xformslib.flflimage \textit{(module)}!xformslib.flflimage.flimage\_dup \textit{(function)}}

    \vspace{0.5ex}

\hspace{.8\funcindent}\begin{boxedminipage}{\funcwidth}

    \raggedright \textbf{flimage\_dup}(\textit{pImage})

    \vspace{-1.5ex}

    \rule{\textwidth}{0.5\fboxrule}
\setlength{\parskip}{2ex}

Duplicates an image and returns the duplicated image. At the moment,
only the first image is duplicated even if the input image has multiple
frames. Furthermore, markers and annotations are not duplicated.

-{}-
\setlength{\parskip}{1ex}
      \textbf{Parameters}
      \vspace{-1ex}

      \begin{quote}
        \begin{Ventry}{xxxxxx}

          \item[pImage]


image to be duplicated
            {\it (type=pointer to xfdata.FL\_IMAGE)}

        \end{Ventry}

      \end{quote}

      \textbf{Return Value}
    \vspace{-1ex}

      \begin{quote}

duplicated image class instance
      {\it (type=pointer to xfdata.FL\_IMAGE)}

      \end{quote}

\textbf{Note:} 
e.g. pimg2 = flimage\_dup(pimg1)


\textbf{Status:} 
Untested + Doc + NoDemo = NOT OK


    \end{boxedminipage}

    \label{xformslib:flflimage:fl_get_submatrix}
    \index{xformslib \textit{(package)}!xformslib.flflimage \textit{(module)}!xformslib.flflimage.fl\_get\_submatrix \textit{(function)}}

    \vspace{0.5ex}

\hspace{.8\funcindent}\begin{boxedminipage}{\funcwidth}

    \raggedright \textbf{fl\_get\_submatrix}(\textit{inmtrx}, \textit{nrows}, \textit{ncols}, \textit{r1}, \textit{c1}, \textit{rs}, \textit{cs}, \textit{elemsize})

    \vspace{-1.5ex}

    \rule{\textwidth}{0.5\fboxrule}
\setlength{\parskip}{2ex}

Grabs a piece of an image matrix. The piece is of nrows by ncols in
size.

-{}-
\setlength{\parskip}{1ex}
      \textbf{Parameters}
      \vspace{-1ex}

      \begin{quote}
        \begin{Ventry}{xxxxxxxx}

          \item[inmtrx]


image matrix
            {\it (type=\emph{todo})}

          \item[nrows]


number of rows
            {\it (type=int)}

          \item[ncols]


number of columns
            {\it (type=int)}

          \item[r1]


initial row?
            {\it (type=int)}

          \item[c1]


initial column?
            {\it (type=int)}

          \item[rs]


final row?
            {\it (type=int)}

          \item[cs]


final column?
            {\it (type=int)}

          \item[elemsize]


size of matrix in bytes
            {\it (type=int\_pos)}

        \end{Ventry}

      \end{quote}

      \textbf{Return Value}
    \vspace{-1ex}

      \begin{quote}

matrix?
      {\it (type=\emph{todo})}

      \end{quote}

\textbf{Note:} 
e.g. \emph{todo}


\textbf{Status:} 
Untested + NoDoc + NoDemo = NOT OK


    \end{boxedminipage}

    \label{xformslib:flflimage:fl_j2pass_quantize_packed}
    \index{xformslib \textit{(package)}!xformslib.flflimage \textit{(module)}!xformslib.flflimage.fl\_j2pass\_quantize\_packed \textit{(function)}}

    \vspace{0.5ex}

\hspace{.8\funcindent}\begin{boxedminipage}{\funcwidth}

    \raggedright \textbf{fl\_j2pass\_quantize\_packed}(\textit{packed}, \textit{w}, \textit{h}, \textit{maxcolr}, \textit{ci}, \textit{actualcolr}, \textit{redlut}, \textit{greenlut}, \textit{bluelut}, \textit{pImage})

    \vspace{-1.5ex}

    \rule{\textwidth}{0.5\fboxrule}
\setlength{\parskip}{2ex}

\emph{todo}

-{}-
\setlength{\parskip}{1ex}
      \textbf{Parameters}
      \vspace{-1ex}

      \begin{quote}
        \begin{Ventry}{xxxxxxxxxx}

          \item[packed]


\emph{todo}
            {\it (type=\emph{todo})}

          \item[w]


\emph{todo}
            {\it (type=int)}

          \item[h]


\emph{todo}
            {\it (type=int)}

          \item[maxcolr]


\emph{todo}
            {\it (type=int)}

          \item[ci]


\emph{todo}
            {\it (type=\emph{todo})}

          \item[actualcolr]


\emph{todo}
            {\it (type=int)}

          \item[redlut]


\emph{todo}
            {\it (type=int)}

          \item[greenlut]


\emph{todo}
            {\it (type=int)}

          \item[bluelut]


\emph{todo}
            {\it (type=int)}

          \item[pImage]


image
            {\it (type=pointer to xfdata.FL\_IMAGE)}

        \end{Ventry}

      \end{quote}

      \textbf{Return Value}
    \vspace{-1ex}

      \begin{quote}

num.
      {\it (type=int)}

      \end{quote}

\textbf{Note:} 
e.g. \emph{todo}


\textbf{Status:} 
Untested + NoDoc + NoDemo = NOT OK


    \end{boxedminipage}

    \label{xformslib:flflimage:fl_j2pass_quantize_rgb}
    \index{xformslib \textit{(package)}!xformslib.flflimage \textit{(module)}!xformslib.flflimage.fl\_j2pass\_quantize\_rgb \textit{(function)}}

    \vspace{0.5ex}

\hspace{.8\funcindent}\begin{boxedminipage}{\funcwidth}

    \raggedright \textbf{fl\_j2pass\_quantize\_rgb}(\textit{red}, \textit{green}, \textit{blue}, \textit{w}, \textit{h}, \textit{maxcolr}, \textit{ci}, \textit{actualcolr}, \textit{redlut}, \textit{greenlut}, \textit{bluelut}, \textit{pImage})

    \vspace{-1.5ex}

    \rule{\textwidth}{0.5\fboxrule}
\setlength{\parskip}{2ex}

\emph{todo}

-{}-
\setlength{\parskip}{1ex}
      \textbf{Parameters}
      \vspace{-1ex}

      \begin{quote}
        \begin{Ventry}{xxxxxxxxxx}

          \item[red]


\emph{todo}
            {\it (type=\emph{todo})}

          \item[green]


\emph{todo}
            {\it (type=\emph{todo})}

          \item[blue]


\emph{todo}
            {\it (type=\emph{todo})}

          \item[w]


\emph{todo}
            {\it (type=int)}

          \item[h]


\emph{todo}
            {\it (type=int)}

          \item[maxcolr]


\emph{todo}
            {\it (type=int)}

          \item[ci]


\emph{todo}
            {\it (type=\emph{todo})}

          \item[actualcolr]


\emph{todo}
            {\it (type=int)}

          \item[redlut]


\emph{todo}
            {\it (type=int)}

          \item[greenlut]


\emph{todo}
            {\it (type=int)}

          \item[bluelut]


\emph{todo}
            {\it (type=int)}

          \item[pImage]


image
            {\it (type=pointer to xfdata.FL\_IMAGE)}

        \end{Ventry}

      \end{quote}

      \textbf{Return Value}
    \vspace{-1ex}

      \begin{quote}

num.
      {\it (type=int)}

      \end{quote}

\textbf{Note:} 
e.g. \emph{todo}


\textbf{Status:} 
Untested + NoDoc + NoDemo = NOT OK


    \end{boxedminipage}

    \label{xformslib:flflimage:fl_make_submatrix}
    \index{xformslib \textit{(package)}!xformslib.flflimage \textit{(module)}!xformslib.flflimage.fl\_make\_submatrix \textit{(function)}}

    \vspace{0.5ex}

\hspace{.8\funcindent}\begin{boxedminipage}{\funcwidth}

    \raggedright \textbf{fl\_make\_submatrix}(\textit{inmtrx}, \textit{nrows}, \textit{ncols}, \textit{r1}, \textit{c1}, \textit{rs}, \textit{cs}, \textit{elemsize})

    \vspace{-1.5ex}

    \rule{\textwidth}{0.5\fboxrule}
\setlength{\parskip}{2ex}

\emph{todo}

-{}-
\setlength{\parskip}{1ex}
      \textbf{Parameters}
      \vspace{-1ex}

      \begin{quote}
        \begin{Ventry}{xxxxxxxx}

          \item[inmtrx]


image matrix
            {\it (type=\emph{todo})}

          \item[nrows]


number of rows
            {\it (type=int)}

          \item[ncols]


number of columns
            {\it (type=int)}

          \item[r1]


initial row?
            {\it (type=int)}

          \item[c1]


initial column?
            {\it (type=int)}

          \item[rs]


final row?
            {\it (type=int)}

          \item[cs]


final column?
            {\it (type=int)}

          \item[elemsize]


size of matrix in bytes
            {\it (type=int\_pos)}

        \end{Ventry}

      \end{quote}

      \textbf{Return Value}
    \vspace{-1ex}

      \begin{quote}

matrix?
      {\it (type=\emph{todo})}

      \end{quote}

\textbf{Note:} 
e.g. \emph{todo}


\textbf{Status:} 
Untested + NoDoc + NoDemo = NOT OK


    \end{boxedminipage}

    \label{xformslib:flflimage:fl_pack_bits}
    \index{xformslib \textit{(package)}!xformslib.flflimage \textit{(module)}!xformslib.flflimage.fl\_pack\_bits \textit{(function)}}

    \vspace{0.5ex}

\hspace{.8\funcindent}\begin{boxedminipage}{\funcwidth}

    \raggedright \textbf{fl\_pack\_bits}(\textit{inval}, \textit{lng})

    \vspace{-1.5ex}

    \rule{\textwidth}{0.5\fboxrule}
\setlength{\parskip}{2ex}

Packs color index (0 or 1) into bytes.

-{}-
\setlength{\parskip}{1ex}
      \textbf{Parameters}
      \vspace{-1ex}

      \begin{quote}
        \begin{Ventry}{xxxxx}

          \item[inval]


input value to be packed
            {\it (type=short\_pos)}

          \item[lng]


number of indexes
            {\it (type=int)}

        \end{Ventry}

      \end{quote}

      \textbf{Return Value}
    \vspace{-1ex}

      \begin{quote}

output value after packing (outval)
      {\it (type=byte\_pos)}

      \end{quote}

\textbf{Note:} 
e.g. \emph{todo}


\textbf{Attention:} 
API change from XForms - upstream was
fl\_pack\_bits(outval, inval, lng)


\textbf{Status:} 
Untested + Doc + NoDemo = NOT OK


    \end{boxedminipage}

    \label{xformslib:flflimage:fl_unpack_bits}
    \index{xformslib \textit{(package)}!xformslib.flflimage \textit{(module)}!xformslib.flflimage.fl\_unpack\_bits \textit{(function)}}

    \vspace{0.5ex}

\hspace{.8\funcindent}\begin{boxedminipage}{\funcwidth}

    \raggedright \textbf{fl\_unpack\_bits}(\textit{inval}, \textit{lng})

    \vspace{-1.5ex}

    \rule{\textwidth}{0.5\fboxrule}
\setlength{\parskip}{2ex}

Unpacks packed bits into color indexes (0 or 1).

-{}-
\setlength{\parskip}{1ex}
      \textbf{Parameters}
      \vspace{-1ex}

      \begin{quote}
        \begin{Ventry}{xxxxx}

          \item[inval]


input value to be unpacked
            {\it (type=byte\_pos)}

          \item[lng]


length of packed bytes.
            {\it (type=int)}

        \end{Ventry}

      \end{quote}

      \textbf{Return Value}
    \vspace{-1ex}

      \begin{quote}

output value after unpacking (outval)
      {\it (type=short\_pos)}

      \end{quote}

\textbf{Note:} 
e.g. \emph{todo}


\textbf{Attention:} 
API change from XForms - upstream was
fl\_unpack\_bits(outval, inval, lng)


\textbf{Status:} 
Untested + Doc + NoDemo = NOT OK


    \end{boxedminipage}

    \label{xformslib:flflimage:fl_value_to_bits}
    \index{xformslib \textit{(package)}!xformslib.flflimage \textit{(module)}!xformslib.flflimage.fl\_value\_to\_bits \textit{(function)}}

    \vspace{0.5ex}

\hspace{.8\funcindent}\begin{boxedminipage}{\funcwidth}

    \raggedright \textbf{fl\_value\_to\_bits}(\textit{val})

    \vspace{-1.5ex}

    \rule{\textwidth}{0.5\fboxrule}
\setlength{\parskip}{2ex}

\emph{todo}

-{}-
\setlength{\parskip}{1ex}
      \textbf{Parameters}
      \vspace{-1ex}

      \begin{quote}
        \begin{Ventry}{xxx}

          \item[val]


value to convert to bits
            {\it (type=int\_pos)}

        \end{Ventry}

      \end{quote}

      \textbf{Return Value}
    \vspace{-1ex}

      \begin{quote}

num.
      {\it (type=int\_pos)}

      \end{quote}

\textbf{Note:} 
e.g. \emph{todo}


\textbf{Status:} 
Untested + NoDoc + NoDemo = NOT OK


    \end{boxedminipage}

    \label{xformslib:flflimage:flimage_add_comments}
    \index{xformslib \textit{(package)}!xformslib.flflimage \textit{(module)}!xformslib.flflimage.flimage\_add\_comments \textit{(function)}}

    \vspace{0.5ex}

\hspace{.8\funcindent}\begin{boxedminipage}{\funcwidth}

    \raggedright \textbf{flimage\_add\_comments}(\textit{pImage}, \textit{text}, \textit{lng})

    \vspace{-1.5ex}

    \rule{\textwidth}{0.5\fboxrule}
\setlength{\parskip}{2ex}

Adds a comment to an image.

-{}-
\setlength{\parskip}{1ex}
      \textbf{Parameters}
      \vspace{-1ex}

      \begin{quote}
        \begin{Ventry}{xxxxxx}

          \item[pImage]


image
            {\it (type=pointer to xfdata.FL\_IMAGE)}

          \item[text]


comment to be added
            {\it (type=str)}

          \item[lng]


length of comment
            {\it (type=int)}

        \end{Ventry}

      \end{quote}

\textbf{Note:} 
e.g. \emph{todo}


\textbf{Status:} 
Untested + Doc + NoDemo = NOT OK


    \end{boxedminipage}

    \label{xformslib:flflimage:flimage_color_to_pixel}
    \index{xformslib \textit{(package)}!xformslib.flflimage \textit{(module)}!xformslib.flflimage.flimage\_color\_to\_pixel \textit{(function)}}

    \vspace{0.5ex}

\hspace{.8\funcindent}\begin{boxedminipage}{\funcwidth}

    \raggedright \textbf{flimage\_color\_to\_pixel}(\textit{pImage}, \textit{r}, \textit{g}, \textit{b})

    \vspace{-1.5ex}

    \rule{\textwidth}{0.5\fboxrule}
\setlength{\parskip}{2ex}

Convert an RGB triple to a pixel.

-{}-
\setlength{\parskip}{1ex}
      \textbf{Parameters}
      \vspace{-1ex}

      \begin{quote}
        \begin{Ventry}{xxxxxx}

          \item[pImage]


image
            {\it (type=pointer to xfdata.FL\_IMAGE)}

          \item[r]


value for red color
            {\it (type=int)}

          \item[g]


value for green color
            {\it (type=int)}

          \item[b]


value for blue color
            {\it (type=int)}

        \end{Ventry}

      \end{quote}

      \textbf{Return Value}
    \vspace{-1ex}

      \begin{quote}

pixel id, new pixel
      {\it (type=long\_pos, int)}

      \end{quote}

\textbf{Note:} 
e.g. \emph{todo}


\textbf{Attention:} 
API change from XForms - upstream was
flimage\_color\_to\_pixel(pImage, r, g, b, newpix)


\textbf{Status:} 
Untested + Doc + NoDemo = NOT OK


    \end{boxedminipage}

    \label{xformslib:flflimage:flimage_combine}
    \index{xformslib \textit{(package)}!xformslib.flflimage \textit{(module)}!xformslib.flflimage.flimage\_combine \textit{(function)}}

    \vspace{0.5ex}

\hspace{.8\funcindent}\begin{boxedminipage}{\funcwidth}

    \raggedright \textbf{flimage\_combine}(\textit{pImage1}, \textit{pImage2}, \textit{alpha})

    \vspace{-1.5ex}

    \rule{\textwidth}{0.5\fboxrule}
\setlength{\parskip}{2ex}

Combines two images with alpha level?, returning a new image.

-{}-
\setlength{\parskip}{1ex}
      \textbf{Parameters}
      \vspace{-1ex}

      \begin{quote}
        \begin{Ventry}{xxxxxxx}

          \item[pImage1]


first image to combine
            {\it (type=pointer to xfdata.FL\_IMAGE)}

          \item[pImage2]


second image to combine
            {\it (type=pointer to xfdata.FL\_IMAGE)}

          \item[alpha]


alpha level?
            {\it (type=float)}

        \end{Ventry}

      \end{quote}

      \textbf{Return Value}
    \vspace{-1ex}

      \begin{quote}

image class instance
      {\it (type=pointer to xfdata.FL\_IMAGE)}

      \end{quote}

\textbf{Note:} 
e.g. \emph{todo}


\textbf{Status:} 
Untested + NoDoc + NoDemo = NOT OK


    \end{boxedminipage}

    \label{xformslib:flflimage:flimage_display_markers}
    \index{xformslib \textit{(package)}!xformslib.flflimage \textit{(module)}!xformslib.flflimage.flimage\_display\_markers \textit{(function)}}

    \vspace{0.5ex}

\hspace{.8\funcindent}\begin{boxedminipage}{\funcwidth}

    \raggedright \textbf{flimage\_display\_markers}(\textit{pImage})

    \vspace{-1.5ex}

    \rule{\textwidth}{0.5\fboxrule}
\setlength{\parskip}{2ex}

Displays markers added to an image.

-{}-
\setlength{\parskip}{1ex}
      \textbf{Parameters}
      \vspace{-1ex}

      \begin{quote}
        \begin{Ventry}{xxxxxx}

          \item[pImage]


image
            {\it (type=pointer to xfdata.FL\_IMAGE)}

        \end{Ventry}

      \end{quote}

\textbf{Note:} 
e.g. flimage\_display\_markers(pimg)


\textbf{Status:} 
Untested + Doc + NoDemo = NOT OK


    \end{boxedminipage}

    \label{xformslib:flflimage:flimage_dup_}
    \index{xformslib \textit{(package)}!xformslib.flflimage \textit{(module)}!xformslib.flflimage.flimage\_dup\_ \textit{(function)}}

    \vspace{0.5ex}

\hspace{.8\funcindent}\begin{boxedminipage}{\funcwidth}

    \raggedright \textbf{flimage\_dup\_}(\textit{pImage}, \textit{pix})

    \vspace{-1.5ex}

    \rule{\textwidth}{0.5\fboxrule}
\setlength{\parskip}{2ex}

Duplicates an image, with or without the pixels

-{}-
\setlength{\parskip}{1ex}
      \textbf{Parameters}
      \vspace{-1ex}

      \begin{quote}
        \begin{Ventry}{xxxxxx}

          \item[pImage]


image
            {\it (type=pointer to xfdata.FL\_IMAGE)}

          \item[pix]


pixel
            {\it (type=int)}

        \end{Ventry}

      \end{quote}

      \textbf{Return Value}
    \vspace{-1ex}

      \begin{quote}

image class instance
      {\it (type=pointer to xfdata.FL\_IMAGE)}

      \end{quote}

\textbf{Note:} 
e.g. \emph{todo}


\textbf{Status:} 
Untested + NoDoc + NoDemo = NOT OK


    \end{boxedminipage}

    \label{xformslib:flflimage:flimage_enable_bmp}
    \index{xformslib \textit{(package)}!xformslib.flflimage \textit{(module)}!xformslib.flflimage.flimage\_enable\_bmp \textit{(function)}}

    \vspace{0.5ex}

\hspace{.8\funcindent}\begin{boxedminipage}{\funcwidth}

    \raggedright \textbf{flimage\_enable\_bmp}()

    \vspace{-1.5ex}

    \rule{\textwidth}{0.5\fboxrule}
\setlength{\parskip}{2ex}

Enables use of BMP (Windows/OS2 Bitmap) image format.

-{}-
\setlength{\parskip}{1ex}
\textbf{Note:} 
e.g. flimage\_enable\_bmp()


\textbf{Status:} 
Tested + Doc + NoDemo = OK


    \end{boxedminipage}

    \label{xformslib:flflimage:flimage_enable_fits}
    \index{xformslib \textit{(package)}!xformslib.flflimage \textit{(module)}!xformslib.flflimage.flimage\_enable\_fits \textit{(function)}}

    \vspace{0.5ex}

\hspace{.8\funcindent}\begin{boxedminipage}{\funcwidth}

    \raggedright \textbf{flimage\_enable\_fits}()

    \vspace{-1.5ex}

    \rule{\textwidth}{0.5\fboxrule}
\setlength{\parskip}{2ex}

Enables use of NASA/NOTS standard FITS image format.

-{}-
\setlength{\parskip}{1ex}
\textbf{Note:} 
e.g. flimage\_enable\_fits()


\textbf{Status:} 
Tested + Doc + NoDemo = OK


    \end{boxedminipage}

    \label{xformslib:flflimage:flimage_enable_genesis}
    \index{xformslib \textit{(package)}!xformslib.flflimage \textit{(module)}!xformslib.flflimage.flimage\_enable\_genesis \textit{(function)}}

    \vspace{0.5ex}

\hspace{.8\funcindent}\begin{boxedminipage}{\funcwidth}

    \raggedright \textbf{flimage\_enable\_genesis}()

    \vspace{-1.5ex}

    \rule{\textwidth}{0.5\fboxrule}
\setlength{\parskip}{2ex}

Enables use of Genesis image format.

-{}-
\setlength{\parskip}{1ex}
\textbf{Note:} 
e.g. flimage\_enable\_genesis()


\textbf{Status:} 
Tested + Doc + NoDemo = OK


    \end{boxedminipage}

    \label{xformslib:flflimage:flimage_enable_gif}
    \index{xformslib \textit{(package)}!xformslib.flflimage \textit{(module)}!xformslib.flflimage.flimage\_enable\_gif \textit{(function)}}

    \vspace{0.5ex}

\hspace{.8\funcindent}\begin{boxedminipage}{\funcwidth}

    \raggedright \textbf{flimage\_enable\_gif}()

    \vspace{-1.5ex}

    \rule{\textwidth}{0.5\fboxrule}
\setlength{\parskip}{2ex}

Enables use of GIF (Compuserve Graphics Interchange format) image
format.

-{}-
\setlength{\parskip}{1ex}
\textbf{Note:} 
e.g. flimage\_enable\_gif()


\textbf{Status:} 
Tested + Doc + NoDemo = OK


    \end{boxedminipage}

    \label{xformslib:flflimage:flimage_enable_gzip}
    \index{xformslib \textit{(package)}!xformslib.flflimage \textit{(module)}!xformslib.flflimage.flimage\_enable\_gzip \textit{(function)}}

    \vspace{0.5ex}

\hspace{.8\funcindent}\begin{boxedminipage}{\funcwidth}

    \raggedright \textbf{flimage\_enable\_gzip}()

    \vspace{-1.5ex}

    \rule{\textwidth}{0.5\fboxrule}
\setlength{\parskip}{2ex}

Enables use of gzip compression filter for images.

-{}-
\setlength{\parskip}{1ex}
\textbf{Note:} 
e.g. flimage\_enable\_gzip()


\textbf{Status:} 
Tested + Doc + NoDemo = OK


    \end{boxedminipage}

    \label{xformslib:flflimage:flimage_enable_jpeg}
    \index{xformslib \textit{(package)}!xformslib.flflimage \textit{(module)}!xformslib.flflimage.flimage\_enable\_jpeg \textit{(function)}}

    \vspace{0.5ex}

\hspace{.8\funcindent}\begin{boxedminipage}{\funcwidth}

    \raggedright \textbf{flimage\_enable\_jpeg}()

    \vspace{-1.5ex}

    \rule{\textwidth}{0.5\fboxrule}
\setlength{\parskip}{2ex}

Enables use of JPEG/JFIF (Joint Photographic Experts Group) image
format.

-{}-
\setlength{\parskip}{1ex}
\textbf{Note:} 
e.g. flimage\_enable\_jpeg()


\textbf{Status:} 
Tested + Doc + NoDemo = OK


    \end{boxedminipage}

    \label{xformslib:flflimage:flimage_enable_png}
    \index{xformslib \textit{(package)}!xformslib.flflimage \textit{(module)}!xformslib.flflimage.flimage\_enable\_png \textit{(function)}}

    \vspace{0.5ex}

\hspace{.8\funcindent}\begin{boxedminipage}{\funcwidth}

    \raggedright \textbf{flimage\_enable\_png}()

    \vspace{-1.5ex}

    \rule{\textwidth}{0.5\fboxrule}
\setlength{\parskip}{2ex}

Enables use of PNG (Portable Network Graphics) image format. It
requires netpbm library to be installed.

-{}-
\setlength{\parskip}{1ex}
\textbf{Note:} 
e.g. flimage\_enable\_png()


\textbf{Status:} 
Tested + Doc + NoDemo = OK


    \end{boxedminipage}

    \label{xformslib:flflimage:flimage_enable_ps}
    \index{xformslib \textit{(package)}!xformslib.flflimage \textit{(module)}!xformslib.flflimage.flimage\_enable\_ps \textit{(function)}}

    \vspace{0.5ex}

\hspace{.8\funcindent}\begin{boxedminipage}{\funcwidth}

    \raggedright \textbf{flimage\_enable\_ps}()

    \vspace{-1.5ex}

    \rule{\textwidth}{0.5\fboxrule}
\setlength{\parskip}{2ex}

Enables use of PS (Adobe PostScript) image format. It needs gs
(ghostscript) for reading.

-{}-
\setlength{\parskip}{1ex}
\textbf{Note:} 
e.g. flimage\_enable\_ps()


\textbf{Status:} 
Tested + Doc + NoDemo = OK


    \end{boxedminipage}

    \label{xformslib:flflimage:flimage_enable_sgi}
    \index{xformslib \textit{(package)}!xformslib.flflimage \textit{(module)}!xformslib.flflimage.flimage\_enable\_sgi \textit{(function)}}

    \vspace{0.5ex}

\hspace{.8\funcindent}\begin{boxedminipage}{\funcwidth}

    \raggedright \textbf{flimage\_enable\_sgi}()

    \vspace{-1.5ex}

    \rule{\textwidth}{0.5\fboxrule}
\setlength{\parskip}{2ex}

Enables use of SGI (Silicon Graphics-Iris) image format. It requires
pbmplus/netpbm library to be installed.

-{}-
\setlength{\parskip}{1ex}
\textbf{Note:} 
e.g. flimage\_enable\_sgi()


\textbf{Status:} 
Tested + Doc + NoDemo = OK


    \end{boxedminipage}

    \label{xformslib:flflimage:flimage_enable_tiff}
    \index{xformslib \textit{(package)}!xformslib.flflimage \textit{(module)}!xformslib.flflimage.flimage\_enable\_tiff \textit{(function)}}

    \vspace{0.5ex}

\hspace{.8\funcindent}\begin{boxedminipage}{\funcwidth}

    \raggedright \textbf{flimage\_enable\_tiff}()

    \vspace{-1.5ex}

    \rule{\textwidth}{0.5\fboxrule}
\setlength{\parskip}{2ex}

Enables use of TIFF (Tagged Image file, with no compression) image
format.

-{}-
\setlength{\parskip}{1ex}
\textbf{Note:} 
e.g. flimage\_enable\_tiff()


\textbf{Status:} 
Tested + Doc + NoDemo = OK


    \end{boxedminipage}

    \label{xformslib:flflimage:flimage_enable_xbm}
    \index{xformslib \textit{(package)}!xformslib.flflimage \textit{(module)}!xformslib.flflimage.flimage\_enable\_xbm \textit{(function)}}

    \vspace{0.5ex}

\hspace{.8\funcindent}\begin{boxedminipage}{\funcwidth}

    \raggedright \textbf{flimage\_enable\_xbm}()

    \vspace{-1.5ex}

    \rule{\textwidth}{0.5\fboxrule}
\setlength{\parskip}{2ex}

Enables use of XBM (X Window Bitmap) image format.

-{}-
\setlength{\parskip}{1ex}
\textbf{Note:} 
e.g. flimage\_enable\_xbm()


\textbf{Status:} 
Tested + Doc + NoDemo = OK


    \end{boxedminipage}

    \label{xformslib:flflimage:flimage_enable_xpm}
    \index{xformslib \textit{(package)}!xformslib.flflimage \textit{(module)}!xformslib.flflimage.flimage\_enable\_xpm \textit{(function)}}

    \vspace{0.5ex}

\hspace{.8\funcindent}\begin{boxedminipage}{\funcwidth}

    \raggedright \textbf{flimage\_enable\_xpm}()

    \vspace{-1.5ex}

    \rule{\textwidth}{0.5\fboxrule}
\setlength{\parskip}{2ex}

Enables use of XPM3 (X Window PixMap) image format.

-{}-
\setlength{\parskip}{1ex}
\textbf{Note:} 
e.g. flimage\_enable\_xpm()


\textbf{Status:} 
Tested + Doc + NoDemo = OK


    \end{boxedminipage}

    \label{xformslib:flflimage:flimage_enable_xwd}
    \index{xformslib \textit{(package)}!xformslib.flflimage \textit{(module)}!xformslib.flflimage.flimage\_enable\_xwd \textit{(function)}}

    \vspace{0.5ex}

\hspace{.8\funcindent}\begin{boxedminipage}{\funcwidth}

    \raggedright \textbf{flimage\_enable\_xwd}()

    \vspace{-1.5ex}

    \rule{\textwidth}{0.5\fboxrule}
\setlength{\parskip}{2ex}

Enables use of XWD (X Window Dump) image format.

-{}-
\setlength{\parskip}{1ex}
\textbf{Note:} 
e.g. flimage\_enable\_xwd()


\textbf{Status:} 
Tested + Doc + NoDemo = OK


    \end{boxedminipage}

    \label{xformslib:flflimage:flimage_free_ci}
    \index{xformslib \textit{(package)}!xformslib.flflimage \textit{(module)}!xformslib.flflimage.flimage\_free\_ci \textit{(function)}}

    \vspace{0.5ex}

\hspace{.8\funcindent}\begin{boxedminipage}{\funcwidth}

    \raggedright \textbf{flimage\_free\_ci}(\textit{pImage})

    \vspace{-1.5ex}

    \rule{\textwidth}{0.5\fboxrule}
\setlength{\parskip}{2ex}

Frees an image of type xfdata.FL\_IMAGE\_CI?

-{}-
\setlength{\parskip}{1ex}
      \textbf{Parameters}
      \vspace{-1ex}

      \begin{quote}
        \begin{Ventry}{xxxxxx}

          \item[pImage]


image
            {\it (type=pointer to xfdata.FL\_IMAGE)}

        \end{Ventry}

      \end{quote}

\textbf{Note:} 
e.g. \emph{todo}


\textbf{Status:} 
Untested + NoDoc + NoDemo = NOT OK


    \end{boxedminipage}

    \label{xformslib:flflimage:flimage_free_gray}
    \index{xformslib \textit{(package)}!xformslib.flflimage \textit{(module)}!xformslib.flflimage.flimage\_free\_gray \textit{(function)}}

    \vspace{0.5ex}

\hspace{.8\funcindent}\begin{boxedminipage}{\funcwidth}

    \raggedright \textbf{flimage\_free\_gray}(\textit{pImage})

    \vspace{-1.5ex}

    \rule{\textwidth}{0.5\fboxrule}
\setlength{\parskip}{2ex}

Frees an image of type xfdata.FL\_IMAGE\_GRAY?.

-{}-
\setlength{\parskip}{1ex}
      \textbf{Parameters}
      \vspace{-1ex}

      \begin{quote}
        \begin{Ventry}{xxxxxx}

          \item[pImage]


image
            {\it (type=pointer to xfdata.FL\_IMAGE)}

        \end{Ventry}

      \end{quote}

\textbf{Note:} 
e.g. \emph{todo}


\textbf{Status:} 
Untested + NoDoc + NoDemo = NOT OK


    \end{boxedminipage}

    \label{xformslib:flflimage:flimage_free_linearlut}
    \index{xformslib \textit{(package)}!xformslib.flflimage \textit{(module)}!xformslib.flflimage.flimage\_free\_linearlut \textit{(function)}}

    \vspace{0.5ex}

\hspace{.8\funcindent}\begin{boxedminipage}{\funcwidth}

    \raggedright \textbf{flimage\_free\_linearlut}(\textit{pImage})

    \vspace{-1.5ex}

    \rule{\textwidth}{0.5\fboxrule}
\setlength{\parskip}{2ex}

\emph{todo}

-{}-
\setlength{\parskip}{1ex}
      \textbf{Parameters}
      \vspace{-1ex}

      \begin{quote}
        \begin{Ventry}{xxxxxx}

          \item[pImage]


image
            {\it (type=pointer to xfdata.FL\_IMAGE)}

        \end{Ventry}

      \end{quote}

\textbf{Note:} 
e.g. \emph{todo}


\textbf{Status:} 
Untested + NoDoc + NoDemo = NOT OK


    \end{boxedminipage}

    \label{xformslib:flflimage:flimage_free_rgb}
    \index{xformslib \textit{(package)}!xformslib.flflimage \textit{(module)}!xformslib.flflimage.flimage\_free\_rgb \textit{(function)}}

    \vspace{0.5ex}

\hspace{.8\funcindent}\begin{boxedminipage}{\funcwidth}

    \raggedright \textbf{flimage\_free\_rgb}(\textit{pImage})

    \vspace{-1.5ex}

    \rule{\textwidth}{0.5\fboxrule}
\setlength{\parskip}{2ex}

Frees an image of type xfdata.FL\_IMAGE\_RGB?.

-{}-
\setlength{\parskip}{1ex}
      \textbf{Parameters}
      \vspace{-1ex}

      \begin{quote}
        \begin{Ventry}{xxxxxx}

          \item[pImage]


image
            {\it (type=pointer to xfdata.FL\_IMAGE)}

        \end{Ventry}

      \end{quote}

\textbf{Note:} 
e.g. \emph{todo}


\textbf{Status:} 
Untested + NoDoc + NoDemo = NOT OK


    \end{boxedminipage}

    \label{xformslib:flflimage:flimage_freemem}
    \index{xformslib \textit{(package)}!xformslib.flflimage \textit{(module)}!xformslib.flflimage.flimage\_freemem \textit{(function)}}

    \vspace{0.5ex}

\hspace{.8\funcindent}\begin{boxedminipage}{\funcwidth}

    \raggedright \textbf{flimage\_freemem}(\textit{pImage})

    \vspace{-1.5ex}

    \rule{\textwidth}{0.5\fboxrule}
\setlength{\parskip}{2ex}

Frees all allocated memory associated with the image.

-{}-
\setlength{\parskip}{1ex}
      \textbf{Parameters}
      \vspace{-1ex}

      \begin{quote}
        \begin{Ventry}{xxxxxx}

          \item[pImage]


image
            {\it (type=pointer to xfdata.FL\_IMAGE)}

        \end{Ventry}

      \end{quote}

\textbf{Note:} 
e.g. \emph{todo}


\textbf{Status:} 
Untested + Doc + NoDemo = NOT OK


    \end{boxedminipage}

    \label{xformslib:flflimage:flimage_get_closest_color_from_map}
    \index{xformslib \textit{(package)}!xformslib.flflimage \textit{(module)}!xformslib.flflimage.flimage\_get\_closest\_color\_from\_map \textit{(function)}}

    \vspace{0.5ex}

\hspace{.8\funcindent}\begin{boxedminipage}{\funcwidth}

    \raggedright \textbf{flimage\_get\_closest\_color\_from\_map}(\textit{pImage}, \textit{colr})

    \vspace{-1.5ex}

    \rule{\textwidth}{0.5\fboxrule}
\setlength{\parskip}{2ex}

Gets closest color from color map?

-{}-
\setlength{\parskip}{1ex}
      \textbf{Parameters}
      \vspace{-1ex}

      \begin{quote}
        \begin{Ventry}{xxxxxx}

          \item[pImage]


image
            {\it (type=pointer to xfdata.FL\_IMAGE)}

          \item[colr]


color to evaluate
            {\it (type=int\_pos)}

        \end{Ventry}

      \end{quote}

      \textbf{Return Value}
    \vspace{-1ex}

      \begin{quote}

color that is close?
      {\it (type=int)}

      \end{quote}

\textbf{Note:} 
e.g. \emph{todo}


\textbf{Status:} 
Untested + NoDoc + NoDemo = NOT OK


    \end{boxedminipage}

    \label{xformslib:flflimage:flimage_get_linearlut}
    \index{xformslib \textit{(package)}!xformslib.flflimage \textit{(module)}!xformslib.flflimage.flimage\_get\_linearlut \textit{(function)}}

    \vspace{0.5ex}

\hspace{.8\funcindent}\begin{boxedminipage}{\funcwidth}

    \raggedright \textbf{flimage\_get\_linearlut}(\textit{pImage})

    \vspace{-1.5ex}

    \rule{\textwidth}{0.5\fboxrule}
\setlength{\parskip}{2ex}

\emph{todo}

-{}-
\setlength{\parskip}{1ex}
      \textbf{Parameters}
      \vspace{-1ex}

      \begin{quote}
        \begin{Ventry}{xxxxxx}

          \item[pImage]


image
            {\it (type=pointer to xfdata.FL\_IMAGE)}

        \end{Ventry}

      \end{quote}

      \textbf{Return Value}
    \vspace{-1ex}

      \begin{quote}

num., or -1 (on failure?)
      {\it (type=int)}

      \end{quote}

\textbf{Note:} 
e.g. \emph{todo}


\textbf{Status:} 
Untested + NoDoc + NoDemo = NOT OK


    \end{boxedminipage}

    \label{xformslib:flflimage:flimage_invalidate_pixels}
    \index{xformslib \textit{(package)}!xformslib.flflimage \textit{(module)}!xformslib.flflimage.flimage\_invalidate\_pixels \textit{(function)}}

    \vspace{0.5ex}

\hspace{.8\funcindent}\begin{boxedminipage}{\funcwidth}

    \raggedright \textbf{flimage\_invalidate\_pixels}(\textit{pImage})

    \vspace{-1.5ex}

    \rule{\textwidth}{0.5\fboxrule}
\setlength{\parskip}{2ex}

Invalidates/frees all other types of image, before we modify the
current image.

-{}-
\setlength{\parskip}{1ex}
      \textbf{Parameters}
      \vspace{-1ex}

      \begin{quote}
        \begin{Ventry}{xxxxxx}

          \item[pImage]


image
            {\it (type=pointer to xfdata.FL\_IMAGE)}

        \end{Ventry}

      \end{quote}

\textbf{Note:} 
e.g. \emph{todo}


\textbf{Status:} 
Untested + NoDoc + NoDemo = NOT OK


    \end{boxedminipage}

    \label{xformslib:flflimage:flimage_open}
    \index{xformslib \textit{(package)}!xformslib.flflimage \textit{(module)}!xformslib.flflimage.flimage\_open \textit{(function)}}

    \vspace{0.5ex}

\hspace{.8\funcindent}\begin{boxedminipage}{\funcwidth}

    \raggedright \textbf{flimage\_open}(\textit{fname})

    \vspace{-1.5ex}

    \rule{\textwidth}{0.5\fboxrule}
\setlength{\parskip}{2ex}

Opens an image file.

-{}-
\setlength{\parskip}{1ex}
      \textbf{Parameters}
      \vspace{-1ex}

      \begin{quote}
        \begin{Ventry}{xxxxx}

          \item[fname]


name of file to open
            {\it (type=str)}

        \end{Ventry}

      \end{quote}

      \textbf{Return Value}
    \vspace{-1ex}

      \begin{quote}

image class instance opened, or None (on failure)
      {\it (type=pointer to xfdata.FL\_IMAGE)}

      \end{quote}

\textbf{Note:} 
e.g. pimg = flimage\_open(``something.ppm'')


\textbf{Status:} 
Untested + Doc + NoDemo = NOT OK


    \end{boxedminipage}

    \label{xformslib:flflimage:flimage_read_annotation}
    \index{xformslib \textit{(package)}!xformslib.flflimage \textit{(module)}!xformslib.flflimage.flimage\_read\_annotation \textit{(function)}}

    \vspace{0.5ex}

\hspace{.8\funcindent}\begin{boxedminipage}{\funcwidth}

    \raggedright \textbf{flimage\_read\_annotation}(\textit{pImage})

    \vspace{-1.5ex}

    \rule{\textwidth}{0.5\fboxrule}
\setlength{\parskip}{2ex}

Reads annotation in the image?

-{}-
\setlength{\parskip}{1ex}
      \textbf{Parameters}
      \vspace{-1ex}

      \begin{quote}
        \begin{Ventry}{xxxxxx}

          \item[pImage]


image
            {\it (type=pointer to xfdata.FL\_IMAGE)}

        \end{Ventry}

      \end{quote}

      \textbf{Return Value}
    \vspace{-1ex}

      \begin{quote}

0, or -1 (on failure)
      {\it (type=int)}

      \end{quote}

\textbf{Note:} 
e.g. \emph{todo}


\textbf{Status:} 
Untested + NoDoc + NoDemo = NOT OK


    \end{boxedminipage}

    \label{xformslib:flflimage:flimage_replace_image}
    \index{xformslib \textit{(package)}!xformslib.flflimage \textit{(module)}!xformslib.flflimage.flimage\_replace\_image \textit{(function)}}

    \vspace{0.5ex}

\hspace{.8\funcindent}\begin{boxedminipage}{\funcwidth}

    \raggedright \textbf{flimage\_replace\_image}(\textit{pImage}, \textit{w}, \textit{h}, \textit{r}, \textit{g}, \textit{b})

    \vspace{-1.5ex}

    \rule{\textwidth}{0.5\fboxrule}
\setlength{\parskip}{2ex}

Replace an image?

-{}-
\setlength{\parskip}{1ex}
      \textbf{Parameters}
      \vspace{-1ex}

      \begin{quote}
        \begin{Ventry}{xxxxxx}

          \item[pImage]


image
            {\it (type=pointer to xfdata.FL\_IMAGE)}

          \item[w]


width
            {\it (type=int)}

          \item[h]


heigth
            {\it (type=int)}

          \item[r]


value for red color
            {\it (type=int)}

          \item[g]


value for green color
            {\it (type=int)}

          \item[b]


value for blue color
            {\it (type=int)}

        \end{Ventry}

      \end{quote}

\textbf{Note:} 
e.g. \emph{todo}


\textbf{Status:} 
Untested + NoDoc + NoDemo = NOT OK


    \end{boxedminipage}

    \label{xformslib:flflimage:flimage_swapbuffer}
    \index{xformslib \textit{(package)}!xformslib.flflimage \textit{(module)}!xformslib.flflimage.flimage\_swapbuffer \textit{(function)}}

    \vspace{0.5ex}

\hspace{.8\funcindent}\begin{boxedminipage}{\funcwidth}

    \raggedright \textbf{flimage\_swapbuffer}(\textit{pImage})

    \vspace{-1.5ex}

    \rule{\textwidth}{0.5\fboxrule}
\setlength{\parskip}{2ex}

Swaps buffer of an image?

-{}-
\setlength{\parskip}{1ex}
      \textbf{Parameters}
      \vspace{-1ex}

      \begin{quote}
        \begin{Ventry}{xxxxxx}

          \item[pImage]


image
            {\it (type=pointer to xfdata.FL\_IMAGE)}

        \end{Ventry}

      \end{quote}

      \textbf{Return Value}
    \vspace{-1ex}

      \begin{quote}

0
      {\it (type=int)}

      \end{quote}

\textbf{Note:} 
e.g. \emph{todo}


\textbf{Status:} 
Untested + NoDoc + NoDemo = NOT OK


    \end{boxedminipage}

    \label{xformslib:flflimage:flimage_to_ximage}
    \index{xformslib \textit{(package)}!xformslib.flflimage \textit{(module)}!xformslib.flflimage.flimage\_to\_ximage \textit{(function)}}

    \vspace{0.5ex}

\hspace{.8\funcindent}\begin{boxedminipage}{\funcwidth}

    \raggedright \textbf{flimage\_to\_ximage}(\textit{pImage}, \textit{win}, \textit{pXWindowAttributes})

    \vspace{-1.5ex}

    \rule{\textwidth}{0.5\fboxrule}
\setlength{\parskip}{2ex}

Converts an FL\_IMAGE into an XImage.

-{}-
\setlength{\parskip}{1ex}
      \textbf{Parameters}
      \vspace{-1ex}

      \begin{quote}
        \begin{Ventry}{xxxxxxxxxxxxxxxxxx}

          \item[pImage]


image
            {\it (type=pointer to xfdata.FL\_IMAGE)}

          \item[win]


long\_pos
            {\it (type=window id)}

          \item[pXWindowAttributes]


class instance
            {\it (type=pointer to xfdata.XWindowAttributes)}

        \end{Ventry}

      \end{quote}

      \textbf{Return Value}
    \vspace{-1ex}

      \begin{quote}

num.
      {\it (type=int)}

      \end{quote}

\textbf{Note:} 
e.g. \emph{todo}


\textbf{Status:} 
Untested + Doc + NoDemo = NOT OK


    \end{boxedminipage}

    \label{xformslib:flflimage:flimage_write_annotation}
    \index{xformslib \textit{(package)}!xformslib.flflimage \textit{(module)}!xformslib.flflimage.flimage\_write\_annotation \textit{(function)}}

    \vspace{0.5ex}

\hspace{.8\funcindent}\begin{boxedminipage}{\funcwidth}

    \raggedright \textbf{flimage\_write\_annotation}(\textit{pImage})

    \vspace{-1.5ex}

    \rule{\textwidth}{0.5\fboxrule}
\setlength{\parskip}{2ex}

Writes annotation in the image?

-{}-
\setlength{\parskip}{1ex}
      \textbf{Parameters}
      \vspace{-1ex}

      \begin{quote}
        \begin{Ventry}{xxxxxx}

          \item[pImage]


image
            {\it (type=pointer to xfdata.FL\_IMAGE)}

        \end{Ventry}

      \end{quote}

      \textbf{Return Value}
    \vspace{-1ex}

      \begin{quote}

0, or -1 (on failure)
      {\it (type=int)}

      \end{quote}

\textbf{Note:} 
e.g. \emph{todo}


\textbf{Status:} 
Untested + NoDoc + NoDemo = NOT OK


    \end{boxedminipage}


%%%%%%%%%%%%%%%%%%%%%%%%%%%%%%%%%%%%%%%%%%%%%%%%%%%%%%%%%%%%%%%%%%%%%%%%%%%
%%                               Variables                               %%
%%%%%%%%%%%%%%%%%%%%%%%%%%%%%%%%%%%%%%%%%%%%%%%%%%%%%%%%%%%%%%%%%%%%%%%%%%%

  \subsection{Variables}

    \vspace{-1cm}
\hspace{\varindent}\begin{longtable}{|p{\varnamewidth}|p{\vardescrwidth}|l}
\cline{1-2}
\cline{1-2} \centering \textbf{Name} & \centering \textbf{Description}& \\
\cline{1-2}
\endhead\cline{1-2}\multicolumn{3}{r}{\small\textit{continued on next page}}\\\endfoot\cline{1-2}
\endlastfoot\raggedright \_\-\_\-p\-a\-c\-k\-a\-g\-e\-\_\-\_\- & \raggedright \textbf{Value:} 
{\tt \texttt{'}\texttt{xformslib}\texttt{'}}&\\
\cline{1-2}
\end{longtable}

    \index{xformslib \textit{(package)}!xformslib.flflimage \textit{(module)}|)}
