%
% API Documentation for API Documentation
% Module xformslib.flinput
%
% Generated by epydoc 3.0.1
% [Fri May 21 15:38:49 2010]
%

%%%%%%%%%%%%%%%%%%%%%%%%%%%%%%%%%%%%%%%%%%%%%%%%%%%%%%%%%%%%%%%%%%%%%%%%%%%
%%                          Module Description                           %%
%%%%%%%%%%%%%%%%%%%%%%%%%%%%%%%%%%%%%%%%%%%%%%%%%%%%%%%%%%%%%%%%%%%%%%%%%%%

    \index{xformslib \textit{(package)}!xformslib.flinput \textit{(module)}|(}
\section{Module xformslib.flinput}

    \label{xformslib:flinput}

xforms-python's functions to manage input objects.

Copyright (C) 2009, 2010  Luca Lazzaroni ``LukenShiro''
e-mail: <\href{mailto:lukenshiro@ngi.it}{lukenshiro@ngi.it}>

This program is free software: you can redistribute it and/or modify
it under the terms of the GNU Lesser General Public License as
published by the Free Software Foundation, version 2.1 of the License.

This program is distributed in the hope that it will be useful,
but WITHOUT ANY WARRANTY; without even the implied warranty of
MERCHANTABILITY or FITNESS FOR A PARTICULAR PURPOSE. See the
GNU Lesser General Public License for more details.

You should have received a copy of the GNU LGPL along with this
program. If not, see <\href{http://www.gnu.org/licenses/}{http://www.gnu.org/licenses/}>.

See CREDITS file to read acknowledgements and thanks to XForms,
ctypes and other developers.

%%%%%%%%%%%%%%%%%%%%%%%%%%%%%%%%%%%%%%%%%%%%%%%%%%%%%%%%%%%%%%%%%%%%%%%%%%%
%%                               Functions                               %%
%%%%%%%%%%%%%%%%%%%%%%%%%%%%%%%%%%%%%%%%%%%%%%%%%%%%%%%%%%%%%%%%%%%%%%%%%%%

  \subsection{Functions}

    \label{xformslib:flinput:fl_add_input}
    \index{xformslib \textit{(package)}!xformslib.flinput \textit{(module)}!xformslib.flinput.fl\_add\_input \textit{(function)}}

    \vspace{0.5ex}

\hspace{.8\funcindent}\begin{boxedminipage}{\funcwidth}

    \raggedright \textbf{fl\_add\_input}(\textit{inputtype}, \textit{x}, \textit{y}, \textit{w}, \textit{h}, \textit{label})

    \vspace{-1.5ex}

    \rule{\textwidth}{0.5\fboxrule}
\setlength{\parskip}{2ex}

Adds an input object.

-{}-
\setlength{\parskip}{1ex}
      \textbf{Parameters}
      \vspace{-1ex}

      \begin{quote}
        \begin{Ventry}{xxxxxxxxx}

          \item[inputtype]


type of input to be added. Values (from xfdata.py) FL\_NORMAL\_INPUT,
FL\_FLOAT\_INPUT, FL\_INT\_INPUT, FL\_DATE\_INPUT, FL\_MULTILINE\_INPUT,
FL\_HIDDEN\_INPUT, FL\_SECRET\_INPUT
            {\it (type=int)}

          \item[x]


horizontal position (upper-left corner)
            {\it (type=int)}

          \item[y]


vertical position (upper-left corner)
            {\it (type=int)}

          \item[w]


width in coord units
            {\it (type=int)}

          \item[h]


height in coord units
            {\it (type=int)}

          \item[label]


text label of input
            {\it (type=str)}

        \end{Ventry}

      \end{quote}

      \textbf{Return Value}
    \vspace{-1ex}

      \begin{quote}

object created (pFlObject)
      {\it (type=pointer to xfdata.FL\_OBJECT)}

      \end{quote}

\textbf{Note:} 
e.g. \emph{todo}


\textbf{Status:} 
Tested + Doc + Demo = OK


    \end{boxedminipage}

    \label{xformslib:flinput:fl_set_input}
    \index{xformslib \textit{(package)}!xformslib.flinput \textit{(module)}!xformslib.flinput.fl\_set\_input \textit{(function)}}

    \vspace{0.5ex}

\hspace{.8\funcindent}\begin{boxedminipage}{\funcwidth}

    \raggedright \textbf{fl\_set\_input}(\textit{pFlObject}, \textit{text})

    \vspace{-1.5ex}

    \rule{\textwidth}{0.5\fboxrule}
\setlength{\parskip}{2ex}

Sets the particular input string, with no checks for validity. An
empty string can be used to clear an input field.

-{}-
\setlength{\parskip}{1ex}
      \textbf{Parameters}
      \vspace{-1ex}

      \begin{quote}
        \begin{Ventry}{xxxxxxxxx}

          \item[pFlObject]


input object
            {\it (type=pointer to xfdata.FL\_OBJECT)}

          \item[text]


input text
            {\it (type=str)}

        \end{Ventry}

      \end{quote}

\textbf{Note:} 
e.g. \emph{todo}


\textbf{Status:} 
Tested + Doc + Demo = OK


    \end{boxedminipage}

    \label{xformslib:flinput:fl_set_input_color}
    \index{xformslib \textit{(package)}!xformslib.flinput \textit{(module)}!xformslib.flinput.fl\_set\_input\_color \textit{(function)}}

    \vspace{0.5ex}

\hspace{.8\funcindent}\begin{boxedminipage}{\funcwidth}

    \raggedright \textbf{fl\_set\_input\_color}(\textit{pFlObject}, \textit{txtcolr}, \textit{curscolr})

    \vspace{-1.5ex}

    \rule{\textwidth}{0.5\fboxrule}
\setlength{\parskip}{2ex}

Sets text and cursor colors to be used in input object.

-{}-
\setlength{\parskip}{1ex}
      \textbf{Parameters}
      \vspace{-1ex}

      \begin{quote}
        \begin{Ventry}{xxxxxxxxx}

          \item[pFlObject]


input object
            {\it (type=pointer to xfdata.FL\_OBJECT)}

          \item[txtcolr]


color value for text
            {\it (type=long\_pos)}

          \item[curscolr]


color value for cursor
            {\it (type=long\_pos)}

        \end{Ventry}

      \end{quote}

\textbf{Note:} 
e.g. \emph{todo}


\textbf{Status:} 
Untested + NoDoc + NoDemo = NOT OK


    \end{boxedminipage}

    \label{xformslib:flinput:fl_get_input_color}
    \index{xformslib \textit{(package)}!xformslib.flinput \textit{(module)}!xformslib.flinput.fl\_get\_input\_color \textit{(function)}}

    \vspace{0.5ex}

\hspace{.8\funcindent}\begin{boxedminipage}{\funcwidth}

    \raggedright \textbf{fl\_get\_input\_color}(\textit{pFlObject})

    \vspace{-1.5ex}

    \rule{\textwidth}{0.5\fboxrule}
\setlength{\parskip}{2ex}

Obtains current color for text and cursor of an input object.

-{}-
\setlength{\parskip}{1ex}
      \textbf{Parameters}
      \vspace{-1ex}

      \begin{quote}
        \begin{Ventry}{xxxxxxxxx}

          \item[pFlObject]


input object
            {\it (type=pointer to xfdata.FL\_OBJECT)}

        \end{Ventry}

      \end{quote}

      \textbf{Return Value}
    \vspace{-1ex}

      \begin{quote}

color value for text, color value for cursor
      {\it (type=long\_pos, long\_pos)}

      \end{quote}

\textbf{Note:} 
e.g. \emph{todo}


\textbf{Attention:} 
API change from XForms - upstream was
fl\_get\_input\_color(pFlObject, textcolr, curscolr)


\textbf{Status:} 
Untested + NoDoc + NoDemo = NOT OK


    \end{boxedminipage}

    \label{xformslib:flinput:fl_set_input_scroll}
    \index{xformslib \textit{(package)}!xformslib.flinput \textit{(module)}!xformslib.flinput.fl\_set\_input\_scroll \textit{(function)}}

    \vspace{0.5ex}

\hspace{.8\funcindent}\begin{boxedminipage}{\funcwidth}

    \raggedright \textbf{fl\_set\_input\_scroll}(\textit{pFlObject}, \textit{yesno})

    \vspace{-1.5ex}

    \rule{\textwidth}{0.5\fboxrule}
\setlength{\parskip}{2ex}

Turn on/off scrolling for an input field (for both multiline and single
line input field).

-{}-
\setlength{\parskip}{1ex}
      \textbf{Parameters}
      \vspace{-1ex}

      \begin{quote}
        \begin{Ventry}{xxxxxxxxx}

          \item[pFlObject]


input object
            {\it (type=pointer to xfdata.FL\_OBJECT)}

          \item[yesno]


flag to enable/disable scrolling. Values 0 (disabled) or 1 (enabled)
            {\it (type=int)}

        \end{Ventry}

      \end{quote}

\textbf{Note:} 
e.g. \emph{todo}


\textbf{Status:} 
Untested + NoDoc + NoDemo = NOT OK


    \end{boxedminipage}

    \label{xformslib:flinput:fl_set_input_cursorpos}
    \index{xformslib \textit{(package)}!xformslib.flinput \textit{(module)}!xformslib.flinput.fl\_set\_input\_cursorpos \textit{(function)}}

    \vspace{0.5ex}

\hspace{.8\funcindent}\begin{boxedminipage}{\funcwidth}

    \raggedright \textbf{fl\_set\_input\_cursorpos}(\textit{pFlObject}, \textit{xpos}, \textit{ypos})

    \vspace{-1.5ex}

    \rule{\textwidth}{0.5\fboxrule}
\setlength{\parskip}{2ex}

Moves the cursor within the input field.

-{}-
\setlength{\parskip}{1ex}
      \textbf{Parameters}
      \vspace{-1ex}

      \begin{quote}
        \begin{Ventry}{xxxxxxxxx}

          \item[pFlObject]


input object
            {\it (type=pointer to xfdata.FL\_OBJECT)}

          \item[xpos]


horizontal cursor position in characters
            {\it (type=int)}

          \item[ypos]


vertical cursor position in lines
            {\it (type=int)}

        \end{Ventry}

      \end{quote}

\textbf{Note:} 
e.g. \emph{todo}


\textbf{Status:} 
Untested + Doc + NoDemo = NOT OK


    \end{boxedminipage}

    \label{xformslib:flinput:fl_set_input_selected}
    \index{xformslib \textit{(package)}!xformslib.flinput \textit{(module)}!xformslib.flinput.fl\_set\_input\_selected \textit{(function)}}

    \vspace{0.5ex}

\hspace{.8\funcindent}\begin{boxedminipage}{\funcwidth}

    \raggedright \textbf{fl\_set\_input\_selected}(\textit{pFlObject}, \textit{yesno})

    \vspace{-1.5ex}

    \rule{\textwidth}{0.5\fboxrule}
\setlength{\parskip}{2ex}

Selects or deselects the current input. It places the cursor at
the end of the string when selected.

-{}-
\setlength{\parskip}{1ex}
      \textbf{Parameters}
      \vspace{-1ex}

      \begin{quote}
        \begin{Ventry}{xxxxxxxxx}

          \item[pFlObject]


input object
            {\it (type=pointer to xfdata.FL\_OBJECT)}

          \item[yesno]


flag to deselect/select input. Values 0 (deselected) or 1 (selected)
            {\it (type=int)}

        \end{Ventry}

      \end{quote}

\textbf{Note:} 
e.g. \emph{todo}


\textbf{Status:} 
Untested + NoDoc + NoDemo = NOT OK


    \end{boxedminipage}

    \label{xformslib:flinput:fl_set_input_selected_range}
    \index{xformslib \textit{(package)}!xformslib.flinput \textit{(module)}!xformslib.flinput.fl\_set\_input\_selected\_range \textit{(function)}}

    \vspace{0.5ex}

\hspace{.8\funcindent}\begin{boxedminipage}{\funcwidth}

    \raggedright \textbf{fl\_set\_input\_selected\_range}(\textit{pFlObject}, \textit{begin}, \textit{end})

    \vspace{-1.5ex}

    \rule{\textwidth}{0.5\fboxrule}
\setlength{\parskip}{2ex}

Selects or deselects the current input of part of it. When begin is
0 and end is the last character number, all input is selected. It places
the cursor at the beginning of selected string.

-{}-
\setlength{\parskip}{1ex}
      \textbf{Parameters}
      \vspace{-1ex}

      \begin{quote}
        \begin{Ventry}{xxxxxxxxx}

          \item[pFlObject]


input object
            {\it (type=pointer to xfdata.FL\_OBJECT)}

          \item[begin]


starting value in characters
            {\it (type=int)}

          \item[end]


ending value in characters
            {\it (type=int)}

        \end{Ventry}

      \end{quote}

\textbf{Note:} 
e.g. \emph{todo}


\textbf{Status:} 
Untested + Doc + NoDemo = NOT OK


    \end{boxedminipage}

    \label{xformslib:flinput:fl_get_input_selected_range}
    \index{xformslib \textit{(package)}!xformslib.flinput \textit{(module)}!xformslib.flinput.fl\_get\_input\_selected\_range \textit{(function)}}

    \vspace{0.5ex}

\hspace{.8\funcindent}\begin{boxedminipage}{\funcwidth}

    \raggedright \textbf{fl\_get\_input\_selected\_range}(\textit{pFlObject})

    \vspace{-1.5ex}

    \rule{\textwidth}{0.5\fboxrule}
\setlength{\parskip}{2ex}

Obtains the currently selected range, either selected by the
application or by the user.

-{}-
\setlength{\parskip}{1ex}
      \textbf{Parameters}
      \vspace{-1ex}

      \begin{quote}
        \begin{Ventry}{xxxxxxxxx}

          \item[pFlObject]


input object
            {\it (type=pointer to xfdata.FL\_OBJECT)}

        \end{Ventry}

      \end{quote}

      \textbf{Return Value}
    \vspace{-1ex}

      \begin{quote}

selected string, starting value, ending value of selection
in characters
      {\it (type=str, int, int)}

      \end{quote}

\textbf{Note:} 
e.g. \emph{todo}


\textbf{Attention:} 
API change from XForms - upstream was
fl\_get\_input\_selected\_range(pFlObject, begin, end)


\textbf{Status:} 
Untested + Doc + NoDemo = NOT OK


    \end{boxedminipage}

    \label{xformslib:flinput:fl_set_input_maxchars}
    \index{xformslib \textit{(package)}!xformslib.flinput \textit{(module)}!xformslib.flinput.fl\_set\_input\_maxchars \textit{(function)}}

    \vspace{0.5ex}

\hspace{.8\funcindent}\begin{boxedminipage}{\funcwidth}

    \raggedright \textbf{fl\_set\_input\_maxchars}(\textit{pFlObject}, \textit{maxchars})

    \vspace{-1.5ex}

    \rule{\textwidth}{0.5\fboxrule}
\setlength{\parskip}{2ex}

Limits the number of characters per line for input fields of type
xfdata.FL\_NORMAL\_INPUT. Note that input objects of type
xfdata.FL\_DATE\_INPUT are limited to 10 characters per default and those
of type xfdata.FL\_SECRET\_INPUT to 16.

-{}-
\setlength{\parskip}{1ex}
      \textbf{Parameters}
      \vspace{-1ex}

      \begin{quote}
        \begin{Ventry}{xxxxxxxxx}

          \item[pFlObject]


input object
            {\it (type=pointer to xfdata.FL\_OBJECT)}

          \item[maxchars]


maximum characters to be set. If it's 0, limit is reset to infinite.
            {\it (type=int)}

        \end{Ventry}

      \end{quote}

\textbf{Note:} 
e.g. \emph{todo}


\textbf{Status:} 
Untested + Doc + NoDemo = NOT OK


    \end{boxedminipage}

    \label{xformslib:flinput:fl_set_input_format}
    \index{xformslib \textit{(package)}!xformslib.flinput \textit{(module)}!xformslib.flinput.fl\_set\_input\_format \textit{(function)}}

    \vspace{0.5ex}

\hspace{.8\funcindent}\begin{boxedminipage}{\funcwidth}

    \raggedright \textbf{fl\_set\_input\_format}(\textit{pFlObject}, \textit{fmt}, \textit{sep})

    \vspace{-1.5ex}

    \rule{\textwidth}{0.5\fboxrule}
\setlength{\parskip}{2ex}

Sets the format used for an input object. Currently used only for
xfdata.FL\_DATE\_INPUT objects.

-{}-
\setlength{\parskip}{1ex}
      \textbf{Parameters}
      \vspace{-1ex}

      \begin{quote}
        \begin{Ventry}{xxx}

          \item[fmt]


format for the input. Values (from xfdata.py) FL\_INPUT\_DDMM,
FL\_INPUT\_MMDD
            {\it (type=int)}

          \item[sep]


printable single character used as separator
            {\it (type=int or char)}

        \end{Ventry}

      \end{quote}

\textbf{Note:} 
e.g. \emph{todo}


\textbf{Status:} 
Untested + Doc + NoDemo = NOT OK


    \end{boxedminipage}

    \label{xformslib:flinput:fl_set_input_hscrollbar}
    \index{xformslib \textit{(package)}!xformslib.flinput \textit{(module)}!xformslib.flinput.fl\_set\_input\_hscrollbar \textit{(function)}}

    \vspace{0.5ex}

\hspace{.8\funcindent}\begin{boxedminipage}{\funcwidth}

    \raggedright \textbf{fl\_set\_input\_hscrollbar}(\textit{pFlObject}, \textit{pref})

    \vspace{-1.5ex}

    \rule{\textwidth}{0.5\fboxrule}
\setlength{\parskip}{2ex}

Sets horizontal scrollbar settings. By default, if an input field of
type xfdata.FL\_MULTILINE\_INPUT contains more text than can be shown,
scrollbars will appear with which the user can scroll the text around
horizontally. Turning off scrollbars for an input field does not turn off
scrolling, the user can still scroll the field using cursor and other keys.

-{}-
\setlength{\parskip}{1ex}
      \textbf{Parameters}
      \vspace{-1ex}

      \begin{quote}
        \begin{Ventry}{xxxxxxxxx}

          \item[pFlObject]


input object
            {\it (type=pointer to xfdata.FL\_OBJECT)}

          \item[pref]


how is horizontal scrollbar shown. Values (from xfdata.py) FL\_AUTO,
FL\_ON, FL\_OFF
            {\it (type=int)}

        \end{Ventry}

      \end{quote}

\textbf{Note:} 
e.g. \emph{todo}


\textbf{Status:} 
Untested + NoDoc + NoDemo = NOT OK


    \end{boxedminipage}

    \label{xformslib:flinput:fl_set_input_vscrollbar}
    \index{xformslib \textit{(package)}!xformslib.flinput \textit{(module)}!xformslib.flinput.fl\_set\_input\_vscrollbar \textit{(function)}}

    \vspace{0.5ex}

\hspace{.8\funcindent}\begin{boxedminipage}{\funcwidth}

    \raggedright \textbf{fl\_set\_input\_vscrollbar}(\textit{pFlObject}, \textit{pref})

    \vspace{-1.5ex}

    \rule{\textwidth}{0.5\fboxrule}
\setlength{\parskip}{2ex}

Sets vertical scrollbar settings. By default, if an input field of type
xfdata.FL\_MULTILINE\_INPUT contains more text than can be shown, scrollbars
will appear with which the user can scroll the text around vertically.
Turning off scrollbars for an input field does not turn off scrolling, the
user can still scroll the field using cursor and other keys.

-{}-
\setlength{\parskip}{1ex}
      \textbf{Parameters}
      \vspace{-1ex}

      \begin{quote}
        \begin{Ventry}{xxxxxxxxx}

          \item[pFlObject]


input object
            {\it (type=pointer to xfdata.FL\_OBJECT)}

          \item[pref]


how is vertical scrollbar shown. Values (from xfdata.py) FL\_AUTO,
FL\_ON, FL\_OFF
            {\it (type=int)}

        \end{Ventry}

      \end{quote}

\textbf{Note:} 
e.g. \emph{todo}


\textbf{Status:} 
Untested + Doc + NoDemo = NOT OK


    \end{boxedminipage}

    \label{xformslib:flinput:fl_set_input_topline}
    \index{xformslib \textit{(package)}!xformslib.flinput \textit{(module)}!xformslib.flinput.fl\_set\_input\_topline \textit{(function)}}

    \vspace{0.5ex}

\hspace{.8\funcindent}\begin{boxedminipage}{\funcwidth}

    \raggedright \textbf{fl\_set\_input\_topline}(\textit{pFlObject}, \textit{linenum})

    \vspace{-1.5ex}

    \rule{\textwidth}{0.5\fboxrule}
\setlength{\parskip}{2ex}

Scrolls vertically an input object (for input fields of type
xfdata.FL\_MULTILINE\_INPUT only).

-{}-
\setlength{\parskip}{1ex}
      \textbf{Parameters}
      \vspace{-1ex}

      \begin{quote}
        \begin{Ventry}{xxxxxxxxx}

          \item[pFlObject]


input object
            {\it (type=pointer to xfdata.FL\_OBJECT)}

          \item[linenum]


the new top line (starting from 1) in the input field.
            {\it (type=int)}

        \end{Ventry}

      \end{quote}

\textbf{Note:} 
e.g. \emph{todo}


\textbf{Status:} 
Untested + NoDoc + NoDemo = NOT OK


    \end{boxedminipage}

    \label{xformslib:flinput:fl_set_input_scrollbarsize}
    \index{xformslib \textit{(package)}!xformslib.flinput \textit{(module)}!xformslib.flinput.fl\_set\_input\_scrollbarsize \textit{(function)}}

    \vspace{0.5ex}

\hspace{.8\funcindent}\begin{boxedminipage}{\funcwidth}

    \raggedright \textbf{fl\_set\_input\_scrollbarsize}(\textit{pFlObject}, \textit{hh}, \textit{vw})

    \vspace{-1.5ex}

    \rule{\textwidth}{0.5\fboxrule}
\setlength{\parskip}{2ex}

Changes the size of the scrollbars. By default, the scrollbar size is
dependent on the initial size of the input box.

-{}-
\setlength{\parskip}{1ex}
      \textbf{Parameters}
      \vspace{-1ex}

      \begin{quote}
        \begin{Ventry}{xxxxxxxxx}

          \item[pFlObject]


input object
            {\it (type=pointer to xfdata.FL\_OBJECT)}

          \item[hh]


horizontal height of scrollbar in pixels
            {\it (type=int)}

          \item[vw]


vertical width of scrollbar in pixels
            {\it (type=int)}

        \end{Ventry}

      \end{quote}

\textbf{Note:} 
e.g. \emph{todo}


\textbf{Status:} 
Untested + Doc + NoDemo = NOT OK


    \end{boxedminipage}

    \label{xformslib:flinput:fl_get_input_scrollbarsize}
    \index{xformslib \textit{(package)}!xformslib.flinput \textit{(module)}!xformslib.flinput.fl\_get\_input\_scrollbarsize \textit{(function)}}

    \vspace{0.5ex}

\hspace{.8\funcindent}\begin{boxedminipage}{\funcwidth}

    \raggedright \textbf{fl\_get\_input\_scrollbarsize}(\textit{pFlObject})

    \vspace{-1.5ex}

    \rule{\textwidth}{0.5\fboxrule}
\setlength{\parskip}{2ex}

Obtains the current settings for the horizontal scrollbar height and
the vertical scrollbar width.

-{}-
\setlength{\parskip}{1ex}
      \textbf{Parameters}
      \vspace{-1ex}

      \begin{quote}
        \begin{Ventry}{xxxxxxxxx}

          \item[pFlObject]


input object
            {\it (type=pointer to xfdata.FL\_OBJECT)}

        \end{Ventry}

      \end{quote}

      \textbf{Return Value}
    \vspace{-1ex}

      \begin{quote}

horizontal height (hh), vertical width (vw)
      {\it (type=int, int)}

      \end{quote}

\textbf{Attention:} 
API change from XForms - upstream was
fl\_get\_input\_scrollbarsize(pFlObject, hh, vw)


\textbf{Note:} 
e.g. \emph{todo}


\textbf{Status:} 
Untested + Doc + NoDemo = NOT OK


    \end{boxedminipage}

    \label{xformslib:flinput:fl_set_input_xoffset}
    \index{xformslib \textit{(package)}!xformslib.flinput \textit{(module)}!xformslib.flinput.fl\_set\_input\_xoffset \textit{(function)}}

    \vspace{0.5ex}

\hspace{.8\funcindent}\begin{boxedminipage}{\funcwidth}

    \raggedright \textbf{fl\_set\_input\_xoffset}(\textit{pFlObject}, \textit{offset})

    \vspace{-1.5ex}

    \rule{\textwidth}{0.5\fboxrule}
\setlength{\parskip}{2ex}

Scroll programmatically horizontally (to the left).

-{}-
\setlength{\parskip}{1ex}
      \textbf{Parameters}
      \vspace{-1ex}

      \begin{quote}
        \begin{Ventry}{xxxxxxxxx}

          \item[pFlObject]


input object
            {\it (type=pointer to xfdata.FL\_OBJECT)}

          \item[offset]


positive number of pixels to scroll to the left relative to the
nominal position of the text lines.
            {\it (type=int)}

        \end{Ventry}

      \end{quote}

\textbf{Note:} 
e.g. \emph{todo}


\textbf{Status:} 
Untested + NoDoc + NoDemo = NOT OK


    \end{boxedminipage}

    \label{xformslib:flinput:fl_get_input_xoffset}
    \index{xformslib \textit{(package)}!xformslib.flinput \textit{(module)}!xformslib.flinput.fl\_get\_input\_xoffset \textit{(function)}}

    \vspace{0.5ex}

\hspace{.8\funcindent}\begin{boxedminipage}{\funcwidth}

    \raggedright \textbf{fl\_get\_input\_xoffset}(\textit{pFlObject})

    \vspace{-1.5ex}

    \rule{\textwidth}{0.5\fboxrule}
\setlength{\parskip}{2ex}

Obtains the current horizontal scrolling offset.

-{}-
\setlength{\parskip}{1ex}
      \textbf{Parameters}
      \vspace{-1ex}

      \begin{quote}
        \begin{Ventry}{xxxxxxxxx}

          \item[pFlObject]


input object
            {\it (type=pointer to xfdata.FL\_OBJECT)}

        \end{Ventry}

      \end{quote}

      \textbf{Return Value}
    \vspace{-1ex}

      \begin{quote}

horizontal offset of input.
      {\it (type=int)}

      \end{quote}

\textbf{Note:} 
e.g. \emph{todo}


\textbf{Status:} 
Untested + NoDoc + NoDemo = NOT OK


    \end{boxedminipage}

    \label{xformslib:flinput:fl_set_input_fieldchar}
    \index{xformslib \textit{(package)}!xformslib.flinput \textit{(module)}!xformslib.flinput.fl\_set\_input\_fieldchar \textit{(function)}}

    \vspace{0.5ex}

\hspace{.8\funcindent}\begin{boxedminipage}{\funcwidth}

    \raggedright \textbf{fl\_set\_input\_fieldchar}(\textit{pFlObject}, \textit{fldchar})

    \vspace{-1.5ex}

    \rule{\textwidth}{0.5\fboxrule}
\setlength{\parskip}{2ex}

Changes the character used to draw the text, for secret input field. By
default text is drawn using spaces.

-{}-
\setlength{\parskip}{1ex}
      \textbf{Parameters}
      \vspace{-1ex}

      \begin{quote}
        \begin{Ventry}{xxxxxxxxx}

          \item[pFlObject]


input object
            {\it (type=pointer to xfdata.FL\_OBJECT)}

          \item[fldchar]


character to use in secret input fields
            {\it (type=int or char)}

        \end{Ventry}

      \end{quote}

      \textbf{Return Value}
    \vspace{-1ex}

      \begin{quote}

old field character (ordinal form)
      {\it (type=int)}

      \end{quote}

\textbf{Note:} 
e.g. \emph{todo}


\textbf{Status:} 
Untested + NoDoc + NoDemo = NOT OK


    \end{boxedminipage}

    \label{xformslib:flinput:fl_get_input_topline}
    \index{xformslib \textit{(package)}!xformslib.flinput \textit{(module)}!xformslib.flinput.fl\_get\_input\_topline \textit{(function)}}

    \vspace{0.5ex}

\hspace{.8\funcindent}\begin{boxedminipage}{\funcwidth}

    \raggedright \textbf{fl\_get\_input\_topline}(\textit{pFlObject})

    \vspace{-1.5ex}

    \rule{\textwidth}{0.5\fboxrule}
\setlength{\parskip}{2ex}

Obtains the current topline in the input field.

-{}-
\setlength{\parskip}{1ex}
      \textbf{Parameters}
      \vspace{-1ex}

      \begin{quote}
        \begin{Ventry}{xxxxxxxxx}

          \item[pFlObject]


input object
            {\it (type=pointer to xfdata.FL\_OBJECT)}

        \end{Ventry}

      \end{quote}

      \textbf{Return Value}
    \vspace{-1ex}

      \begin{quote}

num.
      {\it (type=int)}

      \end{quote}

\textbf{Note:} 
e.g. \emph{todo}


\textbf{Status:} 
Untested + NoDoc + NoDemo = NOT OK


    \end{boxedminipage}

    \label{xformslib:flinput:fl_get_input_screenlines}
    \index{xformslib \textit{(package)}!xformslib.flinput \textit{(module)}!xformslib.flinput.fl\_get\_input\_screenlines \textit{(function)}}

    \vspace{0.5ex}

\hspace{.8\funcindent}\begin{boxedminipage}{\funcwidth}

    \raggedright \textbf{fl\_get\_input\_screenlines}(\textit{pFlObject})

    \vspace{-1.5ex}

    \rule{\textwidth}{0.5\fboxrule}
\setlength{\parskip}{2ex}

Obtains the number of lines that fit inside the input box.

-{}-
\setlength{\parskip}{1ex}
      \textbf{Parameters}
      \vspace{-1ex}

      \begin{quote}
        \begin{Ventry}{xxxxxxxxx}

          \item[pFlObject]


input object
            {\it (type=pointer to xfdata.FL\_OBJECT)}

        \end{Ventry}

      \end{quote}

      \textbf{Return Value}
    \vspace{-1ex}

      \begin{quote}

num.
      {\it (type=int)}

      \end{quote}

\textbf{Note:} 
e.g. \emph{todo}


\textbf{Status:} 
Untested + NoDoc + NoDemo = NOT OK


    \end{boxedminipage}

    \label{xformslib:flinput:fl_get_input_cursorpos}
    \index{xformslib \textit{(package)}!xformslib.flinput \textit{(module)}!xformslib.flinput.fl\_get\_input\_cursorpos \textit{(function)}}

    \vspace{0.5ex}

\hspace{.8\funcindent}\begin{boxedminipage}{\funcwidth}

    \raggedright \textbf{fl\_get\_input\_cursorpos}(\textit{pFlObject})

    \vspace{-1.5ex}

    \rule{\textwidth}{0.5\fboxrule}
\setlength{\parskip}{2ex}

Obtains the cursor position measured in number of characters (including
newline characters) in front of the cursor.

-{}-
\setlength{\parskip}{1ex}
      \textbf{Parameters}
      \vspace{-1ex}

      \begin{quote}
        \begin{Ventry}{xxxxxxxxx}

          \item[pFlObject]


input object
            {\it (type=pointer to xfdata.FL\_OBJECT)}

        \end{Ventry}

      \end{quote}

      \textbf{Return Value}
    \vspace{-1ex}

      \begin{quote}

num. or -1 (if the input field does not have input focus thus
does not have a cursor), horizontal position (x), vertical position
(y)
      {\it (type=int, int, int)}

      \end{quote}

\textbf{Note:} 
e.g. \emph{todo}


\textbf{Attention:} 
API change from XForms - upstream was
fl\_get\_input\_cursorpos(pFlObject, x, y)


\textbf{Status:} 
Tested + NoDoc + Demo = OK


    \end{boxedminipage}

    \label{xformslib:flinput:fl_set_input_cursor_visible}
    \index{xformslib \textit{(package)}!xformslib.flinput \textit{(module)}!xformslib.flinput.fl\_set\_input\_cursor\_visible \textit{(function)}}

    \vspace{0.5ex}

\hspace{.8\funcindent}\begin{boxedminipage}{\funcwidth}

    \raggedright \textbf{fl\_set\_input\_cursor\_visible}(\textit{pFlObject}, \textit{yesno})

    \vspace{-1.5ex}

    \rule{\textwidth}{0.5\fboxrule}
\setlength{\parskip}{2ex}

Turns on/off the cursor of the input field.

-{}-
\setlength{\parskip}{1ex}
      \textbf{Parameters}
      \vspace{-1ex}

      \begin{quote}
        \begin{Ventry}{xxxxxxxxx}

          \item[pFlObject]


input object
            {\it (type=pointer to xfdata.FL\_OBJECT)}

          \item[yesno]


flag to set/unset cursor visibility. Values 1 (visible) or 0 (not
visible)
            {\it (type=int)}

        \end{Ventry}

      \end{quote}

\textbf{Note:} 
e.g. \emph{todo}


\textbf{Status:} 
Untested + NoDoc + NoDemo = NOT OK


    \end{boxedminipage}

    \label{xformslib:flinput:fl_get_input_numberoflines}
    \index{xformslib \textit{(package)}!xformslib.flinput \textit{(module)}!xformslib.flinput.fl\_get\_input\_numberoflines \textit{(function)}}

    \vspace{0.5ex}

\hspace{.8\funcindent}\begin{boxedminipage}{\funcwidth}

    \raggedright \textbf{fl\_get\_input\_numberoflines}(\textit{pFlObject})

    \vspace{-1.5ex}

    \rule{\textwidth}{0.5\fboxrule}
\setlength{\parskip}{2ex}

Obtains the number of lines in the input field.

-{}-
\setlength{\parskip}{1ex}
      \textbf{Parameters}
      \vspace{-1ex}

      \begin{quote}
        \begin{Ventry}{xxxxxxxxx}

          \item[pFlObject]


input object
            {\it (type=pointer to xfdata.FL\_OBJECT)}

        \end{Ventry}

      \end{quote}

      \textbf{Return Value}
    \vspace{-1ex}

      \begin{quote}

number of lines in input
      {\it (type=int)}

      \end{quote}

\textbf{Note:} 
e.g. \emph{todo}


\textbf{Status:} 
Untested + NoDoc + NoDemo = NOT OK


    \end{boxedminipage}

    \label{xformslib:flinput:fl_get_input_format}
    \index{xformslib \textit{(package)}!xformslib.flinput \textit{(module)}!xformslib.flinput.fl\_get\_input\_format \textit{(function)}}

    \vspace{0.5ex}

\hspace{.8\funcindent}\begin{boxedminipage}{\funcwidth}

    \raggedright \textbf{fl\_get\_input\_format}(\textit{pFlObject})

    \vspace{-1.5ex}

    \rule{\textwidth}{0.5\fboxrule}
\setlength{\parskip}{2ex}

Provides means for the validator to retrieve some information about
user preference or other state dependent informations.

-{}-
\setlength{\parskip}{1ex}
      \textbf{Parameters}
      \vspace{-1ex}

      \begin{quote}
        \begin{Ventry}{xxxxxxxxx}

          \item[pFlObject]


input object
            {\it (type=pointer to xfdata.FL\_OBJECT)}

        \end{Ventry}

      \end{quote}

      \textbf{Return Value}
    \vspace{-1ex}

      \begin{quote}

format, separator (in ordinal form)
      {\it (type=int, int)}

      \end{quote}

\textbf{Note:} 
e.g. \emph{todo}


\textbf{Attention:} 
API change from XForms - upstream was
fl\_get\_input\_format(pFlObject, fmt, sep)


\textbf{Status:} 
Untested + Doc + NoDemo = NOT OK


    \end{boxedminipage}

    \label{xformslib:flinput:fl_get_input}
    \index{xformslib \textit{(package)}!xformslib.flinput \textit{(module)}!xformslib.flinput.fl\_get\_input \textit{(function)}}

    \vspace{0.5ex}

\hspace{.8\funcindent}\begin{boxedminipage}{\funcwidth}

    \raggedright \textbf{fl\_get\_input}(\textit{pFlObject})

    \vspace{-1.5ex}

    \rule{\textwidth}{0.5\fboxrule}
\setlength{\parskip}{2ex}

Obtains the text string in the field (when the user has changed it).

-{}-
\setlength{\parskip}{1ex}
      \textbf{Parameters}
      \vspace{-1ex}

      \begin{quote}
        \begin{Ventry}{xxxxxxxxx}

          \item[pFlObject]


input object
            {\it (type=pointer to xfdata.FL\_OBJECT)}

        \end{Ventry}

      \end{quote}

      \textbf{Return Value}
    \vspace{-1ex}

      \begin{quote}

user input string
      {\it (type=str)}

      \end{quote}

\textbf{Note:} 
e.g. \emph{todo}


\textbf{Status:} 
Tested + NoDoc + Demo = OK


    \end{boxedminipage}

    \label{xformslib:flinput:fl_set_input_filter}
    \index{xformslib \textit{(package)}!xformslib.flinput \textit{(module)}!xformslib.flinput.fl\_set\_input\_filter \textit{(function)}}

    \vspace{0.5ex}

\hspace{.8\funcindent}\begin{boxedminipage}{\funcwidth}

    \raggedright \textbf{fl\_set\_input\_filter}(\textit{pFlObject}, \textit{py\_InputValidator})

    \vspace{-1.5ex}

    \rule{\textwidth}{0.5\fboxrule}
\setlength{\parskip}{2ex}

Sets up a validator function, that is is called whenever a new
(regular) character is entered.
-{}-
\setlength{\parskip}{1ex}
      \textbf{Parameters}
      \vspace{-1ex}

      \begin{quote}
        \begin{Ventry}{xxxxxxxxxxxxxxxxx}

          \item[pFlObject]


input object
            {\it (type=pointer to xfdata.FL\_OBJECT)}

          \item[py\_InputValidator]


name referring to function(pFlObject, oldstr, c, int) -> oldfilt
oldstr is the string in the input field before the newly typed
character c was added to form the new string cur. If the new character
is not an acceptable character for the input field, the filter
function should return xfdata.FL\_INVALID otherwise xfdata.FL\_VALID.
If xfdata.FL\_INVALID is returned, the new character is discarded and
the input field remains unmodified. The function returns the old
filter. While the built-in filters also sound the keyboard bell, this
doesn?t happen if a custom filter only returns FL\_INVALID. To also
sound the keyboard bell you can logically OR it with xfdata.FL\_INVALID
| xfdata.FL\_RINGBELL. This still leaves the possibility that the
input is valid for every character entered, but the string is invalid
for the field because it is incomplete. E.g. 12.0e is valid for a
float input field for every character typed, but the final string is
not a valid floating point number. To guard again this, the filter
function is also called just prior to returning the object with the
argument c (for the newly entered character) set to zero. If the
validator returns xfdata.FL\_INVALID the object is not returned to the
application program, but input focus can change to the next input
field. If the return value xfdata.FL\_INVALID | xfdata.FL\_RINGBELL,
the keyboard bell is sounded and the object is not returned to the
application program and the input focus remains in the object.
            {\it (type=python function to set a validator, returned value)}

        \end{Ventry}

      \end{quote}

      \textbf{Return Value}
    \vspace{-1ex}

      \begin{quote}

old filter function for input
      {\it (type=xfdata.FL\_INPUTVALIDATOR)}

      \end{quote}

\textbf{Note:} 
e.g. \emph{todo}


\textbf{Status:} 
Untested + NoDoc + NoDemo = NOT OK


    \end{boxedminipage}

    \label{xformslib:flinput:fl_validate_input}
    \index{xformslib \textit{(package)}!xformslib.flinput \textit{(module)}!xformslib.flinput.fl\_validate\_input \textit{(function)}}

    \vspace{0.5ex}

\hspace{.8\funcindent}\begin{boxedminipage}{\funcwidth}

    \raggedright \textbf{fl\_validate\_input}(\textit{pFlObject})

    \vspace{-1.5ex}

    \rule{\textwidth}{0.5\fboxrule}
\setlength{\parskip}{2ex}

Tests if the value in an input field is valid, according to the
pre-defined validator function.

-{}-
\setlength{\parskip}{1ex}
      \textbf{Parameters}
      \vspace{-1ex}

      \begin{quote}
        \begin{Ventry}{xxxxxxxxx}

          \item[pFlObject]


input object
            {\it (type=pointer to xfdata.FL\_OBJECT)}

        \end{Ventry}

      \end{quote}

      \textbf{Return Value}
    \vspace{-1ex}

      \begin{quote}

(from xfdata.py) FL\_VALID (if value is valid or if there is no
validator function set for the input), otherwise FL\_INVALID
      {\it (type=int)}

      \end{quote}

\textbf{Note:} 
e.g. \emph{todo}


\textbf{Status:} 
Untested + Doc + NoDemo = NOT OK


    \end{boxedminipage}

    \label{xformslib:flbasic:fl_set_object_shortcut}
    \index{xformslib \textit{(package)}!xformslib.flbasic \textit{(module)}!xformslib.flbasic.fl\_set\_object\_shortcut \textit{(function)}}

    \vspace{0.5ex}

\hspace{.8\funcindent}\begin{boxedminipage}{\funcwidth}

    \raggedright \textbf{fl\_set\_input\_shortcut}(\textit{pFlObject}, \textit{shctxt}, \textit{showit})

    \vspace{-1.5ex}

    \rule{\textwidth}{0.5\fboxrule}
\setlength{\parskip}{2ex}

Sets a shortcut, binding a key or a series of keys to an object. It
resets any previous defined shortcuts for the object. Using e.g. ``acE\#d\textasciicircum{}h''
the keys 'a', 'c', 'E', <Alt>d and <Ctrl>h are associated with the object.
The precise format is as follows: any character in the string is considered
as a shortcut, except '\textasciicircum{}' and '\#', which stand for combinations with the
<Ctrl> and <Alt> keys; the case of the key following '\#' or '\textasciicircum{}' is not
important, i.e. no distinction is made between e.g. ``\textasciicircum{}C'' and ``\textasciicircum{}c'', both
encode the key combination <Ctrl>C as well as <Ctrl>C.) The key '\textasciicircum{}' itself
can be set as a shortcut key by using ``\textasciicircum{}\textasciicircum{}'' in the string defining the
shortcut. The key '\#' can be obtained as a shortcut by using the string
``\textasciicircum{}\#''. So, e.g. ``\#\textasciicircum{}\#'' encodes <ALT>\#. The <Esc> key can be given as ``\textasciicircum{}{[}''.
Another special character not mentioned yet is '\&', which indicates
function and arrow keys. Use a sequence starting with '\&' and directly
followed by a number between 1 and 35 to represent one of the function
keys. For example, ``\&2'' stands for the <F2> function key. The four cursors
keys (up, down, right, and left) can be given as ``\&A'', ``\&B'', ``\&C'' and ``\&D'',
respectively. The key '\&' itself can be obtained as a shortcut by
prefixing it with '\textasciicircum{}'.

-{}-
\setlength{\parskip}{1ex}
      \textbf{Parameters}
      \vspace{-1ex}

      \begin{quote}
        \begin{Ventry}{xxxxxxxxx}

          \item[pFlObject]


object
            {\it (type=pointer to xfdata.FL\_OBJECT)}

          \item[shctxt]


shortcut text to be set
            {\it (type=str)}

          \item[showit]


flag if shortcut letter has to be underlined or not if a match exists
(only the 1st alphanumeric character is used). Values 0 (underline not
shown) or 1 (shown)
            {\it (type=int)}

        \end{Ventry}

      \end{quote}

\textbf{Note:} 
e.g. fl\_set\_object\_shortcut(pobj6, ``aA\#A\textasciicircum{}A'', 1)


\textbf{Status:} 
Tested + Doc + NoDemo = OK


    \end{boxedminipage}

    \label{xformslib:flinput:fl_set_input_editkeymap}
    \index{xformslib \textit{(package)}!xformslib.flinput \textit{(module)}!xformslib.flinput.fl\_set\_input\_editkeymap \textit{(function)}}

    \vspace{0.5ex}

\hspace{.8\funcindent}\begin{boxedminipage}{\funcwidth}

    \raggedright \textbf{fl\_set\_input\_editkeymap}(\textit{pEditKeymap})

    \vspace{-1.5ex}

    \rule{\textwidth}{0.5\fboxrule}
\setlength{\parskip}{2ex}

Changes the default edit keymaps. Edit keymap is global and affects all
input field within the application. All cursor keys (<Left>, <Home> etc.)
are reserved and their meanings hard-coded, thus can?t be used in the
mapping. For example, if you try to set del\_prev\_char to <Home>, pressing
the <Home> key will not delete the previous character. In filling the
keymap structure, ASCII characters (i.e. characters with values below 128,
including the control characters with values below 32) should be specified
by their ASCII codes (<Ctrl> C is 3 etc.), while all others by their
Keysyms (XK\_F1 etc.). Control and special character combinations can be
obtained by adding xfdata.FL\_CONTROL\_MASK to the Keysym. To specify Meta
add xfdata.FL\_ALT\_MASK to the key value.
-{}-
\setlength{\parskip}{1ex}
      \textbf{Parameters}
      \vspace{-1ex}

      \begin{quote}
        \begin{Ventry}{xxxxxxxxxxx}

          \item[pEditKeymap]


class instance (filled or partially filled). If None, it restores the
default keymap.
            {\it (type=pointer to xfdata.FL\_EditKeymap)}

        \end{Ventry}

      \end{quote}

\textbf{Note:} 
e.g. \emph{todo}


\textbf{Status:} 
Untested + NoDoc + NoDemo = NOT OK


    \end{boxedminipage}

    \index{xformslib \textit{(package)}!xformslib.flinput \textit{(module)}|)}
