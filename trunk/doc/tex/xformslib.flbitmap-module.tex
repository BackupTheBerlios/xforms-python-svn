%
% API Documentation for API Documentation
% Module xformslib.flbitmap
%
% Generated by epydoc 3.0.1
% [Sun Jan 31 21:45:30 2010]
%

%%%%%%%%%%%%%%%%%%%%%%%%%%%%%%%%%%%%%%%%%%%%%%%%%%%%%%%%%%%%%%%%%%%%%%%%%%%
%%                          Module Description                           %%
%%%%%%%%%%%%%%%%%%%%%%%%%%%%%%%%%%%%%%%%%%%%%%%%%%%%%%%%%%%%%%%%%%%%%%%%%%%

    \index{xformslib \textit{(package)}!xformslib.flbitmap \textit{(module)}|(}
\section{Module xformslib.flbitmap}

    \label{xformslib:flbitmap}
xforms-python - Python wrapper for XForms (X11) GUI C toolkit library using
ctypes

Copyright (C) 2009, 2010  Luca Lazzaroni "LukenShiro" e-mail: 
{\textless}lukenshiro@ngi.it{\textgreater}

This program is free software: you can redistribute it and/or modify it 
under the terms of the GNU Lesser General Public License as published by 
the Free Software Foundation, version 2.1 of the License.

This program is distributed in the hope that it will be useful, but WITHOUT
ANY WARRANTY; without even the implied warranty of MERCHANTABILITY or 
FITNESS FOR A PARTICULAR PURPOSE. See the GNU Lesser General Public License
for more details.

You should have received a copy of the GNU LGPL along with this program. If
not, see {\textless}http://www.gnu.org/licenses/{\textgreater}.

See CREDITS file to read acknowledgements and thanks to XForms, ctypes and 
other developers.


%%%%%%%%%%%%%%%%%%%%%%%%%%%%%%%%%%%%%%%%%%%%%%%%%%%%%%%%%%%%%%%%%%%%%%%%%%%
%%                               Functions                               %%
%%%%%%%%%%%%%%%%%%%%%%%%%%%%%%%%%%%%%%%%%%%%%%%%%%%%%%%%%%%%%%%%%%%%%%%%%%%

  \subsection{Functions}

    \label{xformslib:flbitmap:fl_add_bitmap}
    \index{xformslib \textit{(package)}!xformslib.flbitmap \textit{(module)}!xformslib.flbitmap.fl\_add\_bitmap \textit{(function)}}

    \vspace{0.5ex}

\hspace{.8\funcindent}\begin{boxedminipage}{\funcwidth}

    \raggedright \textbf{fl\_add\_bitmap}(\textit{bitmaptype}, \textit{x}, \textit{y}, \textit{w}, \textit{h}, \textit{label})

    \vspace{-1.5ex}

    \rule{\textwidth}{0.5\fboxrule}
\setlength{\parskip}{2ex}
    Adds a bitmap object.

\setlength{\parskip}{1ex}
      \textbf{Parameters}
      \vspace{-1ex}

      \begin{quote}
        \begin{Ventry}{xxxxxxxxxx}

          \item[bitmaptype]

          type of bitmap to be added. Values (from xfdata) module) 
          FL\_NORMAL\_BITMAP

            {\it (type=int)}

          \item[x]

          horizontal position (upper-left corner)

            {\it (type=int)}

          \item[y]

          vertical position of bitmap (upper-left corner)

            {\it (type=int)}

          \item[w]

          width in coord units

            {\it (type=int)}

          \item[h]

          height in coord units

            {\it (type=int)}

          \item[label]

          text label of bitmap

            {\it (type=str)}

        \end{Ventry}

      \end{quote}

      \textbf{Return Value}
    \vspace{-1ex}

      \begin{quote}
      object created (pFlObject)

      {\it (type=pointer to xfdata.FL\_OBJECT)}

      \end{quote}

\textbf{Example:} fl\_add\_bitmap(xfdata.FL\_NORMAL\_BITMAP, 320, 200, 100, 100,         
"MyBitmap")



\textbf{Status:} Tested + NoDoc + Demo = OK



    \end{boxedminipage}

    \label{xformslib:flbitmap:fl_set_bitmap_data}
    \index{xformslib \textit{(package)}!xformslib.flbitmap \textit{(module)}!xformslib.flbitmap.fl\_set\_bitmap\_data \textit{(function)}}

    \vspace{0.5ex}

\hspace{.8\funcindent}\begin{boxedminipage}{\funcwidth}

    \raggedright \textbf{fl\_set\_bitmap\_data}(\textit{pFlObject}, \textit{w}, \textit{h}, \textit{xbmcontents})

    \vspace{-1.5ex}

    \rule{\textwidth}{0.5\fboxrule}
\setlength{\parskip}{2ex}
    Sets the actual bitmap being displayed from specified contents. A 
    number of bitmaps can be found in '/usr/include/X11/bitmaps' or similar
    places. The X program 'bitmap' can be used to create bitmaps.

\setlength{\parskip}{1ex}
      \textbf{Parameters}
      \vspace{-1ex}

      \begin{quote}
        \begin{Ventry}{xxxxxxxxxxx}

          \item[pFlObject]

          pointer to object

            {\it (type=pointer to xfdata.FL\_OBJECT)}

          \item[w]

          width of bitmap in cood units

            {\it (type=int)}

          \item[h]

          height of bitmap in coord units

            {\it (type=int)}

          \item[xbmcontents]

          bitmap data used for contents

            {\it (type=str of ubytes characters)}

        \end{Ventry}

      \end{quote}

\textbf{Example:} ??



\textbf{Status:} Untested + Doc + NoDemo = NOT OK



    \end{boxedminipage}

    \label{xformslib:flbitmap:fl_set_bitmap_file}
    \index{xformslib \textit{(package)}!xformslib.flbitmap \textit{(module)}!xformslib.flbitmap.fl\_set\_bitmap\_file \textit{(function)}}

    \vspace{0.5ex}

\hspace{.8\funcindent}\begin{boxedminipage}{\funcwidth}

    \raggedright \textbf{fl\_set\_bitmap\_file}(\textit{pFlObject}, \textit{fname})

    \vspace{-1.5ex}

    \rule{\textwidth}{0.5\fboxrule}
\setlength{\parskip}{2ex}
    Sets the actual bitmap being displayed from a specified .xbm file. A 
    number of bitmaps can be found in '/usr/include/X11/bitmaps' or similar
    places. The X program 'bitmap' can be used to create bitmaps.

\setlength{\parskip}{1ex}
      \textbf{Parameters}
      \vspace{-1ex}

      \begin{quote}
        \begin{Ventry}{xxxxxxxxx}

          \item[pFlObject]

          bitmap object

            {\it (type=pointer to xfdata.FL\_OBJECT)}

          \item[fname]

          name (path included if necessary) of bitmap (.xbm format) file

            {\it (type=str)}

        \end{Ventry}

      \end{quote}

\textbf{Example:} fl\_set\_bitmap\_file(xbmobj, "mybitmapfile.xbm")



\textbf{Status:} Tested + Doc + Demo = OK



    \end{boxedminipage}

    \label{xformslib:flbitmap:fl_set_bitmap_file}
    \index{xformslib \textit{(package)}!xformslib.flbitmap \textit{(module)}!xformslib.flbitmap.fl\_set\_bitmap\_file \textit{(function)}}

    \vspace{0.5ex}

\hspace{.8\funcindent}\begin{boxedminipage}{\funcwidth}

    \raggedright \textbf{fl\_set\_bitmapbutton\_file}(\textit{pFlObject}, \textit{fname})

    \vspace{-1.5ex}

    \rule{\textwidth}{0.5\fboxrule}
\setlength{\parskip}{2ex}
    Sets the actual bitmap being displayed from a specified .xbm file. A 
    number of bitmaps can be found in '/usr/include/X11/bitmaps' or similar
    places. The X program 'bitmap' can be used to create bitmaps.

\setlength{\parskip}{1ex}
      \textbf{Parameters}
      \vspace{-1ex}

      \begin{quote}
        \begin{Ventry}{xxxxxxxxx}

          \item[pFlObject]

          bitmap object

            {\it (type=pointer to xfdata.FL\_OBJECT)}

          \item[fname]

          name (path included if necessary) of bitmap (.xbm format) file

            {\it (type=str)}

        \end{Ventry}

      \end{quote}

\textbf{Example:} fl\_set\_bitmap\_file(xbmobj, "mybitmapfile.xbm")



\textbf{Status:} Tested + Doc + Demo = OK



    \end{boxedminipage}

    \label{xformslib:flbitmap:fl_set_bitmap_file}
    \index{xformslib \textit{(package)}!xformslib.flbitmap \textit{(module)}!xformslib.flbitmap.fl\_set\_bitmap\_file \textit{(function)}}

    \vspace{0.5ex}

\hspace{.8\funcindent}\begin{boxedminipage}{\funcwidth}

    \raggedright \textbf{fl\_set\_bitmapbutton\_datafile}(\textit{pFlObject}, \textit{fname})

    \vspace{-1.5ex}

    \rule{\textwidth}{0.5\fboxrule}
\setlength{\parskip}{2ex}
    Sets the actual bitmap being displayed from a specified .xbm file. A 
    number of bitmaps can be found in '/usr/include/X11/bitmaps' or similar
    places. The X program 'bitmap' can be used to create bitmaps.

\setlength{\parskip}{1ex}
      \textbf{Parameters}
      \vspace{-1ex}

      \begin{quote}
        \begin{Ventry}{xxxxxxxxx}

          \item[pFlObject]

          bitmap object

            {\it (type=pointer to xfdata.FL\_OBJECT)}

          \item[fname]

          name (path included if necessary) of bitmap (.xbm format) file

            {\it (type=str)}

        \end{Ventry}

      \end{quote}

\textbf{Example:} fl\_set\_bitmap\_file(xbmobj, "mybitmapfile.xbm")



\textbf{Status:} Tested + Doc + Demo = OK



    \end{boxedminipage}

    \label{xformslib:flbitmap:fl_read_bitmapfile}
    \index{xformslib \textit{(package)}!xformslib.flbitmap \textit{(module)}!xformslib.flbitmap.fl\_read\_bitmapfile \textit{(function)}}

    \vspace{0.5ex}

\hspace{.8\funcindent}\begin{boxedminipage}{\funcwidth}

    \raggedright \textbf{fl\_read\_bitmapfile}(\textit{win}, \textit{filename})

    \vspace{-1.5ex}

    \rule{\textwidth}{0.5\fboxrule}
\setlength{\parskip}{2ex}
    Makes a bitmap from a bitmap file.

\setlength{\parskip}{1ex}
      \textbf{Parameters}
      \vspace{-1ex}

      \begin{quote}
        \begin{Ventry}{xxxxxxxx}

          \item[win]

          window id

            {\it (type=long\_pos)}

          \item[filename]

          name of bitmap (.xbm format) file

            {\it (type=str)}

        \end{Ventry}

      \end{quote}

      \textbf{Return Value}
    \vspace{-1ex}

      \begin{quote}
      pixmap, w, h, hotx, hoty

      {\it (type=long\_pos, int\_pos, int\_pos, int, int)}

      \end{quote}

\textbf{Example:} pmap, w, h, hotx, hoty = fl\_read\_bitmapfile(win0, "xbmfile.xbm")



\textbf{Attention:} API change from XForms - upstream was fl\_read\_bitmapfile(win, filename, 
w, h, hotx, hoty)



\textbf{Status:} Tested + Doc + NoDemo = OK



    \end{boxedminipage}

    \label{xformslib:flbitmap:fl_create_from_bitmapdata}
    \index{xformslib \textit{(package)}!xformslib.flbitmap \textit{(module)}!xformslib.flbitmap.fl\_create\_from\_bitmapdata \textit{(function)}}

    \vspace{0.5ex}

\hspace{.8\funcindent}\begin{boxedminipage}{\funcwidth}

    \raggedright \textbf{fl\_create\_from\_bitmapdata}(\textit{win}, \textit{data}, \textit{w}, \textit{h})

    \vspace{-1.5ex}

    \rule{\textwidth}{0.5\fboxrule}
\setlength{\parskip}{2ex}
    Makes a bitmap from bitmap contents data.

\setlength{\parskip}{1ex}
      \textbf{Parameters}
      \vspace{-1ex}

      \begin{quote}
        \begin{Ventry}{xxxx}

          \item[win]

          window

            {\it (type=long\_pos)}

          \item[data]

          bitmap contents data

            {\it (type=str of ubyte)}

          \item[w]

          width of bitmap in coord units

            {\it (type=int\_pos)}

          \item[h]

          height of bitmap in coord units

            {\it (type=int\_pos)}

        \end{Ventry}

      \end{quote}

      \textbf{Return Value}
    \vspace{-1ex}

      \begin{quote}
      pixmap created

      {\it (type=long\_pos)}

      \end{quote}

\textbf{Example:} ??



\textbf{Status:} Untested + Doc + NoDemo = NOT OK



    \end{boxedminipage}

    \label{xformslib:flbitmap:fl_add_pixmap}
    \index{xformslib \textit{(package)}!xformslib.flbitmap \textit{(module)}!xformslib.flbitmap.fl\_add\_pixmap \textit{(function)}}

    \vspace{0.5ex}

\hspace{.8\funcindent}\begin{boxedminipage}{\funcwidth}

    \raggedright \textbf{fl\_add\_pixmap}(\textit{pixmaptype}, \textit{x}, \textit{y}, \textit{w}, \textit{h}, \textit{label})

    \vspace{-1.5ex}

    \rule{\textwidth}{0.5\fboxrule}
\setlength{\parskip}{2ex}
    Adds a pixmap object.

\setlength{\parskip}{1ex}
      \textbf{Parameters}
      \vspace{-1ex}

      \begin{quote}
        \begin{Ventry}{xxxxxxxxxx}

          \item[pixmaptype]

          type of pixmap to be added. Values (from xfdata module) 
          FL\_NORMAL\_PIXMAP

            {\it (type=int)}

          \item[x]

          horizontal position (upper-left corner) @type x int

          \item[y]

          vertical position of bitmap (upper-left corner) @type y int

          \item[w]

          width in coord units @type w int

          \item[h]

          height in coord units @type h int

          \item[label]

          text label of pixmap

            {\it (type=str)}

        \end{Ventry}

      \end{quote}

      \textbf{Return Value}
    \vspace{-1ex}

      \begin{quote}
      object created (pFlObject)

      {\it (type=pointer to xfdata.FL\_OBJECT)}

      \end{quote}

\textbf{Example:} fl\_add\_pixmap(xfdata.FL\_NORMAL\_PIXMAP, 320, 200, 100, 100,         
"MyPixmap")



\textbf{Status:} Tested + Doc + NoDemo = OK



    \end{boxedminipage}

    \label{xformslib:flbitmap:fl_set_pixmap_data}
    \index{xformslib \textit{(package)}!xformslib.flbitmap \textit{(module)}!xformslib.flbitmap.fl\_set\_pixmap\_data \textit{(function)}}

    \vspace{0.5ex}

\hspace{.8\funcindent}\begin{boxedminipage}{\funcwidth}

    \raggedright \textbf{fl\_set\_pixmap\_data}(\textit{pFlObject}, \textit{bits})

    \vspace{-1.5ex}

    \rule{\textwidth}{0.5\fboxrule}
\setlength{\parskip}{2ex}
    Sets the actual bitmap being displayed from specified data. A number of
    pixmaps can be found in '/usr/include/X11/pixmaps' or similar places.

\setlength{\parskip}{1ex}
      \textbf{Parameters}
      \vspace{-1ex}

      \begin{quote}
        \begin{Ventry}{xxxxxxxxx}

          \item[pFlObject]

          pixmap object

            {\it (type=pointer to xfdata.FL\_OBJECT)}

          \item[bits]

          bits contents of pixmap

            {\it (type=str of ubyte)}

        \end{Ventry}

      \end{quote}

\textbf{Example:} ??



\textbf{Status:} Untested + Doc + NoDemo = NOT OK



    \end{boxedminipage}

    \label{xformslib:flbitmap:fl_set_pixmap_file}
    \index{xformslib \textit{(package)}!xformslib.flbitmap \textit{(module)}!xformslib.flbitmap.fl\_set\_pixmap\_file \textit{(function)}}

    \vspace{0.5ex}

\hspace{.8\funcindent}\begin{boxedminipage}{\funcwidth}

    \raggedright \textbf{fl\_set\_pixmap\_file}(\textit{pFlObject}, \textit{fname})

    \vspace{-1.5ex}

    \rule{\textwidth}{0.5\fboxrule}
\setlength{\parskip}{2ex}
    Sets the actual bitmap being displayed from a specified .xpm file. A 
    number of pixmaps can be found in '/usr/include/X11/pixmaps' or similar
    places.

\setlength{\parskip}{1ex}
      \textbf{Parameters}
      \vspace{-1ex}

      \begin{quote}
        \begin{Ventry}{xxxxxxxxx}

          \item[pFlObject]

          pixmap object

            {\it (type=pointer to xfdata.FL\_OBJECT)}

          \item[fname]

          name (path included if necessary) of pixmap (.xpm format) file

            {\it (type=str)}

        \end{Ventry}

      \end{quote}

\textbf{Example:} fl\_set\_pixmap\_file(xpmobj, "mypixmapfile.xpm")



\textbf{Status:} Tested + Doc + Demo = OK



    \end{boxedminipage}

    \label{xformslib:flbitmap:fl_set_pixmap_file}
    \index{xformslib \textit{(package)}!xformslib.flbitmap \textit{(module)}!xformslib.flbitmap.fl\_set\_pixmap\_file \textit{(function)}}

    \vspace{0.5ex}

\hspace{.8\funcindent}\begin{boxedminipage}{\funcwidth}

    \raggedright \textbf{fl\_set\_pixmapbutton\_file}(\textit{pFlObject}, \textit{fname})

    \vspace{-1.5ex}

    \rule{\textwidth}{0.5\fboxrule}
\setlength{\parskip}{2ex}
    Sets the actual bitmap being displayed from a specified .xpm file. A 
    number of pixmaps can be found in '/usr/include/X11/pixmaps' or similar
    places.

\setlength{\parskip}{1ex}
      \textbf{Parameters}
      \vspace{-1ex}

      \begin{quote}
        \begin{Ventry}{xxxxxxxxx}

          \item[pFlObject]

          pixmap object

            {\it (type=pointer to xfdata.FL\_OBJECT)}

          \item[fname]

          name (path included if necessary) of pixmap (.xpm format) file

            {\it (type=str)}

        \end{Ventry}

      \end{quote}

\textbf{Example:} fl\_set\_pixmap\_file(xpmobj, "mypixmapfile.xpm")



\textbf{Status:} Tested + Doc + Demo = OK



    \end{boxedminipage}

    \label{xformslib:flbitmap:fl_set_pixmap_file}
    \index{xformslib \textit{(package)}!xformslib.flbitmap \textit{(module)}!xformslib.flbitmap.fl\_set\_pixmap\_file \textit{(function)}}

    \vspace{0.5ex}

\hspace{.8\funcindent}\begin{boxedminipage}{\funcwidth}

    \raggedright \textbf{fl\_set\_pixmapbutton\_datafile}(\textit{pFlObject}, \textit{fname})

    \vspace{-1.5ex}

    \rule{\textwidth}{0.5\fboxrule}
\setlength{\parskip}{2ex}
    Sets the actual bitmap being displayed from a specified .xpm file. A 
    number of pixmaps can be found in '/usr/include/X11/pixmaps' or similar
    places.

\setlength{\parskip}{1ex}
      \textbf{Parameters}
      \vspace{-1ex}

      \begin{quote}
        \begin{Ventry}{xxxxxxxxx}

          \item[pFlObject]

          pixmap object

            {\it (type=pointer to xfdata.FL\_OBJECT)}

          \item[fname]

          name (path included if necessary) of pixmap (.xpm format) file

            {\it (type=str)}

        \end{Ventry}

      \end{quote}

\textbf{Example:} fl\_set\_pixmap\_file(xpmobj, "mypixmapfile.xpm")



\textbf{Status:} Tested + Doc + Demo = OK



    \end{boxedminipage}

    \label{xformslib:flbitmap:fl_set_pixmap_align}
    \index{xformslib \textit{(package)}!xformslib.flbitmap \textit{(module)}!xformslib.flbitmap.fl\_set\_pixmap\_align \textit{(function)}}

    \vspace{0.5ex}

\hspace{.8\funcindent}\begin{boxedminipage}{\funcwidth}

    \raggedright \textbf{fl\_set\_pixmap\_align}(\textit{pFlObject}, \textit{align}, \textit{xmargin}, \textit{ymargin})

    \vspace{-1.5ex}

    \rule{\textwidth}{0.5\fboxrule}
\setlength{\parskip}{2ex}
    Changes alignment of a pixmap. By default it is displayed centered 
    inside the bounding box.

\setlength{\parskip}{1ex}
      \textbf{Parameters}
      \vspace{-1ex}

      \begin{quote}
        \begin{Ventry}{xxxxxxxxx}

          \item[pFlObject]

          pixmap object

            {\it (type=pointer to xfdata.FL\_OBJECT)}

          \item[align]

          alignment of pixmap. Values (from xfdata module) 
          FL\_ALIGN\_CENTER, FL\_ALIGN\_TOP, FL\_ALIGN\_BOTTOM, 
          FL\_ALIGN\_LEFT, FL\_ALIGN\_RIGHT, FL\_ALIGN\_LEFT\_TOP, 
          FL\_ALIGN\_RIGHT\_TOP, FL\_ALIGN\_LEFT\_BOTTOM, 
          FL\_ALIGN\_RIGHT\_BOTTOM, FL\_ALIGN\_INSIDE, FL\_ALIGN\_VERT

            {\it (type=int)}

        \end{Ventry}

      \end{quote}

\textbf{Example:} fl\_set\_pixmap\_align(xpmobj, xfdata.FL\_ALIGN\_CENTER, 10, 10)



\textbf{Status:} Tested + Doc + Demo = OK



    \end{boxedminipage}

    \label{xformslib:flbitmap:fl_set_pixmap_align}
    \index{xformslib \textit{(package)}!xformslib.flbitmap \textit{(module)}!xformslib.flbitmap.fl\_set\_pixmap\_align \textit{(function)}}

    \vspace{0.5ex}

\hspace{.8\funcindent}\begin{boxedminipage}{\funcwidth}

    \raggedright \textbf{fl\_set\_pixmapbutton\_align}(\textit{pFlObject}, \textit{align}, \textit{xmargin}, \textit{ymargin})

    \vspace{-1.5ex}

    \rule{\textwidth}{0.5\fboxrule}
\setlength{\parskip}{2ex}
    Changes alignment of a pixmap. By default it is displayed centered 
    inside the bounding box.

\setlength{\parskip}{1ex}
      \textbf{Parameters}
      \vspace{-1ex}

      \begin{quote}
        \begin{Ventry}{xxxxxxxxx}

          \item[pFlObject]

          pixmap object

            {\it (type=pointer to xfdata.FL\_OBJECT)}

          \item[align]

          alignment of pixmap. Values (from xfdata module) 
          FL\_ALIGN\_CENTER, FL\_ALIGN\_TOP, FL\_ALIGN\_BOTTOM, 
          FL\_ALIGN\_LEFT, FL\_ALIGN\_RIGHT, FL\_ALIGN\_LEFT\_TOP, 
          FL\_ALIGN\_RIGHT\_TOP, FL\_ALIGN\_LEFT\_BOTTOM, 
          FL\_ALIGN\_RIGHT\_BOTTOM, FL\_ALIGN\_INSIDE, FL\_ALIGN\_VERT

            {\it (type=int)}

        \end{Ventry}

      \end{quote}

\textbf{Example:} fl\_set\_pixmap\_align(xpmobj, xfdata.FL\_ALIGN\_CENTER, 10, 10)



\textbf{Status:} Tested + Doc + Demo = OK



    \end{boxedminipage}

    \label{xformslib:flbitmap:fl_set_pixmap_pixmap}
    \index{xformslib \textit{(package)}!xformslib.flbitmap \textit{(module)}!xformslib.flbitmap.fl\_set\_pixmap\_pixmap \textit{(function)}}

    \vspace{0.5ex}

\hspace{.8\funcindent}\begin{boxedminipage}{\funcwidth}

    \raggedright \textbf{fl\_set\_pixmap\_pixmap}(\textit{pFlObject}, \textit{idnum}, \textit{mask})

    \vspace{-1.5ex}

    \rule{\textwidth}{0.5\fboxrule}
\setlength{\parskip}{2ex}
\setlength{\parskip}{1ex}
      \textbf{Parameters}
      \vspace{-1ex}

      \begin{quote}
        \begin{Ventry}{xxxxxxxxx}

          \item[pFlObject]

          pixmap object

            {\it (type=pointer to xfdata.FL\_OBJECT)}

          \item[idnum]

          ?

            {\it (type=long\_pos)}

          \item[mask]

          ?

            {\it (type=long\_pos)}

        \end{Ventry}

      \end{quote}

\textbf{Example:} ??



\textbf{Status:} Untested + NoDoc + NoDemo = NOT OK



    \end{boxedminipage}

    \label{xformslib:flbitmap:fl_set_pixmap_pixmap}
    \index{xformslib \textit{(package)}!xformslib.flbitmap \textit{(module)}!xformslib.flbitmap.fl\_set\_pixmap\_pixmap \textit{(function)}}

    \vspace{0.5ex}

\hspace{.8\funcindent}\begin{boxedminipage}{\funcwidth}

    \raggedright \textbf{fl\_set\_pixmapbutton\_pixmap}(\textit{pFlObject}, \textit{idnum}, \textit{mask})

    \vspace{-1.5ex}

    \rule{\textwidth}{0.5\fboxrule}
\setlength{\parskip}{2ex}
\setlength{\parskip}{1ex}
      \textbf{Parameters}
      \vspace{-1ex}

      \begin{quote}
        \begin{Ventry}{xxxxxxxxx}

          \item[pFlObject]

          pixmap object

            {\it (type=pointer to xfdata.FL\_OBJECT)}

          \item[idnum]

          ?

            {\it (type=long\_pos)}

          \item[mask]

          ?

            {\it (type=long\_pos)}

        \end{Ventry}

      \end{quote}

\textbf{Example:} ??



\textbf{Status:} Untested + NoDoc + NoDemo = NOT OK



    \end{boxedminipage}

    \label{xformslib:flbitmap:fl_set_pixmap_colorcloseness}
    \index{xformslib \textit{(package)}!xformslib.flbitmap \textit{(module)}!xformslib.flbitmap.fl\_set\_pixmap\_colorcloseness \textit{(function)}}

    \vspace{0.5ex}

\hspace{.8\funcindent}\begin{boxedminipage}{\funcwidth}

    \raggedright \textbf{fl\_set\_pixmap\_colorcloseness}(\textit{red}, \textit{green}, \textit{blue})

    \vspace{-1.5ex}

    \rule{\textwidth}{0.5\fboxrule}
\setlength{\parskip}{2ex}
\setlength{\parskip}{1ex}
      \textbf{Parameters}
      \vspace{-1ex}

      \begin{quote}
        \begin{Ventry}{xxxxx}

          \item[red]

          ?

            {\it (type=int)}

          \item[green]

          ?

            {\it (type=int)}

          \item[blue]

          ?

            {\it (type=int)}

        \end{Ventry}

      \end{quote}

\textbf{Example:} ?



\textbf{Status:} Untested + NoDoc + NoDemo = NOT OK



    \end{boxedminipage}

    \label{xformslib:flbitmap:fl_free_pixmap_pixmap}
    \index{xformslib \textit{(package)}!xformslib.flbitmap \textit{(module)}!xformslib.flbitmap.fl\_free\_pixmap\_pixmap \textit{(function)}}

    \vspace{0.5ex}

\hspace{.8\funcindent}\begin{boxedminipage}{\funcwidth}

    \raggedright \textbf{fl\_free\_pixmap\_pixmap}(\textit{pFlObject})

    \vspace{-1.5ex}

    \rule{\textwidth}{0.5\fboxrule}
\setlength{\parskip}{2ex}
\setlength{\parskip}{1ex}
      \textbf{Parameters}
      \vspace{-1ex}

      \begin{quote}
        \begin{Ventry}{xxxxxxxxx}

          \item[pFlObject]

          pixmap object

            {\it (type=pointer to xfdata.FL\_OBJECT)}

        \end{Ventry}

      \end{quote}

\textbf{Example:} ??



\textbf{Status:} Tested + NoDoc + Demo = OK



    \end{boxedminipage}

    \label{xformslib:flbitmap:fl_free_pixmap_pixmap}
    \index{xformslib \textit{(package)}!xformslib.flbitmap \textit{(module)}!xformslib.flbitmap.fl\_free\_pixmap\_pixmap \textit{(function)}}

    \vspace{0.5ex}

\hspace{.8\funcindent}\begin{boxedminipage}{\funcwidth}

    \raggedright \textbf{fl\_free\_pixmapbutton\_pixmap}(\textit{pFlObject})

    \vspace{-1.5ex}

    \rule{\textwidth}{0.5\fboxrule}
\setlength{\parskip}{2ex}
\setlength{\parskip}{1ex}
      \textbf{Parameters}
      \vspace{-1ex}

      \begin{quote}
        \begin{Ventry}{xxxxxxxxx}

          \item[pFlObject]

          pixmap object

            {\it (type=pointer to xfdata.FL\_OBJECT)}

        \end{Ventry}

      \end{quote}

\textbf{Example:} ??



\textbf{Status:} Tested + NoDoc + Demo = OK



    \end{boxedminipage}

    \label{xformslib:flbitmap:fl_get_pixmap_pixmap}
    \index{xformslib \textit{(package)}!xformslib.flbitmap \textit{(module)}!xformslib.flbitmap.fl\_get\_pixmap\_pixmap \textit{(function)}}

    \vspace{0.5ex}

\hspace{.8\funcindent}\begin{boxedminipage}{\funcwidth}

    \raggedright \textbf{fl\_get\_pixmap\_pixmap}(\textit{pFlObject})

    \vspace{-1.5ex}

    \rule{\textwidth}{0.5\fboxrule}
\setlength{\parskip}{2ex}
\setlength{\parskip}{1ex}
      \textbf{Parameters}
      \vspace{-1ex}

      \begin{quote}
        \begin{Ventry}{xxxxxxxxx}

          \item[pFlObject]

          pixmap object

            {\it (type=pointer to xfdata.FL\_OBJECT)}

        \end{Ventry}

      \end{quote}

      \textbf{Return Value}
    \vspace{-1ex}

      \begin{quote}
      pixmap id, pixmap id, pixmap mask

      {\it (type=long\_pos, long\_pos, long\_pos)}

      \end{quote}

\textbf{Example:} ??



\textbf{Attention:} API change from XForms - upstream was fl\_get\_pixmap\_pixmap(pFlObject, p,
m)



\textbf{Status:} Untested + NoDoc + NoDemo = NOT OK



    \end{boxedminipage}

    \label{xformslib:flbitmap:fl_get_pixmap_pixmap}
    \index{xformslib \textit{(package)}!xformslib.flbitmap \textit{(module)}!xformslib.flbitmap.fl\_get\_pixmap\_pixmap \textit{(function)}}

    \vspace{0.5ex}

\hspace{.8\funcindent}\begin{boxedminipage}{\funcwidth}

    \raggedright \textbf{fl\_get\_pixmapbutton\_pixmap}(\textit{pFlObject})

    \vspace{-1.5ex}

    \rule{\textwidth}{0.5\fboxrule}
\setlength{\parskip}{2ex}
\setlength{\parskip}{1ex}
      \textbf{Parameters}
      \vspace{-1ex}

      \begin{quote}
        \begin{Ventry}{xxxxxxxxx}

          \item[pFlObject]

          pixmap object

            {\it (type=pointer to xfdata.FL\_OBJECT)}

        \end{Ventry}

      \end{quote}

      \textbf{Return Value}
    \vspace{-1ex}

      \begin{quote}
      pixmap id, pixmap id, pixmap mask

      {\it (type=long\_pos, long\_pos, long\_pos)}

      \end{quote}

\textbf{Example:} ??



\textbf{Attention:} API change from XForms - upstream was fl\_get\_pixmap\_pixmap(pFlObject, p,
m)



\textbf{Status:} Untested + NoDoc + NoDemo = NOT OK



    \end{boxedminipage}

    \label{xformslib:flbitmap:fl_read_pixmapfile}
    \index{xformslib \textit{(package)}!xformslib.flbitmap \textit{(module)}!xformslib.flbitmap.fl\_read\_pixmapfile \textit{(function)}}

    \vspace{0.5ex}

\hspace{.8\funcindent}\begin{boxedminipage}{\funcwidth}

    \raggedright \textbf{fl\_read\_pixmapfile}(\textit{win}, \textit{filename}, \textit{tran})

    \vspace{-1.5ex}

    \rule{\textwidth}{0.5\fboxrule}
\setlength{\parskip}{2ex}
    Makes a pixmap from a pixmap file.

\setlength{\parskip}{1ex}
      \textbf{Parameters}
      \vspace{-1ex}

      \begin{quote}
        \begin{Ventry}{xxxxxxxx}

          \item[win]

          window id

            {\it (type=long\_pos)}

          \item[filename]

          name of pixmap (.xpm format) file

            {\it (type=str)}

          \item[tran]

          color value

            {\it (type=long\_pos)}

        \end{Ventry}

      \end{quote}

      \textbf{Return Value}
    \vspace{-1ex}

      \begin{quote}
      pixmap, w, h, shapemask, hotx, hoty

      {\it (type=long\_pos, int\_pos, int\_pos, long\_pos, int, int)}

      \end{quote}

\textbf{Example:} pmap, w, h, shapmsk, hotx, hoty = fl\_read\_pixmapfile(win0,         
"xpmfile.xpm", xfdata.FL\_WHITE)



\textbf{Attention:} API change from XForms - upstream was fl\_read\_pixmapfile(win, filename, 
w, h, shape\_mask, hotx, hoty, tran)



\textbf{Status:} Tested + Doc + Demo = OK



    \end{boxedminipage}

    \label{xformslib:flbitmap:fl_create_from_pixmapdata}
    \index{xformslib \textit{(package)}!xformslib.flbitmap \textit{(module)}!xformslib.flbitmap.fl\_create\_from\_pixmapdata \textit{(function)}}

    \vspace{0.5ex}

\hspace{.8\funcindent}\begin{boxedminipage}{\funcwidth}

    \raggedright \textbf{fl\_create\_from\_pixmapdata}(\textit{win}, \textit{data}, \textit{w}, \textit{h}, \textit{sm}, \textit{hotx}, \textit{hoty}, \textit{tran})

    \vspace{-1.5ex}

    \rule{\textwidth}{0.5\fboxrule}
\setlength{\parskip}{2ex}
    Makes a pixmap from pixmap contents data.

\setlength{\parskip}{1ex}
      \textbf{Parameters}
      \vspace{-1ex}

      \begin{quote}
        \begin{Ventry}{xxxx}

          \item[win]

          window

            {\it (type=long\_pos)}

          \item[data]

          bitmap contents data

            {\it (type=str of ubyte)}

          \item[w]

          width of bitmap in coord units

            {\it (type=int\_pos)}

          \item[h]

          height of bitmap in coord units

            {\it (type=int\_pos)}

          \item[sm]

          shape mask

            {\it (type=long\_pos)}

          \item[hotx]

          ?

            {\it (type=int)}

          \item[hoty]

          ?

            {\it (type=int)}

          \item[tran]

          color value

            {\it (type=long\_pos)}

        \end{Ventry}

      \end{quote}

      \textbf{Return Value}
    \vspace{-1ex}

      \begin{quote}
      pixmap created

      {\it (type=long\_pos)}

      \end{quote}

\textbf{Example:} ??



\textbf{Status:} Untested + Doc + NoDemo = NOT OK



    \end{boxedminipage}

    \label{xformslib:flbitmap:fl_free_pixmap}
    \index{xformslib \textit{(package)}!xformslib.flbitmap \textit{(module)}!xformslib.flbitmap.fl\_free\_pixmap \textit{(function)}}

    \vspace{0.5ex}

\hspace{.8\funcindent}\begin{boxedminipage}{\funcwidth}

    \raggedright \textbf{fl\_free\_pixmap}(\textit{idnum})

    \vspace{-1.5ex}

    \rule{\textwidth}{0.5\fboxrule}
\setlength{\parskip}{2ex}
    Frees the pixmap.

\setlength{\parskip}{1ex}
      \textbf{Parameters}
      \vspace{-1ex}

      \begin{quote}
        \begin{Ventry}{xxxxx}

          \item[idnum]

          Pixmap id to be freed

            {\it (type=long\_pos)}

        \end{Ventry}

      \end{quote}

\textbf{Example:} fl\_free\_pixmap(pmap)



\textbf{Status:} Untested + Doc + NoDemo = NOT OK



    \end{boxedminipage}


%%%%%%%%%%%%%%%%%%%%%%%%%%%%%%%%%%%%%%%%%%%%%%%%%%%%%%%%%%%%%%%%%%%%%%%%%%%
%%                               Variables                               %%
%%%%%%%%%%%%%%%%%%%%%%%%%%%%%%%%%%%%%%%%%%%%%%%%%%%%%%%%%%%%%%%%%%%%%%%%%%%

  \subsection{Variables}

    \vspace{-1cm}
\hspace{\varindent}\begin{longtable}{|p{\varnamewidth}|p{\vardescrwidth}|l}
\cline{1-2}
\cline{1-2} \centering \textbf{Name} & \centering \textbf{Description}& \\
\cline{1-2}
\endhead\cline{1-2}\multicolumn{3}{r}{\small\textit{continued on next page}}\\\endfoot\cline{1-2}
\endlastfoot\raggedright \_\-\_\-p\-a\-c\-k\-a\-g\-e\-\_\-\_\- & \raggedright \textbf{Value:} 
{\tt \texttt{'}\texttt{xformslib}\texttt{'}}&\\
\cline{1-2}
\end{longtable}

    \index{xformslib \textit{(package)}!xformslib.flbitmap \textit{(module)}|)}
