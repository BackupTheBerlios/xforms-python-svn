%
% API Documentation for API Documentation
% Module xformslib.flcanvas
%
% Generated by epydoc 3.0.1
% [Fri May 21 19:29:31 2010]
%

%%%%%%%%%%%%%%%%%%%%%%%%%%%%%%%%%%%%%%%%%%%%%%%%%%%%%%%%%%%%%%%%%%%%%%%%%%%
%%                          Module Description                           %%
%%%%%%%%%%%%%%%%%%%%%%%%%%%%%%%%%%%%%%%%%%%%%%%%%%%%%%%%%%%%%%%%%%%%%%%%%%%

    \index{xformslib \textit{(package)}!xformslib.flcanvas \textit{(module)}|(}
\section{Module xformslib.flcanvas}

    \label{xformslib:flcanvas}

xforms-python's functions to manage canvas objects.

Copyright (C) 2009, 2010  Luca Lazzaroni ``LukenShiro''
e-mail: <\href{mailto:lukenshiro@ngi.it}{lukenshiro@ngi.it}>

This program is free software: you can redistribute it and/or modify
it under the terms of the GNU Lesser General Public License as
published by the Free Software Foundation, version 2.1 of the License.

This program is distributed in the hope that it will be useful,
but WITHOUT ANY WARRANTY; without even the implied warranty of
MERCHANTABILITY or FITNESS FOR A PARTICULAR PURPOSE. See the
GNU Lesser General Public License for more details.

You should have received a copy of the GNU LGPL along with this
program. If not, see <\href{http://www.gnu.org/licenses/}{http://www.gnu.org/licenses/}>.

See CREDITS file to read acknowledgements and thanks to XForms,
ctypes and other developers.

%%%%%%%%%%%%%%%%%%%%%%%%%%%%%%%%%%%%%%%%%%%%%%%%%%%%%%%%%%%%%%%%%%%%%%%%%%%
%%                               Functions                               %%
%%%%%%%%%%%%%%%%%%%%%%%%%%%%%%%%%%%%%%%%%%%%%%%%%%%%%%%%%%%%%%%%%%%%%%%%%%%

  \subsection{Functions}

    \label{xformslib:flcanvas:fl_create_generic_canvas}
    \index{xformslib \textit{(package)}!xformslib.flcanvas \textit{(module)}!xformslib.flcanvas.fl\_create\_generic\_canvas \textit{(function)}}

    \vspace{0.5ex}

\hspace{.8\funcindent}\begin{boxedminipage}{\funcwidth}

    \raggedright \textbf{fl\_create\_generic\_canvas}(\textit{canvasclass}, \textit{canvastype}, \textit{x}, \textit{y}, \textit{w}, \textit{h}, \textit{label})

    \vspace{-1.5ex}

    \rule{\textwidth}{0.5\fboxrule}
\setlength{\parskip}{2ex}

Creates a generic canvas object.

-{}-
\setlength{\parskip}{1ex}
      \textbf{Parameters}
      \vspace{-1ex}

      \begin{quote}
        \begin{Ventry}{xxxxxxxxxxx}

          \item[canvasclass]


value of a new canvas class
            {\it (type=int)}

          \item[canvastype]


type of canvas to be created. Values (from xfdata.py) FL\_NORMAL\_CANVAS,
FL\_SCROLLED\_CANVAS (not enabled)
            {\it (type=int)}

          \item[x]


horizontal position (upper-left corner)
            {\it (type=int)}

          \item[y]


vertical position (upper-left corner)
            {\it (type=int)}

          \item[w]


width in coord units
            {\it (type=int)}

          \item[h]


height in coord units
            {\it (type=int)}

          \item[label]


text label of canvas
            {\it (type=str)}

        \end{Ventry}

      \end{quote}

      \textbf{Return Value}
    \vspace{-1ex}

      \begin{quote}

canvas object created (pFlObject)
      {\it (type=pointer to xfdata.FL\_OBJECT)}

      \end{quote}

\textbf{Note:} 
e.g. \emph{todo}


\textbf{Status:} 
Untested + Doc + NoDemo = NOT OK


    \end{boxedminipage}

    \label{xformslib:flcanvas:fl_add_canvas}
    \index{xformslib \textit{(package)}!xformslib.flcanvas \textit{(module)}!xformslib.flcanvas.fl\_add\_canvas \textit{(function)}}

    \vspace{0.5ex}

\hspace{.8\funcindent}\begin{boxedminipage}{\funcwidth}

    \raggedright \textbf{fl\_add\_canvas}(\textit{canvastype}, \textit{x}, \textit{y}, \textit{w}, \textit{h}, \textit{label})

    \vspace{-1.5ex}

    \rule{\textwidth}{0.5\fboxrule}
\setlength{\parskip}{2ex}

Adds a canvas object.

-{}-
\setlength{\parskip}{1ex}
      \textbf{Parameters}
      \vspace{-1ex}

      \begin{quote}
        \begin{Ventry}{xxxxxxxxxx}

          \item[canvastype]


type of canvas to be added. Values (from xfdata.py) FL\_NORMAL\_CANVAS,
FL\_SCROLLED\_CANVAS (not enabled)
            {\it (type=int)}

          \item[x]


horizontal position (upper-left corner)
            {\it (type=int)}

          \item[y]


vertical position (upper-left corner)
            {\it (type=int)}

          \item[w]


width in coord units
            {\it (type=int)}

          \item[h]


height in coord units
            {\it (type=int)}

          \item[label]


text label of canvas
            {\it (type=str)}

        \end{Ventry}

      \end{quote}

      \textbf{Return Value}
    \vspace{-1ex}

      \begin{quote}

canvas object added (pFlObject)
      {\it (type=pointer to xfdata.FL\_OBJECT)}

      \end{quote}

\textbf{Note:} 
e.g. canvobj = fl\_add\_canvas(xfdata.FL\_NORMAL\_CANVAS, 150, 210,
320, 200, ``My Canvas'')


\textbf{Status:} 
Tested + Doc + NoDemo = OK


    \end{boxedminipage}

    \label{xformslib:flcanvas:fl_set_canvas_colormap}
    \index{xformslib \textit{(package)}!xformslib.flcanvas \textit{(module)}!xformslib.flcanvas.fl\_set\_canvas\_colormap \textit{(function)}}

    \vspace{0.5ex}

\hspace{.8\funcindent}\begin{boxedminipage}{\funcwidth}

    \raggedright \textbf{fl\_set\_canvas\_colormap}(\textit{pFlObject}, \textit{colormap})

    \vspace{-1.5ex}

    \rule{\textwidth}{0.5\fboxrule}
\setlength{\parskip}{2ex}

Sets the color property of canvas. Caution: when the canvas window goes
away, e.g. as a result of a call of fl\_hide\_form(), the colormap associated
with the canvas is freed (destroyed); this likely will cause problems if a
single colormap is used for multiple canvases as each canvas will attempt
to free the same colormap, resulting in an X error.

-{}-
\setlength{\parskip}{1ex}
      \textbf{Parameters}
      \vspace{-1ex}

      \begin{quote}
        \begin{Ventry}{xxxxxxxxx}

          \item[pFlObject]


canvas object
            {\it (type=pointer to xfdata.FL\_OBJECT)}

          \item[colormap]


colormap of canvas
            {\it (type=long\_pos)}

        \end{Ventry}

      \end{quote}

\textbf{Note:} 
e.g. \emph{todo}


\textbf{Status:} 
Untested + Doc + NoDemo = NOT OK


    \end{boxedminipage}

    \label{xformslib:flcanvas:fl_set_canvas_visual}
    \index{xformslib \textit{(package)}!xformslib.flcanvas \textit{(module)}!xformslib.flcanvas.fl\_set\_canvas\_visual \textit{(function)}}

    \vspace{0.5ex}

\hspace{.8\funcindent}\begin{boxedminipage}{\funcwidth}

    \raggedright \textbf{fl\_set\_canvas\_visual}(\textit{pFlObject}, \textit{pVisual})

    \vspace{-1.5ex}

    \rule{\textwidth}{0.5\fboxrule}
\setlength{\parskip}{2ex}
%
\begin{description}
\item[{Sets visual property of canvas. Changing visual does not generally make}] \leavevmode 
sense once the canvas window is created (which happens when the parent
form is shown). Also, typically if you change the canvas visual, you
probably should also change the canvas depth to match the visual.

\end{description}

-{}-
\setlength{\parskip}{1ex}
      \textbf{Parameters}
      \vspace{-1ex}

      \begin{quote}
        \begin{Ventry}{xxxxxxxxx}

          \item[pFlObject]


canvas object
            {\it (type=pointer to xfdata.FL\_OBJECT)}

          \item[pVisual]


class instance
            {\it (type=pointer to xfdata.Visual)}

        \end{Ventry}

      \end{quote}

\textbf{Note:} 
e.g. \emph{todo}


\textbf{Status:} 
Untested + Doc + NoDemo = NOT OK


    \end{boxedminipage}

    \label{xformslib:flcanvas:fl_set_canvas_depth}
    \index{xformslib \textit{(package)}!xformslib.flcanvas \textit{(module)}!xformslib.flcanvas.fl\_set\_canvas\_depth \textit{(function)}}

    \vspace{0.5ex}

\hspace{.8\funcindent}\begin{boxedminipage}{\funcwidth}

    \raggedright \textbf{fl\_set\_canvas\_depth}(\textit{pFlObject}, \textit{depth})

    \vspace{-1.5ex}

    \rule{\textwidth}{0.5\fboxrule}
\setlength{\parskip}{2ex}

Sets the depth of canvas object. Changing depth does not generally make
sense once the canvas window is created (which happens when the parent
form is shown).

-{}-
\setlength{\parskip}{1ex}
      \textbf{Parameters}
      \vspace{-1ex}

      \begin{quote}
        \begin{Ventry}{xxxxxxxxx}

          \item[pFlObject]


canvas object
            {\it (type=pointer to xfdata.FL\_OBJECT)}

          \item[depth]


depth value of canvas. Values (from xfdata.py) e.g. 8, 16, 24?, 32, ...
            {\it (type=int)}

        \end{Ventry}

      \end{quote}

\textbf{Note:} 
e.g. fl\_set\_canvas\_depth(canvobj, 32)


\textbf{Status:} 
Tested + Doc + NoDemo = OK


    \end{boxedminipage}

    \label{xformslib:flcanvas:fl_set_canvas_attributes}
    \index{xformslib \textit{(package)}!xformslib.flcanvas \textit{(module)}!xformslib.flcanvas.fl\_set\_canvas\_attributes \textit{(function)}}

    \vspace{0.5ex}

\hspace{.8\funcindent}\begin{boxedminipage}{\funcwidth}

    \raggedright \textbf{fl\_set\_canvas\_attributes}(\textit{pFlObject}, \textit{mask}, \textit{pXSetWindowAttributes})

    \vspace{-1.5ex}

    \rule{\textwidth}{0.5\fboxrule}
\setlength{\parskip}{2ex}
%
\begin{description}
\item[{Modifies attributes of a canvas object (e.g. visual, depth and}] \leavevmode 
colormap etc.). By default, upon canvas creation, all its window related
attributes are inherited from its parent (i.e. the window of the form the
canvas belongs to). You should not use this function to modify events.

\end{description}

-{}-
\setlength{\parskip}{1ex}
      \textbf{Parameters}
      \vspace{-1ex}

      \begin{quote}
        \begin{Ventry}{xxxxxxxxxxxxxxxxxxxxx}

          \item[pFlObject]


canvas object
            {\it (type=pointer to xfdata.FL\_OBJECT)}

          \item[mask]


mask num.
            {\it (type=int\_pos)}

          \item[pXSetWindowAttributes]


class instance
            {\it (type=pointer to xfdata.XSetWindowAttributes)}

        \end{Ventry}

      \end{quote}

\textbf{Note:} 
e.g. \emph{todo}


\textbf{Status:} 
Untested + Doc + NoDemo = NOT OK


    \end{boxedminipage}

    \label{xformslib:flcanvas:fl_add_canvas_handler}
    \index{xformslib \textit{(package)}!xformslib.flcanvas \textit{(module)}!xformslib.flcanvas.fl\_add\_canvas\_handler \textit{(function)}}

    \vspace{0.5ex}

\hspace{.8\funcindent}\begin{boxedminipage}{\funcwidth}

    \raggedright \textbf{fl\_add\_canvas\_handler}(\textit{pFlObject}, \textit{xev}, \textit{py\_HandleCanvas}, \textit{udata})

    \vspace{-1.5ex}

    \rule{\textwidth}{0.5\fboxrule}
\setlength{\parskip}{2ex}

Sets a callback to be invoked for a specific X event.

-{}-
\setlength{\parskip}{1ex}
      \textbf{Parameters}
      \vspace{-1ex}

      \begin{quote}
        \begin{Ventry}{xxxxxxxxxxxxxxx}

          \item[pFlObject]


canvas object
            {\it (type=pointer to xfdata.FL\_OBJECT)}

          \item[xev]


X event number. Values (from X11): Expose, etc.. ??
            {\it (type=int)}

          \item[py\_HandleCanvas]


name referring to function(pFlObject, win, num, num, pXEvent,
vdata) -> num
            {\it (type=python function to handle canvas, returning value)}

        \end{Ventry}

      \end{quote}

      \textbf{Return Value}
    \vspace{-1ex}

      \begin{quote}

old canvas handler function
      {\it (type=xfdata.FL\_HANDLE\_CANVAS)}

      \end{quote}

\textbf{Note:} 
e.g. \emph{todo}


\textbf{Status:} 
Untested + Doc + NoDemo = NOT OK


    \end{boxedminipage}

    \label{xformslib:flcanvas:fl_get_canvas_id}
    \index{xformslib \textit{(package)}!xformslib.flcanvas \textit{(module)}!xformslib.flcanvas.fl\_get\_canvas\_id \textit{(function)}}

    \vspace{0.5ex}

\hspace{.8\funcindent}\begin{boxedminipage}{\funcwidth}

    \raggedright \textbf{fl\_get\_canvas\_id}(\textit{pFlObject})

    \vspace{-1.5ex}

    \rule{\textwidth}{0.5\fboxrule}
\setlength{\parskip}{2ex}

Returns the window id of the canvas object.

-{}-
\setlength{\parskip}{1ex}
      \textbf{Parameters}
      \vspace{-1ex}

      \begin{quote}
        \begin{Ventry}{xxxxxxxxx}

          \item[pFlObject]


canvas object
            {\it (type=pointer to xfdata.FL\_OBJECT)}

        \end{Ventry}

      \end{quote}

      \textbf{Return Value}
    \vspace{-1ex}

      \begin{quote}

window id (win)
      {\it (type=long\_pos)}

      \end{quote}

\textbf{Note:} 
e.g. canvwin = fl\_get\_canvas\_id(pobj)


\textbf{Status:} 
Tested + Doc + NoDemo = OK


    \end{boxedminipage}

    \label{xformslib:flcanvas:fl_get_canvas_colormap}
    \index{xformslib \textit{(package)}!xformslib.flcanvas \textit{(module)}!xformslib.flcanvas.fl\_get\_canvas\_colormap \textit{(function)}}

    \vspace{0.5ex}

\hspace{.8\funcindent}\begin{boxedminipage}{\funcwidth}

    \raggedright \textbf{fl\_get\_canvas\_colormap}(\textit{pFlObject})

    \vspace{-1.5ex}

    \rule{\textwidth}{0.5\fboxrule}
\setlength{\parskip}{2ex}

Obtains the colormap of a canvas object.

-{}-
\setlength{\parskip}{1ex}
      \textbf{Parameters}
      \vspace{-1ex}

      \begin{quote}
        \begin{Ventry}{xxxxxxxxx}

          \item[pFlObject]


canvas object
            {\it (type=pointer to xfdata.FL\_OBJECT)}

        \end{Ventry}

      \end{quote}

      \textbf{Return Value}
    \vspace{-1ex}

      \begin{quote}

colormap
      {\it (type=long\_pos)}

      \end{quote}

\textbf{Note:} 
e.g. \emph{todo}


\textbf{Status:} 
Untested + Doc + NoDemo = NOT OK


    \end{boxedminipage}

    \label{xformslib:flcanvas:fl_get_canvas_depth}
    \index{xformslib \textit{(package)}!xformslib.flcanvas \textit{(module)}!xformslib.flcanvas.fl\_get\_canvas\_depth \textit{(function)}}

    \vspace{0.5ex}

\hspace{.8\funcindent}\begin{boxedminipage}{\funcwidth}

    \raggedright \textbf{fl\_get\_canvas\_depth}(\textit{pFlObject})

    \vspace{-1.5ex}

    \rule{\textwidth}{0.5\fboxrule}
\setlength{\parskip}{2ex}

Obtains the depth of a canvas object (e.g. 8, 16, 24?, 32 ..).

-{}-
\setlength{\parskip}{1ex}
      \textbf{Parameters}
      \vspace{-1ex}

      \begin{quote}
        \begin{Ventry}{xxxxxxxxx}

          \item[pFlObject]


canvas object
            {\it (type=pointer to xfdata.FL\_OBJECT)}

        \end{Ventry}

      \end{quote}

      \textbf{Return Value}
    \vspace{-1ex}

      \begin{quote}

depth num.
      {\it (type=int)}

      \end{quote}

\textbf{Note:} 
e.g. canvdph = fl\_get\_canvas\_depth(canvobj)


\textbf{Status:} 
Tested + Doc + NoDemo = OK


    \end{boxedminipage}

    \label{xformslib:flcanvas:fl_remove_canvas_handler}
    \index{xformslib \textit{(package)}!xformslib.flcanvas \textit{(module)}!xformslib.flcanvas.fl\_remove\_canvas\_handler \textit{(function)}}

    \vspace{0.5ex}

\hspace{.8\funcindent}\begin{boxedminipage}{\funcwidth}

    \raggedright \textbf{fl\_remove\_canvas\_handler}(\textit{pFlObject}, \textit{xev}, \textit{py\_HandleCanvas})

    \vspace{-1.5ex}

    \rule{\textwidth}{0.5\fboxrule}
\setlength{\parskip}{2ex}

Removes a particular handler for a specified X event.

-{}-
\setlength{\parskip}{1ex}
      \textbf{Parameters}
      \vspace{-1ex}

      \begin{quote}
        \begin{Ventry}{xxxxxxxxxxxxxxx}

          \item[pFlObject]


canvas object
            {\it (type=pointer to xfdata.FL\_OBJECT)}

          \item[xev]


X event number. If it is invalid, removes all handlers and their
corresponding event mask.
            {\it (type=int)}

          \item[py\_HandleCanvas]


name referring to  function(pFlObject, win, num, num, pXEvent,
vdata) -> num
            {\it (type=python function to handle canvas)}

        \end{Ventry}

      \end{quote}

\textbf{Note:} 
e.g. \emph{todo}


\textbf{Status:} 
Untested + Doc + NoDemo = NOT OK


    \end{boxedminipage}

    \label{xformslib:flcanvas:fl_hide_canvas}
    \index{xformslib \textit{(package)}!xformslib.flcanvas \textit{(module)}!xformslib.flcanvas.fl\_hide\_canvas \textit{(function)}}

    \vspace{0.5ex}

\hspace{.8\funcindent}\begin{boxedminipage}{\funcwidth}

    \raggedright \textbf{fl\_hide\_canvas}(\textit{pFlObject})

    \vspace{-1.5ex}

    \rule{\textwidth}{0.5\fboxrule}
\setlength{\parskip}{2ex}

Hides a canvas object.

-{}-
\setlength{\parskip}{1ex}
      \textbf{Parameters}
      \vspace{-1ex}

      \begin{quote}
        \begin{Ventry}{xxxxxxxxx}

          \item[pFlObject]


canvas object
            {\it (type=pointer to xfdata.FL\_OBJECT)}

        \end{Ventry}

      \end{quote}

\textbf{Note:} 
e.g. fl\_hide\_canvas(canvobj)


\textbf{Status:} 
Tested + Doc + NoDemo = OK


    \end{boxedminipage}

    \label{xformslib:flcanvas:fl_share_canvas_colormap}
    \index{xformslib \textit{(package)}!xformslib.flcanvas \textit{(module)}!xformslib.flcanvas.fl\_share\_canvas\_colormap \textit{(function)}}

    \vspace{0.5ex}

\hspace{.8\funcindent}\begin{boxedminipage}{\funcwidth}

    \raggedright \textbf{fl\_share\_canvas\_colormap}(\textit{pFlObject}, \textit{colormap})

    \vspace{-1.5ex}

    \rule{\textwidth}{0.5\fboxrule}
\setlength{\parskip}{2ex}

Sets the color property of canvas. It also sets a internal flag so the
colormap isn't freed when the canvas goes away.

-{}-
\setlength{\parskip}{1ex}
      \textbf{Parameters}
      \vspace{-1ex}

      \begin{quote}
        \begin{Ventry}{xxxxxxxxx}

          \item[pFlObject]


canvas object
            {\it (type=pointer to xfdata.FL\_OBJECT)}

          \item[colormap]


color map
            {\it (type=long\_pos)}

        \end{Ventry}

      \end{quote}

\textbf{Note:} 
e.g. \emph{todo}


\textbf{Status:} 
Untested + Doc + NoDemo = NOT OK


    \end{boxedminipage}

    \label{xformslib:flcanvas:fl_clear_canvas}
    \index{xformslib \textit{(package)}!xformslib.flcanvas \textit{(module)}!xformslib.flcanvas.fl\_clear\_canvas \textit{(function)}}

    \vspace{0.5ex}

\hspace{.8\funcindent}\begin{boxedminipage}{\funcwidth}

    \raggedright \textbf{fl\_clear\_canvas}(\textit{pFlObject})

    \vspace{-1.5ex}

    \rule{\textwidth}{0.5\fboxrule}
\setlength{\parskip}{2ex}

Clears the canvas to the background color. If no background is
defined uses black.

-{}-
\setlength{\parskip}{1ex}
      \textbf{Parameters}
      \vspace{-1ex}

      \begin{quote}
        \begin{Ventry}{xxxxxxxxx}

          \item[pFlObject]


canvas object
            {\it (type=pointer to xfdata.FL\_OBJECT)}

        \end{Ventry}

      \end{quote}

\textbf{Note:} 
e.g. fl\_clear\_canvas(canvobj)


\textbf{Status:} 
Tested + Doc + NoDemo = OK


    \end{boxedminipage}

    \label{xformslib:flcanvas:fl_modify_canvas_prop}
    \index{xformslib \textit{(package)}!xformslib.flcanvas \textit{(module)}!xformslib.flcanvas.fl\_modify\_canvas\_prop \textit{(function)}}

    \vspace{0.5ex}

\hspace{.8\funcindent}\begin{boxedminipage}{\funcwidth}

    \raggedright \textbf{fl\_modify\_canvas\_prop}(\textit{pFlObject}, \textit{py\_initModifyCanvasProp}, \textit{py\_activateModifyCanvasProp}, \textit{py\_cleanupModifyCanvasProp})

    \vspace{-1.5ex}

    \rule{\textwidth}{0.5\fboxrule}
\setlength{\parskip}{2ex}

Modifies init, activate and cleanup handler properties of a canvas
object.

-{}-
\setlength{\parskip}{1ex}
      \textbf{Parameters}
      \vspace{-1ex}

      \begin{quote}
        \begin{Ventry}{xxxxxxxxxxxxxxxxxxxxxxxxxxx}

          \item[pFlObject]


canvas object
            {\it (type=pointer to xfdata.FL\_OBJECT)}

          \item[py\_initModifyCanvasProp]


name referring to function(pFlObject) -> num.
            {\it (type=python function callback, returned value)}

          \item[py\_activateModifyCanvasProp]


name referring to function(pFlObject) -> num.
            {\it (type=python function callback, returned value)}

          \item[py\_cleanupModifyCanvasProp]


name referring to function(pFlObject) -> num.
            {\it (type=python function callback, returned value)}

        \end{Ventry}

      \end{quote}

\textbf{Note:} 
e.g. \emph{todo}


\textbf{Status:} 
Untested + Doc + NoDemo = NOT OK


    \end{boxedminipage}

    \label{xformslib:flcanvas:fl_canvas_yield_to_shortcut}
    \index{xformslib \textit{(package)}!xformslib.flcanvas \textit{(module)}!xformslib.flcanvas.fl\_canvas\_yield\_to\_shortcut \textit{(function)}}

    \vspace{0.5ex}

\hspace{.8\funcindent}\begin{boxedminipage}{\funcwidth}

    \raggedright \textbf{fl\_canvas\_yield\_to\_shortcut}(\textit{pFlObject}, \textit{yesno})

    \vspace{-1.5ex}

    \rule{\textwidth}{0.5\fboxrule}
\setlength{\parskip}{2ex}

Enables or disables keyboard input's stealing by canvas. By default,
objects with shortcuts appearing on the same form as the canvas will
``steal'' keyboard inputs if they match the shortcuts.

-{}-
\setlength{\parskip}{1ex}
      \textbf{Parameters}
      \vspace{-1ex}

      \begin{quote}
        \begin{Ventry}{xxxxxxxxx}

          \item[pFlObject]


canvas object
            {\it (type=pointer to xfdata.FL\_OBJECT)}

          \item[yesno]


flag to enable/disable keyboard inputs stealing by canvas. Values 0
(to disable) or 1 (to enable)
            {\it (type=int)}

        \end{Ventry}

      \end{quote}

\textbf{Note:} 
e.g. \emph{todo}


\textbf{Status:} 
Untested + Doc + NoDemo = NOT OK


    \end{boxedminipage}

    \index{xformslib \textit{(package)}!xformslib.flcanvas \textit{(module)}|)}
