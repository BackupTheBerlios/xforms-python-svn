%
% API Documentation for API Documentation
% Module xformslib.flpopup
%
% Generated by epydoc 3.0.1
% [Tue May 18 00:48:12 2010]
%

%%%%%%%%%%%%%%%%%%%%%%%%%%%%%%%%%%%%%%%%%%%%%%%%%%%%%%%%%%%%%%%%%%%%%%%%%%%
%%                          Module Description                           %%
%%%%%%%%%%%%%%%%%%%%%%%%%%%%%%%%%%%%%%%%%%%%%%%%%%%%%%%%%%%%%%%%%%%%%%%%%%%

    \index{xformslib \textit{(package)}!xformslib.flpopup \textit{(module)}|(}
\section{Module xformslib.flpopup}

    \label{xformslib:flpopup}

xforms-python's functions to manage popups.

Copyright (C) 2009, 2010  Luca Lazzaroni ``LukenShiro''
e-mail: <\href{mailto:lukenshiro@ngi.it}{lukenshiro@ngi.it}>

This program is free software: you can redistribute it and/or modify
it under the terms of the GNU Lesser General Public License as
published by the Free Software Foundation, version 2.1 of the License.

This program is distributed in the hope that it will be useful,
but WITHOUT ANY WARRANTY; without even the implied warranty of
MERCHANTABILITY or FITNESS FOR A PARTICULAR PURPOSE. See the
GNU Lesser General Public License for more details.

You should have received a copy of the GNU LGPL along with this
program. If not, see <\href{http://www.gnu.org/licenses/}{http://www.gnu.org/licenses/}>.

See CREDITS file to read acknowledgements and thanks to XForms,
ctypes and other developers.

%%%%%%%%%%%%%%%%%%%%%%%%%%%%%%%%%%%%%%%%%%%%%%%%%%%%%%%%%%%%%%%%%%%%%%%%%%%
%%                               Functions                               %%
%%%%%%%%%%%%%%%%%%%%%%%%%%%%%%%%%%%%%%%%%%%%%%%%%%%%%%%%%%%%%%%%%%%%%%%%%%%

  \subsection{Functions}

    \label{xformslib:flpopup:fl_popup_add}
    \index{xformslib \textit{(package)}!xformslib.flpopup \textit{(module)}!xformslib.flpopup.fl\_popup\_add \textit{(function)}}

    \vspace{0.5ex}

\hspace{.8\funcindent}\begin{boxedminipage}{\funcwidth}

    \raggedright \textbf{fl\_popup\_add}(\textit{win}, \textit{title})

    \vspace{-1.5ex}

    \rule{\textwidth}{0.5\fboxrule}
\setlength{\parskip}{2ex}

Defines a new popup. There is no built-in limit to the number
of popups that can be created.

-{}-
\setlength{\parskip}{1ex}
      \textbf{Parameters}
      \vspace{-1ex}

      \begin{quote}
        \begin{Ventry}{xxxxx}

          \item[win]


window of a parent object; use flxbasic.FL\_ObjWin() to find out about
it. You can also use either 'fl\_root' or 'None' for the root window.
            {\it (type=long\_pos)}

          \item[title]


text of title that gets shown at the top of the popup in a frame. If
not wanted, pass an empty string or 'None'. It may contain embedded
newline characters, this allows to create titles that span more than
one line.
            {\it (type=str)}

        \end{Ventry}

      \end{quote}

      \textbf{Return Value}
    \vspace{-1ex}

      \begin{quote}

new popup, or None (on failure)
      {\it (type=pointer to xfdata.FL\_POPUP)}

      \end{quote}

\textbf{Note:} 
e.g. \emph{todo}


\textbf{Status:} 
Untested + NoDoc + NoDemo = NOT OK


    \end{boxedminipage}

    \label{xformslib:flpopup:fl_popup_add_entries}
    \index{xformslib \textit{(package)}!xformslib.flpopup \textit{(module)}!xformslib.flpopup.fl\_popup\_add\_entries \textit{(function)}}

    \vspace{0.5ex}

\hspace{.8\funcindent}\begin{boxedminipage}{\funcwidth}

    \raggedright \textbf{fl\_popup\_add\_entries}(\textit{pPopup}, \textit{entrytxt})

    \vspace{-1.5ex}

    \rule{\textwidth}{0.5\fboxrule}
\setlength{\parskip}{2ex}

Adds one or more entries to a popup.

-{}-
\setlength{\parskip}{1ex}
      \textbf{Parameters}
      \vspace{-1ex}

      \begin{quote}
        \begin{Ventry}{xxxxxxxx}

          \item[pPopup]


popup class instance
            {\it (type=pointer to xfdata.FL\_POPUP)}

          \item[entrytxt]


text of the entry to be added. It may contain newline characters which
allows to create entries that span more than a single line (among
special sequences only \%S is supported)
            {\it (type=str)}

        \end{Ventry}

      \end{quote}

      \textbf{Return Value}
    \vspace{-1ex}

      \begin{quote}

popup entry, or None (on failure)
      {\it (type=pointer to xfdata.FL\_POPUP)}

      \end{quote}

\textbf{Note:} 
e.g. \emph{todo}


\textbf{Status:} 
Untested + NoDoc + NoDemo = NOT OK


    \end{boxedminipage}

    \label{xformslib:flpopup:fl_popup_insert_entries}
    \index{xformslib \textit{(package)}!xformslib.flpopup \textit{(module)}!xformslib.flpopup.fl\_popup\_insert\_entries \textit{(function)}}

    \vspace{0.5ex}

\hspace{.8\funcindent}\begin{boxedminipage}{\funcwidth}

    \raggedright \textbf{fl\_popup\_insert\_entries}(\textit{pPopup}, \textit{pPopupEntry}, \textit{entrytxt})

    \vspace{-1.5ex}

    \rule{\textwidth}{0.5\fboxrule}
\setlength{\parskip}{2ex}

Inserts one or more entries into a popup.

-{}-
\setlength{\parskip}{1ex}
      \textbf{Parameters}
      \vspace{-1ex}

      \begin{quote}
        \begin{Ventry}{xxxxxxxxxxx}

          \item[pPopup]


popup class instance
            {\it (type=pointer to xfdata.FL\_POPUP)}

          \item[pPopupEntry]


popup entry after which entry is inserted. If it's 'None', it inserts
items at the very start.
            {\it (type=pointer to xfdata.FL\_POPUP\_ENTRY)}

          \item[entrytxt]


text of the entry to be added. It may contain newline characters which
allows to create entries that span more than a single line (among
special sequences only \%S is supported)
            {\it (type=str)}

        \end{Ventry}

      \end{quote}

      \textbf{Return Value}
    \vspace{-1ex}

      \begin{quote}

popup entry inserted, or None (on failure)
      {\it (type=pointer to xfdata.FL\_POPUP\_ENTRY)}

      \end{quote}

\textbf{Note:} 
e.g. \emph{todo}


\textbf{Status:} 
Untested + NoDoc + NoDemo = NOT OK


    \end{boxedminipage}

    \label{xformslib:flpopup:fl_popup_create}
    \index{xformslib \textit{(package)}!xformslib.flpopup \textit{(module)}!xformslib.flpopup.fl\_popup\_create \textit{(function)}}

    \vspace{0.5ex}

\hspace{.8\funcindent}\begin{boxedminipage}{\funcwidth}

    \raggedright \textbf{fl\_popup\_create}(\textit{win}, \textit{title}, \textit{pPopupItem})

    \vspace{-1.5ex}

    \rule{\textwidth}{0.5\fboxrule}
\setlength{\parskip}{2ex}

Creates a popup. It doesn't allow to associate values or pointers to
user data to individual entries, set titles for sub-popups, have radio
entries belong to different groups or set enter or leave callback
functions.

-{}-
\setlength{\parskip}{1ex}
      \textbf{Parameters}
      \vspace{-1ex}

      \begin{quote}
        \begin{Ventry}{xxxxxxxxxx}

          \item[win]


window of a parent object (use flxbasic.FL\_ObjWin() to find out about
it). You can also use either 'fl\_root' or 'None' for the root window.
            {\it (type=long\_pos)}

          \item[title]


text of title that gets shown at the top of the popup in a frame. If
not wanted, pass an empty string or 'None'. It may contain embedded
newline characters, this allows to create titles that span more than
one line.
            {\it (type=str)}

          \item[pPopupItem]


popup item
            {\it (type=pointer to xfdata.FL\_POPUP\_ITEM)}

        \end{Ventry}

      \end{quote}

      \textbf{Return Value}
    \vspace{-1ex}

      \begin{quote}

popup created, or None (on failure)
      {\it (type=pointer to xfdata.FL\_POPUP)}

      \end{quote}

\textbf{Note:} 
e.g. \emph{todo}


\textbf{Status:} 
Untested + NoDoc + NoDemo = NOT OK


    \end{boxedminipage}

    \label{xformslib:flpopup:fl_popup_add_items}
    \index{xformslib \textit{(package)}!xformslib.flpopup \textit{(module)}!xformslib.flpopup.fl\_popup\_add\_items \textit{(function)}}

    \vspace{0.5ex}

\hspace{.8\funcindent}\begin{boxedminipage}{\funcwidth}

    \raggedright \textbf{fl\_popup\_add\_items}(\textit{pPopup}, \textit{pPopupItem})

    \vspace{-1.5ex}

    \rule{\textwidth}{0.5\fboxrule}
\setlength{\parskip}{2ex}

Adds one or more items to a popup.

-{}-
\setlength{\parskip}{1ex}
      \textbf{Parameters}
      \vspace{-1ex}

      \begin{quote}
        \begin{Ventry}{xxxxxxxxxx}

          \item[pPopup]


popup class instance
            {\it (type=pointer to xfdata.FL\_POPUP)}

          \item[pPopupItem]


popup item
            {\it (type=pointer to xfdata.FL\_POPUP\_ITEM)}

        \end{Ventry}

      \end{quote}

      \textbf{Return Value}
    \vspace{-1ex}

      \begin{quote}

popup entry, or None
      {\it (type=pointer to xfdata.FL\_POPUP\_ENTRY)}

      \end{quote}

\textbf{Note:} 
e.g. \emph{todo}


\textbf{Status:} 
Untested + NoDoc + NoDemo = NOT OK


    \end{boxedminipage}

    \label{xformslib:flpopup:fl_popup_insert_items}
    \index{xformslib \textit{(package)}!xformslib.flpopup \textit{(module)}!xformslib.flpopup.fl\_popup\_insert\_items \textit{(function)}}

    \vspace{0.5ex}

\hspace{.8\funcindent}\begin{boxedminipage}{\funcwidth}

    \raggedright \textbf{fl\_popup\_insert\_items}(\textit{pPopup}, \textit{pPopupEntry}, \textit{pPopupItem})

    \vspace{-1.5ex}

    \rule{\textwidth}{0.5\fboxrule}
\setlength{\parskip}{2ex}

Inserts entries into a popup.

-{}-
\setlength{\parskip}{1ex}
      \textbf{Parameters}
      \vspace{-1ex}

      \begin{quote}
        \begin{Ventry}{xxxxxxxxxxx}

          \item[pPopup]


popup class instance
            {\it (type=pointer to xfdata.FL\_POPUP)}

          \item[pPopupEntry]


popup entry after which items are inserted. If it's 'None', items are
inserted at the very start.
            {\it (type=pointer to xfdata.FL\_POPUP\_ENTRY)}

          \item[pPopupItem]


popup item
            {\it (type=pointer to xfdata.FL\_POPUP\_ITEM)}

        \end{Ventry}

      \end{quote}

      \textbf{Return Value}
    \vspace{-1ex}

      \begin{quote}

popup entry
      {\it (type=pointer to xfdata.FL\_POPUP\_ENTRY)}

      \end{quote}

\textbf{Note:} 
e.g. \emph{todo}


\textbf{Status:} 
Untested + NoDoc + NoDemo = NOT OK


    \end{boxedminipage}

    \label{xformslib:flpopup:fl_popup_delete}
    \index{xformslib \textit{(package)}!xformslib.flpopup \textit{(module)}!xformslib.flpopup.fl\_popup\_delete \textit{(function)}}

    \vspace{0.5ex}

\hspace{.8\funcindent}\begin{boxedminipage}{\funcwidth}

    \raggedright \textbf{fl\_popup\_delete}(\textit{pPopup})

    \vspace{-1.5ex}

    \rule{\textwidth}{0.5\fboxrule}
\setlength{\parskip}{2ex}

Deletes a popup. It?s not possible to call the function while the
popup is still visible on the screen. Calling it from any callback
function is problematic unless you know for sure that the popup to be
deleted (and sub-popups of it) won't be used later and thus normally
should be avoided.

-{}-
\setlength{\parskip}{1ex}
      \textbf{Parameters}
      \vspace{-1ex}

      \begin{quote}
        \begin{Ventry}{xxxxxx}

          \item[pPopup]


popup class instance
            {\it (type=pointer to xfdata.FL\_POPUP)}

        \end{Ventry}

      \end{quote}

      \textbf{Return Value}
    \vspace{-1ex}

      \begin{quote}

0, or -1 (on failure)
      {\it (type=int)}

      \end{quote}

\textbf{Note:} 
e.g. \emph{todo}


\textbf{Status:} 
Untested + NoDoc + NoDemo = NOT OK


    \end{boxedminipage}

    \label{xformslib:flpopup:fl_popup_entry_delete}
    \index{xformslib \textit{(package)}!xformslib.flpopup \textit{(module)}!xformslib.flpopup.fl\_popup\_entry\_delete \textit{(function)}}

    \vspace{0.5ex}

\hspace{.8\funcindent}\begin{boxedminipage}{\funcwidth}

    \raggedright \textbf{fl\_popup\_entry\_delete}(\textit{pPopupEntry})

    \vspace{-1.5ex}

    \rule{\textwidth}{0.5\fboxrule}
\setlength{\parskip}{2ex}

Removes an existing popup entry.

-{}-
\setlength{\parskip}{1ex}
      \textbf{Parameters}
      \vspace{-1ex}

      \begin{quote}
        \begin{Ventry}{xxxxxxxxxxx}

          \item[pPopupEntry]


popup entry to be removed.
            {\it (type=pointer to xfdata.FL\_POPUP\_ENTRY)}

        \end{Ventry}

      \end{quote}

      \textbf{Return Value}
    \vspace{-1ex}

      \begin{quote}

0, or -1 (on failure)
      {\it (type=int)}

      \end{quote}

\textbf{Note:} 
e.g. \emph{todo}


\textbf{Status:} 
Untested + NoDoc + NoDemo = NOT OK


    \end{boxedminipage}

    \label{xformslib:flpopup:fl_popup_do}
    \index{xformslib \textit{(package)}!xformslib.flpopup \textit{(module)}!xformslib.flpopup.fl\_popup\_do \textit{(function)}}

    \vspace{0.5ex}

\hspace{.8\funcindent}\begin{boxedminipage}{\funcwidth}

    \raggedright \textbf{fl\_popup\_do}(\textit{pPopup})

    \vspace{-1.5ex}

    \rule{\textwidth}{0.5\fboxrule}
\setlength{\parskip}{2ex}

Shows the created popup and returns when the the user is done with the
popup and it has been removed from the screen. Only idle callbacks and
timers etc. are executed in the background while a popup is being shown.

-{}-
\setlength{\parskip}{1ex}
      \textbf{Parameters}
      \vspace{-1ex}

      \begin{quote}
        \begin{Ventry}{xxxxxx}

          \item[pPopup]


popup class instance
            {\it (type=pointer to xfdata.FL\_POPUP)}

        \end{Ventry}

      \end{quote}

      \textbf{Return Value}
    \vspace{-1ex}

      \begin{quote}

popup return class instance
      {\it (type=pointer to xfdata.FL\_POPUP\_RETURN)}

      \end{quote}

\textbf{Note:} 
e.g. \emph{todo}


\textbf{Status:} 
Untested + NoDoc + NoDemo = NOT OK


    \end{boxedminipage}

    \label{xformslib:flpopup:fl_popup_set_position}
    \index{xformslib \textit{(package)}!xformslib.flpopup \textit{(module)}!xformslib.flpopup.fl\_popup\_set\_position \textit{(function)}}

    \vspace{0.5ex}

\hspace{.8\funcindent}\begin{boxedminipage}{\funcwidth}

    \raggedright \textbf{fl\_popup\_set\_position}(\textit{pPopup}, \textit{x}, \textit{y})

    \vspace{-1.5ex}

    \rule{\textwidth}{0.5\fboxrule}
\setlength{\parskip}{2ex}

Sets position where the popup is supposed to appear (if never called,
the popup appears at the mouse position).

-{}-
\setlength{\parskip}{1ex}
      \textbf{Parameters}
      \vspace{-1ex}

      \begin{quote}
        \begin{Ventry}{xxxxxx}

          \item[pPopup]


popup class instance
            {\it (type=pointer to xfdata.FL\_POPUP)}

          \item[x]


horizontal position (upper-left corner)
            {\it (type=int)}

          \item[y]


vertical position (upper-left corner)
            {\it (type=int)}

        \end{Ventry}

      \end{quote}

\textbf{Note:} 
e.g. \emph{todo}


\textbf{Status:} 
Untested + NoDoc + NoDemo = NOT OK


    \end{boxedminipage}

    \label{xformslib:flpopup:fl_popup_get_policy}
    \index{xformslib \textit{(package)}!xformslib.flpopup \textit{(module)}!xformslib.flpopup.fl\_popup\_get\_policy \textit{(function)}}

    \vspace{0.5ex}

\hspace{.8\funcindent}\begin{boxedminipage}{\funcwidth}

    \raggedright \textbf{fl\_popup\_get\_policy}(\textit{pPopup})

    \vspace{-1.5ex}

    \rule{\textwidth}{0.5\fboxrule}
\setlength{\parskip}{2ex}

\emph{todo}

-{}-
\setlength{\parskip}{1ex}
      \textbf{Parameters}
      \vspace{-1ex}

      \begin{quote}
        \begin{Ventry}{xxxxxx}

          \item[pPopup]


popup class instance
            {\it (type=pointer to xfdata.FL\_POPUP)}

        \end{Ventry}

      \end{quote}

      \textbf{Return Value}
    \vspace{-1ex}

      \begin{quote}

num.
      {\it (type=int)}

      \end{quote}

\textbf{Note:} 
e.g. \emph{todo}


\textbf{Status:} 
Untested + NoDoc + NoDemo = NOT OK


    \end{boxedminipage}

    \label{xformslib:flpopup:fl_popup_set_policy}
    \index{xformslib \textit{(package)}!xformslib.flpopup \textit{(module)}!xformslib.flpopup.fl\_popup\_set\_policy \textit{(function)}}

    \vspace{0.5ex}

\hspace{.8\funcindent}\begin{boxedminipage}{\funcwidth}

    \raggedright \textbf{fl\_popup\_set\_policy}(\textit{pPopup}, \textit{policy})

    \vspace{-1.5ex}

    \rule{\textwidth}{0.5\fboxrule}
\setlength{\parskip}{2ex}

Sets policy for handling the popup (i.e. does it get closed when the
user releases the mouse button outside an active entry or not?).

-{}-
\setlength{\parskip}{1ex}
      \textbf{Parameters}
      \vspace{-1ex}

      \begin{quote}
        \begin{Ventry}{xxxxxx}

          \item[pPopup]


popup class instance
            {\it (type=pointer to xfdata.FL\_POPUP)}

          \item[policy]


policy to be set. Values (from xfdata.py) FL\_POPUP\_NORMAL\_SELECT,
FL\_POPUP\_DRAG\_SELECT
            {\it (type=int)}

        \end{Ventry}

      \end{quote}

      \textbf{Return Value}
    \vspace{-1ex}

      \begin{quote}

num.
      {\it (type=int)}

      \end{quote}

\textbf{Note:} 
e.g. \emph{todo}


\textbf{Status:} 
Untested + NoDoc + NoDemo = NOT OK


    \end{boxedminipage}

    \label{xformslib:flpopup:fl_popup_set_callback}
    \index{xformslib \textit{(package)}!xformslib.flpopup \textit{(module)}!xformslib.flpopup.fl\_popup\_set\_callback \textit{(function)}}

    \vspace{0.5ex}

\hspace{.8\funcindent}\begin{boxedminipage}{\funcwidth}

    \raggedright \textbf{fl\_popup\_set\_callback}(\textit{pPopup}, \textit{py\_PopupCb})

    \vspace{-1.5ex}

    \rule{\textwidth}{0.5\fboxrule}
\setlength{\parskip}{2ex}

\emph{todo}

-{}-
\setlength{\parskip}{1ex}
      \textbf{Parameters}
      \vspace{-1ex}

      \begin{quote}
        \begin{Ventry}{xxxxxxxxxx}

          \item[pPopup]


popup class instance
            {\it (type=pointer to xfdata.FL\_POPUP)}

          \item[py\_PopupCb]


name referring to function(pPopupReturn) -> num.
            {\it (type=python function callback, returning value)}

        \end{Ventry}

      \end{quote}

      \textbf{Return Value}
    \vspace{-1ex}

      \begin{quote}

old popup callback
      {\it (type=pointer ot xfdata.FL\_POPUP\_CB)}

      \end{quote}

\textbf{Note:} 
e.g. \emph{todo}


\textbf{Status:} 
Untested + NoDoc + NoDemo = NOT OK


    \end{boxedminipage}

    \label{xformslib:flpopup:fl_popup_get_title_font}
    \index{xformslib \textit{(package)}!xformslib.flpopup \textit{(module)}!xformslib.flpopup.fl\_popup\_get\_title\_font \textit{(function)}}

    \vspace{0.5ex}

\hspace{.8\funcindent}\begin{boxedminipage}{\funcwidth}

    \raggedright \textbf{fl\_popup\_get\_title\_font}(\textit{pPopup})

    \vspace{-1.5ex}

    \rule{\textwidth}{0.5\fboxrule}
\setlength{\parskip}{2ex}

\emph{todo}

-{}-
\setlength{\parskip}{1ex}
      \textbf{Parameters}
      \vspace{-1ex}

      \begin{quote}
        \begin{Ventry}{xxxxxx}

          \item[pPopup]


popup class instance
            {\it (type=pointer to xfdata.FL\_POPUP)}

        \end{Ventry}

      \end{quote}

      \textbf{Return Value}
    \vspace{-1ex}

      \begin{quote}

style, size
      {\it (type=int, int)}

      \end{quote}

\textbf{Note:} 
e.g. \emph{todo}


\textbf{Attention:} 
API change from XForms - upstream was
fl\_popup\_get\_title\_font(pPopup, style, size)


\textbf{Status:} 
Untested + NoDoc + NoDemo = NOT OK


    \end{boxedminipage}

    \label{xformslib:flpopup:fl_popup_set_title_font}
    \index{xformslib \textit{(package)}!xformslib.flpopup \textit{(module)}!xformslib.flpopup.fl\_popup\_set\_title\_font \textit{(function)}}

    \vspace{0.5ex}

\hspace{.8\funcindent}\begin{boxedminipage}{\funcwidth}

    \raggedright \textbf{fl\_popup\_set\_title\_font}(\textit{pPopup}, \textit{style}, \textit{size})

    \vspace{-1.5ex}

    \rule{\textwidth}{0.5\fboxrule}
\setlength{\parskip}{2ex}

\emph{todo}

-{}-
\setlength{\parskip}{1ex}
      \textbf{Parameters}
      \vspace{-1ex}

      \begin{quote}
        \begin{Ventry}{xxxxxx}

          \item[pPopup]


popup class instance
            {\it (type=pointer to xfdata.FL\_POPUP)}

          \item[style]


\emph{todo}
            {\it (type=int)}

          \item[size]


\emph{todo}
            {\it (type=int)}

        \end{Ventry}

      \end{quote}

\textbf{Note:} 
e.g. \emph{todo}


\textbf{Status:} 
Tested + NoDoc + Demo = OK


    \end{boxedminipage}

    \label{xformslib:flpopup:fl_popup_entry_get_font}
    \index{xformslib \textit{(package)}!xformslib.flpopup \textit{(module)}!xformslib.flpopup.fl\_popup\_entry\_get\_font \textit{(function)}}

    \vspace{0.5ex}

\hspace{.8\funcindent}\begin{boxedminipage}{\funcwidth}

    \raggedright \textbf{fl\_popup\_entry\_get\_font}(\textit{pPopup})

    \vspace{-1.5ex}

    \rule{\textwidth}{0.5\fboxrule}
\setlength{\parskip}{2ex}

\emph{todo}

-{}-
\setlength{\parskip}{1ex}
      \textbf{Parameters}
      \vspace{-1ex}

      \begin{quote}
        \begin{Ventry}{xxxxxx}

          \item[pPopup]


popup class instance
            {\it (type=pointer to xfdata.FL\_POPUP)}

        \end{Ventry}

      \end{quote}

      \textbf{Return Value}
    \vspace{-1ex}

      \begin{quote}

style, size
      {\it (type=int, int)}

      \end{quote}

\textbf{Note:} 
e.g. \emph{todo}


\textbf{Attention:} 
API change from XForms - upstream was
fl\_popup\_entry\_get\_font(pPopup, style, size)


\textbf{Status:} 
Untested + NoDoc + NoDemo = NOT OK


    \end{boxedminipage}

    \label{xformslib:flpopup:fl_popup_entry_set_font}
    \index{xformslib \textit{(package)}!xformslib.flpopup \textit{(module)}!xformslib.flpopup.fl\_popup\_entry\_set\_font \textit{(function)}}

    \vspace{0.5ex}

\hspace{.8\funcindent}\begin{boxedminipage}{\funcwidth}

    \raggedright \textbf{fl\_popup\_entry\_set\_font}(\textit{pPopup}, \textit{style}, \textit{size})

    \vspace{-1.5ex}

    \rule{\textwidth}{0.5\fboxrule}
\setlength{\parskip}{2ex}

Sets the font style and size of a popup entry.

-{}-
\setlength{\parskip}{1ex}
      \textbf{Parameters}
      \vspace{-1ex}

      \begin{quote}
        \begin{Ventry}{xxxxxx}

          \item[pPopup]


popup class instance
            {\it (type=pointer to xfdata.FL\_POPUP)}

          \item[style]


style of the popup entry \emph{todo}
            {\it (type=int)}

          \item[size]


size of the popup entry \emph{todo}
            {\it (type=int)}

        \end{Ventry}

      \end{quote}

\textbf{Note:} 
e.g. \emph{todo}


\textbf{Status:} 
Untested + NoDoc + NoDemo = NOT OK


    \end{boxedminipage}

    \label{xformslib:flpopup:fl_popup_get_bw}
    \index{xformslib \textit{(package)}!xformslib.flpopup \textit{(module)}!xformslib.flpopup.fl\_popup\_get\_bw \textit{(function)}}

    \vspace{0.5ex}

\hspace{.8\funcindent}\begin{boxedminipage}{\funcwidth}

    \raggedright \textbf{fl\_popup\_get\_bw}(\textit{pPopup})

    \vspace{-1.5ex}

    \rule{\textwidth}{0.5\fboxrule}
\setlength{\parskip}{2ex}

Returns the border width of a popup.

-{}-
\setlength{\parskip}{1ex}
      \textbf{Parameters}
      \vspace{-1ex}

      \begin{quote}
        \begin{Ventry}{xxxxxx}

          \item[pPopup]


popup class instance
            {\it (type=pointer to xfdata.FL\_POPUP)}

        \end{Ventry}

      \end{quote}

      \textbf{Return Value}
    \vspace{-1ex}

      \begin{quote}

borderwidth (bw)
      {\it (type=)}

      \end{quote}

\textbf{Note:} 
e.g. \emph{todo}


\textbf{Status:} 
Untested + NoDoc + NoDemo = NOT OK


    \end{boxedminipage}

    \label{xformslib:flpopup:fl_popup_set_bw}
    \index{xformslib \textit{(package)}!xformslib.flpopup \textit{(module)}!xformslib.flpopup.fl\_popup\_set\_bw \textit{(function)}}

    \vspace{0.5ex}

\hspace{.8\funcindent}\begin{boxedminipage}{\funcwidth}

    \raggedright \textbf{fl\_popup\_set\_bw}(\textit{pPopup}, \textit{bw})

    \vspace{-1.5ex}

    \rule{\textwidth}{0.5\fboxrule}
\setlength{\parskip}{2ex}

Sets the border width of a popup.

-{}-
\setlength{\parskip}{1ex}
      \textbf{Parameters}
      \vspace{-1ex}

      \begin{quote}
        \begin{Ventry}{xxxxxx}

          \item[pPopup]


popup class instance
            {\it (type=pointer to xfdata.FL\_POPUP)}

          \item[bw]


border width value to be set
            {\it (type=int)}

        \end{Ventry}

      \end{quote}

      \textbf{Return Value}
    \vspace{-1ex}

      \begin{quote}

num.
      {\it (type=int)}

      \end{quote}

\textbf{Note:} 
e.g. \emph{todo}


\textbf{Status:} 
Tested + NoDoc + Demo = OK


    \end{boxedminipage}

    \label{xformslib:flpopup:fl_popup_get_color}
    \index{xformslib \textit{(package)}!xformslib.flpopup \textit{(module)}!xformslib.flpopup.fl\_popup\_get\_color \textit{(function)}}

    \vspace{0.5ex}

\hspace{.8\funcindent}\begin{boxedminipage}{\funcwidth}

    \raggedright \textbf{fl\_popup\_get\_color}(\textit{pPopup}, \textit{colrpos})

    \vspace{-1.5ex}

    \rule{\textwidth}{0.5\fboxrule}
\setlength{\parskip}{2ex}

\emph{todo}

-{}-
\setlength{\parskip}{1ex}
      \textbf{Parameters}
      \vspace{-1ex}

      \begin{quote}
        \begin{Ventry}{xxxxxxx}

          \item[pPopup]


popup class instance
            {\it (type=pointer to xfdata.FL\_POPUP)}

          \item[colrpos]


color type position. Values (from xfdata.py) FL\_POPUP\_BACKGROUND\_COLOR,
FL\_POPUP\_HIGHLIGHT\_COLOR, FL\_POPUP\_TITLE\_COLOR, FL\_POPUP\_TEXT\_COLOR,
FL\_POPUP\_HIGHLIGHT\_TEXT\_COLOR, FL\_POPUP\_DISABLED\_TEXT\_COLOR,
FL\_POPUP\_RADIO\_COLOR
            {\it (type=int)}

        \end{Ventry}

      \end{quote}

      \textbf{Return Value}
    \vspace{-1ex}

      \begin{quote}

color
      {\it (type=long\_pos)}

      \end{quote}

\textbf{Note:} 
e.g. \emph{todo}


\textbf{Status:} 
Untested + NoDoc + NoDemo = NOT OK


    \end{boxedminipage}

    \label{xformslib:flpopup:fl_popup_set_color}
    \index{xformslib \textit{(package)}!xformslib.flpopup \textit{(module)}!xformslib.flpopup.fl\_popup\_set\_color \textit{(function)}}

    \vspace{0.5ex}

\hspace{.8\funcindent}\begin{boxedminipage}{\funcwidth}

    \raggedright \textbf{fl\_popup\_set\_color}(\textit{pPopup}, \textit{colrpos}, \textit{colr})

    \vspace{-1.5ex}

    \rule{\textwidth}{0.5\fboxrule}
\setlength{\parskip}{2ex}

\emph{todo}

-{}-
\setlength{\parskip}{1ex}
      \textbf{Parameters}
      \vspace{-1ex}

      \begin{quote}
        \begin{Ventry}{xxxxxxx}

          \item[pPopup]


popup class instance
            {\it (type=pointer to xfdata.FL\_POPUP)}

          \item[colrpos]


color type position. Values (from xfdata.py) FL\_POPUP\_BACKGROUND\_COLOR,
FL\_POPUP\_HIGHLIGHT\_COLOR, FL\_POPUP\_TITLE\_COLOR, FL\_POPUP\_TEXT\_COLOR,
FL\_POPUP\_HIGHLIGHT\_TEXT\_COLOR, FL\_POPUP\_DISABLED\_TEXT\_COLOR,
FL\_POPUP\_RADIO\_COLOR
            {\it (type=int)}

          \item[colr]


color value to be set
            {\it (type=long\_pos)}

        \end{Ventry}

      \end{quote}

      \textbf{Return Value}
    \vspace{-1ex}

      \begin{quote}

color
      {\it (type=long\_pos)}

      \end{quote}

\textbf{Note:} 
e.g. \emph{todo}


\textbf{Status:} 
Untested + NoDoc + NoDemo = NOT OK


    \end{boxedminipage}

    \label{xformslib:flpopup:fl_popup_set_cursor}
    \index{xformslib \textit{(package)}!xformslib.flpopup \textit{(module)}!xformslib.flpopup.fl\_popup\_set\_cursor \textit{(function)}}

    \vspace{0.5ex}

\hspace{.8\funcindent}\begin{boxedminipage}{\funcwidth}

    \raggedright \textbf{fl\_popup\_set\_cursor}(\textit{pPopup}, \textit{cursnum})

    \vspace{-1.5ex}

    \rule{\textwidth}{0.5\fboxrule}
\setlength{\parskip}{2ex}

Changes the cursor displayed when a popup is shown.

-{}-
\setlength{\parskip}{1ex}
      \textbf{Parameters}
      \vspace{-1ex}

      \begin{quote}
        \begin{Ventry}{xxxxxxx}

          \item[pPopup]


popup class instance
            {\it (type=pointer to xfdata.FL\_POPUP)}

          \item[cursnum]


id of a symbolic cursor shape's name
            {\it (type=int)}

        \end{Ventry}

      \end{quote}

\textbf{Note:} 
e.g. \emph{todo}


\textbf{Status:} 
Untested + NoDoc + NoDemo = NOT OK


    \end{boxedminipage}

    \label{xformslib:flpopup:fl_popup_get_title}
    \index{xformslib \textit{(package)}!xformslib.flpopup \textit{(module)}!xformslib.flpopup.fl\_popup\_get\_title \textit{(function)}}

    \vspace{0.5ex}

\hspace{.8\funcindent}\begin{boxedminipage}{\funcwidth}

    \raggedright \textbf{fl\_popup\_get\_title}(\textit{pPopup})

    \vspace{-1.5ex}

    \rule{\textwidth}{0.5\fboxrule}
\setlength{\parskip}{2ex}

Obtains the title of a popup.

-{}-
\setlength{\parskip}{1ex}
      \textbf{Parameters}
      \vspace{-1ex}

      \begin{quote}
        \begin{Ventry}{xxxxxx}

          \item[pPopup]


popup class instance
            {\it (type=pointer to xfdata.FL\_POPUP)}

        \end{Ventry}

      \end{quote}

      \textbf{Return Value}
    \vspace{-1ex}

      \begin{quote}

title string
      {\it (type=str)}

      \end{quote}

\textbf{Note:} 
e.g. \emph{todo}


\textbf{Status:} 
Untested + NoDoc + NoDemo = NOT OK


    \end{boxedminipage}

    \label{xformslib:flpopup:fl_popup_set_title}
    \index{xformslib \textit{(package)}!xformslib.flpopup \textit{(module)}!xformslib.flpopup.fl\_popup\_set\_title \textit{(function)}}

    \vspace{0.5ex}

\hspace{.8\funcindent}\begin{boxedminipage}{\funcwidth}

    \raggedright \textbf{fl\_popup\_set\_title}(\textit{pPopup}, \textit{title})

    \vspace{-1.5ex}

    \rule{\textwidth}{0.5\fboxrule}
\setlength{\parskip}{2ex}

Sets the title of a popup.

-{}-
\setlength{\parskip}{1ex}
      \textbf{Parameters}
      \vspace{-1ex}

      \begin{quote}
        \begin{Ventry}{xxxxxx}

          \item[pPopup]


popup class instance
            {\it (type=pointer to xfdata.FL\_POPUP)}

          \item[title]


title of the popup
            {\it (type=str)}

        \end{Ventry}

      \end{quote}

      \textbf{Return Value}
    \vspace{-1ex}

      \begin{quote}

popup class instance
      {\it (type=pointer to xfdata.FL\_POPUP)}

      \end{quote}

\textbf{Note:} 
e.g. \emph{todo}


\textbf{Status:} 
Untested + NoDoc + NoDemo = NOT OK


    \end{boxedminipage}

    \label{xformslib:flpopup:fl_popup_entry_set_callback}
    \index{xformslib \textit{(package)}!xformslib.flpopup \textit{(module)}!xformslib.flpopup.fl\_popup\_entry\_set\_callback \textit{(function)}}

    \vspace{0.5ex}

\hspace{.8\funcindent}\begin{boxedminipage}{\funcwidth}

    \raggedright \textbf{fl\_popup\_entry\_set\_callback}(\textit{pPopupEntry}, \textit{py\_PopupCb})

    \vspace{-1.5ex}

    \rule{\textwidth}{0.5\fboxrule}
\setlength{\parskip}{2ex}

\emph{todo}

-{}-
\setlength{\parskip}{1ex}
      \textbf{Parameters}
      \vspace{-1ex}

      \begin{quote}
        \begin{Ventry}{xxxxxxxxxxx}

          \item[pPopupEntry]


popup entry
            {\it (type=pointer to xfdata.FL\_POPUP\_ENTRY)}

          \item[py\_PopupCb]


name referring to function(pPopupReturn) -> num.
            {\it (type=python callback function, returning value)}

        \end{Ventry}

      \end{quote}

      \textbf{Return Value}
    \vspace{-1ex}

      \begin{quote}

old popup callback
      {\it (type=pointer to xfdata.FL\_POPUP\_CB)}

      \end{quote}

\textbf{Note:} 
e.g. \emph{todo}


\textbf{Status:} 
Tested + NoDoc + Demo = OK


    \end{boxedminipage}

    \label{xformslib:flpopup:fl_popup_entry_set_enter_callback}
    \index{xformslib \textit{(package)}!xformslib.flpopup \textit{(module)}!xformslib.flpopup.fl\_popup\_entry\_set\_enter\_callback \textit{(function)}}

    \vspace{0.5ex}

\hspace{.8\funcindent}\begin{boxedminipage}{\funcwidth}

    \raggedright \textbf{fl\_popup\_entry\_set\_enter\_callback}(\textit{pPopupEntry}, \textit{py\_PopupCb})

    \vspace{-1.5ex}

    \rule{\textwidth}{0.5\fboxrule}
\setlength{\parskip}{2ex}

\emph{todo}

-{}-
\setlength{\parskip}{1ex}
      \textbf{Parameters}
      \vspace{-1ex}

      \begin{quote}
        \begin{Ventry}{xxxxxxxxxxx}

          \item[pPopupEntry]


popup entry
            {\it (type=pointer to xfdata.FL\_POPUP\_ENTRY)}

          \item[py\_PopupCb]


name referring to function(pPopupReturn) -> num.
            {\it (type=python callback function, returning value)}

        \end{Ventry}

      \end{quote}

      \textbf{Return Value}
    \vspace{-1ex}

      \begin{quote}

old popup callback
      {\it (type=pointer to xfdata.FL\_POPUP\_CB)}

      \end{quote}

\textbf{Note:} 
e.g. \emph{todo}


\textbf{Status:} 
Untested + NoDoc + NoDemo = NOT OK


    \end{boxedminipage}

    \label{xformslib:flpopup:fl_popup_entry_set_leave_callback}
    \index{xformslib \textit{(package)}!xformslib.flpopup \textit{(module)}!xformslib.flpopup.fl\_popup\_entry\_set\_leave\_callback \textit{(function)}}

    \vspace{0.5ex}

\hspace{.8\funcindent}\begin{boxedminipage}{\funcwidth}

    \raggedright \textbf{fl\_popup\_entry\_set\_leave\_callback}(\textit{pPopupEntry}, \textit{py\_PopupCb})

    \vspace{-1.5ex}

    \rule{\textwidth}{0.5\fboxrule}
\setlength{\parskip}{2ex}

\emph{todo}

-{}-
\setlength{\parskip}{1ex}
      \textbf{Parameters}
      \vspace{-1ex}

      \begin{quote}
        \begin{Ventry}{xxxxxxxxxxx}

          \item[pPopupEntry]


popup entry
            {\it (type=pointer to xfdata.FL\_POPUP\_ENTRY)}

          \item[py\_PopupCb]


name referring to function(pPopupReturn) -> num.
            {\it (type=python callback function, returning value)}

        \end{Ventry}

      \end{quote}

      \textbf{Return Value}
    \vspace{-1ex}

      \begin{quote}

old popup callback
      {\it (type=pointer to xfdata.FL\_POPUP\_CB)}

      \end{quote}

\textbf{Note:} 
e.g. \emph{todo}


\textbf{Status:} 
Untested + NoDoc + NoDemo = NOT OK


    \end{boxedminipage}

    \label{xformslib:flpopup:fl_popup_entry_get_state}
    \index{xformslib \textit{(package)}!xformslib.flpopup \textit{(module)}!xformslib.flpopup.fl\_popup\_entry\_get\_state \textit{(function)}}

    \vspace{0.5ex}

\hspace{.8\funcindent}\begin{boxedminipage}{\funcwidth}

    \raggedright \textbf{fl\_popup\_entry\_get\_state}(\textit{pPopupEntry})

    \vspace{-1.5ex}

    \rule{\textwidth}{0.5\fboxrule}
\setlength{\parskip}{2ex}

\emph{todo}

-{}-
\setlength{\parskip}{1ex}
      \textbf{Parameters}
      \vspace{-1ex}

      \begin{quote}
        \begin{Ventry}{xxxxxxxxxxx}

          \item[pPopupEntry]


popup entry
            {\it (type=pointer to xfdata.FL\_POPUP\_ENTRY)}

        \end{Ventry}

      \end{quote}

      \textbf{Return Value}
    \vspace{-1ex}

      \begin{quote}

state
      {\it (type=int\_pos)}

      \end{quote}

\textbf{Note:} 
e.g. \emph{todo}


\textbf{Status:} 
Untested + NoDoc + NoDemo = NOT OK


    \end{boxedminipage}

    \label{xformslib:flpopup:fl_popup_entry_set_state}
    \index{xformslib \textit{(package)}!xformslib.flpopup \textit{(module)}!xformslib.flpopup.fl\_popup\_entry\_set\_state \textit{(function)}}

    \vspace{0.5ex}

\hspace{.8\funcindent}\begin{boxedminipage}{\funcwidth}

    \raggedright \textbf{fl\_popup\_entry\_set\_state}(\textit{pPopupEntry}, \textit{state})

    \vspace{-1.5ex}

    \rule{\textwidth}{0.5\fboxrule}
\setlength{\parskip}{2ex}

\emph{todo}

-{}-
\setlength{\parskip}{1ex}
      \textbf{Parameters}
      \vspace{-1ex}

      \begin{quote}
        \begin{Ventry}{xxxxxxxxxxx}

          \item[pPopupEntry]


popup entry
            {\it (type=pointer to xfdata.FL\_POPUP\_ENTRY)}

          \item[state]


state to be set. \emph{todo}
            {\it (type=int\_pos)}

        \end{Ventry}

      \end{quote}

      \textbf{Return Value}
    \vspace{-1ex}

      \begin{quote}

old state
      {\it (type=int\_pos)}

      \end{quote}

\textbf{Note:} 
e.g. \emph{todo}


\textbf{Status:} 
Tested + NoDoc + Demo = OK


    \end{boxedminipage}

    \label{xformslib:flpopup:fl_popup_entry_clear_state}
    \index{xformslib \textit{(package)}!xformslib.flpopup \textit{(module)}!xformslib.flpopup.fl\_popup\_entry\_clear\_state \textit{(function)}}

    \vspace{0.5ex}

\hspace{.8\funcindent}\begin{boxedminipage}{\funcwidth}

    \raggedright \textbf{fl\_popup\_entry\_clear\_state}(\textit{pPopupEntry}, \textit{state})

    \vspace{-1.5ex}

    \rule{\textwidth}{0.5\fboxrule}
\setlength{\parskip}{2ex}

\emph{todo}

-{}-
\setlength{\parskip}{1ex}
      \textbf{Parameters}
      \vspace{-1ex}

      \begin{quote}
        \begin{Ventry}{xxxxxxxxxxx}

          \item[pPopupEntry]


popup entry
            {\it (type=pointer to xfdata.FL\_POPUP\_ENTRY)}

          \item[state]


state to be \emph{todo}
            {\it (type=int\_pos)}

        \end{Ventry}

      \end{quote}

      \textbf{Return Value}
    \vspace{-1ex}

      \begin{quote}

state?
      {\it (type=int\_pos)}

      \end{quote}

\textbf{Note:} 
e.g. \emph{todo}


\textbf{Status:} 
Untested + NoDoc + NoDemo = NOT OK


    \end{boxedminipage}

    \label{xformslib:flpopup:fl_popup_entry_raise_state}
    \index{xformslib \textit{(package)}!xformslib.flpopup \textit{(module)}!xformslib.flpopup.fl\_popup\_entry\_raise\_state \textit{(function)}}

    \vspace{0.5ex}

\hspace{.8\funcindent}\begin{boxedminipage}{\funcwidth}

    \raggedright \textbf{fl\_popup\_entry\_raise\_state}(\textit{pPopupEntry}, \textit{state})

    \vspace{-1.5ex}

    \rule{\textwidth}{0.5\fboxrule}
\setlength{\parskip}{2ex}

\emph{todo}

-{}-
\setlength{\parskip}{1ex}
      \textbf{Parameters}
      \vspace{-1ex}

      \begin{quote}
        \begin{Ventry}{xxxxxxxxxxx}

          \item[pPopupEntry]


popup entry
            {\it (type=pointer to xfdata.FL\_POPUP\_ENTRY)}

          \item[state]


state to be \emph{todo}
            {\it (type=int\_pos)}

        \end{Ventry}

      \end{quote}

      \textbf{Return Value}
    \vspace{-1ex}

      \begin{quote}

state?
      {\it (type=int\_pos)}

      \end{quote}

\textbf{Note:} 
e.g. \emph{todo}


\textbf{Status:} 
Untested + NoDoc + NoDemo = NOT OK


    \end{boxedminipage}

    \label{xformslib:flpopup:fl_popup_entry_toggle_state}
    \index{xformslib \textit{(package)}!xformslib.flpopup \textit{(module)}!xformslib.flpopup.fl\_popup\_entry\_toggle\_state \textit{(function)}}

    \vspace{0.5ex}

\hspace{.8\funcindent}\begin{boxedminipage}{\funcwidth}

    \raggedright \textbf{fl\_popup\_entry\_toggle\_state}(\textit{pPopupEntry}, \textit{state})

    \vspace{-1.5ex}

    \rule{\textwidth}{0.5\fboxrule}
\setlength{\parskip}{2ex}

\emph{todo}

-{}-
\setlength{\parskip}{1ex}
      \textbf{Parameters}
      \vspace{-1ex}

      \begin{quote}
        \begin{Ventry}{xxxxxxxxxxx}

          \item[pPopupEntry]


popup entry
            {\it (type=pointer to xfdata.FL\_POPUP\_ENTRY)}

          \item[state]


state to be \emph{todo}
            {\it (type=int\_pos)}

        \end{Ventry}

      \end{quote}

      \textbf{Return Value}
    \vspace{-1ex}

      \begin{quote}

state?
      {\it (type=int\_pos)}

      \end{quote}

\textbf{Note:} 
e.g. \emph{todo}


\textbf{Status:} 
Untested + NoDoc + NoDemo = NOT OK


    \end{boxedminipage}

    \label{xformslib:flpopup:fl_popup_entry_set_text}
    \index{xformslib \textit{(package)}!xformslib.flpopup \textit{(module)}!xformslib.flpopup.fl\_popup\_entry\_set\_text \textit{(function)}}

    \vspace{0.5ex}

\hspace{.8\funcindent}\begin{boxedminipage}{\funcwidth}

    \raggedright \textbf{fl\_popup\_entry\_set\_text}(\textit{pPopupEntry}, \textit{text})

    \vspace{-1.5ex}

    \rule{\textwidth}{0.5\fboxrule}
\setlength{\parskip}{2ex}

\emph{todo}

-{}-
\setlength{\parskip}{1ex}
      \textbf{Parameters}
      \vspace{-1ex}

      \begin{quote}
        \begin{Ventry}{xxxxxxxxxxx}

          \item[pPopupEntry]


popup entry
            {\it (type=pointer to xfdata.FL\_POPUP\_ENTRY)}

          \item[text]


text for the entry
            {\it (type=str)}

        \end{Ventry}

      \end{quote}

      \textbf{Return Value}
    \vspace{-1ex}

      \begin{quote}

num.
      {\it (type=int)}

      \end{quote}

\textbf{Note:} 
e.g. \emph{todo}


\textbf{Status:} 
Untested + NoDoc + NoDemo = NOT OK


    \end{boxedminipage}

    \label{xformslib:flpopup:fl_popup_entry_set_shortcut}
    \index{xformslib \textit{(package)}!xformslib.flpopup \textit{(module)}!xformslib.flpopup.fl\_popup\_entry\_set\_shortcut \textit{(function)}}

    \vspace{0.5ex}

\hspace{.8\funcindent}\begin{boxedminipage}{\funcwidth}

    \raggedright \textbf{fl\_popup\_entry\_set\_shortcut}(\textit{pPopupEntry}, \textit{textsc})

    \vspace{-1.5ex}

    \rule{\textwidth}{0.5\fboxrule}
\setlength{\parskip}{2ex}

\emph{todo}

-{}-
\setlength{\parskip}{1ex}
      \textbf{Parameters}
      \vspace{-1ex}

      \begin{quote}
        \begin{Ventry}{xxxxxxxxxxx}

          \item[pPopupEntry]


popup entry
            {\it (type=pointer to xfdata.FL\_POPUP\_ENTRY)}

          \item[textsc]


text for the shortcut
            {\it (type=str)}

        \end{Ventry}

      \end{quote}

\textbf{Note:} 
e.g. \emph{todo}


\textbf{Status:} 
Tested + NoDoc + Demo = OK


    \end{boxedminipage}

    \label{xformslib:flpopup:fl_popup_entry_set_value}
    \index{xformslib \textit{(package)}!xformslib.flpopup \textit{(module)}!xformslib.flpopup.fl\_popup\_entry\_set\_value \textit{(function)}}

    \vspace{0.5ex}

\hspace{.8\funcindent}\begin{boxedminipage}{\funcwidth}

    \raggedright \textbf{fl\_popup\_entry\_set\_value}(\textit{pPopupEntry}, \textit{val})

    \vspace{-1.5ex}

    \rule{\textwidth}{0.5\fboxrule}
\setlength{\parskip}{2ex}

\emph{todo}

-{}-
\setlength{\parskip}{1ex}
      \textbf{Parameters}
      \vspace{-1ex}

      \begin{quote}
        \begin{Ventry}{xxxxxxxxxxx}

          \item[pPopupEntry]


popup entry
            {\it (type=pointer to xfdata.FL\_POPUP\_ENTRY)}

          \item[val]


value?
            {\it (type=long)}

        \end{Ventry}

      \end{quote}

      \textbf{Return Value}
    \vspace{-1ex}

      \begin{quote}

num.
      {\it (type=int)}

      \end{quote}

\textbf{Note:} 
e.g. \emph{todo}


\textbf{Status:} 
Untested + NoDoc + NoDemo = NOT OK


    \end{boxedminipage}

    \label{xformslib:flpopup:fl_popup_entry_set_user_data}
    \index{xformslib \textit{(package)}!xformslib.flpopup \textit{(module)}!xformslib.flpopup.fl\_popup\_entry\_set\_user\_data \textit{(function)}}

    \vspace{0.5ex}

\hspace{.8\funcindent}\begin{boxedminipage}{\funcwidth}

    \raggedright \textbf{fl\_popup\_entry\_set\_user\_data}(\textit{pPopupEntry}, \textit{vdata})

    \vspace{-1.5ex}

    \rule{\textwidth}{0.5\fboxrule}
\setlength{\parskip}{2ex}

\emph{todo}

-{}-
\setlength{\parskip}{1ex}
      \textbf{Parameters}
      \vspace{-1ex}

      \begin{quote}
        \begin{Ventry}{xxxxxxxxxxx}

          \item[pPopupEntry]


popup entry
            {\it (type=pointer to xfdata.FL\_POPUP\_ENTRY)}

          \item[vdata]


user data to be passed to function; callback has to take care of
type check
            {\it (type=any type (e.g. 'None', int, str, etc..))}

        \end{Ventry}

      \end{quote}

      \textbf{Return Value}
    \vspace{-1ex}

      \begin{quote}

\emph{todo}
      {\it (type=pointer to void?)}

      \end{quote}

\textbf{Note:} 
e.g. \emph{todo}


\textbf{Status:} 
Untested + NoDoc + NoDemo = NOT OK


    \end{boxedminipage}

    \label{xformslib:flpopup:fl_popup_entry_get_by_position}
    \index{xformslib \textit{(package)}!xformslib.flpopup \textit{(module)}!xformslib.flpopup.fl\_popup\_entry\_get\_by\_position \textit{(function)}}

    \vspace{0.5ex}

\hspace{.8\funcindent}\begin{boxedminipage}{\funcwidth}

    \raggedright \textbf{fl\_popup\_entry\_get\_by\_position}(\textit{pPopup}, \textit{numpos})

    \vspace{-1.5ex}

    \rule{\textwidth}{0.5\fboxrule}
\setlength{\parskip}{2ex}

\emph{todo}

-{}-
\setlength{\parskip}{1ex}
      \textbf{Parameters}
      \vspace{-1ex}

      \begin{quote}
        \begin{Ventry}{xxxxxx}

          \item[pPopup]


popup class instance
            {\it (type=pointer to xfdata.FL\_POPUP)}

          \item[numpos]


position number?
            {\it (type=int)}

        \end{Ventry}

      \end{quote}

      \textbf{Return Value}
    \vspace{-1ex}

      \begin{quote}

popup entry
      {\it (type=pointer to xfdata.FL\_POPUP\_ENTRY)}

      \end{quote}

\textbf{Note:} 
e.g. \emph{todo}


\textbf{Status:} 
Untested + NoDoc + NoDemo = NOT OK


    \end{boxedminipage}

    \label{xformslib:flpopup:fl_popup_entry_get_by_value}
    \index{xformslib \textit{(package)}!xformslib.flpopup \textit{(module)}!xformslib.flpopup.fl\_popup\_entry\_get\_by\_value \textit{(function)}}

    \vspace{0.5ex}

\hspace{.8\funcindent}\begin{boxedminipage}{\funcwidth}

    \raggedright \textbf{fl\_popup\_entry\_get\_by\_value}(\textit{pPopup}, \textit{val})

    \vspace{-1.5ex}

    \rule{\textwidth}{0.5\fboxrule}
\setlength{\parskip}{2ex}

\emph{todo}

-{}-
\setlength{\parskip}{1ex}
      \textbf{Parameters}
      \vspace{-1ex}

      \begin{quote}
        \begin{Ventry}{xxxxxx}

          \item[pPopup]


popup class instance
            {\it (type=pointer to xfdata.FL\_POPUP)}

          \item[val]


value?
            {\it (type=long)}

        \end{Ventry}

      \end{quote}

      \textbf{Return Value}
    \vspace{-1ex}

      \begin{quote}

popup entry
      {\it (type=pointer to xfdata.FL\_POPUP\_ENTRY)}

      \end{quote}

\textbf{Note:} 
e.g. \emph{todo}


\textbf{Status:} 
Untested + NoDoc + NoDemo = NOT OK


    \end{boxedminipage}

    \label{xformslib:flpopup:fl_popup_entry_get_by_user_data}
    \index{xformslib \textit{(package)}!xformslib.flpopup \textit{(module)}!xformslib.flpopup.fl\_popup\_entry\_get\_by\_user\_data \textit{(function)}}

    \vspace{0.5ex}

\hspace{.8\funcindent}\begin{boxedminipage}{\funcwidth}

    \raggedright \textbf{fl\_popup\_entry\_get\_by\_user\_data}(\textit{pPopup}, \textit{vdata})

    \vspace{-1.5ex}

    \rule{\textwidth}{0.5\fboxrule}
\setlength{\parskip}{2ex}

\emph{todo}

-{}-
\setlength{\parskip}{1ex}
      \textbf{Parameters}
      \vspace{-1ex}

      \begin{quote}
        \begin{Ventry}{xxxxxx}

          \item[pPopup]


popup class instance
            {\it (type=pointer to xfdata.FL\_POPUP)}

          \item[vdata]


user data to be passed to function; callback has to take care of
type check
            {\it (type=any type (e.g. 'None', int, str, etc..))}

        \end{Ventry}

      \end{quote}

      \textbf{Return Value}
    \vspace{-1ex}

      \begin{quote}

popup entry
      {\it (type=pointer to xfdata.FL\_POPUP\_ENTRY)}

      \end{quote}

\textbf{Note:} 
e.g. \emph{todo}


\textbf{Status:} 
Untested + NoDoc + NoDemo = NOT OK


    \end{boxedminipage}

    \label{xformslib:flpopup:fl_popup_entry_get_by_text}
    \index{xformslib \textit{(package)}!xformslib.flpopup \textit{(module)}!xformslib.flpopup.fl\_popup\_entry\_get\_by\_text \textit{(function)}}

    \vspace{0.5ex}

\hspace{.8\funcindent}\begin{boxedminipage}{\funcwidth}

    \raggedright \textbf{fl\_popup\_entry\_get\_by\_text}(\textit{pPopup}, \textit{text})

    \vspace{-1.5ex}

    \rule{\textwidth}{0.5\fboxrule}
\setlength{\parskip}{2ex}

\emph{todo}

-{}-
\setlength{\parskip}{1ex}
      \textbf{Parameters}
      \vspace{-1ex}

      \begin{quote}
        \begin{Ventry}{xxxxxx}

          \item[pPopup]


popup class instance
            {\it (type=pointer to xfdata.FL\_POPUP)}

          \item[text]


text
            {\it (type=str)}

        \end{Ventry}

      \end{quote}

      \textbf{Return Value}
    \vspace{-1ex}

      \begin{quote}

popup entry
      {\it (type=pointer to xfdata.FL\_POPUP\_ENTRY)}

      \end{quote}

\textbf{Note:} 
e.g. \emph{todo}


\textbf{Status:} 
Untested + NoDoc + NoDemo = NOT OK


    \end{boxedminipage}

    \label{xformslib:flpopup:fl_popup_entry_get_by_label}
    \index{xformslib \textit{(package)}!xformslib.flpopup \textit{(module)}!xformslib.flpopup.fl\_popup\_entry\_get\_by\_label \textit{(function)}}

    \vspace{0.5ex}

\hspace{.8\funcindent}\begin{boxedminipage}{\funcwidth}

    \raggedright \textbf{fl\_popup\_entry\_get\_by\_label}(\textit{pPopup}, \textit{label})

    \vspace{-1.5ex}

    \rule{\textwidth}{0.5\fboxrule}
\setlength{\parskip}{2ex}

\emph{todo}

-{}-
\setlength{\parskip}{1ex}
      \textbf{Parameters}
      \vspace{-1ex}

      \begin{quote}
        \begin{Ventry}{xxxxxx}

          \item[pPopup]


popup class instance
            {\it (type=pointer to xfdata.FL\_POPUP)}

          \item[label]


label
            {\it (type=str)}

        \end{Ventry}

      \end{quote}

      \textbf{Return Value}
    \vspace{-1ex}

      \begin{quote}

popup entry
      {\it (type=pointer to xfdata.FL\_POPUP\_ENTRY)}

      \end{quote}

\textbf{Note:} 
e.g. \emph{todo}


\textbf{Status:} 
Untested + NoDoc + NoDemo = NOT OK


    \end{boxedminipage}

    \label{xformslib:flpopup:fl_popup_entry_get_group}
    \index{xformslib \textit{(package)}!xformslib.flpopup \textit{(module)}!xformslib.flpopup.fl\_popup\_entry\_get\_group \textit{(function)}}

    \vspace{0.5ex}

\hspace{.8\funcindent}\begin{boxedminipage}{\funcwidth}

    \raggedright \textbf{fl\_popup\_entry\_get\_group}(\textit{pPopupEntry})

    \vspace{-1.5ex}

    \rule{\textwidth}{0.5\fboxrule}
\setlength{\parskip}{2ex}

\emph{todo}

-{}-
\setlength{\parskip}{1ex}
      \textbf{Parameters}
      \vspace{-1ex}

      \begin{quote}
        \begin{Ventry}{xxxxxxxxxxx}

          \item[pPopupEntry]


popup entry
            {\it (type=pointer to xfdata.FL\_POPUP\_ENTRY)}

        \end{Ventry}

      \end{quote}

      \textbf{Return Value}
    \vspace{-1ex}

      \begin{quote}

num.
      {\it (type=int)}

      \end{quote}

\textbf{Note:} 
e.g. \emph{todo}


\textbf{Status:} 
Untested + NoDoc + NoDemo = NOT OK


    \end{boxedminipage}

    \label{xformslib:flpopup:fl_popup_entry_set_group}
    \index{xformslib \textit{(package)}!xformslib.flpopup \textit{(module)}!xformslib.flpopup.fl\_popup\_entry\_set\_group \textit{(function)}}

    \vspace{0.5ex}

\hspace{.8\funcindent}\begin{boxedminipage}{\funcwidth}

    \raggedright \textbf{fl\_popup\_entry\_set\_group}(\textit{pPopupEntry}, \textit{num})

    \vspace{-1.5ex}

    \rule{\textwidth}{0.5\fboxrule}
\setlength{\parskip}{2ex}

\emph{todo}

-{}-
\setlength{\parskip}{1ex}
      \textbf{Parameters}
      \vspace{-1ex}

      \begin{quote}
        \begin{Ventry}{xxxxxxxxxxx}

          \item[pPopupEntry]


popup entry
            {\it (type=pointer to xfdata.FL\_POPUP\_ENTRY)}

          \item[num]


\emph{todo}
            {\it (type=int)}

        \end{Ventry}

      \end{quote}

      \textbf{Return Value}
    \vspace{-1ex}

      \begin{quote}

num.
      {\it (type=int)}

      \end{quote}

\textbf{Note:} 
e.g. \emph{todo}


\textbf{Status:} 
Untested + NoDoc + NoDemo = NOT OK


    \end{boxedminipage}

    \label{xformslib:flpopup:fl_popup_entry_get_subpopup}
    \index{xformslib \textit{(package)}!xformslib.flpopup \textit{(module)}!xformslib.flpopup.fl\_popup\_entry\_get\_subpopup \textit{(function)}}

    \vspace{0.5ex}

\hspace{.8\funcindent}\begin{boxedminipage}{\funcwidth}

    \raggedright \textbf{fl\_popup\_entry\_get\_subpopup}(\textit{pPopupEntry})

    \vspace{-1.5ex}

    \rule{\textwidth}{0.5\fboxrule}
\setlength{\parskip}{2ex}

\emph{todo}

-{}-
\setlength{\parskip}{1ex}
      \textbf{Parameters}
      \vspace{-1ex}

      \begin{quote}
        \begin{Ventry}{xxxxxxxxxxx}

          \item[pPopupEntry]


popup entry
            {\it (type=pointer to xfdata.FL\_POPUP\_ENTRY)}

        \end{Ventry}

      \end{quote}

      \textbf{Return Value}
    \vspace{-1ex}

      \begin{quote}

popup class instance
      {\it (type=pointer to xfdata.FL\_POPUP)}

      \end{quote}

\textbf{Note:} 
e.g. \emph{todo}


\textbf{Status:} 
Untested + NoDoc + NoDemo = NOT OK


    \end{boxedminipage}

    \label{xformslib:flpopup:fl_popup_entry_set_subpopup}
    \index{xformslib \textit{(package)}!xformslib.flpopup \textit{(module)}!xformslib.flpopup.fl\_popup\_entry\_set\_subpopup \textit{(function)}}

    \vspace{0.5ex}

\hspace{.8\funcindent}\begin{boxedminipage}{\funcwidth}

    \raggedright \textbf{fl\_popup\_entry\_set\_subpopup}(\textit{pPopupEntry}, \textit{pPopup})

    \vspace{-1.5ex}

    \rule{\textwidth}{0.5\fboxrule}
\setlength{\parskip}{2ex}

\emph{todo}

-{}-
\setlength{\parskip}{1ex}
      \textbf{Parameters}
      \vspace{-1ex}

      \begin{quote}
        \begin{Ventry}{xxxxxxxxxxx}

          \item[pPopupEntry]


popup entry
            {\it (type=pointer to xfdata.FL\_POPUP\_ENTRY)}

          \item[pPopup]


popup class instance
            {\it (type=pointer to xfdata.FL\_POPUP)}

        \end{Ventry}

      \end{quote}

      \textbf{Return Value}
    \vspace{-1ex}

      \begin{quote}

popup class instance
      {\it (type=pointer to xfdata.FL\_POPUP)}

      \end{quote}

\textbf{Note:} 
e.g. \emph{todo}


\textbf{Status:} 
Untested + NoDoc + NoDemo = NOT OK


    \end{boxedminipage}

    \label{xformslib:flpopup:fl_popup_get_size}
    \index{xformslib \textit{(package)}!xformslib.flpopup \textit{(module)}!xformslib.flpopup.fl\_popup\_get\_size \textit{(function)}}

    \vspace{0.5ex}

\hspace{.8\funcindent}\begin{boxedminipage}{\funcwidth}

    \raggedright \textbf{fl\_popup\_get\_size}(\textit{pPopup})

    \vspace{-1.5ex}

    \rule{\textwidth}{0.5\fboxrule}
\setlength{\parskip}{2ex}

\emph{todo}

-{}-
\setlength{\parskip}{1ex}
      \textbf{Parameters}
      \vspace{-1ex}

      \begin{quote}
        \begin{Ventry}{xxxxxx}

          \item[pPopup]


popup class instance
            {\it (type=pointer to xfdata.FL\_POPUP)}

        \end{Ventry}

      \end{quote}

      \textbf{Return Value}
    \vspace{-1ex}

      \begin{quote}

num., width (w), height (h)
      {\it (type=int, int\_pos, int\_pos)}

      \end{quote}

\textbf{Note:} 
e.g. \emph{todo}


\textbf{Attention:} 
API change from XForms - upstream was
fl\_popup\_get\_size(pPopup, w, h)


\textbf{Status:} 
Untested + NoDoc + NoDemo = NOT OK


    \end{boxedminipage}

    \label{xformslib:flpopup:fl_popup_get_min_width}
    \index{xformslib \textit{(package)}!xformslib.flpopup \textit{(module)}!xformslib.flpopup.fl\_popup\_get\_min\_width \textit{(function)}}

    \vspace{0.5ex}

\hspace{.8\funcindent}\begin{boxedminipage}{\funcwidth}

    \raggedright \textbf{fl\_popup\_get\_min\_width}(\textit{pPopup})

    \vspace{-1.5ex}

    \rule{\textwidth}{0.5\fboxrule}
\setlength{\parskip}{2ex}

\emph{todo}

-{}-
\setlength{\parskip}{1ex}
      \textbf{Parameters}
      \vspace{-1ex}

      \begin{quote}
        \begin{Ventry}{xxxxxx}

          \item[pPopup]


popup class instance
            {\it (type=pointer to xfdata.FL\_POPUP)}

        \end{Ventry}

      \end{quote}

      \textbf{Return Value}
    \vspace{-1ex}

      \begin{quote}

width (w)
      {\it (type=int)}

      \end{quote}

\textbf{Note:} 
e.g. \emph{todo}


\textbf{Status:} 
Untested + NoDoc + NoDemo = NOT OK


    \end{boxedminipage}

    \label{xformslib:flpopup:fl_popup_set_min_width}
    \index{xformslib \textit{(package)}!xformslib.flpopup \textit{(module)}!xformslib.flpopup.fl\_popup\_set\_min\_width \textit{(function)}}

    \vspace{0.5ex}

\hspace{.8\funcindent}\begin{boxedminipage}{\funcwidth}

    \raggedright \textbf{fl\_popup\_set\_min\_width}(\textit{pPopup}, \textit{width})

    \vspace{-1.5ex}

    \rule{\textwidth}{0.5\fboxrule}
\setlength{\parskip}{2ex}

\emph{todo}

-{}-
\setlength{\parskip}{1ex}
      \textbf{Parameters}
      \vspace{-1ex}

      \begin{quote}
        \begin{Ventry}{xxxxxx}

          \item[pPopup]


popup class instance
            {\it (type=pointer to xfdata.FL\_POPUP)}

          \item[width]


minimum width to be set
            {\it (type=int)}

        \end{Ventry}

      \end{quote}

      \textbf{Return Value}
    \vspace{-1ex}

      \begin{quote}

old width (w) ?
      {\it (type=int)}

      \end{quote}

\textbf{Note:} 
e.g. \emph{todo}


\textbf{Status:} 
Untested + NoDoc + NoDemo = NOT OK


    \end{boxedminipage}


%%%%%%%%%%%%%%%%%%%%%%%%%%%%%%%%%%%%%%%%%%%%%%%%%%%%%%%%%%%%%%%%%%%%%%%%%%%
%%                               Variables                               %%
%%%%%%%%%%%%%%%%%%%%%%%%%%%%%%%%%%%%%%%%%%%%%%%%%%%%%%%%%%%%%%%%%%%%%%%%%%%

  \subsection{Variables}

    \vspace{-1cm}
\hspace{\varindent}\begin{longtable}{|p{\varnamewidth}|p{\vardescrwidth}|l}
\cline{1-2}
\cline{1-2} \centering \textbf{Name} & \centering \textbf{Description}& \\
\cline{1-2}
\endhead\cline{1-2}\multicolumn{3}{r}{\small\textit{continued on next page}}\\\endfoot\cline{1-2}
\endlastfoot\raggedright \_\-\_\-p\-a\-c\-k\-a\-g\-e\-\_\-\_\- & \raggedright \textbf{Value:} 
{\tt \texttt{'}\texttt{xformslib}\texttt{'}}&\\
\cline{1-2}
\end{longtable}

    \index{xformslib \textit{(package)}!xformslib.flpopup \textit{(module)}|)}
