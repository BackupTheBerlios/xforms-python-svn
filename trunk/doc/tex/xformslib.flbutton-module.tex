%
% API Documentation for API Documentation
% Module xformslib.flbutton
%
% Generated by epydoc 3.0.1
% [Thu May 20 23:16:42 2010]
%

%%%%%%%%%%%%%%%%%%%%%%%%%%%%%%%%%%%%%%%%%%%%%%%%%%%%%%%%%%%%%%%%%%%%%%%%%%%
%%                          Module Description                           %%
%%%%%%%%%%%%%%%%%%%%%%%%%%%%%%%%%%%%%%%%%%%%%%%%%%%%%%%%%%%%%%%%%%%%%%%%%%%

    \index{xformslib \textit{(package)}!xformslib.flbutton \textit{(module)}|(}
\section{Module xformslib.flbutton}

    \label{xformslib:flbutton}

xforms-python's functions to manage button objects.

Copyright (C) 2009, 2010  Luca Lazzaroni ``LukenShiro''
e-mail: <\href{mailto:lukenshiro@ngi.it}{lukenshiro@ngi.it}>

This program is free software: you can redistribute it and/or modify
it under the terms of the GNU Lesser General Public License as
published by the Free Software Foundation, version 2.1 of the License.

This program is distributed in the hope that it will be useful,
but WITHOUT ANY WARRANTY; without even the implied warranty of
MERCHANTABILITY or FITNESS FOR A PARTICULAR PURPOSE. See the
GNU Lesser General Public License for more details.

You should have received a copy of the GNU LGPL along with this
program. If not, see <\href{http://www.gnu.org/licenses/}{http://www.gnu.org/licenses/}>.

See CREDITS file to read acknowledgements and thanks to XForms,
ctypes and other developers.

%%%%%%%%%%%%%%%%%%%%%%%%%%%%%%%%%%%%%%%%%%%%%%%%%%%%%%%%%%%%%%%%%%%%%%%%%%%
%%                               Functions                               %%
%%%%%%%%%%%%%%%%%%%%%%%%%%%%%%%%%%%%%%%%%%%%%%%%%%%%%%%%%%%%%%%%%%%%%%%%%%%

  \subsection{Functions}

    \label{xformslib:flbutton:fl_add_roundbutton}
    \index{xformslib \textit{(package)}!xformslib.flbutton \textit{(module)}!xformslib.flbutton.fl\_add\_roundbutton \textit{(function)}}

    \vspace{0.5ex}

\hspace{.8\funcindent}\begin{boxedminipage}{\funcwidth}

    \raggedright \textbf{fl\_add\_roundbutton}(\textit{buttontype}, \textit{x}, \textit{y}, \textit{w}, \textit{h}, \textit{label})

    \vspace{-1.5ex}

    \rule{\textwidth}{0.5\fboxrule}
\setlength{\parskip}{2ex}

Adds a roundbutton object.

-{}-
\setlength{\parskip}{1ex}
      \textbf{Parameters}
      \vspace{-1ex}

      \begin{quote}
        \begin{Ventry}{xxxxxxxxxx}

          \item[buttontype]


type of button object to be added. Values (from xfdata.py)
FL\_NORMAL\_BUTTON, FL\_PUSH\_BUTTON, FL\_RADIO\_BUTTON, FL\_HIDDEN\_BUTTON,
FL\_TOUCH\_BUTTON, FL\_INOUT\_BUTTON, FL\_RETURN\_BUTTON,
FL\_HIDDEN\_RET\_BUTTON, FL\_MENU\_BUTTON, FL\_TOGGLE\_BUTTON
            {\it (type=int)}

          \item[x]


horizontal position (upper-left corner)
            {\it (type=int)}

          \item[y]


vertical position (upper-left corner)
            {\it (type=int)}

          \item[w]


width in coord units
            {\it (type=int)}

          \item[h]


height in coord units
            {\it (type=int)}

          \item[label]


text label of button
            {\it (type=str)}

        \end{Ventry}

      \end{quote}

      \textbf{Return Value}
    \vspace{-1ex}

      \begin{quote}

button object created (pFlObject)
      {\it (type=pointer to xfdata.FL\_OBJECT)}

      \end{quote}

\textbf{Note:} 
e.g. btnobj = fl\_add\_roundbutton(xfdata.FL\_TOGGLE\_BUTTON, 145,
199, 120, 30, ``MyButton'')


\textbf{Status:} 
Tested + Doc + Demo = OK


    \end{boxedminipage}

    \label{xformslib:flbutton:fl_add_round3dbutton}
    \index{xformslib \textit{(package)}!xformslib.flbutton \textit{(module)}!xformslib.flbutton.fl\_add\_round3dbutton \textit{(function)}}

    \vspace{0.5ex}

\hspace{.8\funcindent}\begin{boxedminipage}{\funcwidth}

    \raggedright \textbf{fl\_add\_round3dbutton}(\textit{buttontype}, \textit{x}, \textit{y}, \textit{w}, \textit{h}, \textit{label})

    \vspace{-1.5ex}

    \rule{\textwidth}{0.5\fboxrule}
\setlength{\parskip}{2ex}

Adds a 3D roundbutton object.

-{}-
\setlength{\parskip}{1ex}
      \textbf{Parameters}
      \vspace{-1ex}

      \begin{quote}
        \begin{Ventry}{xxxxxxxxxx}

          \item[buttontype]


type of button object to be added. Values (from xfdata.py)
FL\_NORMAL\_BUTTON, FL\_PUSH\_BUTTON, FL\_RADIO\_BUTTON, FL\_HIDDEN\_BUTTON,
FL\_TOUCH\_BUTTON, FL\_INOUT\_BUTTON, FL\_RETURN\_BUTTON,
FL\_HIDDEN\_RET\_BUTTON, FL\_MENU\_BUTTON, FL\_TOGGLE\_BUTTON
            {\it (type=int)}

          \item[x]


horizontal position (upper-left corner)
            {\it (type=int)}

          \item[y]


vertical position (upper-left corner)
            {\it (type=int)}

          \item[w]


width in coord units
            {\it (type=int)}

          \item[h]


height in coord units
            {\it (type=int)}

          \item[label]


text label of button
            {\it (type=str)}

        \end{Ventry}

      \end{quote}

      \textbf{Return Value}
    \vspace{-1ex}

      \begin{quote}

button object created (pFlObject)
      {\it (type=pointer to xfdata.FL\_OBJECT)}

      \end{quote}

\textbf{Note:} 
e.g. btnobj = fl\_add\_round3dbutton(xfdata.FL\_TOGGLE\_BUTTON, 145,
199, 120, 30, ``MyButton'')


\textbf{Status:} 
Tested + Doc + Demo = OK


    \end{boxedminipage}

    \label{xformslib:flbutton:fl_add_lightbutton}
    \index{xformslib \textit{(package)}!xformslib.flbutton \textit{(module)}!xformslib.flbutton.fl\_add\_lightbutton \textit{(function)}}

    \vspace{0.5ex}

\hspace{.8\funcindent}\begin{boxedminipage}{\funcwidth}

    \raggedright \textbf{fl\_add\_lightbutton}(\textit{buttontype}, \textit{x}, \textit{y}, \textit{w}, \textit{h}, \textit{label})

    \vspace{-1.5ex}

    \rule{\textwidth}{0.5\fboxrule}
\setlength{\parskip}{2ex}

Adds a lightbutton object (with an on/off light switch).

-{}-
\setlength{\parskip}{1ex}
      \textbf{Parameters}
      \vspace{-1ex}

      \begin{quote}
        \begin{Ventry}{xxxxxxxxxx}

          \item[buttontype]


type of button to be added. Values (from xfdata.py) FL\_NORMAL\_BUTTON,
FL\_PUSH\_BUTTON, FL\_RADIO\_BUTTON, FL\_HIDDEN\_BUTTON, FL\_TOUCH\_BUTTON,
FL\_INOUT\_BUTTON, FL\_RETURN\_BUTTON, FL\_HIDDEN\_RET\_BUTTON,
FL\_MENU\_BUTTON, FL\_TOGGLE\_BUTTON
            {\it (type=int)}

          \item[x]


horizontal position (upper-left corner)
            {\it (type=int)}

          \item[y]


vertical position (upper-left corner)
            {\it (type=int)}

          \item[w]


width in coord units
            {\it (type=int)}

          \item[h]


height in coord units
            {\it (type=int)}

          \item[label]


text label of button
            {\it (type=str)}

        \end{Ventry}

      \end{quote}

      \textbf{Return Value}
    \vspace{-1ex}

      \begin{quote}

button object created (pFlObject)
      {\it (type=pointer to xfdata.FL\_OBJECT)}

      \end{quote}

\textbf{Note:} 
e.g. btnobj = fl\_add\_lightbutton(xfdata.FL\_TOGGLE\_BUTTON, 145,
199, 120, 30, ``MyButton'')


\textbf{Status:} 
Tested + Doc + Demo = OK


    \end{boxedminipage}

    \label{xformslib:flbutton:fl_add_checkbutton}
    \index{xformslib \textit{(package)}!xformslib.flbutton \textit{(module)}!xformslib.flbutton.fl\_add\_checkbutton \textit{(function)}}

    \vspace{0.5ex}

\hspace{.8\funcindent}\begin{boxedminipage}{\funcwidth}

    \raggedright \textbf{fl\_add\_checkbutton}(\textit{buttontype}, \textit{x}, \textit{y}, \textit{w}, \textit{h}, \textit{label})

    \vspace{-1.5ex}

    \rule{\textwidth}{0.5\fboxrule}
\setlength{\parskip}{2ex}

Adds a checkbutton object.

-{}-
\setlength{\parskip}{1ex}
      \textbf{Parameters}
      \vspace{-1ex}

      \begin{quote}
        \begin{Ventry}{xxxxxxxxxx}

          \item[buttontype]


type of button object to be added. Values (from xfdata.py)
FL\_NORMAL\_BUTTON, FL\_PUSH\_BUTTON, FL\_RADIO\_BUTTON, FL\_HIDDEN\_BUTTON,
FL\_TOUCH\_BUTTON, FL\_INOUT\_BUTTON, FL\_RETURN\_BUTTON,
FL\_HIDDEN\_RET\_BUTTON, FL\_MENU\_BUTTON, FL\_TOGGLE\_BUTTON
            {\it (type=int)}

          \item[x]


horizontal position (upper-left corner)
            {\it (type=int)}

          \item[y]


vertical position (upper-left corner)
            {\it (type=int)}

          \item[w]


width in coord units
            {\it (type=int)}

          \item[h]


height in coord units
            {\it (type=int)}

          \item[label]


text label of button
            {\it (type=str)}

        \end{Ventry}

      \end{quote}

      \textbf{Return Value}
    \vspace{-1ex}

      \begin{quote}

button object created (pFlObject)
      {\it (type=pointer to xfdata.FL\_OBJECT)}

      \end{quote}

\textbf{Note:} 
e.g. btnobj = fl\_add\_checkbutton(xfdata.FL\_TOGGLE\_BUTTON, 145,
199, 120, 30, ``MyButton'')


\textbf{Status:} 
Tested + Doc + Demo = OK


    \end{boxedminipage}

    \label{xformslib:flbutton:fl_add_button}
    \index{xformslib \textit{(package)}!xformslib.flbutton \textit{(module)}!xformslib.flbutton.fl\_add\_button \textit{(function)}}

    \vspace{0.5ex}

\hspace{.8\funcindent}\begin{boxedminipage}{\funcwidth}

    \raggedright \textbf{fl\_add\_button}(\textit{buttontype}, \textit{x}, \textit{y}, \textit{w}, \textit{h}, \textit{label})

    \vspace{-1.5ex}

    \rule{\textwidth}{0.5\fboxrule}
\setlength{\parskip}{2ex}

Adds a button object to the current form.

-{}-
\setlength{\parskip}{1ex}
      \textbf{Parameters}
      \vspace{-1ex}

      \begin{quote}
        \begin{Ventry}{xxxxxxxxxx}

          \item[buttontype]


type of button to be added. Values (from xfdata.py) FL\_NORMAL\_BUTTON,
FL\_PUSH\_BUTTON, FL\_RADIO\_BUTTON, FL\_HIDDEN\_BUTTON, FL\_TOUCH\_BUTTON,
FL\_INOUT\_BUTTON, FL\_RETURN\_BUTTON, FL\_HIDDEN\_RET\_BUTTON,
FL\_MENU\_BUTTON, FL\_TOGGLE\_BUTTON
            {\it (type=int)}

          \item[x]


horizontal position (upper-left corner)
            {\it (type=int)}

          \item[y]


vertical position (upper-left corner)
            {\it (type=int)}

          \item[w]


width in coord units
            {\it (type=int)}

          \item[h]


height in coord units
            {\it (type=int)}

          \item[label]


text label of button
            {\it (type=str)}

        \end{Ventry}

      \end{quote}

      \textbf{Return Value}
    \vspace{-1ex}

      \begin{quote}

button object created (pFlObject)
      {\it (type=pointer to xfdata.FL\_OBJECT)}

      \end{quote}

\textbf{Note:} 
e.g. btnobj = fl\_add\_button(xfdata.FL\_TOGGLE\_BUTTON, 145,
199, 120, 30, ``MyButton'')


\textbf{Status:} 
Tested + Doc + Demo = OK


    \end{boxedminipage}

    \label{xformslib:flbutton:fl_add_bitmapbutton}
    \index{xformslib \textit{(package)}!xformslib.flbutton \textit{(module)}!xformslib.flbutton.fl\_add\_bitmapbutton \textit{(function)}}

    \vspace{0.5ex}

\hspace{.8\funcindent}\begin{boxedminipage}{\funcwidth}

    \raggedright \textbf{fl\_add\_bitmapbutton}(\textit{buttontype}, \textit{x}, \textit{y}, \textit{w}, \textit{h}, \textit{label})

    \vspace{-1.5ex}

    \rule{\textwidth}{0.5\fboxrule}
\setlength{\parskip}{2ex}

Adds a bitmapbutton object.

-{}-
\setlength{\parskip}{1ex}
      \textbf{Parameters}
      \vspace{-1ex}

      \begin{quote}
        \begin{Ventry}{xxxxxxxxxx}

          \item[buttontype]


type of button to be added. Values (from xfdata.py) FL\_NORMAL\_BUTTON,
FL\_PUSH\_BUTTON, FL\_RADIO\_BUTTON, FL\_HIDDEN\_BUTTON, FL\_TOUCH\_BUTTON,
FL\_INOUT\_BUTTON, FL\_RETURN\_BUTTON, FL\_HIDDEN\_RET\_BUTTON,
FL\_MENU\_BUTTON, FL\_TOGGLE\_BUTTON
            {\it (type=int)}

          \item[x]


horizontal position (upper-left corner)
            {\it (type=int)}

          \item[y]


vertical position (upper-left corner)
            {\it (type=int)}

          \item[w]


width in coord units
            {\it (type=int)}

          \item[h]


height in coord units
            {\it (type=int)}

          \item[label]


text label of button
            {\it (type=str)}

        \end{Ventry}

      \end{quote}

      \textbf{Return Value}
    \vspace{-1ex}

      \begin{quote}

button object created (pFlObject)
      {\it (type=pointer to xfdata.FL\_OBJECT)}

      \end{quote}

\textbf{Note:} 
e.g. btnobj = fl\_add\_bitmapbutton(xfdata.FL\_TOGGLE\_BUTTON, 145,
199, 120, 30, ``MyButton'')


\textbf{Status:} 
Tested + Doc + Demo = OK


    \end{boxedminipage}

    \label{xformslib:flbutton:fl_add_scrollbutton}
    \index{xformslib \textit{(package)}!xformslib.flbutton \textit{(module)}!xformslib.flbutton.fl\_add\_scrollbutton \textit{(function)}}

    \vspace{0.5ex}

\hspace{.8\funcindent}\begin{boxedminipage}{\funcwidth}

    \raggedright \textbf{fl\_add\_scrollbutton}(\textit{buttontype}, \textit{x}, \textit{y}, \textit{w}, \textit{h}, \textit{label})

    \vspace{-1.5ex}

    \rule{\textwidth}{0.5\fboxrule}
\setlength{\parskip}{2ex}

Adds a scrollbutton object.

-{}-
\setlength{\parskip}{1ex}
      \textbf{Parameters}
      \vspace{-1ex}

      \begin{quote}
        \begin{Ventry}{xxxxxxxxxx}

          \item[buttontype]


type of button to be added. Values (from xfdata.py) FL\_NORMAL\_BUTTON,
FL\_PUSH\_BUTTON, FL\_RADIO\_BUTTON, FL\_HIDDEN\_BUTTON, FL\_TOUCH\_BUTTON,
FL\_INOUT\_BUTTON, FL\_RETURN\_BUTTON, FL\_HIDDEN\_RET\_BUTTON,
FL\_MENU\_BUTTON, FL\_TOGGLE\_BUTTON
            {\it (type=int)}

          \item[x]


horizontal position (upper-left corner)
            {\it (type=int)}

          \item[y]


vertical position (upper-left corner)
            {\it (type=int)}

          \item[w]


width in coord units
            {\it (type=int)}

          \item[h]


height in coord units
            {\it (type=int)}

          \item[label]


text label of button
            {\it (type=str)}

        \end{Ventry}

      \end{quote}

      \textbf{Return Value}
    \vspace{-1ex}

      \begin{quote}

button object created (pFlObject)
      {\it (type=pointer to xfdata.FL\_OBJECT)}

      \end{quote}

\textbf{Note:} 
e.g. btnobj = fl\_add\_scrollbutton(xfdata.FL\_TOGGLE\_BUTTON, 145,
199, 120, 30, ``MyButton'')


\textbf{Status:} 
Tested + Doc + NoDemo = OK


    \end{boxedminipage}

    \label{xformslib:flbutton:fl_add_labelbutton}
    \index{xformslib \textit{(package)}!xformslib.flbutton \textit{(module)}!xformslib.flbutton.fl\_add\_labelbutton \textit{(function)}}

    \vspace{0.5ex}

\hspace{.8\funcindent}\begin{boxedminipage}{\funcwidth}

    \raggedright \textbf{fl\_add\_labelbutton}(\textit{buttontype}, \textit{x}, \textit{y}, \textit{w}, \textit{h}, \textit{label})

    \vspace{-1.5ex}

    \rule{\textwidth}{0.5\fboxrule}
\setlength{\parskip}{2ex}

Adds a labelbutton object.

-{}-
\setlength{\parskip}{1ex}
      \textbf{Parameters}
      \vspace{-1ex}

      \begin{quote}
        \begin{Ventry}{xxxxxxxxxx}

          \item[buttontype]


type of button to be added. Values (from xfdata.py) FL\_NORMAL\_BUTTON,
FL\_PUSH\_BUTTON, FL\_RADIO\_BUTTON, FL\_HIDDEN\_BUTTON, FL\_TOUCH\_BUTTON,
FL\_INOUT\_BUTTON, FL\_RETURN\_BUTTON, FL\_HIDDEN\_RET\_BUTTON,
FL\_MENU\_BUTTON, FL\_TOGGLE\_BUTTON
            {\it (type=int)}

          \item[x]


horizontal position (upper-left corner)
            {\it (type=int)}

          \item[y]


vertical position (upper-left corner)
            {\it (type=int)}

          \item[w]


width in coord units
            {\it (type=int)}

          \item[h]


height in coord units
            {\it (type=int)}

          \item[label]


text label of button
            {\it (type=str)}

        \end{Ventry}

      \end{quote}

      \textbf{Return Value}
    \vspace{-1ex}

      \begin{quote}

button object created (pFlObject)
      {\it (type=pointer to xfdata.FL\_OBJECT)}

      \end{quote}

\textbf{Note:} 
e.g. btnobj = fl\_add\_labelbutton(xfdata.FL\_TOGGLE\_BUTTON, 145,
199, 120, 30, ``MyButton'')


\textbf{Status:} 
Tested + Doc + NoDemo = OK


    \end{boxedminipage}

    \label{xformslib:flbutton:fl_set_bitmapbutton_data}
    \index{xformslib \textit{(package)}!xformslib.flbutton \textit{(module)}!xformslib.flbutton.fl\_set\_bitmapbutton\_data \textit{(function)}}

    \vspace{0.5ex}

\hspace{.8\funcindent}\begin{boxedminipage}{\funcwidth}

    \raggedright \textbf{fl\_set\_bitmapbutton\_data}(\textit{pFlObject}, \textit{w}, \textit{h}, \textit{bits})

    \vspace{-1.5ex}

    \rule{\textwidth}{0.5\fboxrule}
\setlength{\parskip}{2ex}

Sets the bitmap to use for a bitmap button, using some data.

-{}-
\setlength{\parskip}{1ex}
      \textbf{Parameters}
      \vspace{-1ex}

      \begin{quote}
        \begin{Ventry}{xxxxxxxxx}

          \item[pFlObject]


button object
            {\it (type=pointer to xfdata.FL\_OBJECT)}

          \item[w]


width in coord units
            {\it (type=int)}

          \item[h]


height in coord units
            {\it (type=int)}

          \item[bits]


bitmap data
            {\it (type=str of ubytes)}

        \end{Ventry}

      \end{quote}

\textbf{Note:} 
e.g. \emph{todo}


\textbf{Status:} 
Untested + Doc + NoDemo = NOT OK


    \end{boxedminipage}

    \label{xformslib:flbutton:fl_add_pixmapbutton}
    \index{xformslib \textit{(package)}!xformslib.flbutton \textit{(module)}!xformslib.flbutton.fl\_add\_pixmapbutton \textit{(function)}}

    \vspace{0.5ex}

\hspace{.8\funcindent}\begin{boxedminipage}{\funcwidth}

    \raggedright \textbf{fl\_add\_pixmapbutton}(\textit{buttontype}, \textit{x}, \textit{y}, \textit{w}, \textit{h}, \textit{label})

    \vspace{-1.5ex}

    \rule{\textwidth}{0.5\fboxrule}
\setlength{\parskip}{2ex}

Adds a pixmapbutton object.

-{}-
\setlength{\parskip}{1ex}
      \textbf{Parameters}
      \vspace{-1ex}

      \begin{quote}
        \begin{Ventry}{xxxxxxxxxx}

          \item[buttontype]


type of button to be added. Values (from xfdata.py) FL\_NORMAL\_BUTTON,
FL\_PUSH\_BUTTON, FL\_RADIO\_BUTTON, FL\_HIDDEN\_BUTTON, FL\_TOUCH\_BUTTON,
FL\_INOUT\_BUTTON, FL\_RETURN\_BUTTON, FL\_HIDDEN\_RET\_BUTTON,
FL\_MENU\_BUTTON, FL\_TOGGLE\_BUTTON
            {\it (type=int)}

          \item[x]


horizontal position (upper-left corner)
            {\it (type=int)}

          \item[y]


vertical position (upper-left corner)
            {\it (type=int)}

          \item[w]


width in coord units
            {\it (type=int)}

          \item[h]


height in coord units
            {\it (type=int)}

          \item[label]


text label of button
            {\it (type=str)}

        \end{Ventry}

      \end{quote}

      \textbf{Return Value}
    \vspace{-1ex}

      \begin{quote}

button object created (pFlObject)
      {\it (type=pointer to xfdata.FL\_OBJECT)}

      \end{quote}

\textbf{Note:} 
e.g. btnobj = fl\_add\_roundbutton(xfdata.FL\_TOGGLE\_BUTTON, 145,
199, 120, 30, ``MyButton'')


\textbf{Status:} 
Tested + Doc + Demo = OK


    \end{boxedminipage}

    \label{xformslib:flbutton:fl_set_pixmapbutton_focus_outline}
    \index{xformslib \textit{(package)}!xformslib.flbutton \textit{(module)}!xformslib.flbutton.fl\_set\_pixmapbutton\_focus\_outline \textit{(function)}}

    \vspace{0.5ex}

\hspace{.8\funcindent}\begin{boxedminipage}{\funcwidth}

    \raggedright \textbf{fl\_set\_pixmapbutton\_focus\_outline}(\textit{pFlObject}, \textit{yesno})

    \vspace{-1.5ex}

    \rule{\textwidth}{0.5\fboxrule}
\setlength{\parskip}{2ex}

Enables or disables the focus outline of the pixmap button.

-{}-
\setlength{\parskip}{1ex}
      \textbf{Parameters}
      \vspace{-1ex}

      \begin{quote}
        \begin{Ventry}{xxxxxxxxx}

          \item[pFlObject]


button object
            {\it (type=pointer to xfdata.FL\_OBJECT)}

          \item[yesno]


flag to enable (1) or disable (0) the focus outline
            {\it (type=int)}

        \end{Ventry}

      \end{quote}

\textbf{Note:} 
e.g. fl\_set\_pixmapbutton\_focus\_outline(pmapobj, 1)


\textbf{Status:} 
Tested + Doc + NoDemo = OK


    \end{boxedminipage}

    \label{xformslib:flbutton:fl_set_pixmapbutton_focus_outline}
    \index{xformslib \textit{(package)}!xformslib.flbutton \textit{(module)}!xformslib.flbutton.fl\_set\_pixmapbutton\_focus\_outline \textit{(function)}}

    \vspace{0.5ex}

\hspace{.8\funcindent}\begin{boxedminipage}{\funcwidth}

    \raggedright \textbf{fl\_set\_pixmapbutton\_show\_focus}(\textit{pFlObject}, \textit{yesno})

    \vspace{-1.5ex}

    \rule{\textwidth}{0.5\fboxrule}
\setlength{\parskip}{2ex}

Enables or disables the focus outline of the pixmap button.

-{}-
\setlength{\parskip}{1ex}
      \textbf{Parameters}
      \vspace{-1ex}

      \begin{quote}
        \begin{Ventry}{xxxxxxxxx}

          \item[pFlObject]


button object
            {\it (type=pointer to xfdata.FL\_OBJECT)}

          \item[yesno]


flag to enable (1) or disable (0) the focus outline
            {\it (type=int)}

        \end{Ventry}

      \end{quote}

\textbf{Note:} 
e.g. fl\_set\_pixmapbutton\_focus\_outline(pmapobj, 1)


\textbf{Status:} 
Tested + Doc + NoDemo = OK


    \end{boxedminipage}

    \label{xformslib:flbutton:fl_set_pixmapbutton_focus_data}
    \index{xformslib \textit{(package)}!xformslib.flbutton \textit{(module)}!xformslib.flbutton.fl\_set\_pixmapbutton\_focus\_data \textit{(function)}}

    \vspace{0.5ex}

\hspace{.8\funcindent}\begin{boxedminipage}{\funcwidth}

    \raggedright \textbf{fl\_set\_pixmapbutton\_focus\_data}(\textit{pFlObject}, \textit{bits})

    \vspace{-1.5ex}

    \rule{\textwidth}{0.5\fboxrule}
\setlength{\parskip}{2ex}

Sets a different pixmap to show, using data, when the mouse enters
a pixmap button, instead of an outline of the button.

-{}-
\setlength{\parskip}{1ex}
      \textbf{Parameters}
      \vspace{-1ex}

      \begin{quote}
        \begin{Ventry}{xxxxxxxxx}

          \item[pFlObject]


button object
            {\it (type=pointer to xfdata.FL\_OBJECT)}

          \item[bits]


pixmap data
            {\it (type=str of ubytes ?)}

        \end{Ventry}

      \end{quote}

\textbf{Status:} 
Untested + Doc + NoDemo = NOT OK


    \end{boxedminipage}

    \label{xformslib:flbutton:fl_set_pixmapbutton_focus_file}
    \index{xformslib \textit{(package)}!xformslib.flbutton \textit{(module)}!xformslib.flbutton.fl\_set\_pixmapbutton\_focus\_file \textit{(function)}}

    \vspace{0.5ex}

\hspace{.8\funcindent}\begin{boxedminipage}{\funcwidth}

    \raggedright \textbf{fl\_set\_pixmapbutton\_focus\_file}(\textit{pFlObject}, \textit{fname})

    \vspace{-1.5ex}

    \rule{\textwidth}{0.5\fboxrule}
\setlength{\parskip}{2ex}

Sets a different pixmap to show, using a file, when the mouse enters
a pixmap button, instead of an outline of the button.

-{}-
\setlength{\parskip}{1ex}
      \textbf{Parameters}
      \vspace{-1ex}

      \begin{quote}
        \begin{Ventry}{xxxxxxxxx}

          \item[pFlObject]


button object
            {\it (type=pointer to xfdata.FL\_OBJECT)}

          \item[fname]


name of file containing pixmap resource
            {\it (type=str)}

        \end{Ventry}

      \end{quote}

\textbf{Status:} 
Untested + Doc + NoDemo = NOT OK


    \end{boxedminipage}

    \label{xformslib:flbutton:fl_set_pixmapbutton_focus_pixmap}
    \index{xformslib \textit{(package)}!xformslib.flbutton \textit{(module)}!xformslib.flbutton.fl\_set\_pixmapbutton\_focus\_pixmap \textit{(function)}}

    \vspace{0.5ex}

\hspace{.8\funcindent}\begin{boxedminipage}{\funcwidth}

    \raggedright \textbf{fl\_set\_pixmapbutton\_focus\_pixmap}(\textit{pFlObject}, \textit{pix}, \textit{mask})

    \vspace{-1.5ex}

    \rule{\textwidth}{0.5\fboxrule}
\setlength{\parskip}{2ex}

Sets a different pixmap to show, using pixmap id, when the mouse
enters a pixmap button, instead of an outline of the button.

-{}-
\setlength{\parskip}{1ex}
      \textbf{Parameters}
      \vspace{-1ex}

      \begin{quote}
        \begin{Ventry}{xxxxxxxxx}

          \item[pFlObject]


button object
            {\it (type=pointer to xfdata.FL\_OBJECT)}

          \item[pix]


pixmap id
            {\it (type=long\_pos)}

          \item[mask]


pixmap id
            {\it (type=long\_pos)}

        \end{Ventry}

      \end{quote}

\textbf{Note:} 
e.g. \emph{todo}


\textbf{Status:} 
Untested + Doc + NoDemo = NOT OK


    \end{boxedminipage}

    \label{xformslib:flbutton:fl_get_button}
    \index{xformslib \textit{(package)}!xformslib.flbutton \textit{(module)}!xformslib.flbutton.fl\_get\_button \textit{(function)}}

    \vspace{0.5ex}

\hspace{.8\funcindent}\begin{boxedminipage}{\funcwidth}

    \raggedright \textbf{fl\_get\_button}(\textit{pFlObject})

    \vspace{-1.5ex}

    \rule{\textwidth}{0.5\fboxrule}
\setlength{\parskip}{2ex}

Returns the state value of the button.

-{}-
\setlength{\parskip}{1ex}
      \textbf{Parameters}
      \vspace{-1ex}

      \begin{quote}
        \begin{Ventry}{xxxxxxxxx}

          \item[pFlObject]


button object
            {\it (type=pointer to xfdata.FL\_OBJECT)}

        \end{Ventry}

      \end{quote}

      \textbf{Return Value}
    \vspace{-1ex}

      \begin{quote}

0 (not pushed) or 1 (pushed)
      {\it (type=int)}

      \end{quote}

\textbf{Note:} 
e.g. btnstate = fl\_get\_button(btnobj)


\textbf{Status:} 
Tested + Doc + Demo = OK


    \end{boxedminipage}

    \label{xformslib:flbutton:fl_set_button}
    \index{xformslib \textit{(package)}!xformslib.flbutton \textit{(module)}!xformslib.flbutton.fl\_set\_button \textit{(function)}}

    \vspace{0.5ex}

\hspace{.8\funcindent}\begin{boxedminipage}{\funcwidth}

    \raggedright \textbf{fl\_set\_button}(\textit{pFlObject}, \textit{yesno})

    \vspace{-1.5ex}

    \rule{\textwidth}{0.5\fboxrule}
\setlength{\parskip}{2ex}

Sets the button state (not pushed/pushed).

-{}-
\setlength{\parskip}{1ex}
      \textbf{Parameters}
      \vspace{-1ex}

      \begin{quote}
        \begin{Ventry}{xxxxxxxxx}

          \item[pFlObject]


button object
            {\it (type=pointer to xfdata.FL\_OBJECT)}

          \item[yesno]


state of button to be set. Values 0 (if not pushed) or 1 (if pushed)
            {\it (type=int)}

        \end{Ventry}

      \end{quote}

\textbf{Note:} 
e.g. fl\_set\_button(btnobj, 1)


\textbf{Status:} 
Tested + Doc + Demo = OK


    \end{boxedminipage}

    \label{xformslib:flbutton:fl_get_button_numb}
    \index{xformslib \textit{(package)}!xformslib.flbutton \textit{(module)}!xformslib.flbutton.fl\_get\_button\_numb \textit{(function)}}

    \vspace{0.5ex}

\hspace{.8\funcindent}\begin{boxedminipage}{\funcwidth}

    \raggedright \textbf{fl\_get\_button\_numb}(\textit{pFlObject})

    \vspace{-1.5ex}

    \rule{\textwidth}{0.5\fboxrule}
\setlength{\parskip}{2ex}

Returns the number of the last used mouse button. fl\_mouse\_button()
function will also return the mouse number.

-{}-
\setlength{\parskip}{1ex}
      \textbf{Parameters}
      \vspace{-1ex}

      \begin{quote}
        \begin{Ventry}{xxxxxxxxx}

          \item[pFlObject]


button object
            {\it (type=pointer to xfdata.FL\_OBJECT)}

        \end{Ventry}

      \end{quote}

      \textbf{Return Value}
    \vspace{-1ex}

      \begin{quote}

num.
      {\it (type=int)}

      \end{quote}

\textbf{Note:} 
e.g. lastused = fl\_get\_button\_numb(pobj)


\textbf{Status:} 
Tested + Doc + NoDemo = OK


    \end{boxedminipage}

    \label{xformslib:flbutton:fl_create_generic_button}
    \index{xformslib \textit{(package)}!xformslib.flbutton \textit{(module)}!xformslib.flbutton.fl\_create\_generic\_button \textit{(function)}}

    \vspace{0.5ex}

\hspace{.8\funcindent}\begin{boxedminipage}{\funcwidth}

    \raggedright \textbf{fl\_create\_generic\_button}(\textit{btnclass}, \textit{buttontype}, \textit{x}, \textit{y}, \textit{w}, \textit{h}, \textit{label})

    \vspace{-1.5ex}

    \rule{\textwidth}{0.5\fboxrule}
\setlength{\parskip}{2ex}

Creates a generic button object.

-{}-
\setlength{\parskip}{1ex}
      \textbf{Parameters}
      \vspace{-1ex}

      \begin{quote}
        \begin{Ventry}{xxxxxxxxxx}

          \item[btnclass]


value of a new button class
            {\it (type=int)}

          \item[buttontype]


type of button to be created. Values (from xfdata.py) FL\_NORMAL\_BUTTON,
FL\_PUSH\_BUTTON, FL\_RADIO\_BUTTON, FL\_HIDDEN\_BUTTON, FL\_TOUCH\_BUTTON,
FL\_INOUT\_BUTTON, FL\_RETURN\_BUTTON, FL\_HIDDEN\_RET\_BUTTON,
FL\_MENU\_BUTTON, FL\_TOGGLE\_BUTTON
            {\it (type=int)}

          \item[x]


horizontal position (upper-left corner)
            {\it (type=int)}

          \item[y]


vertical position (upper-left corner)
            {\it (type=int)}

          \item[w]


width in coord units
            {\it (type=int)}

          \item[h]


height in coord units
            {\it (type=int)}

          \item[label]


text label of button
            {\it (type=str)}

        \end{Ventry}

      \end{quote}

      \textbf{Return Value}
    \vspace{-1ex}

      \begin{quote}

button object created (pFlObject)
      {\it (type=pointer to xfdata.FL\_OBJECT)}

      \end{quote}

\textbf{Note:} 
e.g. newbtnobj = fl\_add\_roundbutton(1001, xfdata.FL\_TOGGLE\_BUTTON,
145, 199, 120, 30, ``MyButton'')


\textbf{Status:} 
Tested + Doc + NoDemo = OK


    \end{boxedminipage}

    \label{xformslib:flbutton:fl_add_button_class}
    \index{xformslib \textit{(package)}!xformslib.flbutton \textit{(module)}!xformslib.flbutton.fl\_add\_button\_class \textit{(function)}}

    \vspace{0.5ex}

\hspace{.8\funcindent}\begin{boxedminipage}{\funcwidth}

    \raggedright \textbf{fl\_add\_button\_class}(\textit{btnclass}, \textit{py\_DrawButton}, \textit{py\_CleanupButton})

    \vspace{-1.5ex}

    \rule{\textwidth}{0.5\fboxrule}
\setlength{\parskip}{2ex}

Associates a button class with a drawing function.

-{}-
\setlength{\parskip}{1ex}
      \textbf{Parameters}
      \vspace{-1ex}

      \begin{quote}
        \begin{Ventry}{xxxxxxxxxxxxxxxx}

          \item[btnclass]


value of a new button class
            {\it (type=int)}

          \item[py\_DrawButton]


name referring to function(pFlObject)
            {\it (type=python function to draw button, no return)}

          \item[py\_CleanupButton]


name referring to function(pButtonSpec)
            {\it (type=python function to cleanup button, no return)}

        \end{Ventry}

      \end{quote}

\textbf{Notes:}
\begin{quote}
  \begin{itemize}

  \item
    \setlength{\parskip}{0.6ex}

e.g. def drawbtn(pobj): > ...


  \item 
e.g. def cleanbtn(buttonspec): > ...


  \item 
e.g. fl\_add\_button\_class(1001, drawbtn, cleanbtn)


\end{itemize}

\end{quote}

\textbf{Status:} 
Tested + Doc + NoDemo = OK


    \end{boxedminipage}

    \label{xformslib:flbutton:fl_set_button_mouse_buttons}
    \index{xformslib \textit{(package)}!xformslib.flbutton \textit{(module)}!xformslib.flbutton.fl\_set\_button\_mouse\_buttons \textit{(function)}}

    \vspace{0.5ex}

\hspace{.8\funcindent}\begin{boxedminipage}{\funcwidth}

    \raggedright \textbf{fl\_set\_button\_mouse\_buttons}(\textit{pFlObject}, \textit{buttons})

    \vspace{-1.5ex}

    \rule{\textwidth}{0.5\fboxrule}
\setlength{\parskip}{2ex}

Sets up to which mouse buttons the button object will react.

-{}-
\setlength{\parskip}{1ex}
      \textbf{Parameters}
      \vspace{-1ex}

      \begin{quote}
        \begin{Ventry}{xxxxxxxxx}

          \item[pFlObject]


button object
            {\it (type=pointer to xfdata.FL\_OBJECT)}

          \item[buttons]


value of mouse buttons to be set. Values bitwise OR of the numbers 1
for the left mouse button, 2 for the middle, 4 for the right mouse
button, 8 for moving the scroll wheel up ``button'' and 16 for scrolling
down ``button''.
            {\it (type=int\_pos)}

        \end{Ventry}

      \end{quote}

\textbf{Note:} 
e.g. fl\_set\_button\_mouse\_buttons(pobj, 8|16)


\textbf{Status:} 
Tested + Doc + NoDemo = OK


    \end{boxedminipage}

    \label{xformslib:flbutton:fl_get_button_mouse_buttons}
    \index{xformslib \textit{(package)}!xformslib.flbutton \textit{(module)}!xformslib.flbutton.fl\_get\_button\_mouse\_buttons \textit{(function)}}

    \vspace{0.5ex}

\hspace{.8\funcindent}\begin{boxedminipage}{\funcwidth}

    \raggedright \textbf{fl\_get\_button\_mouse\_buttons}(\textit{pFlObject})

    \vspace{-1.5ex}

    \rule{\textwidth}{0.5\fboxrule}
\setlength{\parskip}{2ex}

Returns a value indicating which mouse buttons the button object will
react to (bitwise OR of the numbers 1 for the left mouse button, 2 for
the middle, 4 for the right mouse button, 8 for moving the scroll wheel
up ``button'' and 16 for scrolling down ``button'').

-{}-
\setlength{\parskip}{1ex}
      \textbf{Parameters}
      \vspace{-1ex}

      \begin{quote}
        \begin{Ventry}{xxxxxxxxx}

          \item[pFlObject]


button object
            {\it (type=pointer to xfdata.FL\_OBJECT)}

        \end{Ventry}

      \end{quote}

      \textbf{Return Value}
    \vspace{-1ex}

      \begin{quote}

buttons value
      {\it (type=int\_pos)}

      \end{quote}

\textbf{Note:} 
e.g. moubtn = fl\_get\_button\_mouse\_buttons(pobj)


\textbf{Attention:} 
API change from XForms - upstream was
fl\_get\_button\_mouse\_buttons(pFlObject, buttons)


\textbf{Status:} 
Tested + Doc + NoDemo = OK


    \end{boxedminipage}


%%%%%%%%%%%%%%%%%%%%%%%%%%%%%%%%%%%%%%%%%%%%%%%%%%%%%%%%%%%%%%%%%%%%%%%%%%%
%%                               Variables                               %%
%%%%%%%%%%%%%%%%%%%%%%%%%%%%%%%%%%%%%%%%%%%%%%%%%%%%%%%%%%%%%%%%%%%%%%%%%%%

  \subsection{Variables}

    \vspace{-1cm}
\hspace{\varindent}\begin{longtable}{|p{\varnamewidth}|p{\vardescrwidth}|l}
\cline{1-2}
\cline{1-2} \centering \textbf{Name} & \centering \textbf{Description}& \\
\cline{1-2}
\endhead\cline{1-2}\multicolumn{3}{r}{\small\textit{continued on next page}}\\\endfoot\cline{1-2}
\endlastfoot\raggedright \_\-\_\-p\-a\-c\-k\-a\-g\-e\-\_\-\_\- & \raggedright \textbf{Value:} 
{\tt \texttt{'}\texttt{xformslib}\texttt{'}}&\\
\cline{1-2}
\end{longtable}

    \index{xformslib \textit{(package)}!xformslib.flbutton \textit{(module)}|)}
