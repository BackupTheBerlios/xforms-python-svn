%
% API Documentation for API Documentation
% Module xformslib.library
%
% Generated by epydoc 3.0.1
% [Sat Jan 16 18:52:50 2010]
%

%%%%%%%%%%%%%%%%%%%%%%%%%%%%%%%%%%%%%%%%%%%%%%%%%%%%%%%%%%%%%%%%%%%%%%%%%%%
%%                          Module Description                           %%
%%%%%%%%%%%%%%%%%%%%%%%%%%%%%%%%%%%%%%%%%%%%%%%%%%%%%%%%%%%%%%%%%%%%%%%%%%%

    \index{xformslib \textit{(package)}!xformslib.library \textit{(module)}|(}
\section{Module xformslib.library}

    \label{xformslib:library}
***** xforms-python *****

Python wrapper for XForms (X11) GUI C toolkit library using ctypes

Copyright (C) 2009  Luca Lazzaroni "LukenShiro"  
{\textless}lukenshiro@ngi.it{\textgreater}

This program is free software: you can redistribute it and/or modify it 
under the terms of the GNU Lesser General Public License as published by 
the Free Software Foundation, version 2.1 of the License.

This program is distributed in the hope that it will be useful, but WITHOUT
ANY WARRANTY; without even the implied warranty of MERCHANTABILITY or 
FITNESS FOR A PARTICULAR PURPOSE. See the GNU Lesser General Public License
for more details.

You should have received a copy of the GNU LGPL along with this program. If
not, see {\textless}http://www.gnu.org/licenses/{\textgreater}.

See CREDITS file to read acknowledgements and thanks to XForms, ctypes and 
other developers.

*****************************************************************

\textbf{Version:} 0.3.5\_1.0.93pre3




%%%%%%%%%%%%%%%%%%%%%%%%%%%%%%%%%%%%%%%%%%%%%%%%%%%%%%%%%%%%%%%%%%%%%%%%%%%
%%                               Functions                               %%
%%%%%%%%%%%%%%%%%%%%%%%%%%%%%%%%%%%%%%%%%%%%%%%%%%%%%%%%%%%%%%%%%%%%%%%%%%%

  \subsection{Functions}

    \label{xformslib:library:get_xforms_version}
    \index{xformslib \textit{(package)}!xformslib.library \textit{(module)}!xformslib.library.get\_xforms\_version \textit{(function)}}

    \vspace{0.5ex}

\hspace{.8\funcindent}\begin{boxedminipage}{\funcwidth}

    \raggedright \textbf{get\_xforms\_version}()

    \vspace{-1.5ex}

    \rule{\textwidth}{0.5\fboxrule}
\setlength{\parskip}{2ex}
    Returns version string of installed XForms library/header

\setlength{\parskip}{1ex}
    \end{boxedminipage}

    \label{xformslib:library:verify_version_compatibility}
    \index{xformslib \textit{(package)}!xformslib.library \textit{(module)}!xformslib.library.verify\_version\_compatibility \textit{(function)}}

    \vspace{0.5ex}

\hspace{.8\funcindent}\begin{boxedminipage}{\funcwidth}

    \raggedright \textbf{verify\_version\_compatibility}()

    \vspace{-1.5ex}

    \rule{\textwidth}{0.5\fboxrule}
\setlength{\parskip}{2ex}
    verify compatibility between xforms-python and XForms versions

\setlength{\parskip}{1ex}
    \end{boxedminipage}

    \label{xformslib:library:func_notexisting_placeholder}
    \index{xformslib \textit{(package)}!xformslib.library \textit{(module)}!xformslib.library.func\_notexisting\_placeholder \textit{(function)}}

    \vspace{0.5ex}

\hspace{.8\funcindent}\begin{boxedminipage}{\funcwidth}

    \raggedright \textbf{func\_notexisting\_placeholder}(\textit{cfunction})

    \vspace{-1.5ex}

    \rule{\textwidth}{0.5\fboxrule}
\setlength{\parskip}{2ex}
    Print a warning if called function doesn't exist

\setlength{\parskip}{1ex}
    \end{boxedminipage}

    \label{xformslib:library:keep_cfunc_refs}
    \index{xformslib \textit{(package)}!xformslib.library \textit{(module)}!xformslib.library.keep\_cfunc\_refs \textit{(function)}}

    \vspace{0.5ex}

\hspace{.8\funcindent}\begin{boxedminipage}{\funcwidth}

    \raggedright \textbf{keep\_cfunc\_refs}(*\textit{cfunclist})

    \vspace{-1.5ex}

    \rule{\textwidth}{0.5\fboxrule}
\setlength{\parskip}{2ex}
    Adds a reference for \_cfunc\_refs list of values

\setlength{\parskip}{1ex}
    \end{boxedminipage}

    \label{xformslib:library:keep_elem_refs}
    \index{xformslib \textit{(package)}!xformslib.library \textit{(module)}!xformslib.library.keep\_elem\_refs \textit{(function)}}

    \vspace{0.5ex}

\hspace{.8\funcindent}\begin{boxedminipage}{\funcwidth}

    \raggedright \textbf{keep\_elem\_refs}(*\textit{elemlist})

    \vspace{-1.5ex}

    \rule{\textwidth}{0.5\fboxrule}
\setlength{\parskip}{2ex}
    Adds a reference for \_elem\_refs list of values

\setlength{\parskip}{1ex}
    \end{boxedminipage}

    \label{xformslib:library:load_so_libforms}
    \index{xformslib \textit{(package)}!xformslib.library \textit{(module)}!xformslib.library.load\_so\_libforms \textit{(function)}}

    \vspace{0.5ex}

\hspace{.8\funcindent}\begin{boxedminipage}{\funcwidth}

    \raggedright \textbf{load\_so\_libforms}()

    \vspace{-1.5ex}

    \rule{\textwidth}{0.5\fboxrule}
\setlength{\parskip}{2ex}
    Load libforms.so else raise an error -{\textgreater} library instance

\setlength{\parskip}{1ex}
    \end{boxedminipage}

    \label{xformslib:library:load_so_libflimage}
    \index{xformslib \textit{(package)}!xformslib.library \textit{(module)}!xformslib.library.load\_so\_libflimage \textit{(function)}}

    \vspace{0.5ex}

\hspace{.8\funcindent}\begin{boxedminipage}{\funcwidth}

    \raggedright \textbf{load\_so\_libflimage}()

    \vspace{-1.5ex}

    \rule{\textwidth}{0.5\fboxrule}
\setlength{\parskip}{2ex}
    Load libflimage.so else raise an error -{\textgreater} library instance

\setlength{\parskip}{1ex}
    \end{boxedminipage}

    \label{xformslib:library:load_so_libformsgl}
    \index{xformslib \textit{(package)}!xformslib.library \textit{(module)}!xformslib.library.load\_so\_libformsgl \textit{(function)}}

    \vspace{0.5ex}

\hspace{.8\funcindent}\begin{boxedminipage}{\funcwidth}

    \raggedright \textbf{load\_so\_libformsgl}()

    \vspace{-1.5ex}

    \rule{\textwidth}{0.5\fboxrule}
\setlength{\parskip}{2ex}
    Load libformsGL.so else raise an error -{\textgreater} library instance

\setlength{\parskip}{1ex}
    \end{boxedminipage}

    \label{xformslib:library:load_so_libx11}
    \index{xformslib \textit{(package)}!xformslib.library \textit{(module)}!xformslib.library.load\_so\_libx11 \textit{(function)}}

    \vspace{0.5ex}

\hspace{.8\funcindent}\begin{boxedminipage}{\funcwidth}

    \raggedright \textbf{load\_so\_libx11}()

    \vspace{-1.5ex}

    \rule{\textwidth}{0.5\fboxrule}
\setlength{\parskip}{2ex}
    Load libX11.so.6 else raise an error -{\textgreater} library instance

\setlength{\parskip}{1ex}
    \end{boxedminipage}

    \label{xformslib:library:cfuncproto}
    \index{xformslib \textit{(package)}!xformslib.library \textit{(module)}!xformslib.library.cfuncproto \textit{(function)}}

    \vspace{0.5ex}

\hspace{.8\funcindent}\begin{boxedminipage}{\funcwidth}

    \raggedright \textbf{cfuncproto}(\textit{library}, \textit{cfuncname}, \textit{retval}, \textit{arglist}, \textit{doc}={\tt \texttt{'}\texttt{}\texttt{'}})

    \vspace{-1.5ex}

    \rule{\textwidth}{0.5\fboxrule}
\setlength{\parskip}{2ex}
    Prototype for C functions to be wrapped in python

\setlength{\parskip}{1ex}
    \end{boxedminipage}

    \label{xformslib:library:check_if_initialized}
    \index{xformslib \textit{(package)}!xformslib.library \textit{(module)}!xformslib.library.check\_if\_initialized \textit{(function)}}

    \vspace{0.5ex}

\hspace{.8\funcindent}\begin{boxedminipage}{\funcwidth}

    \raggedright \textbf{check\_if\_initialized}()

    \vspace{-1.5ex}

    \rule{\textwidth}{0.5\fboxrule}
\setlength{\parskip}{2ex}
    Check if fl\_initialize() has been called before caller function. 
    Needed for most functions, except those supposed to be used *BEFORE* 
    initialization.

\setlength{\parskip}{1ex}
    \end{boxedminipage}

    \label{xformslib:library:convert_to_string}
    \index{xformslib \textit{(package)}!xformslib.library \textit{(module)}!xformslib.library.convert\_to\_string \textit{(function)}}

    \vspace{0.5ex}

\hspace{.8\funcindent}\begin{boxedminipage}{\funcwidth}

    \raggedright \textbf{convert\_to\_string}(\textit{paramname})

    \vspace{-1.5ex}

    \rule{\textwidth}{0.5\fboxrule}
\setlength{\parskip}{2ex}
    Converts paramname to python str and to ctypes c\_char\_p

\setlength{\parskip}{1ex}
    \end{boxedminipage}

    \label{xformslib:library:convert_to_int}
    \index{xformslib \textit{(package)}!xformslib.library \textit{(module)}!xformslib.library.convert\_to\_int \textit{(function)}}

    \vspace{0.5ex}

\hspace{.8\funcindent}\begin{boxedminipage}{\funcwidth}

    \raggedright \textbf{convert\_to\_int}(\textit{paramname})

    \vspace{-1.5ex}

    \rule{\textwidth}{0.5\fboxrule}
\setlength{\parskip}{2ex}
    Converts paramname to python int and to ctypes c\_int

\setlength{\parskip}{1ex}
    \end{boxedminipage}

    \label{xformslib:library:convert_to_int}
    \index{xformslib \textit{(package)}!xformslib.library \textit{(module)}!xformslib.library.convert\_to\_int \textit{(function)}}

    \vspace{0.5ex}

\hspace{.8\funcindent}\begin{boxedminipage}{\funcwidth}

    \raggedright \textbf{convert\_to\_FL\_Coord}(\textit{paramname})

    \vspace{-1.5ex}

    \rule{\textwidth}{0.5\fboxrule}
\setlength{\parskip}{2ex}
    Converts paramname to python int and to ctypes c\_int

\setlength{\parskip}{1ex}
    \end{boxedminipage}

    \label{xformslib:library:convert_to_uint}
    \index{xformslib \textit{(package)}!xformslib.library \textit{(module)}!xformslib.library.convert\_to\_uint \textit{(function)}}

    \vspace{0.5ex}

\hspace{.8\funcindent}\begin{boxedminipage}{\funcwidth}

    \raggedright \textbf{convert\_to\_uint}(\textit{paramname})

    \vspace{-1.5ex}

    \rule{\textwidth}{0.5\fboxrule}
\setlength{\parskip}{2ex}
    Converts paramname to python int and to ctypes c\_uint

\setlength{\parskip}{1ex}
    \end{boxedminipage}

    \label{xformslib:library:convert_to_long}
    \index{xformslib \textit{(package)}!xformslib.library \textit{(module)}!xformslib.library.convert\_to\_long \textit{(function)}}

    \vspace{0.5ex}

\hspace{.8\funcindent}\begin{boxedminipage}{\funcwidth}

    \raggedright \textbf{convert\_to\_long}(\textit{paramname})

    \vspace{-1.5ex}

    \rule{\textwidth}{0.5\fboxrule}
\setlength{\parskip}{2ex}
    Converts paramname to python long and to ctypes c\_long

\setlength{\parskip}{1ex}
    \end{boxedminipage}

    \label{xformslib:library:convert_to_ulong}
    \index{xformslib \textit{(package)}!xformslib.library \textit{(module)}!xformslib.library.convert\_to\_ulong \textit{(function)}}

    \vspace{0.5ex}

\hspace{.8\funcindent}\begin{boxedminipage}{\funcwidth}

    \raggedright \textbf{convert\_to\_ulong}(\textit{paramname})

    \vspace{-1.5ex}

    \rule{\textwidth}{0.5\fboxrule}
\setlength{\parskip}{2ex}
    Converts paramname to python long and to ctypes c\_ulong

\setlength{\parskip}{1ex}
    \end{boxedminipage}

    \label{xformslib:library:convert_to_ulong}
    \index{xformslib \textit{(package)}!xformslib.library \textit{(module)}!xformslib.library.convert\_to\_ulong \textit{(function)}}

    \vspace{0.5ex}

\hspace{.8\funcindent}\begin{boxedminipage}{\funcwidth}

    \raggedright \textbf{convert\_to\_FL\_COLOR}(\textit{paramname})

    \vspace{-1.5ex}

    \rule{\textwidth}{0.5\fboxrule}
\setlength{\parskip}{2ex}
    Converts paramname to python long and to ctypes c\_ulong

\setlength{\parskip}{1ex}
    \end{boxedminipage}

    \label{xformslib:library:convert_to_ulong}
    \index{xformslib \textit{(package)}!xformslib.library \textit{(module)}!xformslib.library.convert\_to\_ulong \textit{(function)}}

    \vspace{0.5ex}

\hspace{.8\funcindent}\begin{boxedminipage}{\funcwidth}

    \raggedright \textbf{convert\_to\_Window}(\textit{paramname})

    \vspace{-1.5ex}

    \rule{\textwidth}{0.5\fboxrule}
\setlength{\parskip}{2ex}
    Converts paramname to python long and to ctypes c\_ulong

\setlength{\parskip}{1ex}
    \end{boxedminipage}

    \label{xformslib:library:convert_to_ulong}
    \index{xformslib \textit{(package)}!xformslib.library \textit{(module)}!xformslib.library.convert\_to\_ulong \textit{(function)}}

    \vspace{0.5ex}

\hspace{.8\funcindent}\begin{boxedminipage}{\funcwidth}

    \raggedright \textbf{convert\_to\_Pixmap}(\textit{paramname})

    \vspace{-1.5ex}

    \rule{\textwidth}{0.5\fboxrule}
\setlength{\parskip}{2ex}
    Converts paramname to python long and to ctypes c\_ulong

\setlength{\parskip}{1ex}
    \end{boxedminipage}

    \label{xformslib:library:convert_to_double}
    \index{xformslib \textit{(package)}!xformslib.library \textit{(module)}!xformslib.library.convert\_to\_double \textit{(function)}}

    \vspace{0.5ex}

\hspace{.8\funcindent}\begin{boxedminipage}{\funcwidth}

    \raggedright \textbf{convert\_to\_double}(\textit{paramname})

    \vspace{-1.5ex}

    \rule{\textwidth}{0.5\fboxrule}
\setlength{\parskip}{2ex}
    Converts paramname to python float and to ctypes c\_double

\setlength{\parskip}{1ex}
    \end{boxedminipage}

    \label{xformslib:library:convert_to_float}
    \index{xformslib \textit{(package)}!xformslib.library \textit{(module)}!xformslib.library.convert\_to\_float \textit{(function)}}

    \vspace{0.5ex}

\hspace{.8\funcindent}\begin{boxedminipage}{\funcwidth}

    \raggedright \textbf{convert\_to\_float}(\textit{paramname})

    \vspace{-1.5ex}

    \rule{\textwidth}{0.5\fboxrule}
\setlength{\parskip}{2ex}
    Converts paramname to python float and to ctypes c\_float

\setlength{\parskip}{1ex}
    \end{boxedminipage}

    \label{xformslib:library:convert_to_ubyte}
    \index{xformslib \textit{(package)}!xformslib.library \textit{(module)}!xformslib.library.convert\_to\_ubyte \textit{(function)}}

    \vspace{0.5ex}

\hspace{.8\funcindent}\begin{boxedminipage}{\funcwidth}

    \raggedright \textbf{convert\_to\_ubyte}(\textit{paramname})

    \vspace{-1.5ex}

    \rule{\textwidth}{0.5\fboxrule}
\setlength{\parskip}{2ex}
    Converts paramname to ctypes c\_ubyte

\setlength{\parskip}{1ex}
    \end{boxedminipage}

    \label{xformslib:library:make_int_and_pointer}
    \index{xformslib \textit{(package)}!xformslib.library \textit{(module)}!xformslib.library.make\_int\_and\_pointer \textit{(function)}}

    \vspace{0.5ex}

\hspace{.8\funcindent}\begin{boxedminipage}{\funcwidth}

    \raggedright \textbf{make\_int\_and\_pointer}()

    \vspace{-1.5ex}

    \rule{\textwidth}{0.5\fboxrule}
\setlength{\parskip}{2ex}
    Makes a ctypes c\_int and its pointer, and returns both

\setlength{\parskip}{1ex}
    \end{boxedminipage}

    \label{xformslib:library:make_int_and_pointer}
    \index{xformslib \textit{(package)}!xformslib.library \textit{(module)}!xformslib.library.make\_int\_and\_pointer \textit{(function)}}

    \vspace{0.5ex}

\hspace{.8\funcindent}\begin{boxedminipage}{\funcwidth}

    \raggedright \textbf{make\_FL\_Coord\_and\_pointer}()

    \vspace{-1.5ex}

    \rule{\textwidth}{0.5\fboxrule}
\setlength{\parskip}{2ex}
    Makes a ctypes c\_int and its pointer, and returns both

\setlength{\parskip}{1ex}
    \end{boxedminipage}

    \label{xformslib:library:make_uint_and_pointer}
    \index{xformslib \textit{(package)}!xformslib.library \textit{(module)}!xformslib.library.make\_uint\_and\_pointer \textit{(function)}}

    \vspace{0.5ex}

\hspace{.8\funcindent}\begin{boxedminipage}{\funcwidth}

    \raggedright \textbf{make\_uint\_and\_pointer}()

    \vspace{-1.5ex}

    \rule{\textwidth}{0.5\fboxrule}
\setlength{\parskip}{2ex}
    Makes a ctypes c\_uint and its pointer, and returns both

\setlength{\parskip}{1ex}
    \end{boxedminipage}

    \label{xformslib:library:make_long_and_pointer}
    \index{xformslib \textit{(package)}!xformslib.library \textit{(module)}!xformslib.library.make\_long\_and\_pointer \textit{(function)}}

    \vspace{0.5ex}

\hspace{.8\funcindent}\begin{boxedminipage}{\funcwidth}

    \raggedright \textbf{make\_long\_and\_pointer}()

    \vspace{-1.5ex}

    \rule{\textwidth}{0.5\fboxrule}
\setlength{\parskip}{2ex}
    Makes a ctypes c\_long and its pointer, and returns both

\setlength{\parskip}{1ex}
    \end{boxedminipage}

    \label{xformslib:library:make_ulong_and_pointer}
    \index{xformslib \textit{(package)}!xformslib.library \textit{(module)}!xformslib.library.make\_ulong\_and\_pointer \textit{(function)}}

    \vspace{0.5ex}

\hspace{.8\funcindent}\begin{boxedminipage}{\funcwidth}

    \raggedright \textbf{make\_ulong\_and\_pointer}()

    \vspace{-1.5ex}

    \rule{\textwidth}{0.5\fboxrule}
\setlength{\parskip}{2ex}
    Makes a ctypes c\_ulong and its pointer, and returns both

\setlength{\parskip}{1ex}
    \end{boxedminipage}

    \label{xformslib:library:make_ulong_and_pointer}
    \index{xformslib \textit{(package)}!xformslib.library \textit{(module)}!xformslib.library.make\_ulong\_and\_pointer \textit{(function)}}

    \vspace{0.5ex}

\hspace{.8\funcindent}\begin{boxedminipage}{\funcwidth}

    \raggedright \textbf{make\_Pixmap\_and\_pointer}()

    \vspace{-1.5ex}

    \rule{\textwidth}{0.5\fboxrule}
\setlength{\parskip}{2ex}
    Makes a ctypes c\_ulong and its pointer, and returns both

\setlength{\parskip}{1ex}
    \end{boxedminipage}

    \label{xformslib:library:make_ulong_and_pointer}
    \index{xformslib \textit{(package)}!xformslib.library \textit{(module)}!xformslib.library.make\_ulong\_and\_pointer \textit{(function)}}

    \vspace{0.5ex}

\hspace{.8\funcindent}\begin{boxedminipage}{\funcwidth}

    \raggedright \textbf{make\_FL\_COLOR\_and\_pointer}()

    \vspace{-1.5ex}

    \rule{\textwidth}{0.5\fboxrule}
\setlength{\parskip}{2ex}
    Makes a ctypes c\_ulong and its pointer, and returns both

\setlength{\parskip}{1ex}
    \end{boxedminipage}

    \label{xformslib:library:make_float_and_pointer}
    \index{xformslib \textit{(package)}!xformslib.library \textit{(module)}!xformslib.library.make\_float\_and\_pointer \textit{(function)}}

    \vspace{0.5ex}

\hspace{.8\funcindent}\begin{boxedminipage}{\funcwidth}

    \raggedright \textbf{make\_float\_and\_pointer}()

    \vspace{-1.5ex}

    \rule{\textwidth}{0.5\fboxrule}
\setlength{\parskip}{2ex}
    Makes a ctypes c\_float and its pointer, and returns both

\setlength{\parskip}{1ex}
    \end{boxedminipage}

    \label{xformslib:library:make_double_and_pointer}
    \index{xformslib \textit{(package)}!xformslib.library \textit{(module)}!xformslib.library.make\_double\_and\_pointer \textit{(function)}}

    \vspace{0.5ex}

\hspace{.8\funcindent}\begin{boxedminipage}{\funcwidth}

    \raggedright \textbf{make\_double\_and\_pointer}()

    \vspace{-1.5ex}

    \rule{\textwidth}{0.5\fboxrule}
\setlength{\parskip}{2ex}
    Makes a ctypes c\_double and its pointer, and returns both

\setlength{\parskip}{1ex}
    \end{boxedminipage}

    \label{xformslib:library:check_admitted_listvalues}
    \index{xformslib \textit{(package)}!xformslib.library \textit{(module)}!xformslib.library.check\_admitted\_listvalues \textit{(function)}}

    \vspace{0.5ex}

\hspace{.8\funcindent}\begin{boxedminipage}{\funcwidth}

    \raggedright \textbf{check\_admitted\_listvalues}(\textit{paramname}, *\textit{valueslist})

    \vspace{-1.5ex}

    \rule{\textwidth}{0.5\fboxrule}
\setlength{\parskip}{2ex}
    Check if paramname value is valid in accordance to a list of admissible
    values.

\setlength{\parskip}{1ex}
    \end{boxedminipage}

    \label{xformslib:library:check_if_FL_OBJECT_ptr}
    \index{xformslib \textit{(package)}!xformslib.library \textit{(module)}!xformslib.library.check\_if\_FL\_OBJECT\_ptr \textit{(function)}}

    \vspace{0.5ex}

\hspace{.8\funcindent}\begin{boxedminipage}{\funcwidth}

    \raggedright \textbf{check\_if\_FL\_OBJECT\_ptr}(\textit{paramname})

    \vspace{-1.5ex}

    \rule{\textwidth}{0.5\fboxrule}
\setlength{\parskip}{2ex}
    Check if paramname value is a valid pointer to xfdata.FL\_OBJECT.

\setlength{\parskip}{1ex}
    \end{boxedminipage}

    \label{xformslib:library:check_if_FL_FORM_ptr}
    \index{xformslib \textit{(package)}!xformslib.library \textit{(module)}!xformslib.library.check\_if\_FL\_FORM\_ptr \textit{(function)}}

    \vspace{0.5ex}

\hspace{.8\funcindent}\begin{boxedminipage}{\funcwidth}

    \raggedright \textbf{check\_if\_FL\_FORM\_ptr}(\textit{paramname})

    \vspace{-1.5ex}

    \rule{\textwidth}{0.5\fboxrule}
\setlength{\parskip}{2ex}
    Check if paramname value is a valid pointer to xfdata.FL\_FORM.

\setlength{\parskip}{1ex}
    \end{boxedminipage}

    \label{xformslib:library:donothing_popupcb}
    \index{xformslib \textit{(package)}!xformslib.library \textit{(module)}!xformslib.library.donothing\_popupcb \textit{(function)}}

    \vspace{0.5ex}

\hspace{.8\funcindent}\begin{boxedminipage}{\funcwidth}

    \raggedright \textbf{donothing\_popupcb}(\textit{pPopupReturn})

    \vspace{-1.5ex}

    \rule{\textwidth}{0.5\fboxrule}
\setlength{\parskip}{2ex}
    It replaces a callback function not defined for class instances as e.g.
    xfdata.FL\_POPUP\_ITEM    *temporary*

\setlength{\parskip}{1ex}
    \end{boxedminipage}

    \label{xformslib:library:make_pPopupItem_from_dict}
    \index{xformslib \textit{(package)}!xformslib.library \textit{(module)}!xformslib.library.make\_pPopupItem\_from\_dict \textit{(function)}}

    \vspace{0.5ex}

\hspace{.8\funcindent}\begin{boxedminipage}{\funcwidth}

    \raggedright \textbf{make\_pPopupItem\_from\_dict}(\textit{dictofpopupitems})

    \vspace{-1.5ex}

    \rule{\textwidth}{0.5\fboxrule}
\setlength{\parskip}{2ex}
    Taking a python dict (for one dict item ONLY) with a structure similar 
    to xfdata.FL\_POPUP\_ITEM prepares and returns a C-compatible pointer 
    to xfdata.FL\_POPUP\_ITEM.

\setlength{\parskip}{1ex}
      \textbf{Return Value}
    \vspace{-1ex}

      \begin{quote}
      pPopupItem

      \end{quote}

    \end{boxedminipage}

    \label{xformslib:library:make_pPopupItem_from_list}
    \index{xformslib \textit{(package)}!xformslib.library \textit{(module)}!xformslib.library.make\_pPopupItem\_from\_list \textit{(function)}}

    \vspace{0.5ex}

\hspace{.8\funcindent}\begin{boxedminipage}{\funcwidth}

    \raggedright \textbf{make\_pPopupItem\_from\_list}(\textit{listofpopupitems})

    \vspace{-1.5ex}

    \rule{\textwidth}{0.5\fboxrule}
\setlength{\parskip}{2ex}
    Taking a python single list/several lists of popup items, with elements
    in the same order as xfdata.FL\_POPUP\_ITEM prepares and returns a 
    C-compatible pointer to xfdata.FL\_POPUP\_ITEM.

\setlength{\parskip}{1ex}
      \textbf{Return Value}
    \vspace{-1ex}

      \begin{quote}
      pPopupItem

      \end{quote}

    \end{boxedminipage}

    \label{xformslib:library:FL_IS_UPBOX}
    \index{xformslib \textit{(package)}!xformslib.library \textit{(module)}!xformslib.library.FL\_IS\_UPBOX \textit{(function)}}

    \vspace{0.5ex}

\hspace{.8\funcindent}\begin{boxedminipage}{\funcwidth}

    \raggedright \textbf{FL\_IS\_UPBOX}(\textit{boxtype})

\setlength{\parskip}{2ex}
\setlength{\parskip}{1ex}
    \end{boxedminipage}

    \label{xformslib:library:FL_IS_DOWNBOX}
    \index{xformslib \textit{(package)}!xformslib.library \textit{(module)}!xformslib.library.FL\_IS\_DOWNBOX \textit{(function)}}

    \vspace{0.5ex}

\hspace{.8\funcindent}\begin{boxedminipage}{\funcwidth}

    \raggedright \textbf{FL\_IS\_DOWNBOX}(\textit{boxtype})

\setlength{\parskip}{2ex}
\setlength{\parskip}{1ex}
    \end{boxedminipage}

    \label{xformslib:library:FL_TO_DOWNBOX}
    \index{xformslib \textit{(package)}!xformslib.library \textit{(module)}!xformslib.library.FL\_TO\_DOWNBOX \textit{(function)}}

    \vspace{0.5ex}

\hspace{.8\funcindent}\begin{boxedminipage}{\funcwidth}

    \raggedright \textbf{FL\_TO\_DOWNBOX}(\textit{boxtype})

\setlength{\parskip}{2ex}
\setlength{\parskip}{1ex}
    \end{boxedminipage}

    \label{xformslib:library:special_style}
    \index{xformslib \textit{(package)}!xformslib.library \textit{(module)}!xformslib.library.special\_style \textit{(function)}}

    \vspace{0.5ex}

\hspace{.8\funcindent}\begin{boxedminipage}{\funcwidth}

    \raggedright \textbf{special\_style}(\textit{style})

\setlength{\parskip}{2ex}
\setlength{\parskip}{1ex}
    \end{boxedminipage}

    \label{xformslib:library:fl_object_returned}
    \index{xformslib \textit{(package)}!xformslib.library \textit{(module)}!xformslib.library.fl\_object\_returned \textit{(function)}}

    \vspace{0.5ex}

\hspace{.8\funcindent}\begin{boxedminipage}{\funcwidth}

    \raggedright \textbf{fl\_object\_returned}(\textit{pObject})

\setlength{\parskip}{2ex}
\setlength{\parskip}{1ex}
    \end{boxedminipage}

    \label{xformslib:library:fl_add_io_callback}
    \index{xformslib \textit{(package)}!xformslib.library \textit{(module)}!xformslib.library.fl\_add\_io\_callback \textit{(function)}}

    \vspace{0.5ex}

\hspace{.8\funcindent}\begin{boxedminipage}{\funcwidth}

    \raggedright \textbf{fl\_add\_io\_callback}(\textit{fd}, \textit{mask}, \textit{py\_IoCallback}, \textit{vdata})

    \vspace{-1.5ex}

    \rule{\textwidth}{0.5\fboxrule}
\setlength{\parskip}{2ex}
    Registers an input callback function when input is available from fd.

\setlength{\parskip}{1ex}
      \textbf{Parameters}
      \vspace{-1ex}

      \begin{quote}
        \begin{Ventry}{xxxxxxxxxxxxx}

          \item[fd]

          a valid file descriptor in a unix system 
          ({\textless}int{\textgreater})

          \item[mask]

          under what circumstance the input callback should be invoked 
          ({\textless}int{\textgreater})

            {\it (type=(from xfdata module) FL\_READ, FL\_WRITE, FL\_EXCEPT)}

          \item[py\_IoCallback]

          python function to be invoked - no return

            {\it (type=\_\_ funcname (num, ptr\_void) \_\_)}

          \item[vdata]

          user data argument to be passed to function ({\textless}pointer 
          to void{\textgreater})

        \end{Ventry}

      \end{quote}

\textbf{Example:}
\begin{quote}
  \begin{itemize}

  \item
    \setlength{\parskip}{0.6ex}
def iocb(num, vdata):



  \item {\textbar}-{\textgreater}{\textbar} ...



  \item fdesc = ... function to open file



  \item fl\_add\_io\_callback(fdesc, xfdata.FL\_READ, iocb, None)



\end{itemize}

\end{quote}

\textbf{Status:} Tested + Doc + NoDemo = OK



    \end{boxedminipage}

    \label{xformslib:library:fl_remove_io_callback}
    \index{xformslib \textit{(package)}!xformslib.library \textit{(module)}!xformslib.library.fl\_remove\_io\_callback \textit{(function)}}

    \vspace{0.5ex}

\hspace{.8\funcindent}\begin{boxedminipage}{\funcwidth}

    \raggedright \textbf{fl\_remove\_io\_callback}(\textit{fd}, \textit{mask}, \textit{py\_IoCallback})

    \vspace{-1.5ex}

    \rule{\textwidth}{0.5\fboxrule}
\setlength{\parskip}{2ex}
    Removes the registered callback function when input is available from 
    fd.

\setlength{\parskip}{1ex}
      \textbf{Parameters}
      \vspace{-1ex}

      \begin{quote}
        \begin{Ventry}{xxxxxxxxxxxxx}

          \item[fd]

          a valid file descriptor in a unix system 
          ({\textless}int{\textgreater})

          \item[mask]

          under what circumstance the input callback should be removed 
          ({\textless}int{\textgreater})

            {\it (type=(from xfdata module) FL\_READ, FL\_WRITE, FL\_EXCEPT)}

          \item[py\_IoCallback]

          python function to be removed - no return

            {\it (type=\_\_ funcname (num, ptr\_void) \_\_)}

        \end{Ventry}

      \end{quote}

\textbf{Example:}
\begin{quote}
  \begin{itemize}

  \item
    \setlength{\parskip}{0.6ex}
def iocb(num, vdata):



  \item {\textbar}-{\textgreater}{\textbar} ...



  \item fdesc = ... function to open file



  \item fl\_remove\_io\_callback(fdesc, xfdata.FL\_READ, iocb, None)



\end{itemize}

\end{quote}

\textbf{Status:} Tested + Doc + NoDemo = OK



    \end{boxedminipage}

    \label{xformslib:library:fl_add_signal_callback}
    \index{xformslib \textit{(package)}!xformslib.library \textit{(module)}!xformslib.library.fl\_add\_signal\_callback \textit{(function)}}

    \vspace{0.5ex}

\hspace{.8\funcindent}\begin{boxedminipage}{\funcwidth}

    \raggedright \textbf{fl\_add\_signal\_callback}(\textit{sglnum}, \textit{py\_SignalHandler}, \textit{vdata})

    \vspace{-1.5ex}

    \rule{\textwidth}{0.5\fboxrule}
\setlength{\parskip}{2ex}
    Handles the receipt of a signal by registering a callback function that
    gets called when a signal is caught (only 1 function per signal)

\setlength{\parskip}{1ex}
      \textbf{Parameters}
      \vspace{-1ex}

      \begin{quote}
        \begin{Ventry}{xxxxxxxxxxxxxxxx}

          \item[sglnum]

          signal number ({\textless}int{\textgreater})

            {\it (type=(from signal module) SIGALRM, SIGINT, ...)}

          \item[py\_SignalHandler]

          python function to be invoked after catching the signal - no 
          return

            {\it (type=\_\_ funcname (num, ptr\_void) \_\_)}

          \item[vdata]

          argument to be passed to function ({\textless}pointer to 
          void{\textgreater})

        \end{Ventry}

      \end{quote}

\textbf{Example:}
\begin{quote}
  \begin{itemize}

  \item
    \setlength{\parskip}{0.6ex}
def sglhandl(numsgl, vdata):



  \item {\textbar}-{\textgreater}{\textbar} ...



  \item fl\_add\_signal\_callback(signal.SIGALRM, sglhandl, None)



\end{itemize}

\end{quote}

\textbf{Status:} Tested + Doc + NoDemo = OK



    \end{boxedminipage}

    \label{xformslib:library:fl_remove_signal_callback}
    \index{xformslib \textit{(package)}!xformslib.library \textit{(module)}!xformslib.library.fl\_remove\_signal\_callback \textit{(function)}}

    \vspace{0.5ex}

\hspace{.8\funcindent}\begin{boxedminipage}{\funcwidth}

    \raggedright \textbf{fl\_remove\_signal\_callback}(\textit{sglnum})

    \vspace{-1.5ex}

    \rule{\textwidth}{0.5\fboxrule}
\setlength{\parskip}{2ex}
    Removes a previously registered callback function related to a signal.

\setlength{\parskip}{1ex}
      \textbf{Parameters}
      \vspace{-1ex}

      \begin{quote}
        \begin{Ventry}{xxxxxx}

          \item[sglnum]

          signal number ({\textless}int{\textgreater})

            {\it (type=(from signal module) SIGALRM, SIGINT, ...)}

        \end{Ventry}

      \end{quote}

\textbf{Example:} fl\_remove\_signal\_callback(signal.SIGALRM)



\textbf{Status:} Tested + Doc + NoDemo = OK



    \end{boxedminipage}

    \label{xformslib:library:fl_signal_caught}
    \index{xformslib \textit{(package)}!xformslib.library \textit{(module)}!xformslib.library.fl\_signal\_caught \textit{(function)}}

    \vspace{0.5ex}

\hspace{.8\funcindent}\begin{boxedminipage}{\funcwidth}

    \raggedright \textbf{fl\_signal\_caught}(\textit{sglnum})

    \vspace{-1.5ex}

    \rule{\textwidth}{0.5\fboxrule}
\setlength{\parskip}{2ex}
    Informs the main loop of the delivery of the particular signal. The 
    signal is received by the application program.

\setlength{\parskip}{1ex}
      \textbf{Parameters}
      \vspace{-1ex}

      \begin{quote}
        \begin{Ventry}{xxxxxx}

          \item[sglnum]

          signal number (int\_num)

            {\it (type=(from signal module) SIGALRM, SIGINT, ...)}

        \end{Ventry}

      \end{quote}

\textbf{Example:} fl\_signal\_caught(signal.SIGALRM)



\textbf{Status:} Tested + Doc + NoDemo = OK



    \end{boxedminipage}

    \label{xformslib:library:fl_app_signal_direct}
    \index{xformslib \textit{(package)}!xformslib.library \textit{(module)}!xformslib.library.fl\_app\_signal\_direct \textit{(function)}}

    \vspace{0.5ex}

\hspace{.8\funcindent}\begin{boxedminipage}{\funcwidth}

    \raggedright \textbf{fl\_app\_signal\_direct}(\textit{flag})

    \vspace{-1.5ex}

    \rule{\textwidth}{0.5\fboxrule}
\setlength{\parskip}{2ex}
    Changes the default behavior of the built-in signal facilities (to be 
    called with a true value for flag prior to any use of 
    fl\_add\_signal\_callback)

\setlength{\parskip}{1ex}
      \textbf{Parameters}
      \vspace{-1ex}

      \begin{quote}
        \begin{Ventry}{xxxx}

          \item[flag]

          flag to disable/enable ({\textless}int{\textgreater})

            {\it (type=0 (disabled) or 1 (enabled))}

        \end{Ventry}

      \end{quote}

\textbf{Example:} fl\_app\_signal\_direct(1)



\textbf{Status:} Tested + Doc + NoDemo = OK



    \end{boxedminipage}

    \label{xformslib:library:fl_add_timeout}
    \index{xformslib \textit{(package)}!xformslib.library \textit{(module)}!xformslib.library.fl\_add\_timeout \textit{(function)}}

    \vspace{0.5ex}

\hspace{.8\funcindent}\begin{boxedminipage}{\funcwidth}

    \raggedright \textbf{fl\_add\_timeout}(\textit{msec}, \textit{py\_TimeoutCallback}, \textit{vdata})

    \vspace{-1.5ex}

    \rule{\textwidth}{0.5\fboxrule}
\setlength{\parskip}{2ex}
    Adds a timeout callback after a specified elapsed time.

\setlength{\parskip}{1ex}
      \textbf{Parameters}
      \vspace{-1ex}

      \begin{quote}
        \begin{Ventry}{xxxxxxxxxxxxxxxxxx}

          \item[msec]

          time elapsed in milliseconds ({\textless}long{\textgreater})

          \item[py\_TimeoutCallback]

          python function to be invoked - no return

            {\it (type=\_\_ funcname (num, ptr\_void) \_\_)}

          \item[vdata]

          user data to be passed to function ({\textless}pointer to 
          void{\textgreater})

        \end{Ventry}

      \end{quote}

      \textbf{Return Value}
    \vspace{-1ex}

      \begin{quote}
      timer number id ({\textless}int{\textgreater})

      {\it (type=timer\_id)}

      \end{quote}

\textbf{Example:}
\begin{quote}
  \begin{itemize}

  \item
    \setlength{\parskip}{0.6ex}
def timeoutcb(num, vdata):



  \item {\textbar}-{\textgreater}{\textbar} ...



  \item timnum = fl\_add\_timeout(100, timeoutcb, None)



\end{itemize}

\end{quote}

\textbf{Status:} Tested + Doc + Demo = OK



    \end{boxedminipage}

    \label{xformslib:library:fl_remove_timeout}
    \index{xformslib \textit{(package)}!xformslib.library \textit{(module)}!xformslib.library.fl\_remove\_timeout \textit{(function)}}

    \vspace{0.5ex}

\hspace{.8\funcindent}\begin{boxedminipage}{\funcwidth}

    \raggedright \textbf{fl\_remove\_timeout}(\textit{idnum})

    \vspace{-1.5ex}

    \rule{\textwidth}{0.5\fboxrule}
\setlength{\parskip}{2ex}
    Removes a timeout callback function (created with fl\_add\_timeout).

\setlength{\parskip}{1ex}
      \textbf{Parameters}
      \vspace{-1ex}

      \begin{quote}
        \begin{Ventry}{xxxxx}

          \item[idnum]

          timeout number id ({\textless}int{\textgreater})

        \end{Ventry}

      \end{quote}

\textbf{Example:} fl\_remove\_timeout(timnum)



\textbf{Status:} Tested + Doc + Demo = OK



    \end{boxedminipage}

    \label{xformslib:library:fl_library_version}
    \index{xformslib \textit{(package)}!xformslib.library \textit{(module)}!xformslib.library.fl\_library\_version \textit{(function)}}

    \vspace{0.5ex}

\hspace{.8\funcindent}\begin{boxedminipage}{\funcwidth}

    \raggedright \textbf{fl\_library\_version}()

    \vspace{-1.5ex}

    \rule{\textwidth}{0.5\fboxrule}
\setlength{\parskip}{2ex}
    Returns consolidated, major and minor version informations.

\setlength{\parskip}{1ex}
      \textbf{Return Value}
    \vspace{-1ex}

      \begin{quote}
      ({\textless}int{\textgreater}, {\textless}int{\textgreater}, 
      {\textless}int{\textgreater}) version\_rev (computed as 1000 * 
      version + revision), version (e.g. 1 in 1.x.yy), revision (e.g. 0 in 
      x.0.yy)

      {\it (type=version\_rev\_id, ver, rev)}

      \end{quote}

\textbf{Example:} compver, ver, rev = fl\_library\_version()



\textbf{Attention:} API change from XForms - upstream was fl\_library\_version(ver, rev)



\textbf{Status:} Tested + Doc + NoDemo = OK



    \end{boxedminipage}

    \label{xformslib:library:fl_bgn_form}
    \index{xformslib \textit{(package)}!xformslib.library \textit{(module)}!xformslib.library.fl\_bgn\_form \textit{(function)}}

    \vspace{0.5ex}

\hspace{.8\funcindent}\begin{boxedminipage}{\funcwidth}

    \raggedright \textbf{fl\_bgn\_form}(\textit{formtype}, \textit{w}, \textit{h})

    \vspace{-1.5ex}

    \rule{\textwidth}{0.5\fboxrule}
\setlength{\parskip}{2ex}
    Starts the definition of a form call.

\setlength{\parskip}{1ex}
      \textbf{Parameters}
      \vspace{-1ex}

      \begin{quote}
        \begin{Ventry}{xxxxxxxx}

          \item[formtype]

          type of box that is used as a background 
          ({\textless}int{\textgreater})

            {\it (type=(from xfdata module) FL\_NO\_BOX, FL\_UP\_BOX, FL\_DOWN\_BOX, 
FL\_BORDER\_BOX, FL\_SHADOW\_BOX, FL\_FRAME\_BOX, FL\_ROUNDED\_BOX, 
FL\_EMBOSSED\_BOX, FL\_FLAT\_BOX, FL\_RFLAT\_BOX, FL\_RSHADOW\_BOX, 
FL\_OVAL\_BOX, FL\_ROUNDED3D\_UPBOX, FL\_ROUNDED3D\_DOWNBOX, 
FL\_OVAL3D\_UPBOX, FL\_OVAL3D\_DOWNBOX, FL\_OVAL3D\_FRAMEBOX, 
FL\_OVAL3D\_EMBOSSEDBOX)}

          \item[w]

          width of the new form in coord units 
          ({\textless}int{\textgreater})

          \item[h]

          height of the new form in coord units 
          ({\textless}int{\textgreater})

        \end{Ventry}

      \end{quote}

      \textbf{Return Value}
    \vspace{-1ex}

      \begin{quote}
      form to define ({\textless}pointer to xfdata.FL\_FORM{\textgreater})

      {\it (type=pForm)}

      \end{quote}

\textbf{Example:} pform = fl\_bgn\_form(xfdata.FL\_UP\_BOX, 400, 500)



\textbf{Status:} Tested + Doc + Demo = OK



    \end{boxedminipage}

    \label{xformslib:library:fl_end_form}
    \index{xformslib \textit{(package)}!xformslib.library \textit{(module)}!xformslib.library.fl\_end\_form \textit{(function)}}

    \vspace{0.5ex}

\hspace{.8\funcindent}\begin{boxedminipage}{\funcwidth}

    \raggedright \textbf{fl\_end\_form}()

    \vspace{-1.5ex}

    \rule{\textwidth}{0.5\fboxrule}
\setlength{\parskip}{2ex}
    Ends the definition for a form call, after all required objects have 
    been added to a form call.

\setlength{\parskip}{1ex}
\textbf{Example:} fl\_end\_form()



\textbf{Status:} Tested + Doc + Demo = OK



    \end{boxedminipage}

    \label{xformslib:library:fl_do_forms}
    \index{xformslib \textit{(package)}!xformslib.library \textit{(module)}!xformslib.library.fl\_do\_forms \textit{(function)}}

    \vspace{0.5ex}

\hspace{.8\funcindent}\begin{boxedminipage}{\funcwidth}

    \raggedright \textbf{fl\_do\_forms}()

    \vspace{-1.5ex}

    \rule{\textwidth}{0.5\fboxrule}
\setlength{\parskip}{2ex}
    Starts the main loop of the program and returns only when the state of 
    a xfdata.FL\_OBJECT (that has no callback bound to it) changes.

\setlength{\parskip}{1ex}
      \textbf{Return Value}
    \vspace{-1ex}

      \begin{quote}
      object changed ({\textless}pointer to 
      xfdata.FL\_OBJECT{\textgreater})

      {\it (type=pObject)}

      \end{quote}

\textbf{Example:}
\begin{quote}
  \begin{itemize}

  \item
    \setlength{\parskip}{0.6ex}
while fl\_do\_forms():



  \item {\textbar}-{\textgreater}{\textbar} pass



\end{itemize}

\end{quote}

\textbf{Status:} Tested + Doc + Demo = OK



    \end{boxedminipage}

    \label{xformslib:library:fl_check_forms}
    \index{xformslib \textit{(package)}!xformslib.library \textit{(module)}!xformslib.library.fl\_check\_forms \textit{(function)}}

    \vspace{0.5ex}

\hspace{.8\funcindent}\begin{boxedminipage}{\funcwidth}

    \raggedright \textbf{fl\_check\_forms}()

    \vspace{-1.5ex}

    \rule{\textwidth}{0.5\fboxrule}
\setlength{\parskip}{2ex}
    Returns None immediately unless the state of one of xfdata.FL\_OBJECT 
    (without a callback bound to it) changed.

\setlength{\parskip}{1ex}
      \textbf{Return Value}
    \vspace{-1ex}

      \begin{quote}
      object changed ({\textless}pointer to 
      xfdata.FL\_OBJECT{\textgreater})

      {\it (type=pObject)}

      \end{quote}

\textbf{Example:} pobj = fl\_check\_forms()



\textbf{Status:} Tested + Doc + Demo = OK



    \end{boxedminipage}

    \label{xformslib:library:fl_do_only_forms}
    \index{xformslib \textit{(package)}!xformslib.library \textit{(module)}!xformslib.library.fl\_do\_only\_forms \textit{(function)}}

    \vspace{0.5ex}

\hspace{.8\funcindent}\begin{boxedminipage}{\funcwidth}

    \raggedright \textbf{fl\_do\_only\_forms}()

    \vspace{-1.5ex}

    \rule{\textwidth}{0.5\fboxrule}
\setlength{\parskip}{2ex}
    Starts the main loop of the program and returns only when the state of 
    an object changes that has no callback bound to it. It does not handle 
    user events generated by application windows opened via fl\_winopen() 
    or similar routines.

\setlength{\parskip}{1ex}
      \textbf{Return Value}
    \vspace{-1ex}

      \begin{quote}
      object changed ({\textless}pointer to 
      xfdata.FL\_OBJECT{\textgreater})

      {\it (type=pObject)}

      \end{quote}

\textbf{Example:} pobj = fl\_do\_only\_forms()



\textbf{Status:} Tested + Doc + NoDemo = OK



    \end{boxedminipage}

    \label{xformslib:library:fl_check_only_forms}
    \index{xformslib \textit{(package)}!xformslib.library \textit{(module)}!xformslib.library.fl\_check\_only\_forms \textit{(function)}}

    \vspace{0.5ex}

\hspace{.8\funcindent}\begin{boxedminipage}{\funcwidth}

    \raggedright \textbf{fl\_check\_only\_forms}()

    \vspace{-1.5ex}

    \rule{\textwidth}{0.5\fboxrule}
\setlength{\parskip}{2ex}
    Returns None immediately unless the state of one of the object (without
    a callback bound to it) changed. It does not handle user events 
    generated by application windows opened via fl\_winopen() or similar 
    routines.

\setlength{\parskip}{1ex}
      \textbf{Return Value}
    \vspace{-1ex}

      \begin{quote}
      object changed ({\textless}pointer to 
      xfdata.FL\_OBJECT{\textgreater})

      {\it (type=pObject)}

      \end{quote}

\textbf{Example:} pobj = fl\_check\_only\_forms()



\textbf{Status:} Tested + Doc + NoDemo = OK



    \end{boxedminipage}

    \label{xformslib:library:fl_freeze_form}
    \index{xformslib \textit{(package)}!xformslib.library \textit{(module)}!xformslib.library.fl\_freeze\_form \textit{(function)}}

    \vspace{0.5ex}

\hspace{.8\funcindent}\begin{boxedminipage}{\funcwidth}

    \raggedright \textbf{fl\_freeze\_form}(\textit{pForm})

    \vspace{-1.5ex}

    \rule{\textwidth}{0.5\fboxrule}
\setlength{\parskip}{2ex}
    Redraw of a form is temporarily suspended, while changes are being 
    made, so all changes made are instead buffered internally.

\setlength{\parskip}{1ex}
      \textbf{Parameters}
      \vspace{-1ex}

      \begin{quote}
        \begin{Ventry}{xxxxx}

          \item[pForm]

          form not to be re-drawn temporarily ({\textless}pointer to 
          xfdata.FL\_FORM{\textgreater})

        \end{Ventry}

      \end{quote}

\textbf{Example:} fl\_freeze\_form(pform1)



\textbf{Status:} Tested + Doc + Demo = OK



    \end{boxedminipage}

    \label{xformslib:library:fl_set_focus_object}
    \index{xformslib \textit{(package)}!xformslib.library \textit{(module)}!xformslib.library.fl\_set\_focus\_object \textit{(function)}}

    \vspace{0.5ex}

\hspace{.8\funcindent}\begin{boxedminipage}{\funcwidth}

    \raggedright \textbf{fl\_set\_focus\_object}(\textit{pForm}, \textit{pObject})

    \vspace{-1.5ex}

    \rule{\textwidth}{0.5\fboxrule}
\setlength{\parskip}{2ex}
    Sets the input focus in form to object pObject.

\setlength{\parskip}{1ex}
      \textbf{Parameters}
      \vspace{-1ex}

      \begin{quote}
        \begin{Ventry}{xxxxxxx}

          \item[pForm]

          form whose object has to be focused ({\textless}pointer to 
          xfdata.FL\_FORM{\textgreater})

          \item[pObject]

          object to be focused ({\textless}pointer to 
          xfdata.FL\_OBJECT{\textgreater})

        \end{Ventry}

      \end{quote}

\textbf{Example:} fl\_set\_focus\_object(pform, pobj)



\textbf{Status:} Tested + Doc + NoDemo = OK



    \end{boxedminipage}

    \label{xformslib:library:fl_set_focus_object}
    \index{xformslib \textit{(package)}!xformslib.library \textit{(module)}!xformslib.library.fl\_set\_focus\_object \textit{(function)}}

    \vspace{0.5ex}

\hspace{.8\funcindent}\begin{boxedminipage}{\funcwidth}

    \raggedright \textbf{fl\_set\_object\_focus}(\textit{pForm}, \textit{pObject})

    \vspace{-1.5ex}

    \rule{\textwidth}{0.5\fboxrule}
\setlength{\parskip}{2ex}
    Sets the input focus in form to object pObject.

\setlength{\parskip}{1ex}
      \textbf{Parameters}
      \vspace{-1ex}

      \begin{quote}
        \begin{Ventry}{xxxxxxx}

          \item[pForm]

          form whose object has to be focused ({\textless}pointer to 
          xfdata.FL\_FORM{\textgreater})

          \item[pObject]

          object to be focused ({\textless}pointer to 
          xfdata.FL\_OBJECT{\textgreater})

        \end{Ventry}

      \end{quote}

\textbf{Example:} fl\_set\_focus\_object(pform, pobj)



\textbf{Status:} Tested + Doc + NoDemo = OK



    \end{boxedminipage}

    \label{xformslib:library:fl_get_focus_object}
    \index{xformslib \textit{(package)}!xformslib.library \textit{(module)}!xformslib.library.fl\_get\_focus\_object \textit{(function)}}

    \vspace{0.5ex}

\hspace{.8\funcindent}\begin{boxedminipage}{\funcwidth}

    \raggedright \textbf{fl\_get\_focus\_object}(\textit{pForm})

    \vspace{-1.5ex}

    \rule{\textwidth}{0.5\fboxrule}
\setlength{\parskip}{2ex}
    Obtains the object that has the focus on a form.

\setlength{\parskip}{1ex}
      \textbf{Parameters}
      \vspace{-1ex}

      \begin{quote}
        \begin{Ventry}{xxxxx}

          \item[pForm]

          form that has a focused object in ({\textless}pointer to 
          xfdata.FL\_FORM{\textgreater})

        \end{Ventry}

      \end{quote}

      \textbf{Return Value}
    \vspace{-1ex}

      \begin{quote}
      focused object ({\textless}pointer to 
      xfdata.FL\_OBJECT{\textgreater})

      {\it (type=pObject)}

      \end{quote}

\textbf{Example:} pobj2 = fl\_get\_focus\_object(pform1)



\textbf{Status:} Tested + Doc + NoDemo = OK



    \end{boxedminipage}

    \label{xformslib:library:fl_reset_focus_object}
    \index{xformslib \textit{(package)}!xformslib.library \textit{(module)}!xformslib.library.fl\_reset\_focus\_object \textit{(function)}}

    \vspace{0.5ex}

\hspace{.8\funcindent}\begin{boxedminipage}{\funcwidth}

    \raggedright \textbf{fl\_reset\_focus\_object}(\textit{pObject})

    \vspace{-1.5ex}

    \rule{\textwidth}{0.5\fboxrule}
\setlength{\parskip}{2ex}
    Resets focus on current object, overriding the xfdata.FL\_UNFOCUS 
    event.

\setlength{\parskip}{1ex}
      \textbf{Parameters}
      \vspace{-1ex}

      \begin{quote}
        \begin{Ventry}{xxxxxxx}

          \item[pObject]

          object towards applying event ({\textless}pointer to 
          xfdata.FL\_OBJECT{\textgreater})

        \end{Ventry}

      \end{quote}

\textbf{Example:} fl\_reset\_focus\_object(pobj2)



\textbf{Status:} Tested + Doc + NoDemo = OK



    \end{boxedminipage}

    \label{xformslib:library:fl_set_form_atclose}
    \index{xformslib \textit{(package)}!xformslib.library \textit{(module)}!xformslib.library.fl\_set\_form\_atclose \textit{(function)}}

    \vspace{0.5ex}

\hspace{.8\funcindent}\begin{boxedminipage}{\funcwidth}

    \raggedright \textbf{fl\_set\_form\_atclose}(\textit{pForm}, \textit{py\_FormAtclose}, \textit{vdata})

    \vspace{-1.5ex}

    \rule{\textwidth}{0.5\fboxrule}
\setlength{\parskip}{2ex}
    Calls a callback function before closing the form.

\setlength{\parskip}{1ex}
      \textbf{Parameters}
      \vspace{-1ex}

      \begin{quote}
        \begin{Ventry}{xxxxxxxxxxxxxx}

          \item[pForm]

          form that receives the message ({\textless}pointer to 
          xfdata.FL\_FORM{\textgreater})

          \item[py\_FormAtclose]

          python callback function to be called, with returning value

            {\it (type=\_\_ funcname (pForm, ptr\_void) -{\textgreater} num \_\_)}

          \item[vdata]

          user data to be passed to function ({\textless}pointer to 
          void{\textgreater})

        \end{Ventry}

      \end{quote}

      \textbf{Return Value}
    \vspace{-1ex}

      \begin{quote}
      xfdata.FL\_FORM\_ATCLOSE function

      {\it (type=old FormAtclose func)}

      \end{quote}

\textbf{Example:}
\begin{quote}
  \begin{itemize}

  \item
    \setlength{\parskip}{0.6ex}
def atcolsecb(pform, vdata):



  \item {\textbar}-{\textgreater}{\textbar} ...



  \item {\textbar}-{\textgreater}{\textbar} return 0



  \item oldatclosecb = fl\_set\_form\_atclose(pform1, None)



\end{itemize}

\end{quote}

\textbf{Status:} Tested + Doc + NoDemo = OK



    \end{boxedminipage}

    \label{xformslib:library:fl_set_atclose}
    \index{xformslib \textit{(package)}!xformslib.library \textit{(module)}!xformslib.library.fl\_set\_atclose \textit{(function)}}

    \vspace{0.5ex}

\hspace{.8\funcindent}\begin{boxedminipage}{\funcwidth}

    \raggedright \textbf{fl\_set\_atclose}(\textit{py\_FormAtclose}, \textit{vdata})

    \vspace{-1.5ex}

    \rule{\textwidth}{0.5\fboxrule}
\setlength{\parskip}{2ex}
    Calls a callback function before terminating the application.

\setlength{\parskip}{1ex}
      \textbf{Parameters}
      \vspace{-1ex}

      \begin{quote}
        \begin{Ventry}{xxxxxxxxxxxxxx}

          \item[py\_FormAtclose]

          python callback function to be called, with returning value

            {\it (type=\_\_ funcname (pForm, ptr\_void) -{\textgreater} num \_\_)}

          \item[vdata]

          user data to be passed to function ({\textless}pointer to 
          void{\textgreater})

        \end{Ventry}

      \end{quote}

      \textbf{Return Value}
    \vspace{-1ex}

      \begin{quote}
      old xfdata.FL\_FORM\_ATCLOSE function

      {\it (type=old FormAtclose func)}

      \end{quote}

\textbf{Example:}
\begin{quote}
  \begin{itemize}

  \item
    \setlength{\parskip}{0.6ex}
def atclose(pform, vdata):



  \item {\textbar}-{\textgreater}{\textbar} ...



  \item {\textbar}-{\textgreater}{\textbar} return 0



  \item oldatclosefunc = fl\_set\_atclose(atclose, None)



\end{itemize}

\end{quote}

\textbf{Status:} Tested + Doc + NoDemo = OK



    \end{boxedminipage}

    \label{xformslib:library:fl_set_form_atactivate}
    \index{xformslib \textit{(package)}!xformslib.library \textit{(module)}!xformslib.library.fl\_set\_form\_atactivate \textit{(function)}}

    \vspace{0.5ex}

\hspace{.8\funcindent}\begin{boxedminipage}{\funcwidth}

    \raggedright \textbf{fl\_set\_form\_atactivate}(\textit{pForm}, \textit{py\_FormAtactivate}, \textit{vdata})

    \vspace{-1.5ex}

    \rule{\textwidth}{0.5\fboxrule}
\setlength{\parskip}{2ex}
    Register a callback that is called when activation status of a forms is
    enabled,

\setlength{\parskip}{1ex}
      \textbf{Parameters}
      \vspace{-1ex}

      \begin{quote}
        \begin{Ventry}{xxxxxxxxxxxxxxxxx}

          \item[pForm]

          activated form ({\textless}pointer to 
          xfdata.FL\_FORM{\textgreater})

          \item[py\_FormAtactivate]

          python callback function called - no return

            {\it (type=\_\_ funcname (pForm, ptr\_void) \_\_)}

          \item[vdata]

          user data to be passed to function ({\textless}pointer to 
          void{\textgreater})

        \end{Ventry}

      \end{quote}

      \textbf{Return Value}
    \vspace{-1ex}

      \begin{quote}
      old xfdata.FL\_FORM\_ATACTIVATE function

      {\it (type=old FormAtactivate func)}

      \end{quote}

\textbf{Example:}
\begin{quote}
  \begin{itemize}

  \item
    \setlength{\parskip}{0.6ex}
def atactcb(pform, vdata):



  \item {\textbar}-{\textgreater}{\textbar} ...



  \item oldactfunc = xf.fl\_set\_form\_atdeactivate(pform, atactcb, None)



\end{itemize}

\end{quote}

\textbf{Status:} Tested + Doc + NoDemo = OK



    \end{boxedminipage}

    \label{xformslib:library:fl_set_form_atdeactivate}
    \index{xformslib \textit{(package)}!xformslib.library \textit{(module)}!xformslib.library.fl\_set\_form\_atdeactivate \textit{(function)}}

    \vspace{0.5ex}

\hspace{.8\funcindent}\begin{boxedminipage}{\funcwidth}

    \raggedright \textbf{fl\_set\_form\_atdeactivate}(\textit{pForm}, \textit{py\_FormAtdeactivate}, \textit{vdata})

    \vspace{-1.5ex}

    \rule{\textwidth}{0.5\fboxrule}
\setlength{\parskip}{2ex}
    Register a callback that is called when activation status of a forms is
    disabled.

\setlength{\parskip}{1ex}
      \textbf{Parameters}
      \vspace{-1ex}

      \begin{quote}
        \begin{Ventry}{xxxxxxxxxxxxxxxxxxx}

          \item[pForm]

          de-activated form ({\textless}pointer to 
          xfdata.FL\_FORM{\textgreater})

          \item[py\_FormAtdeactivate]

          python callback function called - no return

            {\it (type=\_\_ funcname (pForm, ptr\_void) \_\_)}

          \item[vdata]

          user data to be passed to function ({\textless}pointer to 
          void{\textgreater})

        \end{Ventry}

      \end{quote}

      \textbf{Return Value}
    \vspace{-1ex}

      \begin{quote}
      old xfdata.FL\_FORM\_ATDEACTIVATE function

      {\it (type=old FormAtdeactivate func)}

      \end{quote}

\textbf{Example:}
\begin{quote}
  \begin{itemize}

  \item
    \setlength{\parskip}{0.6ex}
def atdeactcb(pform, vdata):



  \item {\textbar}-{\textgreater}{\textbar} ...



  \item oldatdeactfunc = xf.fl\_set\_form\_atdeactivate(pform, atdeactiatecb, None)



\end{itemize}

\end{quote}

\textbf{Status:} Tested + Doc + NoDemo = OK



    \end{boxedminipage}

    \label{xformslib:library:fl_unfreeze_form}
    \index{xformslib \textit{(package)}!xformslib.library \textit{(module)}!xformslib.library.fl\_unfreeze\_form \textit{(function)}}

    \vspace{0.5ex}

\hspace{.8\funcindent}\begin{boxedminipage}{\funcwidth}

    \raggedright \textbf{fl\_unfreeze\_form}(\textit{pForm})

    \vspace{-1.5ex}

    \rule{\textwidth}{0.5\fboxrule}
\setlength{\parskip}{2ex}
    Reverts previous freeze (set with fl\_freeze\_form function), all 
    changes made in the meantime in a form are drawn at once.

\setlength{\parskip}{1ex}
      \textbf{Parameters}
      \vspace{-1ex}

      \begin{quote}
        \begin{Ventry}{xxxxx}

          \item[pForm]

          form to be re-drawn after freezing ({\textless}pointer to 
          xfdata.FL\_FORM)

        \end{Ventry}

      \end{quote}

\textbf{Example:} fl\_unfreeze\_form(pform)



\textbf{Status:} Tested + Doc + Demo = OK



    \end{boxedminipage}

    \label{xformslib:library:fl_deactivate_form}
    \index{xformslib \textit{(package)}!xformslib.library \textit{(module)}!xformslib.library.fl\_deactivate\_form \textit{(function)}}

    \vspace{0.5ex}

\hspace{.8\funcindent}\begin{boxedminipage}{\funcwidth}

    \raggedright \textbf{fl\_deactivate\_form}(\textit{pForm})

    \vspace{-1.5ex}

    \rule{\textwidth}{0.5\fboxrule}
\setlength{\parskip}{2ex}
    Deactivates form temporarily, without hiding it, but not allowing a 
    user to interact with elements contained in form (buttons, etc.).

\setlength{\parskip}{1ex}
      \textbf{Parameters}
      \vspace{-1ex}

      \begin{quote}
        \begin{Ventry}{xxxxx}

          \item[pForm]

          form to be de-activated ({\textless}pointer to xfdata.FL\_FORM)

        \end{Ventry}

      \end{quote}

\textbf{Example:} fl\_deactivate\_form(pform)



\textbf{Status:} Tested + Doc + Demo = OK



    \end{boxedminipage}

    \label{xformslib:library:fl_activate_form}
    \index{xformslib \textit{(package)}!xformslib.library \textit{(module)}!xformslib.library.fl\_activate\_form \textit{(function)}}

    \vspace{0.5ex}

\hspace{.8\funcindent}\begin{boxedminipage}{\funcwidth}

    \raggedright \textbf{fl\_activate\_form}(\textit{pForm})

    \vspace{-1.5ex}

    \rule{\textwidth}{0.5\fboxrule}
\setlength{\parskip}{2ex}
    (Re)activates form (deactivated with fl\_deactivate\_form), allowing 
    the user to interact again with elements contained in form (buttons, 
    etc.).

\setlength{\parskip}{1ex}
      \textbf{Parameters}
      \vspace{-1ex}

      \begin{quote}
        \begin{Ventry}{xxxxx}

          \item[pForm]

          form to be re-activated ({\textless}pointer to xfdata.FL\_FORM)

        \end{Ventry}

      \end{quote}

\textbf{Example:} fl\_activate\_form(pform)



\textbf{Status:} Tested + Doc + Demo = OK



    \end{boxedminipage}

    \label{xformslib:library:fl_deactivate_all_forms}
    \index{xformslib \textit{(package)}!xformslib.library \textit{(module)}!xformslib.library.fl\_deactivate\_all\_forms \textit{(function)}}

    \vspace{0.5ex}

\hspace{.8\funcindent}\begin{boxedminipage}{\funcwidth}

    \raggedright \textbf{fl\_deactivate\_all\_forms}()

    \vspace{-1.5ex}

    \rule{\textwidth}{0.5\fboxrule}
\setlength{\parskip}{2ex}
    De-activates all current forms, forbidding any event/user interaction.

\setlength{\parskip}{1ex}
\textbf{Example:} fl\_deactivate\_all\_forms()



\textbf{Status:} Tested + Doc + NoDemo = OK



    \end{boxedminipage}

    \label{xformslib:library:fl_activate_all_forms}
    \index{xformslib \textit{(package)}!xformslib.library \textit{(module)}!xformslib.library.fl\_activate\_all\_forms \textit{(function)}}

    \vspace{0.5ex}

\hspace{.8\funcindent}\begin{boxedminipage}{\funcwidth}

    \raggedright \textbf{fl\_activate\_all\_forms}()

    \vspace{-1.5ex}

    \rule{\textwidth}{0.5\fboxrule}
\setlength{\parskip}{2ex}
    (Re)activates all current forms, allowing event/user interaction.

\setlength{\parskip}{1ex}
\textbf{Example:} fl\_activate\_all\_forms()



\textbf{Status:} Tested + Doc + NoDemo = OK



    \end{boxedminipage}

    \label{xformslib:library:fl_freeze_all_forms}
    \index{xformslib \textit{(package)}!xformslib.library \textit{(module)}!xformslib.library.fl\_freeze\_all\_forms \textit{(function)}}

    \vspace{0.5ex}

\hspace{.8\funcindent}\begin{boxedminipage}{\funcwidth}

    \raggedright \textbf{fl\_freeze\_all\_forms}()

    \vspace{-1.5ex}

    \rule{\textwidth}{0.5\fboxrule}
\setlength{\parskip}{2ex}
    All current forms are not temporarily redrawn, while changes are being 
    made and are instead buffered internally.

\setlength{\parskip}{1ex}
\textbf{Example:} fl\_freeze\_all\_forms()



\textbf{Status:} Tested + Doc + NoDemo = OK



    \end{boxedminipage}

    \label{xformslib:library:fl_unfreeze_all_forms}
    \index{xformslib \textit{(package)}!xformslib.library \textit{(module)}!xformslib.library.fl\_unfreeze\_all\_forms \textit{(function)}}

    \vspace{0.5ex}

\hspace{.8\funcindent}\begin{boxedminipage}{\funcwidth}

    \raggedright \textbf{fl\_unfreeze\_all\_forms}()

    \vspace{-1.5ex}

    \rule{\textwidth}{0.5\fboxrule}
\setlength{\parskip}{2ex}
    All changes made in the meantime in all current forms are drawn at 
    once, reverting previous freeze.

\setlength{\parskip}{1ex}
\textbf{Example:} fl\_unfreeze\_all\_forms()



\textbf{Status:} Tested + Doc + NoDemo = OK



    \end{boxedminipage}

    \label{xformslib:library:fl_scale_form}
    \index{xformslib \textit{(package)}!xformslib.library \textit{(module)}!xformslib.library.fl\_scale\_form \textit{(function)}}

    \vspace{0.5ex}

\hspace{.8\funcindent}\begin{boxedminipage}{\funcwidth}

    \raggedright \textbf{fl\_scale\_form}(\textit{pForm}, \textit{xsc}, \textit{ysc})

    \vspace{-1.5ex}

    \rule{\textwidth}{0.5\fboxrule}
\setlength{\parskip}{2ex}
    Scales a form and the objects on it in size and position, indicating a 
    scaling factor in x- and y-direction (1.1 = 110 percent, 0.5 = 50, 
    etc.) with respect to the current size, and reshapes the window.

\setlength{\parskip}{1ex}
      \textbf{Parameters}
      \vspace{-1ex}

      \begin{quote}
        \begin{Ventry}{xxxxx}

          \item[pForm]

          form to be scaled ({\textless}pointer to 
          xfdata.FL\_FORM{\textgreater})

          \item[xsc]

          scaling factor in horizontal direction 
          ({\textless}float{\textgreater})

          \item[ysc]

          scaling factor in vertical direction 
          ({\textless}float{\textgreater})

        \end{Ventry}

      \end{quote}

\textbf{Example:} fl\_scale\_form(pform, 0.8, 1.2)



\textbf{Status:} Tested + Doc + Demo = OK



    \end{boxedminipage}

    \label{xformslib:library:fl_set_form_position}
    \index{xformslib \textit{(package)}!xformslib.library \textit{(module)}!xformslib.library.fl\_set\_form\_position \textit{(function)}}

    \vspace{0.5ex}

\hspace{.8\funcindent}\begin{boxedminipage}{\funcwidth}

    \raggedright \textbf{fl\_set\_form\_position}(\textit{pForm}, \textit{x}, \textit{y})

    \vspace{-1.5ex}

    \rule{\textwidth}{0.5\fboxrule}
\setlength{\parskip}{2ex}
    Sets position of form, when placing a form on the screen with 
    xfdata.FL\_PLACE\_GEOMETRY as place argument.

\setlength{\parskip}{1ex}
      \textbf{Parameters}
      \vspace{-1ex}

      \begin{quote}
        \begin{Ventry}{xxxxx}

          \item[pForm]

          form whose position is to be set ({\textless}pointer to 
          xfdata.FL\_FORM{\textgreater})

          \item[x]

          horizontal position (upper-left corner) 
          ({\textless}int{\textgreater})

          \item[y]

          vertical position (upper-left corner) 
          ({\textless}int{\textgreater})

        \end{Ventry}

      \end{quote}

\textbf{Example:} fl\_set\_form\_position(pform, 125, 250)



\textbf{Status:} Tested + Doc + Demo = OK



    \end{boxedminipage}

    \label{xformslib:library:fl_set_form_title}
    \index{xformslib \textit{(package)}!xformslib.library \textit{(module)}!xformslib.library.fl\_set\_form\_title \textit{(function)}}

    \vspace{0.5ex}

\hspace{.8\funcindent}\begin{boxedminipage}{\funcwidth}

    \raggedright \textbf{fl\_set\_form\_title}(\textit{pForm}, \textit{title})

    \vspace{-1.5ex}

    \rule{\textwidth}{0.5\fboxrule}
\setlength{\parskip}{2ex}
    Changes the form title (and the icon name) after it is shown.

\setlength{\parskip}{1ex}
      \textbf{Parameters}
      \vspace{-1ex}

      \begin{quote}
        \begin{Ventry}{xxxxx}

          \item[pForm]

          form whose title has to be changed ({\textless}pointer to 
          xfdata.FL\_FORM{\textgreater})

          \item[title]

          new title text for the form ({\textless}string{\textgreater})

        \end{Ventry}

      \end{quote}

\textbf{Example:} fl\_set\_form\_title(pform, "My great form")



\textbf{Status:} Tested + Doc + NoDemo = OK



    \end{boxedminipage}

    \label{xformslib:library:fl_set_app_mainform}
    \index{xformslib \textit{(package)}!xformslib.library \textit{(module)}!xformslib.library.fl\_set\_app\_mainform \textit{(function)}}

    \vspace{0.5ex}

\hspace{.8\funcindent}\begin{boxedminipage}{\funcwidth}

    \raggedright \textbf{fl\_set\_app\_mainform}(\textit{pForm})

    \vspace{-1.5ex}

    \rule{\textwidth}{0.5\fboxrule}
\setlength{\parskip}{2ex}
    Designates the main form. By default, the main form is set 
    automatically by the library to the first full-bordered form shown.

\setlength{\parskip}{1ex}
      \textbf{Parameters}
      \vspace{-1ex}

      \begin{quote}
        \begin{Ventry}{xxxxx}

          \item[pForm]

          form to be set as main one ({\textless}pointer to 
          xfdata.FL\_FORM{\textgreater})

        \end{Ventry}

      \end{quote}

\textbf{Example:} fl\_set\_app\_mainform(pform2)



\textbf{Status:} Tested + Doc + Demo = OK



    \end{boxedminipage}

    \label{xformslib:library:fl_get_app_mainform}
    \index{xformslib \textit{(package)}!xformslib.library \textit{(module)}!xformslib.library.fl\_get\_app\_mainform \textit{(function)}}

    \vspace{0.5ex}

\hspace{.8\funcindent}\begin{boxedminipage}{\funcwidth}

    \raggedright \textbf{fl\_get\_app\_mainform}()

    \vspace{-1.5ex}

    \rule{\textwidth}{0.5\fboxrule}
\setlength{\parskip}{2ex}
    Return the current mainform.

\setlength{\parskip}{1ex}
      \textbf{Return Value}
    \vspace{-1ex}

      \begin{quote}
      main form ({\textless}pointer to xfdata.FL\_FORM{\textgreater})

      {\it (type=pForm)}

      \end{quote}

\textbf{Example:} fl\_get\_app\_mainform()



\textbf{Status:} Tested + Doc + NoDemo = OK



    \end{boxedminipage}

    \label{xformslib:library:fl_set_app_nomainform}
    \index{xformslib \textit{(package)}!xformslib.library \textit{(module)}!xformslib.library.fl\_set\_app\_nomainform \textit{(function)}}

    \vspace{0.5ex}

\hspace{.8\funcindent}\begin{boxedminipage}{\funcwidth}

    \raggedright \textbf{fl\_set\_app\_nomainform}(\textit{flag})

    \vspace{-1.5ex}

    \rule{\textwidth}{0.5\fboxrule}
\setlength{\parskip}{2ex}
    In some situations, either because the concept of an application main 
    form does not apply (for example, an application might have multiple 
    full-bordered windows), or under some (buggy) window managers, the 
    designation of a main form may cause stacking order problems. To 
    workaround these, it can disable the designation of a main form (must 
    be called before any full-bordered form is shown)

\setlength{\parskip}{1ex}
      \textbf{Parameters}
      \vspace{-1ex}

      \begin{quote}
        \begin{Ventry}{xxxx}

          \item[flag]

          flag to disable/enable mainform designation 
          ({\textless}int{\textgreater})

            {\it (type=1 (to disable designation) or 1 (to enable designation))}

        \end{Ventry}

      \end{quote}

\textbf{Example:} fl\_set\_app\_nomainform(1)



\textbf{Status:} Tested + Doc + NoDemo = OK



    \end{boxedminipage}

    \label{xformslib:library:fl_set_form_callback}
    \index{xformslib \textit{(package)}!xformslib.library \textit{(module)}!xformslib.library.fl\_set\_form\_callback \textit{(function)}}

    \vspace{0.5ex}

\hspace{.8\funcindent}\begin{boxedminipage}{\funcwidth}

    \raggedright \textbf{fl\_set\_form\_callback}(\textit{pForm}, \textit{py\_FormCallbackPtr}, \textit{vdata})

    \vspace{-1.5ex}

    \rule{\textwidth}{0.5\fboxrule}
\setlength{\parskip}{2ex}
    Sets the callback function bound to an entire form. Whenever 
    fl\_do\_forms() or fl\_check\_forms() would return an object in form 
    they call the routine callback instead, with the object as an argument.
    So With each form you can associate its own callback routine. For 
    objects that have their own callbacks, the object callbacks have 
    priority over the form callback.

\setlength{\parskip}{1ex}
      \textbf{Parameters}
      \vspace{-1ex}

      \begin{quote}
        \begin{Ventry}{xxxxxxxxxxxxxxxxxx}

          \item[pForm]

          form whose callback has to be set ({\textless}pointer to 
          xfdata.FL\_FORM{\textgreater})

          \item[py\_FormCallbackPtr]

          python callback to be set - no return

            {\it (type=\_\_ funcname (pObject, ptr\_void) \_\_)}

          \item[vdata]

          user data to be passed to function ({\textless}pointer to 
          void{\textgreater}))

        \end{Ventry}

      \end{quote}

\textbf{Example:}
\begin{quote}
  \begin{itemize}

  \item
    \setlength{\parskip}{0.6ex}
def formcb(pobj, vdata):



  \item {\textbar}-{\textgreater}{\textbar} ...



  \item fl\_set\_form\_callback(pform, formcb, None)



\end{itemize}

\end{quote}

\textbf{Status:} Tested + Doc + Demo = OK



    \end{boxedminipage}

    \label{xformslib:library:fl_set_form_callback}
    \index{xformslib \textit{(package)}!xformslib.library \textit{(module)}!xformslib.library.fl\_set\_form\_callback \textit{(function)}}

    \vspace{0.5ex}

\hspace{.8\funcindent}\begin{boxedminipage}{\funcwidth}

    \raggedright \textbf{fl\_set\_form\_call\_back}(\textit{pForm}, \textit{py\_FormCallbackPtr}, \textit{vdata})

    \vspace{-1.5ex}

    \rule{\textwidth}{0.5\fboxrule}
\setlength{\parskip}{2ex}
    Sets the callback function bound to an entire form. Whenever 
    fl\_do\_forms() or fl\_check\_forms() would return an object in form 
    they call the routine callback instead, with the object as an argument.
    So With each form you can associate its own callback routine. For 
    objects that have their own callbacks, the object callbacks have 
    priority over the form callback.

\setlength{\parskip}{1ex}
      \textbf{Parameters}
      \vspace{-1ex}

      \begin{quote}
        \begin{Ventry}{xxxxxxxxxxxxxxxxxx}

          \item[pForm]

          form whose callback has to be set ({\textless}pointer to 
          xfdata.FL\_FORM{\textgreater})

          \item[py\_FormCallbackPtr]

          python callback to be set - no return

            {\it (type=\_\_ funcname (pObject, ptr\_void) \_\_)}

          \item[vdata]

          user data to be passed to function ({\textless}pointer to 
          void{\textgreater}))

        \end{Ventry}

      \end{quote}

\textbf{Example:}
\begin{quote}
  \begin{itemize}

  \item
    \setlength{\parskip}{0.6ex}
def formcb(pobj, vdata):



  \item {\textbar}-{\textgreater}{\textbar} ...



  \item fl\_set\_form\_callback(pform, formcb, None)



\end{itemize}

\end{quote}

\textbf{Status:} Tested + Doc + Demo = OK



    \end{boxedminipage}

    \label{xformslib:library:fl_set_form_size}
    \index{xformslib \textit{(package)}!xformslib.library \textit{(module)}!xformslib.library.fl\_set\_form\_size \textit{(function)}}

    \vspace{0.5ex}

\hspace{.8\funcindent}\begin{boxedminipage}{\funcwidth}

    \raggedright \textbf{fl\_set\_form\_size}(\textit{pForm}, \textit{w}, \textit{h})

    \vspace{-1.5ex}

    \rule{\textwidth}{0.5\fboxrule}
\setlength{\parskip}{2ex}
    Sets the size of form.

\setlength{\parskip}{1ex}
      \textbf{Parameters}
      \vspace{-1ex}

      \begin{quote}
        \begin{Ventry}{xxxxx}

          \item[pForm]

          form whose size has to be set ({\textless}pointer to 
          xfdata.FL\_FORM{\textgreater})

          \item[w]

          width of form in coord units ({\textless}int{\textgreater})

          \item[h]

          height of form in coord units ({\textless}int{\textgreater})

        \end{Ventry}

      \end{quote}

\textbf{Example:} fl\_set\_form\_size(pform, 200, 200)



\textbf{Status:} Tested + Doc + Demo = OK



    \end{boxedminipage}

    \label{xformslib:library:fl_set_form_hotspot}
    \index{xformslib \textit{(package)}!xformslib.library \textit{(module)}!xformslib.library.fl\_set\_form\_hotspot \textit{(function)}}

    \vspace{0.5ex}

\hspace{.8\funcindent}\begin{boxedminipage}{\funcwidth}

    \raggedright \textbf{fl\_set\_form\_hotspot}(\textit{pForm}, \textit{x}, \textit{y})

    \vspace{-1.5ex}

    \rule{\textwidth}{0.5\fboxrule}
\setlength{\parskip}{2ex}
    Sets the position of the hotspot, for showing a form so that a 
    particular point is under the mouse. You have to use 
    xfdata.FL\_PLACE\_HOTSPOT as place argument in fl\_show\_form().

\setlength{\parskip}{1ex}
      \textbf{Parameters}
      \vspace{-1ex}

      \begin{quote}
        \begin{Ventry}{xxxxx}

          \item[pForm]

          form to be set ({\textless}pointer to 
          xfdata.FL\_FORM{\textgreater})

          \item[x]

          horizontal position (upper-left corner) 
          ({\textless}int{\textgreater})

          \item[y]

          vertical position (upper-left corner) 
          ({\textless}int{\textgreater})

        \end{Ventry}

      \end{quote}

\textbf{Example:} fl\_set\_form\_hotspot(pform, 300, 50)



\textbf{Status:} Tested + Doc + NoDemo = OK



    \end{boxedminipage}

    \label{xformslib:library:fl_set_form_hotobject}
    \index{xformslib \textit{(package)}!xformslib.library \textit{(module)}!xformslib.library.fl\_set\_form\_hotobject \textit{(function)}}

    \vspace{0.5ex}

\hspace{.8\funcindent}\begin{boxedminipage}{\funcwidth}

    \raggedright \textbf{fl\_set\_form\_hotobject}(\textit{pForm}, \textit{pObject})

    \vspace{-1.5ex}

    \rule{\textwidth}{0.5\fboxrule}
\setlength{\parskip}{2ex}
    Sets the hotspot for showing a form so that a particular object is 
    under the mouse. You have to use xfdata.FL\_PLACE\_HOTSPOT as place 
    argument in fl\_show\_form().

\setlength{\parskip}{1ex}
      \textbf{Parameters}
      \vspace{-1ex}

      \begin{quote}
        \begin{Ventry}{xxxxxxx}

          \item[pForm]

          form whose object has to be set ({\textless}pointer to 
          xfdata.FL\_FORM{\textgreater})

          \item[pObject]

          object ({\textless}pointer to xfdata.FL\_OBJECT{\textgreater})

        \end{Ventry}

      \end{quote}

\textbf{Example:} fl\_set\_form\_hotobject(pform, pobj)



\textbf{Status:} Tested + Doc + Demo = OK



    \end{boxedminipage}

    \label{xformslib:library:fl_set_form_minsize}
    \index{xformslib \textit{(package)}!xformslib.library \textit{(module)}!xformslib.library.fl\_set\_form\_minsize \textit{(function)}}

    \vspace{0.5ex}

\hspace{.8\funcindent}\begin{boxedminipage}{\funcwidth}

    \raggedright \textbf{fl\_set\_form\_minsize}(\textit{pForm}, \textit{w}, \textit{h})

    \vspace{-1.5ex}

    \rule{\textwidth}{0.5\fboxrule}
\setlength{\parskip}{2ex}
    Sets the minimum size a form can have, if interactive resizing is 
    allowed (e.g., by showing the form with xfdata.FL\_PLACE\_POSITION).

\setlength{\parskip}{1ex}
      \textbf{Parameters}
      \vspace{-1ex}

      \begin{quote}
        \begin{Ventry}{xxxxx}

          \item[pForm]

          form ({\textless}pointer to xfdata.FL\_FORM{\textgreater})

          \item[w]

          width of form in coord units ({\textless}int{\textgreater})

          \item[h]

          height of form in coord units ({\textless}int{\textgreater})

        \end{Ventry}

      \end{quote}

\textbf{Example:} fl\_set\_form\_minsize(pform, 200, 300)



\textbf{Status:} Tested + Doc + NoDemo = OK



    \end{boxedminipage}

    \label{xformslib:library:fl_set_form_maxsize}
    \index{xformslib \textit{(package)}!xformslib.library \textit{(module)}!xformslib.library.fl\_set\_form\_maxsize \textit{(function)}}

    \vspace{0.5ex}

\hspace{.8\funcindent}\begin{boxedminipage}{\funcwidth}

    \raggedright \textbf{fl\_set\_form\_maxsize}(\textit{pForm}, \textit{w}, \textit{h})

    \vspace{-1.5ex}

    \rule{\textwidth}{0.5\fboxrule}
\setlength{\parskip}{2ex}
    Sets the maximum size a form can have, if interactive resizing is 
    allowed (e.g. by showing the form with xfdata.FL\_PLACE\_POSITION).

\setlength{\parskip}{1ex}
      \textbf{Parameters}
      \vspace{-1ex}

      \begin{quote}
        \begin{Ventry}{xxxxx}

          \item[pForm]

          form whose size has to be set ({\textless}pointer to 
          xdata.FL\_FORM{\textgreater})

          \item[w]

          width of form in coord units ({\textless}int{\textgreater})

          \item[h]

          height of form in coord units ({\textless}int{\textgreater})

        \end{Ventry}

      \end{quote}

\textbf{Example:} fl\_set\_form\_maxsize(pform, 400, 450)



\textbf{Status:} Tested + Doc + NoDemo = OK



    \end{boxedminipage}

    \label{xformslib:library:fl_set_form_event_cmask}
    \index{xformslib \textit{(package)}!xformslib.library \textit{(module)}!xformslib.library.fl\_set\_form\_event\_cmask \textit{(function)}}

    \vspace{0.5ex}

\hspace{.8\funcindent}\begin{boxedminipage}{\funcwidth}

    \raggedright \textbf{fl\_set\_form\_event\_cmask}(\textit{pForm}, \textit{cmask})

    \vspace{-1.5ex}

    \rule{\textwidth}{0.5\fboxrule}
\setlength{\parskip}{2ex}
    Sets the event compress mask a form can react to.

\setlength{\parskip}{1ex}
      \textbf{Parameters}
      \vspace{-1ex}

      \begin{quote}
        \begin{Ventry}{xxxxx}

          \item[pForm]

          form ({\textless}pointer to xfdata.FL\_FORM{\textgreater})

          \item[cmask]

          event compress mask for form ({\textless}long\_pos{\textgreater})

            {\it (type=(from xfdata module) one or more OR-ed between NoEventMask, KeyPressMask, 
KeyReleaseMask, ButtonPressMask, ButtonReleaseMask, EnterWindowMask, 
LeaveWindowMask, PointerMotionMask, PointerMotionHintMask, 
Button1MotionMask, Button2MotionMask, Button3MotionMask, Button4MotionMask,
Button5MotionMask, ButtonMotionMask, KeymapStateMask, ExposureMask, 
VisibilityChangeMask, StructureNotifyMask, ResizeRedirectMask, 
SubstructureNotifyMask, SubstructureRedirectMask, FocusChangeMask, 
ColormapChangeMask, OwnerGrabButtonMask, FL\_ALL\_EVENT, ... ?)}

        \end{Ventry}

      \end{quote}

\textbf{Example:} fl\_set\_form\_event\_cmask(pform, xfdata.FL\_ALL\_EVENT)



\textbf{Status:} Tested + Doc + NoDemo = OK



    \end{boxedminipage}

    \label{xformslib:library:fl_get_form_event_cmask}
    \index{xformslib \textit{(package)}!xformslib.library \textit{(module)}!xformslib.library.fl\_get\_form\_event\_cmask \textit{(function)}}

    \vspace{0.5ex}

\hspace{.8\funcindent}\begin{boxedminipage}{\funcwidth}

    \raggedright \textbf{fl\_get\_form\_event\_cmask}(\textit{pForm})

    \vspace{-1.5ex}

    \rule{\textwidth}{0.5\fboxrule}
\setlength{\parskip}{2ex}
    Returns event compress mask a form can react to.

\setlength{\parskip}{1ex}
      \textbf{Parameters}
      \vspace{-1ex}

      \begin{quote}
        \begin{Ventry}{xxxxx}

          \item[pForm]

          pointer to form ({\textless}pointer to 
          xfdata.FL\_FORM{\textgreater})

        \end{Ventry}

      \end{quote}

      \textbf{Return Value}
    \vspace{-1ex}

      \begin{quote}
      event mask {\textless}long\_pos{\textgreater}

      {\it (type=compress mask ID)}

      \end{quote}

\textbf{Example:} cmaskid = fl\_get\_form\_event\_cmask(pform)



\textbf{Status:} Tested + Doc + NoDemo = OK



    \end{boxedminipage}

    \label{xformslib:library:fl_set_form_geometry}
    \index{xformslib \textit{(package)}!xformslib.library \textit{(module)}!xformslib.library.fl\_set\_form\_geometry \textit{(function)}}

    \vspace{0.5ex}

\hspace{.8\funcindent}\begin{boxedminipage}{\funcwidth}

    \raggedright \textbf{fl\_set\_form\_geometry}(\textit{pForm}, \textit{x}, \textit{y}, \textit{w}, \textit{h})

    \vspace{-1.5ex}

    \rule{\textwidth}{0.5\fboxrule}
\setlength{\parskip}{2ex}
    Sets the geometry (position and size) of a form.

\setlength{\parskip}{1ex}
      \textbf{Parameters}
      \vspace{-1ex}

      \begin{quote}
        \begin{Ventry}{xxxxx}

          \item[pForm]

          pointer to form to be set ({\textless}pointer to 
          xfdata.FL\_FORM{\textgreater})

          \item[x]

          horizontal position (upper-left corner) 
          ({\textless}int{\textgreater})

          \item[y]

          vertical position (upper-left corner) 
          ({\textless}int{\textgreater})

          \item[w]

          width of form in coord units ({\textless}int{\textgreater})

          \item[h]

          height of form in coord units ({\textless}int{\textgreater})

        \end{Ventry}

      \end{quote}

\textbf{Example:} fl\_set\_form\_geometry(pform, 300, 400, 150, 150)



\textbf{Status:} Tested + Doc + Demo = OK



    \end{boxedminipage}

    \label{xformslib:library:fl_set_form_geometry}
    \index{xformslib \textit{(package)}!xformslib.library \textit{(module)}!xformslib.library.fl\_set\_form\_geometry \textit{(function)}}

    \vspace{0.5ex}

\hspace{.8\funcindent}\begin{boxedminipage}{\funcwidth}

    \raggedright \textbf{fl\_set\_initial\_placement}(\textit{pForm}, \textit{x}, \textit{y}, \textit{w}, \textit{h})

    \vspace{-1.5ex}

    \rule{\textwidth}{0.5\fboxrule}
\setlength{\parskip}{2ex}
    Sets the geometry (position and size) of a form.

\setlength{\parskip}{1ex}
      \textbf{Parameters}
      \vspace{-1ex}

      \begin{quote}
        \begin{Ventry}{xxxxx}

          \item[pForm]

          pointer to form to be set ({\textless}pointer to 
          xfdata.FL\_FORM{\textgreater})

          \item[x]

          horizontal position (upper-left corner) 
          ({\textless}int{\textgreater})

          \item[y]

          vertical position (upper-left corner) 
          ({\textless}int{\textgreater})

          \item[w]

          width of form in coord units ({\textless}int{\textgreater})

          \item[h]

          height of form in coord units ({\textless}int{\textgreater})

        \end{Ventry}

      \end{quote}

\textbf{Example:} fl\_set\_form\_geometry(pform, 300, 400, 150, 150)



\textbf{Status:} Tested + Doc + Demo = OK



    \end{boxedminipage}

    \label{xformslib:library:fl_show_form}
    \index{xformslib \textit{(package)}!xformslib.library \textit{(module)}!xformslib.library.fl\_show\_form \textit{(function)}}

    \vspace{0.5ex}

\hspace{.8\funcindent}\begin{boxedminipage}{\funcwidth}

    \raggedright \textbf{fl\_show\_form}(\textit{pForm}, \textit{place}, \textit{border}, \textit{title})

    \vspace{-1.5ex}

    \rule{\textwidth}{0.5\fboxrule}
\setlength{\parskip}{2ex}
    Shows the form.

\setlength{\parskip}{1ex}
      \textbf{Parameters}
      \vspace{-1ex}

      \begin{quote}
        \begin{Ventry}{xxxxxx}

          \item[pForm]

          form to be shown ({\textless}pointer to 
          xfdata.FL\_FORM{\textgreater})

          \item[place]

          where form has to be placed ({\textless}int{\textgreater})

            {\it (type=(from xfdata module) FL\_PLACE\_FREE, FL\_PLACE\_MOUSE, FL\_PLACE\_CENTER, 
FL\_PLACE\_POSITION, FL\_PLACE\_SIZE, FL\_PLACE\_GEOMETRY, 
FL\_PLACE\_ASPECT, FL\_PLACE\_FULLSCREEN, FL\_PLACE\_HOTSPOT, 
FL\_PLACE\_ICONIC, FL\_FREE\_SIZE, FL\_PLACE\_FREE\_CENTER, 
FL\_PLACE\_CENTERFREE)}

          \item[border]

          window manager decoration ({\textless}int{\textgreater})

            {\it (type=(from xfdata module) FL\_FULLBORDER, FL\_TRANSIENT, FL\_NOBORDER)}

          \item[title]

          title of form ({\textless}string{\textgreater})

        \end{Ventry}

      \end{quote}

      \textbf{Return Value}
    \vspace{-1ex}

      \begin{quote}
      window id ({\textless}long\_pos{\textgreater})

      {\it (type=win)}

      \end{quote}

\textbf{Example:} wind = fl\_show\_form(pform0, FL\_PLACE\_FREE, FL\_FULLBORDER, "MyForm")



\textbf{Status:} Tested + Doc + Demo = OK



    \end{boxedminipage}

    \label{xformslib:library:fl_hide_form}
    \index{xformslib \textit{(package)}!xformslib.library \textit{(module)}!xformslib.library.fl\_hide\_form \textit{(function)}}

    \vspace{0.5ex}

\hspace{.8\funcindent}\begin{boxedminipage}{\funcwidth}

    \raggedright \textbf{fl\_hide\_form}(\textit{pForm})

    \vspace{-1.5ex}

    \rule{\textwidth}{0.5\fboxrule}
\setlength{\parskip}{2ex}
    Hides the form.

\setlength{\parskip}{1ex}
      \textbf{Parameters}
      \vspace{-1ex}

      \begin{quote}
        \begin{Ventry}{xxxxx}

          \item[pForm]

          form to be hidden ({\textless}pointer to 
          xfdata.FL\_FORM{\textgreater})

        \end{Ventry}

      \end{quote}

\textbf{Example:} fl\_hide\_form(pform0)



\textbf{Status:} Tested + Doc + Demo = OK



    \end{boxedminipage}

    \label{xformslib:library:fl_free_form}
    \index{xformslib \textit{(package)}!xformslib.library \textit{(module)}!xformslib.library.fl\_free\_form \textit{(function)}}

    \vspace{0.5ex}

\hspace{.8\funcindent}\begin{boxedminipage}{\funcwidth}

    \raggedright \textbf{fl\_free\_form}(\textit{pForm})

    \vspace{-1.5ex}

    \rule{\textwidth}{0.5\fboxrule}
\setlength{\parskip}{2ex}
    Frees the memory used by a form, hiding and deleting it together with 
    all its objects.

\setlength{\parskip}{1ex}
      \textbf{Parameters}
      \vspace{-1ex}

      \begin{quote}
        \begin{Ventry}{xxxxx}

          \item[pForm]

          form to be freed ({\textless}pointer to 
          xfdata.FL\_FORM{\textgreater})

        \end{Ventry}

      \end{quote}

\textbf{Example:} fl\_free\_form(pform0)



\textbf{Status:} Tested + Doc + Demo = OK



    \end{boxedminipage}

    \label{xformslib:library:fl_redraw_form}
    \index{xformslib \textit{(package)}!xformslib.library \textit{(module)}!xformslib.library.fl\_redraw\_form \textit{(function)}}

    \vspace{0.5ex}

\hspace{.8\funcindent}\begin{boxedminipage}{\funcwidth}

    \raggedright \textbf{fl\_redraw\_form}(\textit{pForm})

    \vspace{-1.5ex}

    \rule{\textwidth}{0.5\fboxrule}
\setlength{\parskip}{2ex}
    (Re)draws an entire form.

\setlength{\parskip}{1ex}
      \textbf{Parameters}
      \vspace{-1ex}

      \begin{quote}
        \begin{Ventry}{xxxxx}

          \item[pForm]

          form to redraw ({\textless}pointer to 
          xfdata.FL\_FORM{\textgreater})

        \end{Ventry}

      \end{quote}

\textbf{Example:} fl\_redraw\_form(pform0)



\textbf{Status:} Tested + Doc + Demo = OK



    \end{boxedminipage}

    \label{xformslib:library:fl_set_form_dblbuffer}
    \index{xformslib \textit{(package)}!xformslib.library \textit{(module)}!xformslib.library.fl\_set\_form\_dblbuffer \textit{(function)}}

    \vspace{0.5ex}

\hspace{.8\funcindent}\begin{boxedminipage}{\funcwidth}

    \raggedright \textbf{fl\_set\_form\_dblbuffer}(\textit{pForm}, \textit{flag})

    \vspace{-1.5ex}

    \rule{\textwidth}{0.5\fboxrule}
\setlength{\parskip}{2ex}
    Uses double buffering on a per-form basis. Since Xlib doesn't support 
    double buffering, XForms library simulates this functionality with 
    pixmap bit-bliting. In practice, the effect is hardly distinguishable 
    from double buffering and performance is on par with multi-buffering 
    extensions (It is slower than drawing into a window directly on most 
    workstations however). Bear in mind that a pixmap can be resource 
    hungry, so use this option with discretion.

\setlength{\parskip}{1ex}
      \textbf{Parameters}
      \vspace{-1ex}

      \begin{quote}
        \begin{Ventry}{xxxxx}

          \item[pForm]

          form to set ({\textless}pointer to xfdata.FL\_FORM{\textgreater})

          \item[flag]

          flag to disable/enable doublebuffer 
          ({\textless}int{\textgreater})

            {\it (type=0 (disabled) or 1 (enabled))}

        \end{Ventry}

      \end{quote}

\textbf{Example:} fl\_set\_form\_dblbuffer(1)



\textbf{Status:} Tested + Doc + Demo = OK



    \end{boxedminipage}

    \label{xformslib:library:fl_prepare_form_window}
    \index{xformslib \textit{(package)}!xformslib.library \textit{(module)}!xformslib.library.fl\_prepare\_form\_window \textit{(function)}}

    \vspace{0.5ex}

\hspace{.8\funcindent}\begin{boxedminipage}{\funcwidth}

    \raggedright \textbf{fl\_prepare\_form\_window}(\textit{pForm}, \textit{place}, \textit{border}, \textit{title})

    \vspace{-1.5ex}

    \rule{\textwidth}{0.5\fboxrule}
\setlength{\parskip}{2ex}
    Creates a window that obeys any and all constraints just as 
    fl\_show\_form() does but remains unmapped (not shown), returning its 
    window handle. You after need fl\_show\_form\_window() to show it.

\setlength{\parskip}{1ex}
      \textbf{Parameters}
      \vspace{-1ex}

      \begin{quote}
        \begin{Ventry}{xxxxxx}

          \item[pForm]

          form to display ({\textless}pointer to 
          xfdata.FL\_FORM{\textgreater})

          \item[place]

          where has to be placed ({\textless}int{\textgreater})

            {\it (type=(from xfdata module) FL\_PLACE\_FREE, FL\_PLACE\_MOUSE, FL\_PLACE\_CENTER, 
FL\_PLACE\_POSITION, FL\_PLACE\_SIZE, FL\_PLACE\_GEOMETRY, 
FL\_PLACE\_ASPECT, FL\_PLACE\_FULLSCREEN, FL\_PLACE\_HOTSPOT, 
FL\_PLACE\_ICONIC, FL\_FREE\_SIZE, FL\_PLACE\_FREE\_CENTER, 
FL\_PLACE\_CENTERFREE)}

          \item[border]

          window manager decoration ({\textless}int{\textgreater})

            {\it (type=(from xfdata module) FL\_FULLBORDER, FL\_TRANSIENT, FL\_NOBORDER)}

          \item[title]

          text title of form ({\textless}string{\textgreater})

        \end{Ventry}

      \end{quote}

      \textbf{Return Value}
    \vspace{-1ex}

      \begin{quote}
      window id ({\textless}long\_pos{\textgreater})

      {\it (type=win)}

      \end{quote}

\textbf{Example:} wind = fl\_prepare\_form\_window(pform2, FL\_PLACE\_FREE, FL\_FULLBORDER, 
"MyForm")



\textbf{Status:} Tested + Doc + NoDemo = OK



    \end{boxedminipage}

    \label{xformslib:library:fl_show_form_window}
    \index{xformslib \textit{(package)}!xformslib.library \textit{(module)}!xformslib.library.fl\_show\_form\_window \textit{(function)}}

    \vspace{0.5ex}

\hspace{.8\funcindent}\begin{boxedminipage}{\funcwidth}

    \raggedright \textbf{fl\_show\_form\_window}(\textit{pForm})

    \vspace{-1.5ex}

    \rule{\textwidth}{0.5\fboxrule}
\setlength{\parskip}{2ex}
    Maps (shows) a window of form that has been created before with 
    fl\_prepare\_form\_window().

\setlength{\parskip}{1ex}
      \textbf{Parameters}
      \vspace{-1ex}

      \begin{quote}
        \begin{Ventry}{xxxxx}

          \item[pForm]

          form whose window has to be shown ({\textless}pointer to 
          xfdata.FL\_FORM{\textgreater})

        \end{Ventry}

      \end{quote}

      \textbf{Return Value}
    \vspace{-1ex}

      \begin{quote}
      window id ({\textless}long\_pos{\textgreater})

      {\it (type=win)}

      \end{quote}

\textbf{Example:} win1 = fl\_show\_form\_window(pform2)



\textbf{Status:} Tested + Doc + Demo = OK



    \end{boxedminipage}

    \label{xformslib:library:fl_adjust_form_size}
    \index{xformslib \textit{(package)}!xformslib.library \textit{(module)}!xformslib.library.fl\_adjust\_form\_size \textit{(function)}}

    \vspace{0.5ex}

\hspace{.8\funcindent}\begin{boxedminipage}{\funcwidth}

    \raggedright \textbf{fl\_adjust\_form\_size}(\textit{pForm})

    \vspace{-1.5ex}

    \rule{\textwidth}{0.5\fboxrule}
\setlength{\parskip}{2ex}
    Similar to fl\_fit\_object\_label, but will do it for all objects and 
    has a smaller threshold. Mainly intended for compensation for font size
    variations.

\setlength{\parskip}{1ex}
      \textbf{Parameters}
      \vspace{-1ex}

      \begin{quote}
        \begin{Ventry}{xxxxx}

          \item[pForm]

          form whose size has to be adjusted ({\textless}pointer to 
          xfdata.FL\_FORM{\textgreater})

        \end{Ventry}

      \end{quote}

      \textbf{Return Value}
    \vspace{-1ex}

      \begin{quote}
      max factor id ({\textless}float{\textgreater})

      {\it (type=maxfact)}

      \end{quote}

\textbf{Example:} mfactor = fl\_adjust\_form\_size(pform)



\textbf{Status:} Tested + Doc + Demo = OK



    \end{boxedminipage}

    \label{xformslib:library:fl_form_is_visible}
    \index{xformslib \textit{(package)}!xformslib.library \textit{(module)}!xformslib.library.fl\_form\_is\_visible \textit{(function)}}

    \vspace{0.5ex}

\hspace{.8\funcindent}\begin{boxedminipage}{\funcwidth}

    \raggedright \textbf{fl\_form\_is\_visible}(\textit{pForm})

    \vspace{-1.5ex}

    \rule{\textwidth}{0.5\fboxrule}
\setlength{\parskip}{2ex}
    Returns if form is visible or not.

\setlength{\parskip}{1ex}
      \textbf{Parameters}
      \vspace{-1ex}

      \begin{quote}
        \begin{Ventry}{xxxxx}

          \item[pForm]

          form to evaluate ({\textless}pointer to 
          xfdata.FL\_FORM{\textgreater})

        \end{Ventry}

      \end{quote}

      \textbf{Return Value}
    \vspace{-1ex}

      \begin{quote}
      visibility state (0 invisible, non-zero visible) 
      ({\textless}int{\textgreater})

      {\it (type=state)}

      \end{quote}

\textbf{Example:} visib = fl\_form\_is\_visible(pform)



\textbf{Status:} Tested + Doc + NoDemo = OK



    \end{boxedminipage}

    \label{xformslib:library:fl_form_is_iconified}
    \index{xformslib \textit{(package)}!xformslib.library \textit{(module)}!xformslib.library.fl\_form\_is\_iconified \textit{(function)}}

    \vspace{0.5ex}

\hspace{.8\funcindent}\begin{boxedminipage}{\funcwidth}

    \raggedright \textbf{fl\_form\_is\_iconified}(\textit{pForm})

    \vspace{-1.5ex}

    \rule{\textwidth}{0.5\fboxrule}
\setlength{\parskip}{2ex}
    Returns if a form's window is in iconified state or not.

\setlength{\parskip}{1ex}
      \textbf{Parameters}
      \vspace{-1ex}

      \begin{quote}
        \begin{Ventry}{xxxxx}

          \item[pForm]

          form to evaluate ({\textless}pointer to 
          xfdata.FL\_FORM{\textgreater})

        \end{Ventry}

      \end{quote}

      \textbf{Return Value}
    \vspace{-1ex}

      \begin{quote}
      iconic state (0 not iconified, non-zero iconified) 
      ({\textless}int{\textgreater})

      {\it (type=state)}

      \end{quote}

\textbf{Example:} iconif = fl\_form\_is\_iconified(pform)



\textbf{Status:} Tested + Doc + NoDemo = OK



    \end{boxedminipage}

    \label{xformslib:library:fl_register_raw_callback}
    \index{xformslib \textit{(package)}!xformslib.library \textit{(module)}!xformslib.library.fl\_register\_raw\_callback \textit{(function)}}

    \vspace{0.5ex}

\hspace{.8\funcindent}\begin{boxedminipage}{\funcwidth}

    \raggedright \textbf{fl\_register\_raw\_callback}(\textit{pForm}, \textit{mask}, \textit{py\_RawCallback})

    \vspace{-1.5ex}

    \rule{\textwidth}{0.5\fboxrule}
\setlength{\parskip}{2ex}
    Register pre-emptive event handlers. Only one handler is allowed for 
    each eent pair.

\setlength{\parskip}{1ex}
      \textbf{Parameters}
      \vspace{-1ex}

      \begin{quote}
        \begin{Ventry}{xxxxxxxxxxxxxx}

          \item[pForm]

          form ({\textless}pointer to xfdata.FL\_FORM{\textgreater})

          \item[mask]

          key/button/window event mask (press, release, motion, enter, 
          leave etc..) ({\textless}long\_pos{\textgreater})

            {\it (type=(from xfdata module) KeyPressMask and KeyReleaseMask, ButtonPressMask and 
ButtonReleaseMask, EnterWindowMask and LeaveWindowMask, ButtonMotionMask 
and PointerMotionMask, FL\_ALL\_EVENT)}

          \item[py\_RawCallback]

          python callback function, with return value

            {\it (type=\_\_ funcname (pForm, pXEvent) -{\textgreater} num \_\_)}

        \end{Ventry}

      \end{quote}

      \textbf{Return Value}
    \vspace{-1ex}

      \begin{quote}
      xfdata.FL\_RAW\_CALLBACK old function

      {\it (type=old raw\_callback func)}

      \end{quote}

\textbf{Example:}
\begin{quote}
  \begin{itemize}

  \item
    \setlength{\parskip}{0.6ex}
def rawcb(pform, xev):



  \item {\textbar}-{\textgreater}{\textbar} ...



  \item {\textbar}-{\textgreater}{\textbar} return 0



  \item oldrawcb = fl\_register\_callback(pform3, xfdata.KeyPressMask, rawcb)



\end{itemize}

\end{quote}

\textbf{Status:} Tested + Doc + Demo = OK



    \end{boxedminipage}

    \label{xformslib:library:fl_register_raw_callback}
    \index{xformslib \textit{(package)}!xformslib.library \textit{(module)}!xformslib.library.fl\_register\_raw\_callback \textit{(function)}}

    \vspace{0.5ex}

\hspace{.8\funcindent}\begin{boxedminipage}{\funcwidth}

    \raggedright \textbf{fl\_register\_call\_back}(\textit{pForm}, \textit{mask}, \textit{py\_RawCallback})

    \vspace{-1.5ex}

    \rule{\textwidth}{0.5\fboxrule}
\setlength{\parskip}{2ex}
    Register pre-emptive event handlers. Only one handler is allowed for 
    each eent pair.

\setlength{\parskip}{1ex}
      \textbf{Parameters}
      \vspace{-1ex}

      \begin{quote}
        \begin{Ventry}{xxxxxxxxxxxxxx}

          \item[pForm]

          form ({\textless}pointer to xfdata.FL\_FORM{\textgreater})

          \item[mask]

          key/button/window event mask (press, release, motion, enter, 
          leave etc..) ({\textless}long\_pos{\textgreater})

            {\it (type=(from xfdata module) KeyPressMask and KeyReleaseMask, ButtonPressMask and 
ButtonReleaseMask, EnterWindowMask and LeaveWindowMask, ButtonMotionMask 
and PointerMotionMask, FL\_ALL\_EVENT)}

          \item[py\_RawCallback]

          python callback function, with return value

            {\it (type=\_\_ funcname (pForm, pXEvent) -{\textgreater} num \_\_)}

        \end{Ventry}

      \end{quote}

      \textbf{Return Value}
    \vspace{-1ex}

      \begin{quote}
      xfdata.FL\_RAW\_CALLBACK old function

      {\it (type=old raw\_callback func)}

      \end{quote}

\textbf{Example:}
\begin{quote}
  \begin{itemize}

  \item
    \setlength{\parskip}{0.6ex}
def rawcb(pform, xev):



  \item {\textbar}-{\textgreater}{\textbar} ...



  \item {\textbar}-{\textgreater}{\textbar} return 0



  \item oldrawcb = fl\_register\_callback(pform3, xfdata.KeyPressMask, rawcb)



\end{itemize}

\end{quote}

\textbf{Status:} Tested + Doc + Demo = OK



    \end{boxedminipage}

    \label{xformslib:library:fl_bgn_group}
    \index{xformslib \textit{(package)}!xformslib.library \textit{(module)}!xformslib.library.fl\_bgn\_group \textit{(function)}}

    \vspace{0.5ex}

\hspace{.8\funcindent}\begin{boxedminipage}{\funcwidth}

    \raggedright \textbf{fl\_bgn\_group}()

    \vspace{-1.5ex}

    \rule{\textwidth}{0.5\fboxrule}
\setlength{\parskip}{2ex}
    Starts a group of objects definition. It purpose can be e.g. to define 
    a series of objects to be hidden or deactivated or to define a series 
    of radio buttons.

\setlength{\parskip}{1ex}
      \textbf{Return Value}
    \vspace{-1ex}

      \begin{quote}
      group started ({\textless}pointer to xfdata.FL\_OBJECT{\textgreater})

      {\it (type=pObject)}

      \end{quote}

\textbf{Example:} group0 = fl\_bgn\_group()



\textbf{Status:} Tested + Doc + Demo = OK



    \end{boxedminipage}

    \label{xformslib:library:fl_end_group}
    \index{xformslib \textit{(package)}!xformslib.library \textit{(module)}!xformslib.library.fl\_end\_group \textit{(function)}}

    \vspace{0.5ex}

\hspace{.8\funcindent}\begin{boxedminipage}{\funcwidth}

    \raggedright \textbf{fl\_end\_group}()

    \vspace{-1.5ex}

    \rule{\textwidth}{0.5\fboxrule}
\setlength{\parskip}{2ex}
    Ends a group definition.

\setlength{\parskip}{1ex}
\textbf{Example:} fl\_end\_group()



\textbf{Status:} Tested + Doc + Demo = OK



    \end{boxedminipage}

    \label{xformslib:library:fl_addto_group}
    \index{xformslib \textit{(package)}!xformslib.library \textit{(module)}!xformslib.library.fl\_addto\_group \textit{(function)}}

    \vspace{0.5ex}

\hspace{.8\funcindent}\begin{boxedminipage}{\funcwidth}

    \raggedright \textbf{fl\_addto\_group}(\textit{pObject})

    \vspace{-1.5ex}

    \rule{\textwidth}{0.5\fboxrule}
\setlength{\parskip}{2ex}
    Reopens a group (after fl\_end\_group) to allow addition of further 
    objects.

\setlength{\parskip}{1ex}
      \textbf{Parameters}
      \vspace{-1ex}

      \begin{quote}
        \begin{Ventry}{xxxxxxx}

          \item[pObject]

          group object to reopen ({\textless}pointer to 
          xfdata.FL\_OBJECT{\textgreater})

        \end{Ventry}

      \end{quote}

      \textbf{Return Value}
    \vspace{-1ex}

      \begin{quote}
      form ({\textless}pointer to xfdata.FL\_FORM{\textgreater}) or None 
      (on failure)

      {\it (type=pForm)}

      \end{quote}

\textbf{Example:} group1 = fl\_addto\_group(closedgroup)



\textbf{Status:} Tested + Doc + NoDemo = OK



    \end{boxedminipage}

    \label{xformslib:library:fl_get_object_objclass}
    \index{xformslib \textit{(package)}!xformslib.library \textit{(module)}!xformslib.library.fl\_get\_object\_objclass \textit{(function)}}

    \vspace{0.5ex}

\hspace{.8\funcindent}\begin{boxedminipage}{\funcwidth}

    \raggedright \textbf{fl\_get\_object\_objclass}(\textit{pObject})

    \vspace{-1.5ex}

    \rule{\textwidth}{0.5\fboxrule}
\setlength{\parskip}{2ex}
    Return the object class of an object. (e.g. button, lightbutton, box, 
    nmenu, counter, etc.)

\setlength{\parskip}{1ex}
      \textbf{Parameters}
      \vspace{-1ex}

      \begin{quote}
        \begin{Ventry}{xxxxxxx}

          \item[pObject]

          object to evaluate ({\textless}pointer to 
          xfc.FL\_OBJECT{\textgreater})

        \end{Ventry}

      \end{quote}

      \textbf{Return Value}
    \vspace{-1ex}

      \begin{quote}
      objclass id ({\textless}int{\textgreater}) or -1 (on failure)

      {\it (type=id)}

      \end{quote}

\textbf{Example:} obcls = fl\_get\_object\_objclass(pobj)



\textbf{Status:} Tested + Doc + NoDemo = OK



    \end{boxedminipage}

    \label{xformslib:library:fl_get_object_type}
    \index{xformslib \textit{(package)}!xformslib.library \textit{(module)}!xformslib.library.fl\_get\_object\_type \textit{(function)}}

    \vspace{0.5ex}

\hspace{.8\funcindent}\begin{boxedminipage}{\funcwidth}

    \raggedright \textbf{fl\_get\_object\_type}(\textit{pObject})

    \vspace{-1.5ex}

    \rule{\textwidth}{0.5\fboxrule}
\setlength{\parskip}{2ex}
    Return the type of an object (e.g. radio button, multiline input, 
    normal browser, etc..).

\setlength{\parskip}{1ex}
      \textbf{Parameters}
      \vspace{-1ex}

      \begin{quote}
        \begin{Ventry}{xxxxxxx}

          \item[pObject]

          object to evaluate ({\textless}pointer to 
          xfdata.FL\_OBJECT{\textgreater})

        \end{Ventry}

      \end{quote}

      \textbf{Return Value}
    \vspace{-1ex}

      \begin{quote}
      type id ({\textless}int{\textgreater}) or -1 (on failure)

      {\it (type=type id)}

      \end{quote}

\textbf{Example:} obtype = fl\_get\_object\_type(pobj)



\textbf{Status:} Tested + Doc + NoDemo = OK



    \end{boxedminipage}

    \label{xformslib:library:fl_set_object_boxtype}
    \index{xformslib \textit{(package)}!xformslib.library \textit{(module)}!xformslib.library.fl\_set\_object\_boxtype \textit{(function)}}

    \vspace{0.5ex}

\hspace{.8\funcindent}\begin{boxedminipage}{\funcwidth}

    \raggedright \textbf{fl\_set\_object\_boxtype}(\textit{pObject}, \textit{boxtype})

    \vspace{-1.5ex}

    \rule{\textwidth}{0.5\fboxrule}
\setlength{\parskip}{2ex}
    Sets the shape of box of an object. Not all possible boxtypes are 
    suitable for all types of objects.

\setlength{\parskip}{1ex}
      \textbf{Parameters}
      \vspace{-1ex}

      \begin{quote}
        \begin{Ventry}{xxxxxxx}

          \item[pObject]

          object whose boxtype has to be set ({\textless}pointer to 
          xfdata.FL\_OBJECT{\textgreater})

          \item[boxtype]

          type of the box to be set ({\textless}int{\textgreater})

            {\it (type=(from xfdata module) FL\_NO\_BOX, FL\_UP\_BOX, FL\_DOWN\_BOX, 
FL\_BORDER\_BOX, FL\_SHADOW\_BOX, FL\_FRAME\_BOX, FL\_ROUNDED\_BOX, 
FL\_EMBOSSED\_BOX, FL\_FLAT\_BOX, FL\_RFLAT\_BOX, FL\_RSHADOW\_BOX, 
FL\_OVAL\_BOX, FL\_ROUNDED3D\_UPBOX, FL\_ROUNDED3D\_DOWNBOX, 
FL\_OVAL3D\_UPBOX, FL\_OVAL3D\_DOWNBOX, FL\_OVAL3D\_FRAMEBOX, 
FL\_OVAL3D\_EMBOSSEDBOX)}

        \end{Ventry}

      \end{quote}

\textbf{Example:} fl\_set\_object\_boxtype(ptextobj, xfdata.FL\_BORDER\_BOX)



\textbf{Status:} Tested + Doc + Demo = OK



    \end{boxedminipage}

    \label{xformslib:library:fl_get_object_boxtype}
    \index{xformslib \textit{(package)}!xformslib.library \textit{(module)}!xformslib.library.fl\_get\_object\_boxtype \textit{(function)}}

    \vspace{0.5ex}

\hspace{.8\funcindent}\begin{boxedminipage}{\funcwidth}

    \raggedright \textbf{fl\_get\_object\_boxtype}(\textit{pObject})

    \vspace{-1.5ex}

    \rule{\textwidth}{0.5\fboxrule}
\setlength{\parskip}{2ex}
    Returns the current boxtype of an object (e.g. no box, up box, shadow 
    box, etc..).

\setlength{\parskip}{1ex}
      \textbf{Parameters}
      \vspace{-1ex}

      \begin{quote}
        \begin{Ventry}{xxxxxxx}

          \item[pObject]

          object to evaluate ({\textless}pointer to 
          xfdata.FL\_OBJECT{\textgreater})

        \end{Ventry}

      \end{quote}

      \textbf{Return Value}
    \vspace{-1ex}

      \begin{quote}
      boxtype id ({\textless}int{\textgreater}) or -1 (on failure)

      {\it (type=boxtype id)}

      \end{quote}

\textbf{Example:} boxtp = fl\_get\_object\_boxtype(ptextobj)



\textbf{Status:} Tested + Doc + NoDemo = OK



    \end{boxedminipage}

    \label{xformslib:library:fl_set_object_bw}
    \index{xformslib \textit{(package)}!xformslib.library \textit{(module)}!xformslib.library.fl\_set\_object\_bw \textit{(function)}}

    \vspace{0.5ex}

\hspace{.8\funcindent}\begin{boxedminipage}{\funcwidth}

    \raggedright \textbf{fl\_set\_object\_bw}(\textit{pObject}, \textit{bw})

    \vspace{-1.5ex}

    \rule{\textwidth}{0.5\fboxrule}
\setlength{\parskip}{2ex}
    Sets the borderwidth of an object. If requested borderwidth is 0, -1 is
    used. If set to a negative number, all objects appear to have a softer 
    appearance (e.g. -2).

\setlength{\parskip}{1ex}
      \textbf{Parameters}
      \vspace{-1ex}

      \begin{quote}
        \begin{Ventry}{xxxxxxx}

          \item[pObject]

          object ({\textless}pointer to xfdata.FL\_OBJECT{\textgreater})

          \item[bw]

          borderwidth of object to be set ({\textless}int{\textgreater})

        \end{Ventry}

      \end{quote}

\textbf{Example:} fl\_set\_object\_bw(pobj, 2)



\textbf{Status:} Tested + Doc + Demo = OK



    \end{boxedminipage}

    \label{xformslib:library:fl_get_object_bw}
    \index{xformslib \textit{(package)}!xformslib.library \textit{(module)}!xformslib.library.fl\_get\_object\_bw \textit{(function)}}

    \vspace{0.5ex}

\hspace{.8\funcindent}\begin{boxedminipage}{\funcwidth}

    \raggedright \textbf{fl\_get\_object\_bw}(\textit{pObject})

    \vspace{-1.5ex}

    \rule{\textwidth}{0.5\fboxrule}
\setlength{\parskip}{2ex}
    Returns the borderwidth of an object.

\setlength{\parskip}{1ex}
      \textbf{Parameters}
      \vspace{-1ex}

      \begin{quote}
        \begin{Ventry}{xxxxxxx}

          \item[pObject]

          object to evaluate ({\textless}pointer to 
          xfdata.FL\_OBJECT{\textgreater})

        \end{Ventry}

      \end{quote}

      \textbf{Return Value}
    \vspace{-1ex}

      \begin{quote}
      borderwidth ({\textless}int{\textgreater})

      {\it (type=bw)}

      \end{quote}

\textbf{Example:} currbw = fl\_get\_object\_bw(pobj)



\textbf{Attention:} API change from XForms - upstream was fl\_get\_object\_bw(pObject, bw)



\textbf{Status:} Tested + Doc + NoDemo = OK



    \end{boxedminipage}

    \label{xformslib:library:fl_set_object_resize}
    \index{xformslib \textit{(package)}!xformslib.library \textit{(module)}!xformslib.library.fl\_set\_object\_resize \textit{(function)}}

    \vspace{0.5ex}

\hspace{.8\funcindent}\begin{boxedminipage}{\funcwidth}

    \raggedright \textbf{fl\_set\_object\_resize}(\textit{pObject}, \textit{what})

    \vspace{-1.5ex}

    \rule{\textwidth}{0.5\fboxrule}
\setlength{\parskip}{2ex}
    Sets the resize property of an object.

\setlength{\parskip}{1ex}
      \textbf{Parameters}
      \vspace{-1ex}

      \begin{quote}
        \begin{Ventry}{xxxxxxx}

          \item[pObject]

          object to set ({\textless}pointer to 
          xfdata.FL\_OBJECT{\textgreater})

          \item[what]

          resize property ({\textless}int\_pos{\textgreater})

            {\it (type=(from xfdata module) FL\_RESIZE\_NONE, FL\_RESIZE\_X, FL\_RESIZE\_Y, 
FL\_RESIZE\_ALL)}

        \end{Ventry}

      \end{quote}

\textbf{Example:} fl\_set\_object\_resize(pobj, xfdata.FL\_RESIZE\_ALL)



\textbf{Status:} Tested + Doc + Demo = OK



    \end{boxedminipage}

    \label{xformslib:library:fl_get_object_resize}
    \index{xformslib \textit{(package)}!xformslib.library \textit{(module)}!xformslib.library.fl\_get\_object\_resize \textit{(function)}}

    \vspace{0.5ex}

\hspace{.8\funcindent}\begin{boxedminipage}{\funcwidth}

    \raggedright \textbf{fl\_get\_object\_resize}(\textit{pObject})

    \vspace{-1.5ex}

    \rule{\textwidth}{0.5\fboxrule}
\setlength{\parskip}{2ex}
    Returns the resize property of an object (e.g. resize all, resize none,
    etc..).

\setlength{\parskip}{1ex}
      \textbf{Parameters}
      \vspace{-1ex}

      \begin{quote}
        \begin{Ventry}{xxxxxxx}

          \item[pObject]

          object to evaluate ({\textless}pointer to 
          xfdata.FL\_OBJECT{\textgreater})

        \end{Ventry}

      \end{quote}

      \textbf{Return Value}
    \vspace{-1ex}

      \begin{quote}
      resize property ({\textless}int\_pos{\textgreater})

      {\it (type=what)}

      \end{quote}

\textbf{Attention:} API change from XForms - upstream was fl\_get\_object\_resize(pObject, 
what)



\textbf{Example:} reszprop = fl\_get\_object\_resize(pobj)



\textbf{Status:} Tested + Doc + NoDemo = OK



    \end{boxedminipage}

    \label{xformslib:library:fl_set_object_gravity}
    \index{xformslib \textit{(package)}!xformslib.library \textit{(module)}!xformslib.library.fl\_set\_object\_gravity \textit{(function)}}

    \vspace{0.5ex}

\hspace{.8\funcindent}\begin{boxedminipage}{\funcwidth}

    \raggedright \textbf{fl\_set\_object\_gravity}(\textit{pObject}, \textit{nw}, \textit{se})

    \vspace{-1.5ex}

    \rule{\textwidth}{0.5\fboxrule}
\setlength{\parskip}{2ex}
    Sets the gravity properties of an object.

\setlength{\parskip}{1ex}
      \textbf{Parameters}
      \vspace{-1ex}

      \begin{quote}
        \begin{Ventry}{xxxxxxx}

          \item[pObject]

          object to be set ({\textless}pointer to 
          xfdata.FL\_OBJECT{\textgreater})

          \item[nw]

          gravity property for NorthWest 
          ({\textless}int\_pos{\textgreater})

            {\it (type=(from xfdata module) FL\_North, FL\_NorthEast, FL\_NorthWest, FL\_South, 
FL\_SouthEast, FL\_SouthWest, FL\_East, FL\_West, FL\_NoGravity, 
FL\_ForgetGravity)}

          \item[se]

          gravity property for SouthEast 
          ({\textless}int\_pos{\textgreater})

            {\it (type=(from xfdata module) FL\_North, FL\_NorthEast, FL\_NorthWest, FL\_South, 
FL\_SouthEast, FL\_SouthWest, FL\_East, FL\_West, FL\_NoGravity, 
FL\_ForgetGravity)}

        \end{Ventry}

      \end{quote}

\textbf{Example:} fl\_set\_object\_gravity(pobj, xfdata.FL\_North, xfdata.FL\_East)



\textbf{Status:} Tested + Doc + Demo = OK



    \end{boxedminipage}

    \label{xformslib:library:fl_get_object_gravity}
    \index{xformslib \textit{(package)}!xformslib.library \textit{(module)}!xformslib.library.fl\_get\_object\_gravity \textit{(function)}}

    \vspace{0.5ex}

\hspace{.8\funcindent}\begin{boxedminipage}{\funcwidth}

    \raggedright \textbf{fl\_get\_object\_gravity}(\textit{pObject})

    \vspace{-1.5ex}

    \rule{\textwidth}{0.5\fboxrule}
\setlength{\parskip}{2ex}
    Returns the gravity properties of an object (e.g. North, SouthWest, 
    etc..).

\setlength{\parskip}{1ex}
      \textbf{Parameters}
      \vspace{-1ex}

      \begin{quote}
        \begin{Ventry}{xxxxxxx}

          \item[pObject]

          object to set ({\textless}pointer to 
          xfdata.FL\_OBJECT{\textgreater})

        \end{Ventry}

      \end{quote}

      \textbf{Return Value}
    \vspace{-1ex}

      \begin{quote}
      NorthWest and SouthEast gravity ({\textless}int\_pos{\textgreater}, 
      {\textless}int\_pos{\textgreater})

      {\it (type=nw, se)}

      \end{quote}

\textbf{Attention:} API change from XForms - upstream was fl\_get\_object\_gravity(pObject, nw,
se)



\textbf{Example:} nowe, soea = fl\_get\_object\_gravity(pobj)



\textbf{Status:} Tested + Doc + NoDemo = OK



    \end{boxedminipage}

    \label{xformslib:library:fl_set_object_lsize}
    \index{xformslib \textit{(package)}!xformslib.library \textit{(module)}!xformslib.library.fl\_set\_object\_lsize \textit{(function)}}

    \vspace{0.5ex}

\hspace{.8\funcindent}\begin{boxedminipage}{\funcwidth}

    \raggedright \textbf{fl\_set\_object\_lsize}(\textit{pObject}, \textit{size})

    \vspace{-1.5ex}

    \rule{\textwidth}{0.5\fboxrule}
\setlength{\parskip}{2ex}
    Sets the label size of an object.

\setlength{\parskip}{1ex}
      \textbf{Parameters}
      \vspace{-1ex}

      \begin{quote}
        \begin{Ventry}{xxxxxxx}

          \item[pObject]

          object to be set ({\textless}pointer to 
          xfdata.FL\_OBJECT{\textgreater})

          \item[size]

          label size ({\textless}int{\textgreater})

            {\it (type=(from xfdata module) FL\_TINY\_SIZE, FL\_SMALL\_SIZE, FL\_NORMAL\_SIZE, 
FL\_MEDIUM\_SIZE, FL\_LARGE\_SIZE, FL\_HUGE\_SIZE, FL\_DEFAULT\_SIZE)}

        \end{Ventry}

      \end{quote}

\textbf{Example:} fl\_set\_object\_lsize(pobj, xfdata.FL\_MEDIUM\_SIZE)



\textbf{Status:} Tested + Doc + Demo = OK



    \end{boxedminipage}

    \label{xformslib:library:fl_get_object_lsize}
    \index{xformslib \textit{(package)}!xformslib.library \textit{(module)}!xformslib.library.fl\_get\_object\_lsize \textit{(function)}}

    \vspace{0.5ex}

\hspace{.8\funcindent}\begin{boxedminipage}{\funcwidth}

    \raggedright \textbf{fl\_get\_object\_lsize}(\textit{pObject})

    \vspace{-1.5ex}

    \rule{\textwidth}{0.5\fboxrule}
\setlength{\parskip}{2ex}
    Returns the label size of an object.

\setlength{\parskip}{1ex}
      \textbf{Parameters}
      \vspace{-1ex}

      \begin{quote}
        \begin{Ventry}{xxxxxxx}

          \item[pObject]

          object to evaluate ({\textless}pointer to 
          xfdata.FL\_OBJECT{\textgreater})

        \end{Ventry}

      \end{quote}

      \textbf{Return Value}
    \vspace{-1ex}

      \begin{quote}
      label size ({\textless}int{\textgreater})

      {\it (type=size)}

      \end{quote}

\textbf{Example:} lsize = fl\_get\_object\_lsize(pobj)



\textbf{Status:} Tested + Doc + Demo = OK



    \end{boxedminipage}

    \label{xformslib:library:fl_set_object_lstyle}
    \index{xformslib \textit{(package)}!xformslib.library \textit{(module)}!xformslib.library.fl\_set\_object\_lstyle \textit{(function)}}

    \vspace{0.5ex}

\hspace{.8\funcindent}\begin{boxedminipage}{\funcwidth}

    \raggedright \textbf{fl\_set\_object\_lstyle}(\textit{pObject}, \textit{style})

    \vspace{-1.5ex}

    \rule{\textwidth}{0.5\fboxrule}
\setlength{\parskip}{2ex}
    Sets the label style of an object.

\setlength{\parskip}{1ex}
      \textbf{Parameters}
      \vspace{-1ex}

      \begin{quote}
        \begin{Ventry}{xxxxxxx}

          \item[pObject]

          object ot be set ({\textless}pointer to 
          xfdata.FL\_OBJECT{\textgreater})

          \item[style]

          label style ({\textless}int{\textgreater})

            {\it (type=(from xfdata module) FL\_NORMAL\_STYLE, FL\_BOLD\_STYLE, FL\_ITALIC\_STYLE,
FL\_BOLDITALIC\_STYLE, FL\_FIXED\_STYLE, FL\_FIXEDBOLD\_STYLE, 
FL\_FIXEDITALIC\_STYLE, FL\_FIXEDBOLDITALIC\_STYLE, FL\_TIMES\_STYLE, 
FL\_TIMESBOLD\_STYLE, FL\_TIMESITALIC\_STYLE, FL\_TIMESBOLDITALIC\_STYLE, 
FL\_MISC\_STYLE, FL\_MISCBOLD\_STYLE, FL\_MISCITALIC\_STYLE, 
FL\_SYMBOL\_STYLE, FL\_SHADOW\_STYLE, FL\_ENGRAVED\_STYLE, 
FL\_EMBOSSED\_STYLE)}

        \end{Ventry}

      \end{quote}

\textbf{Example:} fl\_set\_object\_lstyle(pobj, xfdata.FL\_TIMESITALIC\_STYLE)



\textbf{Status:} Tested + Doc + Demo = OK



    \end{boxedminipage}

    \label{xformslib:library:fl_get_object_lstyle}
    \index{xformslib \textit{(package)}!xformslib.library \textit{(module)}!xformslib.library.fl\_get\_object\_lstyle \textit{(function)}}

    \vspace{0.5ex}

\hspace{.8\funcindent}\begin{boxedminipage}{\funcwidth}

    \raggedright \textbf{fl\_get\_object\_lstyle}(\textit{pObject})

    \vspace{-1.5ex}

    \rule{\textwidth}{0.5\fboxrule}
\setlength{\parskip}{2ex}
    Returns the label style of an object (e.g. xfdata.FL\_BOLD\_STYLE, 
    xfdata.FL\_NORMAL\_STYLE, etc..).

\setlength{\parskip}{1ex}
      \textbf{Parameters}
      \vspace{-1ex}

      \begin{quote}
        \begin{Ventry}{xxxxxxx}

          \item[pObject]

          object to evaluate ({\textless}pointer to 
          xfdata.FL\_OBJECT{\textgreater})

        \end{Ventry}

      \end{quote}

      \textbf{Return Value}
    \vspace{-1ex}

      \begin{quote}
      label style ({\textless}int{\textgreater})

      {\it (type=style)}

      \end{quote}

\textbf{Example:} lstyle = fl\_get\_object\_lstyle(pobj)



\textbf{Status:} Tested + NoDoc + Demo = OK



    \end{boxedminipage}

    \label{xformslib:library:fl_set_object_lcol}
    \index{xformslib \textit{(package)}!xformslib.library \textit{(module)}!xformslib.library.fl\_set\_object\_lcol \textit{(function)}}

    \vspace{0.5ex}

\hspace{.8\funcindent}\begin{boxedminipage}{\funcwidth}

    \raggedright \textbf{fl\_set\_object\_lcol}(\textit{pObject}, \textit{colr})

    \vspace{-1.5ex}

    \rule{\textwidth}{0.5\fboxrule}
\setlength{\parskip}{2ex}
    Sets the label color of an object.

\setlength{\parskip}{1ex}
      \textbf{Parameters}
      \vspace{-1ex}

      \begin{quote}
        \begin{Ventry}{xxxxxxx}

          \item[pObject]

          object to be set ({\textless}pointer to 
          xfdata.FL\_OBJECT{\textgreater})

          \item[colr]

          label color {\textless}long\_pos{\textgreater}

            {\it (type=(from xfdata module) one of defined colors FL\_BLACK, ... FL\_BLUE, ... 
FL\_GREEN, ... FL\_RED, ... etc..)}

        \end{Ventry}

      \end{quote}

\textbf{Example:} fl\_set\_object\_lcol(pobj, xfdata.FL\_BLUE)



\textbf{Status:} Tested + Doc + Demo = OK



    \end{boxedminipage}

    \label{xformslib:library:fl_set_object_lcol}
    \index{xformslib \textit{(package)}!xformslib.library \textit{(module)}!xformslib.library.fl\_set\_object\_lcol \textit{(function)}}

    \vspace{0.5ex}

\hspace{.8\funcindent}\begin{boxedminipage}{\funcwidth}

    \raggedright \textbf{fl\_set\_object\_lcolor}(\textit{pObject}, \textit{colr})

    \vspace{-1.5ex}

    \rule{\textwidth}{0.5\fboxrule}
\setlength{\parskip}{2ex}
    Sets the label color of an object.

\setlength{\parskip}{1ex}
      \textbf{Parameters}
      \vspace{-1ex}

      \begin{quote}
        \begin{Ventry}{xxxxxxx}

          \item[pObject]

          object to be set ({\textless}pointer to 
          xfdata.FL\_OBJECT{\textgreater})

          \item[colr]

          label color {\textless}long\_pos{\textgreater}

            {\it (type=(from xfdata module) one of defined colors FL\_BLACK, ... FL\_BLUE, ... 
FL\_GREEN, ... FL\_RED, ... etc..)}

        \end{Ventry}

      \end{quote}

\textbf{Example:} fl\_set\_object\_lcol(pobj, xfdata.FL\_BLUE)



\textbf{Status:} Tested + Doc + Demo = OK



    \end{boxedminipage}

    \label{xformslib:library:fl_get_object_lcol}
    \index{xformslib \textit{(package)}!xformslib.library \textit{(module)}!xformslib.library.fl\_get\_object\_lcol \textit{(function)}}

    \vspace{0.5ex}

\hspace{.8\funcindent}\begin{boxedminipage}{\funcwidth}

    \raggedright \textbf{fl\_get\_object\_lcol}(\textit{pObject})

    \vspace{-1.5ex}

    \rule{\textwidth}{0.5\fboxrule}
\setlength{\parskip}{2ex}
    Returns the label color of an object (e.g. xfdata.FL\_WHITE, 
    xfdata.FL\_LIME, etc..).

\setlength{\parskip}{1ex}
      \textbf{Parameters}
      \vspace{-1ex}

      \begin{quote}
        \begin{Ventry}{xxxxxxx}

          \item[pObject]

          object ({\textless}pointer to xfdata.FL\_OBJECT{\textgreater})

        \end{Ventry}

      \end{quote}

      \textbf{Return Value}
    \vspace{-1ex}

      \begin{quote}
      color value ({\textless}long\_pos{\textgreater})

      {\it (type=color)}

      \end{quote}

\textbf{Example:} obcolor = fl\_get\_object\_lcol(pobj)



\textbf{Status:} Tested + Doc + NoDemo = OK



    \end{boxedminipage}

    \label{xformslib:library:fl_set_object_return}
    \index{xformslib \textit{(package)}!xformslib.library \textit{(module)}!xformslib.library.fl\_set\_object\_return \textit{(function)}}

    \vspace{0.5ex}

\hspace{.8\funcindent}\begin{boxedminipage}{\funcwidth}

    \raggedright \textbf{fl\_set\_object\_return}(\textit{pObject}, \textit{when})

    \vspace{-1.5ex}

    \rule{\textwidth}{0.5\fboxrule}
\setlength{\parskip}{2ex}
    Sets the conditions under which an object gets returned (or its 
    callback invoked). If the object has to do additional work on setting 
    te condition (e.g. it has child objects that also need to be set) it 
    has to set up it's own function that then will called in the end. This 
    should only be called once an object has been created completely! Not 
    all return types make sense for all objects.

\setlength{\parskip}{1ex}
      \textbf{Parameters}
      \vspace{-1ex}

      \begin{quote}
        \begin{Ventry}{xxxxxxx}

          \item[pObject]

          object ({\textless}pointer to xfdata.FL\_OBJECT{\textgreater})

          \item[when]

          return type (when it returns) ({\textless}int\_pos{\textgreater})

            {\it (type=(from xfdata module) FL\_RETURN\_NONE, FL\_RETURN\_CHANGED, 
FL\_RETURN\_END, FL\_RETURN\_END\_CHANGED, FL\_RETURN\_SELECTION, 
FL\_RETURN\_DESELECTION, FL\_RETURN\_TRIGGERED, FL\_RETURN\_ALWAYS)}

        \end{Ventry}

      \end{quote}

      \textbf{Return Value}
    \vspace{-1ex}

      \begin{quote}
      return type id ({\textless}int{\textgreater})

      {\it (type=ID num)}

      \end{quote}

\textbf{Example:} fl\_set\_object\_return(pobj, xfdata.FL\_RETURN\_CHANGED)



\textbf{Status:} Tested + Doc + Demo = OK



    \end{boxedminipage}

    \label{xformslib:library:fl_notify_object}
    \index{xformslib \textit{(package)}!xformslib.library \textit{(module)}!xformslib.library.fl\_notify\_object \textit{(function)}}

    \vspace{0.5ex}

\hspace{.8\funcindent}\begin{boxedminipage}{\funcwidth}

    \raggedright \textbf{fl\_notify\_object}(\textit{pObject}, \textit{cause})

    \vspace{-1.5ex}

    \rule{\textwidth}{0.5\fboxrule}
\setlength{\parskip}{2ex}
\setlength{\parskip}{1ex}
      \textbf{Parameters}
      \vspace{-1ex}

      \begin{quote}
        \begin{Ventry}{xxxxxxx}

          \item[pObject]

          pointer to object ({\textless}pointer to 
          xfdata.FL\_OBJECT{\textgreater})

          \item[cause]

          cause for notification ({\textless}int{\textgreater})

            {\it (type=(from xfdata module) FL\_ATTRIB, FL\_RESIZED, FL\_MOVEORIGIN)}

        \end{Ventry}

      \end{quote}

\textbf{Example:} fl\_notify\_object(pobj5, xfdata.FL\_RESIZED)



\textbf{Status:} Tested + NoDoc + NoDemo = NOT OK (not clear purpose)



    \end{boxedminipage}

    \label{xformslib:library:fl_set_object_lalign}
    \index{xformslib \textit{(package)}!xformslib.library \textit{(module)}!xformslib.library.fl\_set\_object\_lalign \textit{(function)}}

    \vspace{0.5ex}

\hspace{.8\funcindent}\begin{boxedminipage}{\funcwidth}

    \raggedright \textbf{fl\_set\_object\_lalign}(\textit{pObject}, \textit{align})

    \vspace{-1.5ex}

    \rule{\textwidth}{0.5\fboxrule}
\setlength{\parskip}{2ex}
    Sets alignment of an object's label.

\setlength{\parskip}{1ex}
      \textbf{Parameters}
      \vspace{-1ex}

      \begin{quote}
        \begin{Ventry}{xxxxxxx}

          \item[pObject]

          object to be set ({\textless}pointer to 
          xfdata.FL\_OBJECT{\textgreater})

          \item[align]

          alignment of label ({\textless}int{\textgreater})

            {\it (type=(from xfdata module) FL\_ALIGN\_CENTER, FL\_ALIGN\_TOP, FL\_ALIGN\_BOTTOM, 
FL\_ALIGN\_LEFT, FL\_ALIGN\_RIGHT, FL\_ALIGN\_LEFT\_TOP, 
FL\_ALIGN\_RIGHT\_TOP, FL\_ALIGN\_LEFT\_BOTTOM, FL\_ALIGN\_RIGHT\_BOTTOM, 
FL\_ALIGN\_INSIDE, FL\_ALIGN\_VERT)}

        \end{Ventry}

      \end{quote}

\textbf{Example:} fl\_set\_object\_lalign(pobj8, xfdata.FL\_ALIGN\_RIGHT)



\textbf{Status:} Tested + Doc + Demo = OK



    \end{boxedminipage}

    \label{xformslib:library:fl_get_object_lalign}
    \index{xformslib \textit{(package)}!xformslib.library \textit{(module)}!xformslib.library.fl\_get\_object\_lalign \textit{(function)}}

    \vspace{0.5ex}

\hspace{.8\funcindent}\begin{boxedminipage}{\funcwidth}

    \raggedright \textbf{fl\_get\_object\_lalign}(\textit{pObject})

    \vspace{-1.5ex}

    \rule{\textwidth}{0.5\fboxrule}
\setlength{\parskip}{2ex}
    Returns alignment of an object's label (e.g. xfdata.FL\_ALIGN\_LEFT, 
    xfdata.FL\_ALIGN\_RIGHT\_TOP, etc..).

\setlength{\parskip}{1ex}
      \textbf{Parameters}
      \vspace{-1ex}

      \begin{quote}
        \begin{Ventry}{xxxxxxx}

          \item[pObject]

          object to be set ({\textless}pointer to 
          xfdata.FL\_OBJECT{\textgreater})

        \end{Ventry}

      \end{quote}

      \textbf{Return Value}
    \vspace{-1ex}

      \begin{quote}
      align num

      \end{quote}

\textbf{Example:} obalign = fl\_get\_object\_lalign(pobj8)



\textbf{Status:} Tested + Doc + Demo = OK



    \end{boxedminipage}

    \label{xformslib:library:fl_set_object_lalign}
    \index{xformslib \textit{(package)}!xformslib.library \textit{(module)}!xformslib.library.fl\_set\_object\_lalign \textit{(function)}}

    \vspace{0.5ex}

\hspace{.8\funcindent}\begin{boxedminipage}{\funcwidth}

    \raggedright \textbf{fl\_set\_object\_align}(\textit{pObject}, \textit{align})

    \vspace{-1.5ex}

    \rule{\textwidth}{0.5\fboxrule}
\setlength{\parskip}{2ex}
    Sets alignment of an object's label.

\setlength{\parskip}{1ex}
      \textbf{Parameters}
      \vspace{-1ex}

      \begin{quote}
        \begin{Ventry}{xxxxxxx}

          \item[pObject]

          object to be set ({\textless}pointer to 
          xfdata.FL\_OBJECT{\textgreater})

          \item[align]

          alignment of label ({\textless}int{\textgreater})

            {\it (type=(from xfdata module) FL\_ALIGN\_CENTER, FL\_ALIGN\_TOP, FL\_ALIGN\_BOTTOM, 
FL\_ALIGN\_LEFT, FL\_ALIGN\_RIGHT, FL\_ALIGN\_LEFT\_TOP, 
FL\_ALIGN\_RIGHT\_TOP, FL\_ALIGN\_LEFT\_BOTTOM, FL\_ALIGN\_RIGHT\_BOTTOM, 
FL\_ALIGN\_INSIDE, FL\_ALIGN\_VERT)}

        \end{Ventry}

      \end{quote}

\textbf{Example:} fl\_set\_object\_lalign(pobj8, xfdata.FL\_ALIGN\_RIGHT)



\textbf{Status:} Tested + Doc + Demo = OK



    \end{boxedminipage}

    \label{xformslib:library:fl_set_object_shortcut}
    \index{xformslib \textit{(package)}!xformslib.library \textit{(module)}!xformslib.library.fl\_set\_object\_shortcut \textit{(function)}}

    \vspace{0.5ex}

\hspace{.8\funcindent}\begin{boxedminipage}{\funcwidth}

    \raggedright \textbf{fl\_set\_object\_shortcut}(\textit{pObject}, \textit{shctxt}, \textit{showit})

    \vspace{-1.5ex}

    \rule{\textwidth}{0.5\fboxrule}
\setlength{\parskip}{2ex}
    Sets a shortcut, binding a key or a series of keys to an object. It 
    resets any previous defined shortcuts for the object. Using e.g. 
    "acE\#d{\textasciicircum}h" the keys 'a', 'c', 'E', 
    {\textless}Alt{\textgreater}d and {\textless}Ctrl{\textgreater}h are 
    associated with the object. The precise format is as follows: any 
    character in the string is considered as a shortcut, except 
    '{\textasciicircum}' and '\#', which stand for combinations with the 
    {\textless}Ctrl{\textgreater} and {\textless}Alt{\textgreater} keys. 
    (the case of the key following '\#' or '{\textasciicircum}' is not 
    important, i.e. no distiction is made between e.g. 
    "{\textasciicircum}C" and "{\textasciicircum}c", both encode the key 
    combination {\textless}Ctrl{\textgreater}C as well as 
    {\textless}Ctrl{\textgreater}C.) The key '{\textasciicircum}' itself 
    can be set as a shortcut key by using 
    "{\textasciicircum}{\textasciicircum}" in the string defining the 
    shortcut. The key '\#' can be obtained as a shortcut by using the 
    string "{\textasciicircum}\#". So, e.g. "\#{\textasciicircum}\#" 
    encodes {\textless}ALT{\textgreater}\#. The 
    {\textless}Esc{\textgreater} key can be given as "{\textasciicircum}[".
    Another special character not mentioned yet is '\&', which indicates 
    function and arrow keys. Use a sequence starting with '\&' and directly
    followed by a number between 1 and 35 to represent one of the function 
    keys. For example, "\&2" stands for the {\textless}F2{\textgreater} 
    function key. The four cursors keys (up, down, right, and left) can be 
    given as "\&A", "\&B", "\&C" and "\&D", respectively. The key '\&' 
    itself can be obtained as a shortcut by prefixing it with 
    '{\textasciicircum}'.

\setlength{\parskip}{1ex}
      \textbf{Parameters}
      \vspace{-1ex}

      \begin{quote}
        \begin{Ventry}{xxxxxxx}

          \item[pObject]

          object ({\textless}pointer to xfdata.FL\_OBJECT{\textgreater})

          \item[shctxt]

          shortcut text to be set ({\textless}string{\textgreater})

          \item[showit]

          flag if shortcut letter has to be underlined or not if a match 
          exists (only the 1st alphanumeric character is used.

            {\it (type=0 (underline not shown) or 1 (shown))}

        \end{Ventry}

      \end{quote}

\textbf{Example:} fl\_set\_object\_shortcut(pobj6, "aA\#A{\textasciicircum}A", 1)



\textbf{Status:} Tested + Doc + NoDemo = OK



    \end{boxedminipage}

    \label{xformslib:library:fl_set_object_shortcutkey}
    \index{xformslib \textit{(package)}!xformslib.library \textit{(module)}!xformslib.library.fl\_set\_object\_shortcutkey \textit{(function)}}

    \vspace{0.5ex}

\hspace{.8\funcindent}\begin{boxedminipage}{\funcwidth}

    \raggedright \textbf{fl\_set\_object\_shortcutkey}(\textit{pObject}, \textit{keysym})

    \vspace{-1.5ex}

    \rule{\textwidth}{0.5\fboxrule}
\setlength{\parskip}{2ex}
    Uses a special key as a shortcut. It always appends the specified key 
    to the current shortcuts. Special keys can't be underlined.

\setlength{\parskip}{1ex}
      \textbf{Parameters}
      \vspace{-1ex}

      \begin{quote}
        \begin{Ventry}{xxxxxxx}

          \item[pObject]

          object ({\textless}pointer to xfdata.FL\_OBJECT{\textgreater})

          \item[keysym]

          X key symbolic num. ({\textless}int\_pos{\textgreater})

            {\it (type=see xfdata module for a (maybe) incomplete list)}

        \end{Ventry}

      \end{quote}

\textbf{Example:} fl\_set\_object\_shortcutkey(pobj, xfdata.XK\_Home)



\textbf{Status:} Tested + Doc + NoDemo = OK



    \end{boxedminipage}

    \label{xformslib:library:fl_set_object_dblbuffer}
    \index{xformslib \textit{(package)}!xformslib.library \textit{(module)}!xformslib.library.fl\_set\_object\_dblbuffer \textit{(function)}}

    \vspace{0.5ex}

\hspace{.8\funcindent}\begin{boxedminipage}{\funcwidth}

    \raggedright \textbf{fl\_set\_object\_dblbuffer}(\textit{pObject}, \textit{flag})

    \vspace{-1.5ex}

    \rule{\textwidth}{0.5\fboxrule}
\setlength{\parskip}{2ex}
    Uses double buffering on a per-object basis. Currently double buffering
    for objects having a non-rectangular box might not work well. A 
    nonrectangular box means that there are regions within the bounding box
    that should not be painted, which is not easily done without complex 
    and expensive clipping and unacceptable inefficiency. Since Xlib 
    doesn't support double buffering, XForms library simulates this 
    functionality with pixmap bit-bliting. In practice, the effect is 
    hardly distinguishable from double buffering and performance is on par 
    with multi-buffering extensions (It is slower than drawing into a 
    window directly on most workstations however). Bear in mind that a 
    pixmap can be resource hungry, so use this option with discretion.

\setlength{\parskip}{1ex}
      \textbf{Parameters}
      \vspace{-1ex}

      \begin{quote}
        \begin{Ventry}{xxxxxxx}

          \item[pObject]

          object ({\textless}pointer to xfdata.FL\_OBJECT{\textgreater})

          \item[flag]

          flag to disable/enable double buffer 
          ({\textless}int{\textgreater})

            {\it (type=0 (disabled) or 1 (enabled))}

        \end{Ventry}

      \end{quote}

\textbf{Example:} fl\_set\_object\_dblbuffer(pobj7, 1)



\textbf{Status:} Tested + Doc + Demo = OK



    \end{boxedminipage}

    \label{xformslib:library:fl_set_object_color}
    \index{xformslib \textit{(package)}!xformslib.library \textit{(module)}!xformslib.library.fl\_set\_object\_color \textit{(function)}}

    \vspace{0.5ex}

\hspace{.8\funcindent}\begin{boxedminipage}{\funcwidth}

    \raggedright \textbf{fl\_set\_object\_color}(\textit{pObject}, \textit{fgcolr}, \textit{bgcolr})

    \vspace{-1.5ex}

    \rule{\textwidth}{0.5\fboxrule}
\setlength{\parskip}{2ex}
    Sets the color of an object.

\setlength{\parskip}{1ex}
      \textbf{Parameters}
      \vspace{-1ex}

      \begin{quote}
        \begin{Ventry}{xxxxxxx}

          \item[pObject]

          object ({\textless}pointer to xfdata.FL\_OBJECT{\textgreater})

          \item[fgcolr]

          foreground color value ({\textless}long\_pos{\textgreater})

          \item[bgcolr]

          background color value ({\textless}long\_pos{\textgreater})

        \end{Ventry}

      \end{quote}

\textbf{Example:} fl\_set\_object\_color(pbutob7, xfdata.FL\_AQUA, xfdata.FL\_WHEAT)



\textbf{Status:} Tested + Doc + NoDemo = OK



    \end{boxedminipage}

    \label{xformslib:library:fl_get_object_color}
    \index{xformslib \textit{(package)}!xformslib.library \textit{(module)}!xformslib.library.fl\_get\_object\_color \textit{(function)}}

    \vspace{0.5ex}

\hspace{.8\funcindent}\begin{boxedminipage}{\funcwidth}

    \raggedright \textbf{fl\_get\_object\_color}(\textit{pObject})

    \vspace{-1.5ex}

    \rule{\textwidth}{0.5\fboxrule}
\setlength{\parskip}{2ex}
    Returns the foreground and background colors of an object.

\setlength{\parskip}{1ex}
      \textbf{Parameters}
      \vspace{-1ex}

      \begin{quote}
        \begin{Ventry}{xxxxxxx}

          \item[pObject]

          object ({\textless}pointer to xfdata.FL\_OBJECT{\textgreater})

        \end{Ventry}

      \end{quote}

      \textbf{Return Value}
    \vspace{-1ex}

      \begin{quote}
      foreground and background color values 
      ({\textless}long\_pos{\textgreater}, 
      {\textless}long\_pos{\textgreater})

      {\it (type=fgcolr, bgcolr)}

      \end{quote}

\textbf{Example:} primcol, secncol = fl\_get\_object\_color(pobj)



\textbf{Attention:} API change from XForms - upstream was fl\_set\_object\_color(pObject, 
fgcolr, bgcolr)



\textbf{Status:} Tested + Doc + Demo = OK



    \end{boxedminipage}

    \label{xformslib:library:fl_set_object_label}
    \index{xformslib \textit{(package)}!xformslib.library \textit{(module)}!xformslib.library.fl\_set\_object\_label \textit{(function)}}

    \vspace{0.5ex}

\hspace{.8\funcindent}\begin{boxedminipage}{\funcwidth}

    \raggedright \textbf{fl\_set\_object\_label}(\textit{pObject}, \textit{label})

    \vspace{-1.5ex}

    \rule{\textwidth}{0.5\fboxrule}
\setlength{\parskip}{2ex}
    Sets the label of an object.

\setlength{\parskip}{1ex}
      \textbf{Parameters}
      \vspace{-1ex}

      \begin{quote}
        \begin{Ventry}{xxxxxxx}

          \item[pObject]

          object ({\textless}pointer to xfdata.FL\_OBJECT{\textgreater})

          \item[label]

          text label of object ({\textless}string{\textgreater})

        \end{Ventry}

      \end{quote}

\textbf{Example:} fl\_set\_object\_label(pobj, "My New Label")



\textbf{Status:} Tested + Doc + NoDemo = OK



    \end{boxedminipage}

    \label{xformslib:library:fl_get_object_label}
    \index{xformslib \textit{(package)}!xformslib.library \textit{(module)}!xformslib.library.fl\_get\_object\_label \textit{(function)}}

    \vspace{0.5ex}

\hspace{.8\funcindent}\begin{boxedminipage}{\funcwidth}

    \raggedright \textbf{fl\_get\_object\_label}(\textit{pObject})

    \vspace{-1.5ex}

    \rule{\textwidth}{0.5\fboxrule}
\setlength{\parskip}{2ex}
    Returns the label of an object.

\setlength{\parskip}{1ex}
      \textbf{Parameters}
      \vspace{-1ex}

      \begin{quote}
        \begin{Ventry}{xxxxxxx}

          \item[pObject]

          object ({\textless}pointer to xfdata.FL\_OBJECT{\textgreater})

        \end{Ventry}

      \end{quote}

      \textbf{Return Value}
    \vspace{-1ex}

      \begin{quote}
      text of label ({\textless}string{\textgreater})

      {\it (type=label)}

      \end{quote}

\textbf{Example:} currlbl = fl\_get\_object\_label(pobj)



\textbf{Status:} Tested + Doc + Demo = OK



    \end{boxedminipage}

    \label{xformslib:library:fl_set_object_helper}
    \index{xformslib \textit{(package)}!xformslib.library \textit{(module)}!xformslib.library.fl\_set\_object\_helper \textit{(function)}}

    \vspace{0.5ex}

\hspace{.8\funcindent}\begin{boxedminipage}{\funcwidth}

    \raggedright \textbf{fl\_set\_object\_helper}(\textit{pObject}, \textit{tip})

    \vspace{-1.5ex}

    \rule{\textwidth}{0.5\fboxrule}
\setlength{\parskip}{2ex}
    Sets the tooltip of an object (with possible embedded newlines in it) 
    that will be shown when the mouse hovers over the object for more than 
    about 600 msec.

\setlength{\parskip}{1ex}
      \textbf{Parameters}
      \vspace{-1ex}

      \begin{quote}
        \begin{Ventry}{xxxxxxx}

          \item[pObject]

          object ({\textless}pointer to xfdata.FL\_OBJECT{\textgreater})

          \item[tip]

          tooltip text for object ({\textless}string{\textgreater})

        \end{Ventry}

      \end{quote}

\textbf{Example:} fl\_set\_object\_helper(pobj, "Button to exit the procedure.")



\textbf{Status:} Tested + Doc + Demo = OK



    \end{boxedminipage}

    \label{xformslib:library:fl_set_object_position}
    \index{xformslib \textit{(package)}!xformslib.library \textit{(module)}!xformslib.library.fl\_set\_object\_position \textit{(function)}}

    \vspace{0.5ex}

\hspace{.8\funcindent}\begin{boxedminipage}{\funcwidth}

    \raggedright \textbf{fl\_set\_object\_position}(\textit{pObject}, \textit{x}, \textit{y})

    \vspace{-1.5ex}

    \rule{\textwidth}{0.5\fboxrule}
\setlength{\parskip}{2ex}
    Sets position of an object.

\setlength{\parskip}{1ex}
      \textbf{Parameters}
      \vspace{-1ex}

      \begin{quote}
        \begin{Ventry}{xxxxxxx}

          \item[pObject]

          object ({\textless}pointer to xfdata.FL\_OBJECT{\textgreater})

          \item[x]

          horizontal position (upper-left corner) 
          ({\textless}int{\textgreater}))

          \item[y]

          vertical position (upper-left corner) 
          ({\textless}int{\textgreater})

        \end{Ventry}

      \end{quote}

\textbf{Example:} fl\_set\_object\_position(pobj, 235, 123)



\textbf{Status:} Tested + Doc + Demo = OK



    \end{boxedminipage}

    \label{xformslib:library:fl_get_object_size}
    \index{xformslib \textit{(package)}!xformslib.library \textit{(module)}!xformslib.library.fl\_get\_object\_size \textit{(function)}}

    \vspace{0.5ex}

\hspace{.8\funcindent}\begin{boxedminipage}{\funcwidth}

    \raggedright \textbf{fl\_get\_object\_size}(\textit{pObject})

    \vspace{-1.5ex}

    \rule{\textwidth}{0.5\fboxrule}
\setlength{\parskip}{2ex}
    Returns the size of an object.

\setlength{\parskip}{1ex}
      \textbf{Parameters}
      \vspace{-1ex}

      \begin{quote}
        \begin{Ventry}{xxxxxxx}

          \item[pObject]

          object to evaluate ({\textless}pointer to 
          xfdata.FL\_OBJECT{\textgreater})

        \end{Ventry}

      \end{quote}

      \textbf{Return Value}
    \vspace{-1ex}

      \begin{quote}
      width and height in coord units ({\textless}int{\textgreater}, 
      {\textless}int{\textgreater})

      {\it (type=width, height)}

      \end{quote}

\textbf{Example:} wid, hei = fl\_get\_object\_size(pobj)



\textbf{Attention:} API change from XForms - upstream was fl\_get\_object\_size(pObject, w, h)



\textbf{Status:} Tested + Doc + NoDemo = OK



    \end{boxedminipage}

    \label{xformslib:library:fl_set_object_size}
    \index{xformslib \textit{(package)}!xformslib.library \textit{(module)}!xformslib.library.fl\_set\_object\_size \textit{(function)}}

    \vspace{0.5ex}

\hspace{.8\funcindent}\begin{boxedminipage}{\funcwidth}

    \raggedright \textbf{fl\_set\_object\_size}(\textit{pObject}, \textit{w}, \textit{h})

    \vspace{-1.5ex}

    \rule{\textwidth}{0.5\fboxrule}
\setlength{\parskip}{2ex}
    Sets the size of an object.

\setlength{\parskip}{1ex}
      \textbf{Parameters}
      \vspace{-1ex}

      \begin{quote}
        \begin{Ventry}{xxxxxxx}

          \item[pObject]

          object ({\textless}pointer to xfdata.FL\_OBJECT{\textgreater})

          \item[w]

          width of object in coord units ({\textless}int{\textgreater})

          \item[h]

          height of object in coord units ({\textless}int{\textgreater})

        \end{Ventry}

      \end{quote}

\textbf{Example:} fl\_set\_object\_size(pobj, 90, 35)



\textbf{Status:} Tested + Doc + NoDemo = OK



    \end{boxedminipage}

    \label{xformslib:library:fl_set_object_automatic}
    \index{xformslib \textit{(package)}!xformslib.library \textit{(module)}!xformslib.library.fl\_set\_object\_automatic \textit{(function)}}

    \vspace{0.5ex}

\hspace{.8\funcindent}\begin{boxedminipage}{\funcwidth}

    \raggedright \textbf{fl\_set\_object\_automatic}(\textit{pObject}, \textit{flag})

    \vspace{-1.5ex}

    \rule{\textwidth}{0.5\fboxrule}
\setlength{\parskip}{2ex}
    Enables or disables an object to receive a xfdata.FL\_STEP event. This 
    should not be used with built-in objects. An object is automatic if it 
    automatically (without user actions) has to change its contents. 
    Automatic objects get a FL\_STEP event about every 50 msec.

\setlength{\parskip}{1ex}
      \textbf{Parameters}
      \vspace{-1ex}

      \begin{quote}
        \begin{Ventry}{xxxxxxx}

          \item[pObject]

          object ({\textless}pointer to xfdata.FL\_OBJECT{\textgreater})

          \item[flag]

          flag if automatic or not ({\textless}int{\textgreater})

            {\it (type=0 (not automatic) or 1 (automatic))}

        \end{Ventry}

      \end{quote}

\textbf{Status:} Tested + Doc + NoDemo = OK



    \end{boxedminipage}

    \label{xformslib:library:fl_object_is_automatic}
    \index{xformslib \textit{(package)}!xformslib.library \textit{(module)}!xformslib.library.fl\_object\_is\_automatic \textit{(function)}}

    \vspace{0.5ex}

\hspace{.8\funcindent}\begin{boxedminipage}{\funcwidth}

    \raggedright \textbf{fl\_object\_is\_automatic}(\textit{pObject})

    \vspace{-1.5ex}

    \rule{\textwidth}{0.5\fboxrule}
\setlength{\parskip}{2ex}
    Returns if an object receives xfdata.FL\_STEP events. An object is 
    automatic if it automatically (without user actions) has to change its 
    contents. Automatic objects get a FL\_STEP event about every 50 msec.

\setlength{\parskip}{1ex}
      \textbf{Parameters}
      \vspace{-1ex}

      \begin{quote}
        \begin{Ventry}{xxxxxxx}

          \item[pObject]

          object to evaluate ({\textless}pointer to 
          xfdata.FL\_OBJECT{\textgreater})

        \end{Ventry}

      \end{quote}

      \textbf{Return Value}
    \vspace{-1ex}

      \begin{quote}
      flag if it's automatic (1) or not (0) ({\textless}int{\textgreater})

      {\it (type=flag num)}

      \end{quote}

\textbf{Example:} if xf.fl\_object\_is\_automatic(pobj): ...



\textbf{Status:} Tested + Doc + NoDemo = OK



    \end{boxedminipage}

    \label{xformslib:library:fl_draw_object_label}
    \index{xformslib \textit{(package)}!xformslib.library \textit{(module)}!xformslib.library.fl\_draw\_object\_label \textit{(function)}}

    \vspace{0.5ex}

\hspace{.8\funcindent}\begin{boxedminipage}{\funcwidth}

    \raggedright \textbf{fl\_draw\_object\_label}(\textit{pObject})

    \vspace{-1.5ex}

    \rule{\textwidth}{0.5\fboxrule}
\setlength{\parskip}{2ex}
    Draws the label according to the alignment, which could be inside or 
    outside of the bounding box.

\setlength{\parskip}{1ex}
      \textbf{Parameters}
      \vspace{-1ex}

      \begin{quote}
        \begin{Ventry}{xxxxxxx}

          \item[pObject]

          object ({\textless}pointer to xfdata.FL\_OBJECT{\textgreater})

        \end{Ventry}

      \end{quote}

\textbf{Example:} fl\_draw\_object\_label(pobj3)



\textbf{Status:} Tested + Doc + NoDemo = OK



    \end{boxedminipage}

    \label{xformslib:library:fl_draw_object_label_outside}
    \index{xformslib \textit{(package)}!xformslib.library \textit{(module)}!xformslib.library.fl\_draw\_object\_label\_outside \textit{(function)}}

    \vspace{0.5ex}

\hspace{.8\funcindent}\begin{boxedminipage}{\funcwidth}

    \raggedright \textbf{fl\_draw\_object\_label\_outside}(\textit{pObject})

    \vspace{-1.5ex}

    \rule{\textwidth}{0.5\fboxrule}
\setlength{\parskip}{2ex}
    Draws the label outside of the bounding box.

\setlength{\parskip}{1ex}
      \textbf{Parameters}
      \vspace{-1ex}

      \begin{quote}
        \begin{Ventry}{xxxxxxx}

          \item[pObject]

          object ({\textless}pointer to xfdata.FL\_OBJECT{\textgreater})

        \end{Ventry}

      \end{quote}

\textbf{Example:} fl\_draw\_object\_label\_outside(pobj3)



\textbf{Status:} Tested + Doc + NoDemo = OK



    \end{boxedminipage}

    \label{xformslib:library:fl_draw_object_label_outside}
    \index{xformslib \textit{(package)}!xformslib.library \textit{(module)}!xformslib.library.fl\_draw\_object\_label\_outside \textit{(function)}}

    \vspace{0.5ex}

\hspace{.8\funcindent}\begin{boxedminipage}{\funcwidth}

    \raggedright \textbf{fl\_draw\_object\_outside\_label}(\textit{pObject})

    \vspace{-1.5ex}

    \rule{\textwidth}{0.5\fboxrule}
\setlength{\parskip}{2ex}
    Draws the label outside of the bounding box.

\setlength{\parskip}{1ex}
      \textbf{Parameters}
      \vspace{-1ex}

      \begin{quote}
        \begin{Ventry}{xxxxxxx}

          \item[pObject]

          object ({\textless}pointer to xfdata.FL\_OBJECT{\textgreater})

        \end{Ventry}

      \end{quote}

\textbf{Example:} fl\_draw\_object\_label\_outside(pobj3)



\textbf{Status:} Tested + Doc + NoDemo = OK



    \end{boxedminipage}

    \label{xformslib:library:fl_get_object_component}
    \index{xformslib \textit{(package)}!xformslib.library \textit{(module)}!xformslib.library.fl\_get\_object\_component \textit{(function)}}

    \vspace{0.5ex}

\hspace{.8\funcindent}\begin{boxedminipage}{\funcwidth}

    \raggedright \textbf{fl\_get\_object\_component}(\textit{pObject}, \textit{objclass}, \textit{compontype}, \textit{num})

    \vspace{-1.5ex}

    \rule{\textwidth}{0.5\fboxrule}
\setlength{\parskip}{2ex}
    Returns the object that is a component of a composite object. E.g. the 
    scrollbar object is made of a slider and two scroll buttons.

\setlength{\parskip}{1ex}
      \textbf{Parameters}
      \vspace{-1ex}

      \begin{quote}
        \begin{Ventry}{xxxxxxxxxx}

          \item[pObject]

          composite object ({\textless}pointer to 
          xfdata.FL\_OBJECT{\textgreater})

          \item[objclass]

          component object's class id ({\textless}int{\textgreater})

          \item[compontype]

          component object's type id ({\textless}int{\textgreater})

          \item[num]

          the sequence number of the desired object in case the composite 
          has more than one object of the same class and type. You cans use
          -1 to indicate any type of specified class 
          ({\textless}int{\textgreater})

        \end{Ventry}

      \end{quote}

      \textbf{Return Value}
    \vspace{-1ex}

      \begin{quote}
      component object found ({\textless}pointer to 
      xfdata.FL\_OBJECT{\textgreater}) or None (no object found)

      {\it (type=pObject)}

      \end{quote}

\textbf{Example:} fl\_get\_object\_component(browserobj, xfdata.FL\_SCROLLBAR, 
xfdata.FL\_HOR\_THIN\_SCROLLBAR, 0)



\textbf{Status:} Untested + Doc + NoDemo = NOT OK



    \end{boxedminipage}

    \label{xformslib:library:fl_for_all_objects}
    \index{xformslib \textit{(package)}!xformslib.library \textit{(module)}!xformslib.library.fl\_for\_all\_objects \textit{(function)}}

    \vspace{0.5ex}

\hspace{.8\funcindent}\begin{boxedminipage}{\funcwidth}

    \raggedright \textbf{fl\_for\_all\_objects}(\textit{pForm}, \textit{py\_operatecb}, \textit{vdata})

    \vspace{-1.5ex}

    \rule{\textwidth}{0.5\fboxrule}
\setlength{\parskip}{2ex}
    Serves as an iterator to change an attribute for all objects on a 
    particular form. Specified operating function is called for every 
    object of the form form unless it returns nonzero, which terminates the
    iterator.

\setlength{\parskip}{1ex}
      \textbf{Parameters}
      \vspace{-1ex}

      \begin{quote}
        \begin{Ventry}{xxxxxxxxxxxx}

          \item[pForm]

          form ({\textless}pointer to xfdata.FL\_FORM{\textgreater})

          \item[py\_operatecb]

          python callback function, returning value

            {\it (type=\_\_ funcname (pObject, ptr\_void) -{\textgreater} num \_\_)}

          \item[vdata]

          user data to be passed ({\textless}pointer to void{\textgreater})

        \end{Ventry}

      \end{quote}

\textbf{Example:}
\begin{quote}
  \begin{itemize}

  \item
    \setlength{\parskip}{0.6ex}
def operatecb(pobj, vdata):



  \item {\textbar}-{\textgreater}{\textbar} ...



  \item {\textbar}-{\textgreater}{\textbar} return 0;



  \item fl\_for\_all\_objects(pform5, operatecb, None)



\end{itemize}

\end{quote}

\textbf{Status:} Untested + Doc + NoDemo = NOT OK



    \end{boxedminipage}

    \label{xformslib:library:fl_set_object_dblclick}
    \index{xformslib \textit{(package)}!xformslib.library \textit{(module)}!xformslib.library.fl\_set\_object\_dblclick \textit{(function)}}

    \vspace{0.5ex}

\hspace{.8\funcindent}\begin{boxedminipage}{\funcwidth}

    \raggedright \textbf{fl\_set\_object\_dblclick}(\textit{pObject}, \textit{timeout})

    \vspace{-1.5ex}

    \rule{\textwidth}{0.5\fboxrule}
\setlength{\parskip}{2ex}
    Sets double click timeout value of an object, enabling or disabling it 
    to receive the xfdata.FL\_DBLCLICK event.

\setlength{\parskip}{1ex}
      \textbf{Parameters}
      \vspace{-1ex}

      \begin{quote}
        \begin{Ventry}{xxxxxxx}

          \item[pObject]

          object ({\textless}pointer to xfdata.FL\_OBJECT{\textgreater})

          \item[timeout]

          maximum time interval (in msec) between two clicks for them to be
          considered a double-click (using 0 disables double-click 
          detection) ({\textless}long\_pos{\textgreater})

        \end{Ventry}

      \end{quote}

\textbf{Example:} fl\_set\_object\_dblclick(pobj, 750)



\textbf{Status:} Tested + Doc + NoDemo = OK



    \end{boxedminipage}

    \label{xformslib:library:fl_get_object_dblclick}
    \index{xformslib \textit{(package)}!xformslib.library \textit{(module)}!xformslib.library.fl\_get\_object\_dblclick \textit{(function)}}

    \vspace{0.5ex}

\hspace{.8\funcindent}\begin{boxedminipage}{\funcwidth}

    \raggedright \textbf{fl\_get\_object\_dblclick}(\textit{pObject})

    \vspace{-1.5ex}

    \rule{\textwidth}{0.5\fboxrule}
\setlength{\parskip}{2ex}
    Return double click timeout value of an object.

\setlength{\parskip}{1ex}
      \textbf{Parameters}
      \vspace{-1ex}

      \begin{quote}
        \begin{Ventry}{xxxxxxx}

          \item[pObject]

          object to evaluate ({\textless}pointer to 
          xfdata.FL\_OBJECT{\textgreater})

        \end{Ventry}

      \end{quote}

      \textbf{Return Value}
    \vspace{-1ex}

      \begin{quote}
      timeout value ({\textless}long\_pos{\textgreater})

      {\it (type=timeout)}

      \end{quote}

\textbf{Example:} dctim = fl\_get\_object\_dblclick(pobj0)



\textbf{Status:} Tested + Doc + NoDemo = OK



    \end{boxedminipage}

    \label{xformslib:library:fl_set_object_geometry}
    \index{xformslib \textit{(package)}!xformslib.library \textit{(module)}!xformslib.library.fl\_set\_object\_geometry \textit{(function)}}

    \vspace{0.5ex}

\hspace{.8\funcindent}\begin{boxedminipage}{\funcwidth}

    \raggedright \textbf{fl\_set\_object\_geometry}(\textit{pObject}, \textit{x}, \textit{y}, \textit{w}, \textit{h})

    \vspace{-1.5ex}

    \rule{\textwidth}{0.5\fboxrule}
\setlength{\parskip}{2ex}
    Sets the geometry (position and size) of an object.

\setlength{\parskip}{1ex}
      \textbf{Parameters}
      \vspace{-1ex}

      \begin{quote}
        \begin{Ventry}{xxxxxxx}

          \item[pObject]

          object to modify ({\textless}pointer to 
          xfdata.FL\_OBJECT{\textgreater})

          \item[x]

          horizontal position (upper-left corner) 
          ({\textless}int{\textgreater})

          \item[y]

          vertical position (upper-left corner) 
          ({\textless}int{\textgreater})

          \item[w]

          width in coord units ({\textless}int{\textgreater})

          \item[h]

          height in coord units ({\textless}int{\textgreater})

        \end{Ventry}

      \end{quote}

\textbf{Example:} fl\_set\_object\_geometry(pobj, 200, 250, 120, 25)



\textbf{Status:} Tested + Doc + Demo = OK



    \end{boxedminipage}

    \label{xformslib:library:fl_move_object}
    \index{xformslib \textit{(package)}!xformslib.library \textit{(module)}!xformslib.library.fl\_move\_object \textit{(function)}}

    \vspace{0.5ex}

\hspace{.8\funcindent}\begin{boxedminipage}{\funcwidth}

    \raggedright \textbf{fl\_move\_object}(\textit{pObject}, \textit{x}, \textit{y})

    \vspace{-1.5ex}

    \rule{\textwidth}{0.5\fboxrule}
\setlength{\parskip}{2ex}
    Moves an object to a new position.

\setlength{\parskip}{1ex}
      \textbf{Parameters}
      \vspace{-1ex}

      \begin{quote}
        \begin{Ventry}{xxxxxxx}

          \item[pObject]

          object to be moved ({\textless}pointer to 
          xfdata.FL\_OBJECT{\textgreater})

          \item[x]

          new horizontal position (upper-left corner) 
          ({\textless}int{\textgreater})

          \item[y]

          new vertical position (upper-left corner) 
          ({\textless}int{\textgreater})

        \end{Ventry}

      \end{quote}

\textbf{Example:} fl\_move\_object(pobj0, 120, 380)



\textbf{Status:} Tested + Doc + NoDemo = OK



    \end{boxedminipage}

    \label{xformslib:library:fl_fit_object_label}
    \index{xformslib \textit{(package)}!xformslib.library \textit{(module)}!xformslib.library.fl\_fit\_object\_label \textit{(function)}}

    \vspace{0.5ex}

\hspace{.8\funcindent}\begin{boxedminipage}{\funcwidth}

    \raggedright \textbf{fl\_fit\_object\_label}(\textit{pObject}, \textit{xmargin}, \textit{ymargin})

    \vspace{-1.5ex}

    \rule{\textwidth}{0.5\fboxrule}
\setlength{\parskip}{2ex}
    Checks if the label of an object fits into it (after x- and y-margin 
    have been added). If not, all objects and the form are enlarged by the 
    necessary factor (but never by more than a factor of 1.5).

\setlength{\parskip}{1ex}
      \textbf{Parameters}
      \vspace{-1ex}

      \begin{quote}
        \begin{Ventry}{xxxxxxx}

          \item[pObject]

          object ({\textless}pointer to xfdata.FL\_OBJECT{\textgreater})

          \item[xmargin]

          horizontal margin of label in coord units 
          ({\textless}int{\textgreater})

          \item[ymargin]

          vertical margin of label in coord units 
          ({\textless}int{\textgreater})

        \end{Ventry}

      \end{quote}

\textbf{Status:} Tested + Doc + NoDemo = OK



    \end{boxedminipage}

    \label{xformslib:library:fl_get_object_geometry}
    \index{xformslib \textit{(package)}!xformslib.library \textit{(module)}!xformslib.library.fl\_get\_object\_geometry \textit{(function)}}

    \vspace{0.5ex}

\hspace{.8\funcindent}\begin{boxedminipage}{\funcwidth}

    \raggedright \textbf{fl\_get\_object\_geometry}(\textit{pObject})

    \vspace{-1.5ex}

    \rule{\textwidth}{0.5\fboxrule}
\setlength{\parskip}{2ex}
    Returns the geometry (position and size) of an object.

\setlength{\parskip}{1ex}
      \textbf{Parameters}
      \vspace{-1ex}

      \begin{quote}
        \begin{Ventry}{xxxxxxx}

          \item[pObject]

          object ({\textless}pointer to xfdata.FL\_OBJECT{\textgreater})

        \end{Ventry}

      \end{quote}

      \textbf{Return Value}
    \vspace{-1ex}

      \begin{quote}
      horizontal, vertical position, width, height 
      ({\textless}int{\textgreater}, {\textless}int{\textgreater}, 
      {\textless}int{\textgreater}, {\textless}int{\textgreater})

      {\it (type=x, y, w, h)}

      \end{quote}

\textbf{Example:} xpos, ypos, wid, hei = fl\_get\_object\_geometry(pobj1)



\textbf{Attention:} API change from XForms - upstream was fl\_get\_object\_geometry(pObject, x,
y, w, h)



\textbf{Status:} Tested + Doc + Demo = OK



    \end{boxedminipage}

    \label{xformslib:library:fl_get_object_position}
    \index{xformslib \textit{(package)}!xformslib.library \textit{(module)}!xformslib.library.fl\_get\_object\_position \textit{(function)}}

    \vspace{0.5ex}

\hspace{.8\funcindent}\begin{boxedminipage}{\funcwidth}

    \raggedright \textbf{fl\_get\_object\_position}(\textit{pObject})

    \vspace{-1.5ex}

    \rule{\textwidth}{0.5\fboxrule}
\setlength{\parskip}{2ex}
    Returns the position of an object.

\setlength{\parskip}{1ex}
      \textbf{Parameters}
      \vspace{-1ex}

      \begin{quote}
        \begin{Ventry}{xxxxxxx}

          \item[pObject]

          object to evaluate ({\textless}pointer to 
          xfdata.FL\_OBJECT{\textgreater})

        \end{Ventry}

      \end{quote}

      \textbf{Return Value}
    \vspace{-1ex}

      \begin{quote}
      horizontal and vertical position ({\textless}int{\textgreater}, 
      {\textless}int{\textgreater})

      {\it (type=x, y)}

      \end{quote}

\textbf{Example:} xpos, ypos = fl\_get\_object\_position(pobj2)



\textbf{Attention:} API change from XForms - upstream was fl\_get\_object\_position(pObject, x,
y)



\textbf{Status:} Tested + Doc + NoDemo = OK



    \end{boxedminipage}

    \label{xformslib:library:fl_get_object_bbox}
    \index{xformslib \textit{(package)}!xformslib.library \textit{(module)}!xformslib.library.fl\_get\_object\_bbox \textit{(function)}}

    \vspace{0.5ex}

\hspace{.8\funcindent}\begin{boxedminipage}{\funcwidth}

    \raggedright \textbf{fl\_get\_object\_bbox}(\textit{pObject})

    \vspace{-1.5ex}

    \rule{\textwidth}{0.5\fboxrule}
\setlength{\parskip}{2ex}
    Returns the bounding box size that has the label, which could be drawn 
    outside of the object figured in.

\setlength{\parskip}{1ex}
      \textbf{Parameters}
      \vspace{-1ex}

      \begin{quote}
        \begin{Ventry}{xxxxxxx}

          \item[pObject]

          object to evaluate ({\textless}pointer to 
          xfdata.FL\_OBJECT{\textgreater})

        \end{Ventry}

      \end{quote}

      \textbf{Return Value}
    \vspace{-1ex}

      \begin{quote}
      horizontal, vertical position, width, height 
      ({\textless}int{\textgreater}, {\textless}int{\textgreater}, 
      {\textless}int{\textgreater}, {\textless}int{\textgreater})

      {\it (type=x, y, w, h)}

      \end{quote}

\textbf{Example:} xpos, ypos, wid, hei = fl\_get\_object\_bbox(pobj4)



\textbf{Attention:} API change from XForms - upstream was fl\_get\_object\_bbox(pObject, x, y, 
w, h)



\textbf{Status:} Tested + Doc + NoDemo = OK



    \end{boxedminipage}

    \label{xformslib:library:fl_get_object_bbox}
    \index{xformslib \textit{(package)}!xformslib.library \textit{(module)}!xformslib.library.fl\_get\_object\_bbox \textit{(function)}}

    \vspace{0.5ex}

\hspace{.8\funcindent}\begin{boxedminipage}{\funcwidth}

    \raggedright \textbf{fl\_compute\_object\_geometry}(\textit{pObject})

    \vspace{-1.5ex}

    \rule{\textwidth}{0.5\fboxrule}
\setlength{\parskip}{2ex}
    Returns the bounding box size that has the label, which could be drawn 
    outside of the object figured in.

\setlength{\parskip}{1ex}
      \textbf{Parameters}
      \vspace{-1ex}

      \begin{quote}
        \begin{Ventry}{xxxxxxx}

          \item[pObject]

          object to evaluate ({\textless}pointer to 
          xfdata.FL\_OBJECT{\textgreater})

        \end{Ventry}

      \end{quote}

      \textbf{Return Value}
    \vspace{-1ex}

      \begin{quote}
      horizontal, vertical position, width, height 
      ({\textless}int{\textgreater}, {\textless}int{\textgreater}, 
      {\textless}int{\textgreater}, {\textless}int{\textgreater})

      {\it (type=x, y, w, h)}

      \end{quote}

\textbf{Example:} xpos, ypos, wid, hei = fl\_get\_object\_bbox(pobj4)



\textbf{Attention:} API change from XForms - upstream was fl\_get\_object\_bbox(pObject, x, y, 
w, h)



\textbf{Status:} Tested + Doc + NoDemo = OK



    \end{boxedminipage}

    \label{xformslib:library:fl_call_object_callback}
    \index{xformslib \textit{(package)}!xformslib.library \textit{(module)}!xformslib.library.fl\_call\_object\_callback \textit{(function)}}

    \vspace{0.5ex}

\hspace{.8\funcindent}\begin{boxedminipage}{\funcwidth}

    \raggedright \textbf{fl\_call\_object\_callback}(\textit{pObject})

    \vspace{-1.5ex}

    \rule{\textwidth}{0.5\fboxrule}
\setlength{\parskip}{2ex}
    Invokes the callback manually (as opposed to invocation by the main 
    loop). If the object does not have a callback associated with it, this 
    call has not effect.

\setlength{\parskip}{1ex}
      \textbf{Parameters}
      \vspace{-1ex}

      \begin{quote}
        \begin{Ventry}{xxxxxxx}

          \item[pObject]

          object ({\textless}pointer to xfdata.FL\_OBJECT{\textgreater})

        \end{Ventry}

      \end{quote}

\textbf{Example:} fl\_call\_object\_callback(pobj\_with\_cb)



\textbf{Status:} Tested + Doc + Demo = OK



    \end{boxedminipage}

    \label{xformslib:library:fl_set_object_prehandler}
    \index{xformslib \textit{(package)}!xformslib.library \textit{(module)}!xformslib.library.fl\_set\_object\_prehandler \textit{(function)}}

    \vspace{0.5ex}

\hspace{.8\funcindent}\begin{boxedminipage}{\funcwidth}

    \raggedright \textbf{fl\_set\_object\_prehandler}(\textit{pObject}, \textit{py\_HandlerPtr})

    \vspace{-1.5ex}

    \rule{\textwidth}{0.5\fboxrule}
\setlength{\parskip}{2ex}
    By-passes the internal event processing for a particular object. The 
    pre-handler will be called before the built-in object handler. By 
    electing to handle some of the events, a pre-handler can, in effect, 
    replace part of the built-in handler.

\setlength{\parskip}{1ex}
      \textbf{Parameters}
      \vspace{-1ex}

      \begin{quote}
        \begin{Ventry}{xxxxxxxxxxxxx}

          \item[pObject]

          object ({\textless}pointer to xfdata.FL\_OBJECT{\textgreater})

          \item[py\_HandlerPtr]

          python callback function, returning value

            {\it (type=\_\_ funcname (pObject, num, coord, coord, num, ptr\_void) -{\textgreater} 
num \_\_)}

        \end{Ventry}

      \end{quote}

      \textbf{Return Value}
    \vspace{-1ex}

      \begin{quote}
      pHandlerPtr

      \end{quote}

\textbf{Example:}
\begin{quote}
  \begin{itemize}

  \item
    \setlength{\parskip}{0.6ex}
def prehandlecb(pobj, num, crd, crd, num2, vdata):



  \item {\textbar}-{\textgreater}{\textbar} ...



  \item {\textbar}-{\textgreater}{\textbar} return 0



  \item fl\_set\_object\_prehandler(pobj2, prehandlecb)



\end{itemize}

\end{quote}

\textbf{Status:} Tested + Doc + Demo = OK



    \end{boxedminipage}

    \label{xformslib:library:fl_set_object_posthandler}
    \index{xformslib \textit{(package)}!xformslib.library \textit{(module)}!xformslib.library.fl\_set\_object\_posthandler \textit{(function)}}

    \vspace{0.5ex}

\hspace{.8\funcindent}\begin{boxedminipage}{\funcwidth}

    \raggedright \textbf{fl\_set\_object\_posthandler}(\textit{pObject}, \textit{py\_HandlerPtr})

    \vspace{-1.5ex}

    \rule{\textwidth}{0.5\fboxrule}
\setlength{\parskip}{2ex}
    By-passes the internal event processing for a particular object. The 
    post-handler will be invoked after the built-in handler finishes. 
    Whenever possible a post-handler should be used instead of a 
    pre-handler.

\setlength{\parskip}{1ex}
      \textbf{Parameters}
      \vspace{-1ex}

      \begin{quote}
        \begin{Ventry}{xxxxxxxxxxxxx}

          \item[pObject]

          pointer to object ({\textless}pointer to 
          xfdata.FL\_OBJECT{\textgreater})

          \item[py\_HandlerPtr]

          python callback function, returning value

            {\it (type=\_\_ funcname (pObject, num, coord, coord, num, ptr\_void) -{\textgreater} 
num \_\_)}

        \end{Ventry}

      \end{quote}

      \textbf{Return Value}
    \vspace{-1ex}

      \begin{quote}
      pHandlerPtr

      \end{quote}

\textbf{Example:}
\begin{quote}
  \begin{itemize}

  \item
    \setlength{\parskip}{0.6ex}
def posthandlecb(pobj, num, crd, crd, num2, vdata):



  \item {\textbar}-{\textgreater}{\textbar} ...



  \item {\textbar}-{\textgreater}{\textbar} return 0



  \item fl\_set\_object\_posthandler(pobj2, posthandlecb)



\end{itemize}

\end{quote}

\textbf{Status:} Tested + Doc + Demo = OK



    \end{boxedminipage}

    \label{xformslib:library:fl_set_object_callback}
    \index{xformslib \textit{(package)}!xformslib.library \textit{(module)}!xformslib.library.fl\_set\_object\_callback \textit{(function)}}

    \vspace{0.5ex}

\hspace{.8\funcindent}\begin{boxedminipage}{\funcwidth}

    \raggedright \textbf{fl\_set\_object\_callback}(\textit{pObject}, \textit{py\_CallbackPtr}, \textit{data})

    \vspace{-1.5ex}

    \rule{\textwidth}{0.5\fboxrule}
\setlength{\parskip}{2ex}
    Calls a callback function bound to an object, if a condition is met.

\setlength{\parskip}{1ex}
      \textbf{Parameters}
      \vspace{-1ex}

      \begin{quote}
        \begin{Ventry}{xxxxxxxxxxxxxx}

          \item[pObject]

          object the callback is bound to ({\textless}pointer to 
          xfdata.FL\_OBJECT{\textgreater})

          \item[py\_CallbackPtr]

          a python function with no () and no args to be used as callback, 
          no return

            {\it (type=\_\_ funcname (pObject, longnum) \_\_)}

          \item[data]

          argument being passed to function {\textless}long{\textgreater}

        \end{Ventry}

      \end{quote}

      \textbf{Return Value}
    \vspace{-1ex}

      \begin{quote}
      old xfdata.FL\_CALLBACKPTR function

      {\it (type=c\_callback func)}

      \end{quote}

\textbf{Example:}
\begin{quote}
  \begin{itemize}

  \item
    \setlength{\parskip}{0.6ex}
def myobcb(pobj, longdata):



  \item {\textbar}-{\textgreater}{\textbar} ...



  \item oldcb = fl\_set\_object\_callback(pobj3, myobjcb, 0)



\end{itemize}

\end{quote}

\textbf{Status:} Tested + Doc + Demo = OK



    \end{boxedminipage}

    \label{xformslib:library:fl_set_object_callback}
    \index{xformslib \textit{(package)}!xformslib.library \textit{(module)}!xformslib.library.fl\_set\_object\_callback \textit{(function)}}

    \vspace{0.5ex}

\hspace{.8\funcindent}\begin{boxedminipage}{\funcwidth}

    \raggedright \textbf{fl\_set\_call\_back}(\textit{pObject}, \textit{py\_CallbackPtr}, \textit{data})

    \vspace{-1.5ex}

    \rule{\textwidth}{0.5\fboxrule}
\setlength{\parskip}{2ex}
    Calls a callback function bound to an object, if a condition is met.

\setlength{\parskip}{1ex}
      \textbf{Parameters}
      \vspace{-1ex}

      \begin{quote}
        \begin{Ventry}{xxxxxxxxxxxxxx}

          \item[pObject]

          object the callback is bound to ({\textless}pointer to 
          xfdata.FL\_OBJECT{\textgreater})

          \item[py\_CallbackPtr]

          a python function with no () and no args to be used as callback, 
          no return

            {\it (type=\_\_ funcname (pObject, longnum) \_\_)}

          \item[data]

          argument being passed to function {\textless}long{\textgreater}

        \end{Ventry}

      \end{quote}

      \textbf{Return Value}
    \vspace{-1ex}

      \begin{quote}
      old xfdata.FL\_CALLBACKPTR function

      {\it (type=c\_callback func)}

      \end{quote}

\textbf{Example:}
\begin{quote}
  \begin{itemize}

  \item
    \setlength{\parskip}{0.6ex}
def myobcb(pobj, longdata):



  \item {\textbar}-{\textgreater}{\textbar} ...



  \item oldcb = fl\_set\_object\_callback(pobj3, myobjcb, 0)



\end{itemize}

\end{quote}

\textbf{Status:} Tested + Doc + Demo = OK



    \end{boxedminipage}

    \label{xformslib:library:fl_redraw_object}
    \index{xformslib \textit{(package)}!xformslib.library \textit{(module)}!xformslib.library.fl\_redraw\_object \textit{(function)}}

    \vspace{0.5ex}

\hspace{.8\funcindent}\begin{boxedminipage}{\funcwidth}

    \raggedright \textbf{fl\_redraw\_object}(\textit{pObject})

    \vspace{-1.5ex}

    \rule{\textwidth}{0.5\fboxrule}
\setlength{\parskip}{2ex}
    Redraws the particular object. If it is a group it redraws the complete
    group. Normally you should never need this routine because all library 
    routines take care of redrawing objects when necessary, but there might
    be situations in which an explicit redraw is required.

\setlength{\parskip}{1ex}
      \textbf{Parameters}
      \vspace{-1ex}

      \begin{quote}
        \begin{Ventry}{xxxxxxx}

          \item[pObject]

          object to redraw ({\textless}pointer to 
          xfdata.FL\_OBJECT{\textgreater})

        \end{Ventry}

      \end{quote}

\textbf{Example:} fl\_redraw\_object(pobj)



\textbf{Status:} Tested + Doc + Demo = OK



    \end{boxedminipage}

    \label{xformslib:library:fl_scale_object}
    \index{xformslib \textit{(package)}!xformslib.library \textit{(module)}!xformslib.library.fl\_scale\_object \textit{(function)}}

    \vspace{0.5ex}

\hspace{.8\funcindent}\begin{boxedminipage}{\funcwidth}

    \raggedright \textbf{fl\_scale\_object}(\textit{pObject}, \textit{xs}, \textit{ys})

    \vspace{-1.5ex}

    \rule{\textwidth}{0.5\fboxrule}
\setlength{\parskip}{2ex}
    Scales (shrinking or enlarging) an object, indicating a scaling factor 
    in x- and y-direction (1.1 = 110 percent, 0.5 = 50, etc.)

\setlength{\parskip}{1ex}
      \textbf{Parameters}
      \vspace{-1ex}

      \begin{quote}
        \begin{Ventry}{xxxxxxx}

          \item[pObject]

          object to be scaled ({\textless}pointer to 
          xfdata.FL\_OBJECT{\textgreater})

          \item[xs]

          new horizontal factor ({\textless}float{\textgreater})

          \item[ys]

          new vertical factor ({\textless}float{\textgreater})

        \end{Ventry}

      \end{quote}

\textbf{Example:} fl\_scale\_object(pobj, 0.8, 1.1)



\textbf{Status:} Tested + Doc + NoDemo = OK



    \end{boxedminipage}

    \label{xformslib:library:fl_show_object}
    \index{xformslib \textit{(package)}!xformslib.library \textit{(module)}!xformslib.library.fl\_show\_object \textit{(function)}}

    \vspace{0.5ex}

\hspace{.8\funcindent}\begin{boxedminipage}{\funcwidth}

    \raggedright \textbf{fl\_show\_object}(\textit{pObject})

    \vspace{-1.5ex}

    \rule{\textwidth}{0.5\fboxrule}
\setlength{\parskip}{2ex}
    Shows an (hidden) object.

\setlength{\parskip}{1ex}
      \textbf{Parameters}
      \vspace{-1ex}

      \begin{quote}
        \begin{Ventry}{xxxxxxx}

          \item[pObject]

          object to be shown ({\textless}pointer to 
          xfdata.FL\_OBJECT{\textgreater})

        \end{Ventry}

      \end{quote}

\textbf{Example:} fl\_show\_object(pobj8)



\textbf{Status:} Tested + Doc + Demo = OK



    \end{boxedminipage}

    \label{xformslib:library:fl_hide_object}
    \index{xformslib \textit{(package)}!xformslib.library \textit{(module)}!xformslib.library.fl\_hide\_object \textit{(function)}}

    \vspace{0.5ex}

\hspace{.8\funcindent}\begin{boxedminipage}{\funcwidth}

    \raggedright \textbf{fl\_hide\_object}(\textit{pObject})

    \vspace{-1.5ex}

    \rule{\textwidth}{0.5\fboxrule}
\setlength{\parskip}{2ex}
    Hides a shown object.

\setlength{\parskip}{1ex}
      \textbf{Parameters}
      \vspace{-1ex}

      \begin{quote}
        \begin{Ventry}{xxxxxxx}

          \item[pObject]

          object to be hidden ({\textless}pointer to 
          xfdata.FL\_OBJECT{\textgreater})

        \end{Ventry}

      \end{quote}

\textbf{Example:} fl\_hide\_object(pobj8)



\textbf{Status:} Tested + Doc + Demo = OK



    \end{boxedminipage}

    \label{xformslib:library:fl_object_is_visible}
    \index{xformslib \textit{(package)}!xformslib.library \textit{(module)}!xformslib.library.fl\_object\_is\_visible \textit{(function)}}

    \vspace{0.5ex}

\hspace{.8\funcindent}\begin{boxedminipage}{\funcwidth}

    \raggedright \textbf{fl\_object\_is\_visible}(\textit{pObject})

    \vspace{-1.5ex}

    \rule{\textwidth}{0.5\fboxrule}
\setlength{\parskip}{2ex}
    Returns if an object is visible or not.

\setlength{\parskip}{1ex}
      \textbf{Parameters}
      \vspace{-1ex}

      \begin{quote}
        \begin{Ventry}{xxxxxxx}

          \item[pObject]

          object to evaluate ({\textless}pointer to 
          xfdata.FL\_OBJECT{\textgreater})

        \end{Ventry}

      \end{quote}

      \textbf{Return Value}
    \vspace{-1ex}

      \begin{quote}
      flag 0 (invisible) or non-zero (visible) 
      ({\textless}int{\textgreater})

      {\it (type=flag)}

      \end{quote}

\textbf{Example:} if not fl\_object\_is\_visible(pobj2): {\textgreater} ...



\textbf{Status:} Tested + Doc + Demo = OK



    \end{boxedminipage}

    \label{xformslib:library:fl_free_object}
    \index{xformslib \textit{(package)}!xformslib.library \textit{(module)}!xformslib.library.fl\_free\_object \textit{(function)}}

    \vspace{0.5ex}

\hspace{.8\funcindent}\begin{boxedminipage}{\funcwidth}

    \raggedright \textbf{fl\_free\_object}(\textit{pObject})

    \vspace{-1.5ex}

    \rule{\textwidth}{0.5\fboxrule}
\setlength{\parskip}{2ex}
    Frees the object and finally destroys it (if necessary deletes the 
    object first).

\setlength{\parskip}{1ex}
      \textbf{Parameters}
      \vspace{-1ex}

      \begin{quote}
        \begin{Ventry}{xxxxxxx}

          \item[pObject]

          object to free ({\textless}pointer to 
          xfdata.FL\_OBJECT{\textgreater})

        \end{Ventry}

      \end{quote}

\textbf{Example:} fl\_free\_object(pobj)



\textbf{Status:} Tested + Doc + NoDemo = OK



    \end{boxedminipage}

    \label{xformslib:library:fl_delete_object}
    \index{xformslib \textit{(package)}!xformslib.library \textit{(module)}!xformslib.library.fl\_delete\_object \textit{(function)}}

    \vspace{0.5ex}

\hspace{.8\funcindent}\begin{boxedminipage}{\funcwidth}

    \raggedright \textbf{fl\_delete\_object}(\textit{pObject})

    \vspace{-1.5ex}

    \rule{\textwidth}{0.5\fboxrule}
\setlength{\parskip}{2ex}
    Deletes an object, breaking its connection to the form, but not 
    destroying it.

\setlength{\parskip}{1ex}
      \textbf{Parameters}
      \vspace{-1ex}

      \begin{quote}
        \begin{Ventry}{xxxxxxx}

          \item[pObject]

          object to delete ({\textless}pointer to 
          xfdata.FL\_OBJECT{\textgreater})

        \end{Ventry}

      \end{quote}

\textbf{Example:} fl\_delete\_object(pobj)



\textbf{Status:} Tested + Doc + NoDemo = OK



    \end{boxedminipage}

    \label{xformslib:library:fl_get_object_return_state}
    \index{xformslib \textit{(package)}!xformslib.library \textit{(module)}!xformslib.library.fl\_get\_object\_return\_state \textit{(function)}}

    \vspace{0.5ex}

\hspace{.8\funcindent}\begin{boxedminipage}{\funcwidth}

    \raggedright \textbf{fl\_get\_object\_return\_state}(\textit{pObject})

    \vspace{-1.5ex}

    \rule{\textwidth}{0.5\fboxrule}
\setlength{\parskip}{2ex}
    Determines the reason an object was returned (or its callback invoked) 
    you. The returned value is logical OR of the conditions that led to the
    object getting returned.

\setlength{\parskip}{1ex}
      \textbf{Parameters}
      \vspace{-1ex}

      \begin{quote}
        \begin{Ventry}{xxxxxxx}

          \item[pObject]

          object to evaluate ({\textless}pointer to 
          xfdata.FL\_OBJECT{\textgreater})

        \end{Ventry}

      \end{quote}

      \textbf{Return Value}
    \vspace{-1ex}

      \begin{quote}
      current return state ({\textless}int{\textgreater})

      {\it (type=ID num)}

      \end{quote}

\textbf{Example:} currstate = fl\_get\_object\_return\_state(pobj5)



\textbf{Status:} Tested + Doc + NoDemo = OK



    \end{boxedminipage}

    \label{xformslib:library:fl_trigger_object}
    \index{xformslib \textit{(package)}!xformslib.library \textit{(module)}!xformslib.library.fl\_trigger\_object \textit{(function)}}

    \vspace{0.5ex}

\hspace{.8\funcindent}\begin{boxedminipage}{\funcwidth}

    \raggedright \textbf{fl\_trigger\_object}(\textit{pObject})

    \vspace{-1.5ex}

    \rule{\textwidth}{0.5\fboxrule}
\setlength{\parskip}{2ex}
    Simulates the action of an object being triggered from within the 
    program. Calling this routine on an object obj results in the object 
    returned to the application program or its callback being called if it 
    exists. Note however, there is no visual feedback, i.e. 
    fl\_trigger\_object(button) will not make the button object named 
    button appearing to be pushed.

\setlength{\parskip}{1ex}
      \textbf{Parameters}
      \vspace{-1ex}

      \begin{quote}
        \begin{Ventry}{xxxxxxx}

          \item[pObject]

          object to trigger ({\textless}pointer to 
          xfdata.FL\_OBJECT{\textgreater})

        \end{Ventry}

      \end{quote}

\textbf{Example:} fl\_trigger\_object(pobj



\textbf{Status:} Tested + Doc + NoDemo = OK



    \end{boxedminipage}

    \label{xformslib:library:fl_activate_object}
    \index{xformslib \textit{(package)}!xformslib.library \textit{(module)}!xformslib.library.fl\_activate\_object \textit{(function)}}

    \vspace{0.5ex}

\hspace{.8\funcindent}\begin{boxedminipage}{\funcwidth}

    \raggedright \textbf{fl\_activate\_object}(\textit{pObject})

    \vspace{-1.5ex}

    \rule{\textwidth}{0.5\fboxrule}
\setlength{\parskip}{2ex}
    (Re)activates an object, (re)enabling user interaction.

\setlength{\parskip}{1ex}
      \textbf{Parameters}
      \vspace{-1ex}

      \begin{quote}
        \begin{Ventry}{xxxxxxx}

          \item[pObject]

          object to activate ({\textless}pointer to 
          xfdata.FL\_OBJECT{\textgreater})

        \end{Ventry}

      \end{quote}

\textbf{Example:} fl\_activate\_object(pobj)



\textbf{Status:} Tested + Doc + Demo = OK



    \end{boxedminipage}

    \label{xformslib:library:fl_deactivate_object}
    \index{xformslib \textit{(package)}!xformslib.library \textit{(module)}!xformslib.library.fl\_deactivate\_object \textit{(function)}}

    \vspace{0.5ex}

\hspace{.8\funcindent}\begin{boxedminipage}{\funcwidth}

    \raggedright \textbf{fl\_deactivate\_object}(\textit{pObject})

    \vspace{-1.5ex}

    \rule{\textwidth}{0.5\fboxrule}
\setlength{\parskip}{2ex}
    Makes a particular object to be temporarily inactive, disabling user 
    interaction, e.g., you want to make it impossible for the user to press
    a particular button or to type input in a particular field. When object
    is a group, the whole group is deactivate.

\setlength{\parskip}{1ex}
      \textbf{Parameters}
      \vspace{-1ex}

      \begin{quote}
        \begin{Ventry}{xxxxxxx}

          \item[pObject]

          object to deactivate ({\textless}pointer to 
          xfdata.FL\_OBJECT{\textgreater})

        \end{Ventry}

      \end{quote}

\textbf{Example:} fl\_deactivate\_object(pactobj)



\textbf{Status:} Tested + Doc + Demo = OK



    \end{boxedminipage}

    \label{xformslib:library:fl_object_is_active}
    \index{xformslib \textit{(package)}!xformslib.library \textit{(module)}!xformslib.library.fl\_object\_is\_active \textit{(function)}}

    \vspace{0.5ex}

\hspace{.8\funcindent}\begin{boxedminipage}{\funcwidth}

    \raggedright \textbf{fl\_object\_is\_active}(\textit{pObject})

    \vspace{-1.5ex}

    \rule{\textwidth}{0.5\fboxrule}
\setlength{\parskip}{2ex}
    Returns if object is active and reacting to events, or not.

\setlength{\parskip}{1ex}
      \textbf{Parameters}
      \vspace{-1ex}

      \begin{quote}
        \begin{Ventry}{xxxxxxx}

          \item[pObject]

          object to evaluate ({\textless}pointer to 
          xfdata.FL\_OBJECT{\textgreater})

        \end{Ventry}

      \end{quote}

      \textbf{Return Value}
    \vspace{-1ex}

      \begin{quote}
      flag 0 (not active) or non-zero (active) 
      ({\textless}int{\textgreater})

      {\it (type=flag)}

      \end{quote}

\textbf{Example:} if not fl\_object\_is\_active(pobj): {\textgreater} ...



\textbf{Status:} Tested + Doc + Demo = OK



    \end{boxedminipage}

    \label{xformslib:library:fl_enumerate_fonts}
    \index{xformslib \textit{(package)}!xformslib.library \textit{(module)}!xformslib.library.fl\_enumerate\_fonts \textit{(function)}}

    \vspace{0.5ex}

\hspace{.8\funcindent}\begin{boxedminipage}{\funcwidth}

    \raggedright \textbf{fl\_enumerate\_fonts}(\textit{py\_output}, \textit{shortform})

    \vspace{-1.5ex}

    \rule{\textwidth}{0.5\fboxrule}
\setlength{\parskip}{2ex}
    Lists built-in fonts.

\setlength{\parskip}{1ex}
      \textbf{Parameters}
      \vspace{-1ex}

      \begin{quote}
        \begin{Ventry}{xxxxxxxxx}

          \item[py\_output]

          python callback function - no return

            {\it (type=\_\_ funcname (string) \_\_)}

          \item[shortform]

          flag to use short form or not ({\textless}int{\textgreater})

            {\it (type=0 (long form used) or non-zero (short form used))}

        \end{Ventry}

      \end{quote}

      \textbf{Return Value}
    \vspace{-1ex}

      \begin{quote}
      number of listed fonts ({\textless}int{\textgreater})

      {\it (type=num)}

      \end{quote}

\textbf{Example:}
\begin{quote}
  \begin{itemize}

  \item
    \setlength{\parskip}{0.6ex}
def pyoutput(strng):



  \item {\textbar}-{\textgreater}{\textbar} print strng



  \item numfonts = fl\_enumerate(pyoutput, 0)



\end{itemize}

\end{quote}

\textbf{Status:} Tested + Doc + Demo = OK



    \end{boxedminipage}

    \label{xformslib:library:fl_set_font_name}
    \index{xformslib \textit{(package)}!xformslib.library \textit{(module)}!xformslib.library.fl\_set\_font\_name \textit{(function)}}

    \vspace{0.5ex}

\hspace{.8\funcindent}\begin{boxedminipage}{\funcwidth}

    \raggedright \textbf{fl\_set\_font\_name}(\textit{fontnum}, \textit{name})

    \vspace{-1.5ex}

    \rule{\textwidth}{0.5\fboxrule}
\setlength{\parskip}{2ex}
    Add a new font (indexed by n) or change an existing font. Preferably 
    the font name contains a '?' in the size position so different sizes 
    can be used. Redraw of all forms is required to actually see the change
    for visible form.

\setlength{\parskip}{1ex}
      \textbf{Parameters}
      \vspace{-1ex}

      \begin{quote}
        \begin{Ventry}{xxxxxxx}

          \item[fontnum]

          font number ({\textless}int{\textgreater})

            {\it (type=value between 0 and xfdata.FL\_MAXFONTS-1)}

          \item[name]

          font name ({\textless}string{\textgreater})

        \end{Ventry}

      \end{quote}

      \textbf{Return Value}
    \vspace{-1ex}

      \begin{quote}
      -1 (on errors) or 0 or 1 ({\textless}int{\textgreater})

      {\it (type=ID num)}

      \end{quote}

\textbf{Example:} fl\_set\_font\_name(40, "symbol-medium-whatever")



\textbf{Status:} Tested + Doc + NoDemo = OK



    \end{boxedminipage}

    \label{xformslib:library:fl_set_font}
    \index{xformslib \textit{(package)}!xformslib.library \textit{(module)}!xformslib.library.fl\_set\_font \textit{(function)}}

    \vspace{0.5ex}

\hspace{.8\funcindent}\begin{boxedminipage}{\funcwidth}

    \raggedright \textbf{fl\_set\_font}(\textit{fontnum}, \textit{size})

    \vspace{-1.5ex}

    \rule{\textwidth}{0.5\fboxrule}
\setlength{\parskip}{2ex}
    Makes the font current.

\setlength{\parskip}{1ex}
      \textbf{Parameters}
      \vspace{-1ex}

      \begin{quote}
        \begin{Ventry}{xxxxxxx}

          \item[fontnum]

          font number ({\textless}int{\textgreater})

          \item[size]

          font size ({\textless}int{\textgreater})

            {\it (type=(from xfdata module) FL\_TINY\_SIZE, FL\_SMALL\_SIZE, FL\_NORMAL\_SIZE, 
FL\_MEDIUM\_SIZE, FL\_LARGE\_SIZE, FL\_HUGE\_SIZE, FL\_DEFAULT\_SIZE)}

        \end{Ventry}

      \end{quote}

\textbf{Example:} fl\_set\_font(5, xfdata.FL\_SMALL\_SIZE)



\textbf{Status:} Tested + Doc + NoDemo = OK



    \end{boxedminipage}

    \label{xformslib:library:fl_get_char_height}
    \index{xformslib \textit{(package)}!xformslib.library \textit{(module)}!xformslib.library.fl\_get\_char\_height \textit{(function)}}

    \vspace{0.5ex}

\hspace{.8\funcindent}\begin{boxedminipage}{\funcwidth}

    \raggedright \textbf{fl\_get\_char\_height}(\textit{style}, \textit{size})

    \vspace{-1.5ex}

    \rule{\textwidth}{0.5\fboxrule}
\setlength{\parskip}{2ex}
    Returns the maximum height of the used font and the height above and 
    below the baseline of the font.

\setlength{\parskip}{1ex}
      \textbf{Parameters}
      \vspace{-1ex}

      \begin{quote}
        \begin{Ventry}{xxxxx}

          \item[style]

          font style ({\textless}int{\textgreater})

            {\it (type=(from xfdata module) FL\_NORMAL\_STYLE, FL\_BOLD\_STYLE, FL\_ITALIC\_STYLE,
FL\_BOLDITALIC\_STYLE, FL\_FIXED\_STYLE, FL\_FIXEDBOLD\_STYLE, 
FL\_FIXEDITALIC\_STYLE, FL\_FIXEDBOLDITALIC\_STYLE, FL\_TIMES\_STYLE, 
FL\_TIMESBOLD\_STYLE, FL\_TIMESITALIC\_STYLE, FL\_TIMESBOLDITALIC\_STYLE, 
FL\_MISC\_STYLE, FL\_MISCBOLD\_STYLE, FL\_MISCITALIC\_STYLE, 
FL\_SYMBOL\_STYLE, FL\_SHADOW\_STYLE, FL\_ENGRAVED\_STYLE, 
FL\_EMBOSSED\_STYLE)}

          \item[size]

          font size ({\textless}int{\textgreater})

            {\it (type=(from xfdata module) FL\_TINY\_SIZE, FL\_SMALL\_SIZE, FL\_NORMAL\_SIZE, 
FL\_MEDIUM\_SIZE, FL\_LARGE\_SIZE, FL\_HUGE\_SIZE, FL\_DEFAULT\_SIZE)}

        \end{Ventry}

      \end{quote}

      \textbf{Return Value}
    \vspace{-1ex}

      \begin{quote}
      height, ascendent and descendent ({\textless}int{\textgreater}, 
      {\textless}int{\textgreater}, {\textless}int{\textgreater})

      {\it (type=h, asc, desc)}

      \end{quote}

\textbf{Example:} hei, asc, desc = fl\_get\_char\_height(xfdata.FL\_BOLD\_STYLE, 
xfdata.FL\_TINY\_SIZE)



\textbf{Attention:} API change from XForms - upstream was fl\_get\_char\_height(style, size, 
asc, desc)



\textbf{Status:} Tested + Doc + NoDemo = OK



    \end{boxedminipage}

    \label{xformslib:library:fl_get_char_width}
    \index{xformslib \textit{(package)}!xformslib.library \textit{(module)}!xformslib.library.fl\_get\_char\_width \textit{(function)}}

    \vspace{0.5ex}

\hspace{.8\funcindent}\begin{boxedminipage}{\funcwidth}

    \raggedright \textbf{fl\_get\_char\_width}(\textit{style}, \textit{size})

    \vspace{-1.5ex}

    \rule{\textwidth}{0.5\fboxrule}
\setlength{\parskip}{2ex}
    Returns the maximum width of the used font.

\setlength{\parskip}{1ex}
      \textbf{Parameters}
      \vspace{-1ex}

      \begin{quote}
        \begin{Ventry}{xxxxx}

          \item[style]

          font style ({\textless}int{\textgreater})

            {\it (type=(from xfdata module) FL\_NORMAL\_STYLE, FL\_BOLD\_STYLE, FL\_ITALIC\_STYLE,
FL\_BOLDITALIC\_STYLE, FL\_FIXED\_STYLE, FL\_FIXEDBOLD\_STYLE, 
FL\_FIXEDITALIC\_STYLE, FL\_FIXEDBOLDITALIC\_STYLE, FL\_TIMES\_STYLE, 
FL\_TIMESBOLD\_STYLE, FL\_TIMESITALIC\_STYLE, FL\_TIMESBOLDITALIC\_STYLE, 
FL\_MISC\_STYLE, FL\_MISCBOLD\_STYLE, FL\_MISCITALIC\_STYLE, 
FL\_SYMBOL\_STYLE, FL\_SHADOW\_STYLE, FL\_ENGRAVED\_STYLE, 
FL\_EMBOSSED\_STYLE)}

          \item[size]

          font size ({\textless}int{\textgreater})

            {\it (type=(from xfdata module) FL\_TINY\_SIZE, FL\_SMALL\_SIZE, FL\_NORMAL\_SIZE, 
FL\_MEDIUM\_SIZE, FL\_LARGE\_SIZE, FL\_HUGE\_SIZE, FL\_DEFAULT\_SIZE)}

        \end{Ventry}

      \end{quote}

      \textbf{Return Value}
    \vspace{-1ex}

      \begin{quote}
      width ({\textless}int{\textgreater})

      {\it (type=w)}

      \end{quote}

\textbf{Example:} wid = fl\_get\_char\_width(xfdata.FL\_TIMES\_STYLE, xfdata.FL\_HUGE\_SIZE)



\textbf{Status:} Tested + Doc + NoDemo = OK



    \end{boxedminipage}

    \label{xformslib:library:fl_get_string_height}
    \index{xformslib \textit{(package)}!xformslib.library \textit{(module)}!xformslib.library.fl\_get\_string\_height \textit{(function)}}

    \vspace{0.5ex}

\hspace{.8\funcindent}\begin{boxedminipage}{\funcwidth}

    \raggedright \textbf{fl\_get\_string\_height}(\textit{style}, \textit{size}, \textit{strng}, \textit{strglen})

    \vspace{-1.5ex}

    \rule{\textwidth}{0.5\fboxrule}
\setlength{\parskip}{2ex}
    Obtains the height information of a specific string and the height 
    above and below the font's baseline.

\setlength{\parskip}{1ex}
      \textbf{Parameters}
      \vspace{-1ex}

      \begin{quote}
        \begin{Ventry}{xxxxxxx}

          \item[style]

          font style ({\textless}int{\textgreater})

            {\it (type=(from xfdata module) FL\_NORMAL\_STYLE, FL\_BOLD\_STYLE, FL\_ITALIC\_STYLE,
FL\_BOLDITALIC\_STYLE, FL\_FIXED\_STYLE, FL\_FIXEDBOLD\_STYLE, 
FL\_FIXEDITALIC\_STYLE, FL\_FIXEDBOLDITALIC\_STYLE, FL\_TIMES\_STYLE, 
FL\_TIMESBOLD\_STYLE, FL\_TIMESITALIC\_STYLE, FL\_TIMESBOLDITALIC\_STYLE, 
FL\_MISC\_STYLE, FL\_MISCBOLD\_STYLE, FL\_MISCITALIC\_STYLE, 
FL\_SYMBOL\_STYLE, FL\_SHADOW\_STYLE, FL\_ENGRAVED\_STYLE, 
FL\_EMBOSSED\_STYLE)}

          \item[size]

          font size ({\textless}int{\textgreater})

            {\it (type=(from xfdata module) FL\_TINY\_SIZE, FL\_SMALL\_SIZE, FL\_NORMAL\_SIZE, 
FL\_MEDIUM\_SIZE, FL\_LARGE\_SIZE, FL\_HUGE\_SIZE, FL\_DEFAULT\_SIZE)}

          \item[strng]

          text ({\textless}string{\textgreater})

          \item[strglen]

          length of string ({\textless}int{\textgreater})

        \end{Ventry}

      \end{quote}

      \textbf{Return Value}
    \vspace{-1ex}

      \begin{quote}
      height, ascendent and descendent ({\textless}int{\textgreater}, 
      {\textless}int{\textgreater}, {\textless}int{\textgreater})

      {\it (type=h, asc, desc)}

      \end{quote}

\textbf{Example:} hei, asc, desc = fl\_get\_string\_height(xfdata.FL\_MISC\_STYLE, 
xfdata.FL\_MEDIUM\_SIZE, "Mystring", 8)



\textbf{Attention:} API change from XForms - upstream was fl\_get\_string\_height(style, size, 
strng, strglen, asc, desc)



\textbf{Status:} Tested + Doc + Demo = OK



    \end{boxedminipage}

    \label{xformslib:library:fl_get_string_width}
    \index{xformslib \textit{(package)}!xformslib.library \textit{(module)}!xformslib.library.fl\_get\_string\_width \textit{(function)}}

    \vspace{0.5ex}

\hspace{.8\funcindent}\begin{boxedminipage}{\funcwidth}

    \raggedright \textbf{fl\_get\_string\_width}(\textit{style}, \textit{size}, \textit{strng}, \textit{strglen})

    \vspace{-1.5ex}

    \rule{\textwidth}{0.5\fboxrule}
\setlength{\parskip}{2ex}
    Obtains the width information for a specific string.

\setlength{\parskip}{1ex}
      \textbf{Parameters}
      \vspace{-1ex}

      \begin{quote}
        \begin{Ventry}{xxxxxxx}

          \item[style]

          font style ({\textless}int{\textgreater})

            {\it (type=(from xfdata module) FL\_NORMAL\_STYLE, FL\_BOLD\_STYLE, FL\_ITALIC\_STYLE,
FL\_BOLDITALIC\_STYLE, FL\_FIXED\_STYLE, FL\_FIXEDBOLD\_STYLE, 
FL\_FIXEDITALIC\_STYLE, FL\_FIXEDBOLDITALIC\_STYLE, FL\_TIMES\_STYLE, 
FL\_TIMESBOLD\_STYLE, FL\_TIMESITALIC\_STYLE, FL\_TIMESBOLDITALIC\_STYLE, 
FL\_MISC\_STYLE, FL\_MISCBOLD\_STYLE, FL\_MISCITALIC\_STYLE, 
FL\_SYMBOL\_STYLE, FL\_SHADOW\_STYLE, FL\_ENGRAVED\_STYLE, 
FL\_EMBOSSED\_STYLE)}

          \item[size]

          font size ({\textless}int{\textgreater})

            {\it (type=(from xfdata module) FL\_TINY\_SIZE, FL\_SMALL\_SIZE, FL\_NORMAL\_SIZE, 
FL\_MEDIUM\_SIZE, FL\_LARGE\_SIZE, FL\_HUGE\_SIZE, FL\_DEFAULT\_SIZE)}

          \item[strng]

          text ({\textless}string{\textgreater})

          \item[strglen]

          length of string ({\textless}int{\textgreater})

        \end{Ventry}

      \end{quote}

      \textbf{Return Value}
    \vspace{-1ex}

      \begin{quote}
      width ({\textless}int{\textgreater})

      {\it (type=w)}

      \end{quote}

\textbf{Example:} wid = fl\_get\_string\_width(xfdata.FL\_MISC\_STYLE, 
xfdata.FL\_MEDIUM\_SIZE, "Mystring", 8)



\textbf{Status:} Tested + Doc + Demo = OK



    \end{boxedminipage}

    \label{xformslib:library:fl_get_string_widthTAB}
    \index{xformslib \textit{(package)}!xformslib.library \textit{(module)}!xformslib.library.fl\_get\_string\_widthTAB \textit{(function)}}

    \vspace{0.5ex}

\hspace{.8\funcindent}\begin{boxedminipage}{\funcwidth}

    \raggedright \textbf{fl\_get\_string\_widthTAB}(\textit{style}, \textit{size}, \textit{strng}, \textit{strglen})

    \vspace{-1.5ex}

    \rule{\textwidth}{0.5\fboxrule}
\setlength{\parskip}{2ex}
\setlength{\parskip}{1ex}
      \textbf{Parameters}
      \vspace{-1ex}

      \begin{quote}
        \begin{Ventry}{xxxxxxx}

          \item[style]

          font style ({\textless}int{\textgreater})

            {\it (type=(from xfdata module) FL\_NORMAL\_STYLE, FL\_BOLD\_STYLE, FL\_ITALIC\_STYLE,
FL\_BOLDITALIC\_STYLE, FL\_FIXED\_STYLE, FL\_FIXEDBOLD\_STYLE, 
FL\_FIXEDITALIC\_STYLE, FL\_FIXEDBOLDITALIC\_STYLE, FL\_TIMES\_STYLE, 
FL\_TIMESBOLD\_STYLE, FL\_TIMESITALIC\_STYLE, FL\_TIMESBOLDITALIC\_STYLE, 
FL\_MISC\_STYLE, FL\_MISCBOLD\_STYLE, FL\_MISCITALIC\_STYLE, 
FL\_SYMBOL\_STYLE, FL\_SHADOW\_STYLE, FL\_ENGRAVED\_STYLE, 
FL\_EMBOSSED\_STYLE)}

          \item[size]

          font size ({\textless}int{\textgreater})

            {\it (type=(from xfdata module) FL\_TINY\_SIZE, FL\_SMALL\_SIZE, FL\_NORMAL\_SIZE, 
FL\_MEDIUM\_SIZE, FL\_LARGE\_SIZE, FL\_HUGE\_SIZE, FL\_DEFAULT\_SIZE)}

          \item[strng]

          text ({\textless}string{\textgreater})

          \item[strglen]

          length of string ({\textless}int{\textgreater})

        \end{Ventry}

      \end{quote}

      \textbf{Return Value}
    \vspace{-1ex}

      \begin{quote}
      width ({\textless}int{\textgreater})

      {\it (type=w)}

      \end{quote}

\textbf{Example:} wid = fl\_get\_string\_width(xfdata.FL\_MISC\_STYLE, 
xfdata.FL\_MEDIUM\_SIZE, "Mystring", 8)



\textbf{Status:} Untested + NoDoc + NoDemo = NOT OK



    \end{boxedminipage}

    \label{xformslib:library:fl_get_string_dimension}
    \index{xformslib \textit{(package)}!xformslib.library \textit{(module)}!xformslib.library.fl\_get\_string\_dimension \textit{(function)}}

    \vspace{0.5ex}

\hspace{.8\funcindent}\begin{boxedminipage}{\funcwidth}

    \raggedright \textbf{fl\_get\_string\_dimension}(\textit{style}, \textit{size}, \textit{strng}, \textit{strglen})

    \vspace{-1.5ex}

    \rule{\textwidth}{0.5\fboxrule}
\setlength{\parskip}{2ex}
    Returns the width and height of a string in one call. In addition, the 
    string passed can contain embedded newline characters and the routine 
    will make proper adjustment so the values returned are (just) large 
    enough to contain the multiple lines of text.

\setlength{\parskip}{1ex}
      \textbf{Parameters}
      \vspace{-1ex}

      \begin{quote}
        \begin{Ventry}{xxxxxxx}

          \item[style]

          font style ({\textless}int{\textgreater})

            {\it (type=(from xfdata module) FL\_NORMAL\_STYLE, FL\_BOLD\_STYLE, FL\_ITALIC\_STYLE,
FL\_BOLDITALIC\_STYLE, FL\_FIXED\_STYLE, FL\_FIXEDBOLD\_STYLE, 
FL\_FIXEDITALIC\_STYLE, FL\_FIXEDBOLDITALIC\_STYLE, FL\_TIMES\_STYLE, 
FL\_TIMESBOLD\_STYLE, FL\_TIMESITALIC\_STYLE, FL\_TIMESBOLDITALIC\_STYLE, 
FL\_MISC\_STYLE, FL\_MISCBOLD\_STYLE, FL\_MISCITALIC\_STYLE, 
FL\_SYMBOL\_STYLE, FL\_SHADOW\_STYLE, FL\_ENGRAVED\_STYLE, 
FL\_EMBOSSED\_STYLE)}

          \item[size]

          font size ({\textless}int{\textgreater})

            {\it (type=(from xfdata module) FL\_TINY\_SIZE, FL\_SMALL\_SIZE, FL\_NORMAL\_SIZE, 
FL\_MEDIUM\_SIZE, FL\_LARGE\_SIZE, FL\_HUGE\_SIZE, FL\_DEFAULT\_SIZE)}

          \item[strng]

          text ({\textless}string{\textgreater})

          \item[strglen]

          length of string ({\textless}int{\textgreater})

        \end{Ventry}

      \end{quote}

      \textbf{Return Value}
    \vspace{-1ex}

      \begin{quote}
      width and height ({\textless}int{\textgreater}, 
      {\textless}int{\textgreater})

      {\it (type=w, h)}

      \end{quote}

\textbf{Example:} fl\_get\_string\_dimension(xfdata.FL\_ENGRAVED\_STYLE, 
xfdata.FL\_DEFAULT\_SIZE, "CustomString", 12)



\textbf{Attention:} API change from upstream fl\_get\_string\_dimension(fntstyle, fntsize, 
strng, strglen, w, h)



\textbf{Status:} Tested + Doc + NoDemo = OK



    \end{boxedminipage}

    \label{xformslib:library:fl_get_string_dimension}
    \index{xformslib \textit{(package)}!xformslib.library \textit{(module)}!xformslib.library.fl\_get\_string\_dimension \textit{(function)}}

    \vspace{0.5ex}

\hspace{.8\funcindent}\begin{boxedminipage}{\funcwidth}

    \raggedright \textbf{fl\_get\_string\_size}(\textit{style}, \textit{size}, \textit{strng}, \textit{strglen})

    \vspace{-1.5ex}

    \rule{\textwidth}{0.5\fboxrule}
\setlength{\parskip}{2ex}
    Returns the width and height of a string in one call. In addition, the 
    string passed can contain embedded newline characters and the routine 
    will make proper adjustment so the values returned are (just) large 
    enough to contain the multiple lines of text.

\setlength{\parskip}{1ex}
      \textbf{Parameters}
      \vspace{-1ex}

      \begin{quote}
        \begin{Ventry}{xxxxxxx}

          \item[style]

          font style ({\textless}int{\textgreater})

            {\it (type=(from xfdata module) FL\_NORMAL\_STYLE, FL\_BOLD\_STYLE, FL\_ITALIC\_STYLE,
FL\_BOLDITALIC\_STYLE, FL\_FIXED\_STYLE, FL\_FIXEDBOLD\_STYLE, 
FL\_FIXEDITALIC\_STYLE, FL\_FIXEDBOLDITALIC\_STYLE, FL\_TIMES\_STYLE, 
FL\_TIMESBOLD\_STYLE, FL\_TIMESITALIC\_STYLE, FL\_TIMESBOLDITALIC\_STYLE, 
FL\_MISC\_STYLE, FL\_MISCBOLD\_STYLE, FL\_MISCITALIC\_STYLE, 
FL\_SYMBOL\_STYLE, FL\_SHADOW\_STYLE, FL\_ENGRAVED\_STYLE, 
FL\_EMBOSSED\_STYLE)}

          \item[size]

          font size ({\textless}int{\textgreater})

            {\it (type=(from xfdata module) FL\_TINY\_SIZE, FL\_SMALL\_SIZE, FL\_NORMAL\_SIZE, 
FL\_MEDIUM\_SIZE, FL\_LARGE\_SIZE, FL\_HUGE\_SIZE, FL\_DEFAULT\_SIZE)}

          \item[strng]

          text ({\textless}string{\textgreater})

          \item[strglen]

          length of string ({\textless}int{\textgreater})

        \end{Ventry}

      \end{quote}

      \textbf{Return Value}
    \vspace{-1ex}

      \begin{quote}
      width and height ({\textless}int{\textgreater}, 
      {\textless}int{\textgreater})

      {\it (type=w, h)}

      \end{quote}

\textbf{Example:} fl\_get\_string\_dimension(xfdata.FL\_ENGRAVED\_STYLE, 
xfdata.FL\_DEFAULT\_SIZE, "CustomString", 12)



\textbf{Attention:} API change from upstream fl\_get\_string\_dimension(fntstyle, fntsize, 
strng, strglen, w, h)



\textbf{Status:} Tested + Doc + NoDemo = OK



    \end{boxedminipage}

    \label{xformslib:library:fl_get_align_xy}
    \index{xformslib \textit{(package)}!xformslib.library \textit{(module)}!xformslib.library.fl\_get\_align\_xy \textit{(function)}}

    \vspace{0.5ex}

\hspace{.8\funcindent}\begin{boxedminipage}{\funcwidth}

    \raggedright \textbf{fl\_get\_align\_xy}(\textit{align}, \textit{x}, \textit{y}, \textit{w}, \textit{h}, \textit{xsize}, \textit{ysize}, \textit{xmargin}, \textit{ymargin})

    \vspace{-1.5ex}

    \rule{\textwidth}{0.5\fboxrule}
\setlength{\parskip}{2ex}
    Obtains the position of where to draw it with a certain alignment and 
    including padding. It works regardless if the object is to be drawn 
    inside or outside of the bounding box

\setlength{\parskip}{1ex}
      \textbf{Parameters}
      \vspace{-1ex}

      \begin{quote}
        \begin{Ventry}{xxxxxxx}

          \item[align]

          alignment ({\textless}int{\textgreater})

            {\it (type=(from xfdata module) FL\_ALIGN\_CENTER, FL\_ALIGN\_TOP, FL\_ALIGN\_BOTTOM, 
FL\_ALIGN\_LEFT, FL\_ALIGN\_RIGHT, FL\_ALIGN\_LEFT\_TOP, 
FL\_ALIGN\_RIGHT\_TOP, FL\_ALIGN\_LEFT\_BOTTOM, FL\_ALIGN\_RIGHT\_BOTTOM, 
FL\_ALIGN\_INSIDE, FL\_ALIGN\_VERT)}

          \item[x]

          horizontal position of bounding box (upper-left corner) 
          ({\textless}int{\textgreater})

          \item[y]

          vertical position of bounding box (upper-left corner) 
          ({\textless}int{\textgreater})

          \item[w]

          width in coord units of bounding box 
          ({\textless}int{\textgreater})

          \item[h]

          height in coord units of bounding box 
          ({\textless}int{\textgreater})

          \item[xsize]

          width of the object to be drawn ({\textless}int{\textgreater})

          \item[ysize]

          height of the object to be drawn ({\textless}int{\textgreater})

          \item[xmargin]

          additional horizontal padding to use 
          ({\textless}int{\textgreater})

          \item[ymargin]

          additional vertical padding to use ({\textless}int{\textgreater})

        \end{Ventry}

      \end{quote}

      \textbf{Return Value}
    \vspace{-1ex}

      \begin{quote}
      horizontal and vertical position used for drawing object 
      ({\textless}int{\textgreater}, {\textless}int{\textgreater})

      {\it (type=xx, yy)}

      \end{quote}

\textbf{Example:} xpos, ypos = fl\_get\_align\_xy(xfdata.FL\_ALIGN\_CENTER, 200, 300, 110, 
30, 120, 40, 15, 15)



\textbf{Attention:} API change from XForms - upstream was fl\_get\_align\_xy(align, x, y, w, h,
xsize, ysize, xoff, yoff, xx, yy)



\textbf{Status:} Tested + Doc + NoDemo = OK



    \end{boxedminipage}

    \label{xformslib:library:fl_drw_text}
    \index{xformslib \textit{(package)}!xformslib.library \textit{(module)}!xformslib.library.fl\_drw\_text \textit{(function)}}

    \vspace{0.5ex}

\hspace{.8\funcindent}\begin{boxedminipage}{\funcwidth}

    \raggedright \textbf{fl\_drw\_text}(\textit{align}, \textit{x}, \textit{y}, \textit{w}, \textit{h}, \textit{colr}, \textit{style}, \textit{size}, \textit{txtstr})

    \vspace{-1.5ex}

    \rule{\textwidth}{0.5\fboxrule}
\setlength{\parskip}{2ex}
    Draws the text inside the bounding box according to the alignment 
    requested. It puts a padding of 5 pixels in vertical direction and 4 in
    horizontal around the text. Thus the bounding box should be 10 pixels 
    wider and 8 pixels higher than required for the text to be drawn. It 
    interprets a text string starting with the character @ differently in 
    drawing some symbols instead.

\setlength{\parskip}{1ex}
      \textbf{Parameters}
      \vspace{-1ex}

      \begin{quote}
        \begin{Ventry}{xxxxxx}

          \item[align]

          alignment of text ({\textless}int{\textgreater})

            {\it (type=(from xfdata module) FL\_ALIGN\_CENTER, FL\_ALIGN\_TOP, FL\_ALIGN\_BOTTOM, 
FL\_ALIGN\_LEFT, FL\_ALIGN\_RIGHT, FL\_ALIGN\_LEFT\_TOP, 
FL\_ALIGN\_RIGHT\_TOP, FL\_ALIGN\_LEFT\_BOTTOM, FL\_ALIGN\_RIGHT\_BOTTOM, 
FL\_ALIGN\_INSIDE, FL\_ALIGN\_VERT)}

          \item[x]

          horizontal position (upper-left corner) 
          ({\textless}int{\textgreater})

          \item[y]

          vertical position (upper-left corner) 
          ({\textless}int{\textgreater})

          \item[w]

          width in coord units ({\textless}int{\textgreater})

          \item[h]

          height in coord units ({\textless}int{\textgreater})

          \item[colr]

          color value ({\textless}long\_pos{\textgreater})

          \item[style]

          font style ({\textless}int{\textgreater})

            {\it (type=(from xfdata module) FL\_NORMAL\_STYLE, FL\_BOLD\_STYLE, FL\_ITALIC\_STYLE,
FL\_BOLDITALIC\_STYLE, FL\_FIXED\_STYLE, FL\_FIXEDBOLD\_STYLE, 
FL\_FIXEDITALIC\_STYLE, FL\_FIXEDBOLDITALIC\_STYLE, FL\_TIMES\_STYLE, 
FL\_TIMESBOLD\_STYLE, FL\_TIMESITALIC\_STYLE, FL\_TIMESBOLDITALIC\_STYLE, 
FL\_MISC\_STYLE, FL\_MISCBOLD\_STYLE, FL\_MISCITALIC\_STYLE, 
FL\_SYMBOL\_STYLE, FL\_SHADOW\_STYLE, FL\_ENGRAVED\_STYLE, 
FL\_EMBOSSED\_STYLE)}

          \item[size]

          font size ({\textless}int{\textgreater})

            {\it (type=(from xfdata module) FL\_TINY\_SIZE, FL\_SMALL\_SIZE, FL\_NORMAL\_SIZE, 
FL\_MEDIUM\_SIZE, FL\_LARGE\_SIZE, FL\_HUGE\_SIZE, FL\_DEFAULT\_SIZE)}

          \item[txtstr]

          text to draw ({\textless}string{\textgreater})

        \end{Ventry}

      \end{quote}

\textbf{Example:} fl\_drw\_text(xfdata.FL\_ALIGN\_BOTTOM, 400, 175, 150, 45, 
xfdata.FL\_GREEN, xfdata.FL\_ITALIC\_STYLE, xfdata.FL\_SMALL\_SIZE, "A Good
Old String")



\textbf{Status:} Tested + Doc + NoDemo = OK



    \end{boxedminipage}

    \label{xformslib:library:fl_drw_text_beside}
    \index{xformslib \textit{(package)}!xformslib.library \textit{(module)}!xformslib.library.fl\_drw\_text\_beside \textit{(function)}}

    \vspace{0.5ex}

\hspace{.8\funcindent}\begin{boxedminipage}{\funcwidth}

    \raggedright \textbf{fl\_drw\_text\_beside}(\textit{align}, \textit{x}, \textit{y}, \textit{w}, \textit{h}, \textit{colr}, \textit{style}, \textit{size}, \textit{txtstr})

    \vspace{-1.5ex}

    \rule{\textwidth}{0.5\fboxrule}
\setlength{\parskip}{2ex}
    Draws the text aligned outside of the box. It interprets a text string 
    starting with the character @ differently in drawing some symbols 
    instead.

\setlength{\parskip}{1ex}
      \textbf{Parameters}
      \vspace{-1ex}

      \begin{quote}
        \begin{Ventry}{xxxxxx}

          \item[align]

          alignment of text ({\textless}int{\textgreater})

            {\it (type=(from xfdata module) FL\_ALIGN\_CENTER, FL\_ALIGN\_TOP, FL\_ALIGN\_BOTTOM, 
FL\_ALIGN\_LEFT, FL\_ALIGN\_RIGHT, FL\_ALIGN\_LEFT\_TOP, 
FL\_ALIGN\_RIGHT\_TOP, FL\_ALIGN\_LEFT\_BOTTOM, FL\_ALIGN\_RIGHT\_BOTTOM, 
FL\_ALIGN\_INSIDE, FL\_ALIGN\_VERT)}

          \item[x]

          horizontal position (upper-left corner) 
          ({\textless}int{\textgreater})

          \item[y]

          vertical position (upper-left corner) 
          ({\textless}int{\textgreater})

          \item[w]

          width in coord units ({\textless}int{\textgreater})

          \item[h]

          height in coord units ({\textless}int{\textgreater})

          \item[colr]

          color value ({\textless}long\_pos{\textgreater})

          \item[style]

          font style ({\textless}int{\textgreater})

            {\it (type=(from xfdata module) FL\_NORMAL\_STYLE, FL\_BOLD\_STYLE, FL\_ITALIC\_STYLE,
FL\_BOLDITALIC\_STYLE, FL\_FIXED\_STYLE, FL\_FIXEDBOLD\_STYLE, 
FL\_FIXEDITALIC\_STYLE, FL\_FIXEDBOLDITALIC\_STYLE, FL\_TIMES\_STYLE, 
FL\_TIMESBOLD\_STYLE, FL\_TIMESITALIC\_STYLE, FL\_TIMESBOLDITALIC\_STYLE, 
FL\_MISC\_STYLE, FL\_MISCBOLD\_STYLE, FL\_MISCITALIC\_STYLE, 
FL\_SYMBOL\_STYLE, FL\_SHADOW\_STYLE, FL\_ENGRAVED\_STYLE, 
FL\_EMBOSSED\_STYLE)}

          \item[size]

          font size ({\textless}int{\textgreater})

            {\it (type=(from xfdata module) FL\_TINY\_SIZE, FL\_SMALL\_SIZE, FL\_NORMAL\_SIZE, 
FL\_MEDIUM\_SIZE, FL\_LARGE\_SIZE, FL\_HUGE\_SIZE, FL\_DEFAULT\_SIZE)}

          \item[txtstr]

          text to draw ({\textless}string{\textgreater})

        \end{Ventry}

      \end{quote}

\textbf{Example:} fl\_drw\_text\_beside(xfdata.FL\_ALIGN\_BOTTOM, 400, 175, 150, 45, 
xfdata.FL\_GREEN, xfdata.FL\_ITALIC\_STYLE, xfdata.FL\_SMALL\_SIZE, "A Good
Old String")



\textbf{Status:} Tested + Doc + NoDemo = OK



    \end{boxedminipage}

    \label{xformslib:library:fl_drw_text_cursor}
    \index{xformslib \textit{(package)}!xformslib.library \textit{(module)}!xformslib.library.fl\_drw\_text\_cursor \textit{(function)}}

    \vspace{0.5ex}

\hspace{.8\funcindent}\begin{boxedminipage}{\funcwidth}

    \raggedright \textbf{fl\_drw\_text\_cursor}(\textit{align}, \textit{x}, \textit{y}, \textit{w}, \textit{h}, \textit{colr}, \textit{style}, \textit{size}, \textit{txtstr}, \textit{curscolr}, \textit{pos})

    \vspace{-1.5ex}

    \rule{\textwidth}{0.5\fboxrule}
\setlength{\parskip}{2ex}
    Draw text and, in addition, a cursor can optionally be drawn. It does 
    no interpretation of the special character @ nor does it add padding 
    around the text.

\setlength{\parskip}{1ex}
      \textbf{Parameters}
      \vspace{-1ex}

      \begin{quote}
        \begin{Ventry}{xxxxxxxx}

          \item[align]

          alignment ({\textless}int{\textgreater})

            {\it (type=(from xfdata module) FL\_ALIGN\_CENTER, FL\_ALIGN\_TOP, FL\_ALIGN\_BOTTOM, 
FL\_ALIGN\_LEFT, FL\_ALIGN\_RIGHT, FL\_ALIGN\_LEFT\_TOP, 
FL\_ALIGN\_RIGHT\_TOP, FL\_ALIGN\_LEFT\_BOTTOM, FL\_ALIGN\_RIGHT\_BOTTOM, 
FL\_ALIGN\_INSIDE, FL\_ALIGN\_VERT)}

          \item[x]

          horizontal position (upper-left corner) 
          ({\textless}int{\textgreater})

          \item[y]

          vertical position (upper-left corner) 
          ({\textless}int{\textgreater})

          \item[w]

          width in coord units ({\textless}int{\textgreater})

          \item[h]

          height in coord units ({\textless}int{\textgreater})

          \item[colr]

          color value ({\textless}long\_pos{\textgreater})

          \item[style]

          font style ({\textless}int{\textgreater})

            {\it (type=(from xfdata module) FL\_NORMAL\_STYLE, FL\_BOLD\_STYLE, FL\_ITALIC\_STYLE,
FL\_BOLDITALIC\_STYLE, FL\_FIXED\_STYLE, FL\_FIXEDBOLD\_STYLE, 
FL\_FIXEDITALIC\_STYLE, FL\_FIXEDBOLDITALIC\_STYLE, FL\_TIMES\_STYLE, 
FL\_TIMESBOLD\_STYLE, FL\_TIMESITALIC\_STYLE, FL\_TIMESBOLDITALIC\_STYLE, 
FL\_MISC\_STYLE, FL\_MISCBOLD\_STYLE, FL\_MISCITALIC\_STYLE, 
FL\_SYMBOL\_STYLE, FL\_SHADOW\_STYLE, FL\_ENGRAVED\_STYLE, 
FL\_EMBOSSED\_STYLE)}

          \item[size]

          font size ({\textless}int{\textgreater})

            {\it (type=(from xfdata module) FL\_TINY\_SIZE, FL\_SMALL\_SIZE, FL\_NORMAL\_SIZE, 
FL\_MEDIUM\_SIZE, FL\_LARGE\_SIZE, FL\_HUGE\_SIZE, FL\_DEFAULT\_SIZE)}

          \item[txtstr]

          text to draw ({\textless}string{\textgreater})

          \item[curscolr]

          color of the cursor ({\textless}int{\textgreater})

          \item[pos]

          position which indicates the index of the character before which 
          to draw the cursor (-1 for not showing it) 
          ({\textless}int{\textgreater})

        \end{Ventry}

      \end{quote}

\textbf{Example:} fl\_drw\_text\_cursor(xfdata.FL\_ALIGN\_BOTTOM, 400, 175, 150, 45, 
xfdata.FL\_GREEN, xfdata.FL\_ITALIC\_STYLE, xfdata.FL\_SMALL\_SIZE, "A Good
Old String", xfdata.FL\_YELLOW, 7)



\textbf{Status:} Tested + Doc + NoDemo = OK



    \end{boxedminipage}

    \label{xformslib:library:fl_drw_box}
    \index{xformslib \textit{(package)}!xformslib.library \textit{(module)}!xformslib.library.fl\_drw\_box \textit{(function)}}

    \vspace{0.5ex}

\hspace{.8\funcindent}\begin{boxedminipage}{\funcwidth}

    \raggedright \textbf{fl\_drw\_box}(\textit{boxtype}, \textit{x}, \textit{y}, \textit{w}, \textit{h}, \textit{colr}, \textit{bw})

    \vspace{-1.5ex}

    \rule{\textwidth}{0.5\fboxrule}
\setlength{\parskip}{2ex}
    Draws the bounding box of an object.

\setlength{\parskip}{1ex}
      \textbf{Parameters}
      \vspace{-1ex}

      \begin{quote}
        \begin{Ventry}{xxxxxxx}

          \item[boxtype]

          type of box to draw ({\textless}int{\textgreater})

            {\it (type=(from xfdata module) FL\_NO\_BOX, FL\_UP\_BOX, FL\_DOWN\_BOX, 
FL\_BORDER\_BOX, FL\_SHADOW\_BOX, FL\_FRAME\_BOX, FL\_ROUNDED\_BOX, 
FL\_EMBOSSED\_BOX, FL\_FLAT\_BOX, FL\_RFLAT\_BOX, FL\_RSHADOW\_BOX, 
FL\_OVAL\_BOX, FL\_ROUNDED3D\_UPBOX, FL\_ROUNDED3D\_DOWNBOX, 
FL\_OVAL3D\_UPBOX, FL\_OVAL3D\_DOWNBOX, FL\_OVAL3D\_FRAMEBOX, 
FL\_OVAL3D\_EMBOSSEDBOX)}

          \item[x]

          horizontal position (upper-left corner) 
          ({\textless}int{\textgreater})

          \item[y]

          vertical position (upper-left corner) 
          ({\textless}int{\textgreater})

          \item[w]

          width in coord units ({\textless}int{\textgreater})

          \item[h]

          height in coord units ({\textless}int{\textgreater})

          \item[colr]

          color value ({\textless}long\_pos{\textgreater})

          \item[bw]

          width of the boundary ({\textless}int{\textgreater})

        \end{Ventry}

      \end{quote}

\textbf{Example:} fl\_drw\_box(xfdata.FL\_DOWN\_BOX, 700, 800, 600, 450, xfdata.FL\_INDIGO, 
3)



\textbf{Status:} Tested + Doc + NoDemo = OK



    \end{boxedminipage}

    \label{xformslib:library:fl_add_symbol}
    \index{xformslib \textit{(package)}!xformslib.library \textit{(module)}!xformslib.library.fl\_add\_symbol \textit{(function)}}

    \vspace{0.5ex}

\hspace{.8\funcindent}\begin{boxedminipage}{\funcwidth}

    \raggedright \textbf{fl\_add\_symbol}(\textit{symbname}, \textit{py\_DrawPtr}, \textit{scalable})

    \vspace{-1.5ex}

    \rule{\textwidth}{0.5\fboxrule}
\setlength{\parskip}{2ex}
    Adds a customly drawn symbol to the system which it can then use to 
    display symbols on objects that are not provided by the library.

\setlength{\parskip}{1ex}
      \textbf{Parameters}
      \vspace{-1ex}

      \begin{quote}
        \begin{Ventry}{xxxxxxxxxx}

          \item[symbname]

          name under which the symbol should be known (at most 15 
          characters), without the leading @ 
          ({\textless}string{\textgreater})

          \item[py\_DrawPtr]

          python function to draw symbol - no return

            {\it (type=\_\_ funcname (coord, coord, coord, coord, angle\_degree\_rotation, colr) 
\_\_)}

          \item[scalable]

          not used, a value of 0 will be fine 
          ({\textless}int{\textgreater})

        \end{Ventry}

      \end{quote}

      \textbf{Return Value}
    \vspace{-1ex}

      \begin{quote}
      num

      \end{quote}

\textbf{Example:}
\begin{quote}
  \begin{itemize}

  \item
    \setlength{\parskip}{0.6ex}
def drawsymb(x, y, w, h, angle, col):



  \item {\textbar}-{\textgreater}{\textbar} ...



  \item xf.fl\_add\_symbol("MySymbol", drawsymb, 0)



\end{itemize}

\end{quote}

\textbf{Status:} Tested + Doc + NoDemo = OK



    \end{boxedminipage}

    \label{xformslib:library:fl_draw_symbol}
    \index{xformslib \textit{(package)}!xformslib.library \textit{(module)}!xformslib.library.fl\_draw\_symbol \textit{(function)}}

    \vspace{0.5ex}

\hspace{.8\funcindent}\begin{boxedminipage}{\funcwidth}

    \raggedright \textbf{fl\_draw\_symbol}(\textit{symbname}, \textit{x}, \textit{y}, \textit{w}, \textit{h}, \textit{colr})

    \vspace{-1.5ex}

    \rule{\textwidth}{0.5\fboxrule}
\setlength{\parskip}{2ex}
    Draws directly a symbol on the screen.

\setlength{\parskip}{1ex}
      \textbf{Parameters}
      \vspace{-1ex}

      \begin{quote}
        \begin{Ventry}{xxxxxxxx}

          \item[symbname]

          name given to the symbol ({\textless}string{\textgreater})

          \item[x]

          horizontal position (upper-left corner) 
          ({\textless}int{\textgreater})

          \item[y]

          vertical position (upper-left corner) 
          ({\textless}int{\textgreater})

          \item[w]

          width in coord units ({\textless}int{\textgreater})

          \item[h]

          height in coord units ({\textless}int{\textgreater})

          \item[colr]

          color value ({\textless}long\_pos{\textgreater})

        \end{Ventry}

      \end{quote}

      \textbf{Return Value}
    \vspace{-1ex}

      \begin{quote}
      1 (on success) or 0 (on failure) ({\textless}int{\textgreater})

      {\it (type=num)}

      \end{quote}

\textbf{Example:} fl\_draw\_symbol("willsym", 120, 120, 15, 20, xfdata.FL\_LIGHTGRAY)



\textbf{Status:} Tested + Doc + NoDemo = OK



    \end{boxedminipage}

    \label{xformslib:library:fl_mapcolor}
    \index{xformslib \textit{(package)}!xformslib.library \textit{(module)}!xformslib.library.fl\_mapcolor \textit{(function)}}

    \vspace{0.5ex}

\hspace{.8\funcindent}\begin{boxedminipage}{\funcwidth}

    \raggedright \textbf{fl\_mapcolor}(\textit{colr}, \textit{r}, \textit{g}, \textit{b})

    \vspace{-1.5ex}

    \rule{\textwidth}{0.5\fboxrule}
\setlength{\parskip}{2ex}
    Changes the colormap and make a color index active so that it can be 
    used in various drawing routines after initialization. It maps a new 
    color using specific values for red, green and blue. In case a request 
    fails, we substitute the closest color. It is recommended that you use 
    an index larger than xfdata.FL\_FREE\_COL1 for your remap request to 
    avoid accidentally free the colors you have not explicitly allocated. 
    Indices larger than 224 are reserved and should not be used.

\setlength{\parskip}{1ex}
      \textbf{Parameters}
      \vspace{-1ex}

      \begin{quote}
        \begin{Ventry}{xxxx}

          \item[colr]

          new color value to be mapped ({\textless}long\_pos{\textgreater})

          \item[r]

          value for red ({\textless}int{\textgreater})

          \item[g]

          value for green ({\textless}int{\textgreater})

          \item[b]

          value for blue ({\textless}int{\textgreater})

        \end{Ventry}

      \end{quote}

      \textbf{Return Value}
    \vspace{-1ex}

      \begin{quote}
      color value ({\textless}long\_pos{\textgreater}) or 0 (on failure)

      {\it (type=color)}

      \end{quote}

\textbf{Example:} fl\_mapcolor(xfdata.FL\_FREE\_COL1, 100, 200, 300)



\textbf{Status:} Tested + Doc + Demo = OK



    \end{boxedminipage}

    \label{xformslib:library:fl_mapcolorname}
    \index{xformslib \textit{(package)}!xformslib.library \textit{(module)}!xformslib.library.fl\_mapcolorname \textit{(function)}}

    \vspace{0.5ex}

\hspace{.8\funcindent}\begin{boxedminipage}{\funcwidth}

    \raggedright \textbf{fl\_mapcolorname}(\textit{colr}, \textit{rgbcolrname})

    \vspace{-1.5ex}

    \rule{\textwidth}{0.5\fboxrule}
\setlength{\parskip}{2ex}
    Sets the color in the colormap indexed by colr to the specified color 
    name. It associates an index with a color name, which may have been 
    obtained via resources.

\setlength{\parskip}{1ex}
      \textbf{Parameters}
      \vspace{-1ex}

      \begin{quote}
        \begin{Ventry}{xxxxxxxxxxx}

          \item[colr]

          color value to be mapped ({\textless}long\_pos{\textgreater})

          \item[rgbcolrname]

          name of mapped color from the systems color database file 
          "rgb.txt" ({\textless}string{\textgreater})

        \end{Ventry}

      \end{quote}

      \textbf{Return Value}
    \vspace{-1ex}

      \begin{quote}
      color pixel value ({\textless}long{\textgreater}) or -1 (on failure)

      {\it (type=num)}

      \end{quote}

\textbf{Example:} fl\_mapcolorname(xfdata.FL\_FREE\_COL3, "PowderBlue")



\textbf{Status:} Tested + Doc + NoDemo = OK



    \end{boxedminipage}

    \label{xformslib:library:fl_mapcolorname}
    \index{xformslib \textit{(package)}!xformslib.library \textit{(module)}!xformslib.library.fl\_mapcolorname \textit{(function)}}

    \vspace{0.5ex}

\hspace{.8\funcindent}\begin{boxedminipage}{\funcwidth}

    \raggedright \textbf{fl\_mapcolor\_name}(\textit{colr}, \textit{rgbcolrname})

    \vspace{-1.5ex}

    \rule{\textwidth}{0.5\fboxrule}
\setlength{\parskip}{2ex}
    Sets the color in the colormap indexed by colr to the specified color 
    name. It associates an index with a color name, which may have been 
    obtained via resources.

\setlength{\parskip}{1ex}
      \textbf{Parameters}
      \vspace{-1ex}

      \begin{quote}
        \begin{Ventry}{xxxxxxxxxxx}

          \item[colr]

          color value to be mapped ({\textless}long\_pos{\textgreater})

          \item[rgbcolrname]

          name of mapped color from the systems color database file 
          "rgb.txt" ({\textless}string{\textgreater})

        \end{Ventry}

      \end{quote}

      \textbf{Return Value}
    \vspace{-1ex}

      \begin{quote}
      color pixel value ({\textless}long{\textgreater}) or -1 (on failure)

      {\it (type=num)}

      \end{quote}

\textbf{Example:} fl\_mapcolorname(xfdata.FL\_FREE\_COL3, "PowderBlue")



\textbf{Status:} Tested + Doc + NoDemo = OK



    \end{boxedminipage}

    \label{xformslib:library:fl_free_colors}
    \index{xformslib \textit{(package)}!xformslib.library \textit{(module)}!xformslib.library.fl\_free\_colors \textit{(function)}}

    \vspace{0.5ex}

\hspace{.8\funcindent}\begin{boxedminipage}{\funcwidth}

    \raggedright \textbf{fl\_free\_colors}(\textit{colr}, \textit{num})

    \vspace{-1.5ex}

    \rule{\textwidth}{0.5\fboxrule}
\setlength{\parskip}{2ex}
    Frees allocated array of colors from the default colormap, if index of 
    colors are known. You shouldn't do that for the reserved colors (i.e. 
    colors with indices below xfdata.FL\_FREE\_COL1).

\setlength{\parskip}{1ex}
      \textbf{Parameters}
      \vspace{-1ex}

      \begin{quote}
        \begin{Ventry}{xxxx}

          \item[colr]

          color value ({\textless}long\_pos{\textgreater})

          \item[num]

          number of colors stored in the array of colors 
          ({\textless}int{\textgreater})

        \end{Ventry}

      \end{quote}

\textbf{Example:} 

\textbf{Status:} Untested + NoDoc + NoDemo = NOT OK



    \end{boxedminipage}

    \label{xformslib:library:fl_free_pixels}
    \index{xformslib \textit{(package)}!xformslib.library \textit{(module)}!xformslib.library.fl\_free\_pixels \textit{(function)}}

    \vspace{0.5ex}

\hspace{.8\funcindent}\begin{boxedminipage}{\funcwidth}

    \raggedright \textbf{fl\_free\_pixels}(\textit{pix}, \textit{num})

    \vspace{-1.5ex}

    \rule{\textwidth}{0.5\fboxrule}
\setlength{\parskip}{2ex}
    Frees allocated colors from the default colormap, if pixel values are 
    known. You shouldn't do that for the reserved colors (i.e. colors with 
    indices below xfdata.FL\_FREE\_COL1).

\setlength{\parskip}{1ex}
      \textbf{Parameters}
      \vspace{-1ex}

      \begin{quote}
        \begin{Ventry}{xxx}

          \item[pix]

          pixel number ({\textless}long\_pos{\textgreater})

          \item[num]

          number of colors stored ({\textless}int{\textgreater})

        \end{Ventry}

      \end{quote}

\textbf{Example:} 

\textbf{Status:} Untested + NoDoc + NoDemo = NOT OK



    \end{boxedminipage}

    \label{xformslib:library:fl_set_color_leak}
    \index{xformslib \textit{(package)}!xformslib.library \textit{(module)}!xformslib.library.fl\_set\_color\_leak \textit{(function)}}

    \vspace{0.5ex}

\hspace{.8\funcindent}\begin{boxedminipage}{\funcwidth}

    \raggedright \textbf{fl\_set\_color\_leak}(\textit{flag})

    \vspace{-1.5ex}

    \rule{\textwidth}{0.5\fboxrule}
\setlength{\parskip}{2ex}
    Enables or disables the leakage of color.

\setlength{\parskip}{1ex}
      \textbf{Parameters}
      \vspace{-1ex}

      \begin{quote}
        \begin{Ventry}{xxxx}

          \item[flag]

          flag to enable/disable leakage of color 
          ({\textless}int{\textgreater})

            {\it (type=0 (to disable) or 1 (to enable))}

        \end{Ventry}

      \end{quote}

\textbf{Example:} fl\_set\_color\_leak(1)



\textbf{Status:} Tested + Doc + NoDemo = OK



    \end{boxedminipage}

    \label{xformslib:library:fl_getmcolor}
    \index{xformslib \textit{(package)}!xformslib.library \textit{(module)}!xformslib.library.fl\_getmcolor \textit{(function)}}

    \vspace{0.5ex}

\hspace{.8\funcindent}\begin{boxedminipage}{\funcwidth}

    \raggedright \textbf{fl\_getmcolor}(\textit{colr})

    \vspace{-1.5ex}

    \rule{\textwidth}{0.5\fboxrule}
\setlength{\parskip}{2ex}
    Obtains the RGB values of an index, returning the pixel value as known 
    by the Xserver. If yu are interested in the internal colormap of XForms
    fl\_get\_icm\_color() is more efficent.

\setlength{\parskip}{1ex}
      \textbf{Parameters}
      \vspace{-1ex}

      \begin{quote}
        \begin{Ventry}{xxxx}

          \item[colr]

          color value ({\textless}long\_pos{\textgreater})

        \end{Ventry}

      \end{quote}

      \textbf{Return Value}
    \vspace{-1ex}

      \begin{quote}
      color pixel, red, green, blue ({\textless}long\_pos{\textgreater}, 
      {\textless}int{\textgreater}, {\textless}int{\textgreater}, 
      {\textless}int{\textgreater})

      {\it (type=num., r, g, b)}

      \end{quote}

\textbf{Example:} pixl, red, green, blue = fl\_getmcolor(xfdata.FL\_VIOLET)



\textbf{Attention:} API change from XForms - upstream was fl\_getmcolor(colr, r, g, b)



\textbf{Status:} Untested + NoDoc + NoDemo = NOT OK



    \end{boxedminipage}

    \label{xformslib:library:fl_get_pixel}
    \index{xformslib \textit{(package)}!xformslib.library \textit{(module)}!xformslib.library.fl\_get\_pixel \textit{(function)}}

    \vspace{0.5ex}

\hspace{.8\funcindent}\begin{boxedminipage}{\funcwidth}

    \raggedright \textbf{fl\_get\_pixel}(\textit{colr})

    \vspace{-1.5ex}

    \rule{\textwidth}{0.5\fboxrule}
\setlength{\parskip}{2ex}
    Obtains the actual pixel value the X server understands. XForms library
    keeps an internal colormap, initialized to predefined colors. The 
    predefined colors do not correspond to pixel values the server 
    understands but are indexes into the colormap. Therefore, they can't be
    used in any of the GC altering or Xlib routines.

\setlength{\parskip}{1ex}
      \textbf{Parameters}
      \vspace{-1ex}

      \begin{quote}
        \begin{Ventry}{xxxx}

          \item[colr]

          color value ({\textless}long\_pos{\textgreater})

        \end{Ventry}

      \end{quote}

      \textbf{Return Value}
    \vspace{-1ex}

      \begin{quote}
      pixel {\textless}long\_pos{\textgreater}

      {\it (type=pixel num)}

      \end{quote}

\textbf{Example:} pixl = fl\_get\_pixel(xfdata.FL\_PEACHPUFF)



\textbf{Status:} Tested + Doc + Demo = OK



    \end{boxedminipage}

    \label{xformslib:library:fl_get_pixel}
    \index{xformslib \textit{(package)}!xformslib.library \textit{(module)}!xformslib.library.fl\_get\_pixel \textit{(function)}}

    \vspace{0.5ex}

\hspace{.8\funcindent}\begin{boxedminipage}{\funcwidth}

    \raggedright \textbf{fl\_get\_flcolor}(\textit{colr})

    \vspace{-1.5ex}

    \rule{\textwidth}{0.5\fboxrule}
\setlength{\parskip}{2ex}
    Obtains the actual pixel value the X server understands. XForms library
    keeps an internal colormap, initialized to predefined colors. The 
    predefined colors do not correspond to pixel values the server 
    understands but are indexes into the colormap. Therefore, they can't be
    used in any of the GC altering or Xlib routines.

\setlength{\parskip}{1ex}
      \textbf{Parameters}
      \vspace{-1ex}

      \begin{quote}
        \begin{Ventry}{xxxx}

          \item[colr]

          color value ({\textless}long\_pos{\textgreater})

        \end{Ventry}

      \end{quote}

      \textbf{Return Value}
    \vspace{-1ex}

      \begin{quote}
      pixel {\textless}long\_pos{\textgreater}

      {\it (type=pixel num)}

      \end{quote}

\textbf{Example:} pixl = fl\_get\_pixel(xfdata.FL\_PEACHPUFF)



\textbf{Status:} Tested + Doc + Demo = OK



    \end{boxedminipage}

    \label{xformslib:library:fl_get_icm_color}
    \index{xformslib \textit{(package)}!xformslib.library \textit{(module)}!xformslib.library.fl\_get\_icm\_color \textit{(function)}}

    \vspace{0.5ex}

\hspace{.8\funcindent}\begin{boxedminipage}{\funcwidth}

    \raggedright \textbf{fl\_get\_icm\_color}(\textit{colr})

    \vspace{-1.5ex}

    \rule{\textwidth}{0.5\fboxrule}
\setlength{\parskip}{2ex}
    Queries the internal colormap handled by XForms, returning red, green 
    and blue values corresponding to color index. Note that it does not 
    communicate with the X server, it only return information about the 
    internal colormap, which is made known to the X server by the 
    initialization routine fl\_initialize().

\setlength{\parskip}{1ex}
      \textbf{Parameters}
      \vspace{-1ex}

      \begin{quote}
        \begin{Ventry}{xxxx}

          \item[colr]

          color value ({\textless}long\_pos{\textgreater})

        \end{Ventry}

      \end{quote}

      \textbf{Return Value}
    \vspace{-1ex}

      \begin{quote}
      red, green, blue ({\textless}int{\textgreater}, 
      {\textless}int{\textgreater}, {\textless}int{\textgreater})

      {\it (type=r, g, b)}

      \end{quote}

\textbf{Example:} red, green, blue = fl\_get\_icm\_color(xfdata.FL\_OLIVE)



\textbf{Attention:} API change from XForms - upstream was fl\_get\_icm\_color(colr, r, g, b)



\textbf{Status:} Tested + Doc + NoDemo = OK



    \end{boxedminipage}

    \label{xformslib:library:fl_set_icm_color}
    \index{xformslib \textit{(package)}!xformslib.library \textit{(module)}!xformslib.library.fl\_set\_icm\_color \textit{(function)}}

    \vspace{0.5ex}

\hspace{.8\funcindent}\begin{boxedminipage}{\funcwidth}

    \raggedright \textbf{fl\_set\_icm\_color}(\textit{colr}, \textit{r}, \textit{g}, \textit{b})

    \vspace{-1.5ex}

    \rule{\textwidth}{0.5\fboxrule}
\setlength{\parskip}{2ex}
    Changes the internal colormap handled by XForms, setting a color index 
    using a red, green and blue values' combination. You have to call 
    fl\_set\_icm\_color() before fl\_initialize() to change XForms's 
    default colormap. Note that it does not communicate with the X server, 
    it only populate the internal colormap, which is made known to the X 
    server by the initialization routine fl\_initialize().

\setlength{\parskip}{1ex}
      \textbf{Parameters}
      \vspace{-1ex}

      \begin{quote}
        \begin{Ventry}{xxxx}

          \item[colr]

          color value ({\textless}long\_pos{\textgreater})

          \item[r]

          red value ({\textless}int{\textgreater})

          \item[g]

          green value ({\textless}int{\textgreater})

          \item[b]

          blue value ({\textless}int{\textgreater})

        \end{Ventry}

      \end{quote}

\textbf{Example:} fl\_set\_icm\_color(xfdata.FL\_FREE\_COL8, 75, 150, 225)



\textbf{Status:} Tested + Doc + NoDemo = OK



    \end{boxedminipage}

    \label{xformslib:library:fl_color}
    \index{xformslib \textit{(package)}!xformslib.library \textit{(module)}!xformslib.library.fl\_color \textit{(function)}}

    \vspace{0.5ex}

\hspace{.8\funcindent}\begin{boxedminipage}{\funcwidth}

    \raggedright \textbf{fl\_color}(\textit{colr})

    \vspace{-1.5ex}

    \rule{\textwidth}{0.5\fboxrule}
\setlength{\parskip}{2ex}
    Sets the foreground color in the XForms library's default GC (gc[0]).

\setlength{\parskip}{1ex}
      \textbf{Parameters}
      \vspace{-1ex}

      \begin{quote}
        \begin{Ventry}{xxxx}

          \item[colr]

          color value ({\textless}long\_pos{\textgreater})

        \end{Ventry}

      \end{quote}

\textbf{Example:} fl\_color(xfdata.FL\_ORANGE)



\textbf{Status:} Tested + Doc + NoDemo = OK



    \end{boxedminipage}

    \label{xformslib:library:fl_bk_color}
    \index{xformslib \textit{(package)}!xformslib.library \textit{(module)}!xformslib.library.fl\_bk\_color \textit{(function)}}

    \vspace{0.5ex}

\hspace{.8\funcindent}\begin{boxedminipage}{\funcwidth}

    \raggedright \textbf{fl\_bk\_color}(\textit{colr})

    \vspace{-1.5ex}

    \rule{\textwidth}{0.5\fboxrule}
\setlength{\parskip}{2ex}
    Sets the background color in the default GC (gc[0]).

\setlength{\parskip}{1ex}
      \textbf{Parameters}
      \vspace{-1ex}

      \begin{quote}
        \begin{Ventry}{xxxx}

          \item[colr]

          color value ({\textless}long\_pos{\textgreater})

        \end{Ventry}

      \end{quote}

\textbf{Example:} fl\_bk\_color(xfdata.FL\_MEDIUMORCHID)



\textbf{Status:} Tested + Doc + NoDemo = OK



    \end{boxedminipage}

    \label{xformslib:library:fl_textcolor}
    \index{xformslib \textit{(package)}!xformslib.library \textit{(module)}!xformslib.library.fl\_textcolor \textit{(function)}}

    \vspace{0.5ex}

\hspace{.8\funcindent}\begin{boxedminipage}{\funcwidth}

    \raggedright \textbf{fl\_textcolor}(\textit{colr})

    \vspace{-1.5ex}

    \rule{\textwidth}{0.5\fboxrule}
\setlength{\parskip}{2ex}
\setlength{\parskip}{1ex}
      \textbf{Parameters}
      \vspace{-1ex}

      \begin{quote}
        \begin{Ventry}{xxxx}

          \item[colr]

          color value ({\textless}long\_pos{\textgreater})

        \end{Ventry}

      \end{quote}

\textbf{Example:} fl\_textcolor(xfdata.FL\_LIGHTCORAL)



\textbf{Status:} Untested + NoDoc + NoDemo = NOT OK



    \end{boxedminipage}

    \label{xformslib:library:fl_bk_textcolor}
    \index{xformslib \textit{(package)}!xformslib.library \textit{(module)}!xformslib.library.fl\_bk\_textcolor \textit{(function)}}

    \vspace{0.5ex}

\hspace{.8\funcindent}\begin{boxedminipage}{\funcwidth}

    \raggedright \textbf{fl\_bk\_textcolor}(\textit{colr})

    \vspace{-1.5ex}

    \rule{\textwidth}{0.5\fboxrule}
\setlength{\parskip}{2ex}
\setlength{\parskip}{1ex}
      \textbf{Parameters}
      \vspace{-1ex}

      \begin{quote}
        \begin{Ventry}{xxxx}

          \item[colr]

          color value ({\textless}long\_pos{\textgreater})

        \end{Ventry}

      \end{quote}

\textbf{Example:} fl\_bk\_textcolor(xfdata.FL\_IVORY)



\textbf{Status:} Untested + NoDoc + NoDemo = NOT OK



    \end{boxedminipage}

    \label{xformslib:library:fl_set_gamma}
    \index{xformslib \textit{(package)}!xformslib.library \textit{(module)}!xformslib.library.fl\_set\_gamma \textit{(function)}}

    \vspace{0.5ex}

\hspace{.8\funcindent}\begin{boxedminipage}{\funcwidth}

    \raggedright \textbf{fl\_set\_gamma}(\textit{r}, \textit{g}, \textit{b})

    \vspace{-1.5ex}

    \rule{\textwidth}{0.5\fboxrule}
\setlength{\parskip}{2ex}
    Adjusts the brightness of the builtin colors. Larger the value, 
    brighter the colors. The default gamma is 1. This has to be called 
    before fl\_initialize().

\setlength{\parskip}{1ex}
      \textbf{Parameters}
      \vspace{-1ex}

      \begin{quote}
        \begin{Ventry}{x}

          \item[r]

          gamma value for red ({\textless}float{\textgreater})

          \item[g]

          gamma value for green ({\textless}float{\textgreater})

          \item[b]

          gamma value for blue ({\textless}float{\textgreater})

        \end{Ventry}

      \end{quote}

\textbf{Example:} fl\_set\_gamma(2.0, 2.0, 2.0)



\textbf{Status:} Tested + Doc + NoDemo = OK



    \end{boxedminipage}

    \label{xformslib:library:FL_max}
    \index{xformslib \textit{(package)}!xformslib.library \textit{(module)}!xformslib.library.FL\_max \textit{(function)}}

    \vspace{0.5ex}

\hspace{.8\funcindent}\begin{boxedminipage}{\funcwidth}

    \raggedright \textbf{FL\_max}(\textit{a}, \textit{b})

\setlength{\parskip}{2ex}
\setlength{\parskip}{1ex}
    \end{boxedminipage}

    \label{xformslib:library:FL_min}
    \index{xformslib \textit{(package)}!xformslib.library \textit{(module)}!xformslib.library.FL\_min \textit{(function)}}

    \vspace{0.5ex}

\hspace{.8\funcindent}\begin{boxedminipage}{\funcwidth}

    \raggedright \textbf{FL\_min}(\textit{a}, \textit{b})

\setlength{\parskip}{2ex}
\setlength{\parskip}{1ex}
    \end{boxedminipage}

    \label{xformslib:library:FL_abs}
    \index{xformslib \textit{(package)}!xformslib.library \textit{(module)}!xformslib.library.FL\_abs \textit{(function)}}

    \vspace{0.5ex}

\hspace{.8\funcindent}\begin{boxedminipage}{\funcwidth}

    \raggedright \textbf{FL\_abs}(\textit{a})

\setlength{\parskip}{2ex}
\setlength{\parskip}{1ex}
    \end{boxedminipage}

    \label{xformslib:library:FL_nint}
    \index{xformslib \textit{(package)}!xformslib.library \textit{(module)}!xformslib.library.FL\_nint \textit{(function)}}

    \vspace{0.5ex}

\hspace{.8\funcindent}\begin{boxedminipage}{\funcwidth}

    \raggedright \textbf{FL\_nint}(\textit{a})

\setlength{\parskip}{2ex}
\setlength{\parskip}{1ex}
    \end{boxedminipage}

    \label{xformslib:library:FL_clamp}
    \index{xformslib \textit{(package)}!xformslib.library \textit{(module)}!xformslib.library.FL\_clamp \textit{(function)}}

    \vspace{0.5ex}

\hspace{.8\funcindent}\begin{boxedminipage}{\funcwidth}

    \raggedright \textbf{FL\_clamp}(\textit{a}, \textit{amin}, \textit{amax})

\setlength{\parskip}{2ex}
\setlength{\parskip}{1ex}
    \end{boxedminipage}

    \label{xformslib:library:FL_crnd}
    \index{xformslib \textit{(package)}!xformslib.library \textit{(module)}!xformslib.library.FL\_crnd \textit{(function)}}

    \vspace{0.5ex}

\hspace{.8\funcindent}\begin{boxedminipage}{\funcwidth}

    \raggedright \textbf{FL\_crnd}(\textit{a})

\setlength{\parskip}{2ex}
\setlength{\parskip}{1ex}
    \end{boxedminipage}

    \label{xformslib:library:fl_add_object}
    \index{xformslib \textit{(package)}!xformslib.library \textit{(module)}!xformslib.library.fl\_add\_object \textit{(function)}}

    \vspace{0.5ex}

\hspace{.8\funcindent}\begin{boxedminipage}{\funcwidth}

    \raggedright \textbf{fl\_add\_object}(\textit{pForm}, \textit{pObject})

    \vspace{-1.5ex}

    \rule{\textwidth}{0.5\fboxrule}
\setlength{\parskip}{2ex}
    The object remains available (except if it's an object that marks the 
    start or end of a group) and can be added again to the same or another 
    form later. Normally, this function is used in object classes to add a 
    newly created object to a form. It may not be used for objects 
    representing the start or end of a group.

\setlength{\parskip}{1ex}
      \textbf{Parameters}
      \vspace{-1ex}

      \begin{quote}
        \begin{Ventry}{xxxxxxx}

          \item[pForm]

          form which an object will be added to ({\textless}pointer to 
          xfdata.FL\_FORM{\textgreater})

          \item[pObject]

          object to be added ({\textless}pointer to 
          xfdata.FL\_OBJECT{\textgreater})

        \end{Ventry}

      \end{quote}

\textbf{Example:} fl\_add\_object(pform2, pobjnew2)



\textbf{Status:} Tested + Doc + NoDemo = OK



    \end{boxedminipage}

    \label{xformslib:library:fl_addto_form}
    \index{xformslib \textit{(package)}!xformslib.library \textit{(module)}!xformslib.library.fl\_addto\_form \textit{(function)}}

    \vspace{0.5ex}

\hspace{.8\funcindent}\begin{boxedminipage}{\funcwidth}

    \raggedright \textbf{fl\_addto\_form}(\textit{pForm})

    \vspace{-1.5ex}

    \rule{\textwidth}{0.5\fboxrule}
\setlength{\parskip}{2ex}
    Reopens a form (after fl\_end\_form) for adding further objects to it.

\setlength{\parskip}{1ex}
      \textbf{Parameters}
      \vspace{-1ex}

      \begin{quote}
        \begin{Ventry}{xxxxx}

          \item[pForm]

          form ({\textless}pointer to xfdata.FL\_FORM{\textgreater})

        \end{Ventry}

      \end{quote}

      \textbf{Return Value}
    \vspace{-1ex}

      \begin{quote}
      form ({\textless}pointer to xfdata.FL\_FORM{\textgreater}) or None 
      (on failure)

      {\it (type=pForm)}

      \end{quote}

\textbf{Example:} form = fl\_addto\_form(closedform)



\textbf{Status:} Tested + Doc + NoDemo = OK



    \end{boxedminipage}

    \label{xformslib:library:fl_make_object}
    \index{xformslib \textit{(package)}!xformslib.library \textit{(module)}!xformslib.library.fl\_make\_object \textit{(function)}}

    \vspace{0.5ex}

\hspace{.8\funcindent}\begin{boxedminipage}{\funcwidth}

    \raggedright \textbf{fl\_make\_object}(\textit{objclass}, \textit{objtype}, \textit{x}, \textit{y}, \textit{w}, \textit{h}, \textit{label}, \textit{py\_HandlePtr})

    \vspace{-1.5ex}

    \rule{\textwidth}{0.5\fboxrule}
\setlength{\parskip}{2ex}
    Makes a custom object.

\setlength{\parskip}{1ex}
      \textbf{Parameters}
      \vspace{-1ex}

      \begin{quote}
        \begin{Ventry}{xxxxxxxxxxxx}

          \item[objclass]

          class type of object to be made ({\textless}int{\textgreater})

          \item[objtype]

          type of the object to be made ({\textless}int{\textgreater})

          \item[x]

          horizontal position of object (upper-left corner) 
          ({\textless}int{\textgreater})

          \item[y]

          vertical position of object (upper-left corner) 
          ({\textless}int{\textgreater})

          \item[w]

          width in coord units ({\textless}int{\textgreater})

          \item[h]

          height coord units ({\textless}int{\textgreater})

          \item[label]

          text label of object ({\textless}string{\textgreater})

          \item[py\_HandlePtr]

          python function for handling object, with returning value

            {\it (type=\_\_ funcname (pObject, num, coord, coord, num, ptr\_void) -{\textgreater} 
num \_\_)}

        \end{Ventry}

      \end{quote}

      \textbf{Return Value}
    \vspace{-1ex}

      \begin{quote}
      object made ({\textless}pointer to xfdata.FL\_OBJECT{\textgreater})

      {\it (type=pObject)}

      \end{quote}

\textbf{Example:}
\begin{quote}
  \begin{itemize}

  \item
    \setlength{\parskip}{0.6ex}
def handlecb(pobj, num, w, h, num, vdata):



  \item {\textbar}-{\textgreater}{\textbar} ...



  \item {\textbar}-{\textgreater}{\textbar} return 0



  \item fl\_make\_object(...)



\end{itemize}

\end{quote}

\textbf{Status:} Untested + NoDoc + NoDemo = NOT OK



    \end{boxedminipage}

    \label{xformslib:library:fl_add_child}
    \index{xformslib \textit{(package)}!xformslib.library \textit{(module)}!xformslib.library.fl\_add\_child \textit{(function)}}

    \vspace{0.5ex}

\hspace{.8\funcindent}\begin{boxedminipage}{\funcwidth}

    \raggedright \textbf{fl\_add\_child}(\textit{pObject1}, \textit{pObject2})

    \vspace{-1.5ex}

    \rule{\textwidth}{0.5\fboxrule}
\setlength{\parskip}{2ex}
\setlength{\parskip}{1ex}
      \textbf{Parameters}
      \vspace{-1ex}

      \begin{quote}
        \begin{Ventry}{xxxxxxxx}

          \item[pObject1]

          father object ({\textless}pointer to 
          xfdata.FL\_OBJECT{\textgreater})

          \item[pObject2]

          child object to add ({\textless}pointer to 
          xfdata.FL\_OBJECT{\textgreater})

        \end{Ventry}

      \end{quote}

\textbf{Example:} 

\textbf{Status:} Untested + NoDoc + NoDemo = NOT OK



    \end{boxedminipage}

    \label{xformslib:library:fl_set_coordunit}
    \index{xformslib \textit{(package)}!xformslib.library \textit{(module)}!xformslib.library.fl\_set\_coordunit \textit{(function)}}

    \vspace{0.5ex}

\hspace{.8\funcindent}\begin{boxedminipage}{\funcwidth}

    \raggedright \textbf{fl\_set\_coordunit}(\textit{unit})

    \vspace{-1.5ex}

    \rule{\textwidth}{0.5\fboxrule}
\setlength{\parskip}{2ex}
    Sets the unit for screen coordinates, instead of default ones (pixels).

\setlength{\parskip}{1ex}
      \textbf{Parameters}
      \vspace{-1ex}

      \begin{quote}
        \begin{Ventry}{xxxx}

          \item[unit]

          coord unit type to set ({\textless}int{\textgreater})

            {\it (type=(from xfdata module) FL\_COORD\_PIXEL, FL\_COORD\_MM, FL\_COORD\_POINT, 
FL\_COORD\_centiMM, FL\_COORD\_centiPOINT)}

        \end{Ventry}

      \end{quote}

\textbf{Example:} fl\_set\_coordunit(xfdata.FL\_COORD\_MM)



\textbf{Status:} Tested + Doc + Demo = OK



    \end{boxedminipage}

    \label{xformslib:library:fl_set_border_width}
    \index{xformslib \textit{(package)}!xformslib.library \textit{(module)}!xformslib.library.fl\_set\_border\_width \textit{(function)}}

    \vspace{0.5ex}

\hspace{.8\funcindent}\begin{boxedminipage}{\funcwidth}

    \raggedright \textbf{fl\_set\_border\_width}(\textit{bw})

    \vspace{-1.5ex}

    \rule{\textwidth}{0.5\fboxrule}
\setlength{\parskip}{2ex}
    Sets the width of the border.  If set to a negative number, all objects
    appear to have a softer appearance.

\setlength{\parskip}{1ex}
      \textbf{Parameters}
      \vspace{-1ex}

      \begin{quote}
        \begin{Ventry}{xx}

          \item[bw]

          value of border width ({\textless}int{\textgreater})

        \end{Ventry}

      \end{quote}

\textbf{Example:} fl\_set\_border\_width(-3)



\textbf{Status:} Tested + Doc + Demo = OK



    \end{boxedminipage}

    \label{xformslib:library:fl_set_scrollbar_type}
    \index{xformslib \textit{(package)}!xformslib.library \textit{(module)}!xformslib.library.fl\_set\_scrollbar\_type \textit{(function)}}

    \vspace{0.5ex}

\hspace{.8\funcindent}\begin{boxedminipage}{\funcwidth}

    \raggedright \textbf{fl\_set\_scrollbar\_type}(\textit{sbtype})

    \vspace{-1.5ex}

    \rule{\textwidth}{0.5\fboxrule}
\setlength{\parskip}{2ex}
    Sets the type of a scrollbar.

\setlength{\parskip}{1ex}
      \textbf{Parameters}
      \vspace{-1ex}

      \begin{quote}
        \begin{Ventry}{xxxxxx}

          \item[sbtype]

          type of scrollbar ({\textless}int{\textgreater})

            {\it (type=(from xfdata module) FL\_VERT\_SCROLLBAR, FL\_HOR\_SCROLLBAR, 
FL\_VERT\_THIN\_SCROLLBAR, FL\_HOR\_THIN\_SCROLLBAR, 
FL\_VERT\_NICE\_SCROLLBAR, FL\_HOR\_NICE\_SCROLLBAR, 
FL\_VERT\_PLAIN\_SCROLLBAR, FL\_HOR\_PLAIN\_SCROLLBAR, 
FL\_HOR\_BASIC\_SCROLLBAR, FL\_VERT\_BASIC\_SCROLLBAR, 
FL\_NORMAL\_SCROLLBAR, FL\_THIN\_SCROLLBAR, FL\_NICE\_SCROLLBAR, 
FL\_PLAIN\_SCROLLBAR)}

        \end{Ventry}

      \end{quote}

\textbf{Example:} fl\_set\_scrollbar\_type(xfdata.FL\_VERT\_BASIC\_SCROLLBAR)



\textbf{Status:} Tested + Doc + NoDemo = OK



    \end{boxedminipage}

    \label{xformslib:library:fl_set_thinscrollbar}
    \index{xformslib \textit{(package)}!xformslib.library \textit{(module)}!xformslib.library.fl\_set\_thinscrollbar \textit{(function)}}

    \vspace{0.5ex}

\hspace{.8\funcindent}\begin{boxedminipage}{\funcwidth}

    \raggedright \textbf{fl\_set\_thinscrollbar}(\textit{flag})

    \vspace{-1.5ex}

    \rule{\textwidth}{0.5\fboxrule}
\setlength{\parskip}{2ex}
    Sets if scrollbar type is thin or normal.

\setlength{\parskip}{1ex}
      \textbf{Parameters}
      \vspace{-1ex}

      \begin{quote}
        \begin{Ventry}{xxxx}

          \item[flag]

          flag if thin scrollbar or not

            {\it (type=1 (for thin) or 0 (for normal))}

        \end{Ventry}

      \end{quote}

\textbf{Example:} fl\_set\_thinscrollbar(1)



\textbf{Status:} Tested + Doc + NoDemo = OK



    \end{boxedminipage}

    \label{xformslib:library:fl_flip_yorigin}
    \index{xformslib \textit{(package)}!xformslib.library \textit{(module)}!xformslib.library.fl\_flip\_yorigin \textit{(function)}}

    \vspace{0.5ex}

\hspace{.8\funcindent}\begin{boxedminipage}{\funcwidth}

    \raggedright \textbf{fl\_flip\_yorigin}()

    \vspace{-1.5ex}

    \rule{\textwidth}{0.5\fboxrule}
\setlength{\parskip}{2ex}
    Sets the origin of XForms coordinates at the lower-left corner of the 
    form (instead of default upper-left corner). It has to be called before
    fl\_initialize().

\setlength{\parskip}{1ex}
\textbf{Example:} fl\_flip\_yorigin()



\textbf{Status:} Tested + Doc + Demo = OK



    \end{boxedminipage}

    \label{xformslib:library:fl_get_coordunit}
    \index{xformslib \textit{(package)}!xformslib.library \textit{(module)}!xformslib.library.fl\_get\_coordunit \textit{(function)}}

    \vspace{0.5ex}

\hspace{.8\funcindent}\begin{boxedminipage}{\funcwidth}

    \raggedright \textbf{fl\_get\_coordunit}()

    \vspace{-1.5ex}

    \rule{\textwidth}{0.5\fboxrule}
\setlength{\parskip}{2ex}
    Returns the unit used for screen coordinates (e.g. 
    xfdata.FL\_COORD\_MM, xfdata.FL\_COORD\_centiPOINT, etc..).

\setlength{\parskip}{1ex}
      \textbf{Return Value}
    \vspace{-1ex}

      \begin{quote}
      current coordinate unit ({\textless}int{\textgreater})

      {\it (type=coord\_unit num)}

      \end{quote}

\textbf{Example:} cunit = fl\_get\_coordunit()



\textbf{Status:} Tested + Doc + Demo = OK



    \end{boxedminipage}

    \label{xformslib:library:fl_get_border_width}
    \index{xformslib \textit{(package)}!xformslib.library \textit{(module)}!xformslib.library.fl\_get\_border\_width \textit{(function)}}

    \vspace{0.5ex}

\hspace{.8\funcindent}\begin{boxedminipage}{\funcwidth}

    \raggedright \textbf{fl\_get\_border\_width}()

    \vspace{-1.5ex}

    \rule{\textwidth}{0.5\fboxrule}
\setlength{\parskip}{2ex}
    Returns the width of border.

\setlength{\parskip}{1ex}
      \textbf{Return Value}
    \vspace{-1ex}

      \begin{quote}
      borderwidth ({\textless}int{\textgreater})

      {\it (type=bw)}

      \end{quote}

\textbf{Status:} Tested + Doc + Demo = OK



    \end{boxedminipage}

    \label{xformslib:library:fl_ringbell}
    \index{xformslib \textit{(package)}!xformslib.library \textit{(module)}!xformslib.library.fl\_ringbell \textit{(function)}}

    \vspace{0.5ex}

\hspace{.8\funcindent}\begin{boxedminipage}{\funcwidth}

    \raggedright \textbf{fl\_ringbell}(\textit{percent})

    \vspace{-1.5ex}

    \rule{\textwidth}{0.5\fboxrule}
\setlength{\parskip}{2ex}
    Sounds the keyboard ringbell (if capable). Note that not all keyboards 
    support volume variations.

\setlength{\parskip}{1ex}
      \textbf{Parameters}
      \vspace{-1ex}

      \begin{quote}
        \begin{Ventry}{xxxxxxx}

          \item[percent]

          volume value for the bell, min -100 (off), max 100, 0 is default.

        \end{Ventry}

      \end{quote}

\textbf{Example:} fl\_ringbell(50)



\textbf{Status:} Tested + Doc + NoDemo = OK



    \end{boxedminipage}

    \label{xformslib:library:fl_gettime}
    \index{xformslib \textit{(package)}!xformslib.library \textit{(module)}!xformslib.library.fl\_gettime \textit{(function)}}

    \vspace{0.5ex}

\hspace{.8\funcindent}\begin{boxedminipage}{\funcwidth}

    \raggedright \textbf{fl\_gettime}()

    \vspace{-1.5ex}

    \rule{\textwidth}{0.5\fboxrule}
\setlength{\parskip}{2ex}
    Returns the current time, expressed in seconds and microseconds since 
    1st January 1970, 00:00 GMT. It is most useful for computing time 
    differences.

\setlength{\parskip}{1ex}
      \textbf{Return Value}
    \vspace{-1ex}

      \begin{quote}
      seconds and microseconds ({\textless}long{\textgreater}, 
      {\textless}long{\textgreater})

      {\it (type=sec, usec)}

      \end{quote}

\textbf{Example:} secs, usecs = fl\_gettime()



\textbf{Attention:} API change from XForms - upstream was fl\_gettime(sec, usec)



\textbf{Status:} Tested + Doc + Demo = OK



    \end{boxedminipage}

    \label{xformslib:library:fl_now}
    \index{xformslib \textit{(package)}!xformslib.library \textit{(module)}!xformslib.library.fl\_now \textit{(function)}}

    \vspace{0.5ex}

\hspace{.8\funcindent}\begin{boxedminipage}{\funcwidth}

    \raggedright \textbf{fl\_now}()

    \vspace{-1.5ex}

    \rule{\textwidth}{0.5\fboxrule}
\setlength{\parskip}{2ex}
    Returns a string form of the current date and time. The format of the 
    string is of the form "Wed Jun 30 21:49:08 1993"

\setlength{\parskip}{1ex}
      \textbf{Return Value}
    \vspace{-1ex}

      \begin{quote}
      current date and time ({\textless}string{\textgreater})

      {\it (type=string)}

      \end{quote}

\textbf{Example:} curdattim = fl\_now()



\textbf{Status:} Tested + Doc + NoDemo = OK



    \end{boxedminipage}

    \label{xformslib:library:fl_whoami}
    \index{xformslib \textit{(package)}!xformslib.library \textit{(module)}!xformslib.library.fl\_whoami \textit{(function)}}

    \vspace{0.5ex}

\hspace{.8\funcindent}\begin{boxedminipage}{\funcwidth}

    \raggedright \textbf{fl\_whoami}()

    \vspace{-1.5ex}

    \rule{\textwidth}{0.5\fboxrule}
\setlength{\parskip}{2ex}
    Returns the user name who is running the application.

\setlength{\parskip}{1ex}
      \textbf{Return Value}
    \vspace{-1ex}

      \begin{quote}
      text of username ({\textless}string{\textgreater})

      {\it (type=string)}

      \end{quote}

\textbf{Example:} usertxt = fl\_whoami()



\textbf{Status:} Tested + Doc + NoDemo = OK



    \end{boxedminipage}

    \label{xformslib:library:fl_mouse_button}
    \index{xformslib \textit{(package)}!xformslib.library \textit{(module)}!xformslib.library.fl\_mouse\_button \textit{(function)}}

    \vspace{0.5ex}

\hspace{.8\funcindent}\begin{boxedminipage}{\funcwidth}

    \raggedright \textbf{fl\_mouse\_button}()

    \vspace{-1.5ex}

    \rule{\textwidth}{0.5\fboxrule}
\setlength{\parskip}{2ex}
    Returns which mouse button was pushed or released (e.g. 
    xfdata.FL\_RIGHT\_MOUSE, xfdata.FL\_MIDDLE\_MOUSE, etc..). Sometimes an
    application program might need to find out more information about the 
    event that triggered a callback, e.g., to implement mouse button number
    sensitive functionalities. This function, if needed, should be called 
    from within a callback. If the callback is triggered by a shortcut, the
    function returns the keysym (ascii value if ASCII) of the key plus  
    FL\_SHORTCUT. For example, if a button has a shortcut 
    {\textless}Ctrl{\textgreater}C (ASCII value is 3), the button number 
    returned upon activation of the shortcut would be xfdata.FL\_SHORTCUT +
    3. xfdata.FL\_SHORTCUT can be used to determine if the callback is 
    triggered by a shortcut or not.

\setlength{\parskip}{1ex}
      \textbf{Return Value}
    \vspace{-1ex}

      \begin{quote}
      which mouse button was pushed or released 
      ({\textless}long{\textgreater})

      {\it (type=num)}

      \end{quote}

\textbf{Example:} mousebtn = fl\_mouse\_button()



\textbf{Status:} Tested + Doc + Demo = OK



    \end{boxedminipage}

    \label{xformslib:library:fl_mouse_button}
    \index{xformslib \textit{(package)}!xformslib.library \textit{(module)}!xformslib.library.fl\_mouse\_button \textit{(function)}}

    \vspace{0.5ex}

\hspace{.8\funcindent}\begin{boxedminipage}{\funcwidth}

    \raggedright \textbf{fl\_mousebutton}()

    \vspace{-1.5ex}

    \rule{\textwidth}{0.5\fboxrule}
\setlength{\parskip}{2ex}
    Returns which mouse button was pushed or released (e.g. 
    xfdata.FL\_RIGHT\_MOUSE, xfdata.FL\_MIDDLE\_MOUSE, etc..). Sometimes an
    application program might need to find out more information about the 
    event that triggered a callback, e.g., to implement mouse button number
    sensitive functionalities. This function, if needed, should be called 
    from within a callback. If the callback is triggered by a shortcut, the
    function returns the keysym (ascii value if ASCII) of the key plus  
    FL\_SHORTCUT. For example, if a button has a shortcut 
    {\textless}Ctrl{\textgreater}C (ASCII value is 3), the button number 
    returned upon activation of the shortcut would be xfdata.FL\_SHORTCUT +
    3. xfdata.FL\_SHORTCUT can be used to determine if the callback is 
    triggered by a shortcut or not.

\setlength{\parskip}{1ex}
      \textbf{Return Value}
    \vspace{-1ex}

      \begin{quote}
      which mouse button was pushed or released 
      ({\textless}long{\textgreater})

      {\it (type=num)}

      \end{quote}

\textbf{Example:} mousebtn = fl\_mouse\_button()



\textbf{Status:} Tested + Doc + Demo = OK



    \end{boxedminipage}

    \label{xformslib:library:fl_set_err_logfp}
    \index{xformslib \textit{(package)}!xformslib.library \textit{(module)}!xformslib.library.fl\_set\_err\_logfp \textit{(function)}}

    \vspace{0.5ex}

\hspace{.8\funcindent}\begin{boxedminipage}{\funcwidth}

    \raggedright \textbf{fl\_set\_err\_logfp}(\textit{pFile})

    \vspace{-1.5ex}

    \rule{\textwidth}{0.5\fboxrule}
\setlength{\parskip}{2ex}
    The default message handler logs the error to a file instead of 
    printing to stderr.

\setlength{\parskip}{1ex}
      \textbf{Parameters}
      \vspace{-1ex}

      \begin{quote}
        \begin{Ventry}{xxxxx}

          \item[pFile]

          opened file in "w" mode ({\textless}pointer to 
          FILE{\textgreater})

        \end{Ventry}

      \end{quote}

\textbf{Example:} fl\_set\_err\_logfp(os.open("/dev/null", os.O\_WRONLY)) ??



\textbf{Status:} Untested + NoDoc + NoDemo = NOT OK (find C-like open for fd)



    \end{boxedminipage}

    \label{xformslib:library:fl_set_error_handler}
    \index{xformslib \textit{(package)}!xformslib.library \textit{(module)}!xformslib.library.fl\_set\_error\_handler \textit{(function)}}

    \vspace{0.5ex}

\hspace{.8\funcindent}\begin{boxedminipage}{\funcwidth}

    \raggedright \textbf{fl\_set\_error\_handler}(\textit{py\_ErrorFunc})

    \vspace{-1.5ex}

    \rule{\textwidth}{0.5\fboxrule}
\setlength{\parskip}{2ex}
    Normally the Forms Library reports errors to stderr. This can be 
    avoided or modified by registering an error handling function. The 
    library will call the user handler function with a string indicating in
    which function an error occured and a formatting string, followed by 
    zero or more arguments. To restore the default handler, call the 
    function again with user handler set to None. You can call this 
    function anytime and as many times as you wish.

\setlength{\parskip}{1ex}
      \textbf{Parameters}
      \vspace{-1ex}

      \begin{quote}
        \begin{Ventry}{xxxxxxxxxxxx}

          \item[py\_ErrorFunc]

          python function for handling error, no return

            {\it (type=\_\_ funcname (strng, strng) \_\_)}

        \end{Ventry}

      \end{quote}

\textbf{Example:}
\begin{quote}
  \begin{itemize}

  \item
    \setlength{\parskip}{0.6ex}
def errhandler(funcnam, errmsg):



  \item {\textbar}-{\textgreater}{\textbar} print "Error caught in \%s: \%s." \% 
(funcnam, errmsg)



  \item fl\_set\_error\_handler(errhandler)



\end{itemize}

\end{quote}

\textbf{Status:} Tested + Doc + NoDemo = OK



    \end{boxedminipage}

    \label{xformslib:library:fl_get_cmdline_args}
    \index{xformslib \textit{(package)}!xformslib.library \textit{(module)}!xformslib.library.fl\_get\_cmdline\_args \textit{(function)}}

    \vspace{0.5ex}

\hspace{.8\funcindent}\begin{boxedminipage}{\funcwidth}

    \raggedright \textbf{fl\_get\_cmdline\_args}(\textit{numargs})

    \vspace{-1.5ex}

    \rule{\textwidth}{0.5\fboxrule}
\setlength{\parskip}{2ex}
\setlength{\parskip}{1ex}
      \textbf{Return Value}
    \vspace{-1ex}

      \begin{quote}
      pointer to string

      \end{quote}

\textbf{Status:} Untested + NoDoc + NoDemo = NOT OK



    \end{boxedminipage}

    \label{xformslib:library:fl_msleep}
    \index{xformslib \textit{(package)}!xformslib.library \textit{(module)}!xformslib.library.fl\_msleep \textit{(function)}}

    \vspace{0.5ex}

\hspace{.8\funcindent}\begin{boxedminipage}{\funcwidth}

    \raggedright \textbf{fl\_msleep}(\textit{msec})

    \vspace{-1.5ex}

    \rule{\textwidth}{0.5\fboxrule}
\setlength{\parskip}{2ex}
    Waits for a number of milliseconds (with the best resolution possible 
    on your system)

\setlength{\parskip}{1ex}
      \textbf{Parameters}
      \vspace{-1ex}

      \begin{quote}
        \begin{Ventry}{xxxx}

          \item[msec]

          milliseconds to sleep

        \end{Ventry}

      \end{quote}

      \textbf{Return Value}
    \vspace{-1ex}

      \begin{quote}
      0 (on success)

      {\it (type=num)}

      \end{quote}

\textbf{Example:} fl\_msleep(200)



\textbf{Status:} Tested + Doc + Demo = OK



    \end{boxedminipage}

    \label{xformslib:library:fl_is_same_object}
    \index{xformslib \textit{(package)}!xformslib.library \textit{(module)}!xformslib.library.fl\_is\_same\_object \textit{(function)}}

    \vspace{0.5ex}

\hspace{.8\funcindent}\begin{boxedminipage}{\funcwidth}

    \raggedright \textbf{fl\_is\_same\_object}(\textit{pObject1}, \textit{pObject2})

    \vspace{-1.5ex}

    \rule{\textwidth}{0.5\fboxrule}
\setlength{\parskip}{2ex}
    Does a comparison between two objects, if they are the same, or not.

\setlength{\parskip}{1ex}
      \textbf{Parameters}
      \vspace{-1ex}

      \begin{quote}
        \begin{Ventry}{xxxxxxxx}

          \item[pObject1]

          1st object to compare ({\textless}pointer to 
          xfdata.FL\_OBJECT{\textgreater})

          \item[pObject2]

          2nd object to compare ({\textless}pointer to 
          xfdata.FL\_OBJECT{\textgreater})

        \end{Ventry}

      \end{quote}

      \textbf{Return Value}
    \vspace{-1ex}

      \begin{quote}
      0 (if they are not the same) or non-zero (if they are)

      {\it (type=num)}

      \end{quote}

\textbf{Status:} Tested + Doc + Demo = OK



    \end{boxedminipage}

    \label{xformslib:library:FL_is_gray}
    \index{xformslib \textit{(package)}!xformslib.library \textit{(module)}!xformslib.library.FL\_is\_gray \textit{(function)}}

    \vspace{0.5ex}

\hspace{.8\funcindent}\begin{boxedminipage}{\funcwidth}

    \raggedright \textbf{FL\_is\_gray}(\textit{v})

\setlength{\parskip}{2ex}
\setlength{\parskip}{1ex}
    \end{boxedminipage}

    \label{xformslib:library:FL_is_rgb}
    \index{xformslib \textit{(package)}!xformslib.library \textit{(module)}!xformslib.library.FL\_is\_rgb \textit{(function)}}

    \vspace{0.5ex}

\hspace{.8\funcindent}\begin{boxedminipage}{\funcwidth}

    \raggedright \textbf{FL\_is\_rgb}(\textit{v})

\setlength{\parskip}{2ex}
\setlength{\parskip}{1ex}
    \end{boxedminipage}

    \label{xformslib:library:fl_get_vclass}
    \index{xformslib \textit{(package)}!xformslib.library \textit{(module)}!xformslib.library.fl\_get\_vclass \textit{(function)}}

    \vspace{0.5ex}

\hspace{.8\funcindent}\begin{boxedminipage}{\funcwidth}

    \raggedright \textbf{fl\_get\_vclass}()

\setlength{\parskip}{2ex}
\setlength{\parskip}{1ex}
    \end{boxedminipage}

    \label{xformslib:library:fl_get_form_vclass}
    \index{xformslib \textit{(package)}!xformslib.library \textit{(module)}!xformslib.library.fl\_get\_form\_vclass \textit{(function)}}

    \vspace{0.5ex}

\hspace{.8\funcindent}\begin{boxedminipage}{\funcwidth}

    \raggedright \textbf{fl\_get\_form\_vclass}(\textit{a})

\setlength{\parskip}{2ex}
\setlength{\parskip}{1ex}
    \end{boxedminipage}

    \label{xformslib:library:fl_get_gc}
    \index{xformslib \textit{(package)}!xformslib.library \textit{(module)}!xformslib.library.fl\_get\_gc \textit{(function)}}

    \vspace{0.5ex}

\hspace{.8\funcindent}\begin{boxedminipage}{\funcwidth}

    \raggedright \textbf{fl\_get\_gc}()

\setlength{\parskip}{2ex}
\setlength{\parskip}{1ex}
    \end{boxedminipage}

    \label{xformslib:library:fl_mode_capable}
    \index{xformslib \textit{(package)}!xformslib.library \textit{(module)}!xformslib.library.fl\_mode\_capable \textit{(function)}}

    \vspace{0.5ex}

\hspace{.8\funcindent}\begin{boxedminipage}{\funcwidth}

    \raggedright \textbf{fl\_mode\_capable}(\textit{mode}, \textit{warn})

    \vspace{-1.5ex}

    \rule{\textwidth}{0.5\fboxrule}
\setlength{\parskip}{2ex}
    Return if the system is capable of displaying in the specified visual 
    class, or not.

\setlength{\parskip}{1ex}
      \textbf{Parameters}
      \vspace{-1ex}

      \begin{quote}
        \begin{Ventry}{xxxx}

          \item[mode]

          visual mode

            {\it (type=(from xfdata module) GrayScale, StaticGray, PseudoColor, StaticColor, 
DirectColor and TrueColor)}

          \item[warn]

          if set a warning is printed out in case the capability asked for 
          isn't available

            {\it (type=0 (don't print warning) or 1 (print warning))}

        \end{Ventry}

      \end{quote}

      \textbf{Return Value}
    \vspace{-1ex}

      \begin{quote}
      1 (if capable) or 0 otherwise ({\textless}int{\textgreater})

      {\it (type=flag)}

      \end{quote}

\textbf{Example:} capable = fl\_mode\_capable(xfdata.GrayScale, 1)



\textbf{Status:} Tested + Doc + NoDemo = OK



    \end{boxedminipage}

    \label{xformslib:library:fl_default_win}
    \index{xformslib \textit{(package)}!xformslib.library \textit{(module)}!xformslib.library.fl\_default\_win \textit{(function)}}

    \vspace{0.5ex}

\hspace{.8\funcindent}\begin{boxedminipage}{\funcwidth}

    \raggedright \textbf{fl\_default\_win}()

\setlength{\parskip}{2ex}
\setlength{\parskip}{1ex}
    \end{boxedminipage}

    \label{xformslib:library:fl_default_window}
    \index{xformslib \textit{(package)}!xformslib.library \textit{(module)}!xformslib.library.fl\_default\_window \textit{(function)}}

    \vspace{0.5ex}

\hspace{.8\funcindent}\begin{boxedminipage}{\funcwidth}

    \raggedright \textbf{fl\_default\_window}()

\setlength{\parskip}{2ex}
\setlength{\parskip}{1ex}
    \end{boxedminipage}

    \label{xformslib:library:fl_rectangle}
    \index{xformslib \textit{(package)}!xformslib.library \textit{(module)}!xformslib.library.fl\_rectangle \textit{(function)}}

    \vspace{0.5ex}

\hspace{.8\funcindent}\begin{boxedminipage}{\funcwidth}

    \raggedright \textbf{fl\_rectangle}(\textit{fill}, \textit{x}, \textit{y}, \textit{w}, \textit{h}, \textit{colr})

    \vspace{-1.5ex}

    \rule{\textwidth}{0.5\fboxrule}
\setlength{\parskip}{2ex}
    Draws a rectangle.

\setlength{\parskip}{1ex}
      \textbf{Parameters}
      \vspace{-1ex}

      \begin{quote}
        \begin{Ventry}{xxxx}

          \item[fill]

          flag if the rectangle has to be filled or just the outline is 
          needed ({\textless}int{\textgreater})

            {\it (type=0 (the outline only) or 1 (filled) ({\textless}int{\textgreater}))}

          \item[x]

          horizontal position (upper-left corner) 
          ({\textless}int{\textgreater})

          \item[y]

          vertical position (upper-left corner) 
          ({\textless}int{\textgreater})

          \item[w]

          width in coord units ({\textless}int{\textgreater})

          \item[h]

          height in coord units ({\textless}int{\textgreater})

          \item[colr]

          color value ({\textless}long\_pos{\textgreater})

        \end{Ventry}

      \end{quote}

\textbf{Example:} fl\_rectangle(1, 100, 200, 300, 200, xfdata.FL\_BEIGE)



\textbf{Status:} Tested + Doc + Demo = OK



    \end{boxedminipage}

    \label{xformslib:library:fl_rectbound}
    \index{xformslib \textit{(package)}!xformslib.library \textit{(module)}!xformslib.library.fl\_rectbound \textit{(function)}}

    \vspace{0.5ex}

\hspace{.8\funcindent}\begin{boxedminipage}{\funcwidth}

    \raggedright \textbf{fl\_rectbound}(\textit{x}, \textit{y}, \textit{w}, \textit{h}, \textit{colr})

    \vspace{-1.5ex}

    \rule{\textwidth}{0.5\fboxrule}
\setlength{\parskip}{2ex}
    Draws a filled rectangle with a black border.

\setlength{\parskip}{1ex}
      \textbf{Parameters}
      \vspace{-1ex}

      \begin{quote}
        \begin{Ventry}{xxxx}

          \item[x]

          horizontal position (upper-left corner) 
          ({\textless}int{\textgreater})

          \item[y]

          vertical position (upper-left corner) 
          ({\textless}int{\textgreater})

          \item[w]

          width in coord units ({\textless}int{\textgreater})

          \item[h]

          height in coord units ({\textless}int{\textgreater})

          \item[colr]

          color value ({\textless}long\_pos{\textgreater})

        \end{Ventry}

      \end{quote}

\textbf{Example:} fl\_rectbound(100, 200, 300, 200, xfdata.FL\_PINK)



\textbf{Status:} Tested + Doc + NoDemo = OK



    \end{boxedminipage}

    \label{xformslib:library:fl_rectf}
    \index{xformslib \textit{(package)}!xformslib.library \textit{(module)}!xformslib.library.fl\_rectf \textit{(function)}}

    \vspace{0.5ex}

\hspace{.8\funcindent}\begin{boxedminipage}{\funcwidth}

    \raggedright \textbf{fl\_rectf}(\textit{x}, \textit{y}, \textit{w}, \textit{h}, \textit{colr})

    \vspace{-1.5ex}

    \rule{\textwidth}{0.5\fboxrule}
\setlength{\parskip}{2ex}
    Draws a filled rectangle on the screen.

\setlength{\parskip}{1ex}
      \textbf{Parameters}
      \vspace{-1ex}

      \begin{quote}
        \begin{Ventry}{xxxx}

          \item[x]

          horizontal position (upper-left corner) 
          ({\textless}int{\textgreater})

          \item[y]

          vertical position (upper-left corner) 
          ({\textless}int{\textgreater})

          \item[w]

          width in coord units ({\textless}int{\textgreater})

          \item[h]

          height in coord units ({\textless}int{\textgreater})

          \item[colr]

          color value ({\textless}long\_pos{\textgreater})

        \end{Ventry}

      \end{quote}

\textbf{Example:} fl\_rectf(150, 220, 300, 200, xfdata.FL\_TOMATO)



\textbf{Status:} Tested + Doc + Demo = OK



    \end{boxedminipage}

    \label{xformslib:library:fl_rect}
    \index{xformslib \textit{(package)}!xformslib.library \textit{(module)}!xformslib.library.fl\_rect \textit{(function)}}

    \vspace{0.5ex}

\hspace{.8\funcindent}\begin{boxedminipage}{\funcwidth}

    \raggedright \textbf{fl\_rect}(\textit{x}, \textit{y}, \textit{w}, \textit{h}, \textit{colr})

    \vspace{-1.5ex}

    \rule{\textwidth}{0.5\fboxrule}
\setlength{\parskip}{2ex}
    Draws a rectangle's outline on the screen.

\setlength{\parskip}{1ex}
      \textbf{Parameters}
      \vspace{-1ex}

      \begin{quote}
        \begin{Ventry}{xxxx}

          \item[x]

          horizontal position (upper-left corner) 
          ({\textless}int{\textgreater})

          \item[y]

          vertical position (upper-left corner) 
          ({\textless}int{\textgreater})

          \item[w]

          width in coord units ({\textless}int{\textgreater})

          \item[h]

          height in coord units ({\textless}int{\textgreater})

          \item[colr]

          color value ({\textless}long\_pos{\textgreater})

        \end{Ventry}

      \end{quote}

\textbf{Example:} fl\_rect(100, 200, 300, 200, xfdata.FL\_SLATEBLUE)



\textbf{Status:} Tested + Doc + Demo = OK



    \end{boxedminipage}

    \label{xformslib:library:fl_roundrectangle}
    \index{xformslib \textit{(package)}!xformslib.library \textit{(module)}!xformslib.library.fl\_roundrectangle \textit{(function)}}

    \vspace{0.5ex}

\hspace{.8\funcindent}\begin{boxedminipage}{\funcwidth}

    \raggedright \textbf{fl\_roundrectangle}(\textit{fill}, \textit{x}, \textit{y}, \textit{w}, \textit{h}, \textit{colr})

    \vspace{-1.5ex}

    \rule{\textwidth}{0.5\fboxrule}
\setlength{\parskip}{2ex}
    Draws a rectangle with rounded corners (filled or just the outline).

\setlength{\parskip}{1ex}
      \textbf{Parameters}
      \vspace{-1ex}

      \begin{quote}
        \begin{Ventry}{xxxx}

          \item[fill]

          flag if the rectangle has to be filled or just the outline is 
          needed ({\textless}int{\textgreater})

            {\it (type=0 (the outline only) or 1 (filled) ({\textless}int{\textgreater}))}

          \item[x]

          horizontal position (upper-left corner) 
          ({\textless}int{\textgreater})

          \item[y]

          vertical position (upper-left corner) 
          ({\textless}int{\textgreater})

          \item[w]

          width in coord units ({\textless}int{\textgreater})

          \item[h]

          height in coord units ({\textless}int{\textgreater})

          \item[colr]

          color value ({\textless}long\_pos{\textgreater})

        \end{Ventry}

      \end{quote}

\textbf{Example:} fl\_roundrectangle(1, 100, 200, 300, 200, xfdata.FL\_MAGENTA)



\textbf{Status:} Tested + Doc + NoDemo = OK



    \end{boxedminipage}

    \label{xformslib:library:fl_roundrectf}
    \index{xformslib \textit{(package)}!xformslib.library \textit{(module)}!xformslib.library.fl\_roundrectf \textit{(function)}}

    \vspace{0.5ex}

\hspace{.8\funcindent}\begin{boxedminipage}{\funcwidth}

    \raggedright \textbf{fl\_roundrectf}(\textit{x}, \textit{y}, \textit{w}, \textit{h}, \textit{colr})

    \vspace{-1.5ex}

    \rule{\textwidth}{0.5\fboxrule}
\setlength{\parskip}{2ex}
    Draws a filled rectangle with rounded corners.

\setlength{\parskip}{1ex}
      \textbf{Parameters}
      \vspace{-1ex}

      \begin{quote}
        \begin{Ventry}{xxxx}

          \item[x]

          horizontal position (upper-left corner) 
          ({\textless}int{\textgreater})

          \item[y]

          vertical position (upper-left corner) 
          ({\textless}int{\textgreater})

          \item[w]

          width in coord units ({\textless}int{\textgreater})

          \item[h]

          height in coord units ({\textless}int{\textgreater})

          \item[colr]

          color value ({\textless}long\_pos{\textgreater})

        \end{Ventry}

      \end{quote}

\textbf{Example:} fl\_roundrectf(100, 200, 300, 200, xfdata.FL\_CYAN)



\textbf{Status:} Tested + Doc + NoDemo = OK



    \end{boxedminipage}

    \label{xformslib:library:fl_roundrect}
    \index{xformslib \textit{(package)}!xformslib.library \textit{(module)}!xformslib.library.fl\_roundrect \textit{(function)}}

    \vspace{0.5ex}

\hspace{.8\funcindent}\begin{boxedminipage}{\funcwidth}

    \raggedright \textbf{fl\_roundrect}(\textit{x}, \textit{y}, \textit{w}, \textit{h}, \textit{colr})

    \vspace{-1.5ex}

    \rule{\textwidth}{0.5\fboxrule}
\setlength{\parskip}{2ex}
    Draws a rectangle's outline with rounded corners.

\setlength{\parskip}{1ex}
      \textbf{Parameters}
      \vspace{-1ex}

      \begin{quote}
        \begin{Ventry}{xxxx}

          \item[x]

          horizontal position (upper-left corner) 
          ({\textless}int{\textgreater})

          \item[y]

          vertical position (upper-left corner) 
          ({\textless}int{\textgreater})

          \item[w]

          width in coord units ({\textless}int{\textgreater})

          \item[h]

          height in coord units ({\textless}int{\textgreater})

          \item[colr]

          color value ({\textless}long\_pos{\textgreater})

        \end{Ventry}

      \end{quote}

\textbf{Example:} fl\_roundrect(100, 200, 300, 200, xfdata.Fl\_INDIANRED)



\textbf{Status:} Tested + Doc + NoDemo = OK



    \end{boxedminipage}

    \label{xformslib:library:fl_polygon}
    \index{xformslib \textit{(package)}!xformslib.library \textit{(module)}!xformslib.library.fl\_polygon \textit{(function)}}

    \vspace{0.5ex}

\hspace{.8\funcindent}\begin{boxedminipage}{\funcwidth}

    \raggedright \textbf{fl\_polygon}(\textit{fill}, \textit{Point}, \textit{numpt}, \textit{colr})

    \vspace{-1.5ex}

    \rule{\textwidth}{0.5\fboxrule}
\setlength{\parskip}{2ex}
    Draws a generic polygon on the screen (filled or just an outline).

\setlength{\parskip}{1ex}
      \textbf{Parameters}
      \vspace{-1ex}

      \begin{quote}
        \begin{Ventry}{xxxxx}

          \item[fill]

          if polygon has to be filled or just an outline is needed 
          ({\textless}int{\textgreater})

            {\it (type=1 (if filled) or 0 (an outline only))}

          \item[Point]

          an array of xfc.FL\_POINT class instance

          \item[numpt]

          number of points ({\textless}int{\textgreater})

          \item[colr]

          value of color to be set ({\textless}long\_pos{\textgreater})

        \end{Ventry}

      \end{quote}

\textbf{Example:} pointmap = (xfdata.FL\_POINT * 4)() ; pointmap[0].x = 12 ; pointmap[0].y = 
32 ; pointmap[1].x = 24 ; pointmap[1].y = 100 ; pointmap[2].x = 87 ; 
pointmap[0].y = 132 ; fl\_polygon(1, pointmap, 3, xfdata.FL\_PALEGREEN)



\textbf{Status:} Tested + Doc + Demo = OK



    \end{boxedminipage}

    \label{xformslib:library:fl_polyf}
    \index{xformslib \textit{(package)}!xformslib.library \textit{(module)}!xformslib.library.fl\_polyf \textit{(function)}}

    \vspace{0.5ex}

\hspace{.8\funcindent}\begin{boxedminipage}{\funcwidth}

    \raggedright \textbf{fl\_polyf}(\textit{Point}, \textit{numpt}, \textit{colr})

    \vspace{-1.5ex}

    \rule{\textwidth}{0.5\fboxrule}
\setlength{\parskip}{2ex}
    Draws a generic filled polygon on the screen.

\setlength{\parskip}{1ex}
      \textbf{Parameters}
      \vspace{-1ex}

      \begin{quote}
        \begin{Ventry}{xxxxx}

          \item[Point]

          an array of xfc.FL\_POINT class instance

          \item[numpt]

          number of points

          \item[colr]

          value of color to be set

        \end{Ventry}

      \end{quote}

\textbf{Example:} pointmap = (xfdata.FL\_POINT * 4)() ; pointmap[0].x = 12 ; pointmap[0].y = 
32 ; pointmap[1].x = 24 ; pointmap[1].y = 100 ; pointmap[2].x = 87 ; 
pointmap[0].y = 132 ; fl\_polyf(pointmap, 3, xfdata.FL\_PALEGREEN)



\textbf{Status:} Tested + Doc + Demo = OK



    \end{boxedminipage}

    \label{xformslib:library:fl_polyl}
    \index{xformslib \textit{(package)}!xformslib.library \textit{(module)}!xformslib.library.fl\_polyl \textit{(function)}}

    \vspace{0.5ex}

\hspace{.8\funcindent}\begin{boxedminipage}{\funcwidth}

    \raggedright \textbf{fl\_polyl}(\textit{Point}, \textit{numpt}, \textit{colr})

    \vspace{-1.5ex}

    \rule{\textwidth}{0.5\fboxrule}
\setlength{\parskip}{2ex}
    Draws a generic polygon's outline on the screen.

\setlength{\parskip}{1ex}
      \textbf{Parameters}
      \vspace{-1ex}

      \begin{quote}
        \begin{Ventry}{xxxxx}

          \item[Point]

          an array of xfc.FL\_POINT class instance

          \item[numpt]

          number of points

          \item[colr]

          value of color to be set

        \end{Ventry}

      \end{quote}

\textbf{Example:} pointmap = (xfdata.FL\_POINT * 4)() ; pointmap[0].x = 12 ; pointmap[0].y = 
32 ; pointmap[1].x = 24 ; pointmap[1].y = 100 ; pointmap[2].x = 87 ; 
pointmap[0].y = 132 ; fl\_polyl(pointmap, 3, xfdata.Fl\_ORCHID)



\textbf{Status:} Tested + Doc + NoDemo = OK



    \end{boxedminipage}

    \label{xformslib:library:fl_polybound}
    \index{xformslib \textit{(package)}!xformslib.library \textit{(module)}!xformslib.library.fl\_polybound \textit{(function)}}

    \vspace{0.5ex}

\hspace{.8\funcindent}\begin{boxedminipage}{\funcwidth}

    \raggedright \textbf{fl\_polybound}(\textit{Point}, \textit{numpt}, \textit{colr})

    \vspace{-1.5ex}

    \rule{\textwidth}{0.5\fboxrule}
\setlength{\parskip}{2ex}
    Draws a generic filled polygon with a black border in the screen.

\setlength{\parskip}{1ex}
      \textbf{Parameters}
      \vspace{-1ex}

      \begin{quote}
        \begin{Ventry}{xxxxx}

          \item[Point]

          an array of xfc.FL\_POINT class instance

          \item[numpt]

          number of points

          \item[colr]

          value of color to be set

        \end{Ventry}

      \end{quote}

\textbf{Example:} pointmap = (xfdata.FL\_POINT * 4)() ; pointmap[0].x = 12 ; pointmap[0].y = 
32 ; pointmap[1].x = 24 ; pointmap[1].y = 100 ; pointmap[2].x = 87 ; 
pointmap[0].y = 132 ; fl\_polybound(pointmap, 3, xfdata.FL\_DARKGOLD)



\textbf{Status:} Tested + Doc + NoDemo = OK



    \end{boxedminipage}

    \label{xformslib:library:fl_lines}
    \index{xformslib \textit{(package)}!xformslib.library \textit{(module)}!xformslib.library.fl\_lines \textit{(function)}}

    \vspace{0.5ex}

\hspace{.8\funcindent}\begin{boxedminipage}{\funcwidth}

    \raggedright \textbf{fl\_lines}(\textit{Point}, \textit{numpt}, \textit{colr})

    \vspace{-1.5ex}

    \rule{\textwidth}{0.5\fboxrule}
\setlength{\parskip}{2ex}
    Draws connected line segments between a number of points

\setlength{\parskip}{1ex}
      \textbf{Parameters}
      \vspace{-1ex}

      \begin{quote}
        \begin{Ventry}{xxxxx}

          \item[Point]

          an array of xfc.FL\_POINT class instance

          \item[numpt]

          number of points

          \item[colr]

          value of color to be set

        \end{Ventry}

      \end{quote}

\textbf{Example:} pointmap = (xfdata.FL\_POINT * 4)() ; pointmap[0].x = 23 ; pointmap[0].y = 
12 ; pointmap[1].x = 56 ; pointmap[1].y = 34 ; pointmap[2].x = 102 ; 
pointmap[0].y = 250 ; fl\_lines(pointmap, 3, xfdata.FL\_DODGERBLUE)



\textbf{Status:} Tested + Doc + NoDemo = OK



    \end{boxedminipage}

    \label{xformslib:library:fl_line}
    \index{xformslib \textit{(package)}!xformslib.library \textit{(module)}!xformslib.library.fl\_line \textit{(function)}}

    \vspace{0.5ex}

\hspace{.8\funcindent}\begin{boxedminipage}{\funcwidth}

    \raggedright \textbf{fl\_line}(\textit{xi}, \textit{yi}, \textit{xf}, \textit{yf}, \textit{colr})

    \vspace{-1.5ex}

    \rule{\textwidth}{0.5\fboxrule}
\setlength{\parskip}{2ex}
    Connects two points with a straight line.

\setlength{\parskip}{1ex}
      \textbf{Parameters}
      \vspace{-1ex}

      \begin{quote}
        \begin{Ventry}{xxxx}

          \item[xi]

          initial horizontal position (upper-left corner) 
          ({\textless}int{\textgreater})

          \item[yi]

          initial vertical position (upper-left corner) 
          ({\textless}int{\textgreater})

          \item[xf]

          final horizontal position (upper-left corner) 
          ({\textless}int{\textgreater})

          \item[yf]

          final vertical position (upper-left corner) 
          ({\textless}int{\textgreater})

          \item[colr]

          color value ({\textless}long\_pos{\textgreater})

        \end{Ventry}

      \end{quote}

\textbf{Example:} fl\_line(100, 100, 200, 200, xfdata.FL\_ANTIQUEWHITE)



\textbf{Status:} Tested + Doc + NoDemo = OK



    \end{boxedminipage}

    \label{xformslib:library:fl_line}
    \index{xformslib \textit{(package)}!xformslib.library \textit{(module)}!xformslib.library.fl\_line \textit{(function)}}

    \vspace{0.5ex}

\hspace{.8\funcindent}\begin{boxedminipage}{\funcwidth}

    \raggedright \textbf{fl\_simple\_line}(\textit{xi}, \textit{yi}, \textit{xf}, \textit{yf}, \textit{colr})

    \vspace{-1.5ex}

    \rule{\textwidth}{0.5\fboxrule}
\setlength{\parskip}{2ex}
    Connects two points with a straight line.

\setlength{\parskip}{1ex}
      \textbf{Parameters}
      \vspace{-1ex}

      \begin{quote}
        \begin{Ventry}{xxxx}

          \item[xi]

          initial horizontal position (upper-left corner) 
          ({\textless}int{\textgreater})

          \item[yi]

          initial vertical position (upper-left corner) 
          ({\textless}int{\textgreater})

          \item[xf]

          final horizontal position (upper-left corner) 
          ({\textless}int{\textgreater})

          \item[yf]

          final vertical position (upper-left corner) 
          ({\textless}int{\textgreater})

          \item[colr]

          color value ({\textless}long\_pos{\textgreater})

        \end{Ventry}

      \end{quote}

\textbf{Example:} fl\_line(100, 100, 200, 200, xfdata.FL\_ANTIQUEWHITE)



\textbf{Status:} Tested + Doc + NoDemo = OK



    \end{boxedminipage}

    \label{xformslib:library:fl_point}
    \index{xformslib \textit{(package)}!xformslib.library \textit{(module)}!xformslib.library.fl\_point \textit{(function)}}

    \vspace{0.5ex}

\hspace{.8\funcindent}\begin{boxedminipage}{\funcwidth}

    \raggedright \textbf{fl\_point}(\textit{x}, \textit{y}, \textit{colr})

    \vspace{-1.5ex}

    \rule{\textwidth}{0.5\fboxrule}
\setlength{\parskip}{2ex}
    Draws one point on the screen.

\setlength{\parskip}{1ex}
      \textbf{Parameters}
      \vspace{-1ex}

      \begin{quote}
        \begin{Ventry}{xxxx}

          \item[x]

          horizontal position (upper-left corner) 
          ({\textless}int{\textgreater})

          \item[y]

          vertical position (upper-left corner) 
          ({\textless}int{\textgreater})

          \item[colr]

          color value ({\textless}long\_pos{\textgreater})

        \end{Ventry}

      \end{quote}

\textbf{Example:} fl\_point(75, 452, xfdata.FL\_CHARTREUSE)



\textbf{Status:} Tested + Doc + NoDemo = OK



    \end{boxedminipage}

    \label{xformslib:library:fl_points}
    \index{xformslib \textit{(package)}!xformslib.library \textit{(module)}!xformslib.library.fl\_points \textit{(function)}}

    \vspace{0.5ex}

\hspace{.8\funcindent}\begin{boxedminipage}{\funcwidth}

    \raggedright \textbf{fl\_points}(\textit{Point}, \textit{numpt}, \textit{colr})

    \vspace{-1.5ex}

    \rule{\textwidth}{0.5\fboxrule}
\setlength{\parskip}{2ex}
    Draws more than one points.

\setlength{\parskip}{1ex}
      \textbf{Parameters}
      \vspace{-1ex}

      \begin{quote}
        \begin{Ventry}{xxxxx}

          \item[Point]

          an array of xfc.FL\_POINT class instance

          \item[numpt]

          number of points ({\textless}int{\textgreater})

          \item[colr]

          value of color to be set ({\textless}long\_pos{\textgreater})

        \end{Ventry}

      \end{quote}

\textbf{Example:} pointmap = (xfdata.FL\_POINT * 3)() ; pointmap[0].x = 23 ; pointmap[0].y = 
12 ; pointmap[1].x = 56 ; pointmap[1].y = 34 ; pointmap[2].x = 102 ; 
pointmap[0].y = 250 ; fl\_points(pointmap, 3, xfdata.FL\_AZURE)



\textbf{Status:} Tested + Doc + NoDemo = OK



    \end{boxedminipage}

    \label{xformslib:library:fl_dashedlinestyle}
    \index{xformslib \textit{(package)}!xformslib.library \textit{(module)}!xformslib.library.fl\_dashedlinestyle \textit{(function)}}

    \vspace{0.5ex}

\hspace{.8\funcindent}\begin{boxedminipage}{\funcwidth}

    \raggedright \textbf{fl\_dashedlinestyle}(\textit{dash}, \textit{ndash})

    \vspace{-1.5ex}

    \rule{\textwidth}{0.5\fboxrule}
\setlength{\parskip}{2ex}
    Changes the dash pattern for xfdata.FL\_USERDASH and xfdata.FL 
    USERDOUBLEDASH. Each element of the array dash is the length of a 
    segment of the pattern in pixels. Dashed lines are drawn as alternating
    segments, each with the length of an element in dash. Thus the overall 
    length of the dash pattern, in pixels, is the sum of all elements of 
    dash. When the pattern is used up but the line to draw is longer it 
    used from the start again. You have to call this one whenever 
    xfdata.FL\_USERDASH is used to set the dash pattern, otherwise whatever
    the last pattern was, it will be used. After the sequence, the pattern 
    repeats.

\setlength{\parskip}{1ex}
      \textbf{Parameters}
      \vspace{-1ex}

      \begin{quote}
        \begin{Ventry}{xxxxx}

          \item[dash]

          sequence list of dashes to use. Use None as default dash pattern 
          ({\textless}list\_of\_int{\textgreater})

          \item[ndash]

          length of dashes list ({\textless}int{\textgreater})

        \end{Ventry}

      \end{quote}

\textbf{Example:} dashlist = [9, 3, 2, 3] fl\_dashedlinestyle(dashlist, 4)



\textbf{Status:} Tested + Doc + NoDemo = OK



    \end{boxedminipage}

    \label{xformslib:library:fl_update_display}
    \index{xformslib \textit{(package)}!xformslib.library \textit{(module)}!xformslib.library.fl\_update\_display \textit{(function)}}

    \vspace{0.5ex}

\hspace{.8\funcindent}\begin{boxedminipage}{\funcwidth}

    \raggedright \textbf{fl\_update\_display}(\textit{block})

    \vspace{-1.5ex}

    \rule{\textwidth}{0.5\fboxrule}
\setlength{\parskip}{2ex}
    Flushes properly the output buffer. It resolves the problem of the form
    being only partially redrawn, due to the two way buffering mechanism of
    Xlib, if fl\_show\_form() is followed by something that blocks (e.g., 
    waiting for a device other than X devices to come online). To be used 
    after fl\_show\_form(). For typical programs that use fl\_do\_forms() 
    or fl\_check\_forms() after fl\_show\_form(), flushing is not necessary
    as the output buffer is flushed automatically. Excessive call to 
    fl\_update\_display() degrades performance.

\setlength{\parskip}{1ex}
      \textbf{Parameters}
      \vspace{-1ex}

      \begin{quote}
        \begin{Ventry}{xxxxx}

          \item[block]

          mode of X buffer flushing ({\textless}int{\textgreater})

            {\it (type=0 (it's flushed so the drawing requests are on their way to the server) or 
1 (it's flushed and waits until all the events are received and processed 
by the server))}

        \end{Ventry}

      \end{quote}

\textbf{Example:} fl\_update\_display()



\textbf{Status:} Tested + Doc + NoDemo = OK



    \end{boxedminipage}

    \label{xformslib:library:fl_diagline}
    \index{xformslib \textit{(package)}!xformslib.library \textit{(module)}!xformslib.library.fl\_diagline \textit{(function)}}

    \vspace{0.5ex}

\hspace{.8\funcindent}\begin{boxedminipage}{\funcwidth}

    \raggedright \textbf{fl\_diagline}(\textit{x}, \textit{y}, \textit{w}, \textit{h}, \textit{colr})

    \vspace{-1.5ex}

    \rule{\textwidth}{0.5\fboxrule}
\setlength{\parskip}{2ex}
    Draws a line along the diagonal of a box (to draw a horizontal line set
    h to 1, not to 0).

\setlength{\parskip}{1ex}
      \textbf{Parameters}
      \vspace{-1ex}

      \begin{quote}
        \begin{Ventry}{xxxx}

          \item[x]

          horizontal position (upper-left corner) 
          ({\textless}int{\textgreater})

          \item[y]

          vertical position (upper-left corner) 
          ({\textless}int{\textgreater})

          \item[w]

          width in coord units ({\textless}int{\textgreater})

          \item[h]

          height in coord units ({\textless}int{\textgreater})

          \item[colr]

          color value ({\textless}long\_pos{\textgreater})

        \end{Ventry}

      \end{quote}

\textbf{Example:} fl\_diagline(180, 90, 5, 2, xfdata.FL\_BISQUE)



\textbf{Status:} Tested + Doc + NoDemo = OK



    \end{boxedminipage}

    \label{xformslib:library:fl_linewidth}
    \index{xformslib \textit{(package)}!xformslib.library \textit{(module)}!xformslib.library.fl\_linewidth \textit{(function)}}

    \vspace{0.5ex}

\hspace{.8\funcindent}\begin{boxedminipage}{\funcwidth}

    \raggedright \textbf{fl\_linewidth}(\textit{w})

    \vspace{-1.5ex}

    \rule{\textwidth}{0.5\fboxrule}
\setlength{\parskip}{2ex}
    Changes the line width.

\setlength{\parskip}{1ex}
      \textbf{Parameters}
      \vspace{-1ex}

      \begin{quote}
        \begin{Ventry}{x}

          \item[w]

          width of line in coord units. 0 to reset to the server's default 
          ({\textless}int{\textgreater})

        \end{Ventry}

      \end{quote}

\textbf{Example:} fl\_linewidth(2)



\textbf{Status:} Tested + Doc + NoDemo = OK



    \end{boxedminipage}

    \label{xformslib:library:fl_linewidth}
    \index{xformslib \textit{(package)}!xformslib.library \textit{(module)}!xformslib.library.fl\_linewidth \textit{(function)}}

    \vspace{0.5ex}

\hspace{.8\funcindent}\begin{boxedminipage}{\funcwidth}

    \raggedright \textbf{fl\_set\_linewidth}(\textit{w})

    \vspace{-1.5ex}

    \rule{\textwidth}{0.5\fboxrule}
\setlength{\parskip}{2ex}
    Changes the line width.

\setlength{\parskip}{1ex}
      \textbf{Parameters}
      \vspace{-1ex}

      \begin{quote}
        \begin{Ventry}{x}

          \item[w]

          width of line in coord units. 0 to reset to the server's default 
          ({\textless}int{\textgreater})

        \end{Ventry}

      \end{quote}

\textbf{Example:} fl\_linewidth(2)



\textbf{Status:} Tested + Doc + NoDemo = OK



    \end{boxedminipage}

    \label{xformslib:library:fl_linestyle}
    \index{xformslib \textit{(package)}!xformslib.library \textit{(module)}!xformslib.library.fl\_linestyle \textit{(function)}}

    \vspace{0.5ex}

\hspace{.8\funcindent}\begin{boxedminipage}{\funcwidth}

    \raggedright \textbf{fl\_linestyle}(\textit{linestyle})

    \vspace{-1.5ex}

    \rule{\textwidth}{0.5\fboxrule}
\setlength{\parskip}{2ex}
    Changes the line style.

\setlength{\parskip}{1ex}
      \textbf{Parameters}
      \vspace{-1ex}

      \begin{quote}
        \begin{Ventry}{xxxxxxxxx}

          \item[linestyle]

          style of the line to draw ({\textless}int{\textgreater})

            {\it (type=(from xfdata module) FL\_SOLID, FL\_USERDASH, FL\_USERDOUBLEDASH, FL\_DOT, 
FL\_DOTDASH, FL\_DASH, FL\_LONGDASH)}

        \end{Ventry}

      \end{quote}

\textbf{Example:} fl\_linestyle(xfdata.FL\_DOT)



\textbf{Status:} Tested + Doc + NoDemo = OK



    \end{boxedminipage}

    \label{xformslib:library:fl_linestyle}
    \index{xformslib \textit{(package)}!xformslib.library \textit{(module)}!xformslib.library.fl\_linestyle \textit{(function)}}

    \vspace{0.5ex}

\hspace{.8\funcindent}\begin{boxedminipage}{\funcwidth}

    \raggedright \textbf{fl\_set\_linestyle}(\textit{linestyle})

    \vspace{-1.5ex}

    \rule{\textwidth}{0.5\fboxrule}
\setlength{\parskip}{2ex}
    Changes the line style.

\setlength{\parskip}{1ex}
      \textbf{Parameters}
      \vspace{-1ex}

      \begin{quote}
        \begin{Ventry}{xxxxxxxxx}

          \item[linestyle]

          style of the line to draw ({\textless}int{\textgreater})

            {\it (type=(from xfdata module) FL\_SOLID, FL\_USERDASH, FL\_USERDOUBLEDASH, FL\_DOT, 
FL\_DOTDASH, FL\_DASH, FL\_LONGDASH)}

        \end{Ventry}

      \end{quote}

\textbf{Example:} fl\_linestyle(xfdata.FL\_DOT)



\textbf{Status:} Tested + Doc + NoDemo = OK



    \end{boxedminipage}

    \label{xformslib:library:fl_drawmode}
    \index{xformslib \textit{(package)}!xformslib.library \textit{(module)}!xformslib.library.fl\_drawmode \textit{(function)}}

    \vspace{0.5ex}

\hspace{.8\funcindent}\begin{boxedminipage}{\funcwidth}

    \raggedright \textbf{fl\_drawmode}(\textit{mode})

    \vspace{-1.5ex}

    \rule{\textwidth}{0.5\fboxrule}
\setlength{\parskip}{2ex}
    Changes the drawing mode so the destination pixel values play a role in
    the final pixel value. By default, all lines are drawn so they 
    overwrite the destination pixel values.

\setlength{\parskip}{1ex}
      \textbf{Parameters}
      \vspace{-1ex}

      \begin{quote}
        \begin{Ventry}{xxxx}

          \item[mode]

          requested mode setting ({\textless}int{\textgreater})

            {\it (type=(from xfdata module) FL\_XOR, FL\_COPY, FL\_AND)}

        \end{Ventry}

      \end{quote}

\textbf{Example:} fl\_drawmode(xfdata.FL\_AND)



\textbf{Status:} Tested + Doc + NoDemo = OK



    \end{boxedminipage}

    \label{xformslib:library:fl_get_linewidth}
    \index{xformslib \textit{(package)}!xformslib.library \textit{(module)}!xformslib.library.fl\_get\_linewidth \textit{(function)}}

    \vspace{0.5ex}

\hspace{.8\funcindent}\begin{boxedminipage}{\funcwidth}

    \raggedright \textbf{fl\_get\_linewidth}()

    \vspace{-1.5ex}

    \rule{\textwidth}{0.5\fboxrule}
\setlength{\parskip}{2ex}
    Returns the width of line.

\setlength{\parskip}{1ex}
      \textbf{Return Value}
    \vspace{-1ex}

      \begin{quote}
      line width ({\textless}int{\textgreater})

      {\it (type=w)}

      \end{quote}

\textbf{Example:} wid = fl\_get\_linewidth()



\textbf{Status:} Tested + Doc + NoDemo = OK



    \end{boxedminipage}

    \label{xformslib:library:fl_get_linestyle}
    \index{xformslib \textit{(package)}!xformslib.library \textit{(module)}!xformslib.library.fl\_get\_linestyle \textit{(function)}}

    \vspace{0.5ex}

\hspace{.8\funcindent}\begin{boxedminipage}{\funcwidth}

    \raggedright \textbf{fl\_get\_linestyle}()

    \vspace{-1.5ex}

    \rule{\textwidth}{0.5\fboxrule}
\setlength{\parskip}{2ex}
    Returns the style of line (e.g. xfdata.FL\_SOLID, xfdata.FL\_DOT, 
    etc..).

\setlength{\parskip}{1ex}
      \textbf{Return Value}
    \vspace{-1ex}

      \begin{quote}
      line style ({\textless}int{\textgreater})

      {\it (type=style)}

      \end{quote}

\textbf{Example:} currstl = fl\_get\_linestyle()



\textbf{Status:} Tested + Doc + NoDemo = OK



    \end{boxedminipage}

    \label{xformslib:library:fl_get_drawmode}
    \index{xformslib \textit{(package)}!xformslib.library \textit{(module)}!xformslib.library.fl\_get\_drawmode \textit{(function)}}

    \vspace{0.5ex}

\hspace{.8\funcindent}\begin{boxedminipage}{\funcwidth}

    \raggedright \textbf{fl\_get\_drawmode}()

    \vspace{-1.5ex}

    \rule{\textwidth}{0.5\fboxrule}
\setlength{\parskip}{2ex}
    Returns the drawing mode of lines (e.g. xfdata.FL\_AND, xfdata.FL\_XOR 
    etc..).

\setlength{\parskip}{1ex}
      \textbf{Return Value}
    \vspace{-1ex}

      \begin{quote}
      drawing mode ({\textless}int{\textgreater})

      {\it (type=mode)}

      \end{quote}

\textbf{Example:} currdrw = fl\_get\_draw\_mode()



\textbf{Status:} Tested + Doc + NoDemo = OK



    \end{boxedminipage}

    \label{xformslib:library:fl_drawmode}
    \index{xformslib \textit{(package)}!xformslib.library \textit{(module)}!xformslib.library.fl\_drawmode \textit{(function)}}

    \vspace{0.5ex}

\hspace{.8\funcindent}\begin{boxedminipage}{\funcwidth}

    \raggedright \textbf{fl\_set\_drawmode}(\textit{mode})

    \vspace{-1.5ex}

    \rule{\textwidth}{0.5\fboxrule}
\setlength{\parskip}{2ex}
    Changes the drawing mode so the destination pixel values play a role in
    the final pixel value. By default, all lines are drawn so they 
    overwrite the destination pixel values.

\setlength{\parskip}{1ex}
      \textbf{Parameters}
      \vspace{-1ex}

      \begin{quote}
        \begin{Ventry}{xxxx}

          \item[mode]

          requested mode setting ({\textless}int{\textgreater})

            {\it (type=(from xfdata module) FL\_XOR, FL\_COPY, FL\_AND)}

        \end{Ventry}

      \end{quote}

\textbf{Example:} fl\_drawmode(xfdata.FL\_AND)



\textbf{Status:} Tested + Doc + NoDemo = OK



    \end{boxedminipage}

    \label{xformslib:library:fl_oval}
    \index{xformslib \textit{(package)}!xformslib.library \textit{(module)}!xformslib.library.fl\_oval \textit{(function)}}

    \vspace{0.5ex}

\hspace{.8\funcindent}\begin{boxedminipage}{\funcwidth}

    \raggedright \textbf{fl\_oval}(\textit{fill}, \textit{x}, \textit{y}, \textit{w}, \textit{h}, \textit{colr})

    \vspace{-1.5ex}

    \rule{\textwidth}{0.5\fboxrule}
\setlength{\parskip}{2ex}
    Draws an ellipse, either filled or open. Use w equal to h to get a 
    circle.

\setlength{\parskip}{1ex}
      \textbf{Parameters}
      \vspace{-1ex}

      \begin{quote}
        \begin{Ventry}{xxxx}

          \item[fill]

          flag if filled or open ellipse ({\textless}int{\textgreater})

            {\it (type=1 (if filled ellipse) or 0 (if open))}

          \item[x]

          horizontal position (upper-left corner) 
          ({\textless}int{\textgreater})

          \item[y]

          vertical position (upper-left corner) 
          ({\textless}int{\textgreater})

          \item[w]

          width in coord units ({\textless}int{\textgreater})

          \item[h]

          height in coord units ({\textless}int{\textgreater})

          \item[colr]

          color value ({\textless}long\_pos{\textgreater})

        \end{Ventry}

      \end{quote}

\textbf{Example:} fl\_oval(1, 125, 256, 145, 320, xfdata.FL\_BURLYWOOD)



\textbf{Status:} Tested + Doc + NoDemo = OK



    \end{boxedminipage}

    \label{xformslib:library:fl_ovalbound}
    \index{xformslib \textit{(package)}!xformslib.library \textit{(module)}!xformslib.library.fl\_ovalbound \textit{(function)}}

    \vspace{0.5ex}

\hspace{.8\funcindent}\begin{boxedminipage}{\funcwidth}

    \raggedright \textbf{fl\_ovalbound}(\textit{x}, \textit{y}, \textit{w}, \textit{h}, \textit{colr})

    \vspace{-1.5ex}

    \rule{\textwidth}{0.5\fboxrule}
\setlength{\parskip}{2ex}
    Draws a filled ellipse with a black outline. Use w equal to h to get a 
    circle.

\setlength{\parskip}{1ex}
      \textbf{Parameters}
      \vspace{-1ex}

      \begin{quote}
        \begin{Ventry}{xxxx}

          \item[x]

          horizontal position (upper-left corner) 
          ({\textless}int{\textgreater})

          \item[y]

          vertical position (upper-left corner) 
          ({\textless}int{\textgreater})

          \item[w]

          width in coord units ({\textless}int{\textgreater})

          \item[h]

          height in coord units ({\textless}int{\textgreater})

          \item[colr]

          color value ({\textless}long\_pos{\textgreater})

        \end{Ventry}

      \end{quote}

\textbf{Example:} fl\_ovalbound(1, 125, 256, 145, 320, xfdata.FL\_BLANCHEDALMOND)



\textbf{Status:} Tested + Doc + NoDemo = OK



    \end{boxedminipage}

    \label{xformslib:library:fl_ovalarc}
    \index{xformslib \textit{(package)}!xformslib.library \textit{(module)}!xformslib.library.fl\_ovalarc \textit{(function)}}

    \vspace{0.5ex}

\hspace{.8\funcindent}\begin{boxedminipage}{\funcwidth}

    \raggedright \textbf{fl\_ovalarc}(\textit{fill}, \textit{x}, \textit{y}, \textit{w}, \textit{h}, \textit{stheta}, \textit{dtheta}, \textit{colr})

    \vspace{-1.5ex}

    \rule{\textwidth}{0.5\fboxrule}
\setlength{\parskip}{2ex}
    Draws an elliptical arc, either filled or open.

\setlength{\parskip}{1ex}
      \textbf{Parameters}
      \vspace{-1ex}

      \begin{quote}
        \begin{Ventry}{xxxxxx}

          \item[fill]

          flag if filled or open ({\textless}int{\textgreater})

            {\it (type=1 (if filled) or 0 (if open))}

          \item[x]

          horizontal position (upper-left corner) 
          ({\textless}int{\textgreater})

          \item[y]

          vertical position (upper-left corner) 
          ({\textless}int{\textgreater})

          \item[w]

          width in coord units ({\textless}int{\textgreater})

          \item[h]

          height in coord units ({\textless}int{\textgreater})

          \item[stheta]

          starting angle, measured in tenth of a degree and with 0 at 3 
          o'clock position ({\textless}int{\textgreater})

          \item[dtheta]

          the directione and the extent of the arc. If positive the arc is 
          drawn in counter-clockwise direction from the starting point, 
          otherwise in clockwise direction. If it is larger than 3600 it is
          truncated to 3600.

          \item[colr]

          color value ({\textless}long\_pos{\textgreater})

        \end{Ventry}

      \end{quote}

\textbf{Example:} fl\_ovalarc(1, 275, 256, 145, 320, 200, 900, xfdata.FL\_DARKSALMON)



\textbf{Status:} Tested + Doc + NoDemo = OK



    \end{boxedminipage}

    \label{xformslib:library:fl_ovalf}
    \index{xformslib \textit{(package)}!xformslib.library \textit{(module)}!xformslib.library.fl\_ovalf \textit{(function)}}

    \vspace{0.5ex}

\hspace{.8\funcindent}\begin{boxedminipage}{\funcwidth}

    \raggedright \textbf{fl\_ovalf}(\textit{x}, \textit{y}, \textit{w}, \textit{h}, \textit{colr})

    \vspace{-1.5ex}

    \rule{\textwidth}{0.5\fboxrule}
\setlength{\parskip}{2ex}
    Draws a filled ellipse. Use w equal to h to get a circle.

\setlength{\parskip}{1ex}
      \textbf{Parameters}
      \vspace{-1ex}

      \begin{quote}
        \begin{Ventry}{xxxx}

          \item[x]

          horizontal position (upper-left corner) 
          ({\textless}int{\textgreater})

          \item[y]

          vertical position (upper-left corner) 
          ({\textless}int{\textgreater})

          \item[w]

          width in coord units ({\textless}int{\textgreater})

          \item[h]

          height in coord units ({\textless}int{\textgreater})

          \item[colr]

          color value ({\textless}long\_pos{\textgreater})

        \end{Ventry}

      \end{quote}

\textbf{Example:} fl\_ovalf(125, 256, 145, 320, xfdata.FL\_CORNFLOWERBLUE)



\textbf{Status:} Tested + Doc + NoDemo = OK



    \end{boxedminipage}

    \label{xformslib:library:fl_ovall}
    \index{xformslib \textit{(package)}!xformslib.library \textit{(module)}!xformslib.library.fl\_ovall \textit{(function)}}

    \vspace{0.5ex}

\hspace{.8\funcindent}\begin{boxedminipage}{\funcwidth}

    \raggedright \textbf{fl\_ovall}(\textit{x}, \textit{y}, \textit{w}, \textit{h}, \textit{colr})

    \vspace{-1.5ex}

    \rule{\textwidth}{0.5\fboxrule}
\setlength{\parskip}{2ex}
    Draws an open ellipse. Use w equal to h to get a circle.

\setlength{\parskip}{1ex}
      \textbf{Parameters}
      \vspace{-1ex}

      \begin{quote}
        \begin{Ventry}{xxxx}

          \item[x]

          horizontal position (upper-left corner) 
          ({\textless}int{\textgreater})

          \item[y]

          vertical position (upper-left corner) 
          ({\textless}int{\textgreater})

          \item[w]

          width in coord units ({\textless}int{\textgreater})

          \item[h]

          height in coord units ({\textless}int{\textgreater})

          \item[colr]

          color value ({\textless}long\_pos{\textgreater})

        \end{Ventry}

      \end{quote}

\textbf{Example:} fl\_ovall(125, 256, 145, 320, xfdata.FL\_DARKERED)



\textbf{Status:} Tested + Doc + NoDemo = OK



    \end{boxedminipage}

    \label{xformslib:library:fl_ovalbound}
    \index{xformslib \textit{(package)}!xformslib.library \textit{(module)}!xformslib.library.fl\_ovalbound \textit{(function)}}

    \vspace{0.5ex}

\hspace{.8\funcindent}\begin{boxedminipage}{\funcwidth}

    \raggedright \textbf{fl\_oval\_bound}(\textit{x}, \textit{y}, \textit{w}, \textit{h}, \textit{colr})

    \vspace{-1.5ex}

    \rule{\textwidth}{0.5\fboxrule}
\setlength{\parskip}{2ex}
    Draws a filled ellipse with a black outline. Use w equal to h to get a 
    circle.

\setlength{\parskip}{1ex}
      \textbf{Parameters}
      \vspace{-1ex}

      \begin{quote}
        \begin{Ventry}{xxxx}

          \item[x]

          horizontal position (upper-left corner) 
          ({\textless}int{\textgreater})

          \item[y]

          vertical position (upper-left corner) 
          ({\textless}int{\textgreater})

          \item[w]

          width in coord units ({\textless}int{\textgreater})

          \item[h]

          height in coord units ({\textless}int{\textgreater})

          \item[colr]

          color value ({\textless}long\_pos{\textgreater})

        \end{Ventry}

      \end{quote}

\textbf{Example:} fl\_ovalbound(1, 125, 256, 145, 320, xfdata.FL\_BLANCHEDALMOND)



\textbf{Status:} Tested + Doc + NoDemo = OK



    \end{boxedminipage}

    \label{xformslib:library:fl_circf}
    \index{xformslib \textit{(package)}!xformslib.library \textit{(module)}!xformslib.library.fl\_circf \textit{(function)}}

    \vspace{0.5ex}

\hspace{.8\funcindent}\begin{boxedminipage}{\funcwidth}

    \raggedright \textbf{fl\_circf}(\textit{x}, \textit{y}, \textit{r}, \textit{colr})

    \vspace{-1.5ex}

    \rule{\textwidth}{0.5\fboxrule}
\setlength{\parskip}{2ex}
    Draws a filled circle.

\setlength{\parskip}{1ex}
      \textbf{Parameters}
      \vspace{-1ex}

      \begin{quote}
        \begin{Ventry}{x}

          \item[x]

          horizontal position of the center of the arc 
          ({\textless}int{\textgreater})

          \item[y]

          vertical position of the center of the arc 
          ({\textless}int{\textgreater})

          \item[r]

          radius of the arc ({\textless}int{\textgreater})

        \end{Ventry}

      \end{quote}

\textbf{Example:} fl\_circf(200, 250, 69, xfdata.FL\_FUCHSIA)



\textbf{Status:} Tested + NoDoc + Demo = OK



    \end{boxedminipage}

    \label{xformslib:library:fl_circ}
    \index{xformslib \textit{(package)}!xformslib.library \textit{(module)}!xformslib.library.fl\_circ \textit{(function)}}

    \vspace{0.5ex}

\hspace{.8\funcindent}\begin{boxedminipage}{\funcwidth}

    \raggedright \textbf{fl\_circ}(\textit{x}, \textit{y}, \textit{r}, \textit{colr})

    \vspace{-1.5ex}

    \rule{\textwidth}{0.5\fboxrule}
\setlength{\parskip}{2ex}
    Draws an open circle.

\setlength{\parskip}{1ex}
      \textbf{Parameters}
      \vspace{-1ex}

      \begin{quote}
        \begin{Ventry}{x}

          \item[x]

          horizontal position of the center of the arc 
          ({\textless}int{\textgreater})

          \item[y]

          vertical position of the center of the arc 
          ({\textless}int{\textgreater})

          \item[r]

          radius of the arc ({\textless}int{\textgreater})

        \end{Ventry}

      \end{quote}

\textbf{Example:} fl\_circ(200, 250, 69, xfdata.FL\_GAINSBORO)



\textbf{Status:} Tested + Doc + NoDemo = OK



    \end{boxedminipage}

    \label{xformslib:library:fl_pieslice}
    \index{xformslib \textit{(package)}!xformslib.library \textit{(module)}!xformslib.library.fl\_pieslice \textit{(function)}}

    \vspace{0.5ex}

\hspace{.8\funcindent}\begin{boxedminipage}{\funcwidth}

    \raggedright \textbf{fl\_pieslice}(\textit{fill}, \textit{x}, \textit{y}, \textit{w}, \textit{h}, \textit{stheta}, \textit{etheta}, \textit{colr})

    \vspace{-1.5ex}

    \rule{\textwidth}{0.5\fboxrule}
\setlength{\parskip}{2ex}
    fl\_pieslice(fill, x, y, w, h, sthetam etheta, colr)

    Draws an elliptical arc, either filled or open.

\setlength{\parskip}{1ex}
      \textbf{Parameters}
      \vspace{-1ex}

      \begin{quote}
        \begin{Ventry}{xxxxxx}

          \item[fill]

          if the arc is filled or open ({\textless}int{\textgreater})

            {\it (type=1 (if filled) or 0 (if open))}

          \item[x]

          horizontal position of the bounding box 
          ({\textless}int{\textgreater})

          \item[y]

          vertical position of the bounding box 
          ({\textless}int{\textgreater})

          \item[h]

          horizontal axe of the ellipse ({\textless}int{\textgreater})

          \item[w]

          vertical axe of the ellipse ({\textless}int{\textgreater})

          \item[stheta]

          starting angle of the arc in units of tenths of a degree (where 0
          stands for a direction of 3 o'clock, i.e. the right-most point of
          a circle) ({\textless}int{\textgreater})

          \item[etheta]

          ending angle of the arc in units of tenths of a degree (where 0 
          stands for a direction of 3 o'clock, i.e. the right-most point of
          a circle) ({\textless}int{\textgreater})

          \item[colr]

          color value ({\textless}long\_pos{\textgreater})

        \end{Ventry}

      \end{quote}

\textbf{Example:} fl\_pieslice(1, 120, 253, 400, 100, 60, 70, xfdata.FL\_GOLD)



\textbf{Status:} Tested + Doc + NoDemo = OK



    \end{boxedminipage}

    \label{xformslib:library:fl_arcf}
    \index{xformslib \textit{(package)}!xformslib.library \textit{(module)}!xformslib.library.fl\_arcf \textit{(function)}}

    \vspace{0.5ex}

\hspace{.8\funcindent}\begin{boxedminipage}{\funcwidth}

    \raggedright \textbf{fl\_arcf}(\textit{x}, \textit{y}, \textit{r}, \textit{stheta}, \textit{etheta}, \textit{colr})

    \vspace{-1.5ex}

    \rule{\textwidth}{0.5\fboxrule}
\setlength{\parskip}{2ex}
    Draws a filled circular arc. If the difference between theta end and 
    theta start is larger than 3600 (360 degrees), drawing is truncated to 
    360 degrees.

\setlength{\parskip}{1ex}
      \textbf{Parameters}
      \vspace{-1ex}

      \begin{quote}
        \begin{Ventry}{xxxxxx}

          \item[x]

          horizontal position of the center of the arc 
          ({\textless}int{\textgreater})

          \item[y]

          vertical position of the center of the arc 
          ({\textless}int{\textgreater})

          \item[r]

          radius of the arc ({\textless}int{\textgreater})

          \item[stheta]

          starting angle of the arc in units of tenths of a degree (where 0
          stands for a direction of 3 o'clock, i.e. the right-most point of
          a circle) ({\textless}int{\textgreater})

          \item[etheta]

          ending angle of the arc in units of tenths of a degree (where 0 
          stands for a direction of 3 o'clock, i.e. the right-most point of
          a circle) ({\textless}int{\textgreater})

          \item[colr]

          color value ({\textless}long\_pos{\textgreater})

        \end{Ventry}

      \end{quote}

\textbf{Example:} fl\_arcf(120, 253, 40, 10, 60, xfdata.FL\_FIREBRICK)



\textbf{Status:} Tested + Doc + NoDemo = OK



    \end{boxedminipage}

    \label{xformslib:library:fl_arc}
    \index{xformslib \textit{(package)}!xformslib.library \textit{(module)}!xformslib.library.fl\_arc \textit{(function)}}

    \vspace{0.5ex}

\hspace{.8\funcindent}\begin{boxedminipage}{\funcwidth}

    \raggedright \textbf{fl\_arc}(\textit{x}, \textit{y}, \textit{r}, \textit{stheta}, \textit{etheta}, \textit{colr})

    \vspace{-1.5ex}

    \rule{\textwidth}{0.5\fboxrule}
\setlength{\parskip}{2ex}
    Draws an open circular arc. If the difference between theta end and 
    theta start is larger than 3600 (360 degrees), drawing is truncated to 
    360 degrees.

\setlength{\parskip}{1ex}
      \textbf{Parameters}
      \vspace{-1ex}

      \begin{quote}
        \begin{Ventry}{xxxxxx}

          \item[x]

          horizontal position of the center of the arc 
          ({\textless}int{\textgreater})

          \item[y]

          vertical position of the center of the arc 
          ({\textless}int{\textgreater})

          \item[r]

          radius of the arc ({\textless}int{\textgreater})

          \item[stheta]

          starting angle of the arc in units of tenths of a degree (where 0
          stands for a direction of 3 o'clock, i.e. the right-most point of
          a circle) ({\textless}int{\textgreater})

          \item[etheta]

          ending angle of the arc in units of tenths of a degree (where 0 
          stands for a direction of 3 o'clock, i.e. the right-most point of
          a circle) ({\textless}int{\textgreater})

          \item[colr]

          color value ({\textless}long\_pos{\textgreater})

        \end{Ventry}

      \end{quote}

\textbf{Example:} fl\_arc(120, 253, 40, 10, 60, xfdata.FL\_FORESTGREEN)



\textbf{Status:} Tested + Doc + NoDemo = OK



    \end{boxedminipage}

    \label{xformslib:library:fl_drw_frame}
    \index{xformslib \textit{(package)}!xformslib.library \textit{(module)}!xformslib.library.fl\_drw\_frame \textit{(function)}}

    \vspace{0.5ex}

\hspace{.8\funcindent}\begin{boxedminipage}{\funcwidth}

    \raggedright \textbf{fl\_drw\_frame}(\textit{boxtype}, \textit{x}, \textit{y}, \textit{w}, \textit{h}, \textit{colr}, \textit{bw})

    \vspace{-1.5ex}

    \rule{\textwidth}{0.5\fboxrule}
\setlength{\parskip}{2ex}
    Draws a frame outside of the bounding box specified.

\setlength{\parskip}{1ex}
      \textbf{Parameters}
      \vspace{-1ex}

      \begin{quote}
        \begin{Ventry}{xxxxxxx}

          \item[boxtype]

          type of frame box ({\textless}int{\textgreater})

            {\it (type=(from xfdata module) FL\_NO\_BOX, FL\_UP\_BOX, FL\_DOWN\_BOX, 
FL\_BORDER\_BOX, FL\_SHADOW\_BOX, FL\_FRAME\_BOX, FL\_ROUNDED\_BOX, 
FL\_EMBOSSED\_BOX, FL\_FLAT\_BOX, FL\_RFLAT\_BOX, FL\_RSHADOW\_BOX, 
FL\_OVAL\_BOX, FL\_ROUNDED3D\_UPBOX, FL\_ROUNDED3D\_DOWNBOX, 
FL\_OVAL3D\_UPBOX, FL\_OVAL3D\_DOWNBOX, FL\_OVAL3D\_FRAMEBOX, 
FL\_OVAL3D\_EMBOSSEDBOX)}

          \item[x]

          horizontal position (upper-left corner) 
          ({\textless}int{\textgreater})

          \item[y]

          vertical position (upper-left corner) 
          ({\textless}int{\textgreater})

          \item[w]

          width in coord units ({\textless}int{\textgreater})

          \item[h]

          height in coord units ({\textless}int{\textgreater})

          \item[colr]

          color value ({\textless}long\_pos{\textgreater})

          \item[bw]

          width of boundary ({\textless}int{\textgreater})

        \end{Ventry}

      \end{quote}

\textbf{Example:} fl\_drw\_frame(xfdata.FL\_UP\_BOX, 470, 560, 170, 280, xfdata.FL\_DIMGRAY, 
2)



\textbf{Status:} Tested + Doc + NoDemo = OK



    \end{boxedminipage}

    \label{xformslib:library:fl_drw_checkbox}
    \index{xformslib \textit{(package)}!xformslib.library \textit{(module)}!xformslib.library.fl\_drw\_checkbox \textit{(function)}}

    \vspace{0.5ex}

\hspace{.8\funcindent}\begin{boxedminipage}{\funcwidth}

    \raggedright \textbf{fl\_drw\_checkbox}(\textit{boxtype}, \textit{x}, \textit{y}, \textit{w}, \textit{h}, \textit{colr}, \textit{bw})

    \vspace{-1.5ex}

    \rule{\textwidth}{0.5\fboxrule}
\setlength{\parskip}{2ex}
    Draws a box retated 45 degrees.

\setlength{\parskip}{1ex}
      \textbf{Parameters}
      \vspace{-1ex}

      \begin{quote}
        \begin{Ventry}{xxxxxxx}

          \item[boxtype]

          type of checkbox to draw ({\textless}int{\textgreater})

            {\it (type=(from xfdata module) FL\_NO\_BOX, FL\_UP\_BOX, FL\_DOWN\_BOX, 
FL\_BORDER\_BOX, FL\_SHADOW\_BOX, FL\_FRAME\_BOX, FL\_ROUNDED\_BOX, 
FL\_EMBOSSED\_BOX, FL\_FLAT\_BOX, FL\_RFLAT\_BOX, FL\_RSHADOW\_BOX, 
FL\_OVAL\_BOX, FL\_ROUNDED3D\_UPBOX, FL\_ROUNDED3D\_DOWNBOX, 
FL\_OVAL3D\_UPBOX, FL\_OVAL3D\_DOWNBOX, FL\_OVAL3D\_FRAMEBOX, 
FL\_OVAL3D\_EMBOSSEDBOX)}

          \item[x]

          horizontal position (upper-left corner) 
          ({\textless}int{\textgreater})

          \item[y]

          vertical position (upper-left corner) 
          ({\textless}int{\textgreater})

          \item[w]

          width in coord units ({\textless}int{\textgreater})

          \item[h]

          height in coord units ({\textless}int{\textgreater})

          \item[colr]

          color value ({\textless}long\_pos{\textgreater})

          \item[bw]

          width of boundary ({\textless}int{\textgreater})

        \end{Ventry}

      \end{quote}

\textbf{Example:} fl\_drw\_checkbox(xfdata.FL\_ROUNDED3D\_UPBOX, 470, 560, 170, 280, 
xfdata.FL\_LEMONCHIFFON, -2)



\textbf{Status:} Tested + Doc + NoDemo = OK



    \end{boxedminipage}

    \label{xformslib:library:fl_get_fontstruct}
    \index{xformslib \textit{(package)}!xformslib.library \textit{(module)}!xformslib.library.fl\_get\_fontstruct \textit{(function)}}

    \vspace{0.5ex}

\hspace{.8\funcindent}\begin{boxedminipage}{\funcwidth}

    \raggedright \textbf{fl\_get\_fontstruct}(\textit{style}, \textit{size})

    \vspace{-1.5ex}

    \rule{\textwidth}{0.5\fboxrule}
\setlength{\parskip}{2ex}
    Returns the X font structure for a particular size and style as used in
    XForms Library.

\setlength{\parskip}{1ex}
      \textbf{Parameters}
      \vspace{-1ex}

      \begin{quote}
        \begin{Ventry}{xxxxx}

          \item[style]

          font style ({\textless}int{\textgreater})

            {\it (type=(from xfdata module) FL\_NORMAL\_STYLE, FL\_BOLD\_STYLE, FL\_ITALIC\_STYLE,
FL\_BOLDITALIC\_STYLE, FL\_FIXED\_STYLE, FL\_FIXEDBOLD\_STYLE, 
FL\_FIXEDITALIC\_STYLE, FL\_FIXEDBOLDITALIC\_STYLE, FL\_TIMES\_STYLE, 
FL\_TIMESBOLD\_STYLE, FL\_TIMESITALIC\_STYLE, FL\_TIMESBOLDITALIC\_STYLE, 
FL\_MISC\_STYLE, FL\_MISCBOLD\_STYLE, FL\_MISCITALIC\_STYLE, 
FL\_SYMBOL\_STYLE, FL\_SHADOW\_STYLE, FL\_ENGRAVED\_STYLE, 
FL\_EMBOSSED\_STYLE)}

          \item[size]

          font size ({\textless}int{\textgreater})

            {\it (type=(from xfdata module) FL\_TINY\_SIZE, FL\_SMALL\_SIZE, FL\_NORMAL\_SIZE, 
FL\_MEDIUM\_SIZE, FL\_LARGE\_SIZE, FL\_HUGE\_SIZE, FL\_DEFAULT\_SIZE)}

        \end{Ventry}

      \end{quote}

      \textbf{Return Value}
    \vspace{-1ex}

      \begin{quote}
      pointer to xfdata.XFontStruct

      {\it (type=XFontStruct class)}

      \end{quote}

\textbf{Status:} Tested + Doc + NoDemo = OK



    \end{boxedminipage}

    \label{xformslib:library:fl_get_fontstruct}
    \index{xformslib \textit{(package)}!xformslib.library \textit{(module)}!xformslib.library.fl\_get\_fontstruct \textit{(function)}}

    \vspace{0.5ex}

\hspace{.8\funcindent}\begin{boxedminipage}{\funcwidth}

    \raggedright \textbf{fl\_get\_font\_struct}(\textit{style}, \textit{size})

    \vspace{-1.5ex}

    \rule{\textwidth}{0.5\fboxrule}
\setlength{\parskip}{2ex}
    Returns the X font structure for a particular size and style as used in
    XForms Library.

\setlength{\parskip}{1ex}
      \textbf{Parameters}
      \vspace{-1ex}

      \begin{quote}
        \begin{Ventry}{xxxxx}

          \item[style]

          font style ({\textless}int{\textgreater})

            {\it (type=(from xfdata module) FL\_NORMAL\_STYLE, FL\_BOLD\_STYLE, FL\_ITALIC\_STYLE,
FL\_BOLDITALIC\_STYLE, FL\_FIXED\_STYLE, FL\_FIXEDBOLD\_STYLE, 
FL\_FIXEDITALIC\_STYLE, FL\_FIXEDBOLDITALIC\_STYLE, FL\_TIMES\_STYLE, 
FL\_TIMESBOLD\_STYLE, FL\_TIMESITALIC\_STYLE, FL\_TIMESBOLDITALIC\_STYLE, 
FL\_MISC\_STYLE, FL\_MISCBOLD\_STYLE, FL\_MISCITALIC\_STYLE, 
FL\_SYMBOL\_STYLE, FL\_SHADOW\_STYLE, FL\_ENGRAVED\_STYLE, 
FL\_EMBOSSED\_STYLE)}

          \item[size]

          font size ({\textless}int{\textgreater})

            {\it (type=(from xfdata module) FL\_TINY\_SIZE, FL\_SMALL\_SIZE, FL\_NORMAL\_SIZE, 
FL\_MEDIUM\_SIZE, FL\_LARGE\_SIZE, FL\_HUGE\_SIZE, FL\_DEFAULT\_SIZE)}

        \end{Ventry}

      \end{quote}

      \textbf{Return Value}
    \vspace{-1ex}

      \begin{quote}
      pointer to xfdata.XFontStruct

      {\it (type=XFontStruct class)}

      \end{quote}

\textbf{Status:} Tested + Doc + NoDemo = OK



    \end{boxedminipage}

    \label{xformslib:library:fl_get_fontstruct}
    \index{xformslib \textit{(package)}!xformslib.library \textit{(module)}!xformslib.library.fl\_get\_fontstruct \textit{(function)}}

    \vspace{0.5ex}

\hspace{.8\funcindent}\begin{boxedminipage}{\funcwidth}

    \raggedright \textbf{fl\_get\_fntstruct}(\textit{style}, \textit{size})

    \vspace{-1.5ex}

    \rule{\textwidth}{0.5\fboxrule}
\setlength{\parskip}{2ex}
    Returns the X font structure for a particular size and style as used in
    XForms Library.

\setlength{\parskip}{1ex}
      \textbf{Parameters}
      \vspace{-1ex}

      \begin{quote}
        \begin{Ventry}{xxxxx}

          \item[style]

          font style ({\textless}int{\textgreater})

            {\it (type=(from xfdata module) FL\_NORMAL\_STYLE, FL\_BOLD\_STYLE, FL\_ITALIC\_STYLE,
FL\_BOLDITALIC\_STYLE, FL\_FIXED\_STYLE, FL\_FIXEDBOLD\_STYLE, 
FL\_FIXEDITALIC\_STYLE, FL\_FIXEDBOLDITALIC\_STYLE, FL\_TIMES\_STYLE, 
FL\_TIMESBOLD\_STYLE, FL\_TIMESITALIC\_STYLE, FL\_TIMESBOLDITALIC\_STYLE, 
FL\_MISC\_STYLE, FL\_MISCBOLD\_STYLE, FL\_MISCITALIC\_STYLE, 
FL\_SYMBOL\_STYLE, FL\_SHADOW\_STYLE, FL\_ENGRAVED\_STYLE, 
FL\_EMBOSSED\_STYLE)}

          \item[size]

          font size ({\textless}int{\textgreater})

            {\it (type=(from xfdata module) FL\_TINY\_SIZE, FL\_SMALL\_SIZE, FL\_NORMAL\_SIZE, 
FL\_MEDIUM\_SIZE, FL\_LARGE\_SIZE, FL\_HUGE\_SIZE, FL\_DEFAULT\_SIZE)}

        \end{Ventry}

      \end{quote}

      \textbf{Return Value}
    \vspace{-1ex}

      \begin{quote}
      pointer to xfdata.XFontStruct

      {\it (type=XFontStruct class)}

      \end{quote}

\textbf{Status:} Tested + Doc + NoDemo = OK



    \end{boxedminipage}

    \label{xformslib:library:fl_get_mouse}
    \index{xformslib \textit{(package)}!xformslib.library \textit{(module)}!xformslib.library.fl\_get\_mouse \textit{(function)}}

    \vspace{0.5ex}

\hspace{.8\funcindent}\begin{boxedminipage}{\funcwidth}

    \raggedright \textbf{fl\_get\_mouse}()

    \vspace{-1.5ex}

    \rule{\textwidth}{0.5\fboxrule}
\setlength{\parskip}{2ex}
    Obtains the current mouse position relative to the root window, and the
    current state of the modifier keys and pointer buttons.

\setlength{\parskip}{1ex}
      \textbf{Return Value}
    \vspace{-1ex}

      \begin{quote}
      window the mouse is in, horizontal and vertical position, keymask 
      ({\textless}long\_pos{\textgreater}, {\textless}int{\textgreater}, 
      {\textless}int{\textgreater}, {\textless}int\_pos{\textgreater})

      {\it (type=win, x, y, keymask)}

      \end{quote}

\textbf{Example:} win, x, y, keym = fl\_get\_mouse()



\textbf{Attention:} API change from XForms - upstream was fl\_get\_mouse(x, y, keymask)



\textbf{Status:} Tested + Doc + NoDemo = OK



    \end{boxedminipage}

    \label{xformslib:library:fl_set_mouse}
    \index{xformslib \textit{(package)}!xformslib.library \textit{(module)}!xformslib.library.fl\_set\_mouse \textit{(function)}}

    \vspace{0.5ex}

\hspace{.8\funcindent}\begin{boxedminipage}{\funcwidth}

    \raggedright \textbf{fl\_set\_mouse}(\textit{x}, \textit{y})

    \vspace{-1.5ex}

    \rule{\textwidth}{0.5\fboxrule}
\setlength{\parskip}{2ex}
    Moves the mouse to a specific location relative to the root window. Use
    this function sparingly, it can be extremely annoying for the user if 
    the mouse position is changed by a program.

\setlength{\parskip}{1ex}
      \textbf{Parameters}
      \vspace{-1ex}

      \begin{quote}
        \begin{Ventry}{x}

          \item[x]

          horizontal position ({\textless}int{\textgreater})

          \item[y]

          vertical position ({\textless}int{\textgreater})

        \end{Ventry}

      \end{quote}

\textbf{Example:} fl\_set\_mouse(200, 120)



\textbf{Status:} Tested + Doc + NoDemo = OK



    \end{boxedminipage}

    \label{xformslib:library:fl_get_win_mouse}
    \index{xformslib \textit{(package)}!xformslib.library \textit{(module)}!xformslib.library.fl\_get\_win\_mouse \textit{(function)}}

    \vspace{0.5ex}

\hspace{.8\funcindent}\begin{boxedminipage}{\funcwidth}

    \raggedright \textbf{fl\_get\_win\_mouse}(\textit{win})

    \vspace{-1.5ex}

    \rule{\textwidth}{0.5\fboxrule}
\setlength{\parskip}{2ex}
    Obtains the position of the mouse relative to a certain window, and the
    current state of the modifier keys and pointer buttons.

\setlength{\parskip}{1ex}
      \textbf{Parameters}
      \vspace{-1ex}

      \begin{quote}
        \begin{Ventry}{xxx}

          \item[win]

          window id ({\textless}long\_pos{\textgreater})

        \end{Ventry}

      \end{quote}

      \textbf{Return Value}
    \vspace{-1ex}

      \begin{quote}
      window the mouse is in, horizontal and vertical position, keymask 
      ({\textless}long\_pos{\textgreater}, {\textless}int{\textgreater}, 
      {\textless}int{\textgreater}, {\textless}int\_pos{\textgreater})

      {\it (type=win, x, y, keymask)}

      \end{quote}

\textbf{Example:} win, x, y, keym = fl\_get\_win\_mouse()



\textbf{Attention:} API change from XForms - upstream was fl\_get\_win\_mouse(win, x, y, 
keymask)



\textbf{Status:} Tested + NoDoc + Demo = OK



    \end{boxedminipage}

    \label{xformslib:library:fl_get_form_mouse}
    \index{xformslib \textit{(package)}!xformslib.library \textit{(module)}!xformslib.library.fl\_get\_form\_mouse \textit{(function)}}

    \vspace{0.5ex}

\hspace{.8\funcindent}\begin{boxedminipage}{\funcwidth}

    \raggedright \textbf{fl\_get\_form\_mouse}(\textit{pForm})

    \vspace{-1.5ex}

    \rule{\textwidth}{0.5\fboxrule}
\setlength{\parskip}{2ex}
    Obtains the position of the mouse relative to a certain form, and the 
    current state of the modifier keys and pointer buttons.

\setlength{\parskip}{1ex}
      \textbf{Parameters}
      \vspace{-1ex}

      \begin{quote}
        \begin{Ventry}{xxxxx}

          \item[pForm]

          form ({\textless}pointer to xfdata.FL\_FORM{\textgreater})

        \end{Ventry}

      \end{quote}

      \textbf{Return Value}
    \vspace{-1ex}

      \begin{quote}
      window the mouse is in, horizontal and vertical position, keymask 
      ({\textless}long\_pos{\textgreater}, {\textless}int{\textgreater}, 
      {\textless}int{\textgreater}, {\textless}int\_pos{\textgreater})

      {\it (type=win, x, y, keymask)}

      \end{quote}

\textbf{Example:} win, x, y, keym = fl\_get\_form\_mouse()



\textbf{Attention:} API change from XForms - upstream was fl\_get\_form\_mouse(fm, x, y, 
keymask)



\textbf{Status:} Tested + Doc + NoDemo = OK



    \end{boxedminipage}

    \label{xformslib:library:fl_win_to_form}
    \index{xformslib \textit{(package)}!xformslib.library \textit{(module)}!xformslib.library.fl\_win\_to\_form \textit{(function)}}

    \vspace{0.5ex}

\hspace{.8\funcindent}\begin{boxedminipage}{\funcwidth}

    \raggedright \textbf{fl\_win\_to\_form}(\textit{win})

    \vspace{-1.5ex}

    \rule{\textwidth}{0.5\fboxrule}
\setlength{\parskip}{2ex}
    Returns the form the specified window belongs to.

\setlength{\parskip}{1ex}
      \textbf{Parameters}
      \vspace{-1ex}

      \begin{quote}
        \begin{Ventry}{xxx}

          \item[win]

          window id ({\textless}long\_pos{\textgreater})

        \end{Ventry}

      \end{quote}

      \textbf{Return Value}
    \vspace{-1ex}

      \begin{quote}
      form ({\textless}pointer to xfdata.FL\_FORM{\textgreater}) or None 
      (on failure)

      {\it (type=pForm)}

      \end{quote}

\textbf{Example:} pform2 = fl\_win\_to\_form(win1)



\textbf{Status:} Tested + Doc + NoDemo = OK



    \end{boxedminipage}

    \label{xformslib:library:fl_set_form_icon}
    \index{xformslib \textit{(package)}!xformslib.library \textit{(module)}!xformslib.library.fl\_set\_form\_icon \textit{(function)}}

    \vspace{0.5ex}

\hspace{.8\funcindent}\begin{boxedminipage}{\funcwidth}

    \raggedright \textbf{fl\_set\_form\_icon}(\textit{pForm}, \textit{icon}, \textit{mask})

    \vspace{-1.5ex}

    \rule{\textwidth}{0.5\fboxrule}
\setlength{\parskip}{2ex}
    Sets or changes the icon shown when a form is iconified.

\setlength{\parskip}{1ex}
      \textbf{Parameters}
      \vspace{-1ex}

      \begin{quote}
        \begin{Ventry}{xxxxx}

          \item[pForm]

          form ({\textless}pointer to xfdata.FL\_FORM{\textgreater})

          \item[icon]

          icon pixmap id ({\textless}long\_pos{\textgreater})

          \item[mask]

          mask pixmap id ({\textless}long\_pos{\textgreater})

        \end{Ventry}

      \end{quote}

\textbf{Status:} Tested + Doc + Demo = OK



    \end{boxedminipage}

    \label{xformslib:library:fl_get_decoration_sizes}
    \index{xformslib \textit{(package)}!xformslib.library \textit{(module)}!xformslib.library.fl\_get\_decoration\_sizes \textit{(function)}}

    \vspace{0.5ex}

\hspace{.8\funcindent}\begin{boxedminipage}{\funcwidth}

    \raggedright \textbf{fl\_get\_decoration\_sizes}(\textit{pForm})

    \vspace{-1.5ex}

    \rule{\textwidth}{0.5\fboxrule}
\setlength{\parskip}{2ex}
    Returns the sizes of the "decorations" the window manager puts around a
    form's window. Returns 0 on success and 1 if the form isn't visible or 
    it's a form embedded into another form.

\setlength{\parskip}{1ex}
      \textbf{Parameters}
      \vspace{-1ex}

      \begin{quote}
        \begin{Ventry}{xxxxx}

          \item[pForm]

          form ({\textless}pointer to xfdata.FL\_FORM{\textgreater})

        \end{Ventry}

      \end{quote}

      \textbf{Return Value}
    \vspace{-1ex}

      \begin{quote}
      num. (0 or 1), top size, right size, bottom size, left size 
      ({\textless}int{\textgreater}, {\textless}int{\textgreater}, 
      {\textless}int{\textgreater}, {\textless}int{\textgreater})

      {\it (type=num., top, right, bottom, left)}

      \end{quote}

\textbf{Attention:} API change from XForms - upstream was fl\_get\_decoration\_sizes(pForm, 
top, right, bottom, left)



\textbf{Status:} Tested + Doc + NoDemo = OK



    \end{boxedminipage}

    \label{xformslib:library:fl_raise_form}
    \index{xformslib \textit{(package)}!xformslib.library \textit{(module)}!xformslib.library.fl\_raise\_form \textit{(function)}}

    \vspace{0.5ex}

\hspace{.8\funcindent}\begin{boxedminipage}{\funcwidth}

    \raggedright \textbf{fl\_raise\_form}(\textit{pForm})

    \vspace{-1.5ex}

    \rule{\textwidth}{0.5\fboxrule}
\setlength{\parskip}{2ex}
    Raises a form to the top of the screen so no other forms obscure it.

\setlength{\parskip}{1ex}
      \textbf{Parameters}
      \vspace{-1ex}

      \begin{quote}
        \begin{Ventry}{xxxxx}

          \item[pForm]

          form to be raised ({\textless}pointer to 
          xfdata.FL\_FORM{\textgreater})

        \end{Ventry}

      \end{quote}

\textbf{Example:} fl\_raise\_form(pform2)



\textbf{Status:} Tested + Doc + NoDemo = OK



    \end{boxedminipage}

    \label{xformslib:library:fl_lower_form}
    \index{xformslib \textit{(package)}!xformslib.library \textit{(module)}!xformslib.library.fl\_lower\_form \textit{(function)}}

    \vspace{0.5ex}

\hspace{.8\funcindent}\begin{boxedminipage}{\funcwidth}

    \raggedright \textbf{fl\_lower\_form}(\textit{pForm})

    \vspace{-1.5ex}

    \rule{\textwidth}{0.5\fboxrule}
\setlength{\parskip}{2ex}
    Lowers a form to the bottom of the stack.

\setlength{\parskip}{1ex}
      \textbf{Parameters}
      \vspace{-1ex}

      \begin{quote}
        \begin{Ventry}{xxxxx}

          \item[pForm]

          form to be lowered ({\textless}pointer to 
          xfdata.FL\_FORM{\textgreater})

        \end{Ventry}

      \end{quote}

\textbf{Example:} fl\_lower\_form(pform2)



\textbf{Status:} Tested + Doc + NoDemo = OK



    \end{boxedminipage}

    \label{xformslib:library:fl_set_foreground}
    \index{xformslib \textit{(package)}!xformslib.library \textit{(module)}!xformslib.library.fl\_set\_foreground \textit{(function)}}

    \vspace{0.5ex}

\hspace{.8\funcindent}\begin{boxedminipage}{\funcwidth}

    \raggedright \textbf{fl\_set\_foreground}(\textit{gc}, \textit{colr})

    \vspace{-1.5ex}

    \rule{\textwidth}{0.5\fboxrule}
\setlength{\parskip}{2ex}
    Sets foreground color in GCs other than the XForms library's default.

\setlength{\parskip}{1ex}
      \textbf{Parameters}
      \vspace{-1ex}

      \begin{quote}
        \begin{Ventry}{xxxx}

          \item[gc]

          Graphics context number

          \item[colr]

          color value to be set as foreground 
          ({\textless}long\_pos{\textgreater})

        \end{Ventry}

      \end{quote}

\textbf{Example:} gc = fl\_state[fl\_get\_vclass()].gc[0] ?? fl\_set\_foreground(gc, 
xfdata.FL\_LAWNGREEN)



\textbf{Status:} Untested + NoDoc + NoDemo = NOT OK (NULL pointer access)



    \end{boxedminipage}

    \label{xformslib:library:fl_set_background}
    \index{xformslib \textit{(package)}!xformslib.library \textit{(module)}!xformslib.library.fl\_set\_background \textit{(function)}}

    \vspace{0.5ex}

\hspace{.8\funcindent}\begin{boxedminipage}{\funcwidth}

    \raggedright \textbf{fl\_set\_background}(\textit{gc}, \textit{colr})

    \vspace{-1.5ex}

    \rule{\textwidth}{0.5\fboxrule}
\setlength{\parskip}{2ex}
    Sets background color in GCs other than the XForms library's default.

\setlength{\parskip}{1ex}
      \textbf{Parameters}
      \vspace{-1ex}

      \begin{quote}
        \begin{Ventry}{xxxx}

          \item[gc]

          Graphics context number

          \item[colr]

          color value to be set as background 
          ({\textless}long\_pos{\textgreater})

        \end{Ventry}

      \end{quote}

\textbf{Example:} gc = fl\_state[fl\_get\_vclass()].gc[0] ?? fl\_set\_foreground(gc, 
xfdata.FL\_HONEYDEW)



\textbf{Status:} Untested + NoDoc + NoDemo = NOT OK (NULL pointer access)



    \end{boxedminipage}

    \label{xformslib:library:fl_wincreate}
    \index{xformslib \textit{(package)}!xformslib.library \textit{(module)}!xformslib.library.fl\_wincreate \textit{(function)}}

    \vspace{0.5ex}

\hspace{.8\funcindent}\begin{boxedminipage}{\funcwidth}

    \raggedright \textbf{fl\_wincreate}(\textit{title})

    \vspace{-1.5ex}

    \rule{\textwidth}{0.5\fboxrule}
\setlength{\parskip}{2ex}
    Creates a window with a specified title.

\setlength{\parskip}{1ex}
      \textbf{Parameters}
      \vspace{-1ex}

      \begin{quote}
        \begin{Ventry}{xxxxx}

          \item[title]

          title of the window ({\textless}string{\textgreater})

        \end{Ventry}

      \end{quote}

      \textbf{Return Value}
    \vspace{-1ex}

      \begin{quote}
      created window id ({\textless}long\_pos{\textgreater})

      {\it (type=win)}

      \end{quote}

\textbf{Example:} win2 = fl\_wincreate("My long title")



\textbf{Status:} Tested + Doc + NoDemo = OK



    \end{boxedminipage}

    \label{xformslib:library:fl_winshow}
    \index{xformslib \textit{(package)}!xformslib.library \textit{(module)}!xformslib.library.fl\_winshow \textit{(function)}}

    \vspace{0.5ex}

\hspace{.8\funcindent}\begin{boxedminipage}{\funcwidth}

    \raggedright \textbf{fl\_winshow}(\textit{win})

    \vspace{-1.5ex}

    \rule{\textwidth}{0.5\fboxrule}
\setlength{\parskip}{2ex}
    Shows the window (created with fl\_wincreate).

\setlength{\parskip}{1ex}
      \textbf{Parameters}
      \vspace{-1ex}

      \begin{quote}
        \begin{Ventry}{xxx}

          \item[win]

          window id to show ({\textless}long\_pos{\textgreater})

        \end{Ventry}

      \end{quote}

      \textbf{Return Value}
    \vspace{-1ex}

      \begin{quote}
      window id shown ({\textless}long\_pos{\textgreater})

      {\it (type=win)}

      \end{quote}

\textbf{Example:} winw = fl\_winshow(win2)



\textbf{Status:} Tested + Doc + NoDemo = OK



    \end{boxedminipage}

    \label{xformslib:library:fl_winopen}
    \index{xformslib \textit{(package)}!xformslib.library \textit{(module)}!xformslib.library.fl\_winopen \textit{(function)}}

    \vspace{0.5ex}

\hspace{.8\funcindent}\begin{boxedminipage}{\funcwidth}

    \raggedright \textbf{fl\_winopen}(\textit{title})

    \vspace{-1.5ex}

    \rule{\textwidth}{0.5\fboxrule}
\setlength{\parskip}{2ex}
    Opens (creates and shows) a toplevel window with the specified title.

\setlength{\parskip}{1ex}
      \textbf{Parameters}
      \vspace{-1ex}

      \begin{quote}
        \begin{Ventry}{xxxxx}

          \item[title]

          title of the window ({\textless}string{\textgreater})

        \end{Ventry}

      \end{quote}

      \textbf{Return Value}
    \vspace{-1ex}

      \begin{quote}
      created window id ({\textless}long\_pos{\textgreater})

      {\it (type=window)}

      \end{quote}

\textbf{Example:} win2 = fl\_winopen("My long title")



\textbf{Status:} Tested + Doc + Demo = OK



    \end{boxedminipage}

    \label{xformslib:library:fl_winhide}
    \index{xformslib \textit{(package)}!xformslib.library \textit{(module)}!xformslib.library.fl\_winhide \textit{(function)}}

    \vspace{0.5ex}

\hspace{.8\funcindent}\begin{boxedminipage}{\funcwidth}

    \raggedright \textbf{fl\_winhide}(\textit{win})

    \vspace{-1.5ex}

    \rule{\textwidth}{0.5\fboxrule}
\setlength{\parskip}{2ex}
    Hides a shown window.

\setlength{\parskip}{1ex}
      \textbf{Parameters}
      \vspace{-1ex}

      \begin{quote}
        \begin{Ventry}{xxx}

          \item[win]

          window id to hide ({\textless}long\_pos{\textgreater})

        \end{Ventry}

      \end{quote}

\textbf{Example:} fl\_winhide(win2)



\textbf{Status:} Tested + Doc + Demo = OK



    \end{boxedminipage}

    \label{xformslib:library:fl_winclose}
    \index{xformslib \textit{(package)}!xformslib.library \textit{(module)}!xformslib.library.fl\_winclose \textit{(function)}}

    \vspace{0.5ex}

\hspace{.8\funcindent}\begin{boxedminipage}{\funcwidth}

    \raggedright \textbf{fl\_winclose}(\textit{win})

    \vspace{-1.5ex}

    \rule{\textwidth}{0.5\fboxrule}
\setlength{\parskip}{2ex}
    Closes (hides and destroys) the specified window.

\setlength{\parskip}{1ex}
      \textbf{Parameters}
      \vspace{-1ex}

      \begin{quote}
        \begin{Ventry}{xxx}

          \item[win]

          window id to close ({\textless}long\_pos{\textgreater})

        \end{Ventry}

      \end{quote}

\textbf{Example:} fl\_winclose(win2)



\textbf{Status:} Tested + Doc + NoDemo = OK



    \end{boxedminipage}

    \label{xformslib:library:fl_winset}
    \index{xformslib \textit{(package)}!xformslib.library \textit{(module)}!xformslib.library.fl\_winset \textit{(function)}}

    \vspace{0.5ex}

\hspace{.8\funcindent}\begin{boxedminipage}{\funcwidth}

    \raggedright \textbf{fl\_winset}(\textit{win})

    \vspace{-1.5ex}

    \rule{\textwidth}{0.5\fboxrule}
\setlength{\parskip}{2ex}
    Sets the "current window", defined as the window the object that uses 
    the drawing routine belongs to.

\setlength{\parskip}{1ex}
      \textbf{Parameters}
      \vspace{-1ex}

      \begin{quote}
        \begin{Ventry}{xxx}

          \item[win]

          window id to set as current ({\textless}long\_pos{\textgreater})

        \end{Ventry}

      \end{quote}

\textbf{Example:} fl\_winset(win3)



\textbf{Status:} Tested + Doc + Demo = OK



    \end{boxedminipage}

    \label{xformslib:library:fl_winreparent}
    \index{xformslib \textit{(package)}!xformslib.library \textit{(module)}!xformslib.library.fl\_winreparent \textit{(function)}}

    \vspace{0.5ex}

\hspace{.8\funcindent}\begin{boxedminipage}{\funcwidth}

    \raggedright \textbf{fl\_winreparent}(\textit{win}, \textit{winnewparent})

    \vspace{-1.5ex}

    \rule{\textwidth}{0.5\fboxrule}
\setlength{\parskip}{2ex}
    Makes a toplevel window a subwindow of another (new parent) window; 
    both the window and the parent window must be valid ones.

\setlength{\parskip}{1ex}
      \textbf{Parameters}
      \vspace{-1ex}

      \begin{quote}
        \begin{Ventry}{xxxxxxxxxxxx}

          \item[win]

          window id to be made a subwindow  
          ({\textless}long\_pos{\textgreater})

          \item[winnewparent]

          window id to become its new parent window 
          ({\textless}long\_pos{\textgreater})

        \end{Ventry}

      \end{quote}

      \textbf{Return Value}
    \vspace{-1ex}

      \begin{quote}
      num. ({\textless}int{\textgreater}) or -1 (on failure)

      {\it (type=num)}

      \end{quote}

\textbf{Example:} exitval = fl\_winreparent(win1, win3)



\textbf{Status:} Tested + Doc + NoDemo = OK



    \end{boxedminipage}

    \label{xformslib:library:fl_winfocus}
    \index{xformslib \textit{(package)}!xformslib.library \textit{(module)}!xformslib.library.fl\_winfocus \textit{(function)}}

    \vspace{0.5ex}

\hspace{.8\funcindent}\begin{boxedminipage}{\funcwidth}

    \raggedright \textbf{fl\_winfocus}(\textit{win})

    \vspace{-1.5ex}

    \rule{\textwidth}{0.5\fboxrule}
\setlength{\parskip}{2ex}
    Keyboard input is directed to the specified window, overriding the 
    keyboard focus assignment.

\setlength{\parskip}{1ex}
      \textbf{Parameters}
      \vspace{-1ex}

      \begin{quote}
        \begin{Ventry}{xxx}

          \item[win]

          window id ({\textless}long\_pos{\textgreater})

        \end{Ventry}

      \end{quote}

\textbf{Example:} fl\_winfocus(win3)



\textbf{Status:} Tested + Doc + NoDemo = OK



    \end{boxedminipage}

    \label{xformslib:library:fl_winget}
    \index{xformslib \textit{(package)}!xformslib.library \textit{(module)}!xformslib.library.fl\_winget \textit{(function)}}

    \vspace{0.5ex}

\hspace{.8\funcindent}\begin{boxedminipage}{\funcwidth}

    \raggedright \textbf{fl\_winget}()

    \vspace{-1.5ex}

    \rule{\textwidth}{0.5\fboxrule}
\setlength{\parskip}{2ex}
    Queries the current window. One caveat about fl\_winget() is that it 
    can return None if called outside of an object's event handler, 
    depending on where the mouse is. Thus, the return value of this 
    function should be checked when called outside of an object handler.

\setlength{\parskip}{1ex}
      \textbf{Return Value}
    \vspace{-1ex}

      \begin{quote}
      window id ({\textless}long\_pos{\textgreater})

      {\it (type=win)}

      \end{quote}

\textbf{Example:} currwin = fl\_winget()



\textbf{Status:} Tested + Doc + NoDemo = OK



    \end{boxedminipage}

    \label{xformslib:library:fl_iconify}
    \index{xformslib \textit{(package)}!xformslib.library \textit{(module)}!xformslib.library.fl\_iconify \textit{(function)}}

    \vspace{0.5ex}

\hspace{.8\funcindent}\begin{boxedminipage}{\funcwidth}

    \raggedright \textbf{fl\_iconify}(\textit{win})

    \vspace{-1.5ex}

    \rule{\textwidth}{0.5\fboxrule}
\setlength{\parskip}{2ex}
    Iconifies the specified window.

\setlength{\parskip}{1ex}
      \textbf{Parameters}
      \vspace{-1ex}

      \begin{quote}
        \begin{Ventry}{xxx}

          \item[win]

          window id ({\textless}long\_pos{\textgreater})

        \end{Ventry}

      \end{quote}

      \textbf{Return Value}
    \vspace{-1ex}

      \begin{quote}
      num. ({\textless}int{\textgreater})

      {\it (type=num)}

      \end{quote}

\textbf{Example:} fl\_iconify(win2)



\textbf{Status:} Tested + Doc + NoDemo = OK



    \end{boxedminipage}

    \label{xformslib:library:fl_winresize}
    \index{xformslib \textit{(package)}!xformslib.library \textit{(module)}!xformslib.library.fl\_winresize \textit{(function)}}

    \vspace{0.5ex}

\hspace{.8\funcindent}\begin{boxedminipage}{\funcwidth}

    \raggedright \textbf{fl\_winresize}(\textit{win}, \textit{w}, \textit{h})

    \vspace{-1.5ex}

    \rule{\textwidth}{0.5\fboxrule}
\setlength{\parskip}{2ex}
    Resizes a window.

\setlength{\parskip}{1ex}
      \textbf{Parameters}
      \vspace{-1ex}

      \begin{quote}
        \begin{Ventry}{xxx}

          \item[win]

          window id to resize ({\textless}long\_pos{\textgreater})

          \item[w]

          new width in coord units ({\textless}int{\textgreater})

          \item[h]

          new height in coord units ({\textless}int{\textgreater})

        \end{Ventry}

      \end{quote}

\textbf{Example:} fl\_winresize(win6, 547, 624)



\textbf{Status:} Tested + Doc + NoDemo = OK



    \end{boxedminipage}

    \label{xformslib:library:fl_winmove}
    \index{xformslib \textit{(package)}!xformslib.library \textit{(module)}!xformslib.library.fl\_winmove \textit{(function)}}

    \vspace{0.5ex}

\hspace{.8\funcindent}\begin{boxedminipage}{\funcwidth}

    \raggedright \textbf{fl\_winmove}(\textit{win}, \textit{x}, \textit{y})

    \vspace{-1.5ex}

    \rule{\textwidth}{0.5\fboxrule}
\setlength{\parskip}{2ex}
    Moves the specified window to a new position.

\setlength{\parskip}{1ex}
      \textbf{Parameters}
      \vspace{-1ex}

      \begin{quote}
        \begin{Ventry}{xxx}

          \item[win]

          window id to move to a new position 
          ({\textless}long\_pos{\textgreater})

          \item[x]

          new horizontal position (upper-left corner) 
          ({\textless}int{\textgreater})

          \item[y]

          new vertical position (upper-left corner) 
          ({\textless}int{\textgreater})

        \end{Ventry}

      \end{quote}

\textbf{Example:} fl\_winmove(win5, 116, 331)



\textbf{Status:} Tested + Doc + NoDemo = OK



    \end{boxedminipage}

    \label{xformslib:library:fl_winreshape}
    \index{xformslib \textit{(package)}!xformslib.library \textit{(module)}!xformslib.library.fl\_winreshape \textit{(function)}}

    \vspace{0.5ex}

\hspace{.8\funcindent}\begin{boxedminipage}{\funcwidth}

    \raggedright \textbf{fl\_winreshape}(\textit{win}, \textit{x}, \textit{y}, \textit{w}, \textit{h})

    \vspace{-1.5ex}

    \rule{\textwidth}{0.5\fboxrule}
\setlength{\parskip}{2ex}
    Reshapes (resizes and moves) a window.

\setlength{\parskip}{1ex}
      \textbf{Parameters}
      \vspace{-1ex}

      \begin{quote}
        \begin{Ventry}{xxx}

          \item[win]

          window id to reshape ({\textless}long\_pos{\textgreater})

          \item[x]

          new horizontal position (upper-left corner) 
          ({\textless}int{\textgreater})

          \item[y]

          new vertical position (upper-left corner) 
          ({\textless}int{\textgreater})

          \item[w]

          width in coord units ({\textless}int{\textgreater})

          \item[h]

          height in coord units ({\textless}int{\textgreater})

        \end{Ventry}

      \end{quote}

\textbf{Example:} fl\_winreshape(win5, 116, 331, 144, 182)



\textbf{Status:} Tested + Doc + NoDemo = OK



    \end{boxedminipage}

    \label{xformslib:library:fl_winicon}
    \index{xformslib \textit{(package)}!xformslib.library \textit{(module)}!xformslib.library.fl\_winicon \textit{(function)}}

    \vspace{0.5ex}

\hspace{.8\funcindent}\begin{boxedminipage}{\funcwidth}

    \raggedright \textbf{fl\_winicon}(\textit{win}, \textit{icon}, \textit{mask})

    \vspace{-1.5ex}

    \rule{\textwidth}{0.5\fboxrule}
\setlength{\parskip}{2ex}
    Installs an icon for the window.

\setlength{\parskip}{1ex}
      \textbf{Parameters}
      \vspace{-1ex}

      \begin{quote}
        \begin{Ventry}{xxxx}

          \item[win]

          window id ({\textless}long\_pos{\textgreater})

          \item[icon]

          pixmap icon id to be installed in window 
          ({\textless}long\_pos{\textgreater})

          \item[mask]

          pixmap mask id ({\textless}long\_pos{\textgreater})

        \end{Ventry}

      \end{quote}

\textbf{Example:} fl\_winicon(win0, ...)



\textbf{Status:} Untested + NoDoc + NoDemo = NOT OK



    \end{boxedminipage}

    \label{xformslib:library:fl_winbackground}
    \index{xformslib \textit{(package)}!xformslib.library \textit{(module)}!xformslib.library.fl\_winbackground \textit{(function)}}

    \vspace{0.5ex}

\hspace{.8\funcindent}\begin{boxedminipage}{\funcwidth}

    \raggedright \textbf{fl\_winbackground}(\textit{win}, \textit{bkcolr})

    \vspace{-1.5ex}

    \rule{\textwidth}{0.5\fboxrule}
\setlength{\parskip}{2ex}
    Sets the background of window to a certain color.

\setlength{\parskip}{1ex}
      \textbf{Parameters}
      \vspace{-1ex}

      \begin{quote}
        \begin{Ventry}{xxxxxx}

          \item[win]

          window id ({\textless}long\_pos{\textgreater})

          \item[bkcolr]

          background color to be set ({\textless}long\_pos{\textgreater})

        \end{Ventry}

      \end{quote}

\textbf{Example:} fl\_winbackground(win1, xfdata.FL\_GHOSTWHITE)



\textbf{Status:} Tested + NoDoc + Demo = OK



    \end{boxedminipage}

    \label{xformslib:library:fl_winbackground}
    \index{xformslib \textit{(package)}!xformslib.library \textit{(module)}!xformslib.library.fl\_winbackground \textit{(function)}}

    \vspace{0.5ex}

\hspace{.8\funcindent}\begin{boxedminipage}{\funcwidth}

    \raggedright \textbf{fl\_win\_background}(\textit{win}, \textit{bkcolr})

    \vspace{-1.5ex}

    \rule{\textwidth}{0.5\fboxrule}
\setlength{\parskip}{2ex}
    Sets the background of window to a certain color.

\setlength{\parskip}{1ex}
      \textbf{Parameters}
      \vspace{-1ex}

      \begin{quote}
        \begin{Ventry}{xxxxxx}

          \item[win]

          window id ({\textless}long\_pos{\textgreater})

          \item[bkcolr]

          background color to be set ({\textless}long\_pos{\textgreater})

        \end{Ventry}

      \end{quote}

\textbf{Example:} fl\_winbackground(win1, xfdata.FL\_GHOSTWHITE)



\textbf{Status:} Tested + NoDoc + Demo = OK



    \end{boxedminipage}

    \label{xformslib:library:fl_winstepsize}
    \index{xformslib \textit{(package)}!xformslib.library \textit{(module)}!xformslib.library.fl\_winstepsize \textit{(function)}}

    \vspace{0.5ex}

\hspace{.8\funcindent}\begin{boxedminipage}{\funcwidth}

    \raggedright \textbf{fl\_winstepsize}(\textit{win}, \textit{xunit}, \textit{yunit})

    \vspace{-1.5ex}

    \rule{\textwidth}{0.5\fboxrule}
\setlength{\parskip}{2ex}
    Sets the steps by which the size of a window can be changed. Changes to
    the window size will be multiples of specified units after this call. 
    Note that this only applies to interactive resizing.

\setlength{\parskip}{1ex}
      \textbf{Parameters}
      \vspace{-1ex}

      \begin{quote}
        \begin{Ventry}{xxxxx}

          \item[xunit]

          number of pixels of changes per unit in horizontal direction 
          ({\textless}int{\textgreater})

          \item[yunit]

          number of pixels of changes per unit in vertical direction 
          ({\textless}int{\textgreater})

        \end{Ventry}

      \end{quote}

\textbf{Example:} fl\_winstepsize(win0, 10, 10)



\textbf{Status:} Tested + Doc + NoDemo = OK



    \end{boxedminipage}

    \label{xformslib:library:fl_winstepsize}
    \index{xformslib \textit{(package)}!xformslib.library \textit{(module)}!xformslib.library.fl\_winstepsize \textit{(function)}}

    \vspace{0.5ex}

\hspace{.8\funcindent}\begin{boxedminipage}{\funcwidth}

    \raggedright \textbf{fl\_winstepunit}(\textit{win}, \textit{xunit}, \textit{yunit})

    \vspace{-1.5ex}

    \rule{\textwidth}{0.5\fboxrule}
\setlength{\parskip}{2ex}
    Sets the steps by which the size of a window can be changed. Changes to
    the window size will be multiples of specified units after this call. 
    Note that this only applies to interactive resizing.

\setlength{\parskip}{1ex}
      \textbf{Parameters}
      \vspace{-1ex}

      \begin{quote}
        \begin{Ventry}{xxxxx}

          \item[xunit]

          number of pixels of changes per unit in horizontal direction 
          ({\textless}int{\textgreater})

          \item[yunit]

          number of pixels of changes per unit in vertical direction 
          ({\textless}int{\textgreater})

        \end{Ventry}

      \end{quote}

\textbf{Example:} fl\_winstepsize(win0, 10, 10)



\textbf{Status:} Tested + Doc + NoDemo = OK



    \end{boxedminipage}

    \label{xformslib:library:fl_winstepsize}
    \index{xformslib \textit{(package)}!xformslib.library \textit{(module)}!xformslib.library.fl\_winstepsize \textit{(function)}}

    \vspace{0.5ex}

\hspace{.8\funcindent}\begin{boxedminipage}{\funcwidth}

    \raggedright \textbf{fl\_set\_winstepunit}(\textit{win}, \textit{xunit}, \textit{yunit})

    \vspace{-1.5ex}

    \rule{\textwidth}{0.5\fboxrule}
\setlength{\parskip}{2ex}
    Sets the steps by which the size of a window can be changed. Changes to
    the window size will be multiples of specified units after this call. 
    Note that this only applies to interactive resizing.

\setlength{\parskip}{1ex}
      \textbf{Parameters}
      \vspace{-1ex}

      \begin{quote}
        \begin{Ventry}{xxxxx}

          \item[xunit]

          number of pixels of changes per unit in horizontal direction 
          ({\textless}int{\textgreater})

          \item[yunit]

          number of pixels of changes per unit in vertical direction 
          ({\textless}int{\textgreater})

        \end{Ventry}

      \end{quote}

\textbf{Example:} fl\_winstepsize(win0, 10, 10)



\textbf{Status:} Tested + Doc + NoDemo = OK



    \end{boxedminipage}

    \label{xformslib:library:fl_winisvalid}
    \index{xformslib \textit{(package)}!xformslib.library \textit{(module)}!xformslib.library.fl\_winisvalid \textit{(function)}}

    \vspace{0.5ex}

\hspace{.8\funcindent}\begin{boxedminipage}{\funcwidth}

    \raggedright \textbf{fl\_winisvalid}(\textit{win})

    \vspace{-1.5ex}

    \rule{\textwidth}{0.5\fboxrule}
\setlength{\parskip}{2ex}
    Checks if a window id is valid or not. Note that excessive use of this 
    function may negatively impact performance.

\setlength{\parskip}{1ex}
      \textbf{Parameters}
      \vspace{-1ex}

      \begin{quote}
        \begin{Ventry}{xxx}

          \item[win]

          window to evaluate ({\textless}long\_pos{\textgreater})

        \end{Ventry}

      \end{quote}

      \textbf{Return Value}
    \vspace{-1ex}

      \begin{quote}
      num. ({\textless}int{\textgreater})

      {\it (type=num)}

      \end{quote}

\textbf{Example:} if fl\_winisvalid(win3): ...



\textbf{Status:} Tested + Doc + NoDemo = OK



    \end{boxedminipage}

    \label{xformslib:library:fl_wintitle}
    \index{xformslib \textit{(package)}!xformslib.library \textit{(module)}!xformslib.library.fl\_wintitle \textit{(function)}}

    \vspace{0.5ex}

\hspace{.8\funcindent}\begin{boxedminipage}{\funcwidth}

    \raggedright \textbf{fl\_wintitle}(\textit{win}, \textit{title})

    \vspace{-1.5ex}

    \rule{\textwidth}{0.5\fboxrule}
\setlength{\parskip}{2ex}
    Changes the window title (and its associated icon title).

\setlength{\parskip}{1ex}
      \textbf{Parameters}
      \vspace{-1ex}

      \begin{quote}
        \begin{Ventry}{xxxxx}

          \item[win]

          window id ({\textless}long\_pos{\textgreater})

          \item[title]

          window title to be set ({\textless}string{\textgreater})

        \end{Ventry}

      \end{quote}

\textbf{Example:} fl\_wintitle("My brand new title")



\textbf{Status:} Tested + Doc + NoDemo = OK



    \end{boxedminipage}

    \label{xformslib:library:fl_winicontitle}
    \index{xformslib \textit{(package)}!xformslib.library \textit{(module)}!xformslib.library.fl\_winicontitle \textit{(function)}}

    \vspace{0.5ex}

\hspace{.8\funcindent}\begin{boxedminipage}{\funcwidth}

    \raggedright \textbf{fl\_winicontitle}(\textit{win}, \textit{title})

    \vspace{-1.5ex}

    \rule{\textwidth}{0.5\fboxrule}
\setlength{\parskip}{2ex}
    Changes only the icon title for the window.

\setlength{\parskip}{1ex}
      \textbf{Parameters}
      \vspace{-1ex}

      \begin{quote}
        \begin{Ventry}{xxxxx}

          \item[win]

          window id ({\textless}long\_pos{\textgreater})

          \item[title]

          icon title to be set ({\textless}string{\textgreater})

        \end{Ventry}

      \end{quote}

\textbf{Example:} fl\_winicontitle("My icon label")



\textbf{Status:} Tested + Doc + NoDemo = OK



    \end{boxedminipage}

    \label{xformslib:library:fl_winposition}
    \index{xformslib \textit{(package)}!xformslib.library \textit{(module)}!xformslib.library.fl\_winposition \textit{(function)}}

    \vspace{0.5ex}

\hspace{.8\funcindent}\begin{boxedminipage}{\funcwidth}

    \raggedright \textbf{fl\_winposition}(\textit{x}, \textit{y})

    \vspace{-1.5ex}

    \rule{\textwidth}{0.5\fboxrule}
\setlength{\parskip}{2ex}
    Sets the position of a window to be opened.

\setlength{\parskip}{1ex}
      \textbf{Parameters}
      \vspace{-1ex}

      \begin{quote}
        \begin{Ventry}{x}

          \item[x]

          horizontal position of window (upper-left corner) 
          ({\textless}int{\textgreater})

          \item[y]

          vertical position of window (upper-left corner) 
          ({\textless}int{\textgreater})

        \end{Ventry}

      \end{quote}

\textbf{Example:} fl\_winposition(140, 123)



\textbf{Status:} Tested + Doc + NoDemo = OK



    \end{boxedminipage}

    \label{xformslib:library:fl_winposition}
    \index{xformslib \textit{(package)}!xformslib.library \textit{(module)}!xformslib.library.fl\_winposition \textit{(function)}}

    \vspace{0.5ex}

\hspace{.8\funcindent}\begin{boxedminipage}{\funcwidth}

    \raggedright \textbf{fl\_pref\_winposition}(\textit{x}, \textit{y})

    \vspace{-1.5ex}

    \rule{\textwidth}{0.5\fboxrule}
\setlength{\parskip}{2ex}
    Sets the position of a window to be opened.

\setlength{\parskip}{1ex}
      \textbf{Parameters}
      \vspace{-1ex}

      \begin{quote}
        \begin{Ventry}{x}

          \item[x]

          horizontal position of window (upper-left corner) 
          ({\textless}int{\textgreater})

          \item[y]

          vertical position of window (upper-left corner) 
          ({\textless}int{\textgreater})

        \end{Ventry}

      \end{quote}

\textbf{Example:} fl\_winposition(140, 123)



\textbf{Status:} Tested + Doc + NoDemo = OK



    \end{boxedminipage}

    \label{xformslib:library:fl_winminsize}
    \index{xformslib \textit{(package)}!xformslib.library \textit{(module)}!xformslib.library.fl\_winminsize \textit{(function)}}

    \vspace{0.5ex}

\hspace{.8\funcindent}\begin{boxedminipage}{\funcwidth}

    \raggedright \textbf{fl\_winminsize}(\textit{win}, \textit{w}, \textit{h})

    \vspace{-1.5ex}

    \rule{\textwidth}{0.5\fboxrule}
\setlength{\parskip}{2ex}
    Sets a constraint for a resizable window whose size will be within a 
    range not less than minumum (to be used before calling fl\_winopen).

\setlength{\parskip}{1ex}
      \textbf{Parameters}
      \vspace{-1ex}

      \begin{quote}
        \begin{Ventry}{xxx}

          \item[win]

          window id to be set ({\textless}long\_pos{\textgreater})

          \item[w]

          minimum width of window in coord units 
          ({\textless}int{\textgreater})

          \item[h]

          minimum height of window in coord units 
          ({\textless}int{\textgreater})

        \end{Ventry}

      \end{quote}

\textbf{Example:} fl\_winminsize(win1, 500, 500)



\textbf{Status:} Tested + Doc + NoDemo = OK



    \end{boxedminipage}

    \label{xformslib:library:fl_winmaxsize}
    \index{xformslib \textit{(package)}!xformslib.library \textit{(module)}!xformslib.library.fl\_winmaxsize \textit{(function)}}

    \vspace{0.5ex}

\hspace{.8\funcindent}\begin{boxedminipage}{\funcwidth}

    \raggedright \textbf{fl\_winmaxsize}(\textit{win}, \textit{w}, \textit{h})

    \vspace{-1.5ex}

    \rule{\textwidth}{0.5\fboxrule}
\setlength{\parskip}{2ex}
    Sets a constraint for a resizable window whose size will be within a 
    range not bigger than maximum (before calling fl\_winopen).

\setlength{\parskip}{1ex}
      \textbf{Parameters}
      \vspace{-1ex}

      \begin{quote}
        \begin{Ventry}{xxx}

          \item[win]

          window id to be set ({\textless}long\_pos{\textgreater})

          \item[w]

          maximum width of window in coord units 
          ({\textless}int{\textgreater})

          \item[h]

          maximum height of window in coord units 
          ({\textless}int{\textgreater})

        \end{Ventry}

      \end{quote}

\textbf{Example:} fl\_winmaxsize(win1, 500, 500)



\textbf{Status:} Tested + Doc + NoDemo = OK



    \end{boxedminipage}

    \label{xformslib:library:fl_winaspect}
    \index{xformslib \textit{(package)}!xformslib.library \textit{(module)}!xformslib.library.fl\_winaspect \textit{(function)}}

    \vspace{0.5ex}

\hspace{.8\funcindent}\begin{boxedminipage}{\funcwidth}

    \raggedright \textbf{fl\_winaspect}(\textit{win}, \textit{x}, \textit{y})

    \vspace{-1.5ex}

    \rule{\textwidth}{0.5\fboxrule}
\setlength{\parskip}{2ex}
    Sets the aspect ratio of the window for later interactive resizing.

\setlength{\parskip}{1ex}
      \textbf{Parameters}
      \vspace{-1ex}

      \begin{quote}
        \begin{Ventry}{xxx}

          \item[win]

          window id to be set ({\textless}long\_pos{\textgreater})

          \item[x]

          horizontal aspect ratio in coord units 
          ({\textless}int{\textgreater})

          \item[y]

          vertical aspect ratio in coord units 
          ({\textless}int{\textgreater})

        \end{Ventry}

      \end{quote}

\textbf{Example:} fl\_winaspect(win0, 2, 4)



\textbf{Status:} Tested + Doc + NoDemo = OK



    \end{boxedminipage}

    \label{xformslib:library:fl_reset_winconstraints}
    \index{xformslib \textit{(package)}!xformslib.library \textit{(module)}!xformslib.library.fl\_reset\_winconstraints \textit{(function)}}

    \vspace{0.5ex}

\hspace{.8\funcindent}\begin{boxedminipage}{\funcwidth}

    \raggedright \textbf{fl\_reset\_winconstraints}(\textit{win})

    \vspace{-1.5ex}

    \rule{\textwidth}{0.5\fboxrule}
\setlength{\parskip}{2ex}
    Changes constraints (size and aspect ratio) on an active window.

\setlength{\parskip}{1ex}
      \textbf{Parameters}
      \vspace{-1ex}

      \begin{quote}
        \begin{Ventry}{xxx}

          \item[win]

          window to be reset ({\textless}long\_pos{\textgreater})

        \end{Ventry}

      \end{quote}

\textbf{Example:} fl\_reset\_constraints(win0)



\textbf{Status:} Tested + Doc + NoDemo = OK



    \end{boxedminipage}

    \label{xformslib:library:fl_winsize}
    \index{xformslib \textit{(package)}!xformslib.library \textit{(module)}!xformslib.library.fl\_winsize \textit{(function)}}

    \vspace{0.5ex}

\hspace{.8\funcindent}\begin{boxedminipage}{\funcwidth}

    \raggedright \textbf{fl\_winsize}(\textit{w}, \textit{h})

    \vspace{-1.5ex}

    \rule{\textwidth}{0.5\fboxrule}
\setlength{\parskip}{2ex}
    Sets the preferred window size (before calling fl\_winopen), and makes 
    the window non-resizeable.

\setlength{\parskip}{1ex}
      \textbf{Parameters}
      \vspace{-1ex}

      \begin{quote}
        \begin{Ventry}{x}

          \item[w]

          width in coord units ({\textless}int{\textgreater})

          \item[h]

          height in coord units ({\textless}int{\textgreater})

        \end{Ventry}

      \end{quote}

\textbf{Example:} fl\_winsize(700, 600)



\textbf{Status:} Tested + Doc + NoDemo = OK



    \end{boxedminipage}

    \label{xformslib:library:fl_winsize}
    \index{xformslib \textit{(package)}!xformslib.library \textit{(module)}!xformslib.library.fl\_winsize \textit{(function)}}

    \vspace{0.5ex}

\hspace{.8\funcindent}\begin{boxedminipage}{\funcwidth}

    \raggedright \textbf{fl\_pref\_winsize}(\textit{w}, \textit{h})

    \vspace{-1.5ex}

    \rule{\textwidth}{0.5\fboxrule}
\setlength{\parskip}{2ex}
    Sets the preferred window size (before calling fl\_winopen), and makes 
    the window non-resizeable.

\setlength{\parskip}{1ex}
      \textbf{Parameters}
      \vspace{-1ex}

      \begin{quote}
        \begin{Ventry}{x}

          \item[w]

          width in coord units ({\textless}int{\textgreater})

          \item[h]

          height in coord units ({\textless}int{\textgreater})

        \end{Ventry}

      \end{quote}

\textbf{Example:} fl\_winsize(700, 600)



\textbf{Status:} Tested + Doc + NoDemo = OK



    \end{boxedminipage}

    \label{xformslib:library:fl_initial_winsize}
    \index{xformslib \textit{(package)}!xformslib.library \textit{(module)}!xformslib.library.fl\_initial\_winsize \textit{(function)}}

    \vspace{0.5ex}

\hspace{.8\funcindent}\begin{boxedminipage}{\funcwidth}

    \raggedright \textbf{fl\_initial\_winsize}(\textit{w}, \textit{h})

    \vspace{-1.5ex}

    \rule{\textwidth}{0.5\fboxrule}
\setlength{\parskip}{2ex}
    Sets the preferred window size (before calling fl\_winopen).

\setlength{\parskip}{1ex}
      \textbf{Parameters}
      \vspace{-1ex}

      \begin{quote}
        \begin{Ventry}{x}

          \item[w]

          width in coord units ({\textless}int{\textgreater})

          \item[h]

          height in coord units ({\textless}int{\textgreater})

        \end{Ventry}

      \end{quote}

\textbf{Example:} fl\_initial\_winsize(700, 600)



\textbf{Status:} Tested + Doc + Demo = OK



    \end{boxedminipage}

    \label{xformslib:library:fl_initial_winstate}
    \index{xformslib \textit{(package)}!xformslib.library \textit{(module)}!xformslib.library.fl\_initial\_winstate \textit{(function)}}

    \vspace{0.5ex}

\hspace{.8\funcindent}\begin{boxedminipage}{\funcwidth}

    \raggedright \textbf{fl\_initial\_winstate}(\textit{state})

    \vspace{-1.5ex}

    \rule{\textwidth}{0.5\fboxrule}
\setlength{\parskip}{2ex}
    Sets initial state of the window.

\setlength{\parskip}{1ex}
      \textbf{Parameters}
      \vspace{-1ex}

      \begin{quote}
        \begin{Ventry}{xxxxx}

          \item[state]

          window state to be set ({\textless}int{\textgreater})

        \end{Ventry}

      \end{quote}

\textbf{Status:} Untested + NoDoc + NoDemo = NOT OK



    \end{boxedminipage}

    \label{xformslib:library:fl_create_colormap}
    \index{xformslib \textit{(package)}!xformslib.library \textit{(module)}!xformslib.library.fl\_create\_colormap \textit{(function)}}

    \vspace{0.5ex}

\hspace{.8\funcindent}\begin{boxedminipage}{\funcwidth}

    \raggedright \textbf{fl\_create\_colormap}(\textit{pXVisualInfo}, \textit{nfill})

    \vspace{-1.5ex}

    \rule{\textwidth}{0.5\fboxrule}
\setlength{\parskip}{2ex}
    Creates a colormap appropriate for a given visual to be used with a 
    canvas.

\setlength{\parskip}{1ex}
      \textbf{Parameters}
      \vspace{-1ex}

      \begin{quote}
        \begin{Ventry}{xxxxxxxxxxxx}

          \item[pXVisualInfo]

          pointer to xfdata.XVisualInfo

          \item[nfill]

          how many colors in the newly created colormap should be filled 
          with XForms' default colors (to avoid flashing effects) 
          ({\textless}int{\textgreater})

        \end{Ventry}

      \end{quote}

      \textbf{Return Value}
    \vspace{-1ex}

      \begin{quote}
      colormap

      \end{quote}

\textbf{Example:} 

\textbf{Status:} Untested + NoDoc + NoDemo = NOT OK



    \end{boxedminipage}

    \label{xformslib:library:fl_wingeometry}
    \index{xformslib \textit{(package)}!xformslib.library \textit{(module)}!xformslib.library.fl\_wingeometry \textit{(function)}}

    \vspace{0.5ex}

\hspace{.8\funcindent}\begin{boxedminipage}{\funcwidth}

    \raggedright \textbf{fl\_wingeometry}(\textit{x}, \textit{y}, \textit{w}, \textit{h})

    \vspace{-1.5ex}

    \rule{\textwidth}{0.5\fboxrule}
\setlength{\parskip}{2ex}
    Sets the initial geometry (position and size) of the window to be 
    opened; the window will not be resizable.

\setlength{\parskip}{1ex}
      \textbf{Parameters}
      \vspace{-1ex}

      \begin{quote}
        \begin{Ventry}{x}

          \item[x]

          horizontal position (upper-left corner) 
          ({\textless}int{\textgreater})

          \item[y]

          vertical position (upper-left corner) 
          ({\textless}int{\textgreater})

          \item[w]

          width in coord units ({\textless}int{\textgreater})

          \item[h]

          height in coord units ({\textless}int{\textgreater})

        \end{Ventry}

      \end{quote}

\textbf{Example:} fl\_wingeometry(192, 231, 450, 550)



\textbf{Status:} Tested + Doc + Demo = OK



    \end{boxedminipage}

    \label{xformslib:library:fl_wingeometry}
    \index{xformslib \textit{(package)}!xformslib.library \textit{(module)}!xformslib.library.fl\_wingeometry \textit{(function)}}

    \vspace{0.5ex}

\hspace{.8\funcindent}\begin{boxedminipage}{\funcwidth}

    \raggedright \textbf{fl\_pref\_wingeometry}(\textit{x}, \textit{y}, \textit{w}, \textit{h})

    \vspace{-1.5ex}

    \rule{\textwidth}{0.5\fboxrule}
\setlength{\parskip}{2ex}
    Sets the initial geometry (position and size) of the window to be 
    opened; the window will not be resizable.

\setlength{\parskip}{1ex}
      \textbf{Parameters}
      \vspace{-1ex}

      \begin{quote}
        \begin{Ventry}{x}

          \item[x]

          horizontal position (upper-left corner) 
          ({\textless}int{\textgreater})

          \item[y]

          vertical position (upper-left corner) 
          ({\textless}int{\textgreater})

          \item[w]

          width in coord units ({\textless}int{\textgreater})

          \item[h]

          height in coord units ({\textless}int{\textgreater})

        \end{Ventry}

      \end{quote}

\textbf{Example:} fl\_wingeometry(192, 231, 450, 550)



\textbf{Status:} Tested + Doc + Demo = OK



    \end{boxedminipage}

    \label{xformslib:library:fl_initial_wingeometry}
    \index{xformslib \textit{(package)}!xformslib.library \textit{(module)}!xformslib.library.fl\_initial\_wingeometry \textit{(function)}}

    \vspace{0.5ex}

\hspace{.8\funcindent}\begin{boxedminipage}{\funcwidth}

    \raggedright \textbf{fl\_initial\_wingeometry}(\textit{x}, \textit{y}, \textit{w}, \textit{h})

    \vspace{-1.5ex}

    \rule{\textwidth}{0.5\fboxrule}
\setlength{\parskip}{2ex}
    Sets the initial geometry (position and size) of the window to be 
    opened.

\setlength{\parskip}{1ex}
      \textbf{Parameters}
      \vspace{-1ex}

      \begin{quote}
        \begin{Ventry}{x}

          \item[x]

          horizontal position (upper-left corner) 
          ({\textless}int{\textgreater})

          \item[y]

          vertical position (upper-left corner) 
          ({\textless}int{\textgreater})

          \item[w]

          width in coord units ({\textless}int{\textgreater})

          \item[h]

          height in coord units ({\textless}int{\textgreater})

        \end{Ventry}

      \end{quote}

\textbf{Example:} fl\_initial\_wingeometry(192, 231, 450, 550)



\textbf{Status:} Tested + Doc + NoDemo = OK



    \end{boxedminipage}

    \label{xformslib:library:fl_noborder}
    \index{xformslib \textit{(package)}!xformslib.library \textit{(module)}!xformslib.library.fl\_noborder \textit{(function)}}

    \vspace{0.5ex}

\hspace{.8\funcindent}\begin{boxedminipage}{\funcwidth}

    \raggedright \textbf{fl\_noborder}()

    \vspace{-1.5ex}

    \rule{\textwidth}{0.5\fboxrule}
\setlength{\parskip}{2ex}
    Suppresses the window manager's decoration (before creating the 
    window).

\setlength{\parskip}{1ex}
\textbf{Example:} fl\_noborder()



\textbf{Status:} Tested + Doc + NoDemo = OK



    \end{boxedminipage}

    \label{xformslib:library:fl_transient}
    \index{xformslib \textit{(package)}!xformslib.library \textit{(module)}!xformslib.library.fl\_transient \textit{(function)}}

    \vspace{0.5ex}

\hspace{.8\funcindent}\begin{boxedminipage}{\funcwidth}

    \raggedright \textbf{fl\_transient}()

    \vspace{-1.5ex}

    \rule{\textwidth}{0.5\fboxrule}
\setlength{\parskip}{2ex}
    Makes a window a transient one (before creating the window).

\setlength{\parskip}{1ex}
\textbf{Example:} fl\_transient()



\textbf{Status:} Tested + Doc + NoDemo = OK



    \end{boxedminipage}

    \label{xformslib:library:fl_get_winsize}
    \index{xformslib \textit{(package)}!xformslib.library \textit{(module)}!xformslib.library.fl\_get\_winsize \textit{(function)}}

    \vspace{0.5ex}

\hspace{.8\funcindent}\begin{boxedminipage}{\funcwidth}

    \raggedright \textbf{fl\_get\_winsize}(\textit{win})

    \vspace{-1.5ex}

    \rule{\textwidth}{0.5\fboxrule}
\setlength{\parskip}{2ex}
    Returns the size of the specified window.

\setlength{\parskip}{1ex}
      \textbf{Parameters}
      \vspace{-1ex}

      \begin{quote}
        \begin{Ventry}{xxx}

          \item[win]

          window id to evaluate ({\textless}long\_pos{\textgreater})

        \end{Ventry}

      \end{quote}

      \textbf{Return Value}
    \vspace{-1ex}

      \begin{quote}
      width and height of window ({\textless}int{\textgreater}, 
      {\textless}int{\textgreater})

      {\it (type=w, h)}

      \end{quote}

\textbf{Example:} wid, hei = fl\_get\_winsize(win0)



\textbf{Attention:} API change from XForms - upstream was fl\_get\_winsize(win, w, h)



\textbf{Status:} Tested + Doc + NoDemo = OK



    \end{boxedminipage}

    \label{xformslib:library:fl_get_winorigin}
    \index{xformslib \textit{(package)}!xformslib.library \textit{(module)}!xformslib.library.fl\_get\_winorigin \textit{(function)}}

    \vspace{0.5ex}

\hspace{.8\funcindent}\begin{boxedminipage}{\funcwidth}

    \raggedright \textbf{fl\_get\_winorigin}(\textit{win})

    \vspace{-1.5ex}

    \rule{\textwidth}{0.5\fboxrule}
\setlength{\parskip}{2ex}
    Returns the origin (position) of the specified window.

\setlength{\parskip}{1ex}
      \textbf{Parameters}
      \vspace{-1ex}

      \begin{quote}
        \begin{Ventry}{xxx}

          \item[win]

          window id to evaluate ({\textless}long\_pos{\textgreater})

        \end{Ventry}

      \end{quote}

      \textbf{Return Value}
    \vspace{-1ex}

      \begin{quote}
      horizontal and vertical position of window 
      ({\textless}int{\textgreater}, {\textless}int{\textgreater})

      {\it (type=x, y)}

      \end{quote}

\textbf{Example:} xpos, ypos = fl\_get\_winorigin(win0)



\textbf{Attention:} API change from XForms - upstream was fl\_get\_winorigin(win, x, y)



\textbf{Status:} Tested + Doc + NoDemo = OK



    \end{boxedminipage}

    \label{xformslib:library:fl_get_wingeometry}
    \index{xformslib \textit{(package)}!xformslib.library \textit{(module)}!xformslib.library.fl\_get\_wingeometry \textit{(function)}}

    \vspace{0.5ex}

\hspace{.8\funcindent}\begin{boxedminipage}{\funcwidth}

    \raggedright \textbf{fl\_get\_wingeometry}(\textit{win})

    \vspace{-1.5ex}

    \rule{\textwidth}{0.5\fboxrule}
\setlength{\parskip}{2ex}
    Returns geometry (position and size) of a window.

\setlength{\parskip}{1ex}
      \textbf{Parameters}
      \vspace{-1ex}

      \begin{quote}
        \begin{Ventry}{xxx}

          \item[win]

          window id to evaluate ({\textless}long\_pos{\textgreater})

        \end{Ventry}

      \end{quote}

      \textbf{Return Value}
    \vspace{-1ex}

      \begin{quote}
      horizontal and vertical position, width and height of window 
      ({\textless}int{\textgreater}, {\textless}int{\textgreater}, 
      {\textless}int{\textgreater}, {\textless}int{\textgreater})

      {\it (type=x, y, w, h)}

      \end{quote}

\textbf{Example:} xpos, ypos, wid, hei = fl\_get\_wingeometry(win0)



\textbf{Attention:} API change from XForms - upstream was fl\_get\_wingeometry(win, x, y, w, h)



\textbf{Status:} Tested + Doc + NoDemo = OK



    \end{boxedminipage}

    \label{xformslib:library:fl_get_display}
    \index{xformslib \textit{(package)}!xformslib.library \textit{(module)}!xformslib.library.fl\_get\_display \textit{(function)}}

    \vspace{0.5ex}

\hspace{.8\funcindent}\begin{boxedminipage}{\funcwidth}

    \raggedright \textbf{fl\_get\_display}()

\setlength{\parskip}{2ex}
\setlength{\parskip}{1ex}
    \end{boxedminipage}

    \label{xformslib:library:FL_FormDisplay}
    \index{xformslib \textit{(package)}!xformslib.library \textit{(module)}!xformslib.library.FL\_FormDisplay \textit{(function)}}

    \vspace{0.5ex}

\hspace{.8\funcindent}\begin{boxedminipage}{\funcwidth}

    \raggedright \textbf{FL\_FormDisplay}(\textit{pForm})

\setlength{\parskip}{2ex}
\setlength{\parskip}{1ex}
    \end{boxedminipage}

    \label{xformslib:library:FL_ObjectDisplay}
    \index{xformslib \textit{(package)}!xformslib.library \textit{(module)}!xformslib.library.FL\_ObjectDisplay \textit{(function)}}

    \vspace{0.5ex}

\hspace{.8\funcindent}\begin{boxedminipage}{\funcwidth}

    \raggedright \textbf{FL\_ObjectDisplay}(\textit{pObject})

\setlength{\parskip}{2ex}
\setlength{\parskip}{1ex}
    \end{boxedminipage}

    \label{xformslib:library:FL_IS_CANVAS}
    \index{xformslib \textit{(package)}!xformslib.library \textit{(module)}!xformslib.library.FL\_IS\_CANVAS \textit{(function)}}

    \vspace{0.5ex}

\hspace{.8\funcindent}\begin{boxedminipage}{\funcwidth}

    \raggedright \textbf{FL\_IS\_CANVAS}(\textit{pObject})

\setlength{\parskip}{2ex}
\setlength{\parskip}{1ex}
    \end{boxedminipage}

    \label{xformslib:library:FL_ObjWin}
    \index{xformslib \textit{(package)}!xformslib.library \textit{(module)}!xformslib.library.FL\_ObjWin \textit{(function)}}

    \vspace{0.5ex}

\hspace{.8\funcindent}\begin{boxedminipage}{\funcwidth}

    \raggedright \textbf{FL\_ObjWin}(\textit{pObject})

    \vspace{-1.5ex}

    \rule{\textwidth}{0.5\fboxrule}
\setlength{\parskip}{2ex}
    Obtains the window id an object belongs to (for general use).

\setlength{\parskip}{1ex}
      \textbf{Parameters}
      \vspace{-1ex}

      \begin{quote}
        \begin{Ventry}{xxxxxxx}

          \item[pObject]

          object ({\textless}pointer to xfdata.FL\_OBJECT{\textgreater})

        \end{Ventry}

      \end{quote}

      \textbf{Return Value}
    \vspace{-1ex}

      \begin{quote}
      window id ({\textless}long\_pos{\textgreater})

      {\it (type=win)}

      \end{quote}

\textbf{Example:} wind = FL\_ObjWin(pobj)



\textbf{Status:} Tested + Doc + Demo = OK



    \end{boxedminipage}

    \label{xformslib:library:fl_get_real_object_window}
    \index{xformslib \textit{(package)}!xformslib.library \textit{(module)}!xformslib.library.fl\_get\_real\_object\_window \textit{(function)}}

    \vspace{0.5ex}

\hspace{.8\funcindent}\begin{boxedminipage}{\funcwidth}

    \raggedright \textbf{fl\_get\_real\_object\_window}(\textit{pObject})

    \vspace{-1.5ex}

    \rule{\textwidth}{0.5\fboxrule}
\setlength{\parskip}{2ex}
    Obtains the real window id an object belongs to (to be used for cursor 
    or pointer routines).

\setlength{\parskip}{1ex}
      \textbf{Parameters}
      \vspace{-1ex}

      \begin{quote}
        \begin{Ventry}{xxxxxxx}

          \item[pObject]

          object ({\textless}pointer to xfdata.FL\_OBJECT{\textgreater})

        \end{Ventry}

      \end{quote}

      \textbf{Return Value}
    \vspace{-1ex}

      \begin{quote}
      window id ({\textless}long\_pos{\textgreater})

      {\it (type=win)}

      \end{quote}

\textbf{Example:} wind = fl\_get\_real\_object\_window(pobj)



\textbf{Status:} Tested + Doc + NoDemo = OK



    \end{boxedminipage}

    \label{xformslib:library:FL_ObjWin}
    \index{xformslib \textit{(package)}!xformslib.library \textit{(module)}!xformslib.library.FL\_ObjWin \textit{(function)}}

    \vspace{0.5ex}

\hspace{.8\funcindent}\begin{boxedminipage}{\funcwidth}

    \raggedright \textbf{FL\_OBJECT\_WID}(\textit{pObject})

    \vspace{-1.5ex}

    \rule{\textwidth}{0.5\fboxrule}
\setlength{\parskip}{2ex}
    Obtains the window id an object belongs to (for general use).

\setlength{\parskip}{1ex}
      \textbf{Parameters}
      \vspace{-1ex}

      \begin{quote}
        \begin{Ventry}{xxxxxxx}

          \item[pObject]

          object ({\textless}pointer to xfdata.FL\_OBJECT{\textgreater})

        \end{Ventry}

      \end{quote}

      \textbf{Return Value}
    \vspace{-1ex}

      \begin{quote}
      window id ({\textless}long\_pos{\textgreater})

      {\it (type=win)}

      \end{quote}

\textbf{Example:} wind = FL\_ObjWin(pobj)



\textbf{Status:} Tested + Doc + Demo = OK



    \end{boxedminipage}

    \label{xformslib:library:fl_XNextEvent}
    \index{xformslib \textit{(package)}!xformslib.library \textit{(module)}!xformslib.library.fl\_XNextEvent \textit{(function)}}

    \vspace{0.5ex}

\hspace{.8\funcindent}\begin{boxedminipage}{\funcwidth}

    \raggedright \textbf{fl\_XNextEvent}(\textit{pXEvent})

    \vspace{-1.5ex}

    \rule{\textwidth}{0.5\fboxrule}
\setlength{\parskip}{2ex}
\setlength{\parskip}{1ex}
      \textbf{Parameters}
      \vspace{-1ex}

      \begin{quote}
        \begin{Ventry}{xxxxxxx}

          \item[pXEvent]

          pointer to XEvent

        \end{Ventry}

      \end{quote}

      \textbf{Return Value}
    \vspace{-1ex}

      \begin{quote}
      event num

      \end{quote}

\textbf{Status:} Untested + NoDoc + NoDemo = NOT OK



    \end{boxedminipage}

    \label{xformslib:library:fl_XPeekEvent}
    \index{xformslib \textit{(package)}!xformslib.library \textit{(module)}!xformslib.library.fl\_XPeekEvent \textit{(function)}}

    \vspace{0.5ex}

\hspace{.8\funcindent}\begin{boxedminipage}{\funcwidth}

    \raggedright \textbf{fl\_XPeekEvent}(\textit{pXEvent})

    \vspace{-1.5ex}

    \rule{\textwidth}{0.5\fboxrule}
\setlength{\parskip}{2ex}
\setlength{\parskip}{1ex}
      \textbf{Parameters}
      \vspace{-1ex}

      \begin{quote}
        \begin{Ventry}{xxxxxxx}

          \item[pXEvent]

          pointer to XEvent

        \end{Ventry}

      \end{quote}

      \textbf{Return Value}
    \vspace{-1ex}

      \begin{quote}
      event num

      \end{quote}

\textbf{Status:} Untested + NoDoc + NoDemo = NOT OK



    \end{boxedminipage}

    \label{xformslib:library:fl_XEventsQueued}
    \index{xformslib \textit{(package)}!xformslib.library \textit{(module)}!xformslib.library.fl\_XEventsQueued \textit{(function)}}

    \vspace{0.5ex}

\hspace{.8\funcindent}\begin{boxedminipage}{\funcwidth}

    \raggedright \textbf{fl\_XEventsQueued}(\textit{mode})

    \vspace{-1.5ex}

    \rule{\textwidth}{0.5\fboxrule}
\setlength{\parskip}{2ex}
\setlength{\parskip}{1ex}
      \textbf{Return Value}
    \vspace{-1ex}

      \begin{quote}
      event num

      \end{quote}

\textbf{Status:} Untested + NoDoc + NoDemo = NOT OK



    \end{boxedminipage}

    \label{xformslib:library:fl_XPutBackEvent}
    \index{xformslib \textit{(package)}!xformslib.library \textit{(module)}!xformslib.library.fl\_XPutBackEvent \textit{(function)}}

    \vspace{0.5ex}

\hspace{.8\funcindent}\begin{boxedminipage}{\funcwidth}

    \raggedright \textbf{fl\_XPutBackEvent}(\textit{pXEvent})

    \vspace{-1.5ex}

    \rule{\textwidth}{0.5\fboxrule}
\setlength{\parskip}{2ex}
\setlength{\parskip}{1ex}
      \textbf{Parameters}
      \vspace{-1ex}

      \begin{quote}
        \begin{Ventry}{xxxxxxx}

          \item[pXEvent]

          pointer to XEvent

        \end{Ventry}

      \end{quote}

\textbf{Status:} Untested + NoDoc + NoDemo = NOT OK



    \end{boxedminipage}

    \label{xformslib:library:fl_last_event}
    \index{xformslib \textit{(package)}!xformslib.library \textit{(module)}!xformslib.library.fl\_last\_event \textit{(function)}}

    \vspace{0.5ex}

\hspace{.8\funcindent}\begin{boxedminipage}{\funcwidth}

    \raggedright \textbf{fl\_last\_event}()

    \vspace{-1.5ex}

    \rule{\textwidth}{0.5\fboxrule}
\setlength{\parskip}{2ex}
\setlength{\parskip}{1ex}
      \textbf{Return Value}
    \vspace{-1ex}

      \begin{quote}
      pXEvent

      \end{quote}

\textbf{Status:} Untested + NoDoc + NoDemo = NOT OK



    \end{boxedminipage}

    \label{xformslib:library:fl_set_event_callback}
    \index{xformslib \textit{(package)}!xformslib.library \textit{(module)}!xformslib.library.fl\_set\_event\_callback \textit{(function)}}

    \vspace{0.5ex}

\hspace{.8\funcindent}\begin{boxedminipage}{\funcwidth}

    \raggedright \textbf{fl\_set\_event\_callback}(\textit{py\_AppEventCb}, \textit{vdata})

    \vspace{-1.5ex}

    \rule{\textwidth}{0.5\fboxrule}
\setlength{\parskip}{2ex}
    Setups an event callback routine. Whenever an event happens the 
    callback function is invoked with the event as the first argument. This
    assumes the application program solicits the events and further, the 
    callback routine should be prepared to handle all XEvent for all 
    non-form windows. The callback function normally should return 0 unless
    the event isn't for one of the applcation-managed windows. This routine
    will be called whenever an XEvent is pending for the application's own 
    window.

\setlength{\parskip}{1ex}
      \textbf{Parameters}
      \vspace{-1ex}

      \begin{quote}
        \begin{Ventry}{xxxxxxxxxxxxx}

          \item[py\_AppEventCb]

          python function callback, returning value

            {\it (type=\_\_ funcname (pXEvent, ptr\_void) -{\textgreater} num. \_\_)}

          \item[vdata]

          user data to be passed to function ({\textless}pointer to 
          void{\textgreater})

        \end{Ventry}

      \end{quote}

      \textbf{Return Value}
    \vspace{-1ex}

      \begin{quote}
      callback ({\textless}pointer to 
      xfdata.FL\_APPEVENT\_CB{\textgreater})

      {\it (type=event callback)}

      \end{quote}

\textbf{Example:}
\begin{quote}
  \begin{itemize}

  \item
    \setlength{\parskip}{0.6ex}
def eventcb(pxev, vdata):



  \item {\textbar}-{\textgreater}{\textbar} ...



  \item {\textbar}-{\textgreater}{\textbar} return 0



  \item fl\_set\_event\_callback(eventcb, None)



\end{itemize}

\end{quote}

\textbf{Status:} Tested + Doc + Demo = OK



    \end{boxedminipage}

    \label{xformslib:library:fl_set_idle_callback}
    \index{xformslib \textit{(package)}!xformslib.library \textit{(module)}!xformslib.library.fl\_set\_idle\_callback \textit{(function)}}

    \vspace{0.5ex}

\hspace{.8\funcindent}\begin{boxedminipage}{\funcwidth}

    \raggedright \textbf{fl\_set\_idle\_callback}(\textit{py\_AppEventCb}, \textit{vdata})

    \vspace{-1.5ex}

    \rule{\textwidth}{0.5\fboxrule}
\setlength{\parskip}{2ex}
    Registers an idle callback. Interaction with it  can used for periodic 
    tasks, e.g. rotating an image, checking the status of some external 
    device or application state etc. An idle callback is an application 
    function that is registered with the system and is called whenever 
    there are no events pending for forms (or application windows). If 
    called with a function as callback who does nothing, it removes idle 
    callback. The time interval between invocations of the idle callback 
    can vary considerably depending on interface activity and other 
    factors. A range between 50 and 300 msec should be expected.

\setlength{\parskip}{1ex}
      \textbf{Parameters}
      \vspace{-1ex}

      \begin{quote}
        \begin{Ventry}{xxxxxxxxxxxxx}

          \item[py\_AppEventCb]

          python function callback, returning unused value

            {\it (type=\_\_ funcname (pXEvent, ptr\_void) -{\textgreater} num \_\_)}

          \item[vdata]

          user data to be passed to function ({\textless}pointer to 
          void{\textgreater})

        \end{Ventry}

      \end{quote}

      \textbf{Return Value}
    \vspace{-1ex}

      \begin{quote}
      event callback func

      \end{quote}

\textbf{Example:}
\begin{quote}
  \begin{itemize}

  \item
    \setlength{\parskip}{0.6ex}
def idlecb(xev, userdata):



  \item {\textbar}-{\textgreater}{\textbar} ...



  \item {\textbar}-{\textgreater}{\textbar} return 0



  \item appevtcb = fl\_set\_idle\_callback(idlecb, None)



  \item def donothing\_idlecb(xev, userdata):



  \item {\textbar}-{\textgreater}{\textbar} pass



  \item removedcb = fl\_set\_idle\_callback(donothing\_idlecb, None)



\end{itemize}

\end{quote}

\textbf{Status:} Tested + Doc + NoDemo = OK



    \end{boxedminipage}

    \label{xformslib:library:fl_addto_selected_xevent}
    \index{xformslib \textit{(package)}!xformslib.library \textit{(module)}!xformslib.library.fl\_addto\_selected\_xevent \textit{(function)}}

    \vspace{0.5ex}

\hspace{.8\funcindent}\begin{boxedminipage}{\funcwidth}

    \raggedright \textbf{fl\_addto\_selected\_xevent}(\textit{win}, \textit{mask})

    \vspace{-1.5ex}

    \rule{\textwidth}{0.5\fboxrule}
\setlength{\parskip}{2ex}
    Adds solicited event masks on the fly without altering other masks 
    already selected.

\setlength{\parskip}{1ex}
      \textbf{Parameters}
      \vspace{-1ex}

      \begin{quote}
        \begin{Ventry}{xxxx}

          \item[win]

          window id ({\textless}long\_pos{\textgreater})

          \item[mask]

          event mask ({\textless}long{\textgreater})

        \end{Ventry}

      \end{quote}

      \textbf{Return Value}
    \vspace{-1ex}

      \begin{quote}
      num. ({\textless}long\_pos{\textgreater})

      {\it (type=num)}

      \end{quote}

\textbf{Example:} lnum = fl\_addto\_selected\_xevent(win7, xfdata.ButtonMotionMask)



\textbf{Status:} Tested + Doc + NoDemo = OK



    \end{boxedminipage}

    \label{xformslib:library:fl_remove_selected_xevent}
    \index{xformslib \textit{(package)}!xformslib.library \textit{(module)}!xformslib.library.fl\_remove\_selected\_xevent \textit{(function)}}

    \vspace{0.5ex}

\hspace{.8\funcindent}\begin{boxedminipage}{\funcwidth}

    \raggedright \textbf{fl\_remove\_selected\_xevent}(\textit{win}, \textit{mask})

    \vspace{-1.5ex}

    \rule{\textwidth}{0.5\fboxrule}
\setlength{\parskip}{2ex}
    Removes solicited event masks on the fly without altering other masks 
    already selected.

\setlength{\parskip}{1ex}
      \textbf{Parameters}
      \vspace{-1ex}

      \begin{quote}
        \begin{Ventry}{xxxx}

          \item[win]

          window id ({\textless}long\_pos{\textgreater})

          \item[mask]

          event mask ({\textless}long{\textgreater})

        \end{Ventry}

      \end{quote}

      \textbf{Return Value}
    \vspace{-1ex}

      \begin{quote}
      num. ({\textless}long\_pos{\textgreater})

      {\it (type=num)}

      \end{quote}

\textbf{Example:} lnum = fl\_remove\_selected\_xevent(win7, xfdata.ButtonMotionMask)



\textbf{Status:} Tested + Doc + NoDemo = OK



    \end{boxedminipage}

    \label{xformslib:library:fl_addto_selected_xevent}
    \index{xformslib \textit{(package)}!xformslib.library \textit{(module)}!xformslib.library.fl\_addto\_selected\_xevent \textit{(function)}}

    \vspace{0.5ex}

\hspace{.8\funcindent}\begin{boxedminipage}{\funcwidth}

    \raggedright \textbf{fl\_add\_selected\_xevent}(\textit{win}, \textit{mask})

    \vspace{-1.5ex}

    \rule{\textwidth}{0.5\fboxrule}
\setlength{\parskip}{2ex}
    Adds solicited event masks on the fly without altering other masks 
    already selected.

\setlength{\parskip}{1ex}
      \textbf{Parameters}
      \vspace{-1ex}

      \begin{quote}
        \begin{Ventry}{xxxx}

          \item[win]

          window id ({\textless}long\_pos{\textgreater})

          \item[mask]

          event mask ({\textless}long{\textgreater})

        \end{Ventry}

      \end{quote}

      \textbf{Return Value}
    \vspace{-1ex}

      \begin{quote}
      num. ({\textless}long\_pos{\textgreater})

      {\it (type=num)}

      \end{quote}

\textbf{Example:} lnum = fl\_addto\_selected\_xevent(win7, xfdata.ButtonMotionMask)



\textbf{Status:} Tested + Doc + NoDemo = OK



    \end{boxedminipage}

    \label{xformslib:library:fl_set_idle_delta}
    \index{xformslib \textit{(package)}!xformslib.library \textit{(module)}!xformslib.library.fl\_set\_idle\_delta \textit{(function)}}

    \vspace{0.5ex}

\hspace{.8\funcindent}\begin{boxedminipage}{\funcwidth}

    \raggedright \textbf{fl\_set\_idle\_delta}(\textit{msec})

    \vspace{-1.5ex}

    \rule{\textwidth}{0.5\fboxrule}
\setlength{\parskip}{2ex}
    Changes what the library considers to be "idle". Be aware that under 
    some conditions ad idle callback can be called sooner than the minimum 
    interval; if the timing of the idle callback is of concerned, timeouts 
    should be used.

\setlength{\parskip}{1ex}
      \textbf{Parameters}
      \vspace{-1ex}

      \begin{quote}
        \begin{Ventry}{xxxx}

          \item[msec]

          minimum time interval of inactivity, after which the main loop is
          considered to be in idle state ({\textless}long{\textgreater})

        \end{Ventry}

      \end{quote}

\textbf{Example:} fl\_set\_idle\_delta(800)



\textbf{Status:} Tested + Doc + NoDemo = OK



    \end{boxedminipage}

    \label{xformslib:library:fl_add_event_callback}
    \index{xformslib \textit{(package)}!xformslib.library \textit{(module)}!xformslib.library.fl\_add\_event\_callback \textit{(function)}}

    \vspace{0.5ex}

\hspace{.8\funcindent}\begin{boxedminipage}{\funcwidth}

    \raggedright \textbf{fl\_add\_event\_callback}(\textit{win}, \textit{evttype}, \textit{py\_AppEventCb}, \textit{vdata})

    \vspace{-1.5ex}

    \rule{\textwidth}{0.5\fboxrule}
\setlength{\parskip}{2ex}
    Adds an event handler for a window. Manipulates the event callback 
    functions for the window specified, which will be called when an event 
    of specified type is pending for the window. It does not solicit any 
    event for the caller, i.e. the XForms library assumes the caller opens 
    the window and solicits all events before calling these routines.

\setlength{\parskip}{1ex}
      \textbf{Parameters}
      \vspace{-1ex}

      \begin{quote}
        \begin{Ventry}{xxxxxxxxxxxxx}

          \item[win]

          window id to add event handler to 
          ({\textless}long\_pos{\textgreater})

          \item[evttype]

          event type number. 0 signifies that a callback for all event for 
          window ({\textless}int{\textgreater})

          \item[py\_AppEventCb]

          python function callback, returning value

            {\it (type=\_\_ funcname (pXEvent, ptr\_void) -{\textgreater} num. \_\_)}

          \item[vdata]

          user data to be passed to function ({\textless}pointer to 
          void{\textgreater})

        \end{Ventry}

      \end{quote}

      \textbf{Return Value}
    \vspace{-1ex}

      \begin{quote}
      callback ({\textless}pointer to 
      xfdata.FL\_APPEVENT\_CB{\textgreater})

      {\it (type=event callback)}

      \end{quote}

\textbf{Example:}
\begin{quote}
  \begin{itemize}

  \item
    \setlength{\parskip}{0.6ex}
def eventcb(pxev, vdata):



  \item {\textbar}-{\textgreater}{\textbar}  ...



  \item {\textbar}-{\textgreater}{\textbar} return 0



  \item fl\_add\_event\_callback(win2, 0, eventcb, None)



\end{itemize}

\end{quote}

\textbf{Status:} Tested + Doc + NoDemo = OK



    \end{boxedminipage}

    \label{xformslib:library:fl_remove_event_callback}
    \index{xformslib \textit{(package)}!xformslib.library \textit{(module)}!xformslib.library.fl\_remove\_event\_callback \textit{(function)}}

    \vspace{0.5ex}

\hspace{.8\funcindent}\begin{boxedminipage}{\funcwidth}

    \raggedright \textbf{fl\_remove\_event\_callback}(\textit{win}, \textit{evttype})

    \vspace{-1.5ex}

    \rule{\textwidth}{0.5\fboxrule}
\setlength{\parskip}{2ex}
    Removes one or all event callbacks for a window and for an event of 
    specified type. May be called with for a window for which no event 
    callbacks have been set.

\setlength{\parskip}{1ex}
      \textbf{Parameters}
      \vspace{-1ex}

      \begin{quote}
        \begin{Ventry}{xxxxxxx}

          \item[win]

          window id ({\textless}long\_pos{\textgreater})

          \item[evttype]

          event type number ({\textless}int{\textgreater})

        \end{Ventry}

      \end{quote}

\textbf{Example:} fl\_remove\_event\_callback(win2, 0)



\textbf{Status:} Tested + Doc + NoDemo = OK



    \end{boxedminipage}

    \label{xformslib:library:fl_activate_event_callbacks}
    \index{xformslib \textit{(package)}!xformslib.library \textit{(module)}!xformslib.library.fl\_activate\_event\_callbacks \textit{(function)}}

    \vspace{0.5ex}

\hspace{.8\funcindent}\begin{boxedminipage}{\funcwidth}

    \raggedright \textbf{fl\_activate\_event\_callbacks}(\textit{win})

    \vspace{-1.5ex}

    \rule{\textwidth}{0.5\fboxrule}
\setlength{\parskip}{2ex}
    Handles event solicitation. Activates the default mapping of events to 
    event masks built-in in the XForms Library, and causes the system to 
    solicit the events for you. Note however, the mapping of events to 
    masks are not unique and depending on applications, the default mapping
    may or may not be the one you want.

\setlength{\parskip}{1ex}
      \textbf{Parameters}
      \vspace{-1ex}

      \begin{quote}
        \begin{Ventry}{xxx}

          \item[win]

          window whose events are referred to 
          ({\textless}long\_pos{\textgreater})

        \end{Ventry}

      \end{quote}

\textbf{Example:} fl\_activate\_event\_callback(win3)



\textbf{Status:} Tested + Doc + NoDemo = OK



    \end{boxedminipage}

    \label{xformslib:library:fl_print_xevent_name}
    \index{xformslib \textit{(package)}!xformslib.library \textit{(module)}!xformslib.library.fl\_print\_xevent\_name \textit{(function)}}

    \vspace{0.5ex}

\hspace{.8\funcindent}\begin{boxedminipage}{\funcwidth}

    \raggedright \textbf{fl\_print\_xevent\_name}(\textit{where}, \textit{pXEvent})

    \vspace{-1.5ex}

    \rule{\textwidth}{0.5\fboxrule}
\setlength{\parskip}{2ex}
    Print the name of an XEvent and some other infos.

\setlength{\parskip}{1ex}
      \textbf{Parameters}
      \vspace{-1ex}

      \begin{quote}
        \begin{Ventry}{xxxxxxx}

          \item[where]

          can indicate where this function is called 
          ({\textless}string{\textgreater})

          \item[pXEvent]

          event ({\textless}pointer to xfdata.XEvent{\textgreater})

        \end{Ventry}

      \end{quote}

      \textbf{Return Value}
    \vspace{-1ex}

      \begin{quote}
      event ({\textless}pointer to xfdata.XEvent{\textgreater})

      {\it (type=pXEvent)}

      \end{quote}

\textbf{Example:} pxev = fl\_print\_xevent\_name("from whatever.py", pxev)



\textbf{Status:} Tested + Doc + NoDemo = OK



    \end{boxedminipage}

    \label{xformslib:library:fl_XFlush}
    \index{xformslib \textit{(package)}!xformslib.library \textit{(module)}!xformslib.library.fl\_XFlush \textit{(function)}}

    \vspace{0.5ex}

\hspace{.8\funcindent}\begin{boxedminipage}{\funcwidth}

    \raggedright \textbf{fl\_XFlush}()

    \vspace{-1.5ex}

    \rule{\textwidth}{0.5\fboxrule}
\setlength{\parskip}{2ex}
    Flushes the output buffer. Convenience replacement for X11 XFlush()

\setlength{\parskip}{1ex}
\textbf{Example:} fl\_XFlush()



\textbf{Status:} Tested + Doc + Demo = OK



    \end{boxedminipage}

    \label{xformslib:library:metakey_down}
    \index{xformslib \textit{(package)}!xformslib.library \textit{(module)}!xformslib.library.metakey\_down \textit{(function)}}

    \vspace{0.5ex}

\hspace{.8\funcindent}\begin{boxedminipage}{\funcwidth}

    \raggedright \textbf{metakey\_down}(\textit{mask})

\setlength{\parskip}{2ex}
\setlength{\parskip}{1ex}
    \end{boxedminipage}

    \label{xformslib:library:shiftkey_down}
    \index{xformslib \textit{(package)}!xformslib.library \textit{(module)}!xformslib.library.shiftkey\_down \textit{(function)}}

    \vspace{0.5ex}

\hspace{.8\funcindent}\begin{boxedminipage}{\funcwidth}

    \raggedright \textbf{shiftkey\_down}(\textit{mask})

\setlength{\parskip}{2ex}
\setlength{\parskip}{1ex}
    \end{boxedminipage}

    \label{xformslib:library:controlkey_down}
    \index{xformslib \textit{(package)}!xformslib.library \textit{(module)}!xformslib.library.controlkey\_down \textit{(function)}}

    \vspace{0.5ex}

\hspace{.8\funcindent}\begin{boxedminipage}{\funcwidth}

    \raggedright \textbf{controlkey\_down}(\textit{mask})

\setlength{\parskip}{2ex}
\setlength{\parskip}{1ex}
    \end{boxedminipage}

    \label{xformslib:library:button_down}
    \index{xformslib \textit{(package)}!xformslib.library \textit{(module)}!xformslib.library.button\_down \textit{(function)}}

    \vspace{0.5ex}

\hspace{.8\funcindent}\begin{boxedminipage}{\funcwidth}

    \raggedright \textbf{button\_down}(\textit{mask})

\setlength{\parskip}{2ex}
\setlength{\parskip}{1ex}
    \end{boxedminipage}

    \label{xformslib:library:fl_initialize}
    \index{xformslib \textit{(package)}!xformslib.library \textit{(module)}!xformslib.library.fl\_initialize \textit{(function)}}

    \vspace{0.5ex}

\hspace{.8\funcindent}\begin{boxedminipage}{\funcwidth}

    \raggedright \textbf{fl\_initialize}(\textit{numargs}, \textit{argslist}, \textit{appname}, \textit{appoptions}, \textit{nappopts})

    \vspace{-1.5ex}

    \rule{\textwidth}{0.5\fboxrule}
\setlength{\parskip}{2ex}
    Initializes XForms library. It should always be called before any other
    calls to the XForms Library (except fl\_set\_defaults() and a few other
    functions that alter some of the defaults of the library. Command line 
    arguments are NOT supported here, but you can always set most of 
    parameters with relative functions.

\setlength{\parskip}{1ex}
      \textbf{Parameters}
      \vspace{-1ex}

      \begin{quote}
        \begin{Ventry}{xxxxxxxxxx}

          \item[numargs]

          number of arguments passed to command line, unused in python 
          ({\textless}int{\textgreater})

          \item[argslist]

          arguments passed to command line, unused in python 
          ({\textless}list of string{\textgreater})

          \item[appname]

          application class name ({\textless}string{\textgreater})

          \item[appoptions]

          options passed, instance of xfdata.FL\_CMD\_OPT

          \item[nappopts]

          number of options ({\textless}int{\textgreater})

        \end{Ventry}

      \end{quote}

      \textbf{Return Value}
    \vspace{-1ex}

      \begin{quote}
      display ({\textless}pointer to xfdata.XDisplay{\textgreater}) or None
      (on falilure, if a connection couldn't be made)

      {\it (type=pDisplay)}

      \end{quote}

\textbf{Example:}
\begin{quote}
  \begin{itemize}

  \item
    \setlength{\parskip}{0.6ex}
import sys



  \item fl\_initialize(len(sys.argv), sys.argv, "MyFormDemo", 0, 0)



\end{itemize}

\end{quote}

\textbf{Status:} HalfTested + Doc + Demo = HALF OK (not for command line args)



    \end{boxedminipage}

    \label{xformslib:library:fl_finish}
    \index{xformslib \textit{(package)}!xformslib.library \textit{(module)}!xformslib.library.fl\_finish \textit{(function)}}

    \vspace{0.5ex}

\hspace{.8\funcindent}\begin{boxedminipage}{\funcwidth}

    \raggedright \textbf{fl\_finish}()

    \vspace{-1.5ex}

    \rule{\textwidth}{0.5\fboxrule}
\setlength{\parskip}{2ex}
    It is a final cleanup routine, restores all X server defaults, shuts 
    down the connection and frees dynamically allocated memory.

\setlength{\parskip}{1ex}
\textbf{Example:} fl\_finish()



\textbf{Status:} Tested + Doc + Demo = OK



    \end{boxedminipage}

    \label{xformslib:library:fl_get_resource}
    \index{xformslib \textit{(package)}!xformslib.library \textit{(module)}!xformslib.library.fl\_get\_resource \textit{(function)}}

    \vspace{0.5ex}

\hspace{.8\funcindent}\begin{boxedminipage}{\funcwidth}

    \raggedright \textbf{fl\_get\_resource}(\textit{rname}, \textit{cname}, \textit{dtype}, \textit{defval}, \textit{val}, \textit{size})

    \vspace{-1.5ex}

    \rule{\textwidth}{0.5\fboxrule}
\setlength{\parskip}{2ex}
\setlength{\parskip}{1ex}
      \textbf{Parameters}
      \vspace{-1ex}

      \begin{quote}
        \begin{Ventry}{xxxxxx}

          \item[rname]

          complete resource name specification (minus the application name)
          and should not contain wildcards of any kind 
          ({\textless}string{\textgreater})

          \item[cname]

          complete resource class specification (minus the application 
          name) and should not contain wildcards of any kind 
          ({\textless}string{\textgreater})

          \item[dtype]

          type of resource ({\textless}int{\textgreater})

            {\it (type=(from xfdata module) FL\_NONE, FL\_SHORT, FL\_BOOL, FL\_INT, FL\_LONG, 
FL\_FLOAT, FL\_STRING)}

          \item[defval]

          ({\textless}string{\textgreater})

          \item[val]

          ({\textless}pointer to void{\textgreater})

          \item[size]

          number of bytes, used only if dtype is FL\_STRING 
          ({\textless}int{\textgreater})

        \end{Ventry}

      \end{quote}

      \textbf{Return Value}
    \vspace{-1ex}

      \begin{quote}
      string representation of the resource value 
      ({\textless}string{\textgreater})

      {\it (type=string)}

      \end{quote}

\textbf{Example:} 

\textbf{Status:} Untested + NoDoc + NoDemo = NOT OK



    \end{boxedminipage}

    \label{xformslib:library:fl_set_resource}
    \index{xformslib \textit{(package)}!xformslib.library \textit{(module)}!xformslib.library.fl\_set\_resource \textit{(function)}}

    \vspace{0.5ex}

\hspace{.8\funcindent}\begin{boxedminipage}{\funcwidth}

    \raggedright \textbf{fl\_set\_resource}(\textit{resstr}, \textit{val})

    \vspace{-1.5ex}

    \rule{\textwidth}{0.5\fboxrule}
\setlength{\parskip}{2ex}
    Changes some of the built-in button labels with proper resource names.

\setlength{\parskip}{1ex}
      \textbf{Parameters}
      \vspace{-1ex}

      \begin{quote}
        \begin{Ventry}{xxxxxx}

          \item[resstr]

          resource name

          \item[val]

          new string value for resource

        \end{Ventry}

      \end{quote}

\textbf{Status:} Tested + NoDoc + Demo = OK



    \end{boxedminipage}

    \label{xformslib:library:fl_get_app_resources}
    \index{xformslib \textit{(package)}!xformslib.library \textit{(module)}!xformslib.library.fl\_get\_app\_resources \textit{(function)}}

    \vspace{0.5ex}

\hspace{.8\funcindent}\begin{boxedminipage}{\funcwidth}

    \raggedright \textbf{fl\_get\_app\_resources}(\textit{pResource}, \textit{n})

    \vspace{-1.5ex}

    \rule{\textwidth}{0.5\fboxrule}
\setlength{\parskip}{2ex}
\setlength{\parskip}{1ex}
\textbf{Status:} Untested + NoDoc + NoDemo = NOT OK



    \end{boxedminipage}

    \label{xformslib:library:fl_set_graphics_mode}
    \index{xformslib \textit{(package)}!xformslib.library \textit{(module)}!xformslib.library.fl\_set\_graphics\_mode \textit{(function)}}

    \vspace{0.5ex}

\hspace{.8\funcindent}\begin{boxedminipage}{\funcwidth}

    \raggedright \textbf{fl\_set\_graphics\_mode}(\textit{mode}, \textit{doublebuf})

    \vspace{-1.5ex}

    \rule{\textwidth}{0.5\fboxrule}
\setlength{\parskip}{2ex}
\setlength{\parskip}{1ex}
\textbf{Status:} Untested + NoDoc + NoDemo = NOT OK



    \end{boxedminipage}

    \label{xformslib:library:fl_set_visualID}
    \index{xformslib \textit{(package)}!xformslib.library \textit{(module)}!xformslib.library.fl\_set\_visualID \textit{(function)}}

    \vspace{0.5ex}

\hspace{.8\funcindent}\begin{boxedminipage}{\funcwidth}

    \raggedright \textbf{fl\_set\_visualID}(\textit{idnum})

    \vspace{-1.5ex}

    \rule{\textwidth}{0.5\fboxrule}
\setlength{\parskip}{2ex}
\setlength{\parskip}{1ex}
\textbf{Status:} Untested + NoDoc + NoDemo = NOT OK



    \end{boxedminipage}

    \label{xformslib:library:fl_keysym_pressed}
    \index{xformslib \textit{(package)}!xformslib.library \textit{(module)}!xformslib.library.fl\_keysym\_pressed \textit{(function)}}

    \vspace{0.5ex}

\hspace{.8\funcindent}\begin{boxedminipage}{\funcwidth}

    \raggedright \textbf{fl\_keysym\_pressed}(\textit{keysym})

    \vspace{-1.5ex}

    \rule{\textwidth}{0.5\fboxrule}
\setlength{\parskip}{2ex}
\setlength{\parskip}{1ex}
      \textbf{Return Value}
    \vspace{-1ex}

      \begin{quote}
      num

      \end{quote}

\textbf{Status:} Untested + NoDoc + NoDemo = NOT OK



    \end{boxedminipage}

    \label{xformslib:library:fl_keysym_pressed}
    \index{xformslib \textit{(package)}!xformslib.library \textit{(module)}!xformslib.library.fl\_keysym\_pressed \textit{(function)}}

    \vspace{0.5ex}

\hspace{.8\funcindent}\begin{boxedminipage}{\funcwidth}

    \raggedright \textbf{fl\_keypressed}(\textit{keysym})

    \vspace{-1.5ex}

    \rule{\textwidth}{0.5\fboxrule}
\setlength{\parskip}{2ex}
\setlength{\parskip}{1ex}
      \textbf{Return Value}
    \vspace{-1ex}

      \begin{quote}
      num

      \end{quote}

\textbf{Status:} Untested + NoDoc + NoDemo = NOT OK



    \end{boxedminipage}

    \label{xformslib:library:fl_set_defaults}
    \index{xformslib \textit{(package)}!xformslib.library \textit{(module)}!xformslib.library.fl\_set\_defaults \textit{(function)}}

    \vspace{0.5ex}

\hspace{.8\funcindent}\begin{boxedminipage}{\funcwidth}

    \raggedright \textbf{fl\_set\_defaults}(\textit{mask}, \textit{pIopt})

    \vspace{-1.5ex}

    \rule{\textwidth}{0.5\fboxrule}
\setlength{\parskip}{2ex}
\setlength{\parskip}{1ex}
\textbf{Status:} Untested + NoDoc + NoDemo = NOT OK



    \end{boxedminipage}

    \label{xformslib:library:fl_set_tabstop}
    \index{xformslib \textit{(package)}!xformslib.library \textit{(module)}!xformslib.library.fl\_set\_tabstop \textit{(function)}}

    \vspace{0.5ex}

\hspace{.8\funcindent}\begin{boxedminipage}{\funcwidth}

    \raggedright \textbf{fl\_set\_tabstop}(\textit{strng})

    \vspace{-1.5ex}

    \rule{\textwidth}{0.5\fboxrule}
\setlength{\parskip}{2ex}
    Adjusts the distance by setting the tab stops. For proportional font, 
    substituting tabs with spaces is not always appropriate because this 
    most likely will fail to align text properly. Instead, a tab is treated
    as an absolute measure of distance, in pixels, and a tab stop will 
    always end at multiples of this distance. The default is "aaaaaaaa", 
    i.e. eight 'a's.

\setlength{\parskip}{1ex}
      \textbf{Parameters}
      \vspace{-1ex}

      \begin{quote}
        \begin{Ventry}{xxxxx}

          \item[strng]

          text string whose width in pixel is to be used as the tab length.
          The font used to calculate the width is the same font that is 
          used to render the string in which the tab is embedded 
          ({\textless}strng{\textgreater})

        \end{Ventry}

      \end{quote}

\textbf{Example:} fl\_set\_tabstop("aaaa")



\textbf{Status:} Tested + Doc + NoDemo = OK



    \end{boxedminipage}

    \label{xformslib:library:fl_get_defaults}
    \index{xformslib \textit{(package)}!xformslib.library \textit{(module)}!xformslib.library.fl\_get\_defaults \textit{(function)}}

    \vspace{0.5ex}

\hspace{.8\funcindent}\begin{boxedminipage}{\funcwidth}

    \raggedright \textbf{fl\_get\_defaults}()

    \vspace{-1.5ex}

    \rule{\textwidth}{0.5\fboxrule}
\setlength{\parskip}{2ex}
    Return program defaults from the resource database.

\setlength{\parskip}{1ex}
      \textbf{Return Value}
    \vspace{-1ex}

      \begin{quote}
      instance of xfdata.FL\_IOPT

      {\it (type=Iopt)}

      \end{quote}

\textbf{Example:} defprgres = fl\_get\_defaults()



\textbf{Attention:} API change from XForms - upstream was fl\_get\_defaults(pIopt)



\textbf{Status:} Tested + Doc + NoDemo = OK



    \end{boxedminipage}

    \label{xformslib:library:fl_get_visual_depth}
    \index{xformslib \textit{(package)}!xformslib.library \textit{(module)}!xformslib.library.fl\_get\_visual\_depth \textit{(function)}}

    \vspace{0.5ex}

\hspace{.8\funcindent}\begin{boxedminipage}{\funcwidth}

    \raggedright \textbf{fl\_get\_visual\_depth}()

    \vspace{-1.5ex}

    \rule{\textwidth}{0.5\fboxrule}
\setlength{\parskip}{2ex}
    Returns the visual depth.

\setlength{\parskip}{1ex}
      \textbf{Return Value}
    \vspace{-1ex}

      \begin{quote}
      visual depth for current mode ({\textless}int{\textgreater})

      {\it (type=depth num)}

      \end{quote}

\textbf{Example:} curdepth = fl\_get\_visual\_depth()



\textbf{Status:} Tested + Doc + Demo = OK



    \end{boxedminipage}

    \label{xformslib:library:fl_vclass_name}
    \index{xformslib \textit{(package)}!xformslib.library \textit{(module)}!xformslib.library.fl\_vclass\_name \textit{(function)}}

    \vspace{0.5ex}

\hspace{.8\funcindent}\begin{boxedminipage}{\funcwidth}

    \raggedright \textbf{fl\_vclass\_name}(\textit{n})

    \vspace{-1.5ex}

    \rule{\textwidth}{0.5\fboxrule}
\setlength{\parskip}{2ex}
\setlength{\parskip}{1ex}
      \textbf{Return Value}
    \vspace{-1ex}

      \begin{quote}
      name string

      \end{quote}

\textbf{Status:} Untested + NoDoc + NoDemo = NOT OK



    \end{boxedminipage}

    \label{xformslib:library:fl_vclass_val}
    \index{xformslib \textit{(package)}!xformslib.library \textit{(module)}!xformslib.library.fl\_vclass\_val \textit{(function)}}

    \vspace{0.5ex}

\hspace{.8\funcindent}\begin{boxedminipage}{\funcwidth}

    \raggedright \textbf{fl\_vclass\_val}(\textit{val})

    \vspace{-1.5ex}

    \rule{\textwidth}{0.5\fboxrule}
\setlength{\parskip}{2ex}
\setlength{\parskip}{1ex}
      \textbf{Return Value}
    \vspace{-1ex}

      \begin{quote}
      num

      \end{quote}

\textbf{Status:} Untested + NoDoc + NoDemo = NOT OK



    \end{boxedminipage}

    \label{xformslib:library:fl_set_ul_property}
    \index{xformslib \textit{(package)}!xformslib.library \textit{(module)}!xformslib.library.fl\_set\_ul\_property \textit{(function)}}

    \vspace{0.5ex}

\hspace{.8\funcindent}\begin{boxedminipage}{\funcwidth}

    \raggedright \textbf{fl\_set\_ul\_property}(\textit{prop}, \textit{thickness})

    \vspace{-1.5ex}

    \rule{\textwidth}{0.5\fboxrule}
\setlength{\parskip}{2ex}
\setlength{\parskip}{1ex}
\textbf{Status:} Untested + NoDoc + NoDemo = NOT OK



    \end{boxedminipage}

    \label{xformslib:library:fl_set_clipping}
    \index{xformslib \textit{(package)}!xformslib.library \textit{(module)}!xformslib.library.fl\_set\_clipping \textit{(function)}}

    \vspace{0.5ex}

\hspace{.8\funcindent}\begin{boxedminipage}{\funcwidth}

    \raggedright \textbf{fl\_set\_clipping}(\textit{x}, \textit{y}, \textit{w}, \textit{h})

    \vspace{-1.5ex}

    \rule{\textwidth}{0.5\fboxrule}
\setlength{\parskip}{2ex}
\setlength{\parskip}{1ex}
\textbf{Status:} Untested + NoDoc + NoDemo = NOT OK



    \end{boxedminipage}

    \label{xformslib:library:fl_set_gc_clipping}
    \index{xformslib \textit{(package)}!xformslib.library \textit{(module)}!xformslib.library.fl\_set\_gc\_clipping \textit{(function)}}

    \vspace{0.5ex}

\hspace{.8\funcindent}\begin{boxedminipage}{\funcwidth}

    \raggedright \textbf{fl\_set\_gc\_clipping}(\textit{gc}, \textit{x}, \textit{y}, \textit{w}, \textit{h})

    \vspace{-1.5ex}

    \rule{\textwidth}{0.5\fboxrule}
\setlength{\parskip}{2ex}
\setlength{\parskip}{1ex}
\textbf{Status:} Untested + NoDoc + NoDemo = NOT OK



    \end{boxedminipage}

    \label{xformslib:library:fl_unset_gc_clipping}
    \index{xformslib \textit{(package)}!xformslib.library \textit{(module)}!xformslib.library.fl\_unset\_gc\_clipping \textit{(function)}}

    \vspace{0.5ex}

\hspace{.8\funcindent}\begin{boxedminipage}{\funcwidth}

    \raggedright \textbf{fl\_unset\_gc\_clipping}(\textit{gc})

    \vspace{-1.5ex}

    \rule{\textwidth}{0.5\fboxrule}
\setlength{\parskip}{2ex}
\setlength{\parskip}{1ex}
\textbf{Status:} Untested + NoDoc + NoDemo = NOT OK



    \end{boxedminipage}

    \label{xformslib:library:fl_set_clippings}
    \index{xformslib \textit{(package)}!xformslib.library \textit{(module)}!xformslib.library.fl\_set\_clippings \textit{(function)}}

    \vspace{0.5ex}

\hspace{.8\funcindent}\begin{boxedminipage}{\funcwidth}

    \raggedright \textbf{fl\_set\_clippings}(\textit{pRect}, \textit{n})

    \vspace{-1.5ex}

    \rule{\textwidth}{0.5\fboxrule}
\setlength{\parskip}{2ex}
\setlength{\parskip}{1ex}
\textbf{Status:} Untested + NoDoc + NoDemo = NOT OK



    \end{boxedminipage}

    \label{xformslib:library:fl_unset_clipping}
    \index{xformslib \textit{(package)}!xformslib.library \textit{(module)}!xformslib.library.fl\_unset\_clipping \textit{(function)}}

    \vspace{0.5ex}

\hspace{.8\funcindent}\begin{boxedminipage}{\funcwidth}

    \raggedright \textbf{fl\_unset\_clipping}()

    \vspace{-1.5ex}

    \rule{\textwidth}{0.5\fboxrule}
\setlength{\parskip}{2ex}
\setlength{\parskip}{1ex}
\textbf{Status:} Untested + NoDoc + NoDemo = NOT OK



    \end{boxedminipage}

    \label{xformslib:library:fl_set_text_clipping}
    \index{xformslib \textit{(package)}!xformslib.library \textit{(module)}!xformslib.library.fl\_set\_text\_clipping \textit{(function)}}

    \vspace{0.5ex}

\hspace{.8\funcindent}\begin{boxedminipage}{\funcwidth}

    \raggedright \textbf{fl\_set\_text\_clipping}(\textit{x}, \textit{y}, \textit{w}, \textit{h})

    \vspace{-1.5ex}

    \rule{\textwidth}{0.5\fboxrule}
\setlength{\parskip}{2ex}
\setlength{\parskip}{1ex}
\textbf{Status:} Untested + NoDoc + NoDemo = NOT OK



    \end{boxedminipage}

    \label{xformslib:library:fl_unset_text_clipping}
    \index{xformslib \textit{(package)}!xformslib.library \textit{(module)}!xformslib.library.fl\_unset\_text\_clipping \textit{(function)}}

    \vspace{0.5ex}

\hspace{.8\funcindent}\begin{boxedminipage}{\funcwidth}

    \raggedright \textbf{fl\_unset\_text\_clipping}()

    \vspace{-1.5ex}

    \rule{\textwidth}{0.5\fboxrule}
\setlength{\parskip}{2ex}
\setlength{\parskip}{1ex}
\textbf{Status:} Untested + NoDoc + NoDemo = NOT OK



    \end{boxedminipage}

    \label{xformslib:library:FL_PCCLAMP}
    \index{xformslib \textit{(package)}!xformslib.library \textit{(module)}!xformslib.library.FL\_PCCLAMP \textit{(function)}}

    \vspace{0.5ex}

\hspace{.8\funcindent}\begin{boxedminipage}{\funcwidth}

    \raggedright \textbf{FL\_PCCLAMP}(\textit{a})

\setlength{\parskip}{2ex}
\setlength{\parskip}{1ex}
    \end{boxedminipage}

    \label{xformslib:library:FL_GETR}
    \index{xformslib \textit{(package)}!xformslib.library \textit{(module)}!xformslib.library.FL\_GETR \textit{(function)}}

    \vspace{0.5ex}

\hspace{.8\funcindent}\begin{boxedminipage}{\funcwidth}

    \raggedright \textbf{FL\_GETR}(\textit{packed})

\setlength{\parskip}{2ex}
\setlength{\parskip}{1ex}
    \end{boxedminipage}

    \label{xformslib:library:FL_GETG}
    \index{xformslib \textit{(package)}!xformslib.library \textit{(module)}!xformslib.library.FL\_GETG \textit{(function)}}

    \vspace{0.5ex}

\hspace{.8\funcindent}\begin{boxedminipage}{\funcwidth}

    \raggedright \textbf{FL\_GETG}(\textit{packed})

\setlength{\parskip}{2ex}
\setlength{\parskip}{1ex}
    \end{boxedminipage}

    \label{xformslib:library:FL_GETB}
    \index{xformslib \textit{(package)}!xformslib.library \textit{(module)}!xformslib.library.FL\_GETB \textit{(function)}}

    \vspace{0.5ex}

\hspace{.8\funcindent}\begin{boxedminipage}{\funcwidth}

    \raggedright \textbf{FL\_GETB}(\textit{packed})

\setlength{\parskip}{2ex}
\setlength{\parskip}{1ex}
    \end{boxedminipage}

    \label{xformslib:library:FL_GETA}
    \index{xformslib \textit{(package)}!xformslib.library \textit{(module)}!xformslib.library.FL\_GETA \textit{(function)}}

    \vspace{0.5ex}

\hspace{.8\funcindent}\begin{boxedminipage}{\funcwidth}

    \raggedright \textbf{FL\_GETA}(\textit{packed})

\setlength{\parskip}{2ex}
\setlength{\parskip}{1ex}
    \end{boxedminipage}

    \label{xformslib:library:FL_PACK3}
    \index{xformslib \textit{(package)}!xformslib.library \textit{(module)}!xformslib.library.FL\_PACK3 \textit{(function)}}

    \vspace{0.5ex}

\hspace{.8\funcindent}\begin{boxedminipage}{\funcwidth}

    \raggedright \textbf{FL\_PACK3}(\textit{r}, \textit{g}, \textit{b})

\setlength{\parskip}{2ex}
\setlength{\parskip}{1ex}
    \end{boxedminipage}

    \label{xformslib:library:FL_PACK3}
    \index{xformslib \textit{(package)}!xformslib.library \textit{(module)}!xformslib.library.FL\_PACK3 \textit{(function)}}

    \vspace{0.5ex}

\hspace{.8\funcindent}\begin{boxedminipage}{\funcwidth}

    \raggedright \textbf{FL\_PACK}(\textit{r}, \textit{g}, \textit{b})

\setlength{\parskip}{2ex}
\setlength{\parskip}{1ex}
    \end{boxedminipage}

    \label{xformslib:library:FL_PACK4}
    \index{xformslib \textit{(package)}!xformslib.library \textit{(module)}!xformslib.library.FL\_PACK4 \textit{(function)}}

    \vspace{0.5ex}

\hspace{.8\funcindent}\begin{boxedminipage}{\funcwidth}

    \raggedright \textbf{FL\_PACK4}(\textit{r}, \textit{g}, \textit{b}, \textit{a})

\setlength{\parskip}{2ex}
\setlength{\parskip}{1ex}
    \end{boxedminipage}

    \label{xformslib:library:FL_UNPACK}
    \index{xformslib \textit{(package)}!xformslib.library \textit{(module)}!xformslib.library.FL\_UNPACK \textit{(function)}}

    \vspace{0.5ex}

\hspace{.8\funcindent}\begin{boxedminipage}{\funcwidth}

    \raggedright \textbf{FL\_UNPACK}(\textit{p}, \textit{r}, \textit{g}, \textit{b})

\setlength{\parskip}{2ex}
\setlength{\parskip}{1ex}
    \end{boxedminipage}

    \label{xformslib:library:FL_UNPACK}
    \index{xformslib \textit{(package)}!xformslib.library \textit{(module)}!xformslib.library.FL\_UNPACK \textit{(function)}}

    \vspace{0.5ex}

\hspace{.8\funcindent}\begin{boxedminipage}{\funcwidth}

    \raggedright \textbf{FL\_UNPACK3}(\textit{p}, \textit{r}, \textit{g}, \textit{b})

\setlength{\parskip}{2ex}
\setlength{\parskip}{1ex}
    \end{boxedminipage}

    \label{xformslib:library:FL_UNPACK4}
    \index{xformslib \textit{(package)}!xformslib.library \textit{(module)}!xformslib.library.FL\_UNPACK4 \textit{(function)}}

    \vspace{0.5ex}

\hspace{.8\funcindent}\begin{boxedminipage}{\funcwidth}

    \raggedright \textbf{FL\_UNPACK4}(\textit{p}, \textit{r}, \textit{g}, \textit{b}, \textit{a})

\setlength{\parskip}{2ex}
\setlength{\parskip}{1ex}
    \end{boxedminipage}

    \label{xformslib:library:fl_popup_add}
    \index{xformslib \textit{(package)}!xformslib.library \textit{(module)}!xformslib.library.fl\_popup\_add \textit{(function)}}

    \vspace{0.5ex}

\hspace{.8\funcindent}\begin{boxedminipage}{\funcwidth}

    \raggedright \textbf{fl\_popup\_add}(\textit{win}, \textit{title})

    \vspace{-1.5ex}

    \rule{\textwidth}{0.5\fboxrule}
\setlength{\parskip}{2ex}
\setlength{\parskip}{1ex}
      \textbf{Parameters}
      \vspace{-1ex}

      \begin{quote}
        \begin{Ventry}{xxxxx}

          \item[win]

          window

          \item[title]

          text of title shown at the top of the popup in a framed box

        \end{Ventry}

      \end{quote}

      \textbf{Return Value}
    \vspace{-1ex}

      \begin{quote}
      pPopup

      \end{quote}

\textbf{Status:} Untested + NoDoc + NoDemo = NOT OK



    \end{boxedminipage}

    \label{xformslib:library:fl_popup_add_entries}
    \index{xformslib \textit{(package)}!xformslib.library \textit{(module)}!xformslib.library.fl\_popup\_add\_entries \textit{(function)}}

    \vspace{0.5ex}

\hspace{.8\funcindent}\begin{boxedminipage}{\funcwidth}

    \raggedright \textbf{fl\_popup\_add\_entries}(\textit{pPopup}, \textit{entrytxt})

    \vspace{-1.5ex}

    \rule{\textwidth}{0.5\fboxrule}
\setlength{\parskip}{2ex}
\setlength{\parskip}{1ex}
      \textbf{Return Value}
    \vspace{-1ex}

      \begin{quote}
      pPopupEntry

      \end{quote}

\textbf{Status:} Untested + NoDoc + NoDemo = NOT OK



    \end{boxedminipage}

    \label{xformslib:library:fl_popup_insert_entries}
    \index{xformslib \textit{(package)}!xformslib.library \textit{(module)}!xformslib.library.fl\_popup\_insert\_entries \textit{(function)}}

    \vspace{0.5ex}

\hspace{.8\funcindent}\begin{boxedminipage}{\funcwidth}

    \raggedright \textbf{fl\_popup\_insert\_entries}(\textit{pPopup}, \textit{pPopupEntry}, \textit{entrytxt})

    \vspace{-1.5ex}

    \rule{\textwidth}{0.5\fboxrule}
\setlength{\parskip}{2ex}
\setlength{\parskip}{1ex}
      \textbf{Return Value}
    \vspace{-1ex}

      \begin{quote}
      pPopupEntry

      \end{quote}

\textbf{Status:} Untested + NoDoc + NoDemo = NOT OK



    \end{boxedminipage}

    \label{xformslib:library:fl_popup_create}
    \index{xformslib \textit{(package)}!xformslib.library \textit{(module)}!xformslib.library.fl\_popup\_create \textit{(function)}}

    \vspace{0.5ex}

\hspace{.8\funcindent}\begin{boxedminipage}{\funcwidth}

    \raggedright \textbf{fl\_popup\_create}(\textit{win}, \textit{text}, \textit{pPopupItem})

    \vspace{-1.5ex}

    \rule{\textwidth}{0.5\fboxrule}
\setlength{\parskip}{2ex}
\setlength{\parskip}{1ex}
      \textbf{Return Value}
    \vspace{-1ex}

      \begin{quote}
      pPopup

      \end{quote}

\textbf{Status:} Untested + NoDoc + NoDemo = NOT OK



    \end{boxedminipage}

    \label{xformslib:library:fl_popup_add_items}
    \index{xformslib \textit{(package)}!xformslib.library \textit{(module)}!xformslib.library.fl\_popup\_add\_items \textit{(function)}}

    \vspace{0.5ex}

\hspace{.8\funcindent}\begin{boxedminipage}{\funcwidth}

    \raggedright \textbf{fl\_popup\_add\_items}(\textit{pPopup}, \textit{pPopupItem})

    \vspace{-1.5ex}

    \rule{\textwidth}{0.5\fboxrule}
\setlength{\parskip}{2ex}
\setlength{\parskip}{1ex}
      \textbf{Return Value}
    \vspace{-1ex}

      \begin{quote}
      pPopupEntry

      \end{quote}

\textbf{Status:} Untested + NoDoc + NoDemo = NOT OK



    \end{boxedminipage}

    \label{xformslib:library:fl_popup_insert_items}
    \index{xformslib \textit{(package)}!xformslib.library \textit{(module)}!xformslib.library.fl\_popup\_insert\_items \textit{(function)}}

    \vspace{0.5ex}

\hspace{.8\funcindent}\begin{boxedminipage}{\funcwidth}

    \raggedright \textbf{fl\_popup\_insert\_items}(\textit{pPopup}, \textit{pPopupEntry}, \textit{pPopupItem})

    \vspace{-1.5ex}

    \rule{\textwidth}{0.5\fboxrule}
\setlength{\parskip}{2ex}
\setlength{\parskip}{1ex}
      \textbf{Return Value}
    \vspace{-1ex}

      \begin{quote}
      pPopupEntry

      \end{quote}

\textbf{Status:} Untested + NoDoc + NoDemo = NOT OK



    \end{boxedminipage}

    \label{xformslib:library:fl_popup_delete}
    \index{xformslib \textit{(package)}!xformslib.library \textit{(module)}!xformslib.library.fl\_popup\_delete \textit{(function)}}

    \vspace{0.5ex}

\hspace{.8\funcindent}\begin{boxedminipage}{\funcwidth}

    \raggedright \textbf{fl\_popup\_delete}(\textit{pPopup})

    \vspace{-1.5ex}

    \rule{\textwidth}{0.5\fboxrule}
\setlength{\parskip}{2ex}
\setlength{\parskip}{1ex}
      \textbf{Return Value}
    \vspace{-1ex}

      \begin{quote}
      num

      \end{quote}

\textbf{Status:} Untested + NoDoc + NoDemo = NOT OK



    \end{boxedminipage}

    \label{xformslib:library:fl_popup_entry_delete}
    \index{xformslib \textit{(package)}!xformslib.library \textit{(module)}!xformslib.library.fl\_popup\_entry\_delete \textit{(function)}}

    \vspace{0.5ex}

\hspace{.8\funcindent}\begin{boxedminipage}{\funcwidth}

    \raggedright \textbf{fl\_popup\_entry\_delete}(\textit{pPopupEntry})

    \vspace{-1.5ex}

    \rule{\textwidth}{0.5\fboxrule}
\setlength{\parskip}{2ex}
\setlength{\parskip}{1ex}
      \textbf{Return Value}
    \vspace{-1ex}

      \begin{quote}
      num

      \end{quote}

\textbf{Status:} Untested + NoDoc + NoDemo = NOT OK



    \end{boxedminipage}

    \label{xformslib:library:fl_popup_do}
    \index{xformslib \textit{(package)}!xformslib.library \textit{(module)}!xformslib.library.fl\_popup\_do \textit{(function)}}

    \vspace{0.5ex}

\hspace{.8\funcindent}\begin{boxedminipage}{\funcwidth}

    \raggedright \textbf{fl\_popup\_do}(\textit{pPopup})

    \vspace{-1.5ex}

    \rule{\textwidth}{0.5\fboxrule}
\setlength{\parskip}{2ex}
\setlength{\parskip}{1ex}
      \textbf{Return Value}
    \vspace{-1ex}

      \begin{quote}
      pPopupReturn

      \end{quote}

\textbf{Status:} Untested + NoDoc + NoDemo = NOT OK



    \end{boxedminipage}

    \label{xformslib:library:fl_popup_set_position}
    \index{xformslib \textit{(package)}!xformslib.library \textit{(module)}!xformslib.library.fl\_popup\_set\_position \textit{(function)}}

    \vspace{0.5ex}

\hspace{.8\funcindent}\begin{boxedminipage}{\funcwidth}

    \raggedright \textbf{fl\_popup\_set\_position}(\textit{pPopup}, \textit{x}, \textit{y})

    \vspace{-1.5ex}

    \rule{\textwidth}{0.5\fboxrule}
\setlength{\parskip}{2ex}
    Sets position where the popup is supposed to appear (if never called 
    the popup appears at the mouse position)

\setlength{\parskip}{1ex}
      \textbf{Parameters}
      \vspace{-1ex}

      \begin{quote}
        \begin{Ventry}{xxxxxx}

          \item[pPopup]

          pointer to Popup

          \item[x]

          horizontal position (upper-left corner)

          \item[y]

          vertical position (upper-left corner)

        \end{Ventry}

      \end{quote}

\textbf{Status:} Untested + NoDoc + NoDemo = NOT OK



    \end{boxedminipage}

    \label{xformslib:library:fl_popup_get_policy}
    \index{xformslib \textit{(package)}!xformslib.library \textit{(module)}!xformslib.library.fl\_popup\_get\_policy \textit{(function)}}

    \vspace{0.5ex}

\hspace{.8\funcindent}\begin{boxedminipage}{\funcwidth}

    \raggedright \textbf{fl\_popup\_get\_policy}(\textit{pPopup})

    \vspace{-1.5ex}

    \rule{\textwidth}{0.5\fboxrule}
\setlength{\parskip}{2ex}
\setlength{\parskip}{1ex}
      \textbf{Parameters}
      \vspace{-1ex}

      \begin{quote}
        \begin{Ventry}{xxxxxx}

          \item[pPopup]

          pointer to Popup

        \end{Ventry}

      \end{quote}

      \textbf{Return Value}
    \vspace{-1ex}

      \begin{quote}
      num

      \end{quote}

\textbf{Status:} Untested + NoDoc + NoDemo = NOT OK



    \end{boxedminipage}

    \label{xformslib:library:fl_popup_set_policy}
    \index{xformslib \textit{(package)}!xformslib.library \textit{(module)}!xformslib.library.fl\_popup\_set\_policy \textit{(function)}}

    \vspace{0.5ex}

\hspace{.8\funcindent}\begin{boxedminipage}{\funcwidth}

    \raggedright \textbf{fl\_popup\_set\_policy}(\textit{pPopup}, \textit{policy})

    \vspace{-1.5ex}

    \rule{\textwidth}{0.5\fboxrule}
\setlength{\parskip}{2ex}
    Sets policy of handling the popup (i.e. does it get closed when the 
    user releases the mouse button outside an active entry or not?)

\setlength{\parskip}{1ex}
      \textbf{Parameters}
      \vspace{-1ex}

      \begin{quote}
        \begin{Ventry}{xxxxxx}

          \item[pPopup]

          pointer to Popup

          \item[policy]

          policy to be set

        \end{Ventry}

      \end{quote}

      \textbf{Return Value}
    \vspace{-1ex}

      \begin{quote}
      num

      \end{quote}

\textbf{Status:} Untested + NoDoc + NoDemo = NOT OK



    \end{boxedminipage}

    \label{xformslib:library:fl_popup_set_callback}
    \index{xformslib \textit{(package)}!xformslib.library \textit{(module)}!xformslib.library.fl\_popup\_set\_callback \textit{(function)}}

    \vspace{0.5ex}

\hspace{.8\funcindent}\begin{boxedminipage}{\funcwidth}

    \raggedright \textbf{fl\_popup\_set\_callback}(\textit{pPopup}, \textit{py\_PopupCb})

    \vspace{-1.5ex}

    \rule{\textwidth}{0.5\fboxrule}
\setlength{\parskip}{2ex}
\setlength{\parskip}{1ex}
      \textbf{Return Value}
    \vspace{-1ex}

      \begin{quote}
      popup callback func

      \end{quote}

\textbf{Status:} Untested + NoDoc + NoDemo = NOT OK



    \end{boxedminipage}

    \label{xformslib:library:fl_popup_get_title_font}
    \index{xformslib \textit{(package)}!xformslib.library \textit{(module)}!xformslib.library.fl\_popup\_get\_title\_font \textit{(function)}}

    \vspace{0.5ex}

\hspace{.8\funcindent}\begin{boxedminipage}{\funcwidth}

    \raggedright \textbf{fl\_popup\_get\_title\_font}(\textit{pPopup})

    \vspace{-1.5ex}

    \rule{\textwidth}{0.5\fboxrule}
\setlength{\parskip}{2ex}
\setlength{\parskip}{1ex}
      \textbf{Return Value}
    \vspace{-1ex}

      \begin{quote}
      style, size

      \end{quote}

\textbf{Attention:} API change from XForms - upstream was fl\_popup\_get\_title\_font(pPopup, 
style, size)



\textbf{Status:} Untested + NoDoc + NoDemo = NOT OK



    \end{boxedminipage}

    \label{xformslib:library:fl_popup_set_title_font}
    \index{xformslib \textit{(package)}!xformslib.library \textit{(module)}!xformslib.library.fl\_popup\_set\_title\_font \textit{(function)}}

    \vspace{0.5ex}

\hspace{.8\funcindent}\begin{boxedminipage}{\funcwidth}

    \raggedright \textbf{fl\_popup\_set\_title\_font}(\textit{pPopup}, \textit{style}, \textit{size})

    \vspace{-1.5ex}

    \rule{\textwidth}{0.5\fboxrule}
\setlength{\parskip}{2ex}
\setlength{\parskip}{1ex}
\textbf{Status:} Tested + NoDoc + Demo = OK



    \end{boxedminipage}

    \label{xformslib:library:fl_popup_entry_get_font}
    \index{xformslib \textit{(package)}!xformslib.library \textit{(module)}!xformslib.library.fl\_popup\_entry\_get\_font \textit{(function)}}

    \vspace{0.5ex}

\hspace{.8\funcindent}\begin{boxedminipage}{\funcwidth}

    \raggedright \textbf{fl\_popup\_entry\_get\_font}(\textit{pPopup})

    \vspace{-1.5ex}

    \rule{\textwidth}{0.5\fboxrule}
\setlength{\parskip}{2ex}
\setlength{\parskip}{1ex}
      \textbf{Return Value}
    \vspace{-1ex}

      \begin{quote}
      style, size

      \end{quote}

\textbf{Attention:} API change from XForms - upstream was fl\_popup\_entry\_get\_font(pPopup, 
style, size)



\textbf{Status:} Untested + NoDoc + NoDemo = NOT OK



    \end{boxedminipage}

    \label{xformslib:library:fl_popup_entry_set_font}
    \index{xformslib \textit{(package)}!xformslib.library \textit{(module)}!xformslib.library.fl\_popup\_entry\_set\_font \textit{(function)}}

    \vspace{0.5ex}

\hspace{.8\funcindent}\begin{boxedminipage}{\funcwidth}

    \raggedright \textbf{fl\_popup\_entry\_set\_font}(\textit{pPopup}, \textit{style}, \textit{size})

    \vspace{-1.5ex}

    \rule{\textwidth}{0.5\fboxrule}
\setlength{\parskip}{2ex}
    Sets the font of a popup entry.

\setlength{\parskip}{1ex}
      \textbf{Parameters}
      \vspace{-1ex}

      \begin{quote}
        \begin{Ventry}{xxxxxx}

          \item[pPopup]

          pointer to Popup

          \item[style]

          style of the popup entry

          \item[size]

          size of the popup entry

        \end{Ventry}

      \end{quote}

\textbf{Status:} Untested + NoDoc + NoDemo = NOT OK



    \end{boxedminipage}

    \label{xformslib:library:fl_popup_get_bw}
    \index{xformslib \textit{(package)}!xformslib.library \textit{(module)}!xformslib.library.fl\_popup\_get\_bw \textit{(function)}}

    \vspace{0.5ex}

\hspace{.8\funcindent}\begin{boxedminipage}{\funcwidth}

    \raggedright \textbf{fl\_popup\_get\_bw}(\textit{pPopup})

    \vspace{-1.5ex}

    \rule{\textwidth}{0.5\fboxrule}
\setlength{\parskip}{2ex}
    Returns the border width of a popup.

\setlength{\parskip}{1ex}
      \textbf{Parameters}
      \vspace{-1ex}

      \begin{quote}
        \begin{Ventry}{xxxxxx}

          \item[pPopup]

          pointer to popup

        \end{Ventry}

      \end{quote}

      \textbf{Return Value}
    \vspace{-1ex}

      \begin{quote}
      borderwidth

      \end{quote}

\textbf{Status:} Untested + NoDoc + NoDemo = NOT OK



    \end{boxedminipage}

    \label{xformslib:library:fl_popup_set_bw}
    \index{xformslib \textit{(package)}!xformslib.library \textit{(module)}!xformslib.library.fl\_popup\_set\_bw \textit{(function)}}

    \vspace{0.5ex}

\hspace{.8\funcindent}\begin{boxedminipage}{\funcwidth}

    \raggedright \textbf{fl\_popup\_set\_bw}(\textit{pPopup}, \textit{bw})

    \vspace{-1.5ex}

    \rule{\textwidth}{0.5\fboxrule}
\setlength{\parskip}{2ex}
    Sets the border width of a popup.

\setlength{\parskip}{1ex}
      \textbf{Parameters}
      \vspace{-1ex}

      \begin{quote}
        \begin{Ventry}{xxxxxx}

          \item[pPopup]

          pointer to popup

          \item[bw]

          border width value to be set

        \end{Ventry}

      \end{quote}

      \textbf{Return Value}
    \vspace{-1ex}

      \begin{quote}
      num

      \end{quote}

\textbf{Status:} Tested + NoDoc + Demo = OK



    \end{boxedminipage}

    \label{xformslib:library:fl_popup_get_color}
    \index{xformslib \textit{(package)}!xformslib.library \textit{(module)}!xformslib.library.fl\_popup\_get\_color \textit{(function)}}

    \vspace{0.5ex}

\hspace{.8\funcindent}\begin{boxedminipage}{\funcwidth}

    \raggedright \textbf{fl\_popup\_get\_color}(\textit{pPopup}, \textit{colrpos})

    \vspace{-1.5ex}

    \rule{\textwidth}{0.5\fboxrule}
\setlength{\parskip}{2ex}
\setlength{\parskip}{1ex}
      \textbf{Parameters}
      \vspace{-1ex}

      \begin{quote}
        \begin{Ventry}{xxxxxxx}

          \item[colrpos]

            {\it (type=[num./int] from xfdata module FL\_POPUP\_BACKGROUND\_COLOR, 
FL\_POPUP\_HIGHLIGHT\_COLOR, FL\_POPUP\_TITLE\_COLOR, 
FL\_POPUP\_TEXT\_COLOR, FL\_POPUP\_HIGHLIGHT\_TEXT\_COLOR, 
FL\_POPUP\_DISABLED\_TEXT\_COLOR, FL\_POPUP\_RADIO\_COLOR)}

        \end{Ventry}

      \end{quote}

      \textbf{Return Value}
    \vspace{-1ex}

      \begin{quote}
      color

      \end{quote}

\textbf{Status:} Untested + NoDoc + NoDemo = NOT OK



    \end{boxedminipage}

    \label{xformslib:library:fl_popup_set_color}
    \index{xformslib \textit{(package)}!xformslib.library \textit{(module)}!xformslib.library.fl\_popup\_set\_color \textit{(function)}}

    \vspace{0.5ex}

\hspace{.8\funcindent}\begin{boxedminipage}{\funcwidth}

    \raggedright \textbf{fl\_popup\_set\_color}(\textit{pPopup}, \textit{colrpos}, \textit{colr})

    \vspace{-1.5ex}

    \rule{\textwidth}{0.5\fboxrule}
\setlength{\parskip}{2ex}
\setlength{\parskip}{1ex}
      \textbf{Parameters}
      \vspace{-1ex}

      \begin{quote}
        \begin{Ventry}{xxxxxxx}

          \item[colrpos]

          popup color type

            {\it (type=[num./int] from xfdata module FL\_POPUP\_BACKGROUND\_COLOR, 
FL\_POPUP\_HIGHLIGHT\_COLOR, FL\_POPUP\_TITLE\_COLOR, 
FL\_POPUP\_TEXT\_COLOR, FL\_POPUP\_HIGHLIGHT\_TEXT\_COLOR, 
FL\_POPUP\_DISABLED\_TEXT\_COLOR, FL\_POPUP\_RADIO\_COLOR)}

          \item[colr]

          color value to be set

        \end{Ventry}

      \end{quote}

      \textbf{Return Value}
    \vspace{-1ex}

      \begin{quote}
      color

      \end{quote}

\textbf{Status:} Untested + NoDoc + NoDemo = NOT OK



    \end{boxedminipage}

    \label{xformslib:library:fl_popup_set_cursor}
    \index{xformslib \textit{(package)}!xformslib.library \textit{(module)}!xformslib.library.fl\_popup\_set\_cursor \textit{(function)}}

    \vspace{0.5ex}

\hspace{.8\funcindent}\begin{boxedminipage}{\funcwidth}

    \raggedright \textbf{fl\_popup\_set\_cursor}(\textit{pPopup}, \textit{cursnum})

    \vspace{-1.5ex}

    \rule{\textwidth}{0.5\fboxrule}
\setlength{\parskip}{2ex}
    Changes the cursor displayed when a popup is shown.

\setlength{\parskip}{1ex}
      \textbf{Parameters}
      \vspace{-1ex}

      \begin{quote}
        \begin{Ventry}{xxxxxxx}

          \item[pPopup]

          pointer to FL\_POPUP

          \item[cursnum]

          id of a symbolic cursor shapes' names

        \end{Ventry}

      \end{quote}

\textbf{Status:} Untested + NoDoc + NoDemo = NOT OK



    \end{boxedminipage}

    \label{xformslib:library:fl_popup_get_title}
    \index{xformslib \textit{(package)}!xformslib.library \textit{(module)}!xformslib.library.fl\_popup\_get\_title \textit{(function)}}

    \vspace{0.5ex}

\hspace{.8\funcindent}\begin{boxedminipage}{\funcwidth}

    \raggedright \textbf{fl\_popup\_get\_title}(\textit{pPopup})

    \vspace{-1.5ex}

    \rule{\textwidth}{0.5\fboxrule}
\setlength{\parskip}{2ex}
    Returns the title of a popup.

\setlength{\parskip}{1ex}
      \textbf{Parameters}
      \vspace{-1ex}

      \begin{quote}
        \begin{Ventry}{xxxxxx}

          \item[pPopup]

          pointer to popup

        \end{Ventry}

      \end{quote}

      \textbf{Return Value}
    \vspace{-1ex}

      \begin{quote}
      title string

      \end{quote}

\textbf{Status:} Untested + NoDoc + NoDemo = NOT OK



    \end{boxedminipage}

    \label{xformslib:library:fl_popup_set_title}
    \index{xformslib \textit{(package)}!xformslib.library \textit{(module)}!xformslib.library.fl\_popup\_set\_title \textit{(function)}}

    \vspace{0.5ex}

\hspace{.8\funcindent}\begin{boxedminipage}{\funcwidth}

    \raggedright \textbf{fl\_popup\_set\_title}(\textit{pPopup}, \textit{title})

    \vspace{-1.5ex}

    \rule{\textwidth}{0.5\fboxrule}
\setlength{\parskip}{2ex}
    Sets the title of a popup.

\setlength{\parskip}{1ex}
      \textbf{Parameters}
      \vspace{-1ex}

      \begin{quote}
        \begin{Ventry}{xxxxxx}

          \item[pPopup]

          pointer to popup

          \item[title]

          title of the popup

        \end{Ventry}

      \end{quote}

      \textbf{Return Value}
    \vspace{-1ex}

      \begin{quote}
      popup

      \end{quote}

\textbf{Status:} Untested + NoDoc + NoDemo = NOT OK



    \end{boxedminipage}

    \label{xformslib:library:fl_popup_entry_set_callback}
    \index{xformslib \textit{(package)}!xformslib.library \textit{(module)}!xformslib.library.fl\_popup\_entry\_set\_callback \textit{(function)}}

    \vspace{0.5ex}

\hspace{.8\funcindent}\begin{boxedminipage}{\funcwidth}

    \raggedright \textbf{fl\_popup\_entry\_set\_callback}(\textit{pPopupEntry}, \textit{py\_PopupCb})

    \vspace{-1.5ex}

    \rule{\textwidth}{0.5\fboxrule}
\setlength{\parskip}{2ex}
\setlength{\parskip}{1ex}
      \textbf{Return Value}
    \vspace{-1ex}

      \begin{quote}
      popup\_callback

      \end{quote}

\textbf{Status:} Tested + NoDoc + Demo = OK



    \end{boxedminipage}

    \label{xformslib:library:fl_popup_entry_set_enter_callback}
    \index{xformslib \textit{(package)}!xformslib.library \textit{(module)}!xformslib.library.fl\_popup\_entry\_set\_enter\_callback \textit{(function)}}

    \vspace{0.5ex}

\hspace{.8\funcindent}\begin{boxedminipage}{\funcwidth}

    \raggedright \textbf{fl\_popup\_entry\_set\_enter\_callback}(\textit{pPopupEntry}, \textit{py\_PopupCb})

    \vspace{-1.5ex}

    \rule{\textwidth}{0.5\fboxrule}
\setlength{\parskip}{2ex}
\setlength{\parskip}{1ex}
      \textbf{Return Value}
    \vspace{-1ex}

      \begin{quote}
      popup\_callback

      \end{quote}

\textbf{Status:} Untested + NoDoc + NoDemo = NOT OK



    \end{boxedminipage}

    \label{xformslib:library:fl_popup_entry_set_leave_callback}
    \index{xformslib \textit{(package)}!xformslib.library \textit{(module)}!xformslib.library.fl\_popup\_entry\_set\_leave\_callback \textit{(function)}}

    \vspace{0.5ex}

\hspace{.8\funcindent}\begin{boxedminipage}{\funcwidth}

    \raggedright \textbf{fl\_popup\_entry\_set\_leave\_callback}(\textit{pPopupEntry}, \textit{py\_PopupCb})

    \vspace{-1.5ex}

    \rule{\textwidth}{0.5\fboxrule}
\setlength{\parskip}{2ex}
\setlength{\parskip}{1ex}
      \textbf{Return Value}
    \vspace{-1ex}

      \begin{quote}
      popup\_callback

      \end{quote}

\textbf{Status:} Untested + NoDoc + NoDemo = NOT OK



    \end{boxedminipage}

    \label{xformslib:library:fl_popup_entry_get_state}
    \index{xformslib \textit{(package)}!xformslib.library \textit{(module)}!xformslib.library.fl\_popup\_entry\_get\_state \textit{(function)}}

    \vspace{0.5ex}

\hspace{.8\funcindent}\begin{boxedminipage}{\funcwidth}

    \raggedright \textbf{fl\_popup\_entry\_get\_state}(\textit{pPopupEntry})

    \vspace{-1.5ex}

    \rule{\textwidth}{0.5\fboxrule}
\setlength{\parskip}{2ex}
\setlength{\parskip}{1ex}
      \textbf{Return Value}
    \vspace{-1ex}

      \begin{quote}
      state num

      \end{quote}

\textbf{Status:} Untested + NoDoc + NoDemo = NOT OK



    \end{boxedminipage}

    \label{xformslib:library:fl_popup_entry_set_state}
    \index{xformslib \textit{(package)}!xformslib.library \textit{(module)}!xformslib.library.fl\_popup\_entry\_set\_state \textit{(function)}}

    \vspace{0.5ex}

\hspace{.8\funcindent}\begin{boxedminipage}{\funcwidth}

    \raggedright \textbf{fl\_popup\_entry\_set\_state}(\textit{pPopupEntry}, \textit{state})

    \vspace{-1.5ex}

    \rule{\textwidth}{0.5\fboxrule}
\setlength{\parskip}{2ex}
\setlength{\parskip}{1ex}
      \textbf{Return Value}
    \vspace{-1ex}

      \begin{quote}
      state num

      \end{quote}

\textbf{Status:} Tested + NoDoc + Demo = OK



    \end{boxedminipage}

    \label{xformslib:library:fl_popup_entry_clear_state}
    \index{xformslib \textit{(package)}!xformslib.library \textit{(module)}!xformslib.library.fl\_popup\_entry\_clear\_state \textit{(function)}}

    \vspace{0.5ex}

\hspace{.8\funcindent}\begin{boxedminipage}{\funcwidth}

    \raggedright \textbf{fl\_popup\_entry\_clear\_state}(\textit{pPopupEntry}, \textit{state})

    \vspace{-1.5ex}

    \rule{\textwidth}{0.5\fboxrule}
\setlength{\parskip}{2ex}
\setlength{\parskip}{1ex}
      \textbf{Return Value}
    \vspace{-1ex}

      \begin{quote}
      state num

      \end{quote}

\textbf{Status:} Untested + NoDoc + NoDemo = NOT OK



    \end{boxedminipage}

    \label{xformslib:library:fl_popup_entry_raise_state}
    \index{xformslib \textit{(package)}!xformslib.library \textit{(module)}!xformslib.library.fl\_popup\_entry\_raise\_state \textit{(function)}}

    \vspace{0.5ex}

\hspace{.8\funcindent}\begin{boxedminipage}{\funcwidth}

    \raggedright \textbf{fl\_popup\_entry\_raise\_state}(\textit{pPopupEntry}, \textit{state})

    \vspace{-1.5ex}

    \rule{\textwidth}{0.5\fboxrule}
\setlength{\parskip}{2ex}
\setlength{\parskip}{1ex}
      \textbf{Return Value}
    \vspace{-1ex}

      \begin{quote}
      state num

      \end{quote}

\textbf{Status:} Untested + NoDoc + NoDemo = NOT OK



    \end{boxedminipage}

    \label{xformslib:library:fl_popup_entry_toggle_state}
    \index{xformslib \textit{(package)}!xformslib.library \textit{(module)}!xformslib.library.fl\_popup\_entry\_toggle\_state \textit{(function)}}

    \vspace{0.5ex}

\hspace{.8\funcindent}\begin{boxedminipage}{\funcwidth}

    \raggedright \textbf{fl\_popup\_entry\_toggle\_state}(\textit{pPopupEntry}, \textit{state})

    \vspace{-1.5ex}

    \rule{\textwidth}{0.5\fboxrule}
\setlength{\parskip}{2ex}
\setlength{\parskip}{1ex}
      \textbf{Return Value}
    \vspace{-1ex}

      \begin{quote}
      num

      \end{quote}

\textbf{Status:} Untested + NoDoc + NoDemo = NOT OK



    \end{boxedminipage}

    \label{xformslib:library:fl_popup_entry_set_text}
    \index{xformslib \textit{(package)}!xformslib.library \textit{(module)}!xformslib.library.fl\_popup\_entry\_set\_text \textit{(function)}}

    \vspace{0.5ex}

\hspace{.8\funcindent}\begin{boxedminipage}{\funcwidth}

    \raggedright \textbf{fl\_popup\_entry\_set\_text}(\textit{p1}, \textit{txtstr})

    \vspace{-1.5ex}

    \rule{\textwidth}{0.5\fboxrule}
\setlength{\parskip}{2ex}
\setlength{\parskip}{1ex}
      \textbf{Return Value}
    \vspace{-1ex}

      \begin{quote}
      num

      \end{quote}

\textbf{Status:} Untested + NoDoc + NoDemo = NOT OK



    \end{boxedminipage}

    \label{xformslib:library:fl_popup_entry_set_shortcut}
    \index{xformslib \textit{(package)}!xformslib.library \textit{(module)}!xformslib.library.fl\_popup\_entry\_set\_shortcut \textit{(function)}}

    \vspace{0.5ex}

\hspace{.8\funcindent}\begin{boxedminipage}{\funcwidth}

    \raggedright \textbf{fl\_popup\_entry\_set\_shortcut}(\textit{pPopupEntry}, \textit{textsc})

    \vspace{-1.5ex}

    \rule{\textwidth}{0.5\fboxrule}
\setlength{\parskip}{2ex}
\setlength{\parskip}{1ex}
\textbf{Status:} Tested + NoDoc + Demo = OK



    \end{boxedminipage}

    \label{xformslib:library:fl_popup_entry_set_value}
    \index{xformslib \textit{(package)}!xformslib.library \textit{(module)}!xformslib.library.fl\_popup\_entry\_set\_value \textit{(function)}}

    \vspace{0.5ex}

\hspace{.8\funcindent}\begin{boxedminipage}{\funcwidth}

    \raggedright \textbf{fl\_popup\_entry\_set\_value}(\textit{pPopupEntry}, \textit{p2})

    \vspace{-1.5ex}

    \rule{\textwidth}{0.5\fboxrule}
\setlength{\parskip}{2ex}
\setlength{\parskip}{1ex}
      \textbf{Return Value}
    \vspace{-1ex}

      \begin{quote}
      num

      \end{quote}

\textbf{Status:} Untested + NoDoc + NoDemo = NOT OK



    \end{boxedminipage}

    \label{xformslib:library:fl_popup_entry_set_user_data}
    \index{xformslib \textit{(package)}!xformslib.library \textit{(module)}!xformslib.library.fl\_popup\_entry\_set\_user\_data \textit{(function)}}

    \vspace{0.5ex}

\hspace{.8\funcindent}\begin{boxedminipage}{\funcwidth}

    \raggedright \textbf{fl\_popup\_entry\_set\_user\_data}(\textit{pPopupEntry}, \textit{vdata})

    \vspace{-1.5ex}

    \rule{\textwidth}{0.5\fboxrule}
\setlength{\parskip}{2ex}
\setlength{\parskip}{1ex}
      \textbf{Return Value}
    \vspace{-1ex}

      \begin{quote}
      ??

      \end{quote}

\textbf{Status:} Untested + NoDoc + NoDemo = NOT OK



    \end{boxedminipage}

    \label{xformslib:library:fl_popup_entry_get_by_position}
    \index{xformslib \textit{(package)}!xformslib.library \textit{(module)}!xformslib.library.fl\_popup\_entry\_get\_by\_position \textit{(function)}}

    \vspace{0.5ex}

\hspace{.8\funcindent}\begin{boxedminipage}{\funcwidth}

    \raggedright \textbf{fl\_popup\_entry\_get\_by\_position}(\textit{pPopup}, \textit{numpos})

    \vspace{-1.5ex}

    \rule{\textwidth}{0.5\fboxrule}
\setlength{\parskip}{2ex}
\setlength{\parskip}{1ex}
      \textbf{Return Value}
    \vspace{-1ex}

      \begin{quote}
      pPopupEntry

      \end{quote}

\textbf{Status:} Untested + NoDoc + NoDemo = NOT OK



    \end{boxedminipage}

    \label{xformslib:library:fl_popup_entry_get_by_value}
    \index{xformslib \textit{(package)}!xformslib.library \textit{(module)}!xformslib.library.fl\_popup\_entry\_get\_by\_value \textit{(function)}}

    \vspace{0.5ex}

\hspace{.8\funcindent}\begin{boxedminipage}{\funcwidth}

    \raggedright \textbf{fl\_popup\_entry\_get\_by\_value}(\textit{pPopup}, \textit{val})

    \vspace{-1.5ex}

    \rule{\textwidth}{0.5\fboxrule}
\setlength{\parskip}{2ex}
\setlength{\parskip}{1ex}
      \textbf{Return Value}
    \vspace{-1ex}

      \begin{quote}
      pPopupEntry

      \end{quote}

\textbf{Status:} Untested + NoDoc + NoDemo = NOT OK



    \end{boxedminipage}

    \label{xformslib:library:fl_popup_entry_get_by_user_data}
    \index{xformslib \textit{(package)}!xformslib.library \textit{(module)}!xformslib.library.fl\_popup\_entry\_get\_by\_user\_data \textit{(function)}}

    \vspace{0.5ex}

\hspace{.8\funcindent}\begin{boxedminipage}{\funcwidth}

    \raggedright \textbf{fl\_popup\_entry\_get\_by\_user\_data}(\textit{pPopup}, \textit{vdata})

    \vspace{-1.5ex}

    \rule{\textwidth}{0.5\fboxrule}
\setlength{\parskip}{2ex}
\setlength{\parskip}{1ex}
      \textbf{Return Value}
    \vspace{-1ex}

      \begin{quote}
      pPopupEntry

      \end{quote}

\textbf{Status:} Untested + NoDoc + NoDemo = NOT OK



    \end{boxedminipage}

    \label{xformslib:library:fl_popup_entry_get_by_text}
    \index{xformslib \textit{(package)}!xformslib.library \textit{(module)}!xformslib.library.fl\_popup\_entry\_get\_by\_text \textit{(function)}}

    \vspace{0.5ex}

\hspace{.8\funcindent}\begin{boxedminipage}{\funcwidth}

    \raggedright \textbf{fl\_popup\_entry\_get\_by\_text}(\textit{pPopup}, \textit{text})

    \vspace{-1.5ex}

    \rule{\textwidth}{0.5\fboxrule}
\setlength{\parskip}{2ex}
\setlength{\parskip}{1ex}
      \textbf{Return Value}
    \vspace{-1ex}

      \begin{quote}
      pPopupEntry

      \end{quote}

\textbf{Status:} Untested + NoDoc + NoDemo = NOT OK



    \end{boxedminipage}

    \label{xformslib:library:fl_popup_entry_get_by_label}
    \index{xformslib \textit{(package)}!xformslib.library \textit{(module)}!xformslib.library.fl\_popup\_entry\_get\_by\_label \textit{(function)}}

    \vspace{0.5ex}

\hspace{.8\funcindent}\begin{boxedminipage}{\funcwidth}

    \raggedright \textbf{fl\_popup\_entry\_get\_by\_label}(\textit{pPopup}, \textit{label})

    \vspace{-1.5ex}

    \rule{\textwidth}{0.5\fboxrule}
\setlength{\parskip}{2ex}
\setlength{\parskip}{1ex}
      \textbf{Return Value}
    \vspace{-1ex}

      \begin{quote}
      pPopupEntry

      \end{quote}

\textbf{Status:} Untested + NoDoc + NoDemo = NOT OK



    \end{boxedminipage}

    \label{xformslib:library:fl_popup_entry_get_group}
    \index{xformslib \textit{(package)}!xformslib.library \textit{(module)}!xformslib.library.fl\_popup\_entry\_get\_group \textit{(function)}}

    \vspace{0.5ex}

\hspace{.8\funcindent}\begin{boxedminipage}{\funcwidth}

    \raggedright \textbf{fl\_popup\_entry\_get\_group}(\textit{pPopupEntry})

    \vspace{-1.5ex}

    \rule{\textwidth}{0.5\fboxrule}
\setlength{\parskip}{2ex}
\setlength{\parskip}{1ex}
      \textbf{Return Value}
    \vspace{-1ex}

      \begin{quote}
      num

      \end{quote}

\textbf{Status:} Untested + NoDoc + NoDemo = NOT OK



    \end{boxedminipage}

    \label{xformslib:library:fl_popup_entry_set_group}
    \index{xformslib \textit{(package)}!xformslib.library \textit{(module)}!xformslib.library.fl\_popup\_entry\_set\_group \textit{(function)}}

    \vspace{0.5ex}

\hspace{.8\funcindent}\begin{boxedminipage}{\funcwidth}

    \raggedright \textbf{fl\_popup\_entry\_set\_group}(\textit{pPopupEntry}, \textit{num})

    \vspace{-1.5ex}

    \rule{\textwidth}{0.5\fboxrule}
\setlength{\parskip}{2ex}
\setlength{\parskip}{1ex}
      \textbf{Return Value}
    \vspace{-1ex}

      \begin{quote}
      num

      \end{quote}

\textbf{Status:} Untested + NoDoc + NoDemo = NOT OK



    \end{boxedminipage}

    \label{xformslib:library:fl_popup_entry_get_subpopup}
    \index{xformslib \textit{(package)}!xformslib.library \textit{(module)}!xformslib.library.fl\_popup\_entry\_get\_subpopup \textit{(function)}}

    \vspace{0.5ex}

\hspace{.8\funcindent}\begin{boxedminipage}{\funcwidth}

    \raggedright \textbf{fl\_popup\_entry\_get\_subpopup}(\textit{pPopupEntry})

    \vspace{-1.5ex}

    \rule{\textwidth}{0.5\fboxrule}
\setlength{\parskip}{2ex}
\setlength{\parskip}{1ex}
      \textbf{Return Value}
    \vspace{-1ex}

      \begin{quote}
      pPopup

      \end{quote}

\textbf{Status:} Untested + NoDoc + NoDemo = NOT OK



    \end{boxedminipage}

    \label{xformslib:library:fl_popup_entry_set_subpopup}
    \index{xformslib \textit{(package)}!xformslib.library \textit{(module)}!xformslib.library.fl\_popup\_entry\_set\_subpopup \textit{(function)}}

    \vspace{0.5ex}

\hspace{.8\funcindent}\begin{boxedminipage}{\funcwidth}

    \raggedright \textbf{fl\_popup\_entry\_set\_subpopup}(\textit{pPopupEntry}, \textit{pPopup})

    \vspace{-1.5ex}

    \rule{\textwidth}{0.5\fboxrule}
\setlength{\parskip}{2ex}
\setlength{\parskip}{1ex}
      \textbf{Return Value}
    \vspace{-1ex}

      \begin{quote}
      pPopup

      \end{quote}

\textbf{Status:} Untested + NoDoc + NoDemo = NOT OK



    \end{boxedminipage}

    \label{xformslib:library:fl_popup_get_size}
    \index{xformslib \textit{(package)}!xformslib.library \textit{(module)}!xformslib.library.fl\_popup\_get\_size \textit{(function)}}

    \vspace{0.5ex}

\hspace{.8\funcindent}\begin{boxedminipage}{\funcwidth}

    \raggedright \textbf{fl\_popup\_get\_size}(\textit{pPopup})

    \vspace{-1.5ex}

    \rule{\textwidth}{0.5\fboxrule}
\setlength{\parskip}{2ex}
\setlength{\parskip}{1ex}
      \textbf{Return Value}
    \vspace{-1ex}

      \begin{quote}
      size num., width, height

      \end{quote}

\textbf{Attention:} API change from XForms - upstream was fl\_popup\_get\_size(pPopup, w, h)



\textbf{Status:} Untested + NoDoc + NoDemo = NOT OK



    \end{boxedminipage}

    \label{xformslib:library:fl_popup_get_min_width}
    \index{xformslib \textit{(package)}!xformslib.library \textit{(module)}!xformslib.library.fl\_popup\_get\_min\_width \textit{(function)}}

    \vspace{0.5ex}

\hspace{.8\funcindent}\begin{boxedminipage}{\funcwidth}

    \raggedright \textbf{fl\_popup\_get\_min\_width}(\textit{pPopup})

    \vspace{-1.5ex}

    \rule{\textwidth}{0.5\fboxrule}
\setlength{\parskip}{2ex}
\setlength{\parskip}{1ex}
      \textbf{Return Value}
    \vspace{-1ex}

      \begin{quote}
      width num

      \end{quote}

\textbf{Status:} Untested + NoDoc + NoDemo = NOT OK



    \end{boxedminipage}

    \label{xformslib:library:fl_popup_set_min_width}
    \index{xformslib \textit{(package)}!xformslib.library \textit{(module)}!xformslib.library.fl\_popup\_set\_min\_width \textit{(function)}}

    \vspace{0.5ex}

\hspace{.8\funcindent}\begin{boxedminipage}{\funcwidth}

    \raggedright \textbf{fl\_popup\_set\_min\_width}(\textit{pPopup}, \textit{minwidth})

    \vspace{-1.5ex}

    \rule{\textwidth}{0.5\fboxrule}
\setlength{\parskip}{2ex}
\setlength{\parskip}{1ex}
      \textbf{Return Value}
    \vspace{-1ex}

      \begin{quote}
      width num

      \end{quote}

\textbf{Status:} Untested + NoDoc + NoDemo = NOT OK



    \end{boxedminipage}

    \label{xformslib:library:fl_add_bitmap}
    \index{xformslib \textit{(package)}!xformslib.library \textit{(module)}!xformslib.library.fl\_add\_bitmap \textit{(function)}}

    \vspace{0.5ex}

\hspace{.8\funcindent}\begin{boxedminipage}{\funcwidth}

    \raggedright \textbf{fl\_add\_bitmap}(\textit{bitmaptype}, \textit{x}, \textit{y}, \textit{w}, \textit{h}, \textit{label})

    \vspace{-1.5ex}

    \rule{\textwidth}{0.5\fboxrule}
\setlength{\parskip}{2ex}
    Adds a bitmap object.

\setlength{\parskip}{1ex}
      \textbf{Parameters}
      \vspace{-1ex}

      \begin{quote}
        \begin{Ventry}{xxxxxxxxxx}

          \item[bitmaptype]

          type of bitmap to be added

            {\it (type=[num./int] xfc.FL\_NORMAL\_BITMAP)}

          \item[x]

          horizontal position (upper-left corner)

          \item[y]

          vertical position of bitmap (upper-left corner)

          \item[w]

          width in coord units

          \item[h]

          height in coord units

          \item[label]

          text label of bitmap

        \end{Ventry}

      \end{quote}

      \textbf{Return Value}
    \vspace{-1ex}

      \begin{quote}
      pObject

      \end{quote}

\textbf{Status:} Tested + NoDoc + Demo = OK



    \end{boxedminipage}

    \label{xformslib:library:fl_set_bitmap_data}
    \index{xformslib \textit{(package)}!xformslib.library \textit{(module)}!xformslib.library.fl\_set\_bitmap\_data \textit{(function)}}

    \vspace{0.5ex}

\hspace{.8\funcindent}\begin{boxedminipage}{\funcwidth}

    \raggedright \textbf{fl\_set\_bitmap\_data}(\textit{pObject}, \textit{w}, \textit{h}, \textit{xbmcontents})

    \vspace{-1.5ex}

    \rule{\textwidth}{0.5\fboxrule}
\setlength{\parskip}{2ex}
    Fills the bitmap object with a bitmap.

\setlength{\parskip}{1ex}
      \textbf{Parameters}
      \vspace{-1ex}

      \begin{quote}
        \begin{Ventry}{xxxxxxxxxxx}

          \item[pObject]

          pointer to object ({\textless}pointer to 
          xfdata.FL\_OBJECT{\textgreater})

          \item[w]

          width of bitmap in cood units

          \item[h]

          height of bitmap in coord units

          \item[xbmcontents]

          bitmap data used for contents in ubytes

        \end{Ventry}

      \end{quote}

\textbf{Status:} Untested + NoDoc + NoDemo = NOT OK



    \end{boxedminipage}

    \label{xformslib:library:fl_set_bitmap_file}
    \index{xformslib \textit{(package)}!xformslib.library \textit{(module)}!xformslib.library.fl\_set\_bitmap\_file \textit{(function)}}

    \vspace{0.5ex}

\hspace{.8\funcindent}\begin{boxedminipage}{\funcwidth}

    \raggedright \textbf{fl\_set\_bitmap\_file}(\textit{pObject}, \textit{fname})

    \vspace{-1.5ex}

    \rule{\textwidth}{0.5\fboxrule}
\setlength{\parskip}{2ex}
\setlength{\parskip}{1ex}
      \textbf{Parameters}
      \vspace{-1ex}

      \begin{quote}
        \begin{Ventry}{xxxxxxx}

          \item[pObject]

          pointer to object ({\textless}pointer to 
          xfdata.FL\_OBJECT{\textgreater})

          \item[fname]

          name of bitmap file

        \end{Ventry}

      \end{quote}

\textbf{Status:} Tested + NoDoc + Demo = OK



    \end{boxedminipage}

    \label{xformslib:library:fl_set_bitmap_file}
    \index{xformslib \textit{(package)}!xformslib.library \textit{(module)}!xformslib.library.fl\_set\_bitmap\_file \textit{(function)}}

    \vspace{0.5ex}

\hspace{.8\funcindent}\begin{boxedminipage}{\funcwidth}

    \raggedright \textbf{fl\_set\_bitmapbutton\_file}(\textit{pObject}, \textit{fname})

    \vspace{-1.5ex}

    \rule{\textwidth}{0.5\fboxrule}
\setlength{\parskip}{2ex}
\setlength{\parskip}{1ex}
      \textbf{Parameters}
      \vspace{-1ex}

      \begin{quote}
        \begin{Ventry}{xxxxxxx}

          \item[pObject]

          pointer to object ({\textless}pointer to 
          xfdata.FL\_OBJECT{\textgreater})

          \item[fname]

          name of bitmap file

        \end{Ventry}

      \end{quote}

\textbf{Status:} Tested + NoDoc + Demo = OK



    \end{boxedminipage}

    \label{xformslib:library:fl_set_bitmap_file}
    \index{xformslib \textit{(package)}!xformslib.library \textit{(module)}!xformslib.library.fl\_set\_bitmap\_file \textit{(function)}}

    \vspace{0.5ex}

\hspace{.8\funcindent}\begin{boxedminipage}{\funcwidth}

    \raggedright \textbf{fl\_set\_bitmapbutton\_datafile}(\textit{pObject}, \textit{fname})

    \vspace{-1.5ex}

    \rule{\textwidth}{0.5\fboxrule}
\setlength{\parskip}{2ex}
\setlength{\parskip}{1ex}
      \textbf{Parameters}
      \vspace{-1ex}

      \begin{quote}
        \begin{Ventry}{xxxxxxx}

          \item[pObject]

          pointer to object ({\textless}pointer to 
          xfdata.FL\_OBJECT{\textgreater})

          \item[fname]

          name of bitmap file

        \end{Ventry}

      \end{quote}

\textbf{Status:} Tested + NoDoc + Demo = OK



    \end{boxedminipage}

    \label{xformslib:library:fl_read_bitmapfile}
    \index{xformslib \textit{(package)}!xformslib.library \textit{(module)}!xformslib.library.fl\_read\_bitmapfile \textit{(function)}}

    \vspace{0.5ex}

\hspace{.8\funcindent}\begin{boxedminipage}{\funcwidth}

    \raggedright \textbf{fl\_read\_bitmapfile}(\textit{win}, \textit{filename}, \textit{w}, \textit{h}, \textit{hotx}, \textit{hoty})

    \vspace{-1.5ex}

    \rule{\textwidth}{0.5\fboxrule}
\setlength{\parskip}{2ex}
\setlength{\parskip}{1ex}
      \textbf{Return Value}
    \vspace{-1ex}

      \begin{quote}
      pixmap

      \end{quote}

\textbf{Status:} Untested + NoDoc + NoDemo = NOT OK



    \end{boxedminipage}

    \label{xformslib:library:fl_create_from_bitmapdata}
    \index{xformslib \textit{(package)}!xformslib.library \textit{(module)}!xformslib.library.fl\_create\_from\_bitmapdata \textit{(function)}}

    \vspace{0.5ex}

\hspace{.8\funcindent}\begin{boxedminipage}{\funcwidth}

    \raggedright \textbf{fl\_create\_from\_bitmapdata}(\textit{win}, \textit{data}, \textit{w}, \textit{h})

    \vspace{-1.5ex}

    \rule{\textwidth}{0.5\fboxrule}
\setlength{\parskip}{2ex}
\setlength{\parskip}{1ex}
      \textbf{Parameters}
      \vspace{-1ex}

      \begin{quote}
        \begin{Ventry}{xxxx}

          \item[win]

          window

          \item[data]

          bitmap data

          \item[w]

          width in coord units

          \item[h]

          height in coord units

        \end{Ventry}

      \end{quote}

      \textbf{Return Value}
    \vspace{-1ex}

      \begin{quote}
      pixmap

      \end{quote}

\textbf{Status:} Untested + NoDoc + NoDemo = NOT OK



    \end{boxedminipage}

    \label{xformslib:library:fl_add_pixmap}
    \index{xformslib \textit{(package)}!xformslib.library \textit{(module)}!xformslib.library.fl\_add\_pixmap \textit{(function)}}

    \vspace{0.5ex}

\hspace{.8\funcindent}\begin{boxedminipage}{\funcwidth}

    \raggedright \textbf{fl\_add\_pixmap}(\textit{pixmaptype}, \textit{x}, \textit{y}, \textit{w}, \textit{h}, \textit{label})

    \vspace{-1.5ex}

    \rule{\textwidth}{0.5\fboxrule}
\setlength{\parskip}{2ex}
    Adds a pixmap object.

\setlength{\parskip}{1ex}
      \textbf{Parameters}
      \vspace{-1ex}

      \begin{quote}
        \begin{Ventry}{xxxxxxxxxx}

          \item[pixmaptype]

          type of pixmap to be added

            {\it (type=[num./int] xfc.FL\_NORMAL\_PIXMAP)}

          \item[x]

          horizontal position (upper-left corner)

          \item[y]

          vertical position of bitmap (upper-left corner)

          \item[w]

          width in coord units

          \item[h]

          height in coord units

          \item[label]

          text label of pixmap

        \end{Ventry}

      \end{quote}

      \textbf{Return Value}
    \vspace{-1ex}

      \begin{quote}
      pObject

      \end{quote}

\textbf{Status:} Untested + NoDoc + NoDemo = NOT OK



    \end{boxedminipage}

    \label{xformslib:library:fl_set_pixmap_data}
    \index{xformslib \textit{(package)}!xformslib.library \textit{(module)}!xformslib.library.fl\_set\_pixmap\_data \textit{(function)}}

    \vspace{0.5ex}

\hspace{.8\funcindent}\begin{boxedminipage}{\funcwidth}

    \raggedright \textbf{fl\_set\_pixmap\_data}(\textit{pObject}, \textit{bits})

    \vspace{-1.5ex}

    \rule{\textwidth}{0.5\fboxrule}
\setlength{\parskip}{2ex}
\setlength{\parskip}{1ex}
      \textbf{Parameters}
      \vspace{-1ex}

      \begin{quote}
        \begin{Ventry}{xxxxxxx}

          \item[pObject]

          pointer to object ({\textless}pointer to 
          xfdata.FL\_OBJECT{\textgreater})

          \item[bits]

          bits contents of pixmap

        \end{Ventry}

      \end{quote}

\textbf{Status:} Untested + NoDoc + NoDemo = NOT OK



    \end{boxedminipage}

    \label{xformslib:library:fl_set_pixmap_file}
    \index{xformslib \textit{(package)}!xformslib.library \textit{(module)}!xformslib.library.fl\_set\_pixmap\_file \textit{(function)}}

    \vspace{0.5ex}

\hspace{.8\funcindent}\begin{boxedminipage}{\funcwidth}

    \raggedright \textbf{fl\_set\_pixmap\_file}(\textit{pObject}, \textit{fname})

    \vspace{-1.5ex}

    \rule{\textwidth}{0.5\fboxrule}
\setlength{\parskip}{2ex}
\setlength{\parskip}{1ex}
      \textbf{Parameters}
      \vspace{-1ex}

      \begin{quote}
        \begin{Ventry}{xxxxxxx}

          \item[pObject]

          pointer to object ({\textless}pointer to 
          xfdata.FL\_OBJECT{\textgreater})

          \item[fname]

          name of the pixmap file

        \end{Ventry}

      \end{quote}

\textbf{Status:} Tested + NoDoc + Demo = OK



    \end{boxedminipage}

    \label{xformslib:library:fl_set_pixmap_file}
    \index{xformslib \textit{(package)}!xformslib.library \textit{(module)}!xformslib.library.fl\_set\_pixmap\_file \textit{(function)}}

    \vspace{0.5ex}

\hspace{.8\funcindent}\begin{boxedminipage}{\funcwidth}

    \raggedright \textbf{fl\_set\_pixmapbutton\_file}(\textit{pObject}, \textit{fname})

    \vspace{-1.5ex}

    \rule{\textwidth}{0.5\fboxrule}
\setlength{\parskip}{2ex}
\setlength{\parskip}{1ex}
      \textbf{Parameters}
      \vspace{-1ex}

      \begin{quote}
        \begin{Ventry}{xxxxxxx}

          \item[pObject]

          pointer to object ({\textless}pointer to 
          xfdata.FL\_OBJECT{\textgreater})

          \item[fname]

          name of the pixmap file

        \end{Ventry}

      \end{quote}

\textbf{Status:} Tested + NoDoc + Demo = OK



    \end{boxedminipage}

    \label{xformslib:library:fl_set_pixmap_file}
    \index{xformslib \textit{(package)}!xformslib.library \textit{(module)}!xformslib.library.fl\_set\_pixmap\_file \textit{(function)}}

    \vspace{0.5ex}

\hspace{.8\funcindent}\begin{boxedminipage}{\funcwidth}

    \raggedright \textbf{fl\_set\_pixmapbutton\_datafile}(\textit{pObject}, \textit{fname})

    \vspace{-1.5ex}

    \rule{\textwidth}{0.5\fboxrule}
\setlength{\parskip}{2ex}
\setlength{\parskip}{1ex}
      \textbf{Parameters}
      \vspace{-1ex}

      \begin{quote}
        \begin{Ventry}{xxxxxxx}

          \item[pObject]

          pointer to object ({\textless}pointer to 
          xfdata.FL\_OBJECT{\textgreater})

          \item[fname]

          name of the pixmap file

        \end{Ventry}

      \end{quote}

\textbf{Status:} Tested + NoDoc + Demo = OK



    \end{boxedminipage}

    \label{xformslib:library:fl_set_pixmap_align}
    \index{xformslib \textit{(package)}!xformslib.library \textit{(module)}!xformslib.library.fl\_set\_pixmap\_align \textit{(function)}}

    \vspace{0.5ex}

\hspace{.8\funcindent}\begin{boxedminipage}{\funcwidth}

    \raggedright \textbf{fl\_set\_pixmap\_align}(\textit{pObject}, \textit{align}, \textit{xmargin}, \textit{ymargin})

    \vspace{-1.5ex}

    \rule{\textwidth}{0.5\fboxrule}
\setlength{\parskip}{2ex}
\setlength{\parskip}{1ex}
      \textbf{Parameters}
      \vspace{-1ex}

      \begin{quote}
        \begin{Ventry}{xxxxxxx}

          \item[pObject]

          pixmap object ({\textless}pointer to 
          xfdata.FL\_OBJECT{\textgreater})

          \item[align]

          alignment of pixmap ({\textless}int{\textgreater})

            {\it (type=(from xfdata module) FL\_ALIGN\_CENTER, FL\_ALIGN\_TOP, FL\_ALIGN\_BOTTOM, 
FL\_ALIGN\_LEFT, FL\_ALIGN\_RIGHT, FL\_ALIGN\_LEFT\_TOP, 
FL\_ALIGN\_RIGHT\_TOP, FL\_ALIGN\_LEFT\_BOTTOM, FL\_ALIGN\_RIGHT\_BOTTOM, 
FL\_ALIGN\_INSIDE, FL\_ALIGN\_VERT)}

        \end{Ventry}

      \end{quote}

\textbf{Status:} Tested + NoDoc + Demo = OK



    \end{boxedminipage}

    \label{xformslib:library:fl_set_pixmap_align}
    \index{xformslib \textit{(package)}!xformslib.library \textit{(module)}!xformslib.library.fl\_set\_pixmap\_align \textit{(function)}}

    \vspace{0.5ex}

\hspace{.8\funcindent}\begin{boxedminipage}{\funcwidth}

    \raggedright \textbf{fl\_set\_pixmapbutton\_align}(\textit{pObject}, \textit{align}, \textit{xmargin}, \textit{ymargin})

    \vspace{-1.5ex}

    \rule{\textwidth}{0.5\fboxrule}
\setlength{\parskip}{2ex}
\setlength{\parskip}{1ex}
      \textbf{Parameters}
      \vspace{-1ex}

      \begin{quote}
        \begin{Ventry}{xxxxxxx}

          \item[pObject]

          pixmap object ({\textless}pointer to 
          xfdata.FL\_OBJECT{\textgreater})

          \item[align]

          alignment of pixmap ({\textless}int{\textgreater})

            {\it (type=(from xfdata module) FL\_ALIGN\_CENTER, FL\_ALIGN\_TOP, FL\_ALIGN\_BOTTOM, 
FL\_ALIGN\_LEFT, FL\_ALIGN\_RIGHT, FL\_ALIGN\_LEFT\_TOP, 
FL\_ALIGN\_RIGHT\_TOP, FL\_ALIGN\_LEFT\_BOTTOM, FL\_ALIGN\_RIGHT\_BOTTOM, 
FL\_ALIGN\_INSIDE, FL\_ALIGN\_VERT)}

        \end{Ventry}

      \end{quote}

\textbf{Status:} Tested + NoDoc + Demo = OK



    \end{boxedminipage}

    \label{xformslib:library:fl_set_pixmap_pixmap}
    \index{xformslib \textit{(package)}!xformslib.library \textit{(module)}!xformslib.library.fl\_set\_pixmap\_pixmap \textit{(function)}}

    \vspace{0.5ex}

\hspace{.8\funcindent}\begin{boxedminipage}{\funcwidth}

    \raggedright \textbf{fl\_set\_pixmap\_pixmap}(\textit{pObject}, \textit{idnum}, \textit{mask})

    \vspace{-1.5ex}

    \rule{\textwidth}{0.5\fboxrule}
\setlength{\parskip}{2ex}
\setlength{\parskip}{1ex}
      \textbf{Parameters}
      \vspace{-1ex}

      \begin{quote}
        \begin{Ventry}{xxxxxxx}

          \item[pObject]

          pointer to object ({\textless}pointer to 
          xfdata.FL\_OBJECT{\textgreater})

        \end{Ventry}

      \end{quote}

\textbf{Status:} Untested + NoDoc + NoDemo = NOT OK



    \end{boxedminipage}

    \label{xformslib:library:fl_set_pixmap_pixmap}
    \index{xformslib \textit{(package)}!xformslib.library \textit{(module)}!xformslib.library.fl\_set\_pixmap\_pixmap \textit{(function)}}

    \vspace{0.5ex}

\hspace{.8\funcindent}\begin{boxedminipage}{\funcwidth}

    \raggedright \textbf{fl\_set\_pixmapbutton\_pixmap}(\textit{pObject}, \textit{idnum}, \textit{mask})

    \vspace{-1.5ex}

    \rule{\textwidth}{0.5\fboxrule}
\setlength{\parskip}{2ex}
\setlength{\parskip}{1ex}
      \textbf{Parameters}
      \vspace{-1ex}

      \begin{quote}
        \begin{Ventry}{xxxxxxx}

          \item[pObject]

          pointer to object ({\textless}pointer to 
          xfdata.FL\_OBJECT{\textgreater})

        \end{Ventry}

      \end{quote}

\textbf{Status:} Untested + NoDoc + NoDemo = NOT OK



    \end{boxedminipage}

    \label{xformslib:library:fl_set_pixmap_colorcloseness}
    \index{xformslib \textit{(package)}!xformslib.library \textit{(module)}!xformslib.library.fl\_set\_pixmap\_colorcloseness \textit{(function)}}

    \vspace{0.5ex}

\hspace{.8\funcindent}\begin{boxedminipage}{\funcwidth}

    \raggedright \textbf{fl\_set\_pixmap\_colorcloseness}(\textit{red}, \textit{green}, \textit{blue})

    \vspace{-1.5ex}

    \rule{\textwidth}{0.5\fboxrule}
\setlength{\parskip}{2ex}
\setlength{\parskip}{1ex}
\textbf{Status:} Untested + NoDoc + NoDemo = NOT OK



    \end{boxedminipage}

    \label{xformslib:library:fl_free_pixmap_pixmap}
    \index{xformslib \textit{(package)}!xformslib.library \textit{(module)}!xformslib.library.fl\_free\_pixmap\_pixmap \textit{(function)}}

    \vspace{0.5ex}

\hspace{.8\funcindent}\begin{boxedminipage}{\funcwidth}

    \raggedright \textbf{fl\_free\_pixmap\_pixmap}(\textit{pObject})

    \vspace{-1.5ex}

    \rule{\textwidth}{0.5\fboxrule}
\setlength{\parskip}{2ex}
\setlength{\parskip}{1ex}
      \textbf{Parameters}
      \vspace{-1ex}

      \begin{quote}
        \begin{Ventry}{xxxxxxx}

          \item[pObject]

          pointer to object ({\textless}pointer to 
          xfdata.FL\_OBJECT{\textgreater})

        \end{Ventry}

      \end{quote}

\textbf{Status:} Tested + NoDoc + Demo = OK



    \end{boxedminipage}

    \label{xformslib:library:fl_free_pixmap_pixmap}
    \index{xformslib \textit{(package)}!xformslib.library \textit{(module)}!xformslib.library.fl\_free\_pixmap\_pixmap \textit{(function)}}

    \vspace{0.5ex}

\hspace{.8\funcindent}\begin{boxedminipage}{\funcwidth}

    \raggedright \textbf{fl\_free\_pixmapbutton\_pixmap}(\textit{pObject})

    \vspace{-1.5ex}

    \rule{\textwidth}{0.5\fboxrule}
\setlength{\parskip}{2ex}
\setlength{\parskip}{1ex}
      \textbf{Parameters}
      \vspace{-1ex}

      \begin{quote}
        \begin{Ventry}{xxxxxxx}

          \item[pObject]

          pointer to object ({\textless}pointer to 
          xfdata.FL\_OBJECT{\textgreater})

        \end{Ventry}

      \end{quote}

\textbf{Status:} Tested + NoDoc + Demo = OK



    \end{boxedminipage}

    \label{xformslib:library:fl_get_pixmap_pixmap}
    \index{xformslib \textit{(package)}!xformslib.library \textit{(module)}!xformslib.library.fl\_get\_pixmap\_pixmap \textit{(function)}}

    \vspace{0.5ex}

\hspace{.8\funcindent}\begin{boxedminipage}{\funcwidth}

    \raggedright \textbf{fl\_get\_pixmap\_pixmap}(\textit{pObject})

    \vspace{-1.5ex}

    \rule{\textwidth}{0.5\fboxrule}
\setlength{\parskip}{2ex}
\setlength{\parskip}{1ex}
      \textbf{Parameters}
      \vspace{-1ex}

      \begin{quote}
        \begin{Ventry}{xxxxxxx}

          \item[pObject]

          object ({\textless}pointer to xfdata.FL\_OBJECT{\textgreater})

        \end{Ventry}

      \end{quote}

      \textbf{Return Value}
    \vspace{-1ex}

      \begin{quote}
      pixmap, Pixmap, Pixmap\_mask

      \end{quote}

\textbf{Attention:} API change from XForms - upstream was fl\_get\_pixmap\_pixmap(pObject, p, 
m)



\textbf{Status:} Untested + NoDoc + NoDemo = NOT OK



    \end{boxedminipage}

    \label{xformslib:library:fl_get_pixmap_pixmap}
    \index{xformslib \textit{(package)}!xformslib.library \textit{(module)}!xformslib.library.fl\_get\_pixmap\_pixmap \textit{(function)}}

    \vspace{0.5ex}

\hspace{.8\funcindent}\begin{boxedminipage}{\funcwidth}

    \raggedright \textbf{fl\_get\_pixmapbutton\_pixmap}(\textit{pObject})

    \vspace{-1.5ex}

    \rule{\textwidth}{0.5\fboxrule}
\setlength{\parskip}{2ex}
\setlength{\parskip}{1ex}
      \textbf{Parameters}
      \vspace{-1ex}

      \begin{quote}
        \begin{Ventry}{xxxxxxx}

          \item[pObject]

          object ({\textless}pointer to xfdata.FL\_OBJECT{\textgreater})

        \end{Ventry}

      \end{quote}

      \textbf{Return Value}
    \vspace{-1ex}

      \begin{quote}
      pixmap, Pixmap, Pixmap\_mask

      \end{quote}

\textbf{Attention:} API change from XForms - upstream was fl\_get\_pixmap\_pixmap(pObject, p, 
m)



\textbf{Status:} Untested + NoDoc + NoDemo = NOT OK



    \end{boxedminipage}

    \label{xformslib:library:fl_read_pixmapfile}
    \index{xformslib \textit{(package)}!xformslib.library \textit{(module)}!xformslib.library.fl\_read\_pixmapfile \textit{(function)}}

    \vspace{0.5ex}

\hspace{.8\funcindent}\begin{boxedminipage}{\funcwidth}

    \raggedright \textbf{fl\_read\_pixmapfile}(\textit{win}, \textit{filename}, \textit{tran})

    \vspace{-1.5ex}

    \rule{\textwidth}{0.5\fboxrule}
\setlength{\parskip}{2ex}
\setlength{\parskip}{1ex}
      \textbf{Return Value}
    \vspace{-1ex}

      \begin{quote}
      pixmap, w, h, shapemask, hotx, hoty

      \end{quote}

\textbf{Attention:} API change from XForms - upstream was fl\_read\_pixmapfile(win, filename, 
w, h, shape\_mask, hotx, hoty, tran)



\textbf{Status:} Tested + NoDoc + Demo = OK



    \end{boxedminipage}

    \label{xformslib:library:fl_create_from_pixmapdata}
    \index{xformslib \textit{(package)}!xformslib.library \textit{(module)}!xformslib.library.fl\_create\_from\_pixmapdata \textit{(function)}}

    \vspace{0.5ex}

\hspace{.8\funcindent}\begin{boxedminipage}{\funcwidth}

    \raggedright \textbf{fl\_create\_from\_pixmapdata}(\textit{win}, \textit{data}, \textit{w}, \textit{h}, \textit{sm}, \textit{hotx}, \textit{hoty}, \textit{tran})

    \vspace{-1.5ex}

    \rule{\textwidth}{0.5\fboxrule}
\setlength{\parskip}{2ex}
\setlength{\parskip}{1ex}
      \textbf{Return Value}
    \vspace{-1ex}

      \begin{quote}
      pixmap

      \end{quote}

\textbf{Status:} Untested + NoDoc + NoDemo = NOT OK



    \end{boxedminipage}

    \label{xformslib:library:fl_free_pixmap}
    \index{xformslib \textit{(package)}!xformslib.library \textit{(module)}!xformslib.library.fl\_free\_pixmap \textit{(function)}}

    \vspace{0.5ex}

\hspace{.8\funcindent}\begin{boxedminipage}{\funcwidth}

    \raggedright \textbf{fl\_free\_pixmap}(\textit{idnum})

    \vspace{-1.5ex}

    \rule{\textwidth}{0.5\fboxrule}
\setlength{\parskip}{2ex}
\setlength{\parskip}{1ex}
      \textbf{Parameters}
      \vspace{-1ex}

      \begin{quote}
        \begin{Ventry}{xxxxx}

          \item[idnum]

          Pixmap id to be freed

        \end{Ventry}

      \end{quote}

\textbf{Status:} Untested + NoDoc + NoDemo = NOT OK



    \end{boxedminipage}

    \label{xformslib:library:fl_add_box}
    \index{xformslib \textit{(package)}!xformslib.library \textit{(module)}!xformslib.library.fl\_add\_box \textit{(function)}}

    \vspace{0.5ex}

\hspace{.8\funcindent}\begin{boxedminipage}{\funcwidth}

    \raggedright \textbf{fl\_add\_box}(\textit{boxtype}, \textit{x}, \textit{y}, \textit{w}, \textit{h}, \textit{label})

    \vspace{-1.5ex}

    \rule{\textwidth}{0.5\fboxrule}
\setlength{\parskip}{2ex}
    Adds a box object.

\setlength{\parskip}{1ex}
      \textbf{Parameters}
      \vspace{-1ex}

      \begin{quote}
        \begin{Ventry}{xxxxxxx}

          \item[boxtype]

          type of the box to be added ({\textless}int{\textgreater})

            {\it (type=(from xfdata module) FL\_NO\_BOX, FL\_UP\_BOX, FL\_DOWN\_BOX, 
FL\_BORDER\_BOX, FL\_SHADOW\_BOX, FL\_FRAME\_BOX, FL\_ROUNDED\_BOX, 
FL\_EMBOSSED\_BOX, FL\_FLAT\_BOX, FL\_RFLAT\_BOX, FL\_RSHADOW\_BOX, 
FL\_OVAL\_BOX, FL\_ROUNDED3D\_UPBOX, FL\_ROUNDED3D\_DOWNBOX, 
FL\_OVAL3D\_UPBOX, FL\_OVAL3D\_DOWNBOX, FL\_OVAL3D\_FRAMEBOX, 
FL\_OVAL3D\_EMBOSSEDBOX)}

          \item[x]

          horizontal position (upper-left corner)

          \item[y]

          vertical position (upper-left corner)

          \item[w]

          width in coord units

          \item[h]

          height in coord units

          \item[label]

          text label of box

        \end{Ventry}

      \end{quote}

      \textbf{Return Value}
    \vspace{-1ex}

      \begin{quote}
      pObject

      \end{quote}

\textbf{Status:} Tested + NoDoc + Demo = OK



    \end{boxedminipage}

    \label{xformslib:library:fl_add_browser}
    \index{xformslib \textit{(package)}!xformslib.library \textit{(module)}!xformslib.library.fl\_add\_browser \textit{(function)}}

    \vspace{0.5ex}

\hspace{.8\funcindent}\begin{boxedminipage}{\funcwidth}

    \raggedright \textbf{fl\_add\_browser}(\textit{browsertype}, \textit{x}, \textit{y}, \textit{w}, \textit{h}, \textit{label})

    \vspace{-1.5ex}

    \rule{\textwidth}{0.5\fboxrule}
\setlength{\parskip}{2ex}
    Adds a browser object.

\setlength{\parskip}{1ex}
      \textbf{Parameters}
      \vspace{-1ex}

      \begin{quote}
        \begin{Ventry}{xxxxxxxxxxx}

          \item[browsertype]

          type of the browser to be added

            {\it (type=[num./int] xfc.FL\_NORMAL\_BROWSER, xfc.FL\_SELECT\_BROWSER, 
xfc.FL\_HOLD\_BROWSER, xfc.FL\_MULTI\_BROWSER)}

          \item[x]

          horizontal position (upper-left corner)

          \item[y]

          vertical position (upper-left corner)

          \item[w]

          width in coord units

          \item[h]

          height in coord units

          \item[label]

          text label of browser

        \end{Ventry}

      \end{quote}

      \textbf{Return Value}
    \vspace{-1ex}

      \begin{quote}
      pObject

      \end{quote}

\textbf{Status:} Tested + NoDoc + Demo = OK



    \end{boxedminipage}

    \label{xformslib:library:fl_clear_browser}
    \index{xformslib \textit{(package)}!xformslib.library \textit{(module)}!xformslib.library.fl\_clear\_browser \textit{(function)}}

    \vspace{0.5ex}

\hspace{.8\funcindent}\begin{boxedminipage}{\funcwidth}

    \raggedright \textbf{fl\_clear\_browser}(\textit{pObject})

    \vspace{-1.5ex}

    \rule{\textwidth}{0.5\fboxrule}
\setlength{\parskip}{2ex}
    Clears contents of a browser object.

\setlength{\parskip}{1ex}
      \textbf{Parameters}
      \vspace{-1ex}

      \begin{quote}
        \begin{Ventry}{xxxxxxx}

          \item[pObject]

          pointer to browser object ({\textless}pointer to 
          xfdata.FL\_OBJECT{\textgreater})

        \end{Ventry}

      \end{quote}

\textbf{Status:} Tested + NoDoc + Demo = OK



    \end{boxedminipage}

    \label{xformslib:library:fl_add_browser_line}
    \index{xformslib \textit{(package)}!xformslib.library \textit{(module)}!xformslib.library.fl\_add\_browser\_line \textit{(function)}}

    \vspace{0.5ex}

\hspace{.8\funcindent}\begin{boxedminipage}{\funcwidth}

    \raggedright \textbf{fl\_add\_browser\_line}(\textit{pObject}, \textit{newtext})

    \vspace{-1.5ex}

    \rule{\textwidth}{0.5\fboxrule}
\setlength{\parskip}{2ex}
    Add a line to a browser object.

\setlength{\parskip}{1ex}
      \textbf{Parameters}
      \vspace{-1ex}

      \begin{quote}
        \begin{Ventry}{xxxxxxx}

          \item[pObject]

          pointer to browser object ({\textless}pointer to 
          xfdata.FL\_OBJECT{\textgreater})

          \item[newtext]

          line of text to be added

        \end{Ventry}

      \end{quote}

\textbf{Status:} Tested + NoDoc + Demo = OK



    \end{boxedminipage}

    \label{xformslib:library:fl_addto_browser}
    \index{xformslib \textit{(package)}!xformslib.library \textit{(module)}!xformslib.library.fl\_addto\_browser \textit{(function)}}

    \vspace{0.5ex}

\hspace{.8\funcindent}\begin{boxedminipage}{\funcwidth}

    \raggedright \textbf{fl\_addto\_browser}(\textit{pObject}, \textit{newtext})

    \vspace{-1.5ex}

    \rule{\textwidth}{0.5\fboxrule}
\setlength{\parskip}{2ex}
\setlength{\parskip}{1ex}
      \textbf{Parameters}
      \vspace{-1ex}

      \begin{quote}
        \begin{Ventry}{xxxxxxx}

          \item[pObject]

          pointer to object ({\textless}pointer to 
          xfdata.FL\_OBJECT{\textgreater})

        \end{Ventry}

      \end{quote}

\textbf{Status:} Tested + NoDoc + Demo = OK



    \end{boxedminipage}

    \label{xformslib:library:fl_addto_browser_chars}
    \index{xformslib \textit{(package)}!xformslib.library \textit{(module)}!xformslib.library.fl\_addto\_browser\_chars \textit{(function)}}

    \vspace{0.5ex}

\hspace{.8\funcindent}\begin{boxedminipage}{\funcwidth}

    \raggedright \textbf{fl\_addto\_browser\_chars}(\textit{pObject}, \textit{browsertext})

    \vspace{-1.5ex}

    \rule{\textwidth}{0.5\fboxrule}
\setlength{\parskip}{2ex}
\setlength{\parskip}{1ex}
\textbf{Status:} Untested + NoDoc + NoDemo = NOT OK



    \end{boxedminipage}

    \label{xformslib:library:fl_addto_browser_chars}
    \index{xformslib \textit{(package)}!xformslib.library \textit{(module)}!xformslib.library.fl\_addto\_browser\_chars \textit{(function)}}

    \vspace{0.5ex}

\hspace{.8\funcindent}\begin{boxedminipage}{\funcwidth}

    \raggedright \textbf{fl\_append\_browser}(\textit{pObject}, \textit{browsertext})

    \vspace{-1.5ex}

    \rule{\textwidth}{0.5\fboxrule}
\setlength{\parskip}{2ex}
\setlength{\parskip}{1ex}
\textbf{Status:} Untested + NoDoc + NoDemo = NOT OK



    \end{boxedminipage}

    \label{xformslib:library:fl_insert_browser_line}
    \index{xformslib \textit{(package)}!xformslib.library \textit{(module)}!xformslib.library.fl\_insert\_browser\_line \textit{(function)}}

    \vspace{0.5ex}

\hspace{.8\funcindent}\begin{boxedminipage}{\funcwidth}

    \raggedright \textbf{fl\_insert\_browser\_line}(\textit{pObject}, \textit{linenum}, \textit{newtext})

    \vspace{-1.5ex}

    \rule{\textwidth}{0.5\fboxrule}
\setlength{\parskip}{2ex}
\setlength{\parskip}{1ex}
      \textbf{Parameters}
      \vspace{-1ex}

      \begin{quote}
        \begin{Ventry}{xxxxxxx}

          \item[pObject]

          pointer to object ({\textless}pointer to 
          xfdata.FL\_OBJECT{\textgreater})

        \end{Ventry}

      \end{quote}

\textbf{Status:} Tested + NoDoc + Demo = OK



    \end{boxedminipage}

    \label{xformslib:library:fl_delete_browser_line}
    \index{xformslib \textit{(package)}!xformslib.library \textit{(module)}!xformslib.library.fl\_delete\_browser\_line \textit{(function)}}

    \vspace{0.5ex}

\hspace{.8\funcindent}\begin{boxedminipage}{\funcwidth}

    \raggedright \textbf{fl\_delete\_browser\_line}(\textit{pObject}, \textit{linenum})

    \vspace{-1.5ex}

    \rule{\textwidth}{0.5\fboxrule}
\setlength{\parskip}{2ex}
\setlength{\parskip}{1ex}
      \textbf{Parameters}
      \vspace{-1ex}

      \begin{quote}
        \begin{Ventry}{xxxxxxx}

          \item[pObject]

          pointer to browser object ({\textless}pointer to 
          xfdata.FL\_OBJECT{\textgreater})

          \item[linenum]

          line number to delete

        \end{Ventry}

      \end{quote}

\textbf{Status:} Tested + NoDoc + Demo = OK



    \end{boxedminipage}

    \label{xformslib:library:fl_replace_browser_line}
    \index{xformslib \textit{(package)}!xformslib.library \textit{(module)}!xformslib.library.fl\_replace\_browser\_line \textit{(function)}}

    \vspace{0.5ex}

\hspace{.8\funcindent}\begin{boxedminipage}{\funcwidth}

    \raggedright \textbf{fl\_replace\_browser\_line}(\textit{pObject}, \textit{linenum}, \textit{newtext})

    \vspace{-1.5ex}

    \rule{\textwidth}{0.5\fboxrule}
\setlength{\parskip}{2ex}
\setlength{\parskip}{1ex}
      \textbf{Parameters}
      \vspace{-1ex}

      \begin{quote}
        \begin{Ventry}{xxxxxxx}

          \item[pObject]

          pointer to browser object ({\textless}pointer to 
          xfdata.FL\_OBJECT{\textgreater})

          \item[linenum]

          line number to replace

          \item[newtext]

          text line used as replacement

        \end{Ventry}

      \end{quote}

\textbf{Status:} Tested + NoDoc + Demo = OK



    \end{boxedminipage}

    \label{xformslib:library:fl_get_browser_line}
    \index{xformslib \textit{(package)}!xformslib.library \textit{(module)}!xformslib.library.fl\_get\_browser\_line \textit{(function)}}

    \vspace{0.5ex}

\hspace{.8\funcindent}\begin{boxedminipage}{\funcwidth}

    \raggedright \textbf{fl\_get\_browser\_line}(\textit{pObject}, \textit{linenum})

    \vspace{-1.5ex}

    \rule{\textwidth}{0.5\fboxrule}
\setlength{\parskip}{2ex}
\setlength{\parskip}{1ex}
      \textbf{Parameters}
      \vspace{-1ex}

      \begin{quote}
        \begin{Ventry}{xxxxxxx}

          \item[pObject]

          pointer to browser object ({\textless}pointer to 
          xfdata.FL\_OBJECT{\textgreater})

          \item[linenum]

          line number to return

        \end{Ventry}

      \end{quote}

      \textbf{Return Value}
    \vspace{-1ex}

      \begin{quote}
      line string

      \end{quote}

\textbf{Status:} Tested + NoDoc + Demo = OK



    \end{boxedminipage}

    \label{xformslib:library:fl_load_browser}
    \index{xformslib \textit{(package)}!xformslib.library \textit{(module)}!xformslib.library.fl\_load\_browser \textit{(function)}}

    \vspace{0.5ex}

\hspace{.8\funcindent}\begin{boxedminipage}{\funcwidth}

    \raggedright \textbf{fl\_load\_browser}(\textit{pObject}, \textit{filename})

    \vspace{-1.5ex}

    \rule{\textwidth}{0.5\fboxrule}
\setlength{\parskip}{2ex}
\setlength{\parskip}{1ex}
      \textbf{Parameters}
      \vspace{-1ex}

      \begin{quote}
        \begin{Ventry}{xxxxxxx}

          \item[pObject]

          pointer to object ({\textless}pointer to 
          xfdata.FL\_OBJECT{\textgreater})

        \end{Ventry}

      \end{quote}

      \textbf{Return Value}
    \vspace{-1ex}

      \begin{quote}
      num

      \end{quote}

\textbf{Status:} Tested + NoDoc + Demo = OK



    \end{boxedminipage}

    \label{xformslib:library:fl_select_browser_line}
    \index{xformslib \textit{(package)}!xformslib.library \textit{(module)}!xformslib.library.fl\_select\_browser\_line \textit{(function)}}

    \vspace{0.5ex}

\hspace{.8\funcindent}\begin{boxedminipage}{\funcwidth}

    \raggedright \textbf{fl\_select\_browser\_line}(\textit{pObject}, \textit{line})

    \vspace{-1.5ex}

    \rule{\textwidth}{0.5\fboxrule}
\setlength{\parskip}{2ex}
\setlength{\parskip}{1ex}
      \textbf{Parameters}
      \vspace{-1ex}

      \begin{quote}
        \begin{Ventry}{xxxxxxx}

          \item[pObject]

          pointer to object ({\textless}pointer to 
          xfdata.FL\_OBJECT{\textgreater})

        \end{Ventry}

      \end{quote}

\textbf{Status:} Untested + NoDoc + NoDemo = NOT OK



    \end{boxedminipage}

    \label{xformslib:library:fl_deselect_browser_line}
    \index{xformslib \textit{(package)}!xformslib.library \textit{(module)}!xformslib.library.fl\_deselect\_browser\_line \textit{(function)}}

    \vspace{0.5ex}

\hspace{.8\funcindent}\begin{boxedminipage}{\funcwidth}

    \raggedright \textbf{fl\_deselect\_browser\_line}(\textit{pObject}, \textit{line})

    \vspace{-1.5ex}

    \rule{\textwidth}{0.5\fboxrule}
\setlength{\parskip}{2ex}
\setlength{\parskip}{1ex}
      \textbf{Parameters}
      \vspace{-1ex}

      \begin{quote}
        \begin{Ventry}{xxxxxxx}

          \item[pObject]

          pointer to object ({\textless}pointer to 
          xfdata.FL\_OBJECT{\textgreater})

        \end{Ventry}

      \end{quote}

\textbf{Status:} Tested + NoDoc + Demo = OK



    \end{boxedminipage}

    \label{xformslib:library:fl_deselect_browser}
    \index{xformslib \textit{(package)}!xformslib.library \textit{(module)}!xformslib.library.fl\_deselect\_browser \textit{(function)}}

    \vspace{0.5ex}

\hspace{.8\funcindent}\begin{boxedminipage}{\funcwidth}

    \raggedright \textbf{fl\_deselect\_browser}(\textit{pObject})

    \vspace{-1.5ex}

    \rule{\textwidth}{0.5\fboxrule}
\setlength{\parskip}{2ex}
\setlength{\parskip}{1ex}
      \textbf{Parameters}
      \vspace{-1ex}

      \begin{quote}
        \begin{Ventry}{xxxxxxx}

          \item[pObject]

          pointer to browser object ({\textless}pointer to 
          xfdata.FL\_OBJECT{\textgreater})

        \end{Ventry}

      \end{quote}

\textbf{Status:} Tested + NoDoc + Demo = OK



    \end{boxedminipage}

    \label{xformslib:library:fl_isselected_browser_line}
    \index{xformslib \textit{(package)}!xformslib.library \textit{(module)}!xformslib.library.fl\_isselected\_browser\_line \textit{(function)}}

    \vspace{0.5ex}

\hspace{.8\funcindent}\begin{boxedminipage}{\funcwidth}

    \raggedright \textbf{fl\_isselected\_browser\_line}(\textit{pObject}, \textit{line})

    \vspace{-1.5ex}

    \rule{\textwidth}{0.5\fboxrule}
\setlength{\parskip}{2ex}
\setlength{\parskip}{1ex}
      \textbf{Parameters}
      \vspace{-1ex}

      \begin{quote}
        \begin{Ventry}{xxxxxxx}

          \item[pObject]

          pointer to object ({\textless}pointer to 
          xfdata.FL\_OBJECT{\textgreater})

        \end{Ventry}

      \end{quote}

      \textbf{Return Value}
    \vspace{-1ex}

      \begin{quote}
      num

      \end{quote}

\textbf{Status:} Untested + NoDoc + NoDemo = NOT OK



    \end{boxedminipage}

    \label{xformslib:library:fl_get_browser_topline}
    \index{xformslib \textit{(package)}!xformslib.library \textit{(module)}!xformslib.library.fl\_get\_browser\_topline \textit{(function)}}

    \vspace{0.5ex}

\hspace{.8\funcindent}\begin{boxedminipage}{\funcwidth}

    \raggedright \textbf{fl\_get\_browser\_topline}(\textit{pObject})

    \vspace{-1.5ex}

    \rule{\textwidth}{0.5\fboxrule}
\setlength{\parskip}{2ex}
\setlength{\parskip}{1ex}
      \textbf{Parameters}
      \vspace{-1ex}

      \begin{quote}
        \begin{Ventry}{xxxxxxx}

          \item[pObject]

          pointer to object ({\textless}pointer to 
          xfdata.FL\_OBJECT{\textgreater})

        \end{Ventry}

      \end{quote}

      \textbf{Return Value}
    \vspace{-1ex}

      \begin{quote}
      num

      \end{quote}

\textbf{Status:} Untested + NoDoc + NoDemo = NOT OK



    \end{boxedminipage}

    \label{xformslib:library:fl_get_browser}
    \index{xformslib \textit{(package)}!xformslib.library \textit{(module)}!xformslib.library.fl\_get\_browser \textit{(function)}}

    \vspace{0.5ex}

\hspace{.8\funcindent}\begin{boxedminipage}{\funcwidth}

    \raggedright \textbf{fl\_get\_browser}(\textit{pObject})

    \vspace{-1.5ex}

    \rule{\textwidth}{0.5\fboxrule}
\setlength{\parskip}{2ex}
\setlength{\parskip}{1ex}
      \textbf{Parameters}
      \vspace{-1ex}

      \begin{quote}
        \begin{Ventry}{xxxxxxx}

          \item[pObject]

          pointer to browser object ({\textless}pointer to 
          xfdata.FL\_OBJECT{\textgreater})

        \end{Ventry}

      \end{quote}

      \textbf{Return Value}
    \vspace{-1ex}

      \begin{quote}
      num

      \end{quote}

\textbf{Status:} Tested + NoDoc + Demo = OK



    \end{boxedminipage}

    \label{xformslib:library:fl_get_browser_maxline}
    \index{xformslib \textit{(package)}!xformslib.library \textit{(module)}!xformslib.library.fl\_get\_browser\_maxline \textit{(function)}}

    \vspace{0.5ex}

\hspace{.8\funcindent}\begin{boxedminipage}{\funcwidth}

    \raggedright \textbf{fl\_get\_browser\_maxline}(\textit{pObject})

    \vspace{-1.5ex}

    \rule{\textwidth}{0.5\fboxrule}
\setlength{\parskip}{2ex}
\setlength{\parskip}{1ex}
      \textbf{Parameters}
      \vspace{-1ex}

      \begin{quote}
        \begin{Ventry}{xxxxxxx}

          \item[pObject]

          pointer to object ({\textless}pointer to 
          xfdata.FL\_OBJECT{\textgreater})

        \end{Ventry}

      \end{quote}

      \textbf{Return Value}
    \vspace{-1ex}

      \begin{quote}
      line num

      \end{quote}

\textbf{Status:} Untested + NoDoc + NoDemo = NOT OK



    \end{boxedminipage}

    \label{xformslib:library:fl_get_browser_screenlines}
    \index{xformslib \textit{(package)}!xformslib.library \textit{(module)}!xformslib.library.fl\_get\_browser\_screenlines \textit{(function)}}

    \vspace{0.5ex}

\hspace{.8\funcindent}\begin{boxedminipage}{\funcwidth}

    \raggedright \textbf{fl\_get\_browser\_screenlines}(\textit{pObject})

    \vspace{-1.5ex}

    \rule{\textwidth}{0.5\fboxrule}
\setlength{\parskip}{2ex}
    Returns an approximation of the number of lines shown in the browser.

\setlength{\parskip}{1ex}
      \textbf{Parameters}
      \vspace{-1ex}

      \begin{quote}
        \begin{Ventry}{xxxxxxx}

          \item[pObject]

          pointer to browser object ({\textless}pointer to 
          xfdata.FL\_OBJECT{\textgreater})

        \end{Ventry}

      \end{quote}

      \textbf{Return Value}
    \vspace{-1ex}

      \begin{quote}
      lines num

      \end{quote}

\textbf{Status:} Untested + NoDoc + NoDemo = NOT OK



    \end{boxedminipage}

    \label{xformslib:library:fl_set_browser_topline}
    \index{xformslib \textit{(package)}!xformslib.library \textit{(module)}!xformslib.library.fl\_set\_browser\_topline \textit{(function)}}

    \vspace{0.5ex}

\hspace{.8\funcindent}\begin{boxedminipage}{\funcwidth}

    \raggedright \textbf{fl\_set\_browser\_topline}(\textit{pObject}, \textit{line})

    \vspace{-1.5ex}

    \rule{\textwidth}{0.5\fboxrule}
\setlength{\parskip}{2ex}
    Moves a line to the top of the browser.

\setlength{\parskip}{1ex}
      \textbf{Parameters}
      \vspace{-1ex}

      \begin{quote}
        \begin{Ventry}{xxxxxxx}

          \item[pObject]

          pointer to browser object ({\textless}pointer to 
          xfdata.FL\_OBJECT{\textgreater})

          \item[line]

          number of text line to be moved to top

        \end{Ventry}

      \end{quote}

\textbf{Status:} Untested + NoDoc + NoDemo = NOT OK



    \end{boxedminipage}

    \label{xformslib:library:fl_set_browser_bottomline}
    \index{xformslib \textit{(package)}!xformslib.library \textit{(module)}!xformslib.library.fl\_set\_browser\_bottomline \textit{(function)}}

    \vspace{0.5ex}

\hspace{.8\funcindent}\begin{boxedminipage}{\funcwidth}

    \raggedright \textbf{fl\_set\_browser\_bottomline}(\textit{pObject}, \textit{line})

    \vspace{-1.5ex}

    \rule{\textwidth}{0.5\fboxrule}
\setlength{\parskip}{2ex}
    Moves a line to the bottom of the browser.

\setlength{\parskip}{1ex}
      \textbf{Parameters}
      \vspace{-1ex}

      \begin{quote}
        \begin{Ventry}{xxxxxxx}

          \item[pObject]

          pointer to browser object ({\textless}pointer to 
          xfdata.FL\_OBJECT{\textgreater})

          \item[line]

          number of text line to be moved to bottom

        \end{Ventry}

      \end{quote}

\textbf{Status:} Untested + NoDoc + NoDemo = NOT OK



    \end{boxedminipage}

    \label{xformslib:library:fl_set_browser_fontsize}
    \index{xformslib \textit{(package)}!xformslib.library \textit{(module)}!xformslib.library.fl\_set\_browser\_fontsize \textit{(function)}}

    \vspace{0.5ex}

\hspace{.8\funcindent}\begin{boxedminipage}{\funcwidth}

    \raggedright \textbf{fl\_set\_browser\_fontsize}(\textit{pObject}, \textit{size})

    \vspace{-1.5ex}

    \rule{\textwidth}{0.5\fboxrule}
\setlength{\parskip}{2ex}
    Sets the font size of a browser object.

\setlength{\parskip}{1ex}
      \textbf{Parameters}
      \vspace{-1ex}

      \begin{quote}
        \begin{Ventry}{xxxxxxx}

          \item[pObject]

          pointer to browser object ({\textless}pointer to 
          xfdata.FL\_OBJECT{\textgreater})

          \item[size]

          font size to be set

        \end{Ventry}

      \end{quote}

\textbf{Status:} Tested + NoDoc + Demo = OK



    \end{boxedminipage}

    \label{xformslib:library:fl_set_browser_fontstyle}
    \index{xformslib \textit{(package)}!xformslib.library \textit{(module)}!xformslib.library.fl\_set\_browser\_fontstyle \textit{(function)}}

    \vspace{0.5ex}

\hspace{.8\funcindent}\begin{boxedminipage}{\funcwidth}

    \raggedright \textbf{fl\_set\_browser\_fontstyle}(\textit{pObject}, \textit{style})

    \vspace{-1.5ex}

    \rule{\textwidth}{0.5\fboxrule}
\setlength{\parskip}{2ex}
    Sets the font style of a browser object.

\setlength{\parskip}{1ex}
      \textbf{Parameters}
      \vspace{-1ex}

      \begin{quote}
        \begin{Ventry}{xxxxxxx}

          \item[pObject]

          pointer to browser object ({\textless}pointer to 
          xfdata.FL\_OBJECT{\textgreater})

          \item[style]

          font style to be set

        \end{Ventry}

      \end{quote}

\textbf{Status:} Untested + NoDoc + NoDemo = NOT OK



    \end{boxedminipage}

    \label{xformslib:library:fl_set_browser_specialkey}
    \index{xformslib \textit{(package)}!xformslib.library \textit{(module)}!xformslib.library.fl\_set\_browser\_specialkey \textit{(function)}}

    \vspace{0.5ex}

\hspace{.8\funcindent}\begin{boxedminipage}{\funcwidth}

    \raggedright \textbf{fl\_set\_browser\_specialkey}(\textit{pObject}, \textit{specialkey})

    \vspace{-1.5ex}

    \rule{\textwidth}{0.5\fboxrule}
\setlength{\parskip}{2ex}
    Sets the escape key used in the text.

\setlength{\parskip}{1ex}
      \textbf{Parameters}
      \vspace{-1ex}

      \begin{quote}
        \begin{Ventry}{xxxxxxxxxx}

          \item[pObject]

          pointer to browser object ({\textless}pointer to 
          xfdata.FL\_OBJECT{\textgreater})

          \item[specialkey]

          escape key to be set

        \end{Ventry}

      \end{quote}

\textbf{Status:} Untested + NoDoc + NoDemo = NOT OK



    \end{boxedminipage}

    \label{xformslib:library:fl_set_browser_vscrollbar}
    \index{xformslib \textit{(package)}!xformslib.library \textit{(module)}!xformslib.library.fl\_set\_browser\_vscrollbar \textit{(function)}}

    \vspace{0.5ex}

\hspace{.8\funcindent}\begin{boxedminipage}{\funcwidth}

    \raggedright \textbf{fl\_set\_browser\_vscrollbar}(\textit{pObject}, \textit{on})

    \vspace{-1.5ex}

    \rule{\textwidth}{0.5\fboxrule}
\setlength{\parskip}{2ex}
\setlength{\parskip}{1ex}
      \textbf{Parameters}
      \vspace{-1ex}

      \begin{quote}
        \begin{Ventry}{xxxxxxx}

          \item[pObject]

          pointer to object ({\textless}pointer to 
          xfdata.FL\_OBJECT{\textgreater})

        \end{Ventry}

      \end{quote}

\textbf{Status:} Tested + NoDoc + Demo = OK



    \end{boxedminipage}

    \label{xformslib:library:fl_set_browser_hscrollbar}
    \index{xformslib \textit{(package)}!xformslib.library \textit{(module)}!xformslib.library.fl\_set\_browser\_hscrollbar \textit{(function)}}

    \vspace{0.5ex}

\hspace{.8\funcindent}\begin{boxedminipage}{\funcwidth}

    \raggedright \textbf{fl\_set\_browser\_hscrollbar}(\textit{pObject}, \textit{on})

    \vspace{-1.5ex}

    \rule{\textwidth}{0.5\fboxrule}
\setlength{\parskip}{2ex}
\setlength{\parskip}{1ex}
      \textbf{Parameters}
      \vspace{-1ex}

      \begin{quote}
        \begin{Ventry}{xxxxxxx}

          \item[pObject]

          pointer to object ({\textless}pointer to 
          xfdata.FL\_OBJECT{\textgreater})

        \end{Ventry}

      \end{quote}

\textbf{Status:} Untested + NoDoc + NoDemo = NOT OK



    \end{boxedminipage}

    \label{xformslib:library:fl_set_browser_line_selectable}
    \index{xformslib \textit{(package)}!xformslib.library \textit{(module)}!xformslib.library.fl\_set\_browser\_line\_selectable \textit{(function)}}

    \vspace{0.5ex}

\hspace{.8\funcindent}\begin{boxedminipage}{\funcwidth}

    \raggedright \textbf{fl\_set\_browser\_line\_selectable}(\textit{pObject}, \textit{line}, \textit{flag})

    \vspace{-1.5ex}

    \rule{\textwidth}{0.5\fboxrule}
\setlength{\parskip}{2ex}
\setlength{\parskip}{1ex}
      \textbf{Parameters}
      \vspace{-1ex}

      \begin{quote}
        \begin{Ventry}{xxxxxxx}

          \item[pObject]

          pointer to object ({\textless}pointer to 
          xfdata.FL\_OBJECT{\textgreater})

        \end{Ventry}

      \end{quote}

\textbf{Status:} Untested + NoDoc + NoDemo = NOT OK



    \end{boxedminipage}

    \label{xformslib:library:fl_get_browser_dimension}
    \index{xformslib \textit{(package)}!xformslib.library \textit{(module)}!xformslib.library.fl\_get\_browser\_dimension \textit{(function)}}

    \vspace{0.5ex}

\hspace{.8\funcindent}\begin{boxedminipage}{\funcwidth}

    \raggedright \textbf{fl\_get\_browser\_dimension}(\textit{pObject})

    \vspace{-1.5ex}

    \rule{\textwidth}{0.5\fboxrule}
\setlength{\parskip}{2ex}
    Returns all dimensions of a browser object.

\setlength{\parskip}{1ex}
      \textbf{Parameters}
      \vspace{-1ex}

      \begin{quote}
        \begin{Ventry}{xxxxxxx}

          \item[pObject]

          pointer to browser object ({\textless}pointer to 
          xfdata.FL\_OBJECT{\textgreater})

        \end{Ventry}

      \end{quote}

      \textbf{Return Value}
    \vspace{-1ex}

      \begin{quote}
      hor.xpos, ver.ypos, width, height

      \end{quote}

\textbf{Attention:} API change from XForms - upstream was fl\_get\_browser\_dimension(pObject, 
x, y, w, h)



\textbf{Status:} Untested + NoDoc + NoDemo = NOT OK



    \end{boxedminipage}

    \label{xformslib:library:fl_set_browser_dblclick_callback}
    \index{xformslib \textit{(package)}!xformslib.library \textit{(module)}!xformslib.library.fl\_set\_browser\_dblclick\_callback \textit{(function)}}

    \vspace{0.5ex}

\hspace{.8\funcindent}\begin{boxedminipage}{\funcwidth}

    \raggedright \textbf{fl\_set\_browser\_dblclick\_callback}(\textit{pObject}, \textit{py\_CallbackPtr}, \textit{data})

    \vspace{-1.5ex}

    \rule{\textwidth}{0.5\fboxrule}
\setlength{\parskip}{2ex}
\setlength{\parskip}{1ex}
      \textbf{Parameters}
      \vspace{-1ex}

      \begin{quote}
        \begin{Ventry}{xxxxxxx}

          \item[pObject]

          pointer to object ({\textless}pointer to 
          xfdata.FL\_OBJECT{\textgreater})

        \end{Ventry}

      \end{quote}

\textbf{Status:} Untested + NoDoc + NoDemo = NOT OK



    \end{boxedminipage}

    \label{xformslib:library:fl_get_browser_xoffset}
    \index{xformslib \textit{(package)}!xformslib.library \textit{(module)}!xformslib.library.fl\_get\_browser\_xoffset \textit{(function)}}

    \vspace{0.5ex}

\hspace{.8\funcindent}\begin{boxedminipage}{\funcwidth}

    \raggedright \textbf{fl\_get\_browser\_xoffset}(\textit{pObject})

    \vspace{-1.5ex}

    \rule{\textwidth}{0.5\fboxrule}
\setlength{\parskip}{2ex}
\setlength{\parskip}{1ex}
      \textbf{Parameters}
      \vspace{-1ex}

      \begin{quote}
        \begin{Ventry}{xxxxxxx}

          \item[pObject]

          pointer to browser object ({\textless}pointer to 
          xfdata.FL\_OBJECT{\textgreater})

        \end{Ventry}

      \end{quote}

      \textbf{Return Value}
    \vspace{-1ex}

      \begin{quote}
      coord num

      \end{quote}

\textbf{Status:} Untested + NoDoc + NoDemo = NOT OK



    \end{boxedminipage}

    \label{xformslib:library:fl_get_browser_rel_xoffset}
    \index{xformslib \textit{(package)}!xformslib.library \textit{(module)}!xformslib.library.fl\_get\_browser\_rel\_xoffset \textit{(function)}}

    \vspace{0.5ex}

\hspace{.8\funcindent}\begin{boxedminipage}{\funcwidth}

    \raggedright \textbf{fl\_get\_browser\_rel\_xoffset}(\textit{pObject})

    \vspace{-1.5ex}

    \rule{\textwidth}{0.5\fboxrule}
\setlength{\parskip}{2ex}
\setlength{\parskip}{1ex}
      \textbf{Parameters}
      \vspace{-1ex}

      \begin{quote}
        \begin{Ventry}{xxxxxxx}

          \item[pObject]

          pointer to browser object ({\textless}pointer to 
          xfdata.FL\_OBJECT{\textgreater})

        \end{Ventry}

      \end{quote}

      \textbf{Return Value}
    \vspace{-1ex}

      \begin{quote}
      num

      \end{quote}

\textbf{Status:} Untested + NoDoc + NoDemo = NOT OK



    \end{boxedminipage}

    \label{xformslib:library:fl_set_browser_xoffset}
    \index{xformslib \textit{(package)}!xformslib.library \textit{(module)}!xformslib.library.fl\_set\_browser\_xoffset \textit{(function)}}

    \vspace{0.5ex}

\hspace{.8\funcindent}\begin{boxedminipage}{\funcwidth}

    \raggedright \textbf{fl\_set\_browser\_xoffset}(\textit{pObject}, \textit{npixels})

    \vspace{-1.5ex}

    \rule{\textwidth}{0.5\fboxrule}
\setlength{\parskip}{2ex}
\setlength{\parskip}{1ex}
      \textbf{Parameters}
      \vspace{-1ex}

      \begin{quote}
        \begin{Ventry}{xxxxxxx}

          \item[pObject]

          pointer to browser object ({\textless}pointer to 
          xfdata.FL\_OBJECT{\textgreater})

        \end{Ventry}

      \end{quote}

\textbf{Status:} Untested + NoDoc + NoDemo = NOT OK



    \end{boxedminipage}

    \label{xformslib:library:fl_set_browser_rel_xoffset}
    \index{xformslib \textit{(package)}!xformslib.library \textit{(module)}!xformslib.library.fl\_set\_browser\_rel\_xoffset \textit{(function)}}

    \vspace{0.5ex}

\hspace{.8\funcindent}\begin{boxedminipage}{\funcwidth}

    \raggedright \textbf{fl\_set\_browser\_rel\_xoffset}(\textit{pObject}, \textit{val})

    \vspace{-1.5ex}

    \rule{\textwidth}{0.5\fboxrule}
\setlength{\parskip}{2ex}
\setlength{\parskip}{1ex}
      \textbf{Parameters}
      \vspace{-1ex}

      \begin{quote}
        \begin{Ventry}{xxxxxxx}

          \item[pObject]

          pointer to browser object ({\textless}pointer to 
          xfdata.FL\_OBJECT{\textgreater})

        \end{Ventry}

      \end{quote}

\textbf{Status:} Untested + NoDoc + NoDemo = NOT OK



    \end{boxedminipage}

    \label{xformslib:library:fl_get_browser_yoffset}
    \index{xformslib \textit{(package)}!xformslib.library \textit{(module)}!xformslib.library.fl\_get\_browser\_yoffset \textit{(function)}}

    \vspace{0.5ex}

\hspace{.8\funcindent}\begin{boxedminipage}{\funcwidth}

    \raggedright \textbf{fl\_get\_browser\_yoffset}(\textit{pObject})

    \vspace{-1.5ex}

    \rule{\textwidth}{0.5\fboxrule}
\setlength{\parskip}{2ex}
\setlength{\parskip}{1ex}
      \textbf{Parameters}
      \vspace{-1ex}

      \begin{quote}
        \begin{Ventry}{xxxxxxx}

          \item[pObject]

          pointer to browser object ({\textless}pointer to 
          xfdata.FL\_OBJECT{\textgreater})

        \end{Ventry}

      \end{quote}

      \textbf{Return Value}
    \vspace{-1ex}

      \begin{quote}
      coord num

      \end{quote}

\textbf{Status:} Tested + NoDoc + Demo = OK



    \end{boxedminipage}

    \label{xformslib:library:fl_get_browser_rel_yoffset}
    \index{xformslib \textit{(package)}!xformslib.library \textit{(module)}!xformslib.library.fl\_get\_browser\_rel\_yoffset \textit{(function)}}

    \vspace{0.5ex}

\hspace{.8\funcindent}\begin{boxedminipage}{\funcwidth}

    \raggedright \textbf{fl\_get\_browser\_rel\_yoffset}(\textit{pObject})

    \vspace{-1.5ex}

    \rule{\textwidth}{0.5\fboxrule}
\setlength{\parskip}{2ex}
\setlength{\parskip}{1ex}
      \textbf{Parameters}
      \vspace{-1ex}

      \begin{quote}
        \begin{Ventry}{xxxxxxx}

          \item[pObject]

          pointer to browser object ({\textless}pointer to 
          xfdata.FL\_OBJECT{\textgreater})

        \end{Ventry}

      \end{quote}

      \textbf{Return Value}
    \vspace{-1ex}

      \begin{quote}
      num

      \end{quote}

\textbf{Status:} Untested + NoDoc + NoDemo = NOT OK



    \end{boxedminipage}

    \label{xformslib:library:fl_set_browser_yoffset}
    \index{xformslib \textit{(package)}!xformslib.library \textit{(module)}!xformslib.library.fl\_set\_browser\_yoffset \textit{(function)}}

    \vspace{0.5ex}

\hspace{.8\funcindent}\begin{boxedminipage}{\funcwidth}

    \raggedright \textbf{fl\_set\_browser\_yoffset}(\textit{pObject}, \textit{npixels})

    \vspace{-1.5ex}

    \rule{\textwidth}{0.5\fboxrule}
\setlength{\parskip}{2ex}
\setlength{\parskip}{1ex}
      \textbf{Parameters}
      \vspace{-1ex}

      \begin{quote}
        \begin{Ventry}{xxxxxxx}

          \item[pObject]

          pointer to browser object ({\textless}pointer to 
          xfdata.FL\_OBJECT{\textgreater})

        \end{Ventry}

      \end{quote}

\textbf{Status:} Tested + NoDoc + Demo = OK



    \end{boxedminipage}

    \label{xformslib:library:fl_set_browser_rel_yoffset}
    \index{xformslib \textit{(package)}!xformslib.library \textit{(module)}!xformslib.library.fl\_set\_browser\_rel\_yoffset \textit{(function)}}

    \vspace{0.5ex}

\hspace{.8\funcindent}\begin{boxedminipage}{\funcwidth}

    \raggedright \textbf{fl\_set\_browser\_rel\_yoffset}(\textit{pObject}, \textit{val})

    \vspace{-1.5ex}

    \rule{\textwidth}{0.5\fboxrule}
\setlength{\parskip}{2ex}
\setlength{\parskip}{1ex}
      \textbf{Parameters}
      \vspace{-1ex}

      \begin{quote}
        \begin{Ventry}{xxxxxxx}

          \item[pObject]

          pointer to browser object ({\textless}pointer to 
          xfdata.FL\_OBJECT{\textgreater})

        \end{Ventry}

      \end{quote}

\textbf{Status:} Untested + NoDoc + NoDemo = NOT OK



    \end{boxedminipage}

    \label{xformslib:library:fl_set_browser_scrollbarsize}
    \index{xformslib \textit{(package)}!xformslib.library \textit{(module)}!xformslib.library.fl\_set\_browser\_scrollbarsize \textit{(function)}}

    \vspace{0.5ex}

\hspace{.8\funcindent}\begin{boxedminipage}{\funcwidth}

    \raggedright \textbf{fl\_set\_browser\_scrollbarsize}(\textit{pObject}, \textit{hh}, \textit{vw})

    \vspace{-1.5ex}

    \rule{\textwidth}{0.5\fboxrule}
\setlength{\parskip}{2ex}
\setlength{\parskip}{1ex}
      \textbf{Parameters}
      \vspace{-1ex}

      \begin{quote}
        \begin{Ventry}{xxxxxxx}

          \item[pObject]

          pointer to browser object ({\textless}pointer to 
          xfdata.FL\_OBJECT{\textgreater})

        \end{Ventry}

      \end{quote}

\textbf{Status:} Untested + NoDoc + NoDemo = NOT OK



    \end{boxedminipage}

    \label{xformslib:library:fl_show_browser_line}
    \index{xformslib \textit{(package)}!xformslib.library \textit{(module)}!xformslib.library.fl\_show\_browser\_line \textit{(function)}}

    \vspace{0.5ex}

\hspace{.8\funcindent}\begin{boxedminipage}{\funcwidth}

    \raggedright \textbf{fl\_show\_browser\_line}(\textit{pObject}, \textit{line})

    \vspace{-1.5ex}

    \rule{\textwidth}{0.5\fboxrule}
\setlength{\parskip}{2ex}
    Bring a browser line into view.

\setlength{\parskip}{1ex}
      \textbf{Parameters}
      \vspace{-1ex}

      \begin{quote}
        \begin{Ventry}{xxxxxxx}

          \item[pObject]

          pointer to browser object ({\textless}pointer to 
          xfdata.FL\_OBJECT{\textgreater})

          \item[line]

          line to show

        \end{Ventry}

      \end{quote}

\textbf{Status:} Untested + NoDoc + NoDemo = NOT OK



    \end{boxedminipage}

    \label{xformslib:library:fl_set_browser_hscroll_callback}
    \index{xformslib \textit{(package)}!xformslib.library \textit{(module)}!xformslib.library.fl\_set\_browser\_hscroll\_callback \textit{(function)}}

    \vspace{0.5ex}

\hspace{.8\funcindent}\begin{boxedminipage}{\funcwidth}

    \raggedright \textbf{fl\_set\_browser\_hscroll\_callback}(\textit{pObject}, \textit{py\_BrowserScrollCallback}, \textit{vdata})

    \vspace{-1.5ex}

    \rule{\textwidth}{0.5\fboxrule}
\setlength{\parskip}{2ex}
\setlength{\parskip}{1ex}
      \textbf{Parameters}
      \vspace{-1ex}

      \begin{quote}
        \begin{Ventry}{xxxxxxxxxxxxxxxxxxxxxxxx}

          \item[pObject]

          pointer to browser object ({\textless}pointer to 
          xfdata.FL\_OBJECT{\textgreater})

          \item[py\_BrowserScrollCallback]

          python function callback

            {\it (type=fn(pObject, num, data))}

          \item[vdata]

          user data argument

        \end{Ventry}

      \end{quote}

\textbf{Status:} Untested + NoDoc + NoDemo = NOT OK



    \end{boxedminipage}

    \label{xformslib:library:fl_set_browser_vscroll_callback}
    \index{xformslib \textit{(package)}!xformslib.library \textit{(module)}!xformslib.library.fl\_set\_browser\_vscroll\_callback \textit{(function)}}

    \vspace{0.5ex}

\hspace{.8\funcindent}\begin{boxedminipage}{\funcwidth}

    \raggedright \textbf{fl\_set\_browser\_vscroll\_callback}(\textit{pObject}, \textit{py\_BrowserScrollCallback}, \textit{vdata})

    \vspace{-1.5ex}

    \rule{\textwidth}{0.5\fboxrule}
\setlength{\parskip}{2ex}
\setlength{\parskip}{1ex}
      \textbf{Parameters}
      \vspace{-1ex}

      \begin{quote}
        \begin{Ventry}{xxxxxxxxxxxxxxxxxxxxxxxx}

          \item[pObject]

          pointer to browser object ({\textless}pointer to 
          xfdata.FL\_OBJECT{\textgreater})

          \item[py\_BrowserScrollCallback]

          python function callback

            {\it (type=fn(pObject, num, data))}

          \item[vdata]

          user data argument

        \end{Ventry}

      \end{quote}

\textbf{Status:} Tested + NoDoc + Demo = OK



    \end{boxedminipage}

    \label{xformslib:library:fl_get_browser_line_yoffset}
    \index{xformslib \textit{(package)}!xformslib.library \textit{(module)}!xformslib.library.fl\_get\_browser\_line\_yoffset \textit{(function)}}

    \vspace{0.5ex}

\hspace{.8\funcindent}\begin{boxedminipage}{\funcwidth}

    \raggedright \textbf{fl\_get\_browser\_line\_yoffset}(\textit{pObject}, \textit{line})

    \vspace{-1.5ex}

    \rule{\textwidth}{0.5\fboxrule}
\setlength{\parskip}{2ex}
\setlength{\parskip}{1ex}
      \textbf{Parameters}
      \vspace{-1ex}

      \begin{quote}
        \begin{Ventry}{xxxxxxx}

          \item[pObject]

          pointer to browser object ({\textless}pointer to 
          xfdata.FL\_OBJECT{\textgreater})

        \end{Ventry}

      \end{quote}

      \textbf{Return Value}
    \vspace{-1ex}

      \begin{quote}
      num

      \end{quote}

\textbf{Status:} Untested + NoDoc + NoDemo = NOT OK



    \end{boxedminipage}

    \label{xformslib:library:fl_get_browser_hscroll_callback}
    \index{xformslib \textit{(package)}!xformslib.library \textit{(module)}!xformslib.library.fl\_get\_browser\_hscroll\_callback \textit{(function)}}

    \vspace{0.5ex}

\hspace{.8\funcindent}\begin{boxedminipage}{\funcwidth}

    \raggedright \textbf{fl\_get\_browser\_hscroll\_callback}(\textit{pObject})

    \vspace{-1.5ex}

    \rule{\textwidth}{0.5\fboxrule}
\setlength{\parskip}{2ex}
\setlength{\parskip}{1ex}
      \textbf{Parameters}
      \vspace{-1ex}

      \begin{quote}
        \begin{Ventry}{xxxxxxx}

          \item[pObject]

          pointer to browser object ({\textless}pointer to 
          xfdata.FL\_OBJECT{\textgreater})

        \end{Ventry}

      \end{quote}

      \textbf{Return Value}
    \vspace{-1ex}

      \begin{quote}
      callback

      \end{quote}

\textbf{Status:} Untested + NoDoc + NoDemo = NOT OK



    \end{boxedminipage}

    \label{xformslib:library:fl_get_browser_vscroll_callback}
    \index{xformslib \textit{(package)}!xformslib.library \textit{(module)}!xformslib.library.fl\_get\_browser\_vscroll\_callback \textit{(function)}}

    \vspace{0.5ex}

\hspace{.8\funcindent}\begin{boxedminipage}{\funcwidth}

    \raggedright \textbf{fl\_get\_browser\_vscroll\_callback}(\textit{pObject})

    \vspace{-1.5ex}

    \rule{\textwidth}{0.5\fboxrule}
\setlength{\parskip}{2ex}
\setlength{\parskip}{1ex}
      \textbf{Parameters}
      \vspace{-1ex}

      \begin{quote}
        \begin{Ventry}{xxxxxxx}

          \item[pObject]

          pointer to browser object ({\textless}pointer to 
          xfdata.FL\_OBJECT{\textgreater})

        \end{Ventry}

      \end{quote}

      \textbf{Return Value}
    \vspace{-1ex}

      \begin{quote}
      callback

      \end{quote}

\textbf{Status:} Untested + NoDoc + NoDemo = NOT OK



    \end{boxedminipage}

    \label{xformslib:library:fl_add_roundbutton}
    \index{xformslib \textit{(package)}!xformslib.library \textit{(module)}!xformslib.library.fl\_add\_roundbutton \textit{(function)}}

    \vspace{0.5ex}

\hspace{.8\funcindent}\begin{boxedminipage}{\funcwidth}

    \raggedright \textbf{fl\_add\_roundbutton}(\textit{buttontype}, \textit{x}, \textit{y}, \textit{w}, \textit{h}, \textit{label})

    \vspace{-1.5ex}

    \rule{\textwidth}{0.5\fboxrule}
\setlength{\parskip}{2ex}
    Adds a roundbutton object.

\setlength{\parskip}{1ex}
      \textbf{Parameters}
      \vspace{-1ex}

      \begin{quote}
        \begin{Ventry}{xxxxxxxxxx}

          \item[buttontype]

          type of button object to be added

            {\it (type=[num./int] from xfdata module FL\_NORMAL\_BUTTON, FL\_PUSH\_BUTTON, 
FL\_RADIO\_BUTTON, FL\_HIDDEN\_BUTTON, FL\_TOUCH\_BUTTON, 
FL\_INOUT\_BUTTON, FL\_RETURN\_BUTTON, FL\_HIDDEN\_RET\_BUTTON, 
FL\_MENU\_BUTTON, FL\_TOGGLE\_BUTTON)}

          \item[x]

          horizontal position (upper-left corner)

          \item[x]

          vertical position (upper-left corner)

          \item[w]

          width in coord units

          \item[h]

          height in coord units

          \item[label]

          text label of button

        \end{Ventry}

      \end{quote}

      \textbf{Return Value}
    \vspace{-1ex}

      \begin{quote}
      pObject

      \end{quote}

\textbf{Status:} Tested + NoDoc + Demo = OK



    \end{boxedminipage}

    \label{xformslib:library:fl_add_round3dbutton}
    \index{xformslib \textit{(package)}!xformslib.library \textit{(module)}!xformslib.library.fl\_add\_round3dbutton \textit{(function)}}

    \vspace{0.5ex}

\hspace{.8\funcindent}\begin{boxedminipage}{\funcwidth}

    \raggedright \textbf{fl\_add\_round3dbutton}(\textit{buttontype}, \textit{x}, \textit{y}, \textit{w}, \textit{h}, \textit{label})

    \vspace{-1.5ex}

    \rule{\textwidth}{0.5\fboxrule}
\setlength{\parskip}{2ex}
    Adds a 3D roundbutton object.

\setlength{\parskip}{1ex}
      \textbf{Parameters}
      \vspace{-1ex}

      \begin{quote}
        \begin{Ventry}{xxxxxxxxxx}

          \item[buttontype]

          type of button object to be added

            {\it (type=[num./int] from xfdata module FL\_NORMAL\_BUTTON, FL\_PUSH\_BUTTON, 
FL\_RADIO\_BUTTON, FL\_HIDDEN\_BUTTON, FL\_TOUCH\_BUTTON, 
FL\_INOUT\_BUTTON, FL\_RETURN\_BUTTON, FL\_HIDDEN\_RET\_BUTTON, 
FL\_MENU\_BUTTON, FL\_TOGGLE\_BUTTON)}

          \item[x]

          horizontal position (upper-left corner)

          \item[x]

          vertical position (upper-left corner)

          \item[w]

          width in coord units

          \item[h]

          height in coord units

          \item[label]

          text label of button

        \end{Ventry}

      \end{quote}

      \textbf{Return Value}
    \vspace{-1ex}

      \begin{quote}
      pObject

      \end{quote}

\textbf{Status:} Tested + NoDoc + Demo = OK



    \end{boxedminipage}

    \label{xformslib:library:fl_add_lightbutton}
    \index{xformslib \textit{(package)}!xformslib.library \textit{(module)}!xformslib.library.fl\_add\_lightbutton \textit{(function)}}

    \vspace{0.5ex}

\hspace{.8\funcindent}\begin{boxedminipage}{\funcwidth}

    \raggedright \textbf{fl\_add\_lightbutton}(\textit{buttontype}, \textit{x}, \textit{y}, \textit{w}, \textit{h}, \textit{label})

    \vspace{-1.5ex}

    \rule{\textwidth}{0.5\fboxrule}
\setlength{\parskip}{2ex}
    Adds a lightbutton object (with an on/off light switch).

\setlength{\parskip}{1ex}
      \textbf{Parameters}
      \vspace{-1ex}

      \begin{quote}
        \begin{Ventry}{xxxxxxxxxx}

          \item[buttontype]

          type of button to be added

            {\it (type=[num./int] xfc.FL\_NORMAL\_BUTTON, xfc.FL\_PUSH\_BUTTON, 
xfc.FL\_RADIO\_BUTTON, xfc.FL\_HIDDEN\_BUTTON, xfc.FL\_TOUCH\_BUTTON, 
xfc.FL\_INOUT\_BUTTON, xfc.FL\_RETURN\_BUTTON, xfc.FL\_HIDDEN\_RET\_BUTTON,
xfc.FL\_MENU\_BUTTON, xfc.FL\_TOGGLE\_BUTTON)}

          \item[x]

          horizontal position (upper-left corner)

          \item[x]

          vertical position (upper-left corner)

          \item[w]

          width in coord units

          \item[h]

          height in coord units

          \item[label]

          text label of button

        \end{Ventry}

      \end{quote}

      \textbf{Return Value}
    \vspace{-1ex}

      \begin{quote}
      pObject

      \end{quote}

\textbf{Status:} Tested + NoDoc + Demo = OK



    \end{boxedminipage}

    \label{xformslib:library:fl_add_checkbutton}
    \index{xformslib \textit{(package)}!xformslib.library \textit{(module)}!xformslib.library.fl\_add\_checkbutton \textit{(function)}}

    \vspace{0.5ex}

\hspace{.8\funcindent}\begin{boxedminipage}{\funcwidth}

    \raggedright \textbf{fl\_add\_checkbutton}(\textit{buttontype}, \textit{x}, \textit{y}, \textit{w}, \textit{h}, \textit{label})

    \vspace{-1.5ex}

    \rule{\textwidth}{0.5\fboxrule}
\setlength{\parskip}{2ex}
    Adds a checkbutton object.

\setlength{\parskip}{1ex}
      \textbf{Parameters}
      \vspace{-1ex}

      \begin{quote}
        \begin{Ventry}{xxxxxxxxxx}

          \item[buttontype]

          type of button object to be added

            {\it (type=[num./int] xfc.FL\_NORMAL\_BUTTON, xfc.FL\_PUSH\_BUTTON, 
xfc.FL\_RADIO\_BUTTON, xfc.FL\_HIDDEN\_BUTTON, xfc.FL\_TOUCH\_BUTTON, 
xfc.FL\_INOUT\_BUTTON, xfc.FL\_RETURN\_BUTTON, xfc.FL\_HIDDEN\_RET\_BUTTON,
xfc.FL\_MENU\_BUTTON, xfc.FL\_TOGGLE\_BUTTON)}

          \item[x]

          horizontal position (upper-left corner)

          \item[x]

          vertical position (upper-left corner)

          \item[w]

          width in coord units

          \item[h]

          height in coord units

          \item[label]

          text label of button

        \end{Ventry}

      \end{quote}

      \textbf{Return Value}
    \vspace{-1ex}

      \begin{quote}
      pObject

      \end{quote}

\textbf{Status:} Tested + NoDoc + Demo = OK



    \end{boxedminipage}

    \label{xformslib:library:fl_add_button}
    \index{xformslib \textit{(package)}!xformslib.library \textit{(module)}!xformslib.library.fl\_add\_button \textit{(function)}}

    \vspace{0.5ex}

\hspace{.8\funcindent}\begin{boxedminipage}{\funcwidth}

    \raggedright \textbf{fl\_add\_button}(\textit{buttontype}, \textit{x}, \textit{y}, \textit{w}, \textit{h}, \textit{label})

    \vspace{-1.5ex}

    \rule{\textwidth}{0.5\fboxrule}
\setlength{\parskip}{2ex}
    Adds a button object to the current form.

\setlength{\parskip}{1ex}
      \textbf{Parameters}
      \vspace{-1ex}

      \begin{quote}
        \begin{Ventry}{xxxxxxxxxx}

          \item[buttontype]

          type of button to be added

            {\it (type=[num./int] xfc.FL\_NORMAL\_BUTTON, xfc.FL\_PUSH\_BUTTON, 
xfc.FL\_RADIO\_BUTTON, xfc.FL\_HIDDEN\_BUTTON, xfc.FL\_TOUCH\_BUTTON, 
xfc.FL\_INOUT\_BUTTON, xfc.FL\_RETURN\_BUTTON, xfc.FL\_HIDDEN\_RET\_BUTTON,
xfc.FL\_MENU\_BUTTON, xfc.FL\_TOGGLE\_BUTTON)}

          \item[x]

          horizontal position (upper-left corner)

          \item[x]

          vertical position (upper-left corner)

          \item[w]

          width in coord units

          \item[h]

          height in coord units

          \item[label]

          text label of button

        \end{Ventry}

      \end{quote}

      \textbf{Return Value}
    \vspace{-1ex}

      \begin{quote}
      pObject

      \end{quote}

\textbf{Status:} Tested + NoDoc + Demo = OK



    \end{boxedminipage}

    \label{xformslib:library:fl_add_bitmapbutton}
    \index{xformslib \textit{(package)}!xformslib.library \textit{(module)}!xformslib.library.fl\_add\_bitmapbutton \textit{(function)}}

    \vspace{0.5ex}

\hspace{.8\funcindent}\begin{boxedminipage}{\funcwidth}

    \raggedright \textbf{fl\_add\_bitmapbutton}(\textit{buttontype}, \textit{x}, \textit{y}, \textit{w}, \textit{h}, \textit{label})

    \vspace{-1.5ex}

    \rule{\textwidth}{0.5\fboxrule}
\setlength{\parskip}{2ex}
    Adds a bitmapbutton object.

\setlength{\parskip}{1ex}
      \textbf{Parameters}
      \vspace{-1ex}

      \begin{quote}
        \begin{Ventry}{xxxxxxxxxx}

          \item[buttontype]

          type of button to be added

            {\it (type=[num./int] xfc.FL\_NORMAL\_BUTTON, xfc.FL\_PUSH\_BUTTON, 
xfc.FL\_RADIO\_BUTTON, xfc.FL\_HIDDEN\_BUTTON, xfc.FL\_TOUCH\_BUTTON, 
xfc.FL\_INOUT\_BUTTON, xfc.FL\_RETURN\_BUTTON, xfc.FL\_HIDDEN\_RET\_BUTTON,
xfc.FL\_MENU\_BUTTON, xfc.FL\_TOGGLE\_BUTTON)}

          \item[x]

          horizontal position (upper-left corner)

          \item[x]

          vertical position (upper-left corner)

          \item[w]

          width in coord units

          \item[h]

          height in coord units

          \item[label]

          text label of button

        \end{Ventry}

      \end{quote}

      \textbf{Return Value}
    \vspace{-1ex}

      \begin{quote}
      pObject

      \end{quote}

\textbf{Status:} Tested + NoDoc + Demo = OK



    \end{boxedminipage}

    \label{xformslib:library:fl_add_scrollbutton}
    \index{xformslib \textit{(package)}!xformslib.library \textit{(module)}!xformslib.library.fl\_add\_scrollbutton \textit{(function)}}

    \vspace{0.5ex}

\hspace{.8\funcindent}\begin{boxedminipage}{\funcwidth}

    \raggedright \textbf{fl\_add\_scrollbutton}(\textit{buttontype}, \textit{x}, \textit{y}, \textit{w}, \textit{h}, \textit{label})

    \vspace{-1.5ex}

    \rule{\textwidth}{0.5\fboxrule}
\setlength{\parskip}{2ex}
    Adds a scrollbutton object.

\setlength{\parskip}{1ex}
      \textbf{Parameters}
      \vspace{-1ex}

      \begin{quote}
        \begin{Ventry}{xxxxxxxxxx}

          \item[buttontype]

          type of button to be added

            {\it (type=[num./int] xfc.FL\_NORMAL\_BUTTON, xfc.FL\_PUSH\_BUTTON, 
xfc.FL\_RADIO\_BUTTON, xfc.FL\_HIDDEN\_BUTTON, xfc.FL\_TOUCH\_BUTTON, 
xfc.FL\_INOUT\_BUTTON, xfc.FL\_RETURN\_BUTTON, xfc.FL\_HIDDEN\_RET\_BUTTON,
xfc.FL\_MENU\_BUTTON, xfc.FL\_TOGGLE\_BUTTON)}

          \item[x]

          horizontal position (upper-left corner)

          \item[x]

          vertical position (upper-left corner)

          \item[w]

          width in coord units

          \item[h]

          height in coord units

          \item[label]

          text label of button

        \end{Ventry}

      \end{quote}

      \textbf{Return Value}
    \vspace{-1ex}

      \begin{quote}
      pObject

      \end{quote}

\textbf{Status:} Untested + NoDoc + NoDemo = NOT OK



    \end{boxedminipage}

    \label{xformslib:library:fl_add_labelbutton}
    \index{xformslib \textit{(package)}!xformslib.library \textit{(module)}!xformslib.library.fl\_add\_labelbutton \textit{(function)}}

    \vspace{0.5ex}

\hspace{.8\funcindent}\begin{boxedminipage}{\funcwidth}

    \raggedright \textbf{fl\_add\_labelbutton}(\textit{buttontype}, \textit{x}, \textit{y}, \textit{w}, \textit{h}, \textit{label})

    \vspace{-1.5ex}

    \rule{\textwidth}{0.5\fboxrule}
\setlength{\parskip}{2ex}
    Adds a labelbutton object.

\setlength{\parskip}{1ex}
      \textbf{Parameters}
      \vspace{-1ex}

      \begin{quote}
        \begin{Ventry}{xxxxxxxxxx}

          \item[buttontype]

          type of button to be added

            {\it (type=[num./int] xfc.FL\_NORMAL\_BUTTON, xfc.FL\_PUSH\_BUTTON, 
xfc.FL\_RADIO\_BUTTON, xfc.FL\_HIDDEN\_BUTTON, xfc.FL\_TOUCH\_BUTTON, 
xfc.FL\_INOUT\_BUTTON, xfc.FL\_RETURN\_BUTTON, xfc.FL\_HIDDEN\_RET\_BUTTON,
xfc.FL\_MENU\_BUTTON, xfc.FL\_TOGGLE\_BUTTON)}

          \item[x]

          horizontal position (upper-left corner)

          \item[x]

          vertical position (upper-left corner)

          \item[w]

          width in coord units

          \item[h]

          height in coord units

          \item[label]

          text label of button

        \end{Ventry}

      \end{quote}

      \textbf{Return Value}
    \vspace{-1ex}

      \begin{quote}
      pObject

      \end{quote}

\textbf{Status:} Untested + NoDoc + NoDemo = NOT OK



    \end{boxedminipage}

    \label{xformslib:library:fl_set_bitmapbutton_data}
    \index{xformslib \textit{(package)}!xformslib.library \textit{(module)}!xformslib.library.fl\_set\_bitmapbutton\_data \textit{(function)}}

    \vspace{0.5ex}

\hspace{.8\funcindent}\begin{boxedminipage}{\funcwidth}

    \raggedright \textbf{fl\_set\_bitmapbutton\_data}(\textit{pObject}, \textit{w}, \textit{h}, \textit{bits})

    \vspace{-1.5ex}

    \rule{\textwidth}{0.5\fboxrule}
\setlength{\parskip}{2ex}
\setlength{\parskip}{1ex}
      \textbf{Parameters}
      \vspace{-1ex}

      \begin{quote}
        \begin{Ventry}{xxxxxxx}

          \item[pObject]

          pointer to object ({\textless}pointer to 
          xfdata.FL\_OBJECT{\textgreater})

        \end{Ventry}

      \end{quote}

\textbf{Status:} Untested + NoDoc + NoDemo = NOT OK



    \end{boxedminipage}

    \label{xformslib:library:fl_add_pixmapbutton}
    \index{xformslib \textit{(package)}!xformslib.library \textit{(module)}!xformslib.library.fl\_add\_pixmapbutton \textit{(function)}}

    \vspace{0.5ex}

\hspace{.8\funcindent}\begin{boxedminipage}{\funcwidth}

    \raggedright \textbf{fl\_add\_pixmapbutton}(\textit{buttontype}, \textit{x}, \textit{y}, \textit{w}, \textit{h}, \textit{label})

    \vspace{-1.5ex}

    \rule{\textwidth}{0.5\fboxrule}
\setlength{\parskip}{2ex}
    Adds a pixmapbutton object.

\setlength{\parskip}{1ex}
      \textbf{Parameters}
      \vspace{-1ex}

      \begin{quote}
        \begin{Ventry}{xxxxxxxxxx}

          \item[buttontype]

          type of button to be added

            {\it (type=[num./int] xfc.FL\_NORMAL\_BUTTON, xfc.FL\_PUSH\_BUTTON, 
xfc.FL\_RADIO\_BUTTON, xfc.FL\_HIDDEN\_BUTTON, xfc.FL\_TOUCH\_BUTTON, 
xfc.FL\_INOUT\_BUTTON, xfc.FL\_RETURN\_BUTTON, xfc.FL\_HIDDEN\_RET\_BUTTON,
xfc.FL\_MENU\_BUTTON, xfc.FL\_TOGGLE\_BUTTON)}

          \item[x]

          horizontal position (upper-left corner)

          \item[x]

          vertical position (upper-left corner)

          \item[w]

          width in coord units

          \item[h]

          height in coord units

          \item[label]

          text label of button

        \end{Ventry}

      \end{quote}

      \textbf{Return Value}
    \vspace{-1ex}

      \begin{quote}
      pObject

      \end{quote}

\textbf{Status:} Tested + NoDoc + Demo = OK



    \end{boxedminipage}

    \label{xformslib:library:fl_set_pixmapbutton_focus_outline}
    \index{xformslib \textit{(package)}!xformslib.library \textit{(module)}!xformslib.library.fl\_set\_pixmapbutton\_focus\_outline \textit{(function)}}

    \vspace{0.5ex}

\hspace{.8\funcindent}\begin{boxedminipage}{\funcwidth}

    \raggedright \textbf{fl\_set\_pixmapbutton\_focus\_outline}(\textit{pObject}, \textit{yes})

    \vspace{-1.5ex}

    \rule{\textwidth}{0.5\fboxrule}
\setlength{\parskip}{2ex}
\setlength{\parskip}{1ex}
      \textbf{Parameters}
      \vspace{-1ex}

      \begin{quote}
        \begin{Ventry}{xxxxxxx}

          \item[pObject]

          pointer to object ({\textless}pointer to 
          xfdata.FL\_OBJECT{\textgreater})

        \end{Ventry}

      \end{quote}

\textbf{Status:} Untested + NoDoc + NoDemo = NOT OK



    \end{boxedminipage}

    \label{xformslib:library:fl_set_pixmap_data}
    \index{xformslib \textit{(package)}!xformslib.library \textit{(module)}!xformslib.library.fl\_set\_pixmap\_data \textit{(function)}}

    \vspace{0.5ex}

\hspace{.8\funcindent}\begin{boxedminipage}{\funcwidth}

    \raggedright \textbf{fl\_set\_pixmapbutton\_data}(\textit{pObject}, \textit{bits})

    \vspace{-1.5ex}

    \rule{\textwidth}{0.5\fboxrule}
\setlength{\parskip}{2ex}
\setlength{\parskip}{1ex}
      \textbf{Parameters}
      \vspace{-1ex}

      \begin{quote}
        \begin{Ventry}{xxxxxxx}

          \item[pObject]

          pointer to object ({\textless}pointer to 
          xfdata.FL\_OBJECT{\textgreater})

          \item[bits]

          bits contents of pixmap

        \end{Ventry}

      \end{quote}

\textbf{Status:} Untested + NoDoc + NoDemo = NOT OK



    \end{boxedminipage}

    \label{xformslib:library:fl_set_pixmapbutton_focus_outline}
    \index{xformslib \textit{(package)}!xformslib.library \textit{(module)}!xformslib.library.fl\_set\_pixmapbutton\_focus\_outline \textit{(function)}}

    \vspace{0.5ex}

\hspace{.8\funcindent}\begin{boxedminipage}{\funcwidth}

    \raggedright \textbf{fl\_set\_pixmapbutton\_show\_focus}(\textit{pObject}, \textit{yes})

    \vspace{-1.5ex}

    \rule{\textwidth}{0.5\fboxrule}
\setlength{\parskip}{2ex}
\setlength{\parskip}{1ex}
      \textbf{Parameters}
      \vspace{-1ex}

      \begin{quote}
        \begin{Ventry}{xxxxxxx}

          \item[pObject]

          pointer to object ({\textless}pointer to 
          xfdata.FL\_OBJECT{\textgreater})

        \end{Ventry}

      \end{quote}

\textbf{Status:} Untested + NoDoc + NoDemo = NOT OK



    \end{boxedminipage}

    \label{xformslib:library:fl_set_pixmapbutton_focus_data}
    \index{xformslib \textit{(package)}!xformslib.library \textit{(module)}!xformslib.library.fl\_set\_pixmapbutton\_focus\_data \textit{(function)}}

    \vspace{0.5ex}

\hspace{.8\funcindent}\begin{boxedminipage}{\funcwidth}

    \raggedright \textbf{fl\_set\_pixmapbutton\_focus\_data}(\textit{pObject}, \textit{bits})

    \vspace{-1.5ex}

    \rule{\textwidth}{0.5\fboxrule}
\setlength{\parskip}{2ex}
\setlength{\parskip}{1ex}
      \textbf{Parameters}
      \vspace{-1ex}

      \begin{quote}
        \begin{Ventry}{xxxxxxx}

          \item[pObject]

          pointer to object ({\textless}pointer to 
          xfdata.FL\_OBJECT{\textgreater})

        \end{Ventry}

      \end{quote}

\textbf{Status:} Untested + NoDoc + NoDemo = NOT OK



    \end{boxedminipage}

    \label{xformslib:library:fl_set_pixmapbutton_focus_file}
    \index{xformslib \textit{(package)}!xformslib.library \textit{(module)}!xformslib.library.fl\_set\_pixmapbutton\_focus\_file \textit{(function)}}

    \vspace{0.5ex}

\hspace{.8\funcindent}\begin{boxedminipage}{\funcwidth}

    \raggedright \textbf{fl\_set\_pixmapbutton\_focus\_file}(\textit{pObject}, \textit{fname})

    \vspace{-1.5ex}

    \rule{\textwidth}{0.5\fboxrule}
\setlength{\parskip}{2ex}
\setlength{\parskip}{1ex}
      \textbf{Parameters}
      \vspace{-1ex}

      \begin{quote}
        \begin{Ventry}{xxxxxxx}

          \item[pObject]

          pointer to object ({\textless}pointer to 
          xfdata.FL\_OBJECT{\textgreater})

        \end{Ventry}

      \end{quote}

\textbf{Status:} Untested + NoDoc + NoDemo = NOT OK



    \end{boxedminipage}

    \label{xformslib:library:fl_set_pixmapbutton_focus_pixmap}
    \index{xformslib \textit{(package)}!xformslib.library \textit{(module)}!xformslib.library.fl\_set\_pixmapbutton\_focus\_pixmap \textit{(function)}}

    \vspace{0.5ex}

\hspace{.8\funcindent}\begin{boxedminipage}{\funcwidth}

    \raggedright \textbf{fl\_set\_pixmapbutton\_focus\_pixmap}(\textit{pObject}, \textit{idnum}, \textit{mask})

    \vspace{-1.5ex}

    \rule{\textwidth}{0.5\fboxrule}
\setlength{\parskip}{2ex}
\setlength{\parskip}{1ex}
      \textbf{Parameters}
      \vspace{-1ex}

      \begin{quote}
        \begin{Ventry}{xxxxxxx}

          \item[pObject]

          pointer to object ({\textless}pointer to 
          xfdata.FL\_OBJECT{\textgreater})

        \end{Ventry}

      \end{quote}

\textbf{Status:} Untested + NoDoc + NoDemo = NOT OK



    \end{boxedminipage}

    \label{xformslib:library:fl_get_button}
    \index{xformslib \textit{(package)}!xformslib.library \textit{(module)}!xformslib.library.fl\_get\_button \textit{(function)}}

    \vspace{0.5ex}

\hspace{.8\funcindent}\begin{boxedminipage}{\funcwidth}

    \raggedright \textbf{fl\_get\_button}(\textit{pObject})

    \vspace{-1.5ex}

    \rule{\textwidth}{0.5\fboxrule}
\setlength{\parskip}{2ex}
    Returns the value of the button.

\setlength{\parskip}{1ex}
      \textbf{Parameters}
      \vspace{-1ex}

      \begin{quote}
        \begin{Ventry}{xxxxxxx}

          \item[pObject]

          pointer to button object ({\textless}pointer to 
          xfdata.FL\_OBJECT{\textgreater})

        \end{Ventry}

      \end{quote}

      \textbf{Return Value}
    \vspace{-1ex}

      \begin{quote}
      num

      \end{quote}

\textbf{Status:} Tested + NoDoc + Demo = OK



    \end{boxedminipage}

    \label{xformslib:library:fl_set_button}
    \index{xformslib \textit{(package)}!xformslib.library \textit{(module)}!xformslib.library.fl\_set\_button \textit{(function)}}

    \vspace{0.5ex}

\hspace{.8\funcindent}\begin{boxedminipage}{\funcwidth}

    \raggedright \textbf{fl\_set\_button}(\textit{pObject}, \textit{pushed})

    \vspace{-1.5ex}

    \rule{\textwidth}{0.5\fboxrule}
\setlength{\parskip}{2ex}
    Sets the button state (not pushed/pushed).

\setlength{\parskip}{1ex}
      \textbf{Parameters}
      \vspace{-1ex}

      \begin{quote}
        \begin{Ventry}{xxxxxxx}

          \item[pObject]

          pointer to button object ({\textless}pointer to 
          xfdata.FL\_OBJECT{\textgreater})

          \item[pushed]

          state of button to be set (0{\textbar}1)

        \end{Ventry}

      \end{quote}

\textbf{Status:} Tested + NoDoc + Demo = OK



    \end{boxedminipage}

    \label{xformslib:library:fl_get_button_numb}
    \index{xformslib \textit{(package)}!xformslib.library \textit{(module)}!xformslib.library.fl\_get\_button\_numb \textit{(function)}}

    \vspace{0.5ex}

\hspace{.8\funcindent}\begin{boxedminipage}{\funcwidth}

    \raggedright \textbf{fl\_get\_button\_numb}(\textit{pObject})

    \vspace{-1.5ex}

    \rule{\textwidth}{0.5\fboxrule}
\setlength{\parskip}{2ex}
    Returns the number of the last used mouse button. fl\_mouse\_button 
    function will also return the mouse number.

\setlength{\parskip}{1ex}
      \textbf{Parameters}
      \vspace{-1ex}

      \begin{quote}
        \begin{Ventry}{xxxxxxx}

          \item[pObject]

          pointer to button object ({\textless}pointer to 
          xfdata.FL\_OBJECT{\textgreater})

        \end{Ventry}

      \end{quote}

      \textbf{Return Value}
    \vspace{-1ex}

      \begin{quote}
      num

      \end{quote}

\textbf{Status:} Untested + NoDoc + NoDemo = NOT OK



    \end{boxedminipage}

    \label{xformslib:library:fl_set_object_shortcut}
    \index{xformslib \textit{(package)}!xformslib.library \textit{(module)}!xformslib.library.fl\_set\_object\_shortcut \textit{(function)}}

    \vspace{0.5ex}

\hspace{.8\funcindent}\begin{boxedminipage}{\funcwidth}

    \raggedright \textbf{fl\_set\_button\_shortcut}(\textit{pObject}, \textit{shctxt}, \textit{showit})

    \vspace{-1.5ex}

    \rule{\textwidth}{0.5\fboxrule}
\setlength{\parskip}{2ex}
    Sets a shortcut, binding a key or a series of keys to an object. It 
    resets any previous defined shortcuts for the object. Using e.g. 
    "acE\#d{\textasciicircum}h" the keys 'a', 'c', 'E', 
    {\textless}Alt{\textgreater}d and {\textless}Ctrl{\textgreater}h are 
    associated with the object. The precise format is as follows: any 
    character in the string is considered as a shortcut, except 
    '{\textasciicircum}' and '\#', which stand for combinations with the 
    {\textless}Ctrl{\textgreater} and {\textless}Alt{\textgreater} keys. 
    (the case of the key following '\#' or '{\textasciicircum}' is not 
    important, i.e. no distiction is made between e.g. 
    "{\textasciicircum}C" and "{\textasciicircum}c", both encode the key 
    combination {\textless}Ctrl{\textgreater}C as well as 
    {\textless}Ctrl{\textgreater}C.) The key '{\textasciicircum}' itself 
    can be set as a shortcut key by using 
    "{\textasciicircum}{\textasciicircum}" in the string defining the 
    shortcut. The key '\#' can be obtained as a shortcut by using the 
    string "{\textasciicircum}\#". So, e.g. "\#{\textasciicircum}\#" 
    encodes {\textless}ALT{\textgreater}\#. The 
    {\textless}Esc{\textgreater} key can be given as "{\textasciicircum}[".
    Another special character not mentioned yet is '\&', which indicates 
    function and arrow keys. Use a sequence starting with '\&' and directly
    followed by a number between 1 and 35 to represent one of the function 
    keys. For example, "\&2" stands for the {\textless}F2{\textgreater} 
    function key. The four cursors keys (up, down, right, and left) can be 
    given as "\&A", "\&B", "\&C" and "\&D", respectively. The key '\&' 
    itself can be obtained as a shortcut by prefixing it with 
    '{\textasciicircum}'.

\setlength{\parskip}{1ex}
      \textbf{Parameters}
      \vspace{-1ex}

      \begin{quote}
        \begin{Ventry}{xxxxxxx}

          \item[pObject]

          object ({\textless}pointer to xfdata.FL\_OBJECT{\textgreater})

          \item[shctxt]

          shortcut text to be set ({\textless}string{\textgreater})

          \item[showit]

          flag if shortcut letter has to be underlined or not if a match 
          exists (only the 1st alphanumeric character is used.

            {\it (type=0 (underline not shown) or 1 (shown))}

        \end{Ventry}

      \end{quote}

\textbf{Example:} fl\_set\_object\_shortcut(pobj6, "aA\#A{\textasciicircum}A", 1)



\textbf{Status:} Tested + Doc + NoDemo = OK



    \end{boxedminipage}

    \label{xformslib:library:fl_create_generic_button}
    \index{xformslib \textit{(package)}!xformslib.library \textit{(module)}!xformslib.library.fl\_create\_generic\_button \textit{(function)}}

    \vspace{0.5ex}

\hspace{.8\funcindent}\begin{boxedminipage}{\funcwidth}

    \raggedright \textbf{fl\_create\_generic\_button}(\textit{btnclass}, \textit{buttontype}, \textit{x}, \textit{y}, \textit{w}, \textit{h}, \textit{label})

    \vspace{-1.5ex}

    \rule{\textwidth}{0.5\fboxrule}
\setlength{\parskip}{2ex}
    Creates a generic button object.

\setlength{\parskip}{1ex}
      \textbf{Parameters}
      \vspace{-1ex}

      \begin{quote}
        \begin{Ventry}{xxxxxxxxxx}

          \item[btnclass]

          value of a new button class

          \item[buttontype]

          type of button to be created

            {\it (type=[num./int] xfc.FL\_NORMAL\_BUTTON, xfc.FL\_PUSH\_BUTTON, 
xfc.FL\_RADIO\_BUTTON, xfc.FL\_HIDDEN\_BUTTON, xfc.FL\_TOUCH\_BUTTON, 
xfc.FL\_INOUT\_BUTTON, xfc.FL\_RETURN\_BUTTON, xfc.FL\_HIDDEN\_RET\_BUTTON,
xfc.FL\_MENU\_BUTTON, xfc.FL\_TOGGLE\_BUTTON)}

          \item[x]

          horizontal position (upper-left corner)

          \item[x]

          vertical position (upper-left corner)

          \item[w]

          width in coord units

          \item[h]

          height in coord units

          \item[label]

          text label of button

        \end{Ventry}

      \end{quote}

      \textbf{Return Value}
    \vspace{-1ex}

      \begin{quote}
      pObject

      \end{quote}

\textbf{Status:} Untested + NoDoc + NoDemo = NOT OK



    \end{boxedminipage}

    \label{xformslib:library:fl_add_button_class}
    \index{xformslib \textit{(package)}!xformslib.library \textit{(module)}!xformslib.library.fl\_add\_button\_class \textit{(function)}}

    \vspace{0.5ex}

\hspace{.8\funcindent}\begin{boxedminipage}{\funcwidth}

    \raggedright \textbf{fl\_add\_button\_class}(\textit{btnclass}, \textit{py\_DrawButton}, \textit{py\_CleanupButton})

    \vspace{-1.5ex}

    \rule{\textwidth}{0.5\fboxrule}
\setlength{\parskip}{2ex}
    Associates a button class with a drawing function.

\setlength{\parskip}{1ex}
      \textbf{Parameters}
      \vspace{-1ex}

      \begin{quote}
        \begin{Ventry}{xxxxxxxxxxxxxxxx}

          \item[btnclass]

          value of a new button class

          \item[py\_DrawButton]

          python function to draw button

            {\it (type=fn(pObject))}

          \item[py\_CleanupButton]

          python function to cleanup button

            {\it (type=fn(pButtonSpec))}

        \end{Ventry}

      \end{quote}

\textbf{Status:} Untested + NoDoc + NoDemo = NOT OK



    \end{boxedminipage}

    \label{xformslib:library:fl_set_button_mouse_buttons}
    \index{xformslib \textit{(package)}!xformslib.library \textit{(module)}!xformslib.library.fl\_set\_button\_mouse\_buttons \textit{(function)}}

    \vspace{0.5ex}

\hspace{.8\funcindent}\begin{boxedminipage}{\funcwidth}

    \raggedright \textbf{fl\_set\_button\_mouse\_buttons}(\textit{pObject}, \textit{buttons})

    \vspace{-1.5ex}

    \rule{\textwidth}{0.5\fboxrule}
\setlength{\parskip}{2ex}
    Function allows to set up to which mouse buttons the button object will
    react.

\setlength{\parskip}{1ex}
      \textbf{Parameters}
      \vspace{-1ex}

      \begin{quote}
        \begin{Ventry}{xxxxxxx}

          \item[pObject]

          pointer to button object ({\textless}pointer to 
          xfdata.FL\_OBJECT{\textgreater})

          \item[buttons]

          value of mouse buttons to be set

        \end{Ventry}

      \end{quote}

\textbf{Status:} Untested + NoDoc + NoDemo = NOT OK



    \end{boxedminipage}

    \label{xformslib:library:fl_get_button_mouse_buttons}
    \index{xformslib \textit{(package)}!xformslib.library \textit{(module)}!xformslib.library.fl\_get\_button\_mouse\_buttons \textit{(function)}}

    \vspace{0.5ex}

\hspace{.8\funcindent}\begin{boxedminipage}{\funcwidth}

    \raggedright \textbf{fl\_get\_button\_mouse\_buttons}(\textit{pObject})

    \vspace{-1.5ex}

    \rule{\textwidth}{0.5\fboxrule}
\setlength{\parskip}{2ex}
    Returns a value indicating which mouse buttons the button object will 
    react to.

\setlength{\parskip}{1ex}
      \textbf{Parameters}
      \vspace{-1ex}

      \begin{quote}
        \begin{Ventry}{xxxxxxx}

          \item[pObject]

          pointer to button object ({\textless}pointer to 
          xfdata.FL\_OBJECT{\textgreater})

        \end{Ventry}

      \end{quote}

      \textbf{Return Value}
    \vspace{-1ex}

      \begin{quote}
      buttons value

      \end{quote}

\textbf{Attention:} API change from XForms - upstream was 
fl\_get\_button\_mouse\_buttons(pObject, buttons)



\textbf{Status:} Untested + NoDoc + NoDemo = NOT OK



    \end{boxedminipage}

    \label{xformslib:library:fl_create_generic_canvas}
    \index{xformslib \textit{(package)}!xformslib.library \textit{(module)}!xformslib.library.fl\_create\_generic\_canvas \textit{(function)}}

    \vspace{0.5ex}

\hspace{.8\funcindent}\begin{boxedminipage}{\funcwidth}

    \raggedright \textbf{fl\_create\_generic\_canvas}(\textit{canvasclass}, \textit{canvastype}, \textit{x}, \textit{y}, \textit{w}, \textit{h}, \textit{label})

    \vspace{-1.5ex}

    \rule{\textwidth}{0.5\fboxrule}
\setlength{\parskip}{2ex}
    Creates a generic canvas object.

\setlength{\parskip}{1ex}
      \textbf{Parameters}
      \vspace{-1ex}

      \begin{quote}
        \begin{Ventry}{xxxxxxxxxxx}

          \item[canvasclass]

          value of a new canvas class

          \item[canvastype]

          type of canvas to be created

            {\it (type=[num./int] xfc.FL\_NORMAL\_CANVAS, xfc.FL\_SCROLLED\_CANVAS)}

          \item[x]

          horizontal position (upper-left corner)

          \item[x]

          vertical position (upper-left corner)

          \item[w]

          width in coord units

          \item[h]

          height in coord units

          \item[label]

          text label of canvas

        \end{Ventry}

      \end{quote}

      \textbf{Return Value}
    \vspace{-1ex}

      \begin{quote}
      pObject

      \end{quote}

\textbf{Status:} Untested + NoDoc + NoDemo = NOT OK



    \end{boxedminipage}

    \label{xformslib:library:fl_add_canvas}
    \index{xformslib \textit{(package)}!xformslib.library \textit{(module)}!xformslib.library.fl\_add\_canvas \textit{(function)}}

    \vspace{0.5ex}

\hspace{.8\funcindent}\begin{boxedminipage}{\funcwidth}

    \raggedright \textbf{fl\_add\_canvas}(\textit{canvastype}, \textit{x}, \textit{y}, \textit{w}, \textit{h}, \textit{label})

    \vspace{-1.5ex}

    \rule{\textwidth}{0.5\fboxrule}
\setlength{\parskip}{2ex}
    Adds a canvas object.

\setlength{\parskip}{1ex}
      \textbf{Parameters}
      \vspace{-1ex}

      \begin{quote}
        \begin{Ventry}{xxxxxxxxxx}

          \item[canvastype]

          type of canvas to be added

            {\it (type=[num./int] xfc.FL\_NORMAL\_CANVAS, xfc.FL\_SCROLLED\_CANVAS)}

          \item[x]

          horizontal position (upper-left corner)

          \item[x]

          vertical position (upper-left corner)

          \item[w]

          width in coord units

          \item[h]

          height in coord units

          \item[label]

          text label of canvas

        \end{Ventry}

      \end{quote}

      \textbf{Return Value}
    \vspace{-1ex}

      \begin{quote}
      pObject

      \end{quote}

\textbf{Status:} Untested + NoDoc + NoDemo = NOT OK



    \end{boxedminipage}

    \label{xformslib:library:fl_set_canvas_colormap}
    \index{xformslib \textit{(package)}!xformslib.library \textit{(module)}!xformslib.library.fl\_set\_canvas\_colormap \textit{(function)}}

    \vspace{0.5ex}

\hspace{.8\funcindent}\begin{boxedminipage}{\funcwidth}

    \raggedright \textbf{fl\_set\_canvas\_colormap}(\textit{pObject}, \textit{colormap})

    \vspace{-1.5ex}

    \rule{\textwidth}{0.5\fboxrule}
\setlength{\parskip}{2ex}
\setlength{\parskip}{1ex}
      \textbf{Parameters}
      \vspace{-1ex}

      \begin{quote}
        \begin{Ventry}{xxxxxxx}

          \item[pObject]

          pointer to object ({\textless}pointer to 
          xfdata.FL\_OBJECT{\textgreater})

        \end{Ventry}

      \end{quote}

\textbf{Status:} Untested + NoDoc + NoDemo = NOT OK



    \end{boxedminipage}

    \label{xformslib:library:fl_set_canvas_visual}
    \index{xformslib \textit{(package)}!xformslib.library \textit{(module)}!xformslib.library.fl\_set\_canvas\_visual \textit{(function)}}

    \vspace{0.5ex}

\hspace{.8\funcindent}\begin{boxedminipage}{\funcwidth}

    \raggedright \textbf{fl\_set\_canvas\_visual}(\textit{pObject}, \textit{pVisual})

    \vspace{-1.5ex}

    \rule{\textwidth}{0.5\fboxrule}
\setlength{\parskip}{2ex}
\setlength{\parskip}{1ex}
      \textbf{Parameters}
      \vspace{-1ex}

      \begin{quote}
        \begin{Ventry}{xxxxxxx}

          \item[pObject]

          pointer to canvas object ({\textless}pointer to 
          xfdata.FL\_OBJECT{\textgreater})

          \item[pVisual]

          pointer to Visual class instance

        \end{Ventry}

      \end{quote}

\textbf{Status:} Untested + NoDoc + NoDemo = NOT OK



    \end{boxedminipage}

    \label{xformslib:library:fl_set_canvas_depth}
    \index{xformslib \textit{(package)}!xformslib.library \textit{(module)}!xformslib.library.fl\_set\_canvas\_depth \textit{(function)}}

    \vspace{0.5ex}

\hspace{.8\funcindent}\begin{boxedminipage}{\funcwidth}

    \raggedright \textbf{fl\_set\_canvas\_depth}(\textit{pObject}, \textit{depth})

    \vspace{-1.5ex}

    \rule{\textwidth}{0.5\fboxrule}
\setlength{\parskip}{2ex}
\setlength{\parskip}{1ex}
      \textbf{Parameters}
      \vspace{-1ex}

      \begin{quote}
        \begin{Ventry}{xxxxxxx}

          \item[pObject]

          pointer to canvas object ({\textless}pointer to 
          xfdata.FL\_OBJECT{\textgreater})

          \item[depth]

          depth value of canvas

        \end{Ventry}

      \end{quote}

\textbf{Status:} Untested + NoDoc + NoDemo = NOT OK



    \end{boxedminipage}

    \label{xformslib:library:fl_set_canvas_attributes}
    \index{xformslib \textit{(package)}!xformslib.library \textit{(module)}!xformslib.library.fl\_set\_canvas\_attributes \textit{(function)}}

    \vspace{0.5ex}

\hspace{.8\funcindent}\begin{boxedminipage}{\funcwidth}

    \raggedright \textbf{fl\_set\_canvas\_attributes}(\textit{pObject}, \textit{mask}, \textit{pXSetWindowAttributes})

    \vspace{-1.5ex}

    \rule{\textwidth}{0.5\fboxrule}
\setlength{\parskip}{2ex}
\setlength{\parskip}{1ex}
      \textbf{Parameters}
      \vspace{-1ex}

      \begin{quote}
        \begin{Ventry}{xxxxxxxxxxxxxxxxxxxxx}

          \item[pObject]

          pointer to canvas object ({\textless}pointer to 
          xfdata.FL\_OBJECT{\textgreater})

          \item[mask]

          mask num.

          \item[pXSetWindowAttributes]

          pointer to XSetWindowAttributes

        \end{Ventry}

      \end{quote}

\textbf{Status:} Untested + NoDoc + NoDemo = NOT OK



    \end{boxedminipage}

    \label{xformslib:library:fl_add_canvas_handler}
    \index{xformslib \textit{(package)}!xformslib.library \textit{(module)}!xformslib.library.fl\_add\_canvas\_handler \textit{(function)}}

    \vspace{0.5ex}

\hspace{.8\funcindent}\begin{boxedminipage}{\funcwidth}

    \raggedright \textbf{fl\_add\_canvas\_handler}(\textit{pObject}, \textit{ev}, \textit{py\_HandleCanvas}, \textit{udata})

    \vspace{-1.5ex}

    \rule{\textwidth}{0.5\fboxrule}
\setlength{\parskip}{2ex}
\setlength{\parskip}{1ex}
      \textbf{Parameters}
      \vspace{-1ex}

      \begin{quote}
        \begin{Ventry}{xxxxxxxxxxxxxxx}

          \item[pObject]

          pointer to canvas object ({\textless}pointer to 
          xfdata.FL\_OBJECT{\textgreater})

          \item[ev]

          event number

          \item[py\_HandleCanvas]

          python function to handle canvas

            {\it (type=fn(pObject, win, num, num, pXEvent, ptr\_void) -{\textgreater} num)}

        \end{Ventry}

      \end{quote}

      \textbf{Return Value}
    \vspace{-1ex}

      \begin{quote}
      canvas handler

      \end{quote}

\textbf{Status:} Untested + NoDoc + NoDemo = NOT OK



    \end{boxedminipage}

    \label{xformslib:library:fl_get_canvas_id}
    \index{xformslib \textit{(package)}!xformslib.library \textit{(module)}!xformslib.library.fl\_get\_canvas\_id \textit{(function)}}

    \vspace{0.5ex}

\hspace{.8\funcindent}\begin{boxedminipage}{\funcwidth}

    \raggedright \textbf{fl\_get\_canvas\_id}(\textit{pObject})

    \vspace{-1.5ex}

    \rule{\textwidth}{0.5\fboxrule}
\setlength{\parskip}{2ex}
    Returns the window ID of the canvas window.

\setlength{\parskip}{1ex}
      \textbf{Parameters}
      \vspace{-1ex}

      \begin{quote}
        \begin{Ventry}{xxxxxxx}

          \item[pObject]

          pointer to canvas object ({\textless}pointer to 
          xfdata.FL\_OBJECT{\textgreater})

        \end{Ventry}

      \end{quote}

      \textbf{Return Value}
    \vspace{-1ex}

      \begin{quote}
      window

      \end{quote}

\textbf{Status:} Untested + NoDoc + NoDemo = NOT OK



    \end{boxedminipage}

    \label{xformslib:library:fl_get_canvas_colormap}
    \index{xformslib \textit{(package)}!xformslib.library \textit{(module)}!xformslib.library.fl\_get\_canvas\_colormap \textit{(function)}}

    \vspace{0.5ex}

\hspace{.8\funcindent}\begin{boxedminipage}{\funcwidth}

    \raggedright \textbf{fl\_get\_canvas\_colormap}(\textit{pObject})

    \vspace{-1.5ex}

    \rule{\textwidth}{0.5\fboxrule}
\setlength{\parskip}{2ex}
    Returns the colormap of a canas object

\setlength{\parskip}{1ex}
      \textbf{Parameters}
      \vspace{-1ex}

      \begin{quote}
        \begin{Ventry}{xxxxxxx}

          \item[pObject]

          pointer to canvas object ({\textless}pointer to 
          xfdata.FL\_OBJECT{\textgreater})

        \end{Ventry}

      \end{quote}

      \textbf{Return Value}
    \vspace{-1ex}

      \begin{quote}
      colormap

      \end{quote}

\textbf{Status:} Untested + NoDoc + NoDemo = NOT OK



    \end{boxedminipage}

    \label{xformslib:library:fl_get_canvas_depth}
    \index{xformslib \textit{(package)}!xformslib.library \textit{(module)}!xformslib.library.fl\_get\_canvas\_depth \textit{(function)}}

    \vspace{0.5ex}

\hspace{.8\funcindent}\begin{boxedminipage}{\funcwidth}

    \raggedright \textbf{fl\_get\_canvas\_depth}(\textit{pObject})

    \vspace{-1.5ex}

    \rule{\textwidth}{0.5\fboxrule}
\setlength{\parskip}{2ex}
    Returns the depth of a canvas object.

\setlength{\parskip}{1ex}
      \textbf{Parameters}
      \vspace{-1ex}

      \begin{quote}
        \begin{Ventry}{xxxxxxx}

          \item[pObject]

          pointer to canvas object ({\textless}pointer to 
          xfdata.FL\_OBJECT{\textgreater})

        \end{Ventry}

      \end{quote}

      \textbf{Return Value}
    \vspace{-1ex}

      \begin{quote}
      depth num

      \end{quote}

\textbf{Status:} Untested + NoDoc + NoDemo = NOT OK



    \end{boxedminipage}

    \label{xformslib:library:fl_remove_canvas_handler}
    \index{xformslib \textit{(package)}!xformslib.library \textit{(module)}!xformslib.library.fl\_remove\_canvas\_handler \textit{(function)}}

    \vspace{0.5ex}

\hspace{.8\funcindent}\begin{boxedminipage}{\funcwidth}

    \raggedright \textbf{fl\_remove\_canvas\_handler}(\textit{pObject}, \textit{ev}, \textit{py\_HandleCanvas})

    \vspace{-1.5ex}

    \rule{\textwidth}{0.5\fboxrule}
\setlength{\parskip}{2ex}
    Remove a particular handler for event ev. If ev is invalid, removes all
    handlers and their corresponding event mask.

\setlength{\parskip}{1ex}
      \textbf{Parameters}
      \vspace{-1ex}

      \begin{quote}
        \begin{Ventry}{xxxxxxxxxxxxxxx}

          \item[pObject]

          pointer to canvas object ({\textless}pointer to 
          xfdata.FL\_OBJECT{\textgreater})

          \item[ev]

          event number

          \item[py\_HandleCanvas]

          python function to handle canvas

            {\it (type=fn(pObject, win, num, num, pXEvent, ptr\_void) -{\textgreater} num)}

        \end{Ventry}

      \end{quote}

\textbf{Status:} Untested + NoDoc + NoDemo = NOT OK



    \end{boxedminipage}

    \label{xformslib:library:fl_hide_canvas}
    \index{xformslib \textit{(package)}!xformslib.library \textit{(module)}!xformslib.library.fl\_hide\_canvas \textit{(function)}}

    \vspace{0.5ex}

\hspace{.8\funcindent}\begin{boxedminipage}{\funcwidth}

    \raggedright \textbf{fl\_hide\_canvas}(\textit{pObject})

    \vspace{-1.5ex}

    \rule{\textwidth}{0.5\fboxrule}
\setlength{\parskip}{2ex}
    Hides a canvas object.

\setlength{\parskip}{1ex}
      \textbf{Parameters}
      \vspace{-1ex}

      \begin{quote}
        \begin{Ventry}{xxxxxxx}

          \item[pObject]

          pointer to canvas object ({\textless}pointer to 
          xfdata.FL\_OBJECT{\textgreater})

        \end{Ventry}

      \end{quote}

\textbf{Status:} Untested + NoDoc + NoDemo = NOT OK



    \end{boxedminipage}

    \label{xformslib:library:fl_share_canvas_colormap}
    \index{xformslib \textit{(package)}!xformslib.library \textit{(module)}!xformslib.library.fl\_share\_canvas\_colormap \textit{(function)}}

    \vspace{0.5ex}

\hspace{.8\funcindent}\begin{boxedminipage}{\funcwidth}

    \raggedright \textbf{fl\_share\_canvas\_colormap}(\textit{pObject}, \textit{colormap})

    \vspace{-1.5ex}

    \rule{\textwidth}{0.5\fboxrule}
\setlength{\parskip}{2ex}
\setlength{\parskip}{1ex}
      \textbf{Parameters}
      \vspace{-1ex}

      \begin{quote}
        \begin{Ventry}{xxxxxxx}

          \item[pObject]

          pointer to object ({\textless}pointer to 
          xfdata.FL\_OBJECT{\textgreater})

        \end{Ventry}

      \end{quote}

\textbf{Status:} Untested + NoDoc + NoDemo = NOT OK



    \end{boxedminipage}

    \label{xformslib:library:fl_clear_canvas}
    \index{xformslib \textit{(package)}!xformslib.library \textit{(module)}!xformslib.library.fl\_clear\_canvas \textit{(function)}}

    \vspace{0.5ex}

\hspace{.8\funcindent}\begin{boxedminipage}{\funcwidth}

    \raggedright \textbf{fl\_clear\_canvas}(\textit{pObject})

    \vspace{-1.5ex}

    \rule{\textwidth}{0.5\fboxrule}
\setlength{\parskip}{2ex}
    Clears the canvas to the background color. If no background is defined 
    uses black.

\setlength{\parskip}{1ex}
      \textbf{Parameters}
      \vspace{-1ex}

      \begin{quote}
        \begin{Ventry}{xxxxxxx}

          \item[pObject]

          pointer to canvas object ({\textless}pointer to 
          xfdata.FL\_OBJECT{\textgreater})

        \end{Ventry}

      \end{quote}

\textbf{Status:} Untested + NoDoc + NoDemo = NOT OK



    \end{boxedminipage}

    \label{xformslib:library:fl_modify_canvas_prop}
    \index{xformslib \textit{(package)}!xformslib.library \textit{(module)}!xformslib.library.fl\_modify\_canvas\_prop \textit{(function)}}

    \vspace{0.5ex}

\hspace{.8\funcindent}\begin{boxedminipage}{\funcwidth}

    \raggedright \textbf{fl\_modify\_canvas\_prop}(\textit{pObject}, \textit{py\_initModifyCanvasProp}, \textit{py\_activateModifyCanvasProp}, \textit{py\_cleanupModifyCanvasProp})

    \vspace{-1.5ex}

    \rule{\textwidth}{0.5\fboxrule}
\setlength{\parskip}{2ex}
\setlength{\parskip}{1ex}
      \textbf{Parameters}
      \vspace{-1ex}

      \begin{quote}
        \begin{Ventry}{xxxxxxxxxxxxxxxxxxxxxxxxxxx}

          \item[pObject]

          pointer to canvas object ({\textless}pointer to 
          xfdata.FL\_OBJECT{\textgreater})

          \item[py\_initModifyCanvasProp]

          python function callback, returning value

          \item[py\_initModifyCanvasProp]

          fn(pObject) -{\textgreater} num.

          \item[py\_activateModifyCanvasProp]

          python function callback, returning value

          \item[py\_activateModifyCanvasProp]

          fn(pObject) -{\textgreater} num.

          \item[py\_cleanupModifyCanvasProp]

          python function callback, returning value

          \item[py\_cleanupModifyCanvasProp]

          fn(pObject) -{\textgreater} num.

        \end{Ventry}

      \end{quote}

\textbf{Status:} Untested + NoDoc + NoDemo = NOT OK



    \end{boxedminipage}

    \label{xformslib:library:fl_canvas_yield_to_shortcut}
    \index{xformslib \textit{(package)}!xformslib.library \textit{(module)}!xformslib.library.fl\_canvas\_yield\_to\_shortcut \textit{(function)}}

    \vspace{0.5ex}

\hspace{.8\funcindent}\begin{boxedminipage}{\funcwidth}

    \raggedright \textbf{fl\_canvas\_yield\_to\_shortcut}(\textit{pObject}, \textit{yes})

    \vspace{-1.5ex}

    \rule{\textwidth}{0.5\fboxrule}
\setlength{\parskip}{2ex}
\setlength{\parskip}{1ex}
      \textbf{Parameters}
      \vspace{-1ex}

      \begin{quote}
        \begin{Ventry}{xxxxxxx}

          \item[pObject]

          pointer to object ({\textless}pointer to 
          xfdata.FL\_OBJECT{\textgreater})

        \end{Ventry}

      \end{quote}

\textbf{Status:} Untested + NoDoc + NoDemo = NOT OK



    \end{boxedminipage}

    \label{xformslib:library:fl_add_glcanvas}
    \index{xformslib \textit{(package)}!xformslib.library \textit{(module)}!xformslib.library.fl\_add\_glcanvas \textit{(function)}}

    \vspace{0.5ex}

\hspace{.8\funcindent}\begin{boxedminipage}{\funcwidth}

    \raggedright \textbf{fl\_add\_glcanvas}(\textit{canvastype}, \textit{x}, \textit{y}, \textit{w}, \textit{h}, \textit{label})

    \vspace{-1.5ex}

    \rule{\textwidth}{0.5\fboxrule}
\setlength{\parskip}{2ex}
    Adds a glcanvas object to the form.

\setlength{\parskip}{1ex}
      \textbf{Parameters}
      \vspace{-1ex}

      \begin{quote}
        \begin{Ventry}{xxxxxxxxxx}

          \item[canvastype]

          type of glcanvas to be added

            {\it (type=[num./int] from xfdata FL\_NORMAL\_CANVAS, FL\_SCROLLED\_CANVAS (not 
enabled))}

          \item[x]

          horizontal position (upper-left corner)

          \item[x]

          vertical position (upper-left corner)

          \item[w]

          width in coord units

          \item[h]

          height in coord units

          \item[label]

          text label of glcanvas

        \end{Ventry}

      \end{quote}

      \textbf{Return Value}
    \vspace{-1ex}

      \begin{quote}
      pObject

      \end{quote}

\textbf{Status:} Untested + NoDoc + NoDemo = NOT OK



    \end{boxedminipage}

    \label{xformslib:library:fl_set_glcanvas_defaults}
    \index{xformslib \textit{(package)}!xformslib.library \textit{(module)}!xformslib.library.fl\_set\_glcanvas\_defaults \textit{(function)}}

    \vspace{0.5ex}

\hspace{.8\funcindent}\begin{boxedminipage}{\funcwidth}

    \raggedright \textbf{fl\_set\_glcanvas\_defaults}(\textit{config})

    \vspace{-1.5ex}

    \rule{\textwidth}{0.5\fboxrule}
\setlength{\parskip}{2ex}
    Modifies the global defaults for glcanvas.

\setlength{\parskip}{1ex}
      \textbf{Parameters}
      \vspace{-1ex}

      \begin{quote}
        \begin{Ventry}{xxxxxx}

          \item[config]

          configuration settings

        \end{Ventry}

      \end{quote}

\textbf{Status:} Untested + NoDoc + NoDemo = NOT OK



    \end{boxedminipage}

    \label{xformslib:library:fl_get_glcanvas_defaults}
    \index{xformslib \textit{(package)}!xformslib.library \textit{(module)}!xformslib.library.fl\_get\_glcanvas\_defaults \textit{(function)}}

    \vspace{0.5ex}

\hspace{.8\funcindent}\begin{boxedminipage}{\funcwidth}

    \raggedright \textbf{fl\_get\_glcanvas\_defaults}()

    \vspace{-1.5ex}

    \rule{\textwidth}{0.5\fboxrule}
\setlength{\parskip}{2ex}
    Returns the global defaults for glcanvas.

\setlength{\parskip}{1ex}
      \textbf{Return Value}
    \vspace{-1ex}

      \begin{quote}
      configuration settings

      \end{quote}

\textbf{Attention:} API change from XForms - upstream was fl\_get\_glcanvas\_defaults(config)



\textbf{Status:} Untested + NoDoc + NoDemo = NOT OK



    \end{boxedminipage}

    \label{xformslib:library:fl_set_glcanvas_attributes}
    \index{xformslib \textit{(package)}!xformslib.library \textit{(module)}!xformslib.library.fl\_set\_glcanvas\_attributes \textit{(function)}}

    \vspace{0.5ex}

\hspace{.8\funcindent}\begin{boxedminipage}{\funcwidth}

    \raggedright \textbf{fl\_set\_glcanvas\_attributes}(\textit{pObject}, \textit{config})

    \vspace{-1.5ex}

    \rule{\textwidth}{0.5\fboxrule}
\setlength{\parskip}{2ex}
    Modifies the default configuration of a particular glcanvas object.

\setlength{\parskip}{1ex}
      \textbf{Parameters}
      \vspace{-1ex}

      \begin{quote}
        \begin{Ventry}{xxxxxxx}

          \item[pObject]

          pointer to glcanvas object ({\textless}pointer to 
          xfdata.FL\_OBJECT{\textgreater})

          \item[config]

          configuration settings to be set

        \end{Ventry}

      \end{quote}

\textbf{Status:} Untested + NoDoc + NoDemo = NOT OK



    \end{boxedminipage}

    \label{xformslib:library:fl_get_glcanvas_attributes}
    \index{xformslib \textit{(package)}!xformslib.library \textit{(module)}!xformslib.library.fl\_get\_glcanvas\_attributes \textit{(function)}}

    \vspace{0.5ex}

\hspace{.8\funcindent}\begin{boxedminipage}{\funcwidth}

    \raggedright \textbf{fl\_get\_glcanvas\_attributes}(\textit{pObject})

    \vspace{-1.5ex}

    \rule{\textwidth}{0.5\fboxrule}
\setlength{\parskip}{2ex}
    Returns the attributes of a glcanvas object.

\setlength{\parskip}{1ex}
      \textbf{Parameters}
      \vspace{-1ex}

      \begin{quote}
        \begin{Ventry}{xxxxxxx}

          \item[pObject]

          glcanvas object ({\textless}pointer to 
          xfdata.FL\_OBJECT{\textgreater})

        \end{Ventry}

      \end{quote}

      \textbf{Return Value}
    \vspace{-1ex}

      \begin{quote}
      attributes

      \end{quote}

\textbf{Attention:} API change from XForms - upstream was 
fl\_get\_glcanvas\_attributes(pObject, attributes)



    \end{boxedminipage}

    \label{xformslib:library:fl_set_glcanvas_direct}
    \index{xformslib \textit{(package)}!xformslib.library \textit{(module)}!xformslib.library.fl\_set\_glcanvas\_direct \textit{(function)}}

    \vspace{0.5ex}

\hspace{.8\funcindent}\begin{boxedminipage}{\funcwidth}

    \raggedright \textbf{fl\_set\_glcanvas\_direct}(\textit{pObject}, \textit{direct})

    \vspace{-1.5ex}

    \rule{\textwidth}{0.5\fboxrule}
\setlength{\parskip}{2ex}
\setlength{\parskip}{1ex}
      \textbf{Parameters}
      \vspace{-1ex}

      \begin{quote}
        \begin{Ventry}{xxxxxxx}

          \item[pObject]

          pointer to glcanvas object ({\textless}pointer to 
          xfdata.FL\_OBJECT{\textgreater})

        \end{Ventry}

      \end{quote}

\textbf{Status:} Untested + NoDoc + NoDemo = NOT OK



    \end{boxedminipage}

    \label{xformslib:library:fl_activate_glcanvas}
    \index{xformslib \textit{(package)}!xformslib.library \textit{(module)}!xformslib.library.fl\_activate\_glcanvas \textit{(function)}}

    \vspace{0.5ex}

\hspace{.8\funcindent}\begin{boxedminipage}{\funcwidth}

    \raggedright \textbf{fl\_activate\_glcanvas}(\textit{pObject})

    \vspace{-1.5ex}

    \rule{\textwidth}{0.5\fboxrule}
\setlength{\parskip}{2ex}
    Activates a glcanvas object, allowing user interaction.

\setlength{\parskip}{1ex}
      \textbf{Parameters}
      \vspace{-1ex}

      \begin{quote}
        \begin{Ventry}{xxxxxxx}

          \item[pObject]

          pointer to glcanvas object ({\textless}pointer to 
          xfdata.FL\_OBJECT{\textgreater})

        \end{Ventry}

      \end{quote}

\textbf{Status:} Untested + NoDoc + NoDemo = NOT OK



    \end{boxedminipage}

    \label{xformslib:library:fl_get_glcanvas_xvisualinfo}
    \index{xformslib \textit{(package)}!xformslib.library \textit{(module)}!xformslib.library.fl\_get\_glcanvas\_xvisualinfo \textit{(function)}}

    \vspace{0.5ex}

\hspace{.8\funcindent}\begin{boxedminipage}{\funcwidth}

    \raggedright \textbf{fl\_get\_glcanvas\_xvisualinfo}(\textit{pObject})

    \vspace{-1.5ex}

    \rule{\textwidth}{0.5\fboxrule}
\setlength{\parskip}{2ex}
    Returns XVisualInfo class of a glcanvas object.

\setlength{\parskip}{1ex}
      \textbf{Parameters}
      \vspace{-1ex}

      \begin{quote}
        \begin{Ventry}{xxxxxxx}

          \item[pObject]

          pointer to glcanvas object ({\textless}pointer to 
          xfdata.FL\_OBJECT{\textgreater})

        \end{Ventry}

      \end{quote}

      \textbf{Return Value}
    \vspace{-1ex}

      \begin{quote}
      xvisualinfo class

      \end{quote}

\textbf{Status:} Untested + NoDoc + NoDemo = NOT OK



    \end{boxedminipage}

    \label{xformslib:library:fl_get_glcanvas_context}
    \index{xformslib \textit{(package)}!xformslib.library \textit{(module)}!xformslib.library.fl\_get\_glcanvas\_context \textit{(function)}}

    \vspace{0.5ex}

\hspace{.8\funcindent}\begin{boxedminipage}{\funcwidth}

    \raggedright \textbf{fl\_get\_glcanvas\_context}(\textit{pObject})

    \vspace{-1.5ex}

    \rule{\textwidth}{0.5\fboxrule}
\setlength{\parskip}{2ex}
\setlength{\parskip}{1ex}
      \textbf{Parameters}
      \vspace{-1ex}

      \begin{quote}
        \begin{Ventry}{xxxxxxx}

          \item[pObject]

          pointer to glcanvas object ({\textless}pointer to 
          xfdata.FL\_OBJECT{\textgreater})

        \end{Ventry}

      \end{quote}

      \textbf{Return Value}
    \vspace{-1ex}

      \begin{quote}
      glxcontext class

      \end{quote}

\textbf{Status:} Untested + NoDoc + NoDemo = NOT OK



    \end{boxedminipage}

    \label{xformslib:library:fl_glwincreate}
    \index{xformslib \textit{(package)}!xformslib.library \textit{(module)}!xformslib.library.fl\_glwincreate \textit{(function)}}

    \vspace{0.5ex}

\hspace{.8\funcindent}\begin{boxedminipage}{\funcwidth}

    \raggedright \textbf{fl\_glwincreate}(\textit{config}, \textit{glxcontext}, \textit{w}, \textit{h})

    \vspace{-1.5ex}

    \rule{\textwidth}{0.5\fboxrule}
\setlength{\parskip}{2ex}
\setlength{\parskip}{1ex}
      \textbf{Return Value}
    \vspace{-1ex}

      \begin{quote}
      window

      \end{quote}

\textbf{Status:} Untested + NoDoc + NoDemo = NOT OK



    \end{boxedminipage}

    \label{xformslib:library:fl_glwinopen}
    \index{xformslib \textit{(package)}!xformslib.library \textit{(module)}!xformslib.library.fl\_glwinopen \textit{(function)}}

    \vspace{0.5ex}

\hspace{.8\funcindent}\begin{boxedminipage}{\funcwidth}

    \raggedright \textbf{fl\_glwinopen}(\textit{config}, \textit{glxcontext}, \textit{w}, \textit{h})

    \vspace{-1.5ex}

    \rule{\textwidth}{0.5\fboxrule}
\setlength{\parskip}{2ex}
\setlength{\parskip}{1ex}
      \textbf{Return Value}
    \vspace{-1ex}

      \begin{quote}
      window

      \end{quote}

\textbf{Status:} Untested + NoDoc + NoDemo = NOT OK



    \end{boxedminipage}

    \label{xformslib:library:fl_add_chart}
    \index{xformslib \textit{(package)}!xformslib.library \textit{(module)}!xformslib.library.fl\_add\_chart \textit{(function)}}

    \vspace{0.5ex}

\hspace{.8\funcindent}\begin{boxedminipage}{\funcwidth}

    \raggedright \textbf{fl\_add\_chart}(\textit{charttype}, \textit{x}, \textit{y}, \textit{w}, \textit{h}, \textit{label})

    \vspace{-1.5ex}

    \rule{\textwidth}{0.5\fboxrule}
\setlength{\parskip}{2ex}
    Adds a chart object.

\setlength{\parskip}{1ex}
      \textbf{Parameters}
      \vspace{-1ex}

      \begin{quote}
        \begin{Ventry}{xxxxxxxxx}

          \item[charttype]

          type of chart to be created

          \item[charttype]

          [num./int] from xfdata module FL\_BAR\_CHART, FL\_HORBAR\_CHART, 
          FL\_LINE\_CHART, FL\_FILL\_CHART, FL\_SPIKE\_CHART, 
          FL\_PIE\_CHART, FL\_SPECIALPIE\_CHART

          \item[x]

          horizontal position (upper-left corner)

          \item[x]

          vertical position (upper-left corner)

          \item[w]

          width in coord units

          \item[h]

          height in coord units

          \item[label]

          text label of chart

        \end{Ventry}

      \end{quote}

      \textbf{Return Value}
    \vspace{-1ex}

      \begin{quote}
      pObject

      \end{quote}

\textbf{Status:} Tested + NoDoc + Demo = OK



    \end{boxedminipage}

    \label{xformslib:library:fl_clear_chart}
    \index{xformslib \textit{(package)}!xformslib.library \textit{(module)}!xformslib.library.fl\_clear\_chart \textit{(function)}}

    \vspace{0.5ex}

\hspace{.8\funcindent}\begin{boxedminipage}{\funcwidth}

    \raggedright \textbf{fl\_clear\_chart}(\textit{pObject})

    \vspace{-1.5ex}

    \rule{\textwidth}{0.5\fboxrule}
\setlength{\parskip}{2ex}
    Clears the contents of a chart.

\setlength{\parskip}{1ex}
      \textbf{Parameters}
      \vspace{-1ex}

      \begin{quote}
        \begin{Ventry}{xxxxxxx}

          \item[pObject]

          pointer to chart object ({\textless}pointer to 
          xfdata.FL\_OBJECT{\textgreater})

        \end{Ventry}

      \end{quote}

\textbf{Status:} Untested + NoDoc + NoDemo = NOT OK



    \end{boxedminipage}

    \label{xformslib:library:fl_add_chart_value}
    \index{xformslib \textit{(package)}!xformslib.library \textit{(module)}!xformslib.library.fl\_add\_chart\_value \textit{(function)}}

    \vspace{0.5ex}

\hspace{.8\funcindent}\begin{boxedminipage}{\funcwidth}

    \raggedright \textbf{fl\_add\_chart\_value}(\textit{pObject}, \textit{val}, \textit{label}, \textit{colr})

    \vspace{-1.5ex}

    \rule{\textwidth}{0.5\fboxrule}
\setlength{\parskip}{2ex}
    Adds an item to the chart.

\setlength{\parskip}{1ex}
      \textbf{Parameters}
      \vspace{-1ex}

      \begin{quote}
        \begin{Ventry}{xxxxxxx}

          \item[pObject]

          pointer to chart object ({\textless}pointer to 
          xfdata.FL\_OBJECT{\textgreater})

          \item[val]

          value of chart item

          \item[label]

          text label of chart object

          \item[colr]

          color num.

        \end{Ventry}

      \end{quote}

\textbf{Status:} Tested + NoDoc + Demo = OK



    \end{boxedminipage}

    \label{xformslib:library:fl_insert_chart_value}
    \index{xformslib \textit{(package)}!xformslib.library \textit{(module)}!xformslib.library.fl\_insert\_chart\_value \textit{(function)}}

    \vspace{0.5ex}

\hspace{.8\funcindent}\begin{boxedminipage}{\funcwidth}

    \raggedright \textbf{fl\_insert\_chart\_value}(\textit{pObject}, \textit{indx}, \textit{val}, \textit{label}, \textit{colr})

    \vspace{-1.5ex}

    \rule{\textwidth}{0.5\fboxrule}
\setlength{\parskip}{2ex}
    Inserts an item before indx to the chart.

\setlength{\parskip}{1ex}
      \textbf{Parameters}
      \vspace{-1ex}

      \begin{quote}
        \begin{Ventry}{xxxxxxx}

          \item[pObject]

          pointer to chart object ({\textless}pointer to 
          xfdata.FL\_OBJECT{\textgreater})

          \item[indx]

          index position of previous item

          \item[val]

          value of chart item

          \item[label]

          text label of chart

          \item[colr]

          color value

        \end{Ventry}

      \end{quote}

\textbf{Status:} Untested + NoDoc + NoDemo = NOT OK



    \end{boxedminipage}

    \label{xformslib:library:fl_replace_chart_value}
    \index{xformslib \textit{(package)}!xformslib.library \textit{(module)}!xformslib.library.fl\_replace\_chart\_value \textit{(function)}}

    \vspace{0.5ex}

\hspace{.8\funcindent}\begin{boxedminipage}{\funcwidth}

    \raggedright \textbf{fl\_replace\_chart\_value}(\textit{pObject}, \textit{indx}, \textit{val}, \textit{label}, \textit{colr})

    \vspace{-1.5ex}

    \rule{\textwidth}{0.5\fboxrule}
\setlength{\parskip}{2ex}
    Replaces value in the chart.

\setlength{\parskip}{1ex}
      \textbf{Parameters}
      \vspace{-1ex}

      \begin{quote}
        \begin{Ventry}{xxxxxxx}

          \item[pObject]

          pointer to chart object ({\textless}pointer to 
          xfdata.FL\_OBJECT{\textgreater})

          \item[indx]

          index position of item to be replaced

          \item[val]

          value of chart item

          \item[label]

          text label of chart

          \item[colr]

          color value

        \end{Ventry}

      \end{quote}

\textbf{Status:} Untested + NoDoc + NoDemo = NOT OK



    \end{boxedminipage}

    \label{xformslib:library:fl_set_chart_bounds}
    \index{xformslib \textit{(package)}!xformslib.library \textit{(module)}!xformslib.library.fl\_set\_chart\_bounds \textit{(function)}}

    \vspace{0.5ex}

\hspace{.8\funcindent}\begin{boxedminipage}{\funcwidth}

    \raggedright \textbf{fl\_set\_chart\_bounds}(\textit{pObject}, \textit{minbound}, \textit{maxbound})

    \vspace{-1.5ex}

    \rule{\textwidth}{0.5\fboxrule}
\setlength{\parskip}{2ex}
    Sets the boundaries/limits for values of a chart object.

\setlength{\parskip}{1ex}
      \textbf{Parameters}
      \vspace{-1ex}

      \begin{quote}
        \begin{Ventry}{xxxxxxxx}

          \item[pObject]

          pointer to chart object ({\textless}pointer to 
          xfdata.FL\_OBJECT{\textgreater})

          \item[minbound]

          minimum bounds to be set

          \item[maxbound]

          maximum bounds to be set

        \end{Ventry}

      \end{quote}

\textbf{Status:} Untested + NoDoc + NoDemo = NOT OK



    \end{boxedminipage}

    \label{xformslib:library:fl_get_chart_bounds}
    \index{xformslib \textit{(package)}!xformslib.library \textit{(module)}!xformslib.library.fl\_get\_chart\_bounds \textit{(function)}}

    \vspace{0.5ex}

\hspace{.8\funcindent}\begin{boxedminipage}{\funcwidth}

    \raggedright \textbf{fl\_get\_chart\_bounds}(\textit{pObject})

    \vspace{-1.5ex}

    \rule{\textwidth}{0.5\fboxrule}
\setlength{\parskip}{2ex}
    Returns the boundaries/limits set for values of a chart object.

\setlength{\parskip}{1ex}
      \textbf{Parameters}
      \vspace{-1ex}

      \begin{quote}
        \begin{Ventry}{xxxxxxx}

          \item[pObject]

          pointer to chart object ({\textless}pointer to 
          xfdata.FL\_OBJECT{\textgreater})

        \end{Ventry}

      \end{quote}

      \textbf{Return Value}
    \vspace{-1ex}

      \begin{quote}
      minbound, maxbound

      \end{quote}

\textbf{Attention:} API change from XForms - upstream was fl\_get\_chart\_bounds(pObject, 
minbound, maxbound)



\textbf{Status:} Untested + NoDoc + NoDemo = NOT OK



    \end{boxedminipage}

    \label{xformslib:library:fl_set_chart_maxnumb}
    \index{xformslib \textit{(package)}!xformslib.library \textit{(module)}!xformslib.library.fl\_set\_chart\_maxnumb \textit{(function)}}

    \vspace{0.5ex}

\hspace{.8\funcindent}\begin{boxedminipage}{\funcwidth}

    \raggedright \textbf{fl\_set\_chart\_maxnumb}(\textit{pObject}, \textit{maxnum})

    \vspace{-1.5ex}

    \rule{\textwidth}{0.5\fboxrule}
\setlength{\parskip}{2ex}
    Sets the maximum number of values displayed in the chart.

\setlength{\parskip}{1ex}
      \textbf{Parameters}
      \vspace{-1ex}

      \begin{quote}
        \begin{Ventry}{xxxxxxx}

          \item[pObject]

          chart object ({\textless}pointer to 
          xfdata.FL\_OBJECT{\textgreater})

          \item[maxnum]

          maximum number of values to display

        \end{Ventry}

      \end{quote}

\textbf{Status:} Untested + NoDoc + NoDemo = NOT OK



    \end{boxedminipage}

    \label{xformslib:library:fl_set_chart_autosize}
    \index{xformslib \textit{(package)}!xformslib.library \textit{(module)}!xformslib.library.fl\_set\_chart\_autosize \textit{(function)}}

    \vspace{0.5ex}

\hspace{.8\funcindent}\begin{boxedminipage}{\funcwidth}

    \raggedright \textbf{fl\_set\_chart\_autosize}(\textit{pObject}, \textit{autosize})

    \vspace{-1.5ex}

    \rule{\textwidth}{0.5\fboxrule}
\setlength{\parskip}{2ex}
    Sets whether the chart should autosize along the x-axis.

\setlength{\parskip}{1ex}
      \textbf{Parameters}
      \vspace{-1ex}

      \begin{quote}
        \begin{Ventry}{xxxxxxxx}

          \item[pObject]

          pointer to chart object ({\textless}pointer to 
          xfdata.FL\_OBJECT{\textgreater})

          \item[autosize]

          autosize flag is enabled/disabled (1{\textbar}0)

        \end{Ventry}

      \end{quote}

\textbf{Status:} Untested + NoDoc + NoDemo = NOT OK



    \end{boxedminipage}

    \label{xformslib:library:fl_set_chart_lstyle}
    \index{xformslib \textit{(package)}!xformslib.library \textit{(module)}!xformslib.library.fl\_set\_chart\_lstyle \textit{(function)}}

    \vspace{0.5ex}

\hspace{.8\funcindent}\begin{boxedminipage}{\funcwidth}

    \raggedright \textbf{fl\_set\_chart\_lstyle}(\textit{pObject}, \textit{lstyle})

    \vspace{-1.5ex}

    \rule{\textwidth}{0.5\fboxrule}
\setlength{\parskip}{2ex}
\setlength{\parskip}{1ex}
      \textbf{Parameters}
      \vspace{-1ex}

      \begin{quote}
        \begin{Ventry}{xxxxxxx}

          \item[pObject]

          pointer to object ({\textless}pointer to 
          xfdata.FL\_OBJECT{\textgreater})

        \end{Ventry}

      \end{quote}

\textbf{Status:} Untested + NoDoc + NoDemo = NOT OK



    \end{boxedminipage}

    \label{xformslib:library:fl_set_chart_lsize}
    \index{xformslib \textit{(package)}!xformslib.library \textit{(module)}!xformslib.library.fl\_set\_chart\_lsize \textit{(function)}}

    \vspace{0.5ex}

\hspace{.8\funcindent}\begin{boxedminipage}{\funcwidth}

    \raggedright \textbf{fl\_set\_chart\_lsize}(\textit{pObject}, \textit{lsize})

    \vspace{-1.5ex}

    \rule{\textwidth}{0.5\fboxrule}
\setlength{\parskip}{2ex}
\setlength{\parskip}{1ex}
      \textbf{Parameters}
      \vspace{-1ex}

      \begin{quote}
        \begin{Ventry}{xxxxxxx}

          \item[pObject]

          pointer to object ({\textless}pointer to 
          xfdata.FL\_OBJECT{\textgreater})

        \end{Ventry}

      \end{quote}

\textbf{Status:} Untested + NoDoc + NoDemo = NOT OK



    \end{boxedminipage}

    \label{xformslib:library:fl_set_chart_lcolor}
    \index{xformslib \textit{(package)}!xformslib.library \textit{(module)}!xformslib.library.fl\_set\_chart\_lcolor \textit{(function)}}

    \vspace{0.5ex}

\hspace{.8\funcindent}\begin{boxedminipage}{\funcwidth}

    \raggedright \textbf{fl\_set\_chart\_lcolor}(\textit{pObject}, \textit{colr})

    \vspace{-1.5ex}

    \rule{\textwidth}{0.5\fboxrule}
\setlength{\parskip}{2ex}
\setlength{\parskip}{1ex}
      \textbf{Parameters}
      \vspace{-1ex}

      \begin{quote}
        \begin{Ventry}{xxxxxxx}

          \item[pObject]

          pointer to chart object ({\textless}pointer to 
          xfdata.FL\_OBJECT{\textgreater})

          \item[colr]

          color value

        \end{Ventry}

      \end{quote}

\textbf{Status:} Untested + NoDoc + NoDemo = NOT OK



    \end{boxedminipage}

    \label{xformslib:library:fl_set_chart_baseline}
    \index{xformslib \textit{(package)}!xformslib.library \textit{(module)}!xformslib.library.fl\_set\_chart\_baseline \textit{(function)}}

    \vspace{0.5ex}

\hspace{.8\funcindent}\begin{boxedminipage}{\funcwidth}

    \raggedright \textbf{fl\_set\_chart\_baseline}(\textit{pObject}, \textit{yesno})

    \vspace{-1.5ex}

    \rule{\textwidth}{0.5\fboxrule}
\setlength{\parskip}{2ex}
\setlength{\parskip}{1ex}
      \textbf{Parameters}
      \vspace{-1ex}

      \begin{quote}
        \begin{Ventry}{xxxxxxx}

          \item[pObject]

          pointer to object ({\textless}pointer to 
          xfdata.FL\_OBJECT{\textgreater})

        \end{Ventry}

      \end{quote}

\textbf{Status:} Untested + NoDoc + NoDemo = NOT OK



    \end{boxedminipage}

    \label{xformslib:library:fl_set_chart_lcolor}
    \index{xformslib \textit{(package)}!xformslib.library \textit{(module)}!xformslib.library.fl\_set\_chart\_lcolor \textit{(function)}}

    \vspace{0.5ex}

\hspace{.8\funcindent}\begin{boxedminipage}{\funcwidth}

    \raggedright \textbf{fl\_set\_chart\_lcol}(\textit{pObject}, \textit{colr})

    \vspace{-1.5ex}

    \rule{\textwidth}{0.5\fboxrule}
\setlength{\parskip}{2ex}
\setlength{\parskip}{1ex}
      \textbf{Parameters}
      \vspace{-1ex}

      \begin{quote}
        \begin{Ventry}{xxxxxxx}

          \item[pObject]

          pointer to chart object ({\textless}pointer to 
          xfdata.FL\_OBJECT{\textgreater})

          \item[colr]

          color value

        \end{Ventry}

      \end{quote}

\textbf{Status:} Untested + NoDoc + NoDemo = NOT OK



    \end{boxedminipage}

    \label{xformslib:library:fl_stuff_clipboard}
    \index{xformslib \textit{(package)}!xformslib.library \textit{(module)}!xformslib.library.fl\_stuff\_clipboard \textit{(function)}}

    \vspace{0.5ex}

\hspace{.8\funcindent}\begin{boxedminipage}{\funcwidth}

    \raggedright \textbf{fl\_stuff\_clipboard}(\textit{pObject}, \textit{clipbdtype}, \textit{data}, \textit{size}, \textit{py\_LoseSelectionCb})

    \vspace{-1.5ex}

    \rule{\textwidth}{0.5\fboxrule}
\setlength{\parskip}{2ex}
\setlength{\parskip}{1ex}
      \textbf{Parameters}
      \vspace{-1ex}

      \begin{quote}
        \begin{Ventry}{xxxxxxxxxxxxxxxxxx}

          \item[pObject]

          pointer to clipboard object ({\textless}pointer to 
          xfdata.FL\_OBJECT{\textgreater})

          \item[clipbdtype]

          type of clipboard (not used)

            {\it (type=num./long)}

          \item[py\_LoseSelectionCb]

          python function callback, returning value

            {\it (type=func(pObject, longnum) -{\textgreater} num.)}

        \end{Ventry}

      \end{quote}

      \textbf{Return Value}
    \vspace{-1ex}

      \begin{quote}
      num

      \end{quote}

\textbf{Status:} Untested + NoDoc + NoDemo = NOT OK



    \end{boxedminipage}

    \label{xformslib:library:fl_request_clipboard}
    \index{xformslib \textit{(package)}!xformslib.library \textit{(module)}!xformslib.library.fl\_request\_clipboard \textit{(function)}}

    \vspace{0.5ex}

\hspace{.8\funcindent}\begin{boxedminipage}{\funcwidth}

    \raggedright \textbf{fl\_request\_clipboard}(\textit{pObject}, \textit{clipbdtype}, \textit{py\_SelectionCb})

    \vspace{-1.5ex}

    \rule{\textwidth}{0.5\fboxrule}
\setlength{\parskip}{2ex}
\setlength{\parskip}{1ex}
      \textbf{Parameters}
      \vspace{-1ex}

      \begin{quote}
        \begin{Ventry}{xxxxxxxxxxxxxx}

          \item[pObject]

          pointer to clipboard object ({\textless}pointer to 
          xfdata.FL\_OBJECT{\textgreater})

          \item[clipbdtype]

          type of clipboard (not used)

          \item[py\_SelectionCb]

          python function callback, returning value

            {\it (type=func(pObject, longnum, ptr\_void, longnum) -{\textgreater} num.)}

        \end{Ventry}

      \end{quote}

      \textbf{Return Value}
    \vspace{-1ex}

      \begin{quote}
      num

      \end{quote}

\textbf{Status:} Untested + NoDoc + NoDemo = NOT OK



    \end{boxedminipage}

    \label{xformslib:library:fl_add_clock}
    \index{xformslib \textit{(package)}!xformslib.library \textit{(module)}!xformslib.library.fl\_add\_clock \textit{(function)}}

    \vspace{0.5ex}

\hspace{.8\funcindent}\begin{boxedminipage}{\funcwidth}

    \raggedright \textbf{fl\_add\_clock}(\textit{clocktype}, \textit{x}, \textit{y}, \textit{w}, \textit{h}, \textit{label})

    \vspace{-1.5ex}

    \rule{\textwidth}{0.5\fboxrule}
\setlength{\parskip}{2ex}
    Adds a clock object.

\setlength{\parskip}{1ex}
      \textbf{Parameters}
      \vspace{-1ex}

      \begin{quote}
        \begin{Ventry}{xxxxxxxxx}

          \item[clocktype]

          type of clock to be added

            {\it (type=[num./int] from xfdata module FL\_ANALOG\_CLOCK, FL\_DIGITAL\_CLOCK)}

          \item[x]

          horizontal position (upper-left corner)

          \item[x]

          vertical position (upper-left corner)

          \item[w]

          width in coord units

          \item[h]

          height in coord units

          \item[label]

          text label of clock

        \end{Ventry}

      \end{quote}

      \textbf{Return Value}
    \vspace{-1ex}

      \begin{quote}
      pObject

      \end{quote}

\textbf{Status:} Tested + NoDoc + Demo = OK



    \end{boxedminipage}

    \label{xformslib:library:fl_get_clock}
    \index{xformslib \textit{(package)}!xformslib.library \textit{(module)}!xformslib.library.fl\_get\_clock \textit{(function)}}

    \vspace{0.5ex}

\hspace{.8\funcindent}\begin{boxedminipage}{\funcwidth}

    \raggedright \textbf{fl\_get\_clock}(\textit{pObject})

    \vspace{-1.5ex}

    \rule{\textwidth}{0.5\fboxrule}
\setlength{\parskip}{2ex}
    Returns time values from a clock object.

\setlength{\parskip}{1ex}
      \textbf{Parameters}
      \vspace{-1ex}

      \begin{quote}
        \begin{Ventry}{xxxxxxx}

          \item[pObject]

          pointer to clock object ({\textless}pointer to 
          xfdata.FL\_OBJECT{\textgreater})

        \end{Ventry}

      \end{quote}

      \textbf{Return Value}
    \vspace{-1ex}

      \begin{quote}
      hr, mn, sec

      \end{quote}

\textbf{Attention:} API change from XForms - upstream was fl\_get\_clock(pObject, hr, mn, sec)



\textbf{Status:} Untested + NoDoc + NoDemo = NOT OK



    \end{boxedminipage}

    \label{xformslib:library:fl_set_clock_adjustment}
    \index{xformslib \textit{(package)}!xformslib.library \textit{(module)}!xformslib.library.fl\_set\_clock\_adjustment \textit{(function)}}

    \vspace{0.5ex}

\hspace{.8\funcindent}\begin{boxedminipage}{\funcwidth}

    \raggedright \textbf{fl\_set\_clock\_adjustment}(\textit{pObject}, \textit{offset})

    \vspace{-1.5ex}

    \rule{\textwidth}{0.5\fboxrule}
\setlength{\parskip}{2ex}
\setlength{\parskip}{1ex}
      \textbf{Parameters}
      \vspace{-1ex}

      \begin{quote}
        \begin{Ventry}{xxxxxxx}

          \item[pObject]

          pointer to object ({\textless}pointer to 
          xfdata.FL\_OBJECT{\textgreater})

        \end{Ventry}

      \end{quote}

      \textbf{Return Value}
    \vspace{-1ex}

      \begin{quote}
      num

      \end{quote}

\textbf{Status:} Untested + NoDoc + NoDemo = NOT OK



    \end{boxedminipage}

    \label{xformslib:library:fl_set_clock_ampm}
    \index{xformslib \textit{(package)}!xformslib.library \textit{(module)}!xformslib.library.fl\_set\_clock\_ampm \textit{(function)}}

    \vspace{0.5ex}

\hspace{.8\funcindent}\begin{boxedminipage}{\funcwidth}

    \raggedright \textbf{fl\_set\_clock\_ampm}(\textit{pObject}, \textit{y})

    \vspace{-1.5ex}

    \rule{\textwidth}{0.5\fboxrule}
\setlength{\parskip}{2ex}
\setlength{\parskip}{1ex}
      \textbf{Parameters}
      \vspace{-1ex}

      \begin{quote}
        \begin{Ventry}{xxxxxxx}

          \item[pObject]

          pointer to object ({\textless}pointer to 
          xfdata.FL\_OBJECT{\textgreater})

        \end{Ventry}

      \end{quote}

\textbf{Status:} Untested + NoDoc + NoDemo = NOT OK



    \end{boxedminipage}

    \label{xformslib:library:fl_add_counter}
    \index{xformslib \textit{(package)}!xformslib.library \textit{(module)}!xformslib.library.fl\_add\_counter \textit{(function)}}

    \vspace{0.5ex}

\hspace{.8\funcindent}\begin{boxedminipage}{\funcwidth}

    \raggedright \textbf{fl\_add\_counter}(\textit{countertype}, \textit{x}, \textit{y}, \textit{w}, \textit{h}, \textit{label})

    \vspace{-1.5ex}

    \rule{\textwidth}{0.5\fboxrule}
\setlength{\parskip}{2ex}
    Adds a counter object.

\setlength{\parskip}{1ex}
      \textbf{Parameters}
      \vspace{-1ex}

      \begin{quote}
        \begin{Ventry}{xxxxxxxxxxx}

          \item[countertype]

          type of counter to be added

            {\it (type=[num./int] from xfdata module FL\_NORMAL\_COUNTER, FL\_SIMPLE\_COUNTER)}

          \item[x]

          horizontal position (upper-left corner)

          \item[x]

          vertical position (upper-left corner)

          \item[w]

          width in coord units

          \item[h]

          height in coord units

          \item[label]

          text label of counter

        \end{Ventry}

      \end{quote}

      \textbf{Return Value}
    \vspace{-1ex}

      \begin{quote}
      pObject

      \end{quote}

\textbf{Status:} Tested + NoDoc + Demo = OK



    \end{boxedminipage}

    \label{xformslib:library:fl_set_counter_value}
    \index{xformslib \textit{(package)}!xformslib.library \textit{(module)}!xformslib.library.fl\_set\_counter\_value \textit{(function)}}

    \vspace{0.5ex}

\hspace{.8\funcindent}\begin{boxedminipage}{\funcwidth}

    \raggedright \textbf{fl\_set\_counter\_value}(\textit{pObject}, \textit{val})

    \vspace{-1.5ex}

    \rule{\textwidth}{0.5\fboxrule}
\setlength{\parskip}{2ex}
\setlength{\parskip}{1ex}
      \textbf{Parameters}
      \vspace{-1ex}

      \begin{quote}
        \begin{Ventry}{xxxxxxx}

          \item[pObject]

          pointer to object ({\textless}pointer to 
          xfdata.FL\_OBJECT{\textgreater})

        \end{Ventry}

      \end{quote}

\textbf{Status:} Tested + NoDoc + Demo = OK



    \end{boxedminipage}

    \label{xformslib:library:fl_set_counter_bounds}
    \index{xformslib \textit{(package)}!xformslib.library \textit{(module)}!xformslib.library.fl\_set\_counter\_bounds \textit{(function)}}

    \vspace{0.5ex}

\hspace{.8\funcindent}\begin{boxedminipage}{\funcwidth}

    \raggedright \textbf{fl\_set\_counter\_bounds}(\textit{pObject}, \textit{minbound}, \textit{maxbound})

    \vspace{-1.5ex}

    \rule{\textwidth}{0.5\fboxrule}
\setlength{\parskip}{2ex}
\setlength{\parskip}{1ex}
      \textbf{Parameters}
      \vspace{-1ex}

      \begin{quote}
        \begin{Ventry}{xxxxxxx}

          \item[pObject]

          pointer to counter object ({\textless}pointer to 
          xfdata.FL\_OBJECT{\textgreater})

        \end{Ventry}

      \end{quote}

\textbf{Status:} Tested + NoDoc + Demo = OK



    \end{boxedminipage}

    \label{xformslib:library:fl_set_counter_step}
    \index{xformslib \textit{(package)}!xformslib.library \textit{(module)}!xformslib.library.fl\_set\_counter\_step \textit{(function)}}

    \vspace{0.5ex}

\hspace{.8\funcindent}\begin{boxedminipage}{\funcwidth}

    \raggedright \textbf{fl\_set\_counter\_step}(\textit{pObject}, \textit{s}, \textit{l})

    \vspace{-1.5ex}

    \rule{\textwidth}{0.5\fboxrule}
\setlength{\parskip}{2ex}
\setlength{\parskip}{1ex}
      \textbf{Parameters}
      \vspace{-1ex}

      \begin{quote}
        \begin{Ventry}{xxxxxxx}

          \item[pObject]

          pointer to object ({\textless}pointer to 
          xfdata.FL\_OBJECT{\textgreater})

        \end{Ventry}

      \end{quote}

\textbf{Status:} Tested + NoDoc + Demo = OK



    \end{boxedminipage}

    \label{xformslib:library:fl_set_counter_precision}
    \index{xformslib \textit{(package)}!xformslib.library \textit{(module)}!xformslib.library.fl\_set\_counter\_precision \textit{(function)}}

    \vspace{0.5ex}

\hspace{.8\funcindent}\begin{boxedminipage}{\funcwidth}

    \raggedright \textbf{fl\_set\_counter\_precision}(\textit{pObject}, \textit{prec})

    \vspace{-1.5ex}

    \rule{\textwidth}{0.5\fboxrule}
\setlength{\parskip}{2ex}
\setlength{\parskip}{1ex}
      \textbf{Parameters}
      \vspace{-1ex}

      \begin{quote}
        \begin{Ventry}{xxxxxxx}

          \item[pObject]

          pointer to object ({\textless}pointer to 
          xfdata.FL\_OBJECT{\textgreater})

        \end{Ventry}

      \end{quote}

\textbf{Status:} Tested + NoDoc + Demo = OK



    \end{boxedminipage}

    \label{xformslib:library:fl_get_counter_precision}
    \index{xformslib \textit{(package)}!xformslib.library \textit{(module)}!xformslib.library.fl\_get\_counter\_precision \textit{(function)}}

    \vspace{0.5ex}

\hspace{.8\funcindent}\begin{boxedminipage}{\funcwidth}

    \raggedright \textbf{fl\_get\_counter\_precision}(\textit{pObject})

    \vspace{-1.5ex}

    \rule{\textwidth}{0.5\fboxrule}
\setlength{\parskip}{2ex}
\setlength{\parskip}{1ex}
      \textbf{Parameters}
      \vspace{-1ex}

      \begin{quote}
        \begin{Ventry}{xxxxxxx}

          \item[pObject]

          pointer to object ({\textless}pointer to 
          xfdata.FL\_OBJECT{\textgreater})

        \end{Ventry}

      \end{quote}

      \textbf{Return Value}
    \vspace{-1ex}

      \begin{quote}
      num

      \end{quote}

\textbf{Status:} Untested + NoDoc + NoDemo = NOT OK



    \end{boxedminipage}

    \label{xformslib:library:fl_get_counter_value}
    \index{xformslib \textit{(package)}!xformslib.library \textit{(module)}!xformslib.library.fl\_get\_counter\_value \textit{(function)}}

    \vspace{0.5ex}

\hspace{.8\funcindent}\begin{boxedminipage}{\funcwidth}

    \raggedright \textbf{fl\_get\_counter\_value}(\textit{pObject})

    \vspace{-1.5ex}

    \rule{\textwidth}{0.5\fboxrule}
\setlength{\parskip}{2ex}
\setlength{\parskip}{1ex}
      \textbf{Parameters}
      \vspace{-1ex}

      \begin{quote}
        \begin{Ventry}{xxxxxxx}

          \item[pObject]

          pointer to object ({\textless}pointer to 
          xfdata.FL\_OBJECT{\textgreater})

        \end{Ventry}

      \end{quote}

      \textbf{Return Value}
    \vspace{-1ex}

      \begin{quote}
      num

      \end{quote}

\textbf{Status:} Tested + NoDoc + Demo = OK



    \end{boxedminipage}

    \label{xformslib:library:fl_get_counter_bounds}
    \index{xformslib \textit{(package)}!xformslib.library \textit{(module)}!xformslib.library.fl\_get\_counter\_bounds \textit{(function)}}

    \vspace{0.5ex}

\hspace{.8\funcindent}\begin{boxedminipage}{\funcwidth}

    \raggedright \textbf{fl\_get\_counter\_bounds}(\textit{pObject})

    \vspace{-1.5ex}

    \rule{\textwidth}{0.5\fboxrule}
\setlength{\parskip}{2ex}
\setlength{\parskip}{1ex}
      \textbf{Parameters}
      \vspace{-1ex}

      \begin{quote}
        \begin{Ventry}{xxxxxxx}

          \item[pObject]

          pointer to object ({\textless}pointer to 
          xfdata.FL\_OBJECT{\textgreater})

        \end{Ventry}

      \end{quote}

      \textbf{Return Value}
    \vspace{-1ex}

      \begin{quote}
      minbound, maxbound

      \end{quote}

\textbf{Attention:} API change from XForms - upstream was fl\_get\_counter\_bounds(pObject, 
minbound, maxbound)



\textbf{Status:} Untested + NoDoc + NoDemo = NOT OK



    \end{boxedminipage}

    \label{xformslib:library:fl_get_counter_step}
    \index{xformslib \textit{(package)}!xformslib.library \textit{(module)}!xformslib.library.fl\_get\_counter\_step \textit{(function)}}

    \vspace{0.5ex}

\hspace{.8\funcindent}\begin{boxedminipage}{\funcwidth}

    \raggedright \textbf{fl\_get\_counter\_step}(\textit{pObject})

    \vspace{-1.5ex}

    \rule{\textwidth}{0.5\fboxrule}
\setlength{\parskip}{2ex}
\setlength{\parskip}{1ex}
      \textbf{Parameters}
      \vspace{-1ex}

      \begin{quote}
        \begin{Ventry}{xxxxxxx}

          \item[pObject]

          pointer to object ({\textless}pointer to 
          xfdata.FL\_OBJECT{\textgreater})

        \end{Ventry}

      \end{quote}

      \textbf{Return Value}
    \vspace{-1ex}

      \begin{quote}
      s, l

      \end{quote}

\textbf{Attention:} API change from XForms - upstream was fl\_get\_counter\_step(pObject, s, l)



\textbf{Status:} Untested + NoDoc + NoDemo = NOT OK



    \end{boxedminipage}

    \label{xformslib:library:fl_set_counter_filter}
    \index{xformslib \textit{(package)}!xformslib.library \textit{(module)}!xformslib.library.fl\_set\_counter\_filter \textit{(function)}}

    \vspace{0.5ex}

\hspace{.8\funcindent}\begin{boxedminipage}{\funcwidth}

    \raggedright \textbf{fl\_set\_counter\_filter}(\textit{pObject}, \textit{py\_ValFilter})

    \vspace{-1.5ex}

    \rule{\textwidth}{0.5\fboxrule}
\setlength{\parskip}{2ex}
\setlength{\parskip}{1ex}
      \textbf{Parameters}
      \vspace{-1ex}

      \begin{quote}
        \begin{Ventry}{xxxxxxx}

          \item[pObject]

          pointer to object ({\textless}pointer to 
          xfdata.FL\_OBJECT{\textgreater})

        \end{Ventry}

      \end{quote}

\textbf{Status:} Untested + NoDoc + NoDemo = NOT OK



    \end{boxedminipage}

    \label{xformslib:library:fl_get_counter_repeat}
    \index{xformslib \textit{(package)}!xformslib.library \textit{(module)}!xformslib.library.fl\_get\_counter\_repeat \textit{(function)}}

    \vspace{0.5ex}

\hspace{.8\funcindent}\begin{boxedminipage}{\funcwidth}

    \raggedright \textbf{fl\_get\_counter\_repeat}(\textit{pObject})

    \vspace{-1.5ex}

    \rule{\textwidth}{0.5\fboxrule}
\setlength{\parskip}{2ex}
\setlength{\parskip}{1ex}
      \textbf{Parameters}
      \vspace{-1ex}

      \begin{quote}
        \begin{Ventry}{xxxxxxx}

          \item[pObject]

          pointer to object ({\textless}pointer to 
          xfdata.FL\_OBJECT{\textgreater})

        \end{Ventry}

      \end{quote}

      \textbf{Return Value}
    \vspace{-1ex}

      \begin{quote}
      num

      \end{quote}

\textbf{Status:} Untested + NoDoc + NoDemo = NOT OK



    \end{boxedminipage}

    \label{xformslib:library:fl_set_counter_repeat}
    \index{xformslib \textit{(package)}!xformslib.library \textit{(module)}!xformslib.library.fl\_set\_counter\_repeat \textit{(function)}}

    \vspace{0.5ex}

\hspace{.8\funcindent}\begin{boxedminipage}{\funcwidth}

    \raggedright \textbf{fl\_set\_counter\_repeat}(\textit{pObject}, \textit{msec})

    \vspace{-1.5ex}

    \rule{\textwidth}{0.5\fboxrule}
\setlength{\parskip}{2ex}
\setlength{\parskip}{1ex}
      \textbf{Parameters}
      \vspace{-1ex}

      \begin{quote}
        \begin{Ventry}{xxxxxxx}

          \item[pObject]

          pointer to object ({\textless}pointer to 
          xfdata.FL\_OBJECT{\textgreater})

        \end{Ventry}

      \end{quote}

\textbf{Status:} Untested + NoDoc + NoDemo = NOT OK



    \end{boxedminipage}

    \label{xformslib:library:fl_get_counter_min_repeat}
    \index{xformslib \textit{(package)}!xformslib.library \textit{(module)}!xformslib.library.fl\_get\_counter\_min\_repeat \textit{(function)}}

    \vspace{0.5ex}

\hspace{.8\funcindent}\begin{boxedminipage}{\funcwidth}

    \raggedright \textbf{fl\_get\_counter\_min\_repeat}(\textit{pObject})

    \vspace{-1.5ex}

    \rule{\textwidth}{0.5\fboxrule}
\setlength{\parskip}{2ex}
\setlength{\parskip}{1ex}
      \textbf{Parameters}
      \vspace{-1ex}

      \begin{quote}
        \begin{Ventry}{xxxxxxx}

          \item[pObject]

          pointer to object ({\textless}pointer to 
          xfdata.FL\_OBJECT{\textgreater})

        \end{Ventry}

      \end{quote}

      \textbf{Return Value}
    \vspace{-1ex}

      \begin{quote}
      num

      \end{quote}

\textbf{Status:} Untested + NoDoc + NoDemo = NOT OK



    \end{boxedminipage}

    \label{xformslib:library:fl_set_counter_min_repeat}
    \index{xformslib \textit{(package)}!xformslib.library \textit{(module)}!xformslib.library.fl\_set\_counter\_min\_repeat \textit{(function)}}

    \vspace{0.5ex}

\hspace{.8\funcindent}\begin{boxedminipage}{\funcwidth}

    \raggedright \textbf{fl\_set\_counter\_min\_repeat}(\textit{pObject}, \textit{msec})

    \vspace{-1.5ex}

    \rule{\textwidth}{0.5\fboxrule}
\setlength{\parskip}{2ex}
\setlength{\parskip}{1ex}
      \textbf{Parameters}
      \vspace{-1ex}

      \begin{quote}
        \begin{Ventry}{xxxxxxx}

          \item[pObject]

          pointer to object ({\textless}pointer to 
          xfdata.FL\_OBJECT{\textgreater})

        \end{Ventry}

      \end{quote}

\textbf{Status:} Untested + NoDoc + NoDemo = NOT OK



    \end{boxedminipage}

    \label{xformslib:library:fl_get_counter_speedjump}
    \index{xformslib \textit{(package)}!xformslib.library \textit{(module)}!xformslib.library.fl\_get\_counter\_speedjump \textit{(function)}}

    \vspace{0.5ex}

\hspace{.8\funcindent}\begin{boxedminipage}{\funcwidth}

    \raggedright \textbf{fl\_get\_counter\_speedjump}(\textit{pObject})

    \vspace{-1.5ex}

    \rule{\textwidth}{0.5\fboxrule}
\setlength{\parskip}{2ex}
\setlength{\parskip}{1ex}
      \textbf{Parameters}
      \vspace{-1ex}

      \begin{quote}
        \begin{Ventry}{xxxxxxx}

          \item[pObject]

          pointer to object ({\textless}pointer to 
          xfdata.FL\_OBJECT{\textgreater})

        \end{Ventry}

      \end{quote}

      \textbf{Return Value}
    \vspace{-1ex}

      \begin{quote}
      num

      \end{quote}

\textbf{Status:} Untested + NoDoc + NoDemo = NOT OK



    \end{boxedminipage}

    \label{xformslib:library:fl_set_counter_speedjump}
    \index{xformslib \textit{(package)}!xformslib.library \textit{(module)}!xformslib.library.fl\_set\_counter\_speedjump \textit{(function)}}

    \vspace{0.5ex}

\hspace{.8\funcindent}\begin{boxedminipage}{\funcwidth}

    \raggedright \textbf{fl\_set\_counter\_speedjump}(\textit{pObject}, \textit{yesno})

    \vspace{-1.5ex}

    \rule{\textwidth}{0.5\fboxrule}
\setlength{\parskip}{2ex}
\setlength{\parskip}{1ex}
      \textbf{Parameters}
      \vspace{-1ex}

      \begin{quote}
        \begin{Ventry}{xxxxxxx}

          \item[pObject]

          pointer to object ({\textless}pointer to 
          xfdata.FL\_OBJECT{\textgreater})

        \end{Ventry}

      \end{quote}

\textbf{Status:} Untested + NoDoc + NoDemo = NOT OK



    \end{boxedminipage}

    \label{xformslib:library:fl_set_cursor}
    \index{xformslib \textit{(package)}!xformslib.library \textit{(module)}!xformslib.library.fl\_set\_cursor \textit{(function)}}

    \vspace{0.5ex}

\hspace{.8\funcindent}\begin{boxedminipage}{\funcwidth}

    \raggedright \textbf{fl\_set\_cursor}(\textit{win}, \textit{cursnum})

    \vspace{-1.5ex}

    \rule{\textwidth}{0.5\fboxrule}
\setlength{\parskip}{2ex}
    Set cursor for window to provided cursor number name. Name is either 
    the standard XC\_ or Form defined.

\setlength{\parskip}{1ex}
      \textbf{Parameters}
      \vspace{-1ex}

      \begin{quote}
        \begin{Ventry}{xxxxxxx}

          \item[win]

          window

          \item[cursnum]

          cursor number

        \end{Ventry}

      \end{quote}

\textbf{Status:} Tested + NoDoc + Demo = OK



    \end{boxedminipage}

    \label{xformslib:library:fl_set_cursor_color}
    \index{xformslib \textit{(package)}!xformslib.library \textit{(module)}!xformslib.library.fl\_set\_cursor\_color \textit{(function)}}

    \vspace{0.5ex}

\hspace{.8\funcindent}\begin{boxedminipage}{\funcwidth}

    \raggedright \textbf{fl\_set\_cursor\_color}(\textit{cursnum}, \textit{fgcolr}, \textit{bgcolr})

    \vspace{-1.5ex}

    \rule{\textwidth}{0.5\fboxrule}
\setlength{\parskip}{2ex}
    Sets foreground and background colors for cursor.

\setlength{\parskip}{1ex}
      \textbf{Parameters}
      \vspace{-1ex}

      \begin{quote}
        \begin{Ventry}{xxxxxxx}

          \item[cursnum]

          cursor number

          \item[fgcolr]

          foreground color to be set

          \item[bgcolr]

          background color to be set

        \end{Ventry}

      \end{quote}

\textbf{Status:} Tested + NoDoc + Demo = OK



    \end{boxedminipage}

    \label{xformslib:library:fl_create_bitmap_cursor}
    \index{xformslib \textit{(package)}!xformslib.library \textit{(module)}!xformslib.library.fl\_create\_bitmap\_cursor \textit{(function)}}

    \vspace{0.5ex}

\hspace{.8\funcindent}\begin{boxedminipage}{\funcwidth}

    \raggedright \textbf{fl\_create\_bitmap\_cursor}(\textit{source}, \textit{maskstr}, \textit{w}, \textit{h}, \textit{hotx}, \textit{hoty})

    \vspace{-1.5ex}

    \rule{\textwidth}{0.5\fboxrule}
\setlength{\parskip}{2ex}
\setlength{\parskip}{1ex}
      \textbf{Return Value}
    \vspace{-1ex}

      \begin{quote}
      num

      \end{quote}

\textbf{Status:} HalfTested + NoDoc + Demo = NOT OK (bitmap after animated problematic)



    \end{boxedminipage}

    \label{xformslib:library:fl_create_animated_cursor}
    \index{xformslib \textit{(package)}!xformslib.library \textit{(module)}!xformslib.library.fl\_create\_animated\_cursor \textit{(function)}}

    \vspace{0.5ex}

\hspace{.8\funcindent}\begin{boxedminipage}{\funcwidth}

    \raggedright \textbf{fl\_create\_animated\_cursor}(\textit{curnums}, \textit{timeout})

    \vspace{-1.5ex}

    \rule{\textwidth}{0.5\fboxrule}
\setlength{\parskip}{2ex}
\setlength{\parskip}{1ex}
      \textbf{Return Value}
    \vspace{-1ex}

      \begin{quote}
      num

      \end{quote}

\textbf{Status:} HalfTested + NoDoc + Demo = NOT OK (curnums data problematic)



    \end{boxedminipage}

    \label{xformslib:library:fl_get_cursor_byname}
    \index{xformslib \textit{(package)}!xformslib.library \textit{(module)}!xformslib.library.fl\_get\_cursor\_byname \textit{(function)}}

    \vspace{0.5ex}

\hspace{.8\funcindent}\begin{boxedminipage}{\funcwidth}

    \raggedright \textbf{fl\_get\_cursor\_byname}(\textit{cursnum})

    \vspace{-1.5ex}

    \rule{\textwidth}{0.5\fboxrule}
\setlength{\parskip}{2ex}
    Return cursor corresponding to number.

\setlength{\parskip}{1ex}
      \textbf{Parameters}
      \vspace{-1ex}

      \begin{quote}
        \begin{Ventry}{xxxxxxx}

          \item[cursnum]

          cursor number

        \end{Ventry}

      \end{quote}

      \textbf{Return Value}
    \vspace{-1ex}

      \begin{quote}
      cursor

      \end{quote}

\textbf{Status:} Untested + NoDoc + NoDemo = NOT OK



    \end{boxedminipage}

    \label{xformslib:library:fl_reset_cursor}
    \index{xformslib \textit{(package)}!xformslib.library \textit{(module)}!xformslib.library.fl\_reset\_cursor \textit{(function)}}

    \vspace{0.5ex}

\hspace{.8\funcindent}\begin{boxedminipage}{\funcwidth}

    \raggedright \textbf{fl\_reset\_cursor}(\textit{win})

    \vspace{-1.5ex}

    \rule{\textwidth}{0.5\fboxrule}
\setlength{\parskip}{2ex}
    Reset used cursor, reverting to default one.

\setlength{\parskip}{1ex}
      \textbf{Parameters}
      \vspace{-1ex}

      \begin{quote}
        \begin{Ventry}{xxx}

          \item[win]

          window

        \end{Ventry}

      \end{quote}

\textbf{Status:} Tested + NoDoc + Demo = OK



    \end{boxedminipage}

    \label{xformslib:library:fl_add_dial}
    \index{xformslib \textit{(package)}!xformslib.library \textit{(module)}!xformslib.library.fl\_add\_dial \textit{(function)}}

    \vspace{0.5ex}

\hspace{.8\funcindent}\begin{boxedminipage}{\funcwidth}

    \raggedright \textbf{fl\_add\_dial}(\textit{dialtype}, \textit{x}, \textit{y}, \textit{w}, \textit{h}, \textit{label})

    \vspace{-1.5ex}

    \rule{\textwidth}{0.5\fboxrule}
\setlength{\parskip}{2ex}
    Adds a dial object.

\setlength{\parskip}{1ex}
      \textbf{Parameters}
      \vspace{-1ex}

      \begin{quote}
        \begin{Ventry}{xxxxxxxx}

          \item[dialtype]

          type of dial to be added

            {\it (type=[num./int] from xfdata module FL\_NORMAL\_DIAL, FL\_LINE\_DIAL, 
FL\_FILL\_DIAL)}

          \item[x]

          horizontal position (upper-left corner)

          \item[y]

          vertical position (upper-left corner)

          \item[w]

          width in coord units

          \item[h]

          height in coord units

          \item[label]

          text label of dial

        \end{Ventry}

      \end{quote}

      \textbf{Return Value}
    \vspace{-1ex}

      \begin{quote}
      pObject

      \end{quote}

\textbf{Status:} Tested + NoDoc + Demo = OK



    \end{boxedminipage}

    \label{xformslib:library:fl_set_dial_value}
    \index{xformslib \textit{(package)}!xformslib.library \textit{(module)}!xformslib.library.fl\_set\_dial\_value \textit{(function)}}

    \vspace{0.5ex}

\hspace{.8\funcindent}\begin{boxedminipage}{\funcwidth}

    \raggedright \textbf{fl\_set\_dial\_value}(\textit{pObject}, \textit{val})

    \vspace{-1.5ex}

    \rule{\textwidth}{0.5\fboxrule}
\setlength{\parskip}{2ex}
\setlength{\parskip}{1ex}
      \textbf{Parameters}
      \vspace{-1ex}

      \begin{quote}
        \begin{Ventry}{xxxxxxx}

          \item[pObject]

          pointer to object ({\textless}pointer to 
          xfdata.FL\_OBJECT{\textgreater})

        \end{Ventry}

      \end{quote}

\textbf{Status:} Tested + NoDoc + Demo = OK



    \end{boxedminipage}

    \label{xformslib:library:fl_get_dial_value}
    \index{xformslib \textit{(package)}!xformslib.library \textit{(module)}!xformslib.library.fl\_get\_dial\_value \textit{(function)}}

    \vspace{0.5ex}

\hspace{.8\funcindent}\begin{boxedminipage}{\funcwidth}

    \raggedright \textbf{fl\_get\_dial\_value}(\textit{pObject})

    \vspace{-1.5ex}

    \rule{\textwidth}{0.5\fboxrule}
\setlength{\parskip}{2ex}
\setlength{\parskip}{1ex}
      \textbf{Parameters}
      \vspace{-1ex}

      \begin{quote}
        \begin{Ventry}{xxxxxxx}

          \item[pObject]

          pointer to dial object ({\textless}pointer to 
          xfdata.FL\_OBJECT{\textgreater})

        \end{Ventry}

      \end{quote}

      \textbf{Return Value}
    \vspace{-1ex}

      \begin{quote}
      num

      \end{quote}

\textbf{Status:} Tested + NoDoc + Demo = OK



    \end{boxedminipage}

    \label{xformslib:library:fl_set_dial_bounds}
    \index{xformslib \textit{(package)}!xformslib.library \textit{(module)}!xformslib.library.fl\_set\_dial\_bounds \textit{(function)}}

    \vspace{0.5ex}

\hspace{.8\funcindent}\begin{boxedminipage}{\funcwidth}

    \raggedright \textbf{fl\_set\_dial\_bounds}(\textit{pObject}, \textit{minbound}, \textit{maxbound})

    \vspace{-1.5ex}

    \rule{\textwidth}{0.5\fboxrule}
\setlength{\parskip}{2ex}
\setlength{\parskip}{1ex}
      \textbf{Parameters}
      \vspace{-1ex}

      \begin{quote}
        \begin{Ventry}{xxxxxxx}

          \item[pObject]

          pointer to object ({\textless}pointer to 
          xfdata.FL\_OBJECT{\textgreater})

        \end{Ventry}

      \end{quote}

\textbf{Status:} Tested + NoDoc + Demo = OK



    \end{boxedminipage}

    \label{xformslib:library:fl_get_dial_bounds}
    \index{xformslib \textit{(package)}!xformslib.library \textit{(module)}!xformslib.library.fl\_get\_dial\_bounds \textit{(function)}}

    \vspace{0.5ex}

\hspace{.8\funcindent}\begin{boxedminipage}{\funcwidth}

    \raggedright \textbf{fl\_get\_dial\_bounds}(\textit{pObject})

    \vspace{-1.5ex}

    \rule{\textwidth}{0.5\fboxrule}
\setlength{\parskip}{2ex}
\setlength{\parskip}{1ex}
      \textbf{Parameters}
      \vspace{-1ex}

      \begin{quote}
        \begin{Ventry}{xxxxxxx}

          \item[pObject]

          pointer to object ({\textless}pointer to 
          xfdata.FL\_OBJECT{\textgreater})

        \end{Ventry}

      \end{quote}

      \textbf{Return Value}
    \vspace{-1ex}

      \begin{quote}
      minbound, maxbound

      \end{quote}

\textbf{Attention:} API change from XForms - upstream was fl\_get\_dial\_bounds(pObject, 
minbound, maxbound)



\textbf{Status:} Untested + NoDoc + NoDemo = NOT OK



    \end{boxedminipage}

    \label{xformslib:library:fl_set_dial_step}
    \index{xformslib \textit{(package)}!xformslib.library \textit{(module)}!xformslib.library.fl\_set\_dial\_step \textit{(function)}}

    \vspace{0.5ex}

\hspace{.8\funcindent}\begin{boxedminipage}{\funcwidth}

    \raggedright \textbf{fl\_set\_dial\_step}(\textit{pObject}, \textit{value})

    \vspace{-1.5ex}

    \rule{\textwidth}{0.5\fboxrule}
\setlength{\parskip}{2ex}
\setlength{\parskip}{1ex}
      \textbf{Parameters}
      \vspace{-1ex}

      \begin{quote}
        \begin{Ventry}{xxxxxxx}

          \item[pObject]

          pointer to object ({\textless}pointer to 
          xfdata.FL\_OBJECT{\textgreater})

        \end{Ventry}

      \end{quote}

\textbf{Status:} Untested + NoDoc + NoDemo = NOT OK



    \end{boxedminipage}

    \label{xformslib:library:fl_set_dial_return}
    \index{xformslib \textit{(package)}!xformslib.library \textit{(module)}!xformslib.library.fl\_set\_dial\_return \textit{(function)}}

    \vspace{0.5ex}

\hspace{.8\funcindent}\begin{boxedminipage}{\funcwidth}

    \raggedright \textbf{fl\_set\_dial\_return}(\textit{pObject}, \textit{value})

    \vspace{-1.5ex}

    \rule{\textwidth}{0.5\fboxrule}
\setlength{\parskip}{2ex}
\setlength{\parskip}{1ex}
      \textbf{Parameters}
      \vspace{-1ex}

      \begin{quote}
        \begin{Ventry}{xxxxxxx}

          \item[pObject]

          pointer to object ({\textless}pointer to 
          xfdata.FL\_OBJECT{\textgreater})

        \end{Ventry}

      \end{quote}

\textbf{Status:} Untested + NoDoc + NoDemo = NOT OK



    \end{boxedminipage}

    \label{xformslib:library:fl_set_dial_angles}
    \index{xformslib \textit{(package)}!xformslib.library \textit{(module)}!xformslib.library.fl\_set\_dial\_angles \textit{(function)}}

    \vspace{0.5ex}

\hspace{.8\funcindent}\begin{boxedminipage}{\funcwidth}

    \raggedright \textbf{fl\_set\_dial\_angles}(\textit{pObject}, \textit{angmin}, \textit{angmax})

    \vspace{-1.5ex}

    \rule{\textwidth}{0.5\fboxrule}
\setlength{\parskip}{2ex}
\setlength{\parskip}{1ex}
      \textbf{Parameters}
      \vspace{-1ex}

      \begin{quote}
        \begin{Ventry}{xxxxxxx}

          \item[pObject]

          pointer to object ({\textless}pointer to 
          xfdata.FL\_OBJECT{\textgreater})

        \end{Ventry}

      \end{quote}

\textbf{Status:} Tested + NoDoc + Demo = OK



    \end{boxedminipage}

    \label{xformslib:library:fl_set_dial_cross}
    \index{xformslib \textit{(package)}!xformslib.library \textit{(module)}!xformslib.library.fl\_set\_dial\_cross \textit{(function)}}

    \vspace{0.5ex}

\hspace{.8\funcindent}\begin{boxedminipage}{\funcwidth}

    \raggedright \textbf{fl\_set\_dial\_cross}(\textit{pObject}, \textit{flag})

    \vspace{-1.5ex}

    \rule{\textwidth}{0.5\fboxrule}
\setlength{\parskip}{2ex}
\setlength{\parskip}{1ex}
      \textbf{Parameters}
      \vspace{-1ex}

      \begin{quote}
        \begin{Ventry}{xxxxxxx}

          \item[pObject]

          pointer to object ({\textless}pointer to 
          xfdata.FL\_OBJECT{\textgreater})

        \end{Ventry}

      \end{quote}

\textbf{Status:} Untested + NoDoc + NoDemo = NOT OK



    \end{boxedminipage}

    \label{xformslib:library:fl_set_dial_cross}
    \index{xformslib \textit{(package)}!xformslib.library \textit{(module)}!xformslib.library.fl\_set\_dial\_cross \textit{(function)}}

    \vspace{0.5ex}

\hspace{.8\funcindent}\begin{boxedminipage}{\funcwidth}

    \raggedright \textbf{fl\_set\_dial\_crossover}(\textit{pObject}, \textit{flag})

    \vspace{-1.5ex}

    \rule{\textwidth}{0.5\fboxrule}
\setlength{\parskip}{2ex}
\setlength{\parskip}{1ex}
      \textbf{Parameters}
      \vspace{-1ex}

      \begin{quote}
        \begin{Ventry}{xxxxxxx}

          \item[pObject]

          pointer to object ({\textless}pointer to 
          xfdata.FL\_OBJECT{\textgreater})

        \end{Ventry}

      \end{quote}

\textbf{Status:} Untested + NoDoc + NoDemo = NOT OK



    \end{boxedminipage}

    \label{xformslib:library:fl_set_dial_direction}
    \index{xformslib \textit{(package)}!xformslib.library \textit{(module)}!xformslib.library.fl\_set\_dial\_direction \textit{(function)}}

    \vspace{0.5ex}

\hspace{.8\funcindent}\begin{boxedminipage}{\funcwidth}

    \raggedright \textbf{fl\_set\_dial\_direction}(\textit{pObject}, \textit{directn})

    \vspace{-1.5ex}

    \rule{\textwidth}{0.5\fboxrule}
\setlength{\parskip}{2ex}
\setlength{\parskip}{1ex}
      \textbf{Parameters}
      \vspace{-1ex}

      \begin{quote}
        \begin{Ventry}{xxxxxxx}

          \item[pObject]

          pointer to object ({\textless}pointer to 
          xfdata.FL\_OBJECT{\textgreater})

        \end{Ventry}

      \end{quote}

\textbf{Status:} Tested + NoDoc + Demo = OK



    \end{boxedminipage}

    \label{xformslib:library:fl_get_dirlist}
    \index{xformslib \textit{(package)}!xformslib.library \textit{(module)}!xformslib.library.fl\_get\_dirlist \textit{(function)}}

    \vspace{0.5ex}

\hspace{.8\funcindent}\begin{boxedminipage}{\funcwidth}

    \raggedright \textbf{fl\_get\_dirlist}(\textit{directory}, \textit{pattern}, \textit{rescan})

    \vspace{-1.5ex}

    \rule{\textwidth}{0.5\fboxrule}
\setlength{\parskip}{2ex}
\setlength{\parskip}{1ex}
      \textbf{Return Value}
    \vspace{-1ex}

      \begin{quote}
      pDirList, n

      \end{quote}

\textbf{Attention:} API change from XForms - upstream was fl\_get\_dirlist(directory, pattern, 
n, rescan)



\textbf{Status:} Untested + NoDoc + NoDemo = NOT OK



    \end{boxedminipage}

    \label{xformslib:library:fl_set_dirlist_filter}
    \index{xformslib \textit{(package)}!xformslib.library \textit{(module)}!xformslib.library.fl\_set\_dirlist\_filter \textit{(function)}}

    \vspace{0.5ex}

\hspace{.8\funcindent}\begin{boxedminipage}{\funcwidth}

    \raggedright \textbf{fl\_set\_dirlist\_filter}(\textit{py\_DirFilter})

    \vspace{-1.5ex}

    \rule{\textwidth}{0.5\fboxrule}
\setlength{\parskip}{2ex}
\setlength{\parskip}{1ex}
      \textbf{Return Value}
    \vspace{-1ex}

      \begin{quote}
      dirlist\_filter func

      \end{quote}

\textbf{Status:} Untested + NoDoc + NoDemo = NOT OK



    \end{boxedminipage}

    \label{xformslib:library:fl_set_dirlist_sort}
    \index{xformslib \textit{(package)}!xformslib.library \textit{(module)}!xformslib.library.fl\_set\_dirlist\_sort \textit{(function)}}

    \vspace{0.5ex}

\hspace{.8\funcindent}\begin{boxedminipage}{\funcwidth}

    \raggedright \textbf{fl\_set\_dirlist\_sort}(\textit{method})

    \vspace{-1.5ex}

    \rule{\textwidth}{0.5\fboxrule}
\setlength{\parskip}{2ex}
\setlength{\parskip}{1ex}
      \textbf{Return Value}
    \vspace{-1ex}

      \begin{quote}
      num

      \end{quote}

\textbf{Status:} Untested + NoDoc + NoDemo = NOT OK



    \end{boxedminipage}

    \label{xformslib:library:fl_set_dirlist_filterdir}
    \index{xformslib \textit{(package)}!xformslib.library \textit{(module)}!xformslib.library.fl\_set\_dirlist\_filterdir \textit{(function)}}

    \vspace{0.5ex}

\hspace{.8\funcindent}\begin{boxedminipage}{\funcwidth}

    \raggedright \textbf{fl\_set\_dirlist\_filterdir}(\textit{yes})

    \vspace{-1.5ex}

    \rule{\textwidth}{0.5\fboxrule}
\setlength{\parskip}{2ex}
\setlength{\parskip}{1ex}
      \textbf{Return Value}
    \vspace{-1ex}

      \begin{quote}
      num

      \end{quote}

\textbf{Status:} Untested + NoDoc + NoDemo = NOT OK



    \end{boxedminipage}

    \label{xformslib:library:fl_free_dirlist}
    \index{xformslib \textit{(package)}!xformslib.library \textit{(module)}!xformslib.library.fl\_free\_dirlist \textit{(function)}}

    \vspace{0.5ex}

\hspace{.8\funcindent}\begin{boxedminipage}{\funcwidth}

    \raggedright \textbf{fl\_free\_dirlist}(\textit{pDirList})

    \vspace{-1.5ex}

    \rule{\textwidth}{0.5\fboxrule}
\setlength{\parskip}{2ex}
\setlength{\parskip}{1ex}
\textbf{Status:} Untested + NoDoc + NoDemo = NOT OK



    \end{boxedminipage}

    \label{xformslib:library:fl_free_all_dirlist}
    \index{xformslib \textit{(package)}!xformslib.library \textit{(module)}!xformslib.library.fl\_free\_all\_dirlist \textit{(function)}}

    \vspace{0.5ex}

\hspace{.8\funcindent}\begin{boxedminipage}{\funcwidth}

    \raggedright \textbf{fl\_free\_all\_dirlist}()

    \vspace{-1.5ex}

    \rule{\textwidth}{0.5\fboxrule}
\setlength{\parskip}{2ex}
\setlength{\parskip}{1ex}
\textbf{Status:} Untested + NoDoc + NoDemo = NOT OK



    \end{boxedminipage}

    \label{xformslib:library:fl_is_valid_dir}
    \index{xformslib \textit{(package)}!xformslib.library \textit{(module)}!xformslib.library.fl\_is\_valid\_dir \textit{(function)}}

    \vspace{0.5ex}

\hspace{.8\funcindent}\begin{boxedminipage}{\funcwidth}

    \raggedright \textbf{fl\_is\_valid\_dir}(\textit{name})

    \vspace{-1.5ex}

    \rule{\textwidth}{0.5\fboxrule}
\setlength{\parskip}{2ex}
\setlength{\parskip}{1ex}
      \textbf{Return Value}
    \vspace{-1ex}

      \begin{quote}
      num

      \end{quote}

\textbf{Status:} Untested + NoDoc + NoDemo = NOT OK



    \end{boxedminipage}

    \label{xformslib:library:fl_fmtime}
    \index{xformslib \textit{(package)}!xformslib.library \textit{(module)}!xformslib.library.fl\_fmtime \textit{(function)}}

    \vspace{0.5ex}

\hspace{.8\funcindent}\begin{boxedminipage}{\funcwidth}

    \raggedright \textbf{fl\_fmtime}(\textit{timestr})

    \vspace{-1.5ex}

    \rule{\textwidth}{0.5\fboxrule}
\setlength{\parskip}{2ex}
\setlength{\parskip}{1ex}
      \textbf{Return Value}
    \vspace{-1ex}

      \begin{quote}
      num

      \end{quote}

\textbf{Status:} Untested + NoDoc + NoDemo = NOT OK



    \end{boxedminipage}

    \label{xformslib:library:fl_fix_dirname}
    \index{xformslib \textit{(package)}!xformslib.library \textit{(module)}!xformslib.library.fl\_fix\_dirname \textit{(function)}}

    \vspace{0.5ex}

\hspace{.8\funcindent}\begin{boxedminipage}{\funcwidth}

    \raggedright \textbf{fl\_fix\_dirname}(\textit{directory})

    \vspace{-1.5ex}

    \rule{\textwidth}{0.5\fboxrule}
\setlength{\parskip}{2ex}
\setlength{\parskip}{1ex}
      \textbf{Return Value}
    \vspace{-1ex}

      \begin{quote}
      dirname string

      \end{quote}

\textbf{Status:} Untested + NoDoc + NoDemo = NOT OK



    \end{boxedminipage}

    \label{xformslib:library:flps_init}
    \index{xformslib \textit{(package)}!xformslib.library \textit{(module)}!xformslib.library.flps\_init \textit{(function)}}

    \vspace{0.5ex}

\hspace{.8\funcindent}\begin{boxedminipage}{\funcwidth}

    \raggedright \textbf{flps\_init}()

    \vspace{-1.5ex}

    \rule{\textwidth}{0.5\fboxrule}
\setlength{\parskip}{2ex}
\setlength{\parskip}{1ex}
      \textbf{Return Value}
    \vspace{-1ex}

      \begin{quote}
      flps\_control class

      \end{quote}

\textbf{Status:} Untested + NoDoc + NoDemo = NOT OK



    \end{boxedminipage}

    \label{xformslib:library:fl_object_ps_dump}
    \index{xformslib \textit{(package)}!xformslib.library \textit{(module)}!xformslib.library.fl\_object\_ps\_dump \textit{(function)}}

    \vspace{0.5ex}

\hspace{.8\funcindent}\begin{boxedminipage}{\funcwidth}

    \raggedright \textbf{fl\_object\_ps\_dump}(\textit{pObject}, \textit{fname})

    \vspace{-1.5ex}

    \rule{\textwidth}{0.5\fboxrule}
\setlength{\parskip}{2ex}
\setlength{\parskip}{1ex}
      \textbf{Parameters}
      \vspace{-1ex}

      \begin{quote}
        \begin{Ventry}{xxxxxxx}

          \item[pObject]

          pointer to object ({\textless}pointer to 
          xfdata.FL\_OBJECT{\textgreater})

        \end{Ventry}

      \end{quote}

      \textbf{Return Value}
    \vspace{-1ex}

      \begin{quote}
      num

      \end{quote}

\textbf{Status:} Untested + NoDoc + NoDemo = NOT OK



    \end{boxedminipage}

    \label{xformslib:library:fl_addto_formbrowser}
    \index{xformslib \textit{(package)}!xformslib.library \textit{(module)}!xformslib.library.fl\_addto\_formbrowser \textit{(function)}}

    \vspace{0.5ex}

\hspace{.8\funcindent}\begin{boxedminipage}{\funcwidth}

    \raggedright \textbf{fl\_addto\_formbrowser}(\textit{pObject}, \textit{pForm})

    \vspace{-1.5ex}

    \rule{\textwidth}{0.5\fboxrule}
\setlength{\parskip}{2ex}
\setlength{\parskip}{1ex}
      \textbf{Parameters}
      \vspace{-1ex}

      \begin{quote}
        \begin{Ventry}{xxxxxxx}

          \item[pObject]

          pointer to object ({\textless}pointer to 
          xfdata.FL\_OBJECT{\textgreater})

        \end{Ventry}

      \end{quote}

      \textbf{Return Value}
    \vspace{-1ex}

      \begin{quote}
      num

      \end{quote}

\textbf{Status:} Untested + NoDoc + NoDemo = NOT OK



    \end{boxedminipage}

    \label{xformslib:library:fl_delete_formbrowser_bynumber}
    \index{xformslib \textit{(package)}!xformslib.library \textit{(module)}!xformslib.library.fl\_delete\_formbrowser\_bynumber \textit{(function)}}

    \vspace{0.5ex}

\hspace{.8\funcindent}\begin{boxedminipage}{\funcwidth}

    \raggedright \textbf{fl\_delete\_formbrowser\_bynumber}(\textit{pObject}, \textit{num})

    \vspace{-1.5ex}

    \rule{\textwidth}{0.5\fboxrule}
\setlength{\parskip}{2ex}
\setlength{\parskip}{1ex}
      \textbf{Parameters}
      \vspace{-1ex}

      \begin{quote}
        \begin{Ventry}{xxxxxxx}

          \item[pObject]

          pointer to object ({\textless}pointer to 
          xfdata.FL\_OBJECT{\textgreater})

        \end{Ventry}

      \end{quote}

      \textbf{Return Value}
    \vspace{-1ex}

      \begin{quote}
      pForm

      \end{quote}

\textbf{Status:} Untested + NoDoc + NoDemo = NOT OK



    \end{boxedminipage}

    \label{xformslib:library:fl_delete_formbrowser}
    \index{xformslib \textit{(package)}!xformslib.library \textit{(module)}!xformslib.library.fl\_delete\_formbrowser \textit{(function)}}

    \vspace{0.5ex}

\hspace{.8\funcindent}\begin{boxedminipage}{\funcwidth}

    \raggedright \textbf{fl\_delete\_formbrowser}(\textit{pObject}, \textit{pForm})

    \vspace{-1.5ex}

    \rule{\textwidth}{0.5\fboxrule}
\setlength{\parskip}{2ex}
\setlength{\parskip}{1ex}
      \textbf{Parameters}
      \vspace{-1ex}

      \begin{quote}
        \begin{Ventry}{xxxxxxx}

          \item[pObject]

          object the formbrowser belongs to ({\textless}pointer to 
          xfdata.FL\_OBJECT{\textgreater})

          \item[pForm]

          form candidate to deletion ({\textless}pointer to 
          xfdata.FL\_FORM{\textgreater})

        \end{Ventry}

      \end{quote}

      \textbf{Return Value}
    \vspace{-1ex}

      \begin{quote}
      num

      \end{quote}

\textbf{Status:} Untested + NoDoc + NoDemo = NOT OK



    \end{boxedminipage}

    \label{xformslib:library:fl_replace_formbrowser}
    \index{xformslib \textit{(package)}!xformslib.library \textit{(module)}!xformslib.library.fl\_replace\_formbrowser \textit{(function)}}

    \vspace{0.5ex}

\hspace{.8\funcindent}\begin{boxedminipage}{\funcwidth}

    \raggedright \textbf{fl\_replace\_formbrowser}(\textit{pObject}, \textit{num}, \textit{pForm})

    \vspace{-1.5ex}

    \rule{\textwidth}{0.5\fboxrule}
\setlength{\parskip}{2ex}
\setlength{\parskip}{1ex}
      \textbf{Parameters}
      \vspace{-1ex}

      \begin{quote}
        \begin{Ventry}{xxxxxxx}

          \item[pObject]

          formbrowser object ({\textless}pointer to 
          xfdata.FL\_OBJECT{\textgreater})

          \item[num]

          formbrowser number to be replaced ({\textless}int{\textgreater})

          \item[pForm]

          form used as replacement ({\textless}pointer to 
          xfdata.FL\_FORM{\textgreater})

        \end{Ventry}

      \end{quote}

\textbf{Status:} Untested + NoDoc + NoDemo = NOT OK



    \end{boxedminipage}

    \label{xformslib:library:fl_insert_formbrowser}
    \index{xformslib \textit{(package)}!xformslib.library \textit{(module)}!xformslib.library.fl\_insert\_formbrowser \textit{(function)}}

    \vspace{0.5ex}

\hspace{.8\funcindent}\begin{boxedminipage}{\funcwidth}

    \raggedright \textbf{fl\_insert\_formbrowser}(\textit{pObject}, \textit{line}, \textit{pForm})

    \vspace{-1.5ex}

    \rule{\textwidth}{0.5\fboxrule}
\setlength{\parskip}{2ex}
\setlength{\parskip}{1ex}
      \textbf{Parameters}
      \vspace{-1ex}

      \begin{quote}
        \begin{Ventry}{xxxxxxx}

          \item[pObject]

          formbrowser object ({\textless}pointer to 
          xfdata.FL\_OBJECT{\textgreater})

          \item[line]

          line after which new form is inserted

          \item[pForm]

          new form to insert ({\textless}pointer to 
          xfdata.FL\_FORM{\textgreater})

        \end{Ventry}

      \end{quote}

      \textbf{Return Value}
    \vspace{-1ex}

      \begin{quote}
      num

      \end{quote}

\textbf{Status:} Untested + NoDoc + NoDemo = NOT OK



    \end{boxedminipage}

    \label{xformslib:library:fl_get_formbrowser_area}
    \index{xformslib \textit{(package)}!xformslib.library \textit{(module)}!xformslib.library.fl\_get\_formbrowser\_area \textit{(function)}}

    \vspace{0.5ex}

\hspace{.8\funcindent}\begin{boxedminipage}{\funcwidth}

    \raggedright \textbf{fl\_get\_formbrowser\_area}(\textit{pObject})

    \vspace{-1.5ex}

    \rule{\textwidth}{0.5\fboxrule}
\setlength{\parskip}{2ex}
\setlength{\parskip}{1ex}
      \textbf{Parameters}
      \vspace{-1ex}

      \begin{quote}
        \begin{Ventry}{xxxxxxx}

          \item[pObject]

          formbrowser object ({\textless}pointer to 
          xfdata.FL\_OBJECT{\textgreater})

        \end{Ventry}

      \end{quote}

      \textbf{Return Value}
    \vspace{-1ex}

      \begin{quote}
      num., x, y, w, h

      \end{quote}

\textbf{Attention:} API change from XForms - upstream was fl\_get\_formbrowser\_area(pObject, 
x, y, w, h)



\textbf{Status:} Untested + NoDoc + NoDemo = NOT OK



    \end{boxedminipage}

    \label{xformslib:library:fl_set_formbrowser_scroll}
    \index{xformslib \textit{(package)}!xformslib.library \textit{(module)}!xformslib.library.fl\_set\_formbrowser\_scroll \textit{(function)}}

    \vspace{0.5ex}

\hspace{.8\funcindent}\begin{boxedminipage}{\funcwidth}

    \raggedright \textbf{fl\_set\_formbrowser\_scroll}(\textit{pObject}, \textit{how})

    \vspace{-1.5ex}

    \rule{\textwidth}{0.5\fboxrule}
\setlength{\parskip}{2ex}
\setlength{\parskip}{1ex}
      \textbf{Parameters}
      \vspace{-1ex}

      \begin{quote}
        \begin{Ventry}{xxxxxxx}

          \item[pObject]

          formbrowser object ({\textless}pointer to 
          xfdata.FL\_OBJECT{\textgreater})

          \item[how]

          ?

        \end{Ventry}

      \end{quote}

\textbf{Status:} Untested + NoDoc + NoDemo = NOT OK



    \end{boxedminipage}

    \label{xformslib:library:fl_set_formbrowser_hscrollbar}
    \index{xformslib \textit{(package)}!xformslib.library \textit{(module)}!xformslib.library.fl\_set\_formbrowser\_hscrollbar \textit{(function)}}

    \vspace{0.5ex}

\hspace{.8\funcindent}\begin{boxedminipage}{\funcwidth}

    \raggedright \textbf{fl\_set\_formbrowser\_hscrollbar}(\textit{pObject}, \textit{how})

    \vspace{-1.5ex}

    \rule{\textwidth}{0.5\fboxrule}
\setlength{\parskip}{2ex}
\setlength{\parskip}{1ex}
      \textbf{Parameters}
      \vspace{-1ex}

      \begin{quote}
        \begin{Ventry}{xxxxxxx}

          \item[pObject]

          formbrowser object ({\textless}pointer to 
          xfdata.FL\_OBJECT{\textgreater})

          \item[how]

          ?

        \end{Ventry}

      \end{quote}

\textbf{Status:} Untested + NoDoc + NoDemo = NOT OK



    \end{boxedminipage}

    \label{xformslib:library:fl_set_formbrowser_vscrollbar}
    \index{xformslib \textit{(package)}!xformslib.library \textit{(module)}!xformslib.library.fl\_set\_formbrowser\_vscrollbar \textit{(function)}}

    \vspace{0.5ex}

\hspace{.8\funcindent}\begin{boxedminipage}{\funcwidth}

    \raggedright \textbf{fl\_set\_formbrowser\_vscrollbar}(\textit{pObject}, \textit{how})

    \vspace{-1.5ex}

    \rule{\textwidth}{0.5\fboxrule}
\setlength{\parskip}{2ex}
\setlength{\parskip}{1ex}
      \textbf{Parameters}
      \vspace{-1ex}

      \begin{quote}
        \begin{Ventry}{xxxxxxx}

          \item[pObject]

          formbrowser object ({\textless}pointer to 
          xfdata.FL\_OBJECT{\textgreater})

          \item[how]

          ?

        \end{Ventry}

      \end{quote}

\textbf{Status:} Untested + NoDoc + NoDemo = NOT OK



    \end{boxedminipage}

    \label{xformslib:library:fl_get_formbrowser_topform}
    \index{xformslib \textit{(package)}!xformslib.library \textit{(module)}!xformslib.library.fl\_get\_formbrowser\_topform \textit{(function)}}

    \vspace{0.5ex}

\hspace{.8\funcindent}\begin{boxedminipage}{\funcwidth}

    \raggedright \textbf{fl\_get\_formbrowser\_topform}(\textit{pObject})

    \vspace{-1.5ex}

    \rule{\textwidth}{0.5\fboxrule}
\setlength{\parskip}{2ex}
\setlength{\parskip}{1ex}
      \textbf{Parameters}
      \vspace{-1ex}

      \begin{quote}
        \begin{Ventry}{xxxxxxx}

          \item[pObject]

          formbrowser object ({\textless}pointer to 
          xfdata.FL\_OBJECT{\textgreater})

        \end{Ventry}

      \end{quote}

      \textbf{Return Value}
    \vspace{-1ex}

      \begin{quote}
      pForm

      \end{quote}

\textbf{Status:} Untested + NoDoc + NoDemo = NOT OK



    \end{boxedminipage}

    \label{xformslib:library:fl_set_formbrowser_topform}
    \index{xformslib \textit{(package)}!xformslib.library \textit{(module)}!xformslib.library.fl\_set\_formbrowser\_topform \textit{(function)}}

    \vspace{0.5ex}

\hspace{.8\funcindent}\begin{boxedminipage}{\funcwidth}

    \raggedright \textbf{fl\_set\_formbrowser\_topform}(\textit{pObject}, \textit{pForm})

    \vspace{-1.5ex}

    \rule{\textwidth}{0.5\fboxrule}
\setlength{\parskip}{2ex}
\setlength{\parskip}{1ex}
      \textbf{Parameters}
      \vspace{-1ex}

      \begin{quote}
        \begin{Ventry}{xxxxxxx}

          \item[pObject]

          formbrowser object ({\textless}pointer to 
          xfdata.FL\_OBJECT{\textgreater})

          \item[pForm]

          form ({\textless}pointer to xfdata.FL\_FORM{\textgreater})

        \end{Ventry}

      \end{quote}

      \textbf{Return Value}
    \vspace{-1ex}

      \begin{quote}
      num

      \end{quote}

\textbf{Status:} Untested + NoDoc + NoDemo = NOT OK



    \end{boxedminipage}

    \label{xformslib:library:fl_set_formbrowser_topform_bynumber}
    \index{xformslib \textit{(package)}!xformslib.library \textit{(module)}!xformslib.library.fl\_set\_formbrowser\_topform\_bynumber \textit{(function)}}

    \vspace{0.5ex}

\hspace{.8\funcindent}\begin{boxedminipage}{\funcwidth}

    \raggedright \textbf{fl\_set\_formbrowser\_topform\_bynumber}(\textit{pObject}, \textit{num})

    \vspace{-1.5ex}

    \rule{\textwidth}{0.5\fboxrule}
\setlength{\parskip}{2ex}
\setlength{\parskip}{1ex}
      \textbf{Parameters}
      \vspace{-1ex}

      \begin{quote}
        \begin{Ventry}{xxxxxxx}

          \item[pObject]

          formbrowser object ({\textless}pointer to 
          xfdata.FL\_OBJECT{\textgreater})

          \item[num]

          ?

        \end{Ventry}

      \end{quote}

      \textbf{Return Value}
    \vspace{-1ex}

      \begin{quote}
      pForm

      \end{quote}

\textbf{Status:} Untested + NoDoc + NoDemo = NOT OK



    \end{boxedminipage}

    \label{xformslib:library:fl_set_formbrowser_xoffset}
    \index{xformslib \textit{(package)}!xformslib.library \textit{(module)}!xformslib.library.fl\_set\_formbrowser\_xoffset \textit{(function)}}

    \vspace{0.5ex}

\hspace{.8\funcindent}\begin{boxedminipage}{\funcwidth}

    \raggedright \textbf{fl\_set\_formbrowser\_xoffset}(\textit{pObject}, \textit{offset})

    \vspace{-1.5ex}

    \rule{\textwidth}{0.5\fboxrule}
\setlength{\parskip}{2ex}
    Scrolls within a formbrowser in horizontal direction.

\setlength{\parskip}{1ex}
      \textbf{Parameters}
      \vspace{-1ex}

      \begin{quote}
        \begin{Ventry}{xxxxxxx}

          \item[pObject]

          formbrowser object ({\textless}pointer to 
          xfdata.FL\_OBJECT{\textgreater})

          \item[offset]

          positive number, measuring in pixels the offset from the the 
          natural position from the left ({\textless}int{\textgreater})

        \end{Ventry}

      \end{quote}

      \textbf{Return Value}
    \vspace{-1ex}

      \begin{quote}
      num

      \end{quote}

\textbf{Status:} Untested + NoDoc + NoDemo = NOT OK



    \end{boxedminipage}

    \label{xformslib:library:fl_set_formbrowser_yoffset}
    \index{xformslib \textit{(package)}!xformslib.library \textit{(module)}!xformslib.library.fl\_set\_formbrowser\_yoffset \textit{(function)}}

    \vspace{0.5ex}

\hspace{.8\funcindent}\begin{boxedminipage}{\funcwidth}

    \raggedright \textbf{fl\_set\_formbrowser\_yoffset}(\textit{pObject}, \textit{offset})

    \vspace{-1.5ex}

    \rule{\textwidth}{0.5\fboxrule}
\setlength{\parskip}{2ex}
    Scrolls within a formbrowser in vertical direction.

\setlength{\parskip}{1ex}
      \textbf{Parameters}
      \vspace{-1ex}

      \begin{quote}
        \begin{Ventry}{xxxxxxx}

          \item[pObject]

          formbrowser object ({\textless}pointer to 
          xfdata.FL\_OBJECT{\textgreater})

          \item[offset]

          positive number, measuring in pixels the offset from the the 
          natural position from the top ({\textless}int{\textgreater})

        \end{Ventry}

      \end{quote}

      \textbf{Return Value}
    \vspace{-1ex}

      \begin{quote}
      num

      \end{quote}

\textbf{Status:} Untested + NoDoc + NoDemo = NOT OK



    \end{boxedminipage}

    \label{xformslib:library:fl_get_formbrowser_xoffset}
    \index{xformslib \textit{(package)}!xformslib.library \textit{(module)}!xformslib.library.fl\_get\_formbrowser\_xoffset \textit{(function)}}

    \vspace{0.5ex}

\hspace{.8\funcindent}\begin{boxedminipage}{\funcwidth}

    \raggedright \textbf{fl\_get\_formbrowser\_xoffset}(\textit{pObject})

    \vspace{-1.5ex}

    \rule{\textwidth}{0.5\fboxrule}
\setlength{\parskip}{2ex}
    Returns the current horizontal offset from left in pixel of a 
    formbrowser.

\setlength{\parskip}{1ex}
      \textbf{Parameters}
      \vspace{-1ex}

      \begin{quote}
        \begin{Ventry}{xxxxxxx}

          \item[pObject]

          formbrowser object ({\textless}pointer to 
          xfdata.FL\_OBJECT{\textgreater})

        \end{Ventry}

      \end{quote}

      \textbf{Return Value}
    \vspace{-1ex}

      \begin{quote}
      num

      \end{quote}

\textbf{Status:} Untested + NoDoc + NoDemo = NOT OK



    \end{boxedminipage}

    \label{xformslib:library:fl_get_formbrowser_yoffset}
    \index{xformslib \textit{(package)}!xformslib.library \textit{(module)}!xformslib.library.fl\_get\_formbrowser\_yoffset \textit{(function)}}

    \vspace{0.5ex}

\hspace{.8\funcindent}\begin{boxedminipage}{\funcwidth}

    \raggedright \textbf{fl\_get\_formbrowser\_yoffset}(\textit{pObject})

    \vspace{-1.5ex}

    \rule{\textwidth}{0.5\fboxrule}
\setlength{\parskip}{2ex}
    Returns the current vertical offset from top in pixel of a formbrowser.

\setlength{\parskip}{1ex}
      \textbf{Parameters}
      \vspace{-1ex}

      \begin{quote}
        \begin{Ventry}{xxxxxxx}

          \item[pObject]

          formbrowser object ({\textless}pointer to 
          xfdata.FL\_OBJECT{\textgreater})

        \end{Ventry}

      \end{quote}

      \textbf{Return Value}
    \vspace{-1ex}

      \begin{quote}
      num

      \end{quote}

\textbf{Status:} Untested + NoDoc + NoDemo = NOT OK



    \end{boxedminipage}

    \label{xformslib:library:fl_find_formbrowser_form_number}
    \index{xformslib \textit{(package)}!xformslib.library \textit{(module)}!xformslib.library.fl\_find\_formbrowser\_form\_number \textit{(function)}}

    \vspace{0.5ex}

\hspace{.8\funcindent}\begin{boxedminipage}{\funcwidth}

    \raggedright \textbf{fl\_find\_formbrowser\_form\_number}(\textit{pObject}, \textit{pForm})

    \vspace{-1.5ex}

    \rule{\textwidth}{0.5\fboxrule}
\setlength{\parskip}{2ex}
\setlength{\parskip}{1ex}
      \textbf{Parameters}
      \vspace{-1ex}

      \begin{quote}
        \begin{Ventry}{xxxxxxx}

          \item[pObject]

          formbrowser object ({\textless}pointer to 
          xfdata.FL\_OBJECT{\textgreater})

          \item[pForm]

          form candidate to be found

        \end{Ventry}

      \end{quote}

      \textbf{Return Value}
    \vspace{-1ex}

      \begin{quote}
      num

      \end{quote}

\textbf{Status:} Untested + NoDoc + NoDemo = NOT OK



    \end{boxedminipage}

    \label{xformslib:library:fl_add_formbrowser}
    \index{xformslib \textit{(package)}!xformslib.library \textit{(module)}!xformslib.library.fl\_add\_formbrowser \textit{(function)}}

    \vspace{0.5ex}

\hspace{.8\funcindent}\begin{boxedminipage}{\funcwidth}

    \raggedright \textbf{fl\_add\_formbrowser}(\textit{frmbrwstype}, \textit{x}, \textit{y}, \textit{w}, \textit{h}, \textit{label})

    \vspace{-1.5ex}

    \rule{\textwidth}{0.5\fboxrule}
\setlength{\parskip}{2ex}
    Adds a formbrowser object.

\setlength{\parskip}{1ex}
      \textbf{Parameters}
      \vspace{-1ex}

      \begin{quote}
        \begin{Ventry}{xxxxxxxxxxx}

          \item[frmbrwstype]

          type of formbrowser to be added ({\textless}int{\textgreater})

            {\it (type=xfdata.FL\_NORMAL\_FORMBROWSER)}

          \item[x]

          horizontal position (upper-left corner) 
          ({\textless}int{\textgreater})

          \item[y]

          vertical position (upper-left corner) 
          ({\textless}int{\textgreater})

          \item[w]

          width in coord units ({\textless}int{\textgreater})

          \item[h]

          height in coord units ({\textless}int{\textgreater})

          \item[label]

          text label of formbrowser ({\textless}string{\textgreater})

        \end{Ventry}

      \end{quote}

      \textbf{Return Value}
    \vspace{-1ex}

      \begin{quote}
      pObject

      \end{quote}

\textbf{Status:} Untested + NoDoc + NoDemo = NOT OK



    \end{boxedminipage}

    \label{xformslib:library:fl_get_formbrowser_numforms}
    \index{xformslib \textit{(package)}!xformslib.library \textit{(module)}!xformslib.library.fl\_get\_formbrowser\_numforms \textit{(function)}}

    \vspace{0.5ex}

\hspace{.8\funcindent}\begin{boxedminipage}{\funcwidth}

    \raggedright \textbf{fl\_get\_formbrowser\_numforms}(\textit{pObject})

    \vspace{-1.5ex}

    \rule{\textwidth}{0.5\fboxrule}
\setlength{\parskip}{2ex}
\setlength{\parskip}{1ex}
      \textbf{Parameters}
      \vspace{-1ex}

      \begin{quote}
        \begin{Ventry}{xxxxxxx}

          \item[pObject]

          formbrowser object ({\textless}pointer to 
          xfdata.FL\_OBJECT{\textgreater})

        \end{Ventry}

      \end{quote}

      \textbf{Return Value}
    \vspace{-1ex}

      \begin{quote}
      forms num

      \end{quote}

\textbf{Status:} Untested + NoDoc + NoDemo = NOT OK



    \end{boxedminipage}

    \label{xformslib:library:fl_get_formbrowser_form}
    \index{xformslib \textit{(package)}!xformslib.library \textit{(module)}!xformslib.library.fl\_get\_formbrowser\_form \textit{(function)}}

    \vspace{0.5ex}

\hspace{.8\funcindent}\begin{boxedminipage}{\funcwidth}

    \raggedright \textbf{fl\_get\_formbrowser\_form}(\textit{pObject}, \textit{num})

    \vspace{-1.5ex}

    \rule{\textwidth}{0.5\fboxrule}
\setlength{\parskip}{2ex}
\setlength{\parskip}{1ex}
      \textbf{Parameters}
      \vspace{-1ex}

      \begin{quote}
        \begin{Ventry}{xxxxxxx}

          \item[pObject]

          formbrowser object ({\textless}pointer to 
          xfdata.FL\_OBJECT{\textgreater})

        \end{Ventry}

      \end{quote}

      \textbf{Return Value}
    \vspace{-1ex}

      \begin{quote}
      pForm

      \end{quote}

\textbf{Status:} Untested + NoDoc + NoDemo = NOT OK



    \end{boxedminipage}

    \label{xformslib:library:fl_add_frame}
    \index{xformslib \textit{(package)}!xformslib.library \textit{(module)}!xformslib.library.fl\_add\_frame \textit{(function)}}

    \vspace{0.5ex}

\hspace{.8\funcindent}\begin{boxedminipage}{\funcwidth}

    \raggedright \textbf{fl\_add\_frame}(\textit{frametype}, \textit{x}, \textit{y}, \textit{w}, \textit{h}, \textit{label})

    \vspace{-1.5ex}

    \rule{\textwidth}{0.5\fboxrule}
\setlength{\parskip}{2ex}
    Adds a frame object.

\setlength{\parskip}{1ex}
      \textbf{Parameters}
      \vspace{-1ex}

      \begin{quote}
        \begin{Ventry}{xxxxxxxxx}

          \item[frametype]

          type of frame to be added

          \item[frametype]

          [num./int] from xfdata module FL\_NO\_FRAME, FL\_UP\_FRAME, 
          FL\_DOWN\_FRAME, FL\_BORDER\_FRAME, FL\_SHADOW\_FRAME, 
          FL\_ENGRAVED\_FRAME, FL\_ROUNDED\_FRAME, FL\_EMBOSSED\_FRAME, 
          FL\_OVAL\_FRAME

          \item[x]

          horizontal position (upper-left corner)

          \item[x]

          vertical position (upper-left corner)

          \item[w]

          width in coord units

          \item[h]

          height in coord units

          \item[label]

          text label of frame

        \end{Ventry}

      \end{quote}

      \textbf{Return Value}
    \vspace{-1ex}

      \begin{quote}
      pObject

      \end{quote}

\textbf{Status:} Tested + NoDoc + Demo = OK



    \end{boxedminipage}

    \label{xformslib:library:fl_add_labelframe}
    \index{xformslib \textit{(package)}!xformslib.library \textit{(module)}!xformslib.library.fl\_add\_labelframe \textit{(function)}}

    \vspace{0.5ex}

\hspace{.8\funcindent}\begin{boxedminipage}{\funcwidth}

    \raggedright \textbf{fl\_add\_labelframe}(\textit{frametype}, \textit{x}, \textit{y}, \textit{w}, \textit{h}, \textit{label})

    \vspace{-1.5ex}

    \rule{\textwidth}{0.5\fboxrule}
\setlength{\parskip}{2ex}
    Adds a labelframe object.

\setlength{\parskip}{1ex}
      \textbf{Parameters}
      \vspace{-1ex}

      \begin{quote}
        \begin{Ventry}{xxxxxxxxx}

          \item[frametype]

          type of labelframe to be added

          \item[frametype]

          [num./int] from xfdata module FL\_NO\_FRAME, FL\_UP\_FRAME, 
          FL\_DOWN\_FRAME, FL\_BORDER\_FRAME, FL\_SHADOW\_FRAME, 
          FL\_ENGRAVED\_FRAME, FL\_ROUNDED\_FRAME, FL\_EMBOSSED\_FRAME, 
          FL\_OVAL\_FRAME

          \item[x]

          horizontal position (upper-left corner)

          \item[x]

          vertical position (upper-left corner)

          \item[w]

          width in coord units

          \item[h]

          height in coord units

          \item[label]

          text label of labelframe

        \end{Ventry}

      \end{quote}

      \textbf{Return Value}
    \vspace{-1ex}

      \begin{quote}
      pObject

      \end{quote}

\textbf{Status:} Tested + NoDoc + Demo = OK



    \end{boxedminipage}

    \label{xformslib:library:fl_add_free}
    \index{xformslib \textit{(package)}!xformslib.library \textit{(module)}!xformslib.library.fl\_add\_free \textit{(function)}}

    \vspace{0.5ex}

\hspace{.8\funcindent}\begin{boxedminipage}{\funcwidth}

    \raggedright \textbf{fl\_add\_free}(\textit{freetype}, \textit{x}, \textit{y}, \textit{w}, \textit{h}, \textit{label}, \textit{py\_HandlePtr})

    \vspace{-1.5ex}

    \rule{\textwidth}{0.5\fboxrule}
\setlength{\parskip}{2ex}
    Adds a free object.

\setlength{\parskip}{1ex}
      \textbf{Parameters}
      \vspace{-1ex}

      \begin{quote}
        \begin{Ventry}{xxxxxxxx}

          \item[freetype]

          type of free to be added

            {\it (type=[num./int] from xfdata module FL\_NORMAL\_FREE, FL\_INACTIVE\_FREE, 
FL\_INPUT\_FREE, FL\_CONTINUOUS\_FREE, FL\_ALL\_FREE, FL\_SLEEPING\_FREE)}

          \item[x]

          horizontal position (upper-left corner)

          \item[x]

          vertical position (upper-left corner)

          \item[w]

          width in coord units

          \item[h]

          height in coord units

          \item[label]

          text label of free

        \end{Ventry}

      \end{quote}

      \textbf{Return Value}
    \vspace{-1ex}

      \begin{quote}
      pObject

      \end{quote}

\textbf{Status:} Tested + NoDoc + Demo = OK



    \end{boxedminipage}

    \label{xformslib:library:fl_set_goodies_font}
    \index{xformslib \textit{(package)}!xformslib.library \textit{(module)}!xformslib.library.fl\_set\_goodies\_font \textit{(function)}}

    \vspace{0.5ex}

\hspace{.8\funcindent}\begin{boxedminipage}{\funcwidth}

    \raggedright \textbf{fl\_set\_goodies\_font}(\textit{style}, \textit{size})

    \vspace{-1.5ex}

    \rule{\textwidth}{0.5\fboxrule}
\setlength{\parskip}{2ex}
\setlength{\parskip}{1ex}
\textbf{Status:} Tested + NoDoc + Demo = OK



    \end{boxedminipage}

    \label{xformslib:library:fl_show_message}
    \index{xformslib \textit{(package)}!xformslib.library \textit{(module)}!xformslib.library.fl\_show\_message \textit{(function)}}

    \vspace{0.5ex}

\hspace{.8\funcindent}\begin{boxedminipage}{\funcwidth}

    \raggedright \textbf{fl\_show\_message}(\textit{msgtxt1}, \textit{msgtxt2}, \textit{msgtxt3})

    \vspace{-1.5ex}

    \rule{\textwidth}{0.5\fboxrule}
\setlength{\parskip}{2ex}
    Shows a message.

\setlength{\parskip}{1ex}
      \textbf{Parameters}
      \vspace{-1ex}

      \begin{quote}
        \begin{Ventry}{xxxxxxx}

          \item[msgtxt1]

          first message to show

          \item[msgtxt2]

          second message to show

          \item[msgtxt3]

          third message to show

        \end{Ventry}

      \end{quote}

\textbf{Status:} Tested + NoDoc + Demo = OK



    \end{boxedminipage}

    \label{xformslib:library:fl_show_messages}
    \index{xformslib \textit{(package)}!xformslib.library \textit{(module)}!xformslib.library.fl\_show\_messages \textit{(function)}}

    \vspace{0.5ex}

\hspace{.8\funcindent}\begin{boxedminipage}{\funcwidth}

    \raggedright \textbf{fl\_show\_messages}(\textit{p1})

    \vspace{-1.5ex}

    \rule{\textwidth}{0.5\fboxrule}
\setlength{\parskip}{2ex}
\setlength{\parskip}{1ex}
\textbf{Status:} Tested + NoDoc + Demo = OK



    \end{boxedminipage}

    \label{xformslib:library:fl_show_msg}
    \index{xformslib \textit{(package)}!xformslib.library \textit{(module)}!xformslib.library.fl\_show\_msg \textit{(function)}}

    \vspace{0.5ex}

\hspace{.8\funcindent}\begin{boxedminipage}{\funcwidth}

    \raggedright \textbf{fl\_show\_msg}(\textit{fmttxt})

    \vspace{-1.5ex}

    \rule{\textwidth}{0.5\fboxrule}
\setlength{\parskip}{2ex}
    Shows a formatted text message.

\setlength{\parskip}{1ex}
      \textbf{Parameters}
      \vspace{-1ex}

      \begin{quote}
        \begin{Ventry}{xxxxxx}

          \item[fmttxt]

          text message to show (with format parameters, e.g. \%s, \%d, \%f 
          etc..)

        \end{Ventry}

      \end{quote}

\textbf{Status:} Untested + NoDoc + NoDemo = NOT OK



    \end{boxedminipage}

    \label{xformslib:library:fl_hide_message}
    \index{xformslib \textit{(package)}!xformslib.library \textit{(module)}!xformslib.library.fl\_hide\_message \textit{(function)}}

    \vspace{0.5ex}

\hspace{.8\funcindent}\begin{boxedminipage}{\funcwidth}

    \raggedright \textbf{fl\_hide\_message}()

    \vspace{-1.5ex}

    \rule{\textwidth}{0.5\fboxrule}
\setlength{\parskip}{2ex}
    Hides a text message already shown.

\setlength{\parskip}{1ex}
\textbf{Status:} Untested + NoDoc + NoDemo = NOT OK



    \end{boxedminipage}

    \label{xformslib:library:fl_hide_message}
    \index{xformslib \textit{(package)}!xformslib.library \textit{(module)}!xformslib.library.fl\_hide\_message \textit{(function)}}

    \vspace{0.5ex}

\hspace{.8\funcindent}\begin{boxedminipage}{\funcwidth}

    \raggedright \textbf{fl\_hide\_msg}()

    \vspace{-1.5ex}

    \rule{\textwidth}{0.5\fboxrule}
\setlength{\parskip}{2ex}
    Hides a text message already shown.

\setlength{\parskip}{1ex}
\textbf{Status:} Untested + NoDoc + NoDemo = NOT OK



    \end{boxedminipage}

    \label{xformslib:library:fl_hide_message}
    \index{xformslib \textit{(package)}!xformslib.library \textit{(module)}!xformslib.library.fl\_hide\_message \textit{(function)}}

    \vspace{0.5ex}

\hspace{.8\funcindent}\begin{boxedminipage}{\funcwidth}

    \raggedright \textbf{fl\_hide\_messages}()

    \vspace{-1.5ex}

    \rule{\textwidth}{0.5\fboxrule}
\setlength{\parskip}{2ex}
    Hides a text message already shown.

\setlength{\parskip}{1ex}
\textbf{Status:} Untested + NoDoc + NoDemo = NOT OK



    \end{boxedminipage}

    \label{xformslib:library:fl_show_question}
    \index{xformslib \textit{(package)}!xformslib.library \textit{(module)}!xformslib.library.fl\_show\_question \textit{(function)}}

    \vspace{0.5ex}

\hspace{.8\funcindent}\begin{boxedminipage}{\funcwidth}

    \raggedright \textbf{fl\_show\_question}(\textit{questmsg}, \textit{p2})

    \vspace{-1.5ex}

    \rule{\textwidth}{0.5\fboxrule}
\setlength{\parskip}{2ex}
    Shows a question message.

\setlength{\parskip}{1ex}
      \textbf{Parameters}
      \vspace{-1ex}

      \begin{quote}
        \begin{Ventry}{xxxxxxxx}

          \item[questmsg]

          text of question message to show

          \item[p2]

          ?

        \end{Ventry}

      \end{quote}

      \textbf{Return Value}
    \vspace{-1ex}

      \begin{quote}
      num

      \end{quote}

\textbf{Status:} Tested + NoDoc + Demo = OK



    \end{boxedminipage}

    \label{xformslib:library:fl_hide_question}
    \index{xformslib \textit{(package)}!xformslib.library \textit{(module)}!xformslib.library.fl\_hide\_question \textit{(function)}}

    \vspace{0.5ex}

\hspace{.8\funcindent}\begin{boxedminipage}{\funcwidth}

    \raggedright \textbf{fl\_hide\_question}()

    \vspace{-1.5ex}

    \rule{\textwidth}{0.5\fboxrule}
\setlength{\parskip}{2ex}
    Hides a question message already shown.

\setlength{\parskip}{1ex}
\textbf{Status:} Untested + NoDoc + NoDemo = NOT OK



    \end{boxedminipage}

    \label{xformslib:library:fl_show_alert}
    \index{xformslib \textit{(package)}!xformslib.library \textit{(module)}!xformslib.library.fl\_show\_alert \textit{(function)}}

    \vspace{0.5ex}

\hspace{.8\funcindent}\begin{boxedminipage}{\funcwidth}

    \raggedright \textbf{fl\_show\_alert}(\textit{title}, \textit{msg1}, \textit{msg2}, \textit{centered})

    \vspace{-1.5ex}

    \rule{\textwidth}{0.5\fboxrule}
\setlength{\parskip}{2ex}
    Shows an alert message.

\setlength{\parskip}{1ex}
      \textbf{Parameters}
      \vspace{-1ex}

      \begin{quote}
        \begin{Ventry}{xxxxxxxx}

          \item[title]

          title of alert

          \item[msg1]

          first message text

          \item[msg2]

          other message text

          \item[centered]

          if alert has to be displayed centered or not

        \end{Ventry}

      \end{quote}

\textbf{Status:} Tested + NoDoc + Demo = OK



    \end{boxedminipage}

    \label{xformslib:library:fl_show_alert2}
    \index{xformslib \textit{(package)}!xformslib.library \textit{(module)}!xformslib.library.fl\_show\_alert2 \textit{(function)}}

    \vspace{0.5ex}

\hspace{.8\funcindent}\begin{boxedminipage}{\funcwidth}

    \raggedright \textbf{fl\_show\_alert2}(\textit{centered}, \textit{fmt})

    \vspace{-1.5ex}

    \rule{\textwidth}{0.5\fboxrule}
\setlength{\parskip}{2ex}
    Shows a formatted alert message.

\setlength{\parskip}{1ex}
      \textbf{Parameters}
      \vspace{-1ex}

      \begin{quote}
        \begin{Ventry}{xxxxxxxx}

          \item[fmt]

          formatted message text

          \item[centered]

          if alert has to be displayed centered or not

        \end{Ventry}

      \end{quote}

\textbf{Status:} Untested + NoDoc + NoDemo = NOT OK



    \end{boxedminipage}

    \label{xformslib:library:fl_hide_alert}
    \index{xformslib \textit{(package)}!xformslib.library \textit{(module)}!xformslib.library.fl\_hide\_alert \textit{(function)}}

    \vspace{0.5ex}

\hspace{.8\funcindent}\begin{boxedminipage}{\funcwidth}

    \raggedright \textbf{fl\_hide\_alert}()

    \vspace{-1.5ex}

    \rule{\textwidth}{0.5\fboxrule}
\setlength{\parskip}{2ex}
    Hides a previously shown alert message.

\setlength{\parskip}{1ex}
\textbf{Status:} Tested + NoDoc + Demo = OK



    \end{boxedminipage}

    \label{xformslib:library:fl_show_input}
    \index{xformslib \textit{(package)}!xformslib.library \textit{(module)}!xformslib.library.fl\_show\_input \textit{(function)}}

    \vspace{0.5ex}

\hspace{.8\funcindent}\begin{boxedminipage}{\funcwidth}

    \raggedright \textbf{fl\_show\_input}(\textit{msgtxt}, \textit{defstr})

    \vspace{-1.5ex}

    \rule{\textwidth}{0.5\fboxrule}
\setlength{\parskip}{2ex}
    Obtains some text from user, showing a default text. It has OK and 
    Cancel buttons.

\setlength{\parskip}{1ex}
      \textbf{Parameters}
      \vspace{-1ex}

      \begin{quote}
        \begin{Ventry}{xxxxxx}

          \item[msgtxt]

          text used to ask for input

          \item[defstr]

          default user answer to show

        \end{Ventry}

      \end{quote}

      \textbf{Return Value}
    \vspace{-1ex}

      \begin{quote}
      input string

      \end{quote}

\textbf{Status:} Tested + NoDoc + Demo = OK



    \end{boxedminipage}

    \label{xformslib:library:fl_hide_input}
    \index{xformslib \textit{(package)}!xformslib.library \textit{(module)}!xformslib.library.fl\_hide\_input \textit{(function)}}

    \vspace{0.5ex}

\hspace{.8\funcindent}\begin{boxedminipage}{\funcwidth}

    \raggedright \textbf{fl\_hide\_input}()

    \vspace{-1.5ex}

    \rule{\textwidth}{0.5\fboxrule}
\setlength{\parskip}{2ex}
    Hides a previously shown input object.

\setlength{\parskip}{1ex}
\textbf{Status:} Untested + NoDoc + NoDemo = NOT OK



    \end{boxedminipage}

    \label{xformslib:library:fl_show_simple_input}
    \index{xformslib \textit{(package)}!xformslib.library \textit{(module)}!xformslib.library.fl\_show\_simple\_input \textit{(function)}}

    \vspace{0.5ex}

\hspace{.8\funcindent}\begin{boxedminipage}{\funcwidth}

    \raggedright \textbf{fl\_show\_simple\_input}(\textit{msgtxt}, \textit{defstr})

    \vspace{-1.5ex}

    \rule{\textwidth}{0.5\fboxrule}
\setlength{\parskip}{2ex}
    Asks the user for textual input. It has an OK button only.

\setlength{\parskip}{1ex}
      \textbf{Parameters}
      \vspace{-1ex}

      \begin{quote}
        \begin{Ventry}{xxxxxx}

          \item[msgtxt]

          message used to ask for input

          \item[defstr]

          default user answer in input

        \end{Ventry}

      \end{quote}

      \textbf{Return Value}
    \vspace{-1ex}

      \begin{quote}
      input string

      \end{quote}

\textbf{Example:} inpstr = fl\_show\_simple\_input("Insert name and surname:", "John Doe")



\textbf{Status:} Tested + Doc + NoDemo = OK



    \end{boxedminipage}

    \label{xformslib:library:fl_show_colormap}
    \index{xformslib \textit{(package)}!xformslib.library \textit{(module)}!xformslib.library.fl\_show\_colormap \textit{(function)}}

    \vspace{0.5ex}

\hspace{.8\funcindent}\begin{boxedminipage}{\funcwidth}

    \raggedright \textbf{fl\_show\_colormap}(\textit{oldcolr})

    \vspace{-1.5ex}

    \rule{\textwidth}{0.5\fboxrule}
\setlength{\parskip}{2ex}
    Shows a colormap color selector from which the user can select a color.

\setlength{\parskip}{1ex}
      \textbf{Parameters}
      \vspace{-1ex}

      \begin{quote}
        \begin{Ventry}{xxxxxxx}

          \item[oldcolr]

          color num. (Not FL\_COLOR)

            {\it (type=num./int)}

        \end{Ventry}

      \end{quote}

      \textbf{Return Value}
    \vspace{-1ex}

      \begin{quote}
      colormap num

      \end{quote}

\textbf{Status:} Tested + NoDoc + Demo = OK



    \end{boxedminipage}

    \label{xformslib:library:fl_show_choices}
    \index{xformslib \textit{(package)}!xformslib.library \textit{(module)}!xformslib.library.fl\_show\_choices \textit{(function)}}

    \vspace{0.5ex}

\hspace{.8\funcindent}\begin{boxedminipage}{\funcwidth}

    \raggedright \textbf{fl\_show\_choices}(\textit{msgtxt}, \textit{p2}, \textit{p3}, \textit{p4}, \textit{p5}, \textit{p6})

    \vspace{-1.5ex}

    \rule{\textwidth}{0.5\fboxrule}
\setlength{\parskip}{2ex}
\setlength{\parskip}{1ex}
      \textbf{Return Value}
    \vspace{-1ex}

      \begin{quote}
      num

      \end{quote}

\textbf{Status:} Tested + NoDoc + Demo = OK



    \end{boxedminipage}

    \label{xformslib:library:fl_show_choice}
    \index{xformslib \textit{(package)}!xformslib.library \textit{(module)}!xformslib.library.fl\_show\_choice \textit{(function)}}

    \vspace{0.5ex}

\hspace{.8\funcindent}\begin{boxedminipage}{\funcwidth}

    \raggedright \textbf{fl\_show\_choice}(\textit{p1}, \textit{p2}, \textit{p3}, \textit{p4}, \textit{p5}, \textit{p6}, \textit{p7}, \textit{p8})

    \vspace{-1.5ex}

    \rule{\textwidth}{0.5\fboxrule}
\setlength{\parskip}{2ex}
\setlength{\parskip}{1ex}
      \textbf{Return Value}
    \vspace{-1ex}

      \begin{quote}
      num

      \end{quote}

\textbf{Status:} Untested + NoDoc + NoDemo = NOT OK



    \end{boxedminipage}

    \label{xformslib:library:fl_hide_choice}
    \index{xformslib \textit{(package)}!xformslib.library \textit{(module)}!xformslib.library.fl\_hide\_choice \textit{(function)}}

    \vspace{0.5ex}

\hspace{.8\funcindent}\begin{boxedminipage}{\funcwidth}

    \raggedright \textbf{fl\_hide\_choice}()

    \vspace{-1.5ex}

    \rule{\textwidth}{0.5\fboxrule}
\setlength{\parskip}{2ex}
\setlength{\parskip}{1ex}
\textbf{Status:} Untested + NoDoc + NoDemo = NOT OK



    \end{boxedminipage}

    \label{xformslib:library:fl_set_choices_shortcut}
    \index{xformslib \textit{(package)}!xformslib.library \textit{(module)}!xformslib.library.fl\_set\_choices\_shortcut \textit{(function)}}

    \vspace{0.5ex}

\hspace{.8\funcindent}\begin{boxedminipage}{\funcwidth}

    \raggedright \textbf{fl\_set\_choices\_shortcut}(\textit{p1}, \textit{p2}, \textit{p3})

    \vspace{-1.5ex}

    \rule{\textwidth}{0.5\fboxrule}
\setlength{\parskip}{2ex}
\setlength{\parskip}{1ex}
\textbf{Status:} Untested + NoDoc + NoDemo = NOT OK



    \end{boxedminipage}

    \label{xformslib:library:fl_set_choices_shortcut}
    \index{xformslib \textit{(package)}!xformslib.library \textit{(module)}!xformslib.library.fl\_set\_choices\_shortcut \textit{(function)}}

    \vspace{0.5ex}

\hspace{.8\funcindent}\begin{boxedminipage}{\funcwidth}

    \raggedright \textbf{fl\_set\_choice\_shortcut}(\textit{p1}, \textit{p2}, \textit{p3})

    \vspace{-1.5ex}

    \rule{\textwidth}{0.5\fboxrule}
\setlength{\parskip}{2ex}
\setlength{\parskip}{1ex}
\textbf{Status:} Untested + NoDoc + NoDemo = NOT OK



    \end{boxedminipage}

    \label{xformslib:library:fl_show_oneliner}
    \index{xformslib \textit{(package)}!xformslib.library \textit{(module)}!xformslib.library.fl\_show\_oneliner \textit{(function)}}

    \vspace{0.5ex}

\hspace{.8\funcindent}\begin{boxedminipage}{\funcwidth}

    \raggedright \textbf{fl\_show\_oneliner}(\textit{p1}, \textit{p2}, \textit{p3})

    \vspace{-1.5ex}

    \rule{\textwidth}{0.5\fboxrule}
\setlength{\parskip}{2ex}
\setlength{\parskip}{1ex}
\textbf{Status:} Tested + NoDoc + Demo = OK



    \end{boxedminipage}

    \label{xformslib:library:fl_hide_oneliner}
    \index{xformslib \textit{(package)}!xformslib.library \textit{(module)}!xformslib.library.fl\_hide\_oneliner \textit{(function)}}

    \vspace{0.5ex}

\hspace{.8\funcindent}\begin{boxedminipage}{\funcwidth}

    \raggedright \textbf{fl\_hide\_oneliner}()

    \vspace{-1.5ex}

    \rule{\textwidth}{0.5\fboxrule}
\setlength{\parskip}{2ex}
\setlength{\parskip}{1ex}
\textbf{Status:} Tested + NoDoc + Demo = OK



    \end{boxedminipage}

    \label{xformslib:library:fl_set_oneliner_font}
    \index{xformslib \textit{(package)}!xformslib.library \textit{(module)}!xformslib.library.fl\_set\_oneliner\_font \textit{(function)}}

    \vspace{0.5ex}

\hspace{.8\funcindent}\begin{boxedminipage}{\funcwidth}

    \raggedright \textbf{fl\_set\_oneliner\_font}(\textit{style}, \textit{size})

    \vspace{-1.5ex}

    \rule{\textwidth}{0.5\fboxrule}
\setlength{\parskip}{2ex}
\setlength{\parskip}{1ex}
      \textbf{Parameters}
      \vspace{-1ex}

      \begin{quote}
        \begin{Ventry}{xxxxx}

          \item[style]

          label style ({\textless}int{\textgreater})

            {\it (type=(from xfdata module) FL\_NORMAL\_STYLE, FL\_BOLD\_STYLE, FL\_ITALIC\_STYLE,
FL\_BOLDITALIC\_STYLE, FL\_FIXED\_STYLE, FL\_FIXEDBOLD\_STYLE, 
FL\_FIXEDITALIC\_STYLE, FL\_FIXEDBOLDITALIC\_STYLE, FL\_TIMES\_STYLE, 
FL\_TIMESBOLD\_STYLE, FL\_TIMESITALIC\_STYLE, FL\_TIMESBOLDITALIC\_STYLE, 
FL\_MISC\_STYLE, FL\_MISCBOLD\_STYLE, FL\_MISCITALIC\_STYLE, 
FL\_SYMBOL\_STYLE, FL\_SHADOW\_STYLE, FL\_ENGRAVED\_STYLE, 
FL\_EMBOSSED\_STYLE)}

          \item[size]

          label size ({\textless}int{\textgreater})

            {\it (type=(from xfdata module) FL\_TINY\_SIZE, FL\_SMALL\_SIZE, FL\_NORMAL\_SIZE, 
FL\_MEDIUM\_SIZE, FL\_LARGE\_SIZE, FL\_HUGE\_SIZE, FL\_DEFAULT\_SIZE)}

        \end{Ventry}

      \end{quote}

\textbf{Status:} Untested + NoDoc + NoDemo = NOT OK



    \end{boxedminipage}

    \label{xformslib:library:fl_set_oneliner_color}
    \index{xformslib \textit{(package)}!xformslib.library \textit{(module)}!xformslib.library.fl\_set\_oneliner\_color \textit{(function)}}

    \vspace{0.5ex}

\hspace{.8\funcindent}\begin{boxedminipage}{\funcwidth}

    \raggedright \textbf{fl\_set\_oneliner\_color}(\textit{fgcolr}, \textit{bgcolr})

    \vspace{-1.5ex}

    \rule{\textwidth}{0.5\fboxrule}
\setlength{\parskip}{2ex}
\setlength{\parskip}{1ex}
\textbf{Status:} Untested + NoDoc + NoDemo = NOT OK



    \end{boxedminipage}

    \label{xformslib:library:fl_set_tooltip_font}
    \index{xformslib \textit{(package)}!xformslib.library \textit{(module)}!xformslib.library.fl\_set\_tooltip\_font \textit{(function)}}

    \vspace{0.5ex}

\hspace{.8\funcindent}\begin{boxedminipage}{\funcwidth}

    \raggedright \textbf{fl\_set\_tooltip\_font}(\textit{style}, \textit{size})

    \vspace{-1.5ex}

    \rule{\textwidth}{0.5\fboxrule}
\setlength{\parskip}{2ex}
\setlength{\parskip}{1ex}
      \textbf{Parameters}
      \vspace{-1ex}

      \begin{quote}
        \begin{Ventry}{xxxxx}

          \item[style]

          label style ({\textless}int{\textgreater})

            {\it (type=(from xfdata module) FL\_NORMAL\_STYLE, FL\_BOLD\_STYLE, FL\_ITALIC\_STYLE,
FL\_BOLDITALIC\_STYLE, FL\_FIXED\_STYLE, FL\_FIXEDBOLD\_STYLE, 
FL\_FIXEDITALIC\_STYLE, FL\_FIXEDBOLDITALIC\_STYLE, FL\_TIMES\_STYLE, 
FL\_TIMESBOLD\_STYLE, FL\_TIMESITALIC\_STYLE, FL\_TIMESBOLDITALIC\_STYLE, 
FL\_MISC\_STYLE, FL\_MISCBOLD\_STYLE, FL\_MISCITALIC\_STYLE, 
FL\_SYMBOL\_STYLE, FL\_SHADOW\_STYLE, FL\_ENGRAVED\_STYLE, 
FL\_EMBOSSED\_STYLE)}

          \item[size]

          label size ({\textless}int{\textgreater})

            {\it (type=(from xfdata module) FL\_TINY\_SIZE, FL\_SMALL\_SIZE, FL\_NORMAL\_SIZE, 
FL\_MEDIUM\_SIZE, FL\_LARGE\_SIZE, FL\_HUGE\_SIZE, FL\_DEFAULT\_SIZE)}

        \end{Ventry}

      \end{quote}

\textbf{Status:} Untested + NoDoc + NoDemo = NOT OK



    \end{boxedminipage}

    \label{xformslib:library:fl_set_tooltip_color}
    \index{xformslib \textit{(package)}!xformslib.library \textit{(module)}!xformslib.library.fl\_set\_tooltip\_color \textit{(function)}}

    \vspace{0.5ex}

\hspace{.8\funcindent}\begin{boxedminipage}{\funcwidth}

    \raggedright \textbf{fl\_set\_tooltip\_color}(\textit{fgcolr}, \textit{bgcolr})

    \vspace{-1.5ex}

    \rule{\textwidth}{0.5\fboxrule}
\setlength{\parskip}{2ex}
\setlength{\parskip}{1ex}
\textbf{Status:} Untested + NoDoc + NoDemo = NOT OK



    \end{boxedminipage}

    \label{xformslib:library:fl_set_tooltip_boxtype}
    \index{xformslib \textit{(package)}!xformslib.library \textit{(module)}!xformslib.library.fl\_set\_tooltip\_boxtype \textit{(function)}}

    \vspace{0.5ex}

\hspace{.8\funcindent}\begin{boxedminipage}{\funcwidth}

    \raggedright \textbf{fl\_set\_tooltip\_boxtype}(\textit{boxtype})

    \vspace{-1.5ex}

    \rule{\textwidth}{0.5\fboxrule}
\setlength{\parskip}{2ex}
\setlength{\parskip}{1ex}
      \textbf{Parameters}
      \vspace{-1ex}

      \begin{quote}
        \begin{Ventry}{xxxxxxx}

          \item[boxtype]

          type of the box to be added ({\textless}int{\textgreater})

            {\it (type=(from xfdata module) FL\_NO\_BOX, FL\_UP\_BOX, FL\_DOWN\_BOX, 
FL\_BORDER\_BOX, FL\_SHADOW\_BOX, FL\_FRAME\_BOX, FL\_ROUNDED\_BOX, 
FL\_EMBOSSED\_BOX, FL\_FLAT\_BOX, FL\_RFLAT\_BOX, FL\_RSHADOW\_BOX, 
FL\_OVAL\_BOX, FL\_ROUNDED3D\_UPBOX, FL\_ROUNDED3D\_DOWNBOX, 
FL\_OVAL3D\_UPBOX, FL\_OVAL3D\_DOWNBOX, FL\_OVAL3D\_FRAMEBOX, 
FL\_OVAL3D\_EMBOSSEDBOX)}

        \end{Ventry}

      \end{quote}

\textbf{Status:} Untested + NoDoc + NoDemo = NOT OK



    \end{boxedminipage}

    \label{xformslib:library:fl_set_tooltip_lalign}
    \index{xformslib \textit{(package)}!xformslib.library \textit{(module)}!xformslib.library.fl\_set\_tooltip\_lalign \textit{(function)}}

    \vspace{0.5ex}

\hspace{.8\funcindent}\begin{boxedminipage}{\funcwidth}

    \raggedright \textbf{fl\_set\_tooltip\_lalign}(\textit{align})

    \vspace{-1.5ex}

    \rule{\textwidth}{0.5\fboxrule}
\setlength{\parskip}{2ex}
\setlength{\parskip}{1ex}
      \textbf{Parameters}
      \vspace{-1ex}

      \begin{quote}
        \begin{Ventry}{xxxxx}

          \item[align]

          alignment of tooltip ({\textless}int{\textgreater})

            {\it (type=(from xfdata module) FL\_ALIGN\_CENTER, FL\_ALIGN\_TOP, FL\_ALIGN\_BOTTOM, 
FL\_ALIGN\_LEFT, FL\_ALIGN\_RIGHT, FL\_ALIGN\_LEFT\_TOP, 
FL\_ALIGN\_RIGHT\_TOP, FL\_ALIGN\_LEFT\_BOTTOM, FL\_ALIGN\_RIGHT\_BOTTOM, 
FL\_ALIGN\_INSIDE, FL\_ALIGN\_VERT)}

        \end{Ventry}

      \end{quote}

\textbf{Status:} Untested + NoDoc + NoDemo = NOT OK



    \end{boxedminipage}

    \label{xformslib:library:fl_exe_command}
    \index{xformslib \textit{(package)}!xformslib.library \textit{(module)}!xformslib.library.fl\_exe\_command \textit{(function)}}

    \vspace{0.5ex}

\hspace{.8\funcindent}\begin{boxedminipage}{\funcwidth}

    \raggedright \textbf{fl\_exe\_command}(\textit{command}, \textit{block})

    \vspace{-1.5ex}

    \rule{\textwidth}{0.5\fboxrule}
\setlength{\parskip}{2ex}
    Forks a new process that runs specified command

\setlength{\parskip}{1ex}
      \textbf{Parameters}
      \vspace{-1ex}

      \begin{quote}
        \begin{Ventry}{xxxxxxx}

          \item[command]

          a shell command line

          \item[block]

          blocking flag indicating if the function should wait for the 
          child process to finish or not ({\textless}int{\textgreater})

            {\it (type=non-zero (for waiting) or 0 (don't wait).)}

        \end{Ventry}

      \end{quote}

      \textbf{Return Value}
    \vspace{-1ex}

      \begin{quote}
      exit status

      \end{quote}

\textbf{Status:} Untested + NoDoc + NoDemo = NOT OK



    \end{boxedminipage}

    \label{xformslib:library:fl_exe_command}
    \index{xformslib \textit{(package)}!xformslib.library \textit{(module)}!xformslib.library.fl\_exe\_command \textit{(function)}}

    \vspace{0.5ex}

\hspace{.8\funcindent}\begin{boxedminipage}{\funcwidth}

    \raggedright \textbf{fl\_open\_command}(\textit{command}, \textit{block})

    \vspace{-1.5ex}

    \rule{\textwidth}{0.5\fboxrule}
\setlength{\parskip}{2ex}
    Forks a new process that runs specified command

\setlength{\parskip}{1ex}
      \textbf{Parameters}
      \vspace{-1ex}

      \begin{quote}
        \begin{Ventry}{xxxxxxx}

          \item[command]

          a shell command line

          \item[block]

          blocking flag indicating if the function should wait for the 
          child process to finish or not ({\textless}int{\textgreater})

            {\it (type=non-zero (for waiting) or 0 (don't wait).)}

        \end{Ventry}

      \end{quote}

      \textbf{Return Value}
    \vspace{-1ex}

      \begin{quote}
      exit status

      \end{quote}

\textbf{Status:} Untested + NoDoc + NoDemo = NOT OK



    \end{boxedminipage}

    \label{xformslib:library:fl_end_command}
    \index{xformslib \textit{(package)}!xformslib.library \textit{(module)}!xformslib.library.fl\_end\_command \textit{(function)}}

    \vspace{0.5ex}

\hspace{.8\funcindent}\begin{boxedminipage}{\funcwidth}

    \raggedright \textbf{fl\_end\_command}(\textit{pid})

    \vspace{-1.5ex}

    \rule{\textwidth}{0.5\fboxrule}
\setlength{\parskip}{2ex}
    Suspends the current process and waits until the child process is 
    completed, then it returns the exit status of the child process or -1 
    if an error has occurred.

\setlength{\parskip}{1ex}
      \textbf{Parameters}
      \vspace{-1ex}

      \begin{quote}
        \begin{Ventry}{xxx}

          \item[pid]

          process id returned by fl\_exe\_command()

        \end{Ventry}

      \end{quote}

      \textbf{Return Value}
    \vspace{-1ex}

      \begin{quote}
      exit status

      \end{quote}

\textbf{Status:} Untested + NoDoc + NoDemo = NOT OK



    \end{boxedminipage}

    \label{xformslib:library:fl_end_command}
    \index{xformslib \textit{(package)}!xformslib.library \textit{(module)}!xformslib.library.fl\_end\_command \textit{(function)}}

    \vspace{0.5ex}

\hspace{.8\funcindent}\begin{boxedminipage}{\funcwidth}

    \raggedright \textbf{fl\_close\_command}(\textit{pid})

    \vspace{-1.5ex}

    \rule{\textwidth}{0.5\fboxrule}
\setlength{\parskip}{2ex}
    Suspends the current process and waits until the child process is 
    completed, then it returns the exit status of the child process or -1 
    if an error has occurred.

\setlength{\parskip}{1ex}
      \textbf{Parameters}
      \vspace{-1ex}

      \begin{quote}
        \begin{Ventry}{xxx}

          \item[pid]

          process id returned by fl\_exe\_command()

        \end{Ventry}

      \end{quote}

      \textbf{Return Value}
    \vspace{-1ex}

      \begin{quote}
      exit status

      \end{quote}

\textbf{Status:} Untested + NoDoc + NoDemo = NOT OK



    \end{boxedminipage}

    \label{xformslib:library:fl_check_command}
    \index{xformslib \textit{(package)}!xformslib.library \textit{(module)}!xformslib.library.fl\_check\_command \textit{(function)}}

    \vspace{0.5ex}

\hspace{.8\funcindent}\begin{boxedminipage}{\funcwidth}

    \raggedright \textbf{fl\_check\_command}(\textit{pid})

    \vspace{-1.5ex}

    \rule{\textwidth}{0.5\fboxrule}
\setlength{\parskip}{2ex}
    Polls the status of a child process. Returns 0 if the child process is 
    finished; 1 if the child process still exists (running or stopped) and 
    -1 if an error has occurred inside the function.

\setlength{\parskip}{1ex}
      \textbf{Parameters}
      \vspace{-1ex}

      \begin{quote}
        \begin{Ventry}{xxx}

          \item[pid]

          process id returned by fl\_exe\_command()

        \end{Ventry}

      \end{quote}

      \textbf{Return Value}
    \vspace{-1ex}

      \begin{quote}
      exit status

      \end{quote}

\textbf{Status:} Untested + NoDoc + NoDemo = NOT OK



    \end{boxedminipage}

    \label{xformslib:library:fl_popen}
    \index{xformslib \textit{(package)}!xformslib.library \textit{(module)}!xformslib.library.fl\_popen \textit{(function)}}

    \vspace{0.5ex}

\hspace{.8\funcindent}\begin{boxedminipage}{\funcwidth}

    \raggedright \textbf{fl\_popen}(\textit{command}, \textit{otype})

    \vspace{-1.5ex}

    \rule{\textwidth}{0.5\fboxrule}
\setlength{\parskip}{2ex}
    Executes the command in a child process, and logs the stderr messages 
    into the command log. If type is "w", stdout will also be logged into 
    the command browser.

\setlength{\parskip}{1ex}
      \textbf{Parameters}
      \vspace{-1ex}

      \begin{quote}
        \begin{Ventry}{xxxxxxx}

          \item[command]

          filename to execute ({\textless}string{\textgreater})

          \item[otype]

          type of opening (e.g. w, r ..) ({\textless}string{\textgreater})

        \end{Ventry}

      \end{quote}

      \textbf{Return Value}
    \vspace{-1ex}

      \begin{quote}
      file opened ({\textless}pointer to FILE{\textgreater})

      {\it (type=pFile)}

      \end{quote}

\textbf{Possible values:}
\begin{quote}
  \begin{itemize}

  \item
    \setlength{\parskip}{0.6ex}
existing filename



  \item letter between 'w', 'r', etc..



\end{itemize}

\end{quote}

\textbf{Status:} Untested + NoDoc + NoDemo = NOT OK



    \end{boxedminipage}

    \label{xformslib:library:fl_pclose}
    \index{xformslib \textit{(package)}!xformslib.library \textit{(module)}!xformslib.library.fl\_pclose \textit{(function)}}

    \vspace{0.5ex}

\hspace{.8\funcindent}\begin{boxedminipage}{\funcwidth}

    \raggedright \textbf{fl\_pclose}(\textit{pFile})

    \vspace{-1.5ex}

    \rule{\textwidth}{0.5\fboxrule}
\setlength{\parskip}{2ex}
    Cleans up the child process executed.

\setlength{\parskip}{1ex}
      \textbf{Parameters}
      \vspace{-1ex}

      \begin{quote}
        \begin{Ventry}{xxxxx}

          \item[pFile]

          opened file stream returned by fl\_popen() ({\textless}pointer to
          FILE{\textgreater})

        \end{Ventry}

      \end{quote}

      \textbf{Return Value}
    \vspace{-1ex}

      \begin{quote}
      non-zero ({\textless}int{\textgreater}) or -1 (on failure)

      {\it (type=num)}

      \end{quote}

\textbf{Example:}
\begin{quote}
  \begin{itemize}

  \item
    \setlength{\parskip}{0.6ex}
if fl\_pclose(pfile) == -1:



  \item ...



\end{itemize}

\end{quote}

\textbf{Status:} Tested + Doc + NoDemo = OK



    \end{boxedminipage}

    \label{xformslib:library:fl_end_all_command}
    \index{xformslib \textit{(package)}!xformslib.library \textit{(module)}!xformslib.library.fl\_end\_all\_command \textit{(function)}}

    \vspace{0.5ex}

\hspace{.8\funcindent}\begin{boxedminipage}{\funcwidth}

    \raggedright \textbf{fl\_end\_all\_command}()

    \vspace{-1.5ex}

    \rule{\textwidth}{0.5\fboxrule}
\setlength{\parskip}{2ex}
    Wait for all the child processes initiated by fl\_exe\_command() to 
    complete. Returns the status of the last child process.

\setlength{\parskip}{1ex}
      \textbf{Return Value}
    \vspace{-1ex}

      \begin{quote}
      exit status

      \end{quote}

\textbf{Status:} Untested + NoDoc + NoDemo = NOT OK



    \end{boxedminipage}

    \label{xformslib:library:fl_show_command_log}
    \index{xformslib \textit{(package)}!xformslib.library \textit{(module)}!xformslib.library.fl\_show\_command\_log \textit{(function)}}

    \vspace{0.5ex}

\hspace{.8\funcindent}\begin{boxedminipage}{\funcwidth}

    \raggedright \textbf{fl\_show\_command\_log}(\textit{border})

    \vspace{-1.5ex}

    \rule{\textwidth}{0.5\fboxrule}
\setlength{\parskip}{2ex}
    Shows the log of the command output.

\setlength{\parskip}{1ex}
      \textbf{Parameters}
      \vspace{-1ex}

      \begin{quote}
        \begin{Ventry}{xxxxxx}

          \item[border]

          window manager decoration ({\textless}int{\textgreater})

            {\it (type=(from xfdata module) \texttt{FL\_FULLBORDER}, \texttt{FL\_TRANSIENT}, 
\texttt{FL\_NOBORDER})}

        \end{Ventry}

      \end{quote}

\textbf{Possible values:} (from xfdata module) \texttt{FL\_FULLBORDER}, \texttt{FL\_TRANSIENT}, 
\texttt{FL\_NOBORDER}



\textbf{Status:} Untested + NoDoc + NoDemo = NOT OK



    \end{boxedminipage}

    \label{xformslib:library:fl_hide_command_log}
    \index{xformslib \textit{(package)}!xformslib.library \textit{(module)}!xformslib.library.fl\_hide\_command\_log \textit{(function)}}

    \vspace{0.5ex}

\hspace{.8\funcindent}\begin{boxedminipage}{\funcwidth}

    \raggedright \textbf{fl\_hide\_command\_log}()

    \vspace{-1.5ex}

    \rule{\textwidth}{0.5\fboxrule}
\setlength{\parskip}{2ex}
    Hides the log of the command output.

\setlength{\parskip}{1ex}
\textbf{Status:} Untested + NoDoc + NoDemo = NOT OK



    \end{boxedminipage}

    \label{xformslib:library:fl_clear_command_log}
    \index{xformslib \textit{(package)}!xformslib.library \textit{(module)}!xformslib.library.fl\_clear\_command\_log \textit{(function)}}

    \vspace{0.5ex}

\hspace{.8\funcindent}\begin{boxedminipage}{\funcwidth}

    \raggedright \textbf{fl\_clear\_command\_log}()

    \vspace{-1.5ex}

    \rule{\textwidth}{0.5\fboxrule}
\setlength{\parskip}{2ex}
    Clears the browser and the logging output displayed within it.

\setlength{\parskip}{1ex}
\textbf{Status:} Untested + NoDoc + NoDemo = NOT OK



    \end{boxedminipage}

    \label{xformslib:library:fl_addto_command_log}
    \index{xformslib \textit{(package)}!xformslib.library \textit{(module)}!xformslib.library.fl\_addto\_command\_log \textit{(function)}}

    \vspace{0.5ex}

\hspace{.8\funcindent}\begin{boxedminipage}{\funcwidth}

    \raggedright \textbf{fl\_addto\_command\_log}(\textit{txtstr})

    \vspace{-1.5ex}

    \rule{\textwidth}{0.5\fboxrule}
\setlength{\parskip}{2ex}
    Adds arbitrary text to the command browser

\setlength{\parskip}{1ex}
      \textbf{Parameters}
      \vspace{-1ex}

      \begin{quote}
        \begin{Ventry}{xxxxxx}

          \item[txtstr]

          text line to be added {\textless}string{\textgreater}

        \end{Ventry}

      \end{quote}

\textbf{Status:} Untested + NoDoc + NoDemo = NOT OK



    \end{boxedminipage}

    \label{xformslib:library:fl_set_command_log_position}
    \index{xformslib \textit{(package)}!xformslib.library \textit{(module)}!xformslib.library.fl\_set\_command\_log\_position \textit{(function)}}

    \vspace{0.5ex}

\hspace{.8\funcindent}\begin{boxedminipage}{\funcwidth}

    \raggedright \textbf{fl\_set\_command\_log\_position}(\textit{x}, \textit{y})

    \vspace{-1.5ex}

    \rule{\textwidth}{0.5\fboxrule}
\setlength{\parskip}{2ex}
    Changes the default placement of the command log.

\setlength{\parskip}{1ex}
      \textbf{Parameters}
      \vspace{-1ex}

      \begin{quote}
        \begin{Ventry}{x}

          \item[x]

          horizontal position (upper-left corner)

          \item[y]

          vertical position (upper-left corner)

        \end{Ventry}

      \end{quote}

\textbf{Status:} Untested + NoDoc + NoDemo = NOT OK



    \end{boxedminipage}

    \label{xformslib:library:fl_get_command_log_fdstruct}
    \index{xformslib \textit{(package)}!xformslib.library \textit{(module)}!xformslib.library.fl\_get\_command\_log\_fdstruct \textit{(function)}}

    \vspace{0.5ex}

\hspace{.8\funcindent}\begin{boxedminipage}{\funcwidth}

    \raggedright \textbf{fl\_get\_command\_log\_fdstruct}()

    \vspace{-1.5ex}

    \rule{\textwidth}{0.5\fboxrule}
\setlength{\parskip}{2ex}
\setlength{\parskip}{1ex}
      \textbf{Return Value}
    \vspace{-1ex}

      \begin{quote}
      pCmdlog

      \end{quote}

\textbf{Status:} Untested + NoDoc + NoDemo = NOT OK



    \end{boxedminipage}

    \label{xformslib:library:fl_use_fselector}
    \index{xformslib \textit{(package)}!xformslib.library \textit{(module)}!xformslib.library.fl\_use\_fselector \textit{(function)}}

    \vspace{0.5ex}

\hspace{.8\funcindent}\begin{boxedminipage}{\funcwidth}

    \raggedright \textbf{fl\_use\_fselector}(\textit{num})

    \vspace{-1.5ex}

    \rule{\textwidth}{0.5\fboxrule}
\setlength{\parskip}{2ex}
\setlength{\parskip}{1ex}
      \textbf{Parameters}
      \vspace{-1ex}

      \begin{quote}
        \begin{Ventry}{xxx}

          \item[num]

          fselector number to use ({\textless}int{\textgreater})

            {\it (type=between 0 and xfdata.FL\_MAX\_FSELECTOR - 1)}

        \end{Ventry}

      \end{quote}

      \textbf{Return Value}
    \vspace{-1ex}

      \begin{quote}
      num

      \end{quote}

\textbf{Status:} Untested + NoDoc + NoDemo = NOT OK



    \end{boxedminipage}

    \label{xformslib:library:fl_show_fselector}
    \index{xformslib \textit{(package)}!xformslib.library \textit{(module)}!xformslib.library.fl\_show\_fselector \textit{(function)}}

    \vspace{0.5ex}

\hspace{.8\funcindent}\begin{boxedminipage}{\funcwidth}

    \raggedright \textbf{fl\_show\_fselector}(\textit{msgtxt}, \textit{p2}, \textit{p3}, \textit{p4})

    \vspace{-1.5ex}

    \rule{\textwidth}{0.5\fboxrule}
\setlength{\parskip}{2ex}
\setlength{\parskip}{1ex}
      \textbf{Parameters}
      \vspace{-1ex}

      \begin{quote}
        \begin{Ventry}{xxxxxx}

          \item[msgtxt]

          ({\textless}string{\textgreater})

          \item[p2]

          ({\textless}string{\textgreater})

          \item[p3]

          ({\textless}string{\textgreater})

        \end{Ventry}

      \end{quote}

      \textbf{Return Value}
    \vspace{-1ex}

      \begin{quote}
      fselector string

      \end{quote}

\textbf{Status:} Tested + NoDoc + Demo = OK



    \end{boxedminipage}

    \label{xformslib:library:fl_show_fselector}
    \index{xformslib \textit{(package)}!xformslib.library \textit{(module)}!xformslib.library.fl\_show\_fselector \textit{(function)}}

    \vspace{0.5ex}

\hspace{.8\funcindent}\begin{boxedminipage}{\funcwidth}

    \raggedright \textbf{fl\_show\_file\_selector}(\textit{msgtxt}, \textit{p2}, \textit{p3}, \textit{p4})

    \vspace{-1.5ex}

    \rule{\textwidth}{0.5\fboxrule}
\setlength{\parskip}{2ex}
\setlength{\parskip}{1ex}
      \textbf{Parameters}
      \vspace{-1ex}

      \begin{quote}
        \begin{Ventry}{xxxxxx}

          \item[msgtxt]

          ({\textless}string{\textgreater})

          \item[p2]

          ({\textless}string{\textgreater})

          \item[p3]

          ({\textless}string{\textgreater})

        \end{Ventry}

      \end{quote}

      \textbf{Return Value}
    \vspace{-1ex}

      \begin{quote}
      fselector string

      \end{quote}

\textbf{Status:} Tested + NoDoc + Demo = OK



    \end{boxedminipage}

    \label{xformslib:library:fl_set_fselector_fontsize}
    \index{xformslib \textit{(package)}!xformslib.library \textit{(module)}!xformslib.library.fl\_set\_fselector\_fontsize \textit{(function)}}

    \vspace{0.5ex}

\hspace{.8\funcindent}\begin{boxedminipage}{\funcwidth}

    \raggedright \textbf{fl\_set\_fselector\_fontsize}(\textit{size})

    \vspace{-1.5ex}

    \rule{\textwidth}{0.5\fboxrule}
\setlength{\parskip}{2ex}
\setlength{\parskip}{1ex}
      \textbf{Parameters}
      \vspace{-1ex}

      \begin{quote}
        \begin{Ventry}{xxxx}

          \item[size]

          label size [{\textless}int{\textgreater}]

            {\it (type=(from xfdata module) FL\_TINY\_SIZE, FL\_SMALL\_SIZE, FL\_NORMAL\_SIZE, 
FL\_MEDIUM\_SIZE, FL\_LARGE\_SIZE, FL\_HUGE\_SIZE, FL\_DEFAULT\_SIZE)}

        \end{Ventry}

      \end{quote}

\textbf{Status:} Untested + NoDoc + NoDemo = NOT OK



    \end{boxedminipage}

    \label{xformslib:library:fl_set_fselector_fontstyle}
    \index{xformslib \textit{(package)}!xformslib.library \textit{(module)}!xformslib.library.fl\_set\_fselector\_fontstyle \textit{(function)}}

    \vspace{0.5ex}

\hspace{.8\funcindent}\begin{boxedminipage}{\funcwidth}

    \raggedright \textbf{fl\_set\_fselector\_fontstyle}(\textit{style})

    \vspace{-1.5ex}

    \rule{\textwidth}{0.5\fboxrule}
\setlength{\parskip}{2ex}
\setlength{\parskip}{1ex}
      \textbf{Parameters}
      \vspace{-1ex}

      \begin{quote}
        \begin{Ventry}{xxxxx}

          \item[style]

          label style ({\textless}int{\textgreater})

            {\it (type=(from xfdata module) FL\_NORMAL\_STYLE, FL\_BOLD\_STYLE, FL\_ITALIC\_STYLE,
FL\_BOLDITALIC\_STYLE, FL\_FIXED\_STYLE, FL\_FIXEDBOLD\_STYLE, 
FL\_FIXEDITALIC\_STYLE, FL\_FIXEDBOLDITALIC\_STYLE, FL\_TIMES\_STYLE, 
FL\_TIMESBOLD\_STYLE, FL\_TIMESITALIC\_STYLE, FL\_TIMESBOLDITALIC\_STYLE, 
FL\_MISC\_STYLE, FL\_MISCBOLD\_STYLE, FL\_MISCITALIC\_STYLE, 
FL\_SYMBOL\_STYLE, FL\_SHADOW\_STYLE, FL\_ENGRAVED\_STYLE, 
FL\_EMBOSSED\_STYLE)}

        \end{Ventry}

      \end{quote}

\textbf{Status:} Untested + NoDoc + NoDemo = NOT OK



    \end{boxedminipage}

    \label{xformslib:library:fl_set_fselector_placement}
    \index{xformslib \textit{(package)}!xformslib.library \textit{(module)}!xformslib.library.fl\_set\_fselector\_placement \textit{(function)}}

    \vspace{0.5ex}

\hspace{.8\funcindent}\begin{boxedminipage}{\funcwidth}

    \raggedright \textbf{fl\_set\_fselector\_placement}(\textit{place})

    \vspace{-1.5ex}

    \rule{\textwidth}{0.5\fboxrule}
\setlength{\parskip}{2ex}
\setlength{\parskip}{1ex}
      \textbf{Parameters}
      \vspace{-1ex}

      \begin{quote}
        \begin{Ventry}{xxxxx}

          \item[place]

          where to place it ({\textless}int{\textgreater})

            {\it (type=(from xfdata module) FL\_PLACE\_FREE, FL\_PLACE\_MOUSE, FL\_PLACE\_CENTER, 
FL\_PLACE\_POSITION, FL\_PLACE\_SIZE, FL\_PLACE\_GEOMETRY, 
FL\_PLACE\_ASPECT, FL\_PLACE\_FULLSCREEN, FL\_PLACE\_HOTSPOT, 
FL\_PLACE\_ICONIC, FL\_FREE\_SIZE, FL\_PLACE\_FREE\_CENTER, 
FL\_PLACE\_CENTERFREE)}

        \end{Ventry}

      \end{quote}

\textbf{Status:} Tested + NoDoc + Demo = OK



    \end{boxedminipage}

    \label{xformslib:library:fl_set_fselector_border}
    \index{xformslib \textit{(package)}!xformslib.library \textit{(module)}!xformslib.library.fl\_set\_fselector\_border \textit{(function)}}

    \vspace{0.5ex}

\hspace{.8\funcindent}\begin{boxedminipage}{\funcwidth}

    \raggedright \textbf{fl\_set\_fselector\_border}(\textit{border})

    \vspace{-1.5ex}

    \rule{\textwidth}{0.5\fboxrule}
\setlength{\parskip}{2ex}
\setlength{\parskip}{1ex}
      \textbf{Parameters}
      \vspace{-1ex}

      \begin{quote}
        \begin{Ventry}{xxxxxx}

          \item[border]

          window manager decoration ({\textless}int{\textgreater})

            {\it (type=(from xfdata module) FL\_FULLBORDER, FL\_TRANSIENT, FL\_NOBORDER)}

        \end{Ventry}

      \end{quote}

\textbf{Status:} Untested + NoDoc + NoDemo = NOT OK



    \end{boxedminipage}

    \label{xformslib:library:fl_set_fselector_transient}
    \index{xformslib \textit{(package)}!xformslib.library \textit{(module)}!xformslib.library.fl\_set\_fselector\_transient \textit{(function)}}

    \vspace{0.5ex}

\hspace{.8\funcindent}\begin{boxedminipage}{\funcwidth}

    \raggedright \textbf{fl\_set\_fselector\_transient}(\textit{flag})

    \vspace{-1.5ex}

    \rule{\textwidth}{0.5\fboxrule}
\setlength{\parskip}{2ex}
\setlength{\parskip}{1ex}
      \textbf{Parameters}
      \vspace{-1ex}

      \begin{quote}
        \begin{Ventry}{xxxx}

          \item[flag]

          flag if transient or not ({\textless}int{\textgreater})

            {\it (type=1 (transient) or 0 (not transient))}

        \end{Ventry}

      \end{quote}

\textbf{Status:} Untested + NoDoc + NoDemo = NOT OK



    \end{boxedminipage}

    \label{xformslib:library:fl_set_fselector_callback}
    \index{xformslib \textit{(package)}!xformslib.library \textit{(module)}!xformslib.library.fl\_set\_fselector\_callback \textit{(function)}}

    \vspace{0.5ex}

\hspace{.8\funcindent}\begin{boxedminipage}{\funcwidth}

    \raggedright \textbf{fl\_set\_fselector\_callback}(\textit{py\_FSCB}, \textit{vdata})

    \vspace{-1.5ex}

    \rule{\textwidth}{0.5\fboxrule}
\setlength{\parskip}{2ex}
\setlength{\parskip}{1ex}
      \textbf{Parameters}
      \vspace{-1ex}

      \begin{quote}
        \begin{Ventry}{xxxxxxx}

          \item[py\_FSCB]

          python function callback, returning value

            {\it (type=\_\_ funcname (string, ptr\_void) -{\textgreater} num \_\_)}

          \item[vdata]

          user data to be passed to function ({\textless}pointer to 
          void{\textgreater})

        \end{Ventry}

      \end{quote}

\textbf{Status:} Tested + NoDoc + Demo = OK



    \end{boxedminipage}

    \label{xformslib:library:fl_set_fselector_callback}
    \index{xformslib \textit{(package)}!xformslib.library \textit{(module)}!xformslib.library.fl\_set\_fselector\_callback \textit{(function)}}

    \vspace{0.5ex}

\hspace{.8\funcindent}\begin{boxedminipage}{\funcwidth}

    \raggedright \textbf{fl\_set\_fselector\_cb}(\textit{py\_FSCB}, \textit{vdata})

    \vspace{-1.5ex}

    \rule{\textwidth}{0.5\fboxrule}
\setlength{\parskip}{2ex}
\setlength{\parskip}{1ex}
      \textbf{Parameters}
      \vspace{-1ex}

      \begin{quote}
        \begin{Ventry}{xxxxxxx}

          \item[py\_FSCB]

          python function callback, returning value

            {\it (type=\_\_ funcname (string, ptr\_void) -{\textgreater} num \_\_)}

          \item[vdata]

          user data to be passed to function ({\textless}pointer to 
          void{\textgreater})

        \end{Ventry}

      \end{quote}

\textbf{Status:} Tested + NoDoc + Demo = OK



    \end{boxedminipage}

    \label{xformslib:library:fl_get_filename}
    \index{xformslib \textit{(package)}!xformslib.library \textit{(module)}!xformslib.library.fl\_get\_filename \textit{(function)}}

    \vspace{0.5ex}

\hspace{.8\funcindent}\begin{boxedminipage}{\funcwidth}

    \raggedright \textbf{fl\_get\_filename}()

    \vspace{-1.5ex}

    \rule{\textwidth}{0.5\fboxrule}
\setlength{\parskip}{2ex}
\setlength{\parskip}{1ex}
      \textbf{Return Value}
    \vspace{-1ex}

      \begin{quote}
      name of file ({\textless}string{\textgreater})

      {\it (type=fname)}

      \end{quote}

\textbf{Status:} Untested + NoDoc + NoDemo = NOT OK



    \end{boxedminipage}

    \label{xformslib:library:fl_get_directory}
    \index{xformslib \textit{(package)}!xformslib.library \textit{(module)}!xformslib.library.fl\_get\_directory \textit{(function)}}

    \vspace{0.5ex}

\hspace{.8\funcindent}\begin{boxedminipage}{\funcwidth}

    \raggedright \textbf{fl\_get\_directory}()

    \vspace{-1.5ex}

    \rule{\textwidth}{0.5\fboxrule}
\setlength{\parskip}{2ex}
\setlength{\parskip}{1ex}
      \textbf{Return Value}
    \vspace{-1ex}

      \begin{quote}
      name of directory ({\textless}string{\textgreater})

      {\it (type=dirname)}

      \end{quote}

\textbf{Status:} Untested + NoDoc + NoDemo = NOT OK



    \end{boxedminipage}

    \label{xformslib:library:fl_get_pattern}
    \index{xformslib \textit{(package)}!xformslib.library \textit{(module)}!xformslib.library.fl\_get\_pattern \textit{(function)}}

    \vspace{0.5ex}

\hspace{.8\funcindent}\begin{boxedminipage}{\funcwidth}

    \raggedright \textbf{fl\_get\_pattern}()

    \vspace{-1.5ex}

    \rule{\textwidth}{0.5\fboxrule}
\setlength{\parskip}{2ex}
\setlength{\parskip}{1ex}
      \textbf{Return Value}
    \vspace{-1ex}

      \begin{quote}
      pattern text ({\textless}string{\textgreater})

      {\it (type=pattern)}

      \end{quote}

\textbf{Status:} Untested + NoDoc + NoDemo = NOT OK



    \end{boxedminipage}

    \label{xformslib:library:fl_set_directory}
    \index{xformslib \textit{(package)}!xformslib.library \textit{(module)}!xformslib.library.fl\_set\_directory \textit{(function)}}

    \vspace{0.5ex}

\hspace{.8\funcindent}\begin{boxedminipage}{\funcwidth}

    \raggedright \textbf{fl\_set\_directory}(\textit{dirname})

    \vspace{-1.5ex}

    \rule{\textwidth}{0.5\fboxrule}
\setlength{\parskip}{2ex}
\setlength{\parskip}{1ex}
      \textbf{Parameters}
      \vspace{-1ex}

      \begin{quote}
        \begin{Ventry}{xxxxxxx}

          \item[dirname]

          name of directory to be set ({\textless}string{\textgreater})

        \end{Ventry}

      \end{quote}

      \textbf{Return Value}
    \vspace{-1ex}

      \begin{quote}
      num

      \end{quote}

\textbf{Status:} Untested + NoDoc + NoDemo = NOT OK



    \end{boxedminipage}

    \label{xformslib:library:fl_set_pattern}
    \index{xformslib \textit{(package)}!xformslib.library \textit{(module)}!xformslib.library.fl\_set\_pattern \textit{(function)}}

    \vspace{0.5ex}

\hspace{.8\funcindent}\begin{boxedminipage}{\funcwidth}

    \raggedright \textbf{fl\_set\_pattern}(\textit{pattern})

    \vspace{-1.5ex}

    \rule{\textwidth}{0.5\fboxrule}
\setlength{\parskip}{2ex}
\setlength{\parskip}{1ex}
      \textbf{Parameters}
      \vspace{-1ex}

      \begin{quote}
        \begin{Ventry}{xxxxxxx}

          \item[pattern]

          text of pattern ({\textless}string{\textgreater})

        \end{Ventry}

      \end{quote}

\textbf{Status:} Untested + NoDoc + NoDemo = NOT OK



    \end{boxedminipage}

    \label{xformslib:library:fl_refresh_fselector}
    \index{xformslib \textit{(package)}!xformslib.library \textit{(module)}!xformslib.library.fl\_refresh\_fselector \textit{(function)}}

    \vspace{0.5ex}

\hspace{.8\funcindent}\begin{boxedminipage}{\funcwidth}

    \raggedright \textbf{fl\_refresh\_fselector}()

    \vspace{-1.5ex}

    \rule{\textwidth}{0.5\fboxrule}
\setlength{\parskip}{2ex}
\setlength{\parskip}{1ex}
\textbf{Status:} Untested + NoDoc + NoDemo = NOT OK



    \end{boxedminipage}

    \label{xformslib:library:fl_add_fselector_appbutton}
    \index{xformslib \textit{(package)}!xformslib.library \textit{(module)}!xformslib.library.fl\_add\_fselector\_appbutton \textit{(function)}}

    \vspace{0.5ex}

\hspace{.8\funcindent}\begin{boxedminipage}{\funcwidth}

    \raggedright \textbf{fl\_add\_fselector\_appbutton}(\textit{label}, \textit{py\_fn}, \textit{vdata})

    \vspace{-1.5ex}

    \rule{\textwidth}{0.5\fboxrule}
\setlength{\parskip}{2ex}
\setlength{\parskip}{1ex}
      \textbf{Parameters}
      \vspace{-1ex}

      \begin{quote}
        \begin{Ventry}{xxxxx}

          \item[label]

          text of label ({\textless}string{\textgreater})

          \item[py\_fn]

          python function callback - no return

            {\it (type=\_\_ funcname (ptr\_void) \_\_)}

          \item[vdata]

          user data to be passed to function ({\textless}pointer to 
          void{\textgreater})

        \end{Ventry}

      \end{quote}

\textbf{Status:} Untested + NoDoc + NoDemo = NOT OK



    \end{boxedminipage}

    \label{xformslib:library:fl_remove_fselector_appbutton}
    \index{xformslib \textit{(package)}!xformslib.library \textit{(module)}!xformslib.library.fl\_remove\_fselector\_appbutton \textit{(function)}}

    \vspace{0.5ex}

\hspace{.8\funcindent}\begin{boxedminipage}{\funcwidth}

    \raggedright \textbf{fl\_remove\_fselector\_appbutton}(\textit{label})

    \vspace{-1.5ex}

    \rule{\textwidth}{0.5\fboxrule}
\setlength{\parskip}{2ex}
\setlength{\parskip}{1ex}
      \textbf{Parameters}
      \vspace{-1ex}

      \begin{quote}
        \begin{Ventry}{xxxxx}

          \item[label]

          text of label ({\textless}string{\textgreater})

        \end{Ventry}

      \end{quote}

\textbf{Status:} Untested + NoDoc + NoDemo = NOT OK



    \end{boxedminipage}

    \label{xformslib:library:fl_disable_fselector_cache}
    \index{xformslib \textit{(package)}!xformslib.library \textit{(module)}!xformslib.library.fl\_disable\_fselector\_cache \textit{(function)}}

    \vspace{0.5ex}

\hspace{.8\funcindent}\begin{boxedminipage}{\funcwidth}

    \raggedright \textbf{fl\_disable\_fselector\_cache}(\textit{yes})

    \vspace{-1.5ex}

    \rule{\textwidth}{0.5\fboxrule}
\setlength{\parskip}{2ex}
\setlength{\parskip}{1ex}
      \textbf{Parameters}
      \vspace{-1ex}

      \begin{quote}
        \begin{Ventry}{xxx}

          \item[yes]

          ({\textless}int{\textgreater})

        \end{Ventry}

      \end{quote}

\textbf{Status:} Untested + NoDoc + NoDemo = NOT OK



    \end{boxedminipage}

    \label{xformslib:library:fl_invalidate_fselector_cache}
    \index{xformslib \textit{(package)}!xformslib.library \textit{(module)}!xformslib.library.fl\_invalidate\_fselector\_cache \textit{(function)}}

    \vspace{0.5ex}

\hspace{.8\funcindent}\begin{boxedminipage}{\funcwidth}

    \raggedright \textbf{fl\_invalidate\_fselector\_cache}()

    \vspace{-1.5ex}

    \rule{\textwidth}{0.5\fboxrule}
\setlength{\parskip}{2ex}
\setlength{\parskip}{1ex}
\textbf{Status:} Untested + NoDoc + NoDemo = NOT OK



    \end{boxedminipage}

    \label{xformslib:library:fl_get_fselector_form}
    \index{xformslib \textit{(package)}!xformslib.library \textit{(module)}!xformslib.library.fl\_get\_fselector\_form \textit{(function)}}

    \vspace{0.5ex}

\hspace{.8\funcindent}\begin{boxedminipage}{\funcwidth}

    \raggedright \textbf{fl\_get\_fselector\_form}()

    \vspace{-1.5ex}

    \rule{\textwidth}{0.5\fboxrule}
\setlength{\parskip}{2ex}
\setlength{\parskip}{1ex}
      \textbf{Return Value}
    \vspace{-1ex}

      \begin{quote}
      pForm

      \end{quote}

\textbf{Status:} Untested + NoDoc + NoDemo = NOT OK



    \end{boxedminipage}

    \label{xformslib:library:fl_get_fselector_fdstruct}
    \index{xformslib \textit{(package)}!xformslib.library \textit{(module)}!xformslib.library.fl\_get\_fselector\_fdstruct \textit{(function)}}

    \vspace{0.5ex}

\hspace{.8\funcindent}\begin{boxedminipage}{\funcwidth}

    \raggedright \textbf{fl\_get\_fselector\_fdstruct}()

    \vspace{-1.5ex}

    \rule{\textwidth}{0.5\fboxrule}
\setlength{\parskip}{2ex}
\setlength{\parskip}{1ex}
      \textbf{Return Value}
    \vspace{-1ex}

      \begin{quote}
      pointer to xfdata.FD\_FSELECTOR

      {\it (type=fselector\_cls)}

      \end{quote}

\textbf{Status:} Untested + NoDoc + NoDemo = NOT OK



    \end{boxedminipage}

    \label{xformslib:library:fl_hide_fselector}
    \index{xformslib \textit{(package)}!xformslib.library \textit{(module)}!xformslib.library.fl\_hide\_fselector \textit{(function)}}

    \vspace{0.5ex}

\hspace{.8\funcindent}\begin{boxedminipage}{\funcwidth}

    \raggedright \textbf{fl\_hide\_fselector}()

    \vspace{-1.5ex}

    \rule{\textwidth}{0.5\fboxrule}
\setlength{\parskip}{2ex}
\setlength{\parskip}{1ex}
\textbf{Status:} Untested + NoDoc + NoDemo = NOT OK



    \end{boxedminipage}

    \label{xformslib:library:fl_set_fselector_filetype_marker}
    \index{xformslib \textit{(package)}!xformslib.library \textit{(module)}!xformslib.library.fl\_set\_fselector\_filetype\_marker \textit{(function)}}

    \vspace{0.5ex}

\hspace{.8\funcindent}\begin{boxedminipage}{\funcwidth}

    \raggedright \textbf{fl\_set\_fselector\_filetype\_marker}(\textit{p1}, \textit{p2}, \textit{p3}, \textit{p4}, \textit{p5})

    \vspace{-1.5ex}

    \rule{\textwidth}{0.5\fboxrule}
\setlength{\parskip}{2ex}
\setlength{\parskip}{1ex}
\textbf{Status:} Untested + NoDoc + NoDemo = NOT OK



    \end{boxedminipage}

    \label{xformslib:library:fl_set_fselector_title}
    \index{xformslib \textit{(package)}!xformslib.library \textit{(module)}!xformslib.library.fl\_set\_fselector\_title \textit{(function)}}

    \vspace{0.5ex}

\hspace{.8\funcindent}\begin{boxedminipage}{\funcwidth}

    \raggedright \textbf{fl\_set\_fselector\_title}(\textit{title})

    \vspace{-1.5ex}

    \rule{\textwidth}{0.5\fboxrule}
\setlength{\parskip}{2ex}
\setlength{\parskip}{1ex}
      \textbf{Parameters}
      \vspace{-1ex}

      \begin{quote}
        \begin{Ventry}{xxxxx}

          \item[title]

          title to be set ({\textless}string{\textgreater})

        \end{Ventry}

      \end{quote}

\textbf{Status:} Untested + NoDoc + NoDemo = NOT OK



    \end{boxedminipage}

    \label{xformslib:library:fl_goodies_atclose}
    \index{xformslib \textit{(package)}!xformslib.library \textit{(module)}!xformslib.library.fl\_goodies\_atclose \textit{(function)}}

    \vspace{0.5ex}

\hspace{.8\funcindent}\begin{boxedminipage}{\funcwidth}

    \raggedright \textbf{fl\_goodies\_atclose}(\textit{pForm}, \textit{vdata})

    \vspace{-1.5ex}

    \rule{\textwidth}{0.5\fboxrule}
\setlength{\parskip}{2ex}
\setlength{\parskip}{1ex}
      \textbf{Parameters}
      \vspace{-1ex}

      \begin{quote}
        \begin{Ventry}{xxxxx}

          \item[pForm]

          form ({\textless}pointer to xfdata.FL\_FORM{\textgreater})

        \end{Ventry}

      \end{quote}

      \textbf{Return Value}
    \vspace{-1ex}

      \begin{quote}
      num

      \end{quote}

\textbf{Status:} Untested + NoDoc + NoDemo = NOT OK



    \end{boxedminipage}

    \label{xformslib:library:fl_add_input}
    \index{xformslib \textit{(package)}!xformslib.library \textit{(module)}!xformslib.library.fl\_add\_input \textit{(function)}}

    \vspace{0.5ex}

\hspace{.8\funcindent}\begin{boxedminipage}{\funcwidth}

    \raggedright \textbf{fl\_add\_input}(\textit{inputtype}, \textit{x}, \textit{y}, \textit{w}, \textit{h}, \textit{label})

    \vspace{-1.5ex}

    \rule{\textwidth}{0.5\fboxrule}
\setlength{\parskip}{2ex}
    Adds an input object.

\setlength{\parskip}{1ex}
      \textbf{Parameters}
      \vspace{-1ex}

      \begin{quote}
        \begin{Ventry}{xxxxxxxxx}

          \item[inputtype]

          type of input to be added ({\textless}int{\textgreater})

            {\it (type=(from xfdata module) FL\_NORMAL\_INPUT, FL\_FLOAT\_INPUT, FL\_INT\_INPUT, 
FL\_DATE\_INPUT, FL\_MULTILINE\_INPUT, FL\_HIDDEN\_INPUT, FL\_SECRET\_INPUT)}

          \item[x]

          horizontal position (upper-left corner) 
          ({\textless}int{\textgreater})

          \item[x]

          vertical position (upper-left corner) 
          ({\textless}int{\textgreater})

          \item[w]

          width in coord units ({\textless}int{\textgreater})

          \item[h]

          height in coord units ({\textless}int{\textgreater})

          \item[label]

          text label of input ({\textless}string{\textgreater})

        \end{Ventry}

      \end{quote}

      \textbf{Return Value}
    \vspace{-1ex}

      \begin{quote}
      object created ({\textless}pointer to 
      xfdata.FL\_OBJECT{\textgreater})

      {\it (type=pObject)}

      \end{quote}

\textbf{Status:} Tested + Doc + Demo = OK



    \end{boxedminipage}

    \label{xformslib:library:fl_set_input}
    \index{xformslib \textit{(package)}!xformslib.library \textit{(module)}!xformslib.library.fl\_set\_input \textit{(function)}}

    \vspace{0.5ex}

\hspace{.8\funcindent}\begin{boxedminipage}{\funcwidth}

    \raggedright \textbf{fl\_set\_input}(\textit{pObject}, \textit{text})

    \vspace{-1.5ex}

    \rule{\textwidth}{0.5\fboxrule}
\setlength{\parskip}{2ex}
    Sets the particular input string, with no checks for validity. An empty
    string can be used to clear an input field.

\setlength{\parskip}{1ex}
      \textbf{Parameters}
      \vspace{-1ex}

      \begin{quote}
        \begin{Ventry}{xxxxxxx}

          \item[pObject]

          input object ({\textless}pointer to 
          xfdata.FL\_OBJECT{\textgreater})

          \item[text]

          input text ({\textless}string{\textgreater})

        \end{Ventry}

      \end{quote}

\textbf{Status:} Tested + Doc + Demo = OK



    \end{boxedminipage}

    \label{xformslib:library:fl_set_input_color}
    \index{xformslib \textit{(package)}!xformslib.library \textit{(module)}!xformslib.library.fl\_set\_input\_color \textit{(function)}}

    \vspace{0.5ex}

\hspace{.8\funcindent}\begin{boxedminipage}{\funcwidth}

    \raggedright \textbf{fl\_set\_input\_color}(\textit{pObject}, \textit{txtcolr}, \textit{curscolr})

    \vspace{-1.5ex}

    \rule{\textwidth}{0.5\fboxrule}
\setlength{\parskip}{2ex}
\setlength{\parskip}{1ex}
      \textbf{Parameters}
      \vspace{-1ex}

      \begin{quote}
        \begin{Ventry}{xxxxxxxx}

          \item[pObject]

          input object ({\textless}pointer to 
          xfdata.FL\_OBJECT{\textgreater})

          \item[txtcolr]

          color value for text {\textless}long(pos){\textgreater}

          \item[curscolr]

          color value for cursor {\textless}long(pos){\textgreater}

        \end{Ventry}

      \end{quote}

\textbf{Status:} Untested + NoDoc + NoDemo = NOT OK



    \end{boxedminipage}

    \label{xformslib:library:fl_get_input_color}
    \index{xformslib \textit{(package)}!xformslib.library \textit{(module)}!xformslib.library.fl\_get\_input\_color \textit{(function)}}

    \vspace{0.5ex}

\hspace{.8\funcindent}\begin{boxedminipage}{\funcwidth}

    \raggedright \textbf{fl\_get\_input\_color}(\textit{pObject})

    \vspace{-1.5ex}

    \rule{\textwidth}{0.5\fboxrule}
\setlength{\parskip}{2ex}
\setlength{\parskip}{1ex}
      \textbf{Parameters}
      \vspace{-1ex}

      \begin{quote}
        \begin{Ventry}{xxxxxxx}

          \item[pObject]

          input object ({\textless}pointer to 
          xfdata.FL\_OBJECT{\textgreater})

        \end{Ventry}

      \end{quote}

      \textbf{Return Value}
    \vspace{-1ex}

      \begin{quote}
      color value for text, color value for cursor {\textless}long(pos), 
      long(pos){\textgreater}

      {\it (type=txtcolr, curscolr)}

      \end{quote}

\textbf{Attention:} API change from XForms - upstream was fl\_get\_input\_color(pObject, 
textcolr, curscolr)



\textbf{Status:} Untested + NoDoc + NoDemo = NOT OK



    \end{boxedminipage}

    \label{xformslib:library:fl_set_input_scroll}
    \index{xformslib \textit{(package)}!xformslib.library \textit{(module)}!xformslib.library.fl\_set\_input\_scroll \textit{(function)}}

    \vspace{0.5ex}

\hspace{.8\funcindent}\begin{boxedminipage}{\funcwidth}

    \raggedright \textbf{fl\_set\_input\_scroll}(\textit{pObject}, \textit{yes})

    \vspace{-1.5ex}

    \rule{\textwidth}{0.5\fboxrule}
\setlength{\parskip}{2ex}
\setlength{\parskip}{1ex}
      \textbf{Parameters}
      \vspace{-1ex}

      \begin{quote}
        \begin{Ventry}{xxxxxxx}

          \item[pObject]

          input object ({\textless}pointer to 
          xfdata.FL\_OBJECT{\textgreater})

          \item[yes]

          ({\textless}int{\textgreater})

        \end{Ventry}

      \end{quote}

\textbf{Status:} Untested + NoDoc + NoDemo = NOT OK



    \end{boxedminipage}

    \label{xformslib:library:fl_set_input_cursorpos}
    \index{xformslib \textit{(package)}!xformslib.library \textit{(module)}!xformslib.library.fl\_set\_input\_cursorpos \textit{(function)}}

    \vspace{0.5ex}

\hspace{.8\funcindent}\begin{boxedminipage}{\funcwidth}

    \raggedright \textbf{fl\_set\_input\_cursorpos}(\textit{pObject}, \textit{xpos}, \textit{ypos})

    \vspace{-1.5ex}

    \rule{\textwidth}{0.5\fboxrule}
\setlength{\parskip}{2ex}
    Moves the cursor within the input field.

\setlength{\parskip}{1ex}
      \textbf{Parameters}
      \vspace{-1ex}

      \begin{quote}
        \begin{Ventry}{xxxxxxx}

          \item[pObject]

          input object ({\textless}pointer to 
          xfdata.FL\_OBJECT{\textgreater})

          \item[xpos]

          horizontal cursor position in characters 
          ({\textless}int{\textgreater})

          \item[ypos]

          vertical cursor position in lines ({\textless}int{\textgreater})

        \end{Ventry}

      \end{quote}

\textbf{Status:} Untested + Doc + NoDemo = NOT OK



    \end{boxedminipage}

    \label{xformslib:library:fl_set_input_selected}
    \index{xformslib \textit{(package)}!xformslib.library \textit{(module)}!xformslib.library.fl\_set\_input\_selected \textit{(function)}}

    \vspace{0.5ex}

\hspace{.8\funcindent}\begin{boxedminipage}{\funcwidth}

    \raggedright \textbf{fl\_set\_input\_selected}(\textit{pObject}, \textit{flag})

    \vspace{-1.5ex}

    \rule{\textwidth}{0.5\fboxrule}
\setlength{\parskip}{2ex}
    Selects or deselects the current input. It places the cursor at the end
    of the string when selected

\setlength{\parskip}{1ex}
      \textbf{Parameters}
      \vspace{-1ex}

      \begin{quote}
        \begin{Ventry}{xxxxxxx}

          \item[pObject]

          input object ({\textless}pointer to 
          xfdata.FL\_OBJECT{\textgreater})

          \item[flag]

          flag to deselect/select ({\textless}int{\textgreater})

            {\it (type=0 or 1)}

        \end{Ventry}

      \end{quote}

\textbf{Status:} Untested + NoDoc + NoDemo = NOT OK



    \end{boxedminipage}

    \label{xformslib:library:fl_set_input_selected_range}
    \index{xformslib \textit{(package)}!xformslib.library \textit{(module)}!xformslib.library.fl\_set\_input\_selected\_range \textit{(function)}}

    \vspace{0.5ex}

\hspace{.8\funcindent}\begin{boxedminipage}{\funcwidth}

    \raggedright \textbf{fl\_set\_input\_selected\_range}(\textit{pObject}, \textit{begin}, \textit{end})

    \vspace{-1.5ex}

    \rule{\textwidth}{0.5\fboxrule}
\setlength{\parskip}{2ex}
    Selects or deselects the current input of part of it. When begin is 0 
    and end is the last character number, all input is selected. It places 
    the cursor at the beginning of selected string.

\setlength{\parskip}{1ex}
      \textbf{Parameters}
      \vspace{-1ex}

      \begin{quote}
        \begin{Ventry}{xxxxxxx}

          \item[pObject]

          input object ({\textless}pointer to 
          xfdata.FL\_OBJECT{\textgreater})

          \item[begin]

          starting value in characters ({\textless}int{\textgreater})

          \item[end]

          ending value in characters ({\textless}int{\textgreater})

        \end{Ventry}

      \end{quote}

\textbf{Status:} Untested + Doc + NoDemo = NOT OK



    \end{boxedminipage}

    \label{xformslib:library:fl_get_input_selected_range}
    \index{xformslib \textit{(package)}!xformslib.library \textit{(module)}!xformslib.library.fl\_get\_input\_selected\_range \textit{(function)}}

    \vspace{0.5ex}

\hspace{.8\funcindent}\begin{boxedminipage}{\funcwidth}

    \raggedright \textbf{fl\_get\_input\_selected\_range}(\textit{pObject})

    \vspace{-1.5ex}

    \rule{\textwidth}{0.5\fboxrule}
\setlength{\parskip}{2ex}
    Obtains the currently selected range, either selected by the 
    application or by the user.

\setlength{\parskip}{1ex}
      \textbf{Parameters}
      \vspace{-1ex}

      \begin{quote}
        \begin{Ventry}{xxxxxxx}

          \item[pObject]

          input object ({\textless}pointer to 
          xfdata.FL\_OBJECT{\textgreater})

        \end{Ventry}

      \end{quote}

      \textbf{Return Value}
    \vspace{-1ex}

      \begin{quote}
      selected string, starting and ending values of selection in 
      characters [string, int\_num, int\_num]

      {\it (type=string, begin, end)}

      \end{quote}

\textbf{Attention:} API change from XForms - upstream was 
fl\_get\_input\_selected\_range(pObject, begin, end)



\textbf{Status:} Untested + Doc + NoDemo = NOT OK



    \end{boxedminipage}

    \label{xformslib:library:fl_set_input_maxchars}
    \index{xformslib \textit{(package)}!xformslib.library \textit{(module)}!xformslib.library.fl\_set\_input\_maxchars \textit{(function)}}

    \vspace{0.5ex}

\hspace{.8\funcindent}\begin{boxedminipage}{\funcwidth}

    \raggedright \textbf{fl\_set\_input\_maxchars}(\textit{pObject}, \textit{maxchars})

    \vspace{-1.5ex}

    \rule{\textwidth}{0.5\fboxrule}
\setlength{\parskip}{2ex}
    Limits the number of characters per line for input fields of type 
    xfdata.FL\_NORMAL\_INPUT. To reset the limit to infinite, set maxchars 
    to 0. Note that input objects of types xfdata.FL\_DATE\_INPUT are 
    limited to 10 characters per default and those of type 
    xfdata.FL\_SECRET\_INPUT to 16.

\setlength{\parskip}{1ex}
      \textbf{Parameters}
      \vspace{-1ex}

      \begin{quote}
        \begin{Ventry}{xxxxxxxx}

          \item[pObject]

          input object ({\textless}pointer to 
          xfdata.FL\_OBJECT{\textgreater})

          \item[maxchars]

          maximum characters to be set ({\textless}int{\textgreater})

        \end{Ventry}

      \end{quote}

\textbf{Status:} Untested + Doc + NoDemo = NOT OK



    \end{boxedminipage}

    \label{xformslib:library:fl_set_input_format}
    \index{xformslib \textit{(package)}!xformslib.library \textit{(module)}!xformslib.library.fl\_set\_input\_format \textit{(function)}}

    \vspace{0.5ex}

\hspace{.8\funcindent}\begin{boxedminipage}{\funcwidth}

    \raggedright \textbf{fl\_set\_input\_format}(\textit{pObject}, \textit{fmt}, \textit{sep})

    \vspace{-1.5ex}

    \rule{\textwidth}{0.5\fboxrule}
\setlength{\parskip}{2ex}
    Sets the format used for an input object. Currently used only for 
    xfdata.FL\_DATE\_INPUT objects.

\setlength{\parskip}{1ex}
      \textbf{Parameters}
      \vspace{-1ex}

      \begin{quote}
        \begin{Ventry}{xxx}

          \item[fmt]

          format for the input ({\textless}int{\textgreater})

            {\it (type=(from xfdata module) FL\_INPUT\_DDMM, FL\_INPUT\_MMDD)}

          \item[sep]

          printable single character used as separator {\textless}int or 
          char{\textgreater}

        \end{Ventry}

      \end{quote}

\textbf{Status:} Untested + Doc + NoDemo = NOT OK



    \end{boxedminipage}

    \label{xformslib:library:fl_set_input_hscrollbar}
    \index{xformslib \textit{(package)}!xformslib.library \textit{(module)}!xformslib.library.fl\_set\_input\_hscrollbar \textit{(function)}}

    \vspace{0.5ex}

\hspace{.8\funcindent}\begin{boxedminipage}{\funcwidth}

    \raggedright \textbf{fl\_set\_input\_hscrollbar}(\textit{pObject}, \textit{pref})

    \vspace{-1.5ex}

    \rule{\textwidth}{0.5\fboxrule}
\setlength{\parskip}{2ex}
\setlength{\parskip}{1ex}
      \textbf{Parameters}
      \vspace{-1ex}

      \begin{quote}
        \begin{Ventry}{xxxxxxx}

          \item[pObject]

          input object ({\textless}pointer to 
          xfdata.FL\_OBJECT{\textgreater})

          \item[pref]

          horizontal scrollbar flag ({\textless}int{\textgreater})

            {\it (type=(from xfdata module) FL\_AUTO, FL\_ON, FL\_OFF)}

        \end{Ventry}

      \end{quote}

\textbf{Status:} Untested + NoDoc + NoDemo = NOT OK



    \end{boxedminipage}

    \label{xformslib:library:fl_set_input_vscrollbar}
    \index{xformslib \textit{(package)}!xformslib.library \textit{(module)}!xformslib.library.fl\_set\_input\_vscrollbar \textit{(function)}}

    \vspace{0.5ex}

\hspace{.8\funcindent}\begin{boxedminipage}{\funcwidth}

    \raggedright \textbf{fl\_set\_input\_vscrollbar}(\textit{pObject}, \textit{pref})

    \vspace{-1.5ex}

    \rule{\textwidth}{0.5\fboxrule}
\setlength{\parskip}{2ex}
\setlength{\parskip}{1ex}
      \textbf{Parameters}
      \vspace{-1ex}

      \begin{quote}
        \begin{Ventry}{xxxxxxx}

          \item[pObject]

          input object ({\textless}pointer to 
          xfdata.FL\_OBJECT{\textgreater})

          \item[pref]

          vertical scrollbar flag ({\textless}int{\textgreater})

            {\it (type=(from xfdata module) FL\_AUTO, FL\_ON, FL\_OFF)}

        \end{Ventry}

      \end{quote}

\textbf{Status:} Untested + NoDoc + NoDemo = NOT OK



    \end{boxedminipage}

    \label{xformslib:library:fl_set_input_topline}
    \index{xformslib \textit{(package)}!xformslib.library \textit{(module)}!xformslib.library.fl\_set\_input\_topline \textit{(function)}}

    \vspace{0.5ex}

\hspace{.8\funcindent}\begin{boxedminipage}{\funcwidth}

    \raggedright \textbf{fl\_set\_input\_topline}(\textit{pObject}, \textit{top})

    \vspace{-1.5ex}

    \rule{\textwidth}{0.5\fboxrule}
\setlength{\parskip}{2ex}
\setlength{\parskip}{1ex}
      \textbf{Parameters}
      \vspace{-1ex}

      \begin{quote}
        \begin{Ventry}{xxxxxxx}

          \item[pObject]

          input object ({\textless}pointer to 
          xfdata.FL\_OBJECT{\textgreater})

          \item[top]

          ? ({\textless}int{\textgreater})

        \end{Ventry}

      \end{quote}

\textbf{Status:} Untested + NoDoc + NoDemo = NOT OK



    \end{boxedminipage}

    \label{xformslib:library:fl_set_input_scrollbarsize}
    \index{xformslib \textit{(package)}!xformslib.library \textit{(module)}!xformslib.library.fl\_set\_input\_scrollbarsize \textit{(function)}}

    \vspace{0.5ex}

\hspace{.8\funcindent}\begin{boxedminipage}{\funcwidth}

    \raggedright \textbf{fl\_set\_input\_scrollbarsize}(\textit{pObject}, \textit{hh}, \textit{vw})

    \vspace{-1.5ex}

    \rule{\textwidth}{0.5\fboxrule}
\setlength{\parskip}{2ex}
\setlength{\parskip}{1ex}
      \textbf{Parameters}
      \vspace{-1ex}

      \begin{quote}
        \begin{Ventry}{xxxxxxx}

          \item[pObject]

          input object ({\textless}pointer to 
          xfdata.FL\_OBJECT{\textgreater})

        \end{Ventry}

      \end{quote}

\textbf{Status:} Untested + NoDoc + NoDemo = NOT OK



    \end{boxedminipage}

    \label{xformslib:library:fl_get_input_scrollbarsize}
    \index{xformslib \textit{(package)}!xformslib.library \textit{(module)}!xformslib.library.fl\_get\_input\_scrollbarsize \textit{(function)}}

    \vspace{0.5ex}

\hspace{.8\funcindent}\begin{boxedminipage}{\funcwidth}

    \raggedright \textbf{fl\_get\_input\_scrollbarsize}(\textit{pObject})

    \vspace{-1.5ex}

    \rule{\textwidth}{0.5\fboxrule}
\setlength{\parskip}{2ex}
\setlength{\parskip}{1ex}
      \textbf{Return Value}
    \vspace{-1ex}

      \begin{quote}
      hh, vw

      \end{quote}

\textbf{Attention:} API change from XForms - upstream was 
fl\_get\_input\_scrollbarsize(pObject, hh, vw)



\textbf{Status:} Untested + NoDoc + NoDemo = NOT OK



    \end{boxedminipage}

    \label{xformslib:library:fl_set_input_xoffset}
    \index{xformslib \textit{(package)}!xformslib.library \textit{(module)}!xformslib.library.fl\_set\_input\_xoffset \textit{(function)}}

    \vspace{0.5ex}

\hspace{.8\funcindent}\begin{boxedminipage}{\funcwidth}

    \raggedright \textbf{fl\_set\_input\_xoffset}(\textit{pObject}, \textit{xoff})

    \vspace{-1.5ex}

    \rule{\textwidth}{0.5\fboxrule}
\setlength{\parskip}{2ex}
\setlength{\parskip}{1ex}
      \textbf{Parameters}
      \vspace{-1ex}

      \begin{quote}
        \begin{Ventry}{xxxxxxx}

          \item[pObject]

          input object ({\textless}pointer to 
          xfdata.FL\_OBJECT{\textgreater})

        \end{Ventry}

      \end{quote}

\textbf{Status:} Untested + NoDoc + NoDemo = NOT OK



    \end{boxedminipage}

    \label{xformslib:library:fl_get_input_xoffset}
    \index{xformslib \textit{(package)}!xformslib.library \textit{(module)}!xformslib.library.fl\_get\_input\_xoffset \textit{(function)}}

    \vspace{0.5ex}

\hspace{.8\funcindent}\begin{boxedminipage}{\funcwidth}

    \raggedright \textbf{fl\_get\_input\_xoffset}(\textit{pObject})

    \vspace{-1.5ex}

    \rule{\textwidth}{0.5\fboxrule}
\setlength{\parskip}{2ex}
\setlength{\parskip}{1ex}
      \textbf{Parameters}
      \vspace{-1ex}

      \begin{quote}
        \begin{Ventry}{xxxxxxx}

          \item[pObject]

          input object ({\textless}pointer to 
          xfdata.FL\_OBJECT{\textgreater})

        \end{Ventry}

      \end{quote}

      \textbf{Return Value}
    \vspace{-1ex}

      \begin{quote}
      num

      \end{quote}

\textbf{Status:} Untested + NoDoc + NoDemo = NOT OK



    \end{boxedminipage}

    \label{xformslib:library:fl_set_input_fieldchar}
    \index{xformslib \textit{(package)}!xformslib.library \textit{(module)}!xformslib.library.fl\_set\_input\_fieldchar \textit{(function)}}

    \vspace{0.5ex}

\hspace{.8\funcindent}\begin{boxedminipage}{\funcwidth}

    \raggedright \textbf{fl\_set\_input\_fieldchar}(\textit{pObject}, \textit{fldchar})

    \vspace{-1.5ex}

    \rule{\textwidth}{0.5\fboxrule}
\setlength{\parskip}{2ex}
\setlength{\parskip}{1ex}
      \textbf{Parameters}
      \vspace{-1ex}

      \begin{quote}
        \begin{Ventry}{xxxxxxx}

          \item[pObject]

          input object ({\textless}pointer to 
          xfdata.FL\_OBJECT{\textgreater})

        \end{Ventry}

      \end{quote}

      \textbf{Return Value}
    \vspace{-1ex}

      \begin{quote}
      num

      \end{quote}

\textbf{Status:} Untested + NoDoc + NoDemo = NOT OK



    \end{boxedminipage}

    \label{xformslib:library:fl_get_input_topline}
    \index{xformslib \textit{(package)}!xformslib.library \textit{(module)}!xformslib.library.fl\_get\_input\_topline \textit{(function)}}

    \vspace{0.5ex}

\hspace{.8\funcindent}\begin{boxedminipage}{\funcwidth}

    \raggedright \textbf{fl\_get\_input\_topline}(\textit{pObject})

    \vspace{-1.5ex}

    \rule{\textwidth}{0.5\fboxrule}
\setlength{\parskip}{2ex}
\setlength{\parskip}{1ex}
      \textbf{Parameters}
      \vspace{-1ex}

      \begin{quote}
        \begin{Ventry}{xxxxxxx}

          \item[pObject]

          input object ({\textless}pointer to 
          xfdata.FL\_OBJECT{\textgreater})

        \end{Ventry}

      \end{quote}

      \textbf{Return Value}
    \vspace{-1ex}

      \begin{quote}
      num

      \end{quote}

\textbf{Status:} Untested + NoDoc + NoDemo = NOT OK



    \end{boxedminipage}

    \label{xformslib:library:fl_get_input_screenlines}
    \index{xformslib \textit{(package)}!xformslib.library \textit{(module)}!xformslib.library.fl\_get\_input\_screenlines \textit{(function)}}

    \vspace{0.5ex}

\hspace{.8\funcindent}\begin{boxedminipage}{\funcwidth}

    \raggedright \textbf{fl\_get\_input\_screenlines}(\textit{pObject})

    \vspace{-1.5ex}

    \rule{\textwidth}{0.5\fboxrule}
\setlength{\parskip}{2ex}
\setlength{\parskip}{1ex}
      \textbf{Parameters}
      \vspace{-1ex}

      \begin{quote}
        \begin{Ventry}{xxxxxxx}

          \item[pObject]

          input object ({\textless}pointer to 
          xfdata.FL\_OBJECT{\textgreater})

        \end{Ventry}

      \end{quote}

      \textbf{Return Value}
    \vspace{-1ex}

      \begin{quote}
      num

      \end{quote}

\textbf{Status:} Untested + NoDoc + NoDemo = NOT OK



    \end{boxedminipage}

    \label{xformslib:library:fl_get_input_cursorpos}
    \index{xformslib \textit{(package)}!xformslib.library \textit{(module)}!xformslib.library.fl\_get\_input\_cursorpos \textit{(function)}}

    \vspace{0.5ex}

\hspace{.8\funcindent}\begin{boxedminipage}{\funcwidth}

    \raggedright \textbf{fl\_get\_input\_cursorpos}(\textit{pObject})

    \vspace{-1.5ex}

    \rule{\textwidth}{0.5\fboxrule}
\setlength{\parskip}{2ex}
\setlength{\parskip}{1ex}
      \textbf{Parameters}
      \vspace{-1ex}

      \begin{quote}
        \begin{Ventry}{xxxxxxx}

          \item[pObject]

          input object ({\textless}pointer to 
          xfdata.FL\_OBJECT{\textgreater})

        \end{Ventry}

      \end{quote}

      \textbf{Return Value}
    \vspace{-1ex}

      \begin{quote}
      num., x, y

      \end{quote}

\textbf{Attention:} API change from XForms - upstream was fl\_get\_input\_cursorpos(pObject, x,
y)



\textbf{Status:} Tested + NoDoc + Demo = OK



    \end{boxedminipage}

    \label{xformslib:library:fl_set_input_cursor_visible}
    \index{xformslib \textit{(package)}!xformslib.library \textit{(module)}!xformslib.library.fl\_set\_input\_cursor\_visible \textit{(function)}}

    \vspace{0.5ex}

\hspace{.8\funcindent}\begin{boxedminipage}{\funcwidth}

    \raggedright \textbf{fl\_set\_input\_cursor\_visible}(\textit{pObject}, \textit{visible})

    \vspace{-1.5ex}

    \rule{\textwidth}{0.5\fboxrule}
\setlength{\parskip}{2ex}
\setlength{\parskip}{1ex}
      \textbf{Parameters}
      \vspace{-1ex}

      \begin{quote}
        \begin{Ventry}{xxxxxxx}

          \item[pObject]

          input object ({\textless}pointer to 
          xfdata.FL\_OBJECT{\textgreater})

        \end{Ventry}

      \end{quote}

\textbf{Status:} Untested + NoDoc + NoDemo = NOT OK



    \end{boxedminipage}

    \label{xformslib:library:fl_get_input_numberoflines}
    \index{xformslib \textit{(package)}!xformslib.library \textit{(module)}!xformslib.library.fl\_get\_input\_numberoflines \textit{(function)}}

    \vspace{0.5ex}

\hspace{.8\funcindent}\begin{boxedminipage}{\funcwidth}

    \raggedright \textbf{fl\_get\_input\_numberoflines}(\textit{pObject})

    \vspace{-1.5ex}

    \rule{\textwidth}{0.5\fboxrule}
\setlength{\parskip}{2ex}
\setlength{\parskip}{1ex}
      \textbf{Parameters}
      \vspace{-1ex}

      \begin{quote}
        \begin{Ventry}{xxxxxxx}

          \item[pObject]

          input object ({\textless}pointer to 
          xfdata.FL\_OBJECT{\textgreater})

        \end{Ventry}

      \end{quote}

      \textbf{Return Value}
    \vspace{-1ex}

      \begin{quote}
      lines num

      \end{quote}

\textbf{Status:} Untested + NoDoc + NoDemo = NOT OK



    \end{boxedminipage}

    \label{xformslib:library:fl_get_input_format}
    \index{xformslib \textit{(package)}!xformslib.library \textit{(module)}!xformslib.library.fl\_get\_input\_format \textit{(function)}}

    \vspace{0.5ex}

\hspace{.8\funcindent}\begin{boxedminipage}{\funcwidth}

    \raggedright \textbf{fl\_get\_input\_format}(\textit{pObject})

    \vspace{-1.5ex}

    \rule{\textwidth}{0.5\fboxrule}
\setlength{\parskip}{2ex}
\setlength{\parskip}{1ex}
      \textbf{Parameters}
      \vspace{-1ex}

      \begin{quote}
        \begin{Ventry}{xxxxxxx}

          \item[pObject]

          input object ({\textless}pointer to 
          xfdata.FL\_OBJECT{\textgreater})

        \end{Ventry}

      \end{quote}

      \textbf{Return Value}
    \vspace{-1ex}

      \begin{quote}
      fmt, sep

      \end{quote}

\textbf{Attention:} API change from XForms - upstream was fl\_get\_input\_format(pObject, fmt, 
sep)



\textbf{Status:} Untested + NoDoc + NoDemo = NOT OK



    \end{boxedminipage}

    \label{xformslib:library:fl_get_input}
    \index{xformslib \textit{(package)}!xformslib.library \textit{(module)}!xformslib.library.fl\_get\_input \textit{(function)}}

    \vspace{0.5ex}

\hspace{.8\funcindent}\begin{boxedminipage}{\funcwidth}

    \raggedright \textbf{fl\_get\_input}(\textit{pObject})

    \vspace{-1.5ex}

    \rule{\textwidth}{0.5\fboxrule}
\setlength{\parskip}{2ex}
\setlength{\parskip}{1ex}
      \textbf{Parameters}
      \vspace{-1ex}

      \begin{quote}
        \begin{Ventry}{xxxxxxx}

          \item[pObject]

          input object ({\textless}pointer to 
          xfdata.FL\_OBJECT{\textgreater})

        \end{Ventry}

      \end{quote}

      \textbf{Return Value}
    \vspace{-1ex}

      \begin{quote}
      input string

      \end{quote}

\textbf{Status:} Tested + NoDoc + Demo = OK



    \end{boxedminipage}

    \label{xformslib:library:fl_set_input_filter}
    \index{xformslib \textit{(package)}!xformslib.library \textit{(module)}!xformslib.library.fl\_set\_input\_filter \textit{(function)}}

    \vspace{0.5ex}

\hspace{.8\funcindent}\begin{boxedminipage}{\funcwidth}

    \raggedright \textbf{fl\_set\_input\_filter}(\textit{pObject}, \textit{py\_InputValidator})

    \vspace{-1.5ex}

    \rule{\textwidth}{0.5\fboxrule}
\setlength{\parskip}{2ex}
\setlength{\parskip}{1ex}
      \textbf{Parameters}
      \vspace{-1ex}

      \begin{quote}
        \begin{Ventry}{xxxxxxx}

          \item[pObject]

          input object ({\textless}pointer to 
          xfdata.FL\_OBJECT{\textgreater})

        \end{Ventry}

      \end{quote}

      \textbf{Return Value}
    \vspace{-1ex}

      \begin{quote}
      input\_filter func

      \end{quote}

\textbf{Status:} Untested + NoDoc + NoDemo = NOT OK



    \end{boxedminipage}

    \label{xformslib:library:fl_validate_input}
    \index{xformslib \textit{(package)}!xformslib.library \textit{(module)}!xformslib.library.fl\_validate\_input \textit{(function)}}

    \vspace{0.5ex}

\hspace{.8\funcindent}\begin{boxedminipage}{\funcwidth}

    \raggedright \textbf{fl\_validate\_input}(\textit{pObject})

    \vspace{-1.5ex}

    \rule{\textwidth}{0.5\fboxrule}
\setlength{\parskip}{2ex}
    Tests if the value in an input field is valid. It returns 
    xfdata.FL\_VALID if the value is valid or if there is no validator 
    function set for the input, otherwise xfdata.FL\_INVALID.

\setlength{\parskip}{1ex}
      \textbf{Parameters}
      \vspace{-1ex}

      \begin{quote}
        \begin{Ventry}{xxxxxxx}

          \item[pObject]

          input object ({\textless}pointer to 
          xfdata.FL\_OBJECT{\textgreater})

        \end{Ventry}

      \end{quote}

      \textbf{Return Value}
    \vspace{-1ex}

      \begin{quote}
      (from xfdata module) FL\_VALID or FL\_INVALID 
      ({\textless}int{\textgreater})

      {\it (type=num)}

      \end{quote}

\textbf{Status:} Untested + Doc + NoDemo = NOT OK



    \end{boxedminipage}

    \label{xformslib:library:fl_set_object_shortcut}
    \index{xformslib \textit{(package)}!xformslib.library \textit{(module)}!xformslib.library.fl\_set\_object\_shortcut \textit{(function)}}

    \vspace{0.5ex}

\hspace{.8\funcindent}\begin{boxedminipage}{\funcwidth}

    \raggedright \textbf{fl\_set\_input\_shortcut}(\textit{pObject}, \textit{shctxt}, \textit{showit})

    \vspace{-1.5ex}

    \rule{\textwidth}{0.5\fboxrule}
\setlength{\parskip}{2ex}
    Sets a shortcut, binding a key or a series of keys to an object. It 
    resets any previous defined shortcuts for the object. Using e.g. 
    "acE\#d{\textasciicircum}h" the keys 'a', 'c', 'E', 
    {\textless}Alt{\textgreater}d and {\textless}Ctrl{\textgreater}h are 
    associated with the object. The precise format is as follows: any 
    character in the string is considered as a shortcut, except 
    '{\textasciicircum}' and '\#', which stand for combinations with the 
    {\textless}Ctrl{\textgreater} and {\textless}Alt{\textgreater} keys. 
    (the case of the key following '\#' or '{\textasciicircum}' is not 
    important, i.e. no distiction is made between e.g. 
    "{\textasciicircum}C" and "{\textasciicircum}c", both encode the key 
    combination {\textless}Ctrl{\textgreater}C as well as 
    {\textless}Ctrl{\textgreater}C.) The key '{\textasciicircum}' itself 
    can be set as a shortcut key by using 
    "{\textasciicircum}{\textasciicircum}" in the string defining the 
    shortcut. The key '\#' can be obtained as a shortcut by using the 
    string "{\textasciicircum}\#". So, e.g. "\#{\textasciicircum}\#" 
    encodes {\textless}ALT{\textgreater}\#. The 
    {\textless}Esc{\textgreater} key can be given as "{\textasciicircum}[".
    Another special character not mentioned yet is '\&', which indicates 
    function and arrow keys. Use a sequence starting with '\&' and directly
    followed by a number between 1 and 35 to represent one of the function 
    keys. For example, "\&2" stands for the {\textless}F2{\textgreater} 
    function key. The four cursors keys (up, down, right, and left) can be 
    given as "\&A", "\&B", "\&C" and "\&D", respectively. The key '\&' 
    itself can be obtained as a shortcut by prefixing it with 
    '{\textasciicircum}'.

\setlength{\parskip}{1ex}
      \textbf{Parameters}
      \vspace{-1ex}

      \begin{quote}
        \begin{Ventry}{xxxxxxx}

          \item[pObject]

          object ({\textless}pointer to xfdata.FL\_OBJECT{\textgreater})

          \item[shctxt]

          shortcut text to be set ({\textless}string{\textgreater})

          \item[showit]

          flag if shortcut letter has to be underlined or not if a match 
          exists (only the 1st alphanumeric character is used.

            {\it (type=0 (underline not shown) or 1 (shown))}

        \end{Ventry}

      \end{quote}

\textbf{Example:} fl\_set\_object\_shortcut(pobj6, "aA\#A{\textasciicircum}A", 1)



\textbf{Status:} Tested + Doc + NoDemo = OK



    \end{boxedminipage}

    \label{xformslib:library:fl_set_input_editkeymap}
    \index{xformslib \textit{(package)}!xformslib.library \textit{(module)}!xformslib.library.fl\_set\_input\_editkeymap \textit{(function)}}

    \vspace{0.5ex}

\hspace{.8\funcindent}\begin{boxedminipage}{\funcwidth}

    \raggedright \textbf{fl\_set\_input\_editkeymap}(\textit{pEditKeymap})

    \vspace{-1.5ex}

    \rule{\textwidth}{0.5\fboxrule}
\setlength{\parskip}{2ex}
\setlength{\parskip}{1ex}
      \textbf{Parameters}
      \vspace{-1ex}

      \begin{quote}
        \begin{Ventry}{xxxxxxxxxxx}

          \item[pEditKeymap]

          EditKeyMap class [pointer to xfdata.FL\_EditKeymap]

        \end{Ventry}

      \end{quote}

\textbf{Status:} Untested + NoDoc + NoDemo = NOT OK



    \end{boxedminipage}

    \label{xformslib:library:fl_add_nmenu}
    \index{xformslib \textit{(package)}!xformslib.library \textit{(module)}!xformslib.library.fl\_add\_nmenu \textit{(function)}}

    \vspace{0.5ex}

\hspace{.8\funcindent}\begin{boxedminipage}{\funcwidth}

    \raggedright \textbf{fl\_add\_nmenu}(\textit{nmenutype}, \textit{x}, \textit{y}, \textit{w}, \textit{h}, \textit{label})

    \vspace{-1.5ex}

    \rule{\textwidth}{0.5\fboxrule}
\setlength{\parskip}{2ex}
    Adds a nmenu object.

\setlength{\parskip}{1ex}
      \textbf{Parameters}
      \vspace{-1ex}

      \begin{quote}
        \begin{Ventry}{xxxxxxxxx}

          \item[nmenutype]

          type of nmenu to be added

          \item[x]

          horizontal position (upper-left corner)

          \item[y]

          vertical position (upper-left corner)

          \item[w]

          width in coord units

          \item[h]

          height in coord units

          \item[label]

          text label of nmenu object

        \end{Ventry}

      \end{quote}

      \textbf{Return Value}
    \vspace{-1ex}

      \begin{quote}
      pObject

      \end{quote}

\textbf{Status:} Tested + NoDoc + Demo = OK



    \end{boxedminipage}

    \label{xformslib:library:fl_clear_nmenu}
    \index{xformslib \textit{(package)}!xformslib.library \textit{(module)}!xformslib.library.fl\_clear\_nmenu \textit{(function)}}

    \vspace{0.5ex}

\hspace{.8\funcindent}\begin{boxedminipage}{\funcwidth}

    \raggedright \textbf{fl\_clear\_nmenu}(\textit{pObject})

    \vspace{-1.5ex}

    \rule{\textwidth}{0.5\fboxrule}
\setlength{\parskip}{2ex}
\setlength{\parskip}{1ex}
      \textbf{Return Value}
    \vspace{-1ex}

      \begin{quote}
      num

      \end{quote}

\textbf{Status:} Untested + NoDoc + NoDemo = NOT OK



    \end{boxedminipage}

    \label{xformslib:library:fl_add_nmenu_items}
    \index{xformslib \textit{(package)}!xformslib.library \textit{(module)}!xformslib.library.fl\_add\_nmenu\_items \textit{(function)}}

    \vspace{0.5ex}

\hspace{.8\funcindent}\begin{boxedminipage}{\funcwidth}

    \raggedright \textbf{fl\_add\_nmenu\_items}(\textit{pObject}, \textit{itemstr})

    \vspace{-1.5ex}

    \rule{\textwidth}{0.5\fboxrule}
\setlength{\parskip}{2ex}
\setlength{\parskip}{1ex}
      \textbf{Parameters}
      \vspace{-1ex}

      \begin{quote}
        \begin{Ventry}{xxxxxxx}

          \item[pObject]

          nmenu object ({\textless}pointer to 
          xfdata.FL\_OBJECT{\textgreater})

        \end{Ventry}

      \end{quote}

      \textbf{Return Value}
    \vspace{-1ex}

      \begin{quote}
      pPopupEntry

      \end{quote}

\textbf{Status:} HalfTested + NoDoc + Demo = NOT OK (sequence param.)



    \end{boxedminipage}

    \label{xformslib:library:fl_insert_nmenu_items}
    \index{xformslib \textit{(package)}!xformslib.library \textit{(module)}!xformslib.library.fl\_insert\_nmenu\_items \textit{(function)}}

    \vspace{0.5ex}

\hspace{.8\funcindent}\begin{boxedminipage}{\funcwidth}

    \raggedright \textbf{fl\_insert\_nmenu\_items}(\textit{pObject}, \textit{pPopupEntry}, \textit{itemstr})

    \vspace{-1.5ex}

    \rule{\textwidth}{0.5\fboxrule}
\setlength{\parskip}{2ex}
\setlength{\parskip}{1ex}
      \textbf{Parameters}
      \vspace{-1ex}

      \begin{quote}
        \begin{Ventry}{xxxxxxx}

          \item[pObject]

          nmenu object ({\textless}pointer to 
          xfdata.FL\_OBJECT{\textgreater})

          \item[itemstr]

          text of the item (among special sequences only \%S is supported)

        \end{Ventry}

      \end{quote}

      \textbf{Return Value}
    \vspace{-1ex}

      \begin{quote}
      pPopupEntry

      \end{quote}

\textbf{Status:} HalfTested + NoDoc + Demo = NOT OK (special sequences)



    \end{boxedminipage}

    \label{xformslib:library:fl_replace_nmenu_item}
    \index{xformslib \textit{(package)}!xformslib.library \textit{(module)}!xformslib.library.fl\_replace\_nmenu\_item \textit{(function)}}

    \vspace{0.5ex}

\hspace{.8\funcindent}\begin{boxedminipage}{\funcwidth}

    \raggedright \textbf{fl\_replace\_nmenu\_item}(\textit{pObject}, \textit{pPopupEntry}, \textit{itemstr})

    \vspace{-1.5ex}

    \rule{\textwidth}{0.5\fboxrule}
\setlength{\parskip}{2ex}
\setlength{\parskip}{1ex}
      \textbf{Parameters}
      \vspace{-1ex}

      \begin{quote}
        \begin{Ventry}{xxxxxxx}

          \item[pObject]

          nmenu object ({\textless}pointer to 
          xfdata.FL\_OBJECT{\textgreater})

        \end{Ventry}

      \end{quote}

      \textbf{Return Value}
    \vspace{-1ex}

      \begin{quote}
      pPopupEntry

      \end{quote}

\textbf{Status:} Untested + NoDoc + NoDemo = NOT OK



    \end{boxedminipage}

    \label{xformslib:library:fl_delete_nmenu_item}
    \index{xformslib \textit{(package)}!xformslib.library \textit{(module)}!xformslib.library.fl\_delete\_nmenu\_item \textit{(function)}}

    \vspace{0.5ex}

\hspace{.8\funcindent}\begin{boxedminipage}{\funcwidth}

    \raggedright \textbf{fl\_delete\_nmenu\_item}(\textit{pObject}, \textit{pPopupEntry})

    \vspace{-1.5ex}

    \rule{\textwidth}{0.5\fboxrule}
\setlength{\parskip}{2ex}
\setlength{\parskip}{1ex}
      \textbf{Return Value}
    \vspace{-1ex}

      \begin{quote}
      num

      \end{quote}

\textbf{Status:} Untested + NoDoc + NoDemo = NOT OK



    \end{boxedminipage}

    \label{xformslib:library:fl_set_nmenu_items}
    \index{xformslib \textit{(package)}!xformslib.library \textit{(module)}!xformslib.library.fl\_set\_nmenu\_items \textit{(function)}}

    \vspace{0.5ex}

\hspace{.8\funcindent}\begin{boxedminipage}{\funcwidth}

    \raggedright \textbf{fl\_set\_nmenu\_items}(\textit{pObject}, \textit{pPopupItem})

    \vspace{-1.5ex}

    \rule{\textwidth}{0.5\fboxrule}
\setlength{\parskip}{2ex}
\setlength{\parskip}{1ex}
      \textbf{Return Value}
    \vspace{-1ex}

      \begin{quote}
      pPopupEntry

      \end{quote}

\textbf{Status:} Untested + NoDoc + NoDemo = NOT OK



    \end{boxedminipage}

    \label{xformslib:library:fl_add_nmenu_items2}
    \index{xformslib \textit{(package)}!xformslib.library \textit{(module)}!xformslib.library.fl\_add\_nmenu\_items2 \textit{(function)}}

    \vspace{0.5ex}

\hspace{.8\funcindent}\begin{boxedminipage}{\funcwidth}

    \raggedright \textbf{fl\_add\_nmenu\_items2}(\textit{pObject}, \textit{pPopupItem})

    \vspace{-1.5ex}

    \rule{\textwidth}{0.5\fboxrule}
\setlength{\parskip}{2ex}
\setlength{\parskip}{1ex}
      \textbf{Parameters}
      \vspace{-1ex}

      \begin{quote}
        \begin{Ventry}{xxxxxxxxxx}

          \item[pObject]

          nmenu object ({\textless}pointer to 
          xfdata.FL\_OBJECT{\textgreater})

          \item[pPopupItem]

          pointer to xfc.FL\_POPUP\_ITEM; it needs to be prepared 
          beforehand with make\_pPopupItem\_from\_list(..) function for 
          single or multiple lists, or with 
          make\_pPopupItem\_from\_dict(..) for a single dict.

        \end{Ventry}

      \end{quote}

      \textbf{Return Value}
    \vspace{-1ex}

      \begin{quote}
      pPopupEntry

      \end{quote}

\textbf{Status:} Tested + NoDoc + Demo = OK



    \end{boxedminipage}

    \label{xformslib:library:fl_insert_nmenu_items2}
    \index{xformslib \textit{(package)}!xformslib.library \textit{(module)}!xformslib.library.fl\_insert\_nmenu\_items2 \textit{(function)}}

    \vspace{0.5ex}

\hspace{.8\funcindent}\begin{boxedminipage}{\funcwidth}

    \raggedright \textbf{fl\_insert\_nmenu\_items2}(\textit{pObject}, \textit{pPopupEntry}, \textit{pPopupItem})

    \vspace{-1.5ex}

    \rule{\textwidth}{0.5\fboxrule}
\setlength{\parskip}{2ex}
\setlength{\parskip}{1ex}
      \textbf{Parameters}
      \vspace{-1ex}

      \begin{quote}
        \begin{Ventry}{xxxxxxxxxx}

          \item[pObject]

          nmenu object ({\textless}pointer to 
          xfdata.FL\_OBJECT{\textgreater})

          \item[pPopupItem]

          pointer to xfc.FL\_POPUP\_ITEM; it needs to be prepared 
          beforehand with make\_pPopupItem\_from\_list(..) function for 
          single or multiple lists, or with 
          make\_pPopupItem\_from\_dict(..) for a single dict.

        \end{Ventry}

      \end{quote}

      \textbf{Return Value}
    \vspace{-1ex}

      \begin{quote}
      pPopupEntry

      \end{quote}

\textbf{Status:} Tested + NoDoc + Demo = OK



    \end{boxedminipage}

    \label{xformslib:library:fl_replace_nmenu_items2}
    \index{xformslib \textit{(package)}!xformslib.library \textit{(module)}!xformslib.library.fl\_replace\_nmenu\_items2 \textit{(function)}}

    \vspace{0.5ex}

\hspace{.8\funcindent}\begin{boxedminipage}{\funcwidth}

    \raggedright \textbf{fl\_replace\_nmenu\_items2}(\textit{pObject}, \textit{pPopupEntry}, \textit{pPopupItem})

    \vspace{-1.5ex}

    \rule{\textwidth}{0.5\fboxrule}
\setlength{\parskip}{2ex}
\setlength{\parskip}{1ex}
      \textbf{Parameters}
      \vspace{-1ex}

      \begin{quote}
        \begin{Ventry}{xxxxxxxxxx}

          \item[pObject]

          nmenu object ({\textless}pointer to 
          xfdata.FL\_OBJECT{\textgreater})

          \item[pPopupItem]

          pointer to xfc.FL\_POPUP\_ITEM; it needs to be prepared 
          beforehand with make\_pPopupItem\_from\_list(..) function for for
          single or multiple lists, or with 
          make\_pPopupItem\_from\_dict(..) for a single dict.

        \end{Ventry}

      \end{quote}

      \textbf{Return Value}
    \vspace{-1ex}

      \begin{quote}
      pPopupEntry

      \end{quote}

\textbf{Status:} Tested + NoDoc + Demo = OK



    \end{boxedminipage}

    \label{xformslib:library:fl_get_nmenu_popup}
    \index{xformslib \textit{(package)}!xformslib.library \textit{(module)}!xformslib.library.fl\_get\_nmenu\_popup \textit{(function)}}

    \vspace{0.5ex}

\hspace{.8\funcindent}\begin{boxedminipage}{\funcwidth}

    \raggedright \textbf{fl\_get\_nmenu\_popup}(\textit{pObject})

    \vspace{-1.5ex}

    \rule{\textwidth}{0.5\fboxrule}
\setlength{\parskip}{2ex}
\setlength{\parskip}{1ex}
      \textbf{Parameters}
      \vspace{-1ex}

      \begin{quote}
        \begin{Ventry}{xxxxxxx}

          \item[pObject]

          nmenu object ({\textless}pointer to 
          xfdata.FL\_OBJECT{\textgreater})

        \end{Ventry}

      \end{quote}

      \textbf{Return Value}
    \vspace{-1ex}

      \begin{quote}
      pPopup

      \end{quote}

\textbf{Status:} Untested + NoDoc + NoDemo = NOT OK



    \end{boxedminipage}

    \label{xformslib:library:fl_set_nmenu_popup}
    \index{xformslib \textit{(package)}!xformslib.library \textit{(module)}!xformslib.library.fl\_set\_nmenu\_popup \textit{(function)}}

    \vspace{0.5ex}

\hspace{.8\funcindent}\begin{boxedminipage}{\funcwidth}

    \raggedright \textbf{fl\_set\_nmenu\_popup}(\textit{pObject}, \textit{pPopup})

    \vspace{-1.5ex}

    \rule{\textwidth}{0.5\fboxrule}
\setlength{\parskip}{2ex}
\setlength{\parskip}{1ex}
      \textbf{Parameters}
      \vspace{-1ex}

      \begin{quote}
        \begin{Ventry}{xxxxxxx}

          \item[pObject]

          nmenu object ({\textless}pointer to 
          xfdata.FL\_OBJECT{\textgreater})

        \end{Ventry}

      \end{quote}

      \textbf{Return Value}
    \vspace{-1ex}

      \begin{quote}
      num

      \end{quote}

\textbf{Status:} Untested + NoDoc + NoDemo = NOT OK



    \end{boxedminipage}

    \label{xformslib:library:fl_get_nmenu_item}
    \index{xformslib \textit{(package)}!xformslib.library \textit{(module)}!xformslib.library.fl\_get\_nmenu\_item \textit{(function)}}

    \vspace{0.5ex}

\hspace{.8\funcindent}\begin{boxedminipage}{\funcwidth}

    \raggedright \textbf{fl\_get\_nmenu\_item}(\textit{pObject})

    \vspace{-1.5ex}

    \rule{\textwidth}{0.5\fboxrule}
\setlength{\parskip}{2ex}
\setlength{\parskip}{1ex}
      \textbf{Parameters}
      \vspace{-1ex}

      \begin{quote}
        \begin{Ventry}{xxxxxxx}

          \item[pObject]

          nmenu object ({\textless}pointer to 
          xfdata.FL\_OBJECT{\textgreater})

        \end{Ventry}

      \end{quote}

      \textbf{Return Value}
    \vspace{-1ex}

      \begin{quote}
      pPopupReturn

      \end{quote}

\textbf{Status:} Tested + NoDoc + Demo = OK



    \end{boxedminipage}

    \label{xformslib:library:fl_get_nmenu_item_by_value}
    \index{xformslib \textit{(package)}!xformslib.library \textit{(module)}!xformslib.library.fl\_get\_nmenu\_item\_by\_value \textit{(function)}}

    \vspace{0.5ex}

\hspace{.8\funcindent}\begin{boxedminipage}{\funcwidth}

    \raggedright \textbf{fl\_get\_nmenu\_item\_by\_value}(\textit{pObject}, \textit{value})

    \vspace{-1.5ex}

    \rule{\textwidth}{0.5\fboxrule}
\setlength{\parskip}{2ex}
\setlength{\parskip}{1ex}
      \textbf{Parameters}
      \vspace{-1ex}

      \begin{quote}
        \begin{Ventry}{xxxxxxx}

          \item[pObject]

          nmenu object ({\textless}pointer to 
          xfdata.FL\_OBJECT{\textgreater})

        \end{Ventry}

      \end{quote}

      \textbf{Return Value}
    \vspace{-1ex}

      \begin{quote}
      pPopupEntry

      \end{quote}

\textbf{Status:} Tested + NoDoc + Demo = OK



    \end{boxedminipage}

    \label{xformslib:library:fl_get_nmenu_item_by_label}
    \index{xformslib \textit{(package)}!xformslib.library \textit{(module)}!xformslib.library.fl\_get\_nmenu\_item\_by\_label \textit{(function)}}

    \vspace{0.5ex}

\hspace{.8\funcindent}\begin{boxedminipage}{\funcwidth}

    \raggedright \textbf{fl\_get\_nmenu\_item\_by\_label}(\textit{pObject}, \textit{label})

    \vspace{-1.5ex}

    \rule{\textwidth}{0.5\fboxrule}
\setlength{\parskip}{2ex}
\setlength{\parskip}{1ex}
      \textbf{Parameters}
      \vspace{-1ex}

      \begin{quote}
        \begin{Ventry}{xxxxxxx}

          \item[pObject]

          nmenu object ({\textless}pointer to 
          xfdata.FL\_OBJECT{\textgreater})

        \end{Ventry}

      \end{quote}

      \textbf{Return Value}
    \vspace{-1ex}

      \begin{quote}
      pPopupEntry

      \end{quote}

\textbf{Status:} Untested + NoDoc + NoDemo = NOT OK



    \end{boxedminipage}

    \label{xformslib:library:fl_get_nmenu_item_by_text}
    \index{xformslib \textit{(package)}!xformslib.library \textit{(module)}!xformslib.library.fl\_get\_nmenu\_item\_by\_text \textit{(function)}}

    \vspace{0.5ex}

\hspace{.8\funcindent}\begin{boxedminipage}{\funcwidth}

    \raggedright \textbf{fl\_get\_nmenu\_item\_by\_text}(\textit{pObject}, \textit{text})

    \vspace{-1.5ex}

    \rule{\textwidth}{0.5\fboxrule}
\setlength{\parskip}{2ex}
\setlength{\parskip}{1ex}
      \textbf{Parameters}
      \vspace{-1ex}

      \begin{quote}
        \begin{Ventry}{xxxxxxx}

          \item[pObject]

          nmenu object ({\textless}pointer to 
          xfdata.FL\_OBJECT{\textgreater})

        \end{Ventry}

      \end{quote}

      \textbf{Return Value}
    \vspace{-1ex}

      \begin{quote}
      pPopupEntry

      \end{quote}

\textbf{Status:} Untested + NoDoc + NoDemo = NOT OK



    \end{boxedminipage}

    \label{xformslib:library:fl_set_nmenu_policy}
    \index{xformslib \textit{(package)}!xformslib.library \textit{(module)}!xformslib.library.fl\_set\_nmenu\_policy \textit{(function)}}

    \vspace{0.5ex}

\hspace{.8\funcindent}\begin{boxedminipage}{\funcwidth}

    \raggedright \textbf{fl\_set\_nmenu\_policy}(\textit{pObject}, \textit{num})

    \vspace{-1.5ex}

    \rule{\textwidth}{0.5\fboxrule}
\setlength{\parskip}{2ex}
\setlength{\parskip}{1ex}
      \textbf{Parameters}
      \vspace{-1ex}

      \begin{quote}
        \begin{Ventry}{xxxxxxx}

          \item[pObject]

          nmenu object ({\textless}pointer to 
          xfdata.FL\_OBJECT{\textgreater})

        \end{Ventry}

      \end{quote}

      \textbf{Return Value}
    \vspace{-1ex}

      \begin{quote}
      num

      \end{quote}

\textbf{Status:} Untested + NoDoc + NoDemo = NOT OK



    \end{boxedminipage}

    \label{xformslib:library:fl_set_nmenu_hl_text_color}
    \index{xformslib \textit{(package)}!xformslib.library \textit{(module)}!xformslib.library.fl\_set\_nmenu\_hl\_text\_color \textit{(function)}}

    \vspace{0.5ex}

\hspace{.8\funcindent}\begin{boxedminipage}{\funcwidth}

    \raggedright \textbf{fl\_set\_nmenu\_hl\_text\_color}(\textit{pObject}, \textit{colr})

    \vspace{-1.5ex}

    \rule{\textwidth}{0.5\fboxrule}
\setlength{\parskip}{2ex}
\setlength{\parskip}{1ex}
      \textbf{Parameters}
      \vspace{-1ex}

      \begin{quote}
        \begin{Ventry}{xxxxxxx}

          \item[pObject]

          nmenu object ({\textless}pointer to 
          xfdata.FL\_OBJECT{\textgreater})

        \end{Ventry}

      \end{quote}

      \textbf{Return Value}
    \vspace{-1ex}

      \begin{quote}
      color

      \end{quote}

\textbf{Status:} Untested + NoDoc + NoDemo = NOT OK



    \end{boxedminipage}

    \label{xformslib:library:fl_add_positioner}
    \index{xformslib \textit{(package)}!xformslib.library \textit{(module)}!xformslib.library.fl\_add\_positioner \textit{(function)}}

    \vspace{0.5ex}

\hspace{.8\funcindent}\begin{boxedminipage}{\funcwidth}

    \raggedright \textbf{fl\_add\_positioner}(\textit{postype}, \textit{x}, \textit{y}, \textit{w}, \textit{h}, \textit{label})

    \vspace{-1.5ex}

    \rule{\textwidth}{0.5\fboxrule}
\setlength{\parskip}{2ex}
    Adds a positioner object.

\setlength{\parskip}{1ex}
      \textbf{Parameters}
      \vspace{-1ex}

      \begin{quote}
        \begin{Ventry}{xxxxxxx}

          \item[postype]

          type of positioner to be added

          \item[x]

          horizontal position (upper-left corner)

          \item[y]

          vertical position (upper-left corner)

          \item[w]

          width in coord units

          \item[h]

          height in coord units

          \item[label]

          text label of positioner

        \end{Ventry}

      \end{quote}

      \textbf{Return Value}
    \vspace{-1ex}

      \begin{quote}
      pObject

      \end{quote}

\textbf{Status:} Tested + NoDoc + Demo = OK



    \end{boxedminipage}

    \label{xformslib:library:fl_set_positioner_xvalue}
    \index{xformslib \textit{(package)}!xformslib.library \textit{(module)}!xformslib.library.fl\_set\_positioner\_xvalue \textit{(function)}}

    \vspace{0.5ex}

\hspace{.8\funcindent}\begin{boxedminipage}{\funcwidth}

    \raggedright \textbf{fl\_set\_positioner\_xvalue}(\textit{pObject}, \textit{val})

    \vspace{-1.5ex}

    \rule{\textwidth}{0.5\fboxrule}
\setlength{\parskip}{2ex}
\setlength{\parskip}{1ex}
      \textbf{Parameters}
      \vspace{-1ex}

      \begin{quote}
        \begin{Ventry}{xxxxxxx}

          \item[pObject]

          positioner object ({\textless}pointer to 
          xfdata.FL\_OBJECT{\textgreater})

        \end{Ventry}

      \end{quote}

\textbf{Status:} Tested + NoDoc + Demo = OK



    \end{boxedminipage}

    \label{xformslib:library:fl_get_positioner_xvalue}
    \index{xformslib \textit{(package)}!xformslib.library \textit{(module)}!xformslib.library.fl\_get\_positioner\_xvalue \textit{(function)}}

    \vspace{0.5ex}

\hspace{.8\funcindent}\begin{boxedminipage}{\funcwidth}

    \raggedright \textbf{fl\_get\_positioner\_xvalue}(\textit{pObject})

    \vspace{-1.5ex}

    \rule{\textwidth}{0.5\fboxrule}
\setlength{\parskip}{2ex}
\setlength{\parskip}{1ex}
      \textbf{Parameters}
      \vspace{-1ex}

      \begin{quote}
        \begin{Ventry}{xxxxxxx}

          \item[pObject]

          positioner object ({\textless}pointer to 
          xfdata.FL\_OBJECT{\textgreater})

        \end{Ventry}

      \end{quote}

      \textbf{Return Value}
    \vspace{-1ex}

      \begin{quote}
      floatnum

      \end{quote}

\textbf{Status:} Tested + NoDoc + Demo = OK



    \end{boxedminipage}

    \label{xformslib:library:fl_set_positioner_xbounds}
    \index{xformslib \textit{(package)}!xformslib.library \textit{(module)}!xformslib.library.fl\_set\_positioner\_xbounds \textit{(function)}}

    \vspace{0.5ex}

\hspace{.8\funcindent}\begin{boxedminipage}{\funcwidth}

    \raggedright \textbf{fl\_set\_positioner\_xbounds}(\textit{pObject}, \textit{minbound}, \textit{maxbound})

    \vspace{-1.5ex}

    \rule{\textwidth}{0.5\fboxrule}
\setlength{\parskip}{2ex}
\setlength{\parskip}{1ex}
      \textbf{Parameters}
      \vspace{-1ex}

      \begin{quote}
        \begin{Ventry}{xxxxxxx}

          \item[pObject]

          positioner object ({\textless}pointer to 
          xfdata.FL\_OBJECT{\textgreater})

        \end{Ventry}

      \end{quote}

\textbf{Status:} Tested + NoDoc + Demo = OK



    \end{boxedminipage}

    \label{xformslib:library:fl_get_positioner_xbounds}
    \index{xformslib \textit{(package)}!xformslib.library \textit{(module)}!xformslib.library.fl\_get\_positioner\_xbounds \textit{(function)}}

    \vspace{0.5ex}

\hspace{.8\funcindent}\begin{boxedminipage}{\funcwidth}

    \raggedright \textbf{fl\_get\_positioner\_xbounds}(\textit{pObject})

    \vspace{-1.5ex}

    \rule{\textwidth}{0.5\fboxrule}
\setlength{\parskip}{2ex}
\setlength{\parskip}{1ex}
      \textbf{Parameters}
      \vspace{-1ex}

      \begin{quote}
        \begin{Ventry}{xxxxxxx}

          \item[pObject]

          positioner object ({\textless}pointer to 
          xfdata.FL\_OBJECT{\textgreater})

        \end{Ventry}

      \end{quote}

      \textbf{Return Value}
    \vspace{-1ex}

      \begin{quote}
      minbound, maxbound

      \end{quote}

\textbf{Attention:} API change from XForms - upstream was fl\_get\_positioner\_xbounds(pObject,
minbound, maxbound)



\textbf{Status:} Untested + NoDoc + NoDemo = NOT OK



    \end{boxedminipage}

    \label{xformslib:library:fl_set_positioner_yvalue}
    \index{xformslib \textit{(package)}!xformslib.library \textit{(module)}!xformslib.library.fl\_set\_positioner\_yvalue \textit{(function)}}

    \vspace{0.5ex}

\hspace{.8\funcindent}\begin{boxedminipage}{\funcwidth}

    \raggedright \textbf{fl\_set\_positioner\_yvalue}(\textit{pObject}, \textit{val})

    \vspace{-1.5ex}

    \rule{\textwidth}{0.5\fboxrule}
\setlength{\parskip}{2ex}
\setlength{\parskip}{1ex}
      \textbf{Parameters}
      \vspace{-1ex}

      \begin{quote}
        \begin{Ventry}{xxxxxxx}

          \item[pObject]

          positioner object ({\textless}pointer to 
          xfdata.FL\_OBJECT{\textgreater})

        \end{Ventry}

      \end{quote}

\textbf{Status:} Tested + NoDoc + Demo = OK



    \end{boxedminipage}

    \label{xformslib:library:fl_get_positioner_yvalue}
    \index{xformslib \textit{(package)}!xformslib.library \textit{(module)}!xformslib.library.fl\_get\_positioner\_yvalue \textit{(function)}}

    \vspace{0.5ex}

\hspace{.8\funcindent}\begin{boxedminipage}{\funcwidth}

    \raggedright \textbf{fl\_get\_positioner\_yvalue}(\textit{pObject})

    \vspace{-1.5ex}

    \rule{\textwidth}{0.5\fboxrule}
\setlength{\parskip}{2ex}
\setlength{\parskip}{1ex}
      \textbf{Parameters}
      \vspace{-1ex}

      \begin{quote}
        \begin{Ventry}{xxxxxxx}

          \item[pObject]

          positioner object ({\textless}pointer to 
          xfdata.FL\_OBJECT{\textgreater})

        \end{Ventry}

      \end{quote}

      \textbf{Return Value}
    \vspace{-1ex}

      \begin{quote}
      floatnum

      \end{quote}

\textbf{Status:} Untested + NoDoc + NoDemo = NOT OK



    \end{boxedminipage}

    \label{xformslib:library:fl_set_positioner_ybounds}
    \index{xformslib \textit{(package)}!xformslib.library \textit{(module)}!xformslib.library.fl\_set\_positioner\_ybounds \textit{(function)}}

    \vspace{0.5ex}

\hspace{.8\funcindent}\begin{boxedminipage}{\funcwidth}

    \raggedright \textbf{fl\_set\_positioner\_ybounds}(\textit{pObject}, \textit{minbound}, \textit{maxbound})

    \vspace{-1.5ex}

    \rule{\textwidth}{0.5\fboxrule}
\setlength{\parskip}{2ex}
\setlength{\parskip}{1ex}
      \textbf{Parameters}
      \vspace{-1ex}

      \begin{quote}
        \begin{Ventry}{xxxxxxx}

          \item[pObject]

          positioner object ({\textless}pointer to 
          xfdata.FL\_OBJECT{\textgreater})

        \end{Ventry}

      \end{quote}

\textbf{Status:} Tested + NoDoc + Demo = OK



    \end{boxedminipage}

    \label{xformslib:library:fl_get_positioner_ybounds}
    \index{xformslib \textit{(package)}!xformslib.library \textit{(module)}!xformslib.library.fl\_get\_positioner\_ybounds \textit{(function)}}

    \vspace{0.5ex}

\hspace{.8\funcindent}\begin{boxedminipage}{\funcwidth}

    \raggedright \textbf{fl\_get\_positioner\_ybounds}(\textit{pObject})

    \vspace{-1.5ex}

    \rule{\textwidth}{0.5\fboxrule}
\setlength{\parskip}{2ex}
\setlength{\parskip}{1ex}
      \textbf{Parameters}
      \vspace{-1ex}

      \begin{quote}
        \begin{Ventry}{xxxxxxx}

          \item[pObject]

          positioner object ({\textless}pointer to 
          xfdata.FL\_OBJECT{\textgreater})

        \end{Ventry}

      \end{quote}

      \textbf{Return Value}
    \vspace{-1ex}

      \begin{quote}
      minbound, maxbound

      \end{quote}

\textbf{Attention:} API change from XForms - upstream was fl\_get\_positioner\_ybounds(pObject,
minbound, maxbound)



\textbf{Status:} Untested + NoDoc + NoDemo = NOT OK



    \end{boxedminipage}

    \label{xformslib:library:fl_set_positioner_xstep}
    \index{xformslib \textit{(package)}!xformslib.library \textit{(module)}!xformslib.library.fl\_set\_positioner\_xstep \textit{(function)}}

    \vspace{0.5ex}

\hspace{.8\funcindent}\begin{boxedminipage}{\funcwidth}

    \raggedright \textbf{fl\_set\_positioner\_xstep}(\textit{pObject}, \textit{value})

    \vspace{-1.5ex}

    \rule{\textwidth}{0.5\fboxrule}
\setlength{\parskip}{2ex}
\setlength{\parskip}{1ex}
      \textbf{Parameters}
      \vspace{-1ex}

      \begin{quote}
        \begin{Ventry}{xxxxxxx}

          \item[pObject]

          positioner object ({\textless}pointer to 
          xfdata.FL\_OBJECT{\textgreater})

        \end{Ventry}

      \end{quote}

\textbf{Status:} Untested + NoDoc + NoDemo = NOT OK



    \end{boxedminipage}

    \label{xformslib:library:fl_set_positioner_ystep}
    \index{xformslib \textit{(package)}!xformslib.library \textit{(module)}!xformslib.library.fl\_set\_positioner\_ystep \textit{(function)}}

    \vspace{0.5ex}

\hspace{.8\funcindent}\begin{boxedminipage}{\funcwidth}

    \raggedright \textbf{fl\_set\_positioner\_ystep}(\textit{pObject}, \textit{value})

    \vspace{-1.5ex}

    \rule{\textwidth}{0.5\fboxrule}
\setlength{\parskip}{2ex}
\setlength{\parskip}{1ex}
      \textbf{Parameters}
      \vspace{-1ex}

      \begin{quote}
        \begin{Ventry}{xxxxxxx}

          \item[pObject]

          positioner object ({\textless}pointer to 
          xfdata.FL\_OBJECT{\textgreater})

        \end{Ventry}

      \end{quote}

\textbf{Status:} Untested + NoDoc + NoDemo = NOT OK



    \end{boxedminipage}

    \label{xformslib:library:fl_set_positioner_return}
    \index{xformslib \textit{(package)}!xformslib.library \textit{(module)}!xformslib.library.fl\_set\_positioner\_return \textit{(function)}}

    \vspace{0.5ex}

\hspace{.8\funcindent}\begin{boxedminipage}{\funcwidth}

    \raggedright \textbf{fl\_set\_positioner\_return}(\textit{pObject}, \textit{value})

    \vspace{-1.5ex}

    \rule{\textwidth}{0.5\fboxrule}
\setlength{\parskip}{2ex}
\setlength{\parskip}{1ex}
      \textbf{Parameters}
      \vspace{-1ex}

      \begin{quote}
        \begin{Ventry}{xxxxxxx}

          \item[pObject]

          positioner object ({\textless}pointer to 
          xfdata.FL\_OBJECT{\textgreater})

          \item[value]

          return type

        \end{Ventry}

      \end{quote}

\textbf{Status:} Untested + NoDoc + NoDemo = NOT OK



    \end{boxedminipage}

    \label{xformslib:library:fl_add_scrollbar}
    \index{xformslib \textit{(package)}!xformslib.library \textit{(module)}!xformslib.library.fl\_add\_scrollbar \textit{(function)}}

    \vspace{0.5ex}

\hspace{.8\funcindent}\begin{boxedminipage}{\funcwidth}

    \raggedright \textbf{fl\_add\_scrollbar}(\textit{scrolltype}, \textit{x}, \textit{y}, \textit{w}, \textit{h}, \textit{label})

    \vspace{-1.5ex}

    \rule{\textwidth}{0.5\fboxrule}
\setlength{\parskip}{2ex}
    Adds a scrollbar object to a form.

\setlength{\parskip}{1ex}
      \textbf{Parameters}
      \vspace{-1ex}

      \begin{quote}
        \begin{Ventry}{xxxxxxxxxx}

          \item[scrolltype]

          type of scrollbar to be added

          \item[x]

          horizontal position (upper-left corner)

          \item[y]

          vertical position (upper-left corner)

          \item[w]

          width in coord units

          \item[h]

          height in coord units

          \item[label]

          label text of scrollbar

        \end{Ventry}

      \end{quote}

      \textbf{Return Value}
    \vspace{-1ex}

      \begin{quote}
      pObject

      \end{quote}

\textbf{Status:} Tested + NoDoc + Demo = OK



    \end{boxedminipage}

    \label{xformslib:library:fl_get_scrollbar_value}
    \index{xformslib \textit{(package)}!xformslib.library \textit{(module)}!xformslib.library.fl\_get\_scrollbar\_value \textit{(function)}}

    \vspace{0.5ex}

\hspace{.8\funcindent}\begin{boxedminipage}{\funcwidth}

    \raggedright \textbf{fl\_get\_scrollbar\_value}(\textit{pObject})

    \vspace{-1.5ex}

    \rule{\textwidth}{0.5\fboxrule}
\setlength{\parskip}{2ex}
    Returns the value of a scrollbar.

\setlength{\parskip}{1ex}
      \textbf{Parameters}
      \vspace{-1ex}

      \begin{quote}
        \begin{Ventry}{xxxxxxx}

          \item[pObject]

          pointer to object

        \end{Ventry}

      \end{quote}

      \textbf{Return Value}
    \vspace{-1ex}

      \begin{quote}
      value[float]

      \end{quote}

\textbf{Status:} Untested + NoDoc + NoDemo = NOT OK



    \end{boxedminipage}

    \label{xformslib:library:fl_set_scrollbar_value}
    \index{xformslib \textit{(package)}!xformslib.library \textit{(module)}!xformslib.library.fl\_set\_scrollbar\_value \textit{(function)}}

    \vspace{0.5ex}

\hspace{.8\funcindent}\begin{boxedminipage}{\funcwidth}

    \raggedright \textbf{fl\_set\_scrollbar\_value}(\textit{pObject}, \textit{val})

    \vspace{-1.5ex}

    \rule{\textwidth}{0.5\fboxrule}
\setlength{\parskip}{2ex}
    Sets the value of a scrollbar.

\setlength{\parskip}{1ex}
      \textbf{Parameters}
      \vspace{-1ex}

      \begin{quote}
        \begin{Ventry}{xxxxxxx}

          \item[pObject]

          pointer to object

          \item[val]

          value of the scrollbar to be set

        \end{Ventry}

      \end{quote}

\textbf{Status:} Tested + NoDoc + Demo = OK



    \end{boxedminipage}

    \label{xformslib:library:fl_set_scrollbar_size}
    \index{xformslib \textit{(package)}!xformslib.library \textit{(module)}!xformslib.library.fl\_set\_scrollbar\_size \textit{(function)}}

    \vspace{0.5ex}

\hspace{.8\funcindent}\begin{boxedminipage}{\funcwidth}

    \raggedright \textbf{fl\_set\_scrollbar\_size}(\textit{pObject}, \textit{val})

    \vspace{-1.5ex}

    \rule{\textwidth}{0.5\fboxrule}
\setlength{\parskip}{2ex}
\setlength{\parskip}{1ex}
\textbf{Status:} Tested + NoDoc + Demo = OK



    \end{boxedminipage}

    \label{xformslib:library:fl_set_scrollbar_increment}
    \index{xformslib \textit{(package)}!xformslib.library \textit{(module)}!xformslib.library.fl\_set\_scrollbar\_increment \textit{(function)}}

    \vspace{0.5ex}

\hspace{.8\funcindent}\begin{boxedminipage}{\funcwidth}

    \raggedright \textbf{fl\_set\_scrollbar\_increment}(\textit{pObject}, \textit{leftbtnval}, \textit{midlbtnval})

    \vspace{-1.5ex}

    \rule{\textwidth}{0.5\fboxrule}
\setlength{\parskip}{2ex}
    Sets the size of the steps of a scrollbar jump.

\setlength{\parskip}{1ex}
      \textbf{Parameters}
      \vspace{-1ex}

      \begin{quote}
        \begin{Ventry}{xxxxxxxxxx}

          \item[pObject]

          pointer to object

          \item[leftbtnval]

          value to increment if the left mouse button is pressed

          \item[midlbtnval]

          value to increment if the middle mouse button is pressed

        \end{Ventry}

      \end{quote}

\textbf{Status:} Untested + NoDoc + NoDemo = NOT OK



    \end{boxedminipage}

    \label{xformslib:library:fl_get_scrollbar_increment}
    \index{xformslib \textit{(package)}!xformslib.library \textit{(module)}!xformslib.library.fl\_get\_scrollbar\_increment \textit{(function)}}

    \vspace{0.5ex}

\hspace{.8\funcindent}\begin{boxedminipage}{\funcwidth}

    \raggedright \textbf{fl\_get\_scrollbar\_increment}(\textit{pObject})

    \vspace{-1.5ex}

    \rule{\textwidth}{0.5\fboxrule}
\setlength{\parskip}{2ex}
    Returns the increment of size of a scrollbar for left and middle mouse 
    buttons.

\setlength{\parskip}{1ex}
      \textbf{Parameters}
      \vspace{-1ex}

      \begin{quote}
        \begin{Ventry}{xxxxxxx}

          \item[pObject]

          pointer to object

        \end{Ventry}

      \end{quote}

      \textbf{Return Value}
    \vspace{-1ex}

      \begin{quote}
      leftbtnval[double], midlbtnval[double]

      \end{quote}

\textbf{Attention:} API change from XForms - upstream was 
fl\_get\_scrollbar\_increment(pObject, leftbtnval, valmidlbtnval)



\textbf{Status:} Untested + NoDoc + NoDemo = NOT OK



    \end{boxedminipage}

    \label{xformslib:library:fl_set_scrollbar_bounds}
    \index{xformslib \textit{(package)}!xformslib.library \textit{(module)}!xformslib.library.fl\_set\_scrollbar\_bounds \textit{(function)}}

    \vspace{0.5ex}

\hspace{.8\funcindent}\begin{boxedminipage}{\funcwidth}

    \raggedright \textbf{fl\_set\_scrollbar\_bounds}(\textit{pObject}, \textit{minbound}, \textit{maxbound})

    \vspace{-1.5ex}

    \rule{\textwidth}{0.5\fboxrule}
\setlength{\parskip}{2ex}
    Sets the bounds/limits of a scrollbar.

\setlength{\parskip}{1ex}
      \textbf{Parameters}
      \vspace{-1ex}

      \begin{quote}
        \begin{Ventry}{xxxxxxxx}

          \item[pObject]

          pointer to object

          \item[minbound]

          minimum bound of scrollbar

          \item[maxbound]

          maximum bound of scrollbar

        \end{Ventry}

      \end{quote}

\textbf{Status:} Untested + NoDoc + NoDemo = NOT OK



    \end{boxedminipage}

    \label{xformslib:library:fl_get_scrollbar_bounds}
    \index{xformslib \textit{(package)}!xformslib.library \textit{(module)}!xformslib.library.fl\_get\_scrollbar\_bounds \textit{(function)}}

    \vspace{0.5ex}

\hspace{.8\funcindent}\begin{boxedminipage}{\funcwidth}

    \raggedright \textbf{fl\_get\_scrollbar\_bounds}(\textit{pObject})

    \vspace{-1.5ex}

    \rule{\textwidth}{0.5\fboxrule}
\setlength{\parskip}{2ex}
    Returns the bounds/limits of a scrollbar.

\setlength{\parskip}{1ex}
      \textbf{Parameters}
      \vspace{-1ex}

      \begin{quote}
        \begin{Ventry}{xxxxxxx}

          \item[pObject]

          pointer to scrollbar object

        \end{Ventry}

      \end{quote}

      \textbf{Return Value}
    \vspace{-1ex}

      \begin{quote}
      minbound, maxbound

      \end{quote}

\textbf{Attention:} API change from XForms - upstream was fl\_get\_scrollbar\_bounds(pObject, 
b1, b2)



    \end{boxedminipage}

    \label{xformslib:library:fl_set_scrollbar_step}
    \index{xformslib \textit{(package)}!xformslib.library \textit{(module)}!xformslib.library.fl\_set\_scrollbar\_step \textit{(function)}}

    \vspace{0.5ex}

\hspace{.8\funcindent}\begin{boxedminipage}{\funcwidth}

    \raggedright \textbf{fl\_set\_scrollbar\_step}(\textit{pObject}, \textit{step})

    \vspace{-1.5ex}

    \rule{\textwidth}{0.5\fboxrule}
\setlength{\parskip}{2ex}
\setlength{\parskip}{1ex}
\textbf{Status:} Untested + NoDoc + NoDemo = NOT OK



    \end{boxedminipage}

    \label{xformslib:library:fl_add_select}
    \index{xformslib \textit{(package)}!xformslib.library \textit{(module)}!xformslib.library.fl\_add\_select \textit{(function)}}

    \vspace{0.5ex}

\hspace{.8\funcindent}\begin{boxedminipage}{\funcwidth}

    \raggedright \textbf{fl\_add\_select}(\textit{selecttype}, \textit{x}, \textit{y}, \textit{w}, \textit{h}, \textit{label})

    \vspace{-1.5ex}

    \rule{\textwidth}{0.5\fboxrule}
\setlength{\parskip}{2ex}
    Adds a select object.

\setlength{\parskip}{1ex}
      \textbf{Parameters}
      \vspace{-1ex}

      \begin{quote}
        \begin{Ventry}{xxxxxxxxxx}

          \item[selecttype]

          type of select to be added

          \item[x]

          horizontal position (upper-left corner)

          \item[x]

          vertical position (upper-left corner)

          \item[w]

          width in coord units

          \item[h]

          height in coord units

          \item[label]

          text label of select

        \end{Ventry}

      \end{quote}

      \textbf{Return Value}
    \vspace{-1ex}

      \begin{quote}
      pObject

      \end{quote}

\textbf{Status:} Tested + NoDoc + Demo = OK



    \end{boxedminipage}

    \label{xformslib:library:fl_clear_select}
    \index{xformslib \textit{(package)}!xformslib.library \textit{(module)}!xformslib.library.fl\_clear\_select \textit{(function)}}

    \vspace{0.5ex}

\hspace{.8\funcindent}\begin{boxedminipage}{\funcwidth}

    \raggedright \textbf{fl\_clear\_select}(\textit{pObject})

    \vspace{-1.5ex}

    \rule{\textwidth}{0.5\fboxrule}
\setlength{\parskip}{2ex}
\setlength{\parskip}{1ex}
      \textbf{Parameters}
      \vspace{-1ex}

      \begin{quote}
        \begin{Ventry}{xxxxxxx}

          \item[pObject]

          pointer to select object

        \end{Ventry}

      \end{quote}

\textbf{Status:} Untested + NoDoc + NoDemo = NOT OK



    \end{boxedminipage}

    \label{xformslib:library:fl_add_select_items}
    \index{xformslib \textit{(package)}!xformslib.library \textit{(module)}!xformslib.library.fl\_add\_select\_items \textit{(function)}}

    \vspace{0.5ex}

\hspace{.8\funcindent}\begin{boxedminipage}{\funcwidth}

    \raggedright \textbf{fl\_add\_select\_items}(\textit{pObject}, \textit{itemstr})

    \vspace{-1.5ex}

    \rule{\textwidth}{0.5\fboxrule}
\setlength{\parskip}{2ex}
\setlength{\parskip}{1ex}
      \textbf{Return Value}
    \vspace{-1ex}

      \begin{quote}
      pPopupEntry

      \end{quote}

\textbf{Status:} HalfTested + NoDoc + Demo = NOT OK (sequence param.)



    \end{boxedminipage}

    \label{xformslib:library:fl_insert_select_items}
    \index{xformslib \textit{(package)}!xformslib.library \textit{(module)}!xformslib.library.fl\_insert\_select\_items \textit{(function)}}

    \vspace{0.5ex}

\hspace{.8\funcindent}\begin{boxedminipage}{\funcwidth}

    \raggedright \textbf{fl\_insert\_select\_items}(\textit{pObject}, \textit{pPopupEntry}, \textit{itemstr})

    \vspace{-1.5ex}

    \rule{\textwidth}{0.5\fboxrule}
\setlength{\parskip}{2ex}
\setlength{\parskip}{1ex}
      \textbf{Parameters}
      \vspace{-1ex}

      \begin{quote}
        \begin{Ventry}{xxxxxxx}

          \item[itemstr]

          text of the item (among special sequences only \%S is supported

        \end{Ventry}

      \end{quote}

      \textbf{Return Value}
    \vspace{-1ex}

      \begin{quote}
      pPopupEntry

      \end{quote}

\textbf{Status:} HalfTested + NoDoc + Demo = NOT OK (special sequence)



    \end{boxedminipage}

    \label{xformslib:library:fl_replace_select_item}
    \index{xformslib \textit{(package)}!xformslib.library \textit{(module)}!xformslib.library.fl\_replace\_select\_item \textit{(function)}}

    \vspace{0.5ex}

\hspace{.8\funcindent}\begin{boxedminipage}{\funcwidth}

    \raggedright \textbf{fl\_replace\_select\_item}(\textit{pObject}, \textit{pPopupEntry}, \textit{itemstr})

    \vspace{-1.5ex}

    \rule{\textwidth}{0.5\fboxrule}
\setlength{\parskip}{2ex}
\setlength{\parskip}{1ex}
      \textbf{Return Value}
    \vspace{-1ex}

      \begin{quote}
      pPopupEntry

      \end{quote}

\textbf{Status:} Untested + NoDoc + NoDemo = NOT OK



    \end{boxedminipage}

    \label{xformslib:library:fl_delete_select_item}
    \index{xformslib \textit{(package)}!xformslib.library \textit{(module)}!xformslib.library.fl\_delete\_select\_item \textit{(function)}}

    \vspace{0.5ex}

\hspace{.8\funcindent}\begin{boxedminipage}{\funcwidth}

    \raggedright \textbf{fl\_delete\_select\_item}(\textit{pObject}, \textit{pPopupEntry})

    \vspace{-1.5ex}

    \rule{\textwidth}{0.5\fboxrule}
\setlength{\parskip}{2ex}
\setlength{\parskip}{1ex}
      \textbf{Return Value}
    \vspace{-1ex}

      \begin{quote}
      num

      \end{quote}

\textbf{Status:} Untested + NoDoc + NoDemo = NOT OK



    \end{boxedminipage}

    \label{xformslib:library:fl_set_select_items}
    \index{xformslib \textit{(package)}!xformslib.library \textit{(module)}!xformslib.library.fl\_set\_select\_items \textit{(function)}}

    \vspace{0.5ex}

\hspace{.8\funcindent}\begin{boxedminipage}{\funcwidth}

    \raggedright \textbf{fl\_set\_select\_items}(\textit{pObject}, \textit{pPopupItem})

    \vspace{-1.5ex}

    \rule{\textwidth}{0.5\fboxrule}
\setlength{\parskip}{2ex}
    (Re)populates a select object popup.

\setlength{\parskip}{1ex}
      \textbf{Parameters}
      \vspace{-1ex}

      \begin{quote}
        \begin{Ventry}{xxxxxxxxxx}

          \item[pObject]

          pointer to select object

          \item[pPopupItem]

          pointer to FL\_POPUP\_ITEM class instance (array of it)

        \end{Ventry}

      \end{quote}

      \textbf{Return Value}
    \vspace{-1ex}

      \begin{quote}
      num

      \end{quote}

\textbf{Status:} Untested + NoDoc + NoDemo = NOT OK



    \end{boxedminipage}

    \label{xformslib:library:fl_get_select_popup}
    \index{xformslib \textit{(package)}!xformslib.library \textit{(module)}!xformslib.library.fl\_get\_select\_popup \textit{(function)}}

    \vspace{0.5ex}

\hspace{.8\funcindent}\begin{boxedminipage}{\funcwidth}

    \raggedright \textbf{fl\_get\_select\_popup}(\textit{pObject})

    \vspace{-1.5ex}

    \rule{\textwidth}{0.5\fboxrule}
\setlength{\parskip}{2ex}
\setlength{\parskip}{1ex}
      \textbf{Return Value}
    \vspace{-1ex}

      \begin{quote}
      pPopup

      \end{quote}

\textbf{Status:} Tested + NoDoc + Demo = OK



    \end{boxedminipage}

    \label{xformslib:library:fl_set_select_popup}
    \index{xformslib \textit{(package)}!xformslib.library \textit{(module)}!xformslib.library.fl\_set\_select\_popup \textit{(function)}}

    \vspace{0.5ex}

\hspace{.8\funcindent}\begin{boxedminipage}{\funcwidth}

    \raggedright \textbf{fl\_set\_select\_popup}(\textit{pObject}, \textit{pPopup})

    \vspace{-1.5ex}

    \rule{\textwidth}{0.5\fboxrule}
\setlength{\parskip}{2ex}
\setlength{\parskip}{1ex}
      \textbf{Return Value}
    \vspace{-1ex}

      \begin{quote}
      num

      \end{quote}

\textbf{Status:} Untested + NoDoc + NoDemo = NOT OK



    \end{boxedminipage}

    \label{xformslib:library:fl_get_select_item}
    \index{xformslib \textit{(package)}!xformslib.library \textit{(module)}!xformslib.library.fl\_get\_select\_item \textit{(function)}}

    \vspace{0.5ex}

\hspace{.8\funcindent}\begin{boxedminipage}{\funcwidth}

    \raggedright \textbf{fl\_get\_select\_item}(\textit{pObject})

    \vspace{-1.5ex}

    \rule{\textwidth}{0.5\fboxrule}
\setlength{\parskip}{2ex}
\setlength{\parskip}{1ex}
      \textbf{Return Value}
    \vspace{-1ex}

      \begin{quote}
      pPopupReturn

      \end{quote}

\textbf{Status:} Tested + NoDoc + Demo = OK



    \end{boxedminipage}

    \label{xformslib:library:fl_set_select_item}
    \index{xformslib \textit{(package)}!xformslib.library \textit{(module)}!xformslib.library.fl\_set\_select\_item \textit{(function)}}

    \vspace{0.5ex}

\hspace{.8\funcindent}\begin{boxedminipage}{\funcwidth}

    \raggedright \textbf{fl\_set\_select\_item}(\textit{pObject}, \textit{pPopupEntry})

    \vspace{-1.5ex}

    \rule{\textwidth}{0.5\fboxrule}
\setlength{\parskip}{2ex}
    Set a new item as currently selected.

\setlength{\parskip}{1ex}
      \textbf{Parameters}
      \vspace{-1ex}

      \begin{quote}
        \begin{Ventry}{xxxxxxxxxxx}

          \item[pObject]

          pointer to select object

          \item[pPopupEntry]

          pointer to FL\_POPUP\_ENTRY class instance

        \end{Ventry}

      \end{quote}

      \textbf{Return Value}
    \vspace{-1ex}

      \begin{quote}
      pPopupReturn

      \end{quote}

\textbf{Status:} HalfTested + NoDoc + Demo = NOT OK (FL\_POPUP\_ENTRY not prepared)



    \end{boxedminipage}

    \label{xformslib:library:fl_get_select_item_by_value}
    \index{xformslib \textit{(package)}!xformslib.library \textit{(module)}!xformslib.library.fl\_get\_select\_item\_by\_value \textit{(function)}}

    \vspace{0.5ex}

\hspace{.8\funcindent}\begin{boxedminipage}{\funcwidth}

    \raggedright \textbf{fl\_get\_select\_item\_by\_value}(\textit{pObject}, \textit{value})

    \vspace{-1.5ex}

    \rule{\textwidth}{0.5\fboxrule}
\setlength{\parskip}{2ex}
\setlength{\parskip}{1ex}
      \textbf{Return Value}
    \vspace{-1ex}

      \begin{quote}
      pPopupEntry

      \end{quote}

\textbf{Status:} Untested + NoDoc + NoDemo = NOT OK



    \end{boxedminipage}

    \label{xformslib:library:fl_get_select_item_by_label}
    \index{xformslib \textit{(package)}!xformslib.library \textit{(module)}!xformslib.library.fl\_get\_select\_item\_by\_label \textit{(function)}}

    \vspace{0.5ex}

\hspace{.8\funcindent}\begin{boxedminipage}{\funcwidth}

    \raggedright \textbf{fl\_get\_select\_item\_by\_label}(\textit{pObject}, \textit{label})

    \vspace{-1.5ex}

    \rule{\textwidth}{0.5\fboxrule}
\setlength{\parskip}{2ex}
\setlength{\parskip}{1ex}
      \textbf{Return Value}
    \vspace{-1ex}

      \begin{quote}
      pPopupEntry

      \end{quote}

\textbf{Status:} Tested + NoDoc + Demo = OK



    \end{boxedminipage}

    \label{xformslib:library:fl_get_select_item_by_text}
    \index{xformslib \textit{(package)}!xformslib.library \textit{(module)}!xformslib.library.fl\_get\_select\_item\_by\_text \textit{(function)}}

    \vspace{0.5ex}

\hspace{.8\funcindent}\begin{boxedminipage}{\funcwidth}

    \raggedright \textbf{fl\_get\_select\_item\_by\_text}(\textit{pObject}, \textit{txtstr})

    \vspace{-1.5ex}

    \rule{\textwidth}{0.5\fboxrule}
\setlength{\parskip}{2ex}
\setlength{\parskip}{1ex}
      \textbf{Return Value}
    \vspace{-1ex}

      \begin{quote}
      pPopupEntry

      \end{quote}

\textbf{Status:} Untested + NoDoc + NoDemo = NOT OK



    \end{boxedminipage}

    \label{xformslib:library:fl_get_select_text_color}
    \index{xformslib \textit{(package)}!xformslib.library \textit{(module)}!xformslib.library.fl\_get\_select\_text\_color \textit{(function)}}

    \vspace{0.5ex}

\hspace{.8\funcindent}\begin{boxedminipage}{\funcwidth}

    \raggedright \textbf{fl\_get\_select\_text\_color}(\textit{pObject})

    \vspace{-1.5ex}

    \rule{\textwidth}{0.5\fboxrule}
\setlength{\parskip}{2ex}
\setlength{\parskip}{1ex}
      \textbf{Return Value}
    \vspace{-1ex}

      \begin{quote}
      color

      \end{quote}

\textbf{Status:} Untested + NoDoc + NoDemo = NOT OK



    \end{boxedminipage}

    \label{xformslib:library:fl_set_select_text_color}
    \index{xformslib \textit{(package)}!xformslib.library \textit{(module)}!xformslib.library.fl\_set\_select\_text\_color \textit{(function)}}

    \vspace{0.5ex}

\hspace{.8\funcindent}\begin{boxedminipage}{\funcwidth}

    \raggedright \textbf{fl\_set\_select\_text\_color}(\textit{pObject}, \textit{colr})

    \vspace{-1.5ex}

    \rule{\textwidth}{0.5\fboxrule}
\setlength{\parskip}{2ex}
\setlength{\parskip}{1ex}
      \textbf{Return Value}
    \vspace{-1ex}

      \begin{quote}
      color

      \end{quote}

\textbf{Status:} Untested + NoDoc + NoDemo = NOT OK



    \end{boxedminipage}

    \label{xformslib:library:fl_get_select_text_font}
    \index{xformslib \textit{(package)}!xformslib.library \textit{(module)}!xformslib.library.fl\_get\_select\_text\_font \textit{(function)}}

    \vspace{0.5ex}

\hspace{.8\funcindent}\begin{boxedminipage}{\funcwidth}

    \raggedright \textbf{fl\_get\_select\_text\_font}(\textit{pObject})

    \vspace{-1.5ex}

    \rule{\textwidth}{0.5\fboxrule}
\setlength{\parskip}{2ex}
\setlength{\parskip}{1ex}
      \textbf{Return Value}
    \vspace{-1ex}

      \begin{quote}
      num, num1, num2

      \end{quote}

\textbf{Attention:} API change from XForms - upstream was fl\_get\_select\_text\_font(pObject, 
p2, p3)



\textbf{Status:} Untested + NoDoc + NoDemo = NOT OK



    \end{boxedminipage}

    \label{xformslib:library:fl_set_select_text_font}
    \index{xformslib \textit{(package)}!xformslib.library \textit{(module)}!xformslib.library.fl\_set\_select\_text\_font \textit{(function)}}

    \vspace{0.5ex}

\hspace{.8\funcindent}\begin{boxedminipage}{\funcwidth}

    \raggedright \textbf{fl\_set\_select\_text\_font}(\textit{pObject}, \textit{p2}, \textit{p3})

    \vspace{-1.5ex}

    \rule{\textwidth}{0.5\fboxrule}
\setlength{\parskip}{2ex}
\setlength{\parskip}{1ex}
      \textbf{Return Value}
    \vspace{-1ex}

      \begin{quote}
      font num

      \end{quote}

\textbf{Status:} Untested + NoDoc + NoDemo = NOT OK



    \end{boxedminipage}

    \label{xformslib:library:fl_get_select_text_align}
    \index{xformslib \textit{(package)}!xformslib.library \textit{(module)}!xformslib.library.fl\_get\_select\_text\_align \textit{(function)}}

    \vspace{0.5ex}

\hspace{.8\funcindent}\begin{boxedminipage}{\funcwidth}

    \raggedright \textbf{fl\_get\_select\_text\_align}(\textit{pObject})

    \vspace{-1.5ex}

    \rule{\textwidth}{0.5\fboxrule}
\setlength{\parskip}{2ex}
\setlength{\parskip}{1ex}
      \textbf{Return Value}
    \vspace{-1ex}

      \begin{quote}
      num

      \end{quote}

\textbf{Status:} Untested + NoDoc + NoDemo = NOT OK



    \end{boxedminipage}

    \label{xformslib:library:fl_set_select_text_align}
    \index{xformslib \textit{(package)}!xformslib.library \textit{(module)}!xformslib.library.fl\_set\_select\_text\_align \textit{(function)}}

    \vspace{0.5ex}

\hspace{.8\funcindent}\begin{boxedminipage}{\funcwidth}

    \raggedright \textbf{fl\_set\_select\_text\_align}(\textit{pObject}, \textit{p2})

    \vspace{-1.5ex}

    \rule{\textwidth}{0.5\fboxrule}
\setlength{\parskip}{2ex}
\setlength{\parskip}{1ex}
      \textbf{Return Value}
    \vspace{-1ex}

      \begin{quote}
      num

      \end{quote}

\textbf{Status:} Untested + NoDoc + NoDemo = NOT OK



    \end{boxedminipage}

    \label{xformslib:library:fl_set_select_policy}
    \index{xformslib \textit{(package)}!xformslib.library \textit{(module)}!xformslib.library.fl\_set\_select\_policy \textit{(function)}}

    \vspace{0.5ex}

\hspace{.8\funcindent}\begin{boxedminipage}{\funcwidth}

    \raggedright \textbf{fl\_set\_select\_policy}(\textit{pObject}, \textit{num})

    \vspace{-1.5ex}

    \rule{\textwidth}{0.5\fboxrule}
\setlength{\parskip}{2ex}
\setlength{\parskip}{1ex}
      \textbf{Return Value}
    \vspace{-1ex}

      \begin{quote}
      num

      \end{quote}

\textbf{Status:} Tested + NoDoc + Demo = OK



    \end{boxedminipage}

    \label{xformslib:library:fl_add_slider}
    \index{xformslib \textit{(package)}!xformslib.library \textit{(module)}!xformslib.library.fl\_add\_slider \textit{(function)}}

    \vspace{0.5ex}

\hspace{.8\funcindent}\begin{boxedminipage}{\funcwidth}

    \raggedright \textbf{fl\_add\_slider}(\textit{slidertype}, \textit{x}, \textit{y}, \textit{w}, \textit{h}, \textit{label})

    \vspace{-1.5ex}

    \rule{\textwidth}{0.5\fboxrule}
\setlength{\parskip}{2ex}
    Adds a slider to a form. No value is displayed.

\setlength{\parskip}{1ex}
      \textbf{Parameters}
      \vspace{-1ex}

      \begin{quote}
        \begin{Ventry}{xxxxxxxxxx}

          \item[slidertype]

          type of the slider to be added

            {\it (type=[num./int] from xfdata module FL\_VERT\_SLIDER, FL\_HOR\_SLIDER, 
FL\_VERT\_FILL\_SLIDER, FL\_HOR\_FILL\_SLIDER, FL\_VERT\_NICE\_SLIDER, 
FL\_HOR\_NICE\_SLIDER, FL\_VERT\_BROWSER\_SLIDER, FL\_HOR\_BROWSER\_SLIDER,
FL\_VERT\_BROWSER\_SLIDER2, FL\_HOR\_BROWSER\_SLIDER2, 
FL\_VERT\_THIN\_SLIDER, FL\_HOR\_THIN\_SLIDER, FL\_VERT\_THIN\_SLIDER, 
FL\_HOR\_THIN\_SLIDER, FL\_VERT\_NICE\_SLIDER2, FL\_HOR\_NICE\_SLIDER2, 
FL\_VERT\_BASIC\_SLIDER, FL\_HOR\_BASIC\_SLIDER)}

          \item[x]

          horizontal position (upper-left corner)

          \item[y]

          vertical position (upper-left corner)

          \item[w]

          width in coord units

          \item[h]

          height in coord units

          \item[label]

          label of the slider (placed below it by default)

        \end{Ventry}

      \end{quote}

      \textbf{Return Value}
    \vspace{-1ex}

      \begin{quote}
      pObject

      \end{quote}

\textbf{Status:} Tested + NoDoc + Demo = OK



    \end{boxedminipage}

    \label{xformslib:library:fl_add_valslider}
    \index{xformslib \textit{(package)}!xformslib.library \textit{(module)}!xformslib.library.fl\_add\_valslider \textit{(function)}}

    \vspace{0.5ex}

\hspace{.8\funcindent}\begin{boxedminipage}{\funcwidth}

    \raggedright \textbf{fl\_add\_valslider}(\textit{slidertype}, \textit{x}, \textit{y}, \textit{w}, \textit{h}, \textit{label})

    \vspace{-1.5ex}

    \rule{\textwidth}{0.5\fboxrule}
\setlength{\parskip}{2ex}
    Adds a slider to a form. Its value is displayed above or to the left of
    the slider.

\setlength{\parskip}{1ex}
      \textbf{Parameters}
      \vspace{-1ex}

      \begin{quote}
        \begin{Ventry}{xxxxxxxxxx}

          \item[slidertype]

          type of the slider

            {\it (type=[num./int] from xfdata module FL\_VERT\_SLIDER, FL\_HOR\_SLIDER, 
FL\_VERT\_FILL\_SLIDER, FL\_HOR\_FILL\_SLIDER, FL\_VERT\_NICE\_SLIDER, 
FL\_HOR\_NICE\_SLIDER, FL\_VERT\_BROWSER\_SLIDER, FL\_HOR\_BROWSER\_SLIDER,
FL\_VERT\_BROWSER\_SLIDER2, FL\_HOR\_BROWSER\_SLIDER2, 
FL\_VERT\_THIN\_SLIDER, FL\_HOR\_THIN\_SLIDER, FL\_VERT\_THIN\_SLIDER, 
FL\_HOR\_THIN\_SLIDER, FL\_VERT\_NICE\_SLIDER2, FL\_HOR\_NICE\_SLIDER2, 
FL\_VERT\_BASIC\_SLIDER, FL\_HOR\_BASIC\_SLIDER)}

          \item[x]

          horizontal position (upper-left corner)

          \item[y]

          vertical position (upper-left corner)

          \item[w]

          width in coord units

          \item[h]

          height in coord units

          \item[label]

          text label of slider

        \end{Ventry}

      \end{quote}

      \textbf{Return Value}
    \vspace{-1ex}

      \begin{quote}
      pObject

      \end{quote}

\textbf{Status:} Tested + NoDoc + Demo = OK



    \end{boxedminipage}

    \label{xformslib:library:fl_set_slider_value}
    \index{xformslib \textit{(package)}!xformslib.library \textit{(module)}!xformslib.library.fl\_set\_slider\_value \textit{(function)}}

    \vspace{0.5ex}

\hspace{.8\funcindent}\begin{boxedminipage}{\funcwidth}

    \raggedright \textbf{fl\_set\_slider\_value}(\textit{pObject}, \textit{val})

    \vspace{-1.5ex}

    \rule{\textwidth}{0.5\fboxrule}
\setlength{\parskip}{2ex}
    Changes the value of a slider.

\setlength{\parskip}{1ex}
      \textbf{Parameters}
      \vspace{-1ex}

      \begin{quote}
        \begin{Ventry}{xxxxxxx}

          \item[pObject]

          pointer to object

          \item[val]

          new value of slider

        \end{Ventry}

      \end{quote}

\textbf{Status:} Tested + NoDoc + Demo = OK



    \end{boxedminipage}

    \label{xformslib:library:fl_get_slider_value}
    \index{xformslib \textit{(package)}!xformslib.library \textit{(module)}!xformslib.library.fl\_get\_slider\_value \textit{(function)}}

    \vspace{0.5ex}

\hspace{.8\funcindent}\begin{boxedminipage}{\funcwidth}

    \raggedright \textbf{fl\_get\_slider\_value}(\textit{pObject})

    \vspace{-1.5ex}

    \rule{\textwidth}{0.5\fboxrule}
\setlength{\parskip}{2ex}
    Returns value of a slider.

\setlength{\parskip}{1ex}
      \textbf{Parameters}
      \vspace{-1ex}

      \begin{quote}
        \begin{Ventry}{xxxxxxx}

          \item[pObject]

          pointer to object

        \end{Ventry}

      \end{quote}

      \textbf{Return Value}
    \vspace{-1ex}

      \begin{quote}
      value[float]

      \end{quote}

\textbf{Status:} Tested + NoDoc + Demo = OK



    \end{boxedminipage}

    \label{xformslib:library:fl_set_slider_bounds}
    \index{xformslib \textit{(package)}!xformslib.library \textit{(module)}!xformslib.library.fl\_set\_slider\_bounds \textit{(function)}}

    \vspace{0.5ex}

\hspace{.8\funcindent}\begin{boxedminipage}{\funcwidth}

    \raggedright \textbf{fl\_set\_slider\_bounds}(\textit{pObject}, \textit{minbound}, \textit{maxbound})

    \vspace{-1.5ex}

    \rule{\textwidth}{0.5\fboxrule}
\setlength{\parskip}{2ex}
    Sets bounds/limits of a slider.

\setlength{\parskip}{1ex}
      \textbf{Parameters}
      \vspace{-1ex}

      \begin{quote}
        \begin{Ventry}{xxxxxxxx}

          \item[pObject]

          pointer to object

          \item[minbound]

          minimum bound of slider

          \item[maxbound]

          maximum bound of slider

        \end{Ventry}

      \end{quote}

\textbf{Status:} Tested + NoDoc + Demo = OK



    \end{boxedminipage}

    \label{xformslib:library:fl_get_slider_bounds}
    \index{xformslib \textit{(package)}!xformslib.library \textit{(module)}!xformslib.library.fl\_get\_slider\_bounds \textit{(function)}}

    \vspace{0.5ex}

\hspace{.8\funcindent}\begin{boxedminipage}{\funcwidth}

    \raggedright \textbf{fl\_get\_slider\_bounds}(\textit{pObject})

    \vspace{-1.5ex}

    \rule{\textwidth}{0.5\fboxrule}
\setlength{\parskip}{2ex}
    Returns bounds/limits of a slider.

\setlength{\parskip}{1ex}
      \textbf{Parameters}
      \vspace{-1ex}

      \begin{quote}
        \begin{Ventry}{xxxxxxx}

          \item[pObject]

          pointer to object

        \end{Ventry}

      \end{quote}

      \textbf{Return Value}
    \vspace{-1ex}

      \begin{quote}
      minbound[float], maxbound[float]

      \end{quote}

\textbf{Attention:} API change from XForms - upstream was fl\_get\_slider\_bounds(pObject, 
minbound, maxbound)



\textbf{Status:} Untested + NoDoc + NoDemo = NOT OK



    \end{boxedminipage}

    \label{xformslib:library:fl_set_slider_step}
    \index{xformslib \textit{(package)}!xformslib.library \textit{(module)}!xformslib.library.fl\_set\_slider\_step \textit{(function)}}

    \vspace{0.5ex}

\hspace{.8\funcindent}\begin{boxedminipage}{\funcwidth}

    \raggedright \textbf{fl\_set\_slider\_step}(\textit{pObject}, \textit{value})

    \vspace{-1.5ex}

    \rule{\textwidth}{0.5\fboxrule}
\setlength{\parskip}{2ex}
\setlength{\parskip}{1ex}
\textbf{Status:} Untested + NoDoc + NoDemo = NOT OK



    \end{boxedminipage}

    \label{xformslib:library:fl_set_slider_increment}
    \index{xformslib \textit{(package)}!xformslib.library \textit{(module)}!xformslib.library.fl\_set\_slider\_increment \textit{(function)}}

    \vspace{0.5ex}

\hspace{.8\funcindent}\begin{boxedminipage}{\funcwidth}

    \raggedright \textbf{fl\_set\_slider\_increment}(\textit{pObject}, \textit{leftbtnval}, \textit{midlbtnval})

    \vspace{-1.5ex}

    \rule{\textwidth}{0.5\fboxrule}
\setlength{\parskip}{2ex}
\setlength{\parskip}{1ex}
\textbf{Status:} Untested + NoDoc + NoDemo = NOT OK



    \end{boxedminipage}

    \label{xformslib:library:fl_get_slider_increment}
    \index{xformslib \textit{(package)}!xformslib.library \textit{(module)}!xformslib.library.fl\_get\_slider\_increment \textit{(function)}}

    \vspace{0.5ex}

\hspace{.8\funcindent}\begin{boxedminipage}{\funcwidth}

    \raggedright \textbf{fl\_get\_slider\_increment}(\textit{pObject})

    \vspace{-1.5ex}

    \rule{\textwidth}{0.5\fboxrule}
\setlength{\parskip}{2ex}
\setlength{\parskip}{1ex}
      \textbf{Return Value}
    \vspace{-1ex}

      \begin{quote}
      leftbtnval, midlbtnval

      \end{quote}

\textbf{Attention:} API change from XForms - upstream was fl\_get\_slider\_increment(pObject, 
leftbtnval, midlbtnval)



\textbf{Status:} Untested + NoDoc + NoDemo = NOT OK



    \end{boxedminipage}

    \label{xformslib:library:fl_set_slider_size}
    \index{xformslib \textit{(package)}!xformslib.library \textit{(module)}!xformslib.library.fl\_set\_slider\_size \textit{(function)}}

    \vspace{0.5ex}

\hspace{.8\funcindent}\begin{boxedminipage}{\funcwidth}

    \raggedright \textbf{fl\_set\_slider\_size}(\textit{pObject}, \textit{size})

    \vspace{-1.5ex}

    \rule{\textwidth}{0.5\fboxrule}
\setlength{\parskip}{2ex}
    Sets the size of a slider.

\setlength{\parskip}{1ex}
      \textbf{Parameters}
      \vspace{-1ex}

      \begin{quote}
        \begin{Ventry}{xxxxxxx}

          \item[pObject]

          pointer to object

          \item[size]

          value of size of the slider

        \end{Ventry}

      \end{quote}

\textbf{Status:} Tested + NoDoc + Demo = OK



    \end{boxedminipage}

    \label{xformslib:library:fl_set_slider_precision}
    \index{xformslib \textit{(package)}!xformslib.library \textit{(module)}!xformslib.library.fl\_set\_slider\_precision \textit{(function)}}

    \vspace{0.5ex}

\hspace{.8\funcindent}\begin{boxedminipage}{\funcwidth}

    \raggedright \textbf{fl\_set\_slider\_precision}(\textit{pObject}, \textit{precnum})

    \vspace{-1.5ex}

    \rule{\textwidth}{0.5\fboxrule}
\setlength{\parskip}{2ex}
    Sets precision with which value a valslider is shown.

\setlength{\parskip}{1ex}
      \textbf{Parameters}
      \vspace{-1ex}

      \begin{quote}
        \begin{Ventry}{xxxxxxx}

          \item[pObject]

          pointer to object

          \item[precnum]

          precision of shown value

        \end{Ventry}

      \end{quote}

\textbf{Status:} Untested + NoDoc + NoDemo = NOT OK



    \end{boxedminipage}

    \label{xformslib:library:fl_set_slider_filter}
    \index{xformslib \textit{(package)}!xformslib.library \textit{(module)}!xformslib.library.fl\_set\_slider\_filter \textit{(function)}}

    \vspace{0.5ex}

\hspace{.8\funcindent}\begin{boxedminipage}{\funcwidth}

    \raggedright \textbf{fl\_set\_slider\_filter}(\textit{pObject}, \textit{py\_ValFilter})

    \vspace{-1.5ex}

    \rule{\textwidth}{0.5\fboxrule}
\setlength{\parskip}{2ex}
    Overrides the default (slider value shown in floating point format) by 
    registering a filter function.

\setlength{\parskip}{1ex}
      \textbf{Parameters}
      \vspace{-1ex}

      \begin{quote}
        \begin{Ventry}{xxxxxxxxxxxx}

          \item[pObject]

          pointer to oject

          \item[py\_ValFilter]

          python function, fn(pObject, valfloat, intprecis) -{\textgreater}
          string

        \end{Ventry}

      \end{quote}

\textbf{Status:} Untested + NoDoc + NoDemo = NOT OK



    \end{boxedminipage}

    \label{xformslib:library:fl_add_spinner}
    \index{xformslib \textit{(package)}!xformslib.library \textit{(module)}!xformslib.library.fl\_add\_spinner \textit{(function)}}

    \vspace{0.5ex}

\hspace{.8\funcindent}\begin{boxedminipage}{\funcwidth}

    \raggedright \textbf{fl\_add\_spinner}(\textit{spinnertype}, \textit{x}, \textit{y}, \textit{w}, \textit{h}, \textit{label})

    \vspace{-1.5ex}

    \rule{\textwidth}{0.5\fboxrule}
\setlength{\parskip}{2ex}
    Adds a spinner object.

\setlength{\parskip}{1ex}
      \textbf{Parameters}
      \vspace{-1ex}

      \begin{quote}
        \begin{Ventry}{xxxxxxxxxxx}

          \item[spinnertype]

          type of spinner to be added

            {\it (type=[num./int] from xfdata module FL\_INT\_SPINNER, FL\_FLOAT\_SPINNER)}

          \item[x]

          horizontal position (upper-left corner)

          \item[x]

          vertical position (upper-left corner)

          \item[w]

          width in coord units

          \item[h]

          height in coord units

          \item[label]

          text label of spinner

        \end{Ventry}

      \end{quote}

      \textbf{Return Value}
    \vspace{-1ex}

      \begin{quote}
      pObject

      \end{quote}

\textbf{Status:} Untested + NoDoc + NoDemo = NOT OK



    \end{boxedminipage}

    \label{xformslib:library:fl_get_spinner_value}
    \index{xformslib \textit{(package)}!xformslib.library \textit{(module)}!xformslib.library.fl\_get\_spinner\_value \textit{(function)}}

    \vspace{0.5ex}

\hspace{.8\funcindent}\begin{boxedminipage}{\funcwidth}

    \raggedright \textbf{fl\_get\_spinner\_value}(\textit{pObject})

    \vspace{-1.5ex}

    \rule{\textwidth}{0.5\fboxrule}
\setlength{\parskip}{2ex}
\setlength{\parskip}{1ex}
      \textbf{Return Value}
    \vspace{-1ex}

      \begin{quote}
      floatval

      \end{quote}

\textbf{Status:} Untested + NoDoc + NoDemo = NOT OK



    \end{boxedminipage}

    \label{xformslib:library:fl_set_spinner_value}
    \index{xformslib \textit{(package)}!xformslib.library \textit{(module)}!xformslib.library.fl\_set\_spinner\_value \textit{(function)}}

    \vspace{0.5ex}

\hspace{.8\funcindent}\begin{boxedminipage}{\funcwidth}

    \raggedright \textbf{fl\_set\_spinner\_value}(\textit{pObject}, \textit{val})

    \vspace{-1.5ex}

    \rule{\textwidth}{0.5\fboxrule}
\setlength{\parskip}{2ex}
\setlength{\parskip}{1ex}
      \textbf{Return Value}
    \vspace{-1ex}

      \begin{quote}
      num

      \end{quote}

\textbf{Status:} Untested + NoDoc + NoDemo = NOT OK



    \end{boxedminipage}

    \label{xformslib:library:fl_set_spinner_bounds}
    \index{xformslib \textit{(package)}!xformslib.library \textit{(module)}!xformslib.library.fl\_set\_spinner\_bounds \textit{(function)}}

    \vspace{0.5ex}

\hspace{.8\funcindent}\begin{boxedminipage}{\funcwidth}

    \raggedright \textbf{fl\_set\_spinner\_bounds}(\textit{pObject}, \textit{minbound}, \textit{maxbound})

    \vspace{-1.5ex}

    \rule{\textwidth}{0.5\fboxrule}
\setlength{\parskip}{2ex}
\setlength{\parskip}{1ex}
\textbf{Status:} Untested + NoDoc + NoDemo = NOT OK



    \end{boxedminipage}

    \label{xformslib:library:fl_get_spinner_bounds}
    \index{xformslib \textit{(package)}!xformslib.library \textit{(module)}!xformslib.library.fl\_get\_spinner\_bounds \textit{(function)}}

    \vspace{0.5ex}

\hspace{.8\funcindent}\begin{boxedminipage}{\funcwidth}

    \raggedright \textbf{fl\_get\_spinner\_bounds}(\textit{pObject})

    \vspace{-1.5ex}

    \rule{\textwidth}{0.5\fboxrule}
\setlength{\parskip}{2ex}
\setlength{\parskip}{1ex}
      \textbf{Return Value}
    \vspace{-1ex}

      \begin{quote}
      minbound, maxbound

      \end{quote}

\textbf{Attention:} API change from XForms - upstream was fl\_get\_spinner\_bounds(pObject, 
minbound, maxbound)



\textbf{Status:} Untested + NoDoc + NoDemo = NOT OK



    \end{boxedminipage}

    \label{xformslib:library:fl_set_spinner_step}
    \index{xformslib \textit{(package)}!xformslib.library \textit{(module)}!xformslib.library.fl\_set\_spinner\_step \textit{(function)}}

    \vspace{0.5ex}

\hspace{.8\funcindent}\begin{boxedminipage}{\funcwidth}

    \raggedright \textbf{fl\_set\_spinner\_step}(\textit{pObject}, \textit{step})

    \vspace{-1.5ex}

    \rule{\textwidth}{0.5\fboxrule}
\setlength{\parskip}{2ex}
\setlength{\parskip}{1ex}
\textbf{Status:} Untested + NoDoc + NoDemo = NOT OK



    \end{boxedminipage}

    \label{xformslib:library:fl_get_spinner_step}
    \index{xformslib \textit{(package)}!xformslib.library \textit{(module)}!xformslib.library.fl\_get\_spinner\_step \textit{(function)}}

    \vspace{0.5ex}

\hspace{.8\funcindent}\begin{boxedminipage}{\funcwidth}

    \raggedright \textbf{fl\_get\_spinner\_step}(\textit{pObject})

    \vspace{-1.5ex}

    \rule{\textwidth}{0.5\fboxrule}
\setlength{\parskip}{2ex}
\setlength{\parskip}{1ex}
      \textbf{Return Value}
    \vspace{-1ex}

      \begin{quote}
      num

      \end{quote}

\textbf{Status:} Untested + NoDoc + NoDemo = NOT OK



    \end{boxedminipage}

    \label{xformslib:library:fl_set_spinner_precision}
    \index{xformslib \textit{(package)}!xformslib.library \textit{(module)}!xformslib.library.fl\_set\_spinner\_precision \textit{(function)}}

    \vspace{0.5ex}

\hspace{.8\funcindent}\begin{boxedminipage}{\funcwidth}

    \raggedright \textbf{fl\_set\_spinner\_precision}(\textit{pObject}, \textit{precnum})

    \vspace{-1.5ex}

    \rule{\textwidth}{0.5\fboxrule}
\setlength{\parskip}{2ex}
\setlength{\parskip}{1ex}
\textbf{Status:} Untested + NoDoc + NoDemo = NOT OK



    \end{boxedminipage}

    \label{xformslib:library:fl_get_spinner_precision}
    \index{xformslib \textit{(package)}!xformslib.library \textit{(module)}!xformslib.library.fl\_get\_spinner\_precision \textit{(function)}}

    \vspace{0.5ex}

\hspace{.8\funcindent}\begin{boxedminipage}{\funcwidth}

    \raggedright \textbf{fl\_get\_spinner\_precision}(\textit{pObject})

    \vspace{-1.5ex}

    \rule{\textwidth}{0.5\fboxrule}
\setlength{\parskip}{2ex}
\setlength{\parskip}{1ex}
      \textbf{Return Value}
    \vspace{-1ex}

      \begin{quote}
      num

      \end{quote}

\textbf{Status:} Untested + NoDoc + NoDemo = NOT OK



    \end{boxedminipage}

    \label{xformslib:library:fl_get_spinner_input}
    \index{xformslib \textit{(package)}!xformslib.library \textit{(module)}!xformslib.library.fl\_get\_spinner\_input \textit{(function)}}

    \vspace{0.5ex}

\hspace{.8\funcindent}\begin{boxedminipage}{\funcwidth}

    \raggedright \textbf{fl\_get\_spinner\_input}(\textit{pObject})

    \vspace{-1.5ex}

    \rule{\textwidth}{0.5\fboxrule}
\setlength{\parskip}{2ex}
\setlength{\parskip}{1ex}
      \textbf{Return Value}
    \vspace{-1ex}

      \begin{quote}
      pObject

      \end{quote}

\textbf{Status:} Untested + NoDoc + NoDemo = NOT OK



    \end{boxedminipage}

    \label{xformslib:library:fl_get_spinner_up_button}
    \index{xformslib \textit{(package)}!xformslib.library \textit{(module)}!xformslib.library.fl\_get\_spinner\_up\_button \textit{(function)}}

    \vspace{0.5ex}

\hspace{.8\funcindent}\begin{boxedminipage}{\funcwidth}

    \raggedright \textbf{fl\_get\_spinner\_up\_button}(\textit{pObject})

    \vspace{-1.5ex}

    \rule{\textwidth}{0.5\fboxrule}
\setlength{\parskip}{2ex}
\setlength{\parskip}{1ex}
      \textbf{Return Value}
    \vspace{-1ex}

      \begin{quote}
      pObject

      \end{quote}

\textbf{Status:} Untested + NoDoc + NoDemo = NOT OK



    \end{boxedminipage}

    \label{xformslib:library:fl_get_spinner_down_button}
    \index{xformslib \textit{(package)}!xformslib.library \textit{(module)}!xformslib.library.fl\_get\_spinner\_down\_button \textit{(function)}}

    \vspace{0.5ex}

\hspace{.8\funcindent}\begin{boxedminipage}{\funcwidth}

    \raggedright \textbf{fl\_get\_spinner\_down\_button}(\textit{pObject})

    \vspace{-1.5ex}

    \rule{\textwidth}{0.5\fboxrule}
\setlength{\parskip}{2ex}
\setlength{\parskip}{1ex}
      \textbf{Return Value}
    \vspace{-1ex}

      \begin{quote}
      pObject

      \end{quote}

\textbf{Status:} Untested + NoDoc + NoDemo = NOT OK



    \end{boxedminipage}

    \label{xformslib:library:fl_add_tabfolder}
    \index{xformslib \textit{(package)}!xformslib.library \textit{(module)}!xformslib.library.fl\_add\_tabfolder \textit{(function)}}

    \vspace{0.5ex}

\hspace{.8\funcindent}\begin{boxedminipage}{\funcwidth}

    \raggedright \textbf{fl\_add\_tabfolder}(\textit{foldertype}, \textit{x}, \textit{y}, \textit{w}, \textit{h}, \textit{label})

    \vspace{-1.5ex}

    \rule{\textwidth}{0.5\fboxrule}
\setlength{\parskip}{2ex}
    Adds a tabfolder object.

\setlength{\parskip}{1ex}
      \textbf{Parameters}
      \vspace{-1ex}

      \begin{quote}
        \begin{Ventry}{xxxxxxxxxx}

          \item[foldertype]

          type of tabfolder to be added

          \item[x]

          horizontal position (upper-left corner)

          \item[x]

          vertical position (upper-left corner)

          \item[w]

          width in coord units

          \item[h]

          height in coord units

          \item[label]

          text label of tabfolder

        \end{Ventry}

      \end{quote}

      \textbf{Return Value}
    \vspace{-1ex}

      \begin{quote}
      pObject

      \end{quote}

\textbf{Status:} Untested + NoDoc + NoDemo = NOT OK



    \end{boxedminipage}

    \label{xformslib:library:fl_addto_tabfolder}
    \index{xformslib \textit{(package)}!xformslib.library \textit{(module)}!xformslib.library.fl\_addto\_tabfolder \textit{(function)}}

    \vspace{0.5ex}

\hspace{.8\funcindent}\begin{boxedminipage}{\funcwidth}

    \raggedright \textbf{fl\_addto\_tabfolder}(\textit{pObject}, \textit{title}, \textit{pForm})

    \vspace{-1.5ex}

    \rule{\textwidth}{0.5\fboxrule}
\setlength{\parskip}{2ex}
\setlength{\parskip}{1ex}
      \textbf{Return Value}
    \vspace{-1ex}

      \begin{quote}
      pObject

      \end{quote}

\textbf{Status:} Tested + NoDoc + Demo = OK



    \end{boxedminipage}

    \label{xformslib:library:fl_get_tabfolder_folder_bynumber}
    \index{xformslib \textit{(package)}!xformslib.library \textit{(module)}!xformslib.library.fl\_get\_tabfolder\_folder\_bynumber \textit{(function)}}

    \vspace{0.5ex}

\hspace{.8\funcindent}\begin{boxedminipage}{\funcwidth}

    \raggedright \textbf{fl\_get\_tabfolder\_folder\_bynumber}(\textit{pObject}, \textit{num})

    \vspace{-1.5ex}

    \rule{\textwidth}{0.5\fboxrule}
\setlength{\parskip}{2ex}
\setlength{\parskip}{1ex}
      \textbf{Return Value}
    \vspace{-1ex}

      \begin{quote}
      pForm

      \end{quote}

\textbf{Status:} Untested + NoDoc + NoDemo = NOT OK



    \end{boxedminipage}

    \label{xformslib:library:fl_get_tabfolder_folder_byname}
    \index{xformslib \textit{(package)}!xformslib.library \textit{(module)}!xformslib.library.fl\_get\_tabfolder\_folder\_byname \textit{(function)}}

    \vspace{0.5ex}

\hspace{.8\funcindent}\begin{boxedminipage}{\funcwidth}

    \raggedright \textbf{fl\_get\_tabfolder\_folder\_byname}(\textit{pObject}, \textit{name})

    \vspace{-1.5ex}

    \rule{\textwidth}{0.5\fboxrule}
\setlength{\parskip}{2ex}
\setlength{\parskip}{1ex}
      \textbf{Return Value}
    \vspace{-1ex}

      \begin{quote}
      pForm

      \end{quote}

\textbf{Status:} Untested + NoDoc + NoDemo = NOT OK



    \end{boxedminipage}

    \label{xformslib:library:fl_delete_folder}
    \index{xformslib \textit{(package)}!xformslib.library \textit{(module)}!xformslib.library.fl\_delete\_folder \textit{(function)}}

    \vspace{0.5ex}

\hspace{.8\funcindent}\begin{boxedminipage}{\funcwidth}

    \raggedright \textbf{fl\_delete\_folder}(\textit{pObject}, \textit{pForm})

    \vspace{-1.5ex}

    \rule{\textwidth}{0.5\fboxrule}
\setlength{\parskip}{2ex}
\setlength{\parskip}{1ex}
\textbf{Status:} Untested + NoDoc + NoDemo = NOT OK



    \end{boxedminipage}

    \label{xformslib:library:fl_delete_folder_bynumber}
    \index{xformslib \textit{(package)}!xformslib.library \textit{(module)}!xformslib.library.fl\_delete\_folder\_bynumber \textit{(function)}}

    \vspace{0.5ex}

\hspace{.8\funcindent}\begin{boxedminipage}{\funcwidth}

    \raggedright \textbf{fl\_delete\_folder\_bynumber}(\textit{pObject}, \textit{num})

    \vspace{-1.5ex}

    \rule{\textwidth}{0.5\fboxrule}
\setlength{\parskip}{2ex}
\setlength{\parskip}{1ex}
\textbf{Status:} Untested + NoDoc + NoDemo = NOT OK



    \end{boxedminipage}

    \label{xformslib:library:fl_delete_folder_byname}
    \index{xformslib \textit{(package)}!xformslib.library \textit{(module)}!xformslib.library.fl\_delete\_folder\_byname \textit{(function)}}

    \vspace{0.5ex}

\hspace{.8\funcindent}\begin{boxedminipage}{\funcwidth}

    \raggedright \textbf{fl\_delete\_folder\_byname}(\textit{pObject}, \textit{name})

    \vspace{-1.5ex}

    \rule{\textwidth}{0.5\fboxrule}
\setlength{\parskip}{2ex}
\setlength{\parskip}{1ex}
\textbf{Status:} Untested + NoDoc + NoDemo = NOT OK



    \end{boxedminipage}

    \label{xformslib:library:fl_set_folder}
    \index{xformslib \textit{(package)}!xformslib.library \textit{(module)}!xformslib.library.fl\_set\_folder \textit{(function)}}

    \vspace{0.5ex}

\hspace{.8\funcindent}\begin{boxedminipage}{\funcwidth}

    \raggedright \textbf{fl\_set\_folder}(\textit{pObject}, \textit{pForm})

    \vspace{-1.5ex}

    \rule{\textwidth}{0.5\fboxrule}
\setlength{\parskip}{2ex}
\setlength{\parskip}{1ex}
\textbf{Status:} Untested + NoDoc + NoDemo = NOT OK



    \end{boxedminipage}

    \label{xformslib:library:fl_set_folder_byname}
    \index{xformslib \textit{(package)}!xformslib.library \textit{(module)}!xformslib.library.fl\_set\_folder\_byname \textit{(function)}}

    \vspace{0.5ex}

\hspace{.8\funcindent}\begin{boxedminipage}{\funcwidth}

    \raggedright \textbf{fl\_set\_folder\_byname}(\textit{pObject}, \textit{name})

    \vspace{-1.5ex}

    \rule{\textwidth}{0.5\fboxrule}
\setlength{\parskip}{2ex}
\setlength{\parskip}{1ex}
\textbf{Status:} Untested + NoDoc + NoDemo = NOT OK



    \end{boxedminipage}

    \label{xformslib:library:fl_set_folder_bynumber}
    \index{xformslib \textit{(package)}!xformslib.library \textit{(module)}!xformslib.library.fl\_set\_folder\_bynumber \textit{(function)}}

    \vspace{0.5ex}

\hspace{.8\funcindent}\begin{boxedminipage}{\funcwidth}

    \raggedright \textbf{fl\_set\_folder\_bynumber}(\textit{pObject}, \textit{num})

    \vspace{-1.5ex}

    \rule{\textwidth}{0.5\fboxrule}
\setlength{\parskip}{2ex}
\setlength{\parskip}{1ex}
\textbf{Status:} Tested + NoDoc + Demo = OK



    \end{boxedminipage}

    \label{xformslib:library:fl_get_folder}
    \index{xformslib \textit{(package)}!xformslib.library \textit{(module)}!xformslib.library.fl\_get\_folder \textit{(function)}}

    \vspace{0.5ex}

\hspace{.8\funcindent}\begin{boxedminipage}{\funcwidth}

    \raggedright \textbf{fl\_get\_folder}(\textit{pObject})

    \vspace{-1.5ex}

    \rule{\textwidth}{0.5\fboxrule}
\setlength{\parskip}{2ex}
\setlength{\parskip}{1ex}
      \textbf{Parameters}
      \vspace{-1ex}

      \begin{quote}
        \begin{Ventry}{xxxxxxx}

          \item[pObject]

          pointer to object

        \end{Ventry}

      \end{quote}

      \textbf{Return Value}
    \vspace{-1ex}

      \begin{quote}
      pForm

      \end{quote}

\textbf{Status:} Untested + NoDoc + NoDemo = NOT OK



    \end{boxedminipage}

    \label{xformslib:library:fl_get_folder_number}
    \index{xformslib \textit{(package)}!xformslib.library \textit{(module)}!xformslib.library.fl\_get\_folder\_number \textit{(function)}}

    \vspace{0.5ex}

\hspace{.8\funcindent}\begin{boxedminipage}{\funcwidth}

    \raggedright \textbf{fl\_get\_folder\_number}(\textit{pObject})

    \vspace{-1.5ex}

    \rule{\textwidth}{0.5\fboxrule}
\setlength{\parskip}{2ex}
\setlength{\parskip}{1ex}
      \textbf{Parameters}
      \vspace{-1ex}

      \begin{quote}
        \begin{Ventry}{xxxxxxx}

          \item[pObject]

          pointer to object

        \end{Ventry}

      \end{quote}

      \textbf{Return Value}
    \vspace{-1ex}

      \begin{quote}
      folder num

      \end{quote}

\textbf{Status:} Untested + NoDoc + NoDemo = NOT OK



    \end{boxedminipage}

    \label{xformslib:library:fl_get_folder_name}
    \index{xformslib \textit{(package)}!xformslib.library \textit{(module)}!xformslib.library.fl\_get\_folder\_name \textit{(function)}}

    \vspace{0.5ex}

\hspace{.8\funcindent}\begin{boxedminipage}{\funcwidth}

    \raggedright \textbf{fl\_get\_folder\_name}(\textit{pObject})

    \vspace{-1.5ex}

    \rule{\textwidth}{0.5\fboxrule}
\setlength{\parskip}{2ex}
\setlength{\parskip}{1ex}
      \textbf{Parameters}
      \vspace{-1ex}

      \begin{quote}
        \begin{Ventry}{xxxxxxx}

          \item[pObject]

          pointer to object

        \end{Ventry}

      \end{quote}

      \textbf{Return Value}
    \vspace{-1ex}

      \begin{quote}
      name string

      \end{quote}

\textbf{Status:} Untested + NoDoc + NoDemo = NOT OK



    \end{boxedminipage}

    \label{xformslib:library:fl_get_tabfolder_numfolders}
    \index{xformslib \textit{(package)}!xformslib.library \textit{(module)}!xformslib.library.fl\_get\_tabfolder\_numfolders \textit{(function)}}

    \vspace{0.5ex}

\hspace{.8\funcindent}\begin{boxedminipage}{\funcwidth}

    \raggedright \textbf{fl\_get\_tabfolder\_numfolders}(\textit{pObject})

    \vspace{-1.5ex}

    \rule{\textwidth}{0.5\fboxrule}
\setlength{\parskip}{2ex}
\setlength{\parskip}{1ex}
      \textbf{Parameters}
      \vspace{-1ex}

      \begin{quote}
        \begin{Ventry}{xxxxxxx}

          \item[pObject]

          pointer to object

        \end{Ventry}

      \end{quote}

      \textbf{Return Value}
    \vspace{-1ex}

      \begin{quote}
      num

      \end{quote}

\textbf{Status:} Untested + NoDoc + NoDemo = NOT OK



    \end{boxedminipage}

    \label{xformslib:library:fl_get_active_folder}
    \index{xformslib \textit{(package)}!xformslib.library \textit{(module)}!xformslib.library.fl\_get\_active\_folder \textit{(function)}}

    \vspace{0.5ex}

\hspace{.8\funcindent}\begin{boxedminipage}{\funcwidth}

    \raggedright \textbf{fl\_get\_active\_folder}(\textit{pObject})

    \vspace{-1.5ex}

    \rule{\textwidth}{0.5\fboxrule}
\setlength{\parskip}{2ex}
\setlength{\parskip}{1ex}
      \textbf{Parameters}
      \vspace{-1ex}

      \begin{quote}
        \begin{Ventry}{xxxxxxx}

          \item[pObject]

          pointer to object

        \end{Ventry}

      \end{quote}

      \textbf{Return Value}
    \vspace{-1ex}

      \begin{quote}
      pForm

      \end{quote}

\textbf{Status:} Tested + NoDoc + Demo = OK



    \end{boxedminipage}

    \label{xformslib:library:fl_get_active_folder_number}
    \index{xformslib \textit{(package)}!xformslib.library \textit{(module)}!xformslib.library.fl\_get\_active\_folder\_number \textit{(function)}}

    \vspace{0.5ex}

\hspace{.8\funcindent}\begin{boxedminipage}{\funcwidth}

    \raggedright \textbf{fl\_get\_active\_folder\_number}(\textit{pObject})

    \vspace{-1.5ex}

    \rule{\textwidth}{0.5\fboxrule}
\setlength{\parskip}{2ex}
\setlength{\parskip}{1ex}
      \textbf{Parameters}
      \vspace{-1ex}

      \begin{quote}
        \begin{Ventry}{xxxxxxx}

          \item[pObject]

          pointer to object

        \end{Ventry}

      \end{quote}

      \textbf{Return Value}
    \vspace{-1ex}

      \begin{quote}
      num

      \end{quote}

\textbf{Status:} Untested + NoDoc + NoDemo = NOT OK



    \end{boxedminipage}

    \label{xformslib:library:fl_get_active_folder_name}
    \index{xformslib \textit{(package)}!xformslib.library \textit{(module)}!xformslib.library.fl\_get\_active\_folder\_name \textit{(function)}}

    \vspace{0.5ex}

\hspace{.8\funcindent}\begin{boxedminipage}{\funcwidth}

    \raggedright \textbf{fl\_get\_active\_folder\_name}(\textit{pObject})

    \vspace{-1.5ex}

    \rule{\textwidth}{0.5\fboxrule}
\setlength{\parskip}{2ex}
\setlength{\parskip}{1ex}
      \textbf{Parameters}
      \vspace{-1ex}

      \begin{quote}
        \begin{Ventry}{xxxxxxx}

          \item[pObject]

          pointer to object

        \end{Ventry}

      \end{quote}

      \textbf{Return Value}
    \vspace{-1ex}

      \begin{quote}
      name string

      \end{quote}

\textbf{Status:} Untested + NoDoc + NoDemo = NOT OK



    \end{boxedminipage}

    \label{xformslib:library:fl_get_folder_area}
    \index{xformslib \textit{(package)}!xformslib.library \textit{(module)}!xformslib.library.fl\_get\_folder\_area \textit{(function)}}

    \vspace{0.5ex}

\hspace{.8\funcindent}\begin{boxedminipage}{\funcwidth}

    \raggedright \textbf{fl\_get\_folder\_area}(\textit{pObject})

    \vspace{-1.5ex}

    \rule{\textwidth}{0.5\fboxrule}
\setlength{\parskip}{2ex}
\setlength{\parskip}{1ex}
      \textbf{Parameters}
      \vspace{-1ex}

      \begin{quote}
        \begin{Ventry}{xxxxxxx}

          \item[pObject]

          pointer to object

        \end{Ventry}

      \end{quote}

      \textbf{Return Value}
    \vspace{-1ex}

      \begin{quote}
      x, y, w, h

      \end{quote}

\textbf{Attention:} API change from XForms - upstream was fl\_get\_folder\_area(pObject, x, y, 
w, h)



\textbf{Status:} Untested + NoDoc + NoDemo = NOT OK



    \end{boxedminipage}

    \label{xformslib:library:fl_replace_folder_bynumber}
    \index{xformslib \textit{(package)}!xformslib.library \textit{(module)}!xformslib.library.fl\_replace\_folder\_bynumber \textit{(function)}}

    \vspace{0.5ex}

\hspace{.8\funcindent}\begin{boxedminipage}{\funcwidth}

    \raggedright \textbf{fl\_replace\_folder\_bynumber}(\textit{pObject}, \textit{num}, \textit{pForm})

    \vspace{-1.5ex}

    \rule{\textwidth}{0.5\fboxrule}
\setlength{\parskip}{2ex}
\setlength{\parskip}{1ex}
      \textbf{Parameters}
      \vspace{-1ex}

      \begin{quote}
        \begin{Ventry}{xxxxxxx}

          \item[pObject]

          pointer to object

        \end{Ventry}

      \end{quote}

\textbf{Status:} Untested + NoDoc + NoDemo = NOT OK



    \end{boxedminipage}

    \label{xformslib:library:fl_set_tabfolder_autofit}
    \index{xformslib \textit{(package)}!xformslib.library \textit{(module)}!xformslib.library.fl\_set\_tabfolder\_autofit \textit{(function)}}

    \vspace{0.5ex}

\hspace{.8\funcindent}\begin{boxedminipage}{\funcwidth}

    \raggedright \textbf{fl\_set\_tabfolder\_autofit}(\textit{pObject}, \textit{num})

    \vspace{-1.5ex}

    \rule{\textwidth}{0.5\fboxrule}
\setlength{\parskip}{2ex}
\setlength{\parskip}{1ex}
      \textbf{Parameters}
      \vspace{-1ex}

      \begin{quote}
        \begin{Ventry}{xxxxxxx}

          \item[pObject]

          pointer to object

        \end{Ventry}

      \end{quote}

      \textbf{Return Value}
    \vspace{-1ex}

      \begin{quote}
      num

      \end{quote}

\textbf{Status:} Untested + NoDoc + NoDemo = NOT OK



    \end{boxedminipage}

    \label{xformslib:library:fl_set_default_tabfolder_corner}
    \index{xformslib \textit{(package)}!xformslib.library \textit{(module)}!xformslib.library.fl\_set\_default\_tabfolder\_corner \textit{(function)}}

    \vspace{0.5ex}

\hspace{.8\funcindent}\begin{boxedminipage}{\funcwidth}

    \raggedright \textbf{fl\_set\_default\_tabfolder\_corner}(\textit{npixels})

    \vspace{-1.5ex}

    \rule{\textwidth}{0.5\fboxrule}
\setlength{\parskip}{2ex}
    Adjusts the corner pixels, changing appearance of the tabs.

\setlength{\parskip}{1ex}
      \textbf{Parameters}
      \vspace{-1ex}

      \begin{quote}
        \begin{Ventry}{xxxxxxx}

          \item[npixels]

          number of corner pixels (default 3)

        \end{Ventry}

      \end{quote}

      \textbf{Return Value}
    \vspace{-1ex}

      \begin{quote}
      old pixels num

      \end{quote}

\textbf{Status:} Untested + NoDoc + NoDemo = NOT OK



    \end{boxedminipage}

    \label{xformslib:library:fl_set_tabfolder_offset}
    \index{xformslib \textit{(package)}!xformslib.library \textit{(module)}!xformslib.library.fl\_set\_tabfolder\_offset \textit{(function)}}

    \vspace{0.5ex}

\hspace{.8\funcindent}\begin{boxedminipage}{\funcwidth}

    \raggedright \textbf{fl\_set\_tabfolder\_offset}(\textit{pObject}, \textit{offset})

    \vspace{-1.5ex}

    \rule{\textwidth}{0.5\fboxrule}
\setlength{\parskip}{2ex}
\setlength{\parskip}{1ex}
      \textbf{Parameters}
      \vspace{-1ex}

      \begin{quote}
        \begin{Ventry}{xxxxxxx}

          \item[pObject]

          pointer to tabfolder object

        \end{Ventry}

      \end{quote}

      \textbf{Return Value}
    \vspace{-1ex}

      \begin{quote}
      num

      \end{quote}

\textbf{Status:} Untested + NoDoc + NoDemo = NOT OK



    \end{boxedminipage}

    \label{xformslib:library:fl_add_text}
    \index{xformslib \textit{(package)}!xformslib.library \textit{(module)}!xformslib.library.fl\_add\_text \textit{(function)}}

    \vspace{0.5ex}

\hspace{.8\funcindent}\begin{boxedminipage}{\funcwidth}

    \raggedright \textbf{fl\_add\_text}(\textit{texttype}, \textit{x}, \textit{y}, \textit{w}, \textit{h}, \textit{label})

    \vspace{-1.5ex}

    \rule{\textwidth}{0.5\fboxrule}
\setlength{\parskip}{2ex}
    Adds a text object.

\setlength{\parskip}{1ex}
      \textbf{Parameters}
      \vspace{-1ex}

      \begin{quote}
        \begin{Ventry}{xxxxxxxx}

          \item[texttype]

          type of text to be added

          \item[texttype]

          [num./int] xfc.FL\_NORMAL\_TEXT

          \item[x]

          horizontal position (upper-left corner)

          \item[x]

          vertical position (upper-left corner)

          \item[w]

          width in coord units

          \item[h]

          height in coord units

          \item[label]

          text label of text

        \end{Ventry}

      \end{quote}

      \textbf{Return Value}
    \vspace{-1ex}

      \begin{quote}
      pObject

      \end{quote}

\textbf{Status:} Tested + NoDoc + Demo = OK



    \end{boxedminipage}

    \label{xformslib:library:fl_get_thumbwheel_value}
    \index{xformslib \textit{(package)}!xformslib.library \textit{(module)}!xformslib.library.fl\_get\_thumbwheel\_value \textit{(function)}}

    \vspace{0.5ex}

\hspace{.8\funcindent}\begin{boxedminipage}{\funcwidth}

    \raggedright \textbf{fl\_get\_thumbwheel\_value}(\textit{pObject})

    \vspace{-1.5ex}

    \rule{\textwidth}{0.5\fboxrule}
\setlength{\parskip}{2ex}
\setlength{\parskip}{1ex}
      \textbf{Parameters}
      \vspace{-1ex}

      \begin{quote}
        \begin{Ventry}{xxxxxxx}

          \item[pObject]

          pointer to object

        \end{Ventry}

      \end{quote}

      \textbf{Return Value}
    \vspace{-1ex}

      \begin{quote}
      num

      \end{quote}

\textbf{Status:} Tested + NoDoc + Demo = OK



    \end{boxedminipage}

    \label{xformslib:library:fl_set_thumbwheel_value}
    \index{xformslib \textit{(package)}!xformslib.library \textit{(module)}!xformslib.library.fl\_set\_thumbwheel\_value \textit{(function)}}

    \vspace{0.5ex}

\hspace{.8\funcindent}\begin{boxedminipage}{\funcwidth}

    \raggedright \textbf{fl\_set\_thumbwheel\_value}(\textit{pObject}, \textit{value})

    \vspace{-1.5ex}

    \rule{\textwidth}{0.5\fboxrule}
\setlength{\parskip}{2ex}
\setlength{\parskip}{1ex}
      \textbf{Parameters}
      \vspace{-1ex}

      \begin{quote}
        \begin{Ventry}{xxxxxxx}

          \item[pObject]

          pointer to object

        \end{Ventry}

      \end{quote}

    \end{boxedminipage}

    \label{xformslib:library:fl_get_thumbwheel_step}
    \index{xformslib \textit{(package)}!xformslib.library \textit{(module)}!xformslib.library.fl\_get\_thumbwheel\_step \textit{(function)}}

    \vspace{0.5ex}

\hspace{.8\funcindent}\begin{boxedminipage}{\funcwidth}

    \raggedright \textbf{fl\_get\_thumbwheel\_step}(\textit{pObject})

    \vspace{-1.5ex}

    \rule{\textwidth}{0.5\fboxrule}
\setlength{\parskip}{2ex}
\setlength{\parskip}{1ex}
      \textbf{Parameters}
      \vspace{-1ex}

      \begin{quote}
        \begin{Ventry}{xxxxxxx}

          \item[pObject]

          pointer to object

        \end{Ventry}

      \end{quote}

      \textbf{Return Value}
    \vspace{-1ex}

      \begin{quote}
      num

      \end{quote}

\textbf{Status:} Untested + NoDoc + NoDemo = NOT OK



    \end{boxedminipage}

    \label{xformslib:library:fl_set_thumbwheel_step}
    \index{xformslib \textit{(package)}!xformslib.library \textit{(module)}!xformslib.library.fl\_set\_thumbwheel\_step \textit{(function)}}

    \vspace{0.5ex}

\hspace{.8\funcindent}\begin{boxedminipage}{\funcwidth}

    \raggedright \textbf{fl\_set\_thumbwheel\_step}(\textit{pObject}, \textit{step})

    \vspace{-1.5ex}

    \rule{\textwidth}{0.5\fboxrule}
\setlength{\parskip}{2ex}
\setlength{\parskip}{1ex}
      \textbf{Parameters}
      \vspace{-1ex}

      \begin{quote}
        \begin{Ventry}{xxxxxxx}

          \item[pObject]

          pointer to object

        \end{Ventry}

      \end{quote}

      \textbf{Return Value}
    \vspace{-1ex}

      \begin{quote}
      num

      \end{quote}

\textbf{Status:} Tested + NoDoc + Demo = OK



    \end{boxedminipage}

    \label{xformslib:library:fl_set_thumbwheel_return}
    \index{xformslib \textit{(package)}!xformslib.library \textit{(module)}!xformslib.library.fl\_set\_thumbwheel\_return \textit{(function)}}

    \vspace{0.5ex}

\hspace{.8\funcindent}\begin{boxedminipage}{\funcwidth}

    \raggedright \textbf{fl\_set\_thumbwheel\_return}(\textit{pObject}, \textit{when})

    \vspace{-1.5ex}

    \rule{\textwidth}{0.5\fboxrule}
\setlength{\parskip}{2ex}
\setlength{\parskip}{1ex}
      \textbf{Parameters}
      \vspace{-1ex}

      \begin{quote}
        \begin{Ventry}{xxxxxxx}

          \item[pObject]

          pointer to object

          \item[when]

          return type (when it returns)

            {\it (type=[num./int] xfc.FL\_RETURN\_NONE, xfc.FL\_RETURN\_CHANGED, 
xfc.FL\_RETURN\_END, xfc.FL\_RETURN\_END\_CHANGED, 
xfc.FL\_RETURN\_SELECTION, xfc.FL\_RETURN\_DESELECTION, 
xfc.FL\_RETURN\_TRIGGERED, xfc.FL\_RETURN\_ALWAYS)}

        \end{Ventry}

      \end{quote}

      \textbf{Return Value}
    \vspace{-1ex}

      \begin{quote}
      num

      \end{quote}

\textbf{Status:} Tested + NoDoc + Demo = OK



    \end{boxedminipage}

    \label{xformslib:library:fl_set_thumbwheel_crossover}
    \index{xformslib \textit{(package)}!xformslib.library \textit{(module)}!xformslib.library.fl\_set\_thumbwheel\_crossover \textit{(function)}}

    \vspace{0.5ex}

\hspace{.8\funcindent}\begin{boxedminipage}{\funcwidth}

    \raggedright \textbf{fl\_set\_thumbwheel\_crossover}(\textit{pObject}, \textit{flag})

    \vspace{-1.5ex}

    \rule{\textwidth}{0.5\fboxrule}
\setlength{\parskip}{2ex}
\setlength{\parskip}{1ex}
      \textbf{Parameters}
      \vspace{-1ex}

      \begin{quote}
        \begin{Ventry}{xxxxxxx}

          \item[pObject]

          pointer to object

        \end{Ventry}

      \end{quote}

      \textbf{Return Value}
    \vspace{-1ex}

      \begin{quote}
      num

      \end{quote}

\textbf{Status:} Untested + NoDoc + NoDemo = NOT OK



    \end{boxedminipage}

    \label{xformslib:library:fl_set_thumbwheel_bounds}
    \index{xformslib \textit{(package)}!xformslib.library \textit{(module)}!xformslib.library.fl\_set\_thumbwheel\_bounds \textit{(function)}}

    \vspace{0.5ex}

\hspace{.8\funcindent}\begin{boxedminipage}{\funcwidth}

    \raggedright \textbf{fl\_set\_thumbwheel\_bounds}(\textit{pObject}, \textit{minbound}, \textit{maxbound})

    \vspace{-1.5ex}

    \rule{\textwidth}{0.5\fboxrule}
\setlength{\parskip}{2ex}
\setlength{\parskip}{1ex}
      \textbf{Parameters}
      \vspace{-1ex}

      \begin{quote}
        \begin{Ventry}{xxxxxxx}

          \item[pObject]

          pointer to object

        \end{Ventry}

      \end{quote}

\textbf{Status:} Untested + NoDoc + NoDemo = NOT OK



    \end{boxedminipage}

    \label{xformslib:library:fl_get_thumbwheel_bounds}
    \index{xformslib \textit{(package)}!xformslib.library \textit{(module)}!xformslib.library.fl\_get\_thumbwheel\_bounds \textit{(function)}}

    \vspace{0.5ex}

\hspace{.8\funcindent}\begin{boxedminipage}{\funcwidth}

    \raggedright \textbf{fl\_get\_thumbwheel\_bounds}(\textit{pObject})

    \vspace{-1.5ex}

    \rule{\textwidth}{0.5\fboxrule}
\setlength{\parskip}{2ex}
\setlength{\parskip}{1ex}
      \textbf{Parameters}
      \vspace{-1ex}

      \begin{quote}
        \begin{Ventry}{xxxxxxx}

          \item[pObject]

          pointer to thumbwheel object

        \end{Ventry}

      \end{quote}

      \textbf{Return Value}
    \vspace{-1ex}

      \begin{quote}
      minbound, maxbound

      \end{quote}

\textbf{Attention:} API change from XForms - upstream was fl\_get\_thumbwheel\_bounds(pObject, 
minbound, maxbound)



\textbf{Status:} Untested + NoDoc + NoDemo = NOT OK



    \end{boxedminipage}

    \label{xformslib:library:fl_add_thumbwheel}
    \index{xformslib \textit{(package)}!xformslib.library \textit{(module)}!xformslib.library.fl\_add\_thumbwheel \textit{(function)}}

    \vspace{0.5ex}

\hspace{.8\funcindent}\begin{boxedminipage}{\funcwidth}

    \raggedright \textbf{fl\_add\_thumbwheel}(\textit{wheeltype}, \textit{x}, \textit{y}, \textit{w}, \textit{h}, \textit{label})

    \vspace{-1.5ex}

    \rule{\textwidth}{0.5\fboxrule}
\setlength{\parskip}{2ex}
    Adds a thumbwheel object.

\setlength{\parskip}{1ex}
      \textbf{Parameters}
      \vspace{-1ex}

      \begin{quote}
        \begin{Ventry}{xxxxxxxxx}

          \item[wheeltype]

          type of thumbwheel to be added

          \item[x]

          horizontal position (upper-left corner)

          \item[x]

          vertical position (upper-left corner)

          \item[w]

          width in coord units

          \item[h]

          height in coord units

          \item[label]

          text label of thumbwheel

        \end{Ventry}

      \end{quote}

      \textbf{Return Value}
    \vspace{-1ex}

      \begin{quote}
      pObject

      \end{quote}

\textbf{Status:} Tested + NoDoc + Demo = OK



    \end{boxedminipage}

    \label{xformslib:library:fl_add_timer}
    \index{xformslib \textit{(package)}!xformslib.library \textit{(module)}!xformslib.library.fl\_add\_timer \textit{(function)}}

    \vspace{0.5ex}

\hspace{.8\funcindent}\begin{boxedminipage}{\funcwidth}

    \raggedright \textbf{fl\_add\_timer}(\textit{timertype}, \textit{x}, \textit{y}, \textit{w}, \textit{h}, \textit{label})

    \vspace{-1.5ex}

    \rule{\textwidth}{0.5\fboxrule}
\setlength{\parskip}{2ex}
    Adds a timer object.

\setlength{\parskip}{1ex}
      \textbf{Parameters}
      \vspace{-1ex}

      \begin{quote}
        \begin{Ventry}{xxxxxxxxx}

          \item[timertype]

          type of timer to be added

          \item[x]

          horizontal position (upper-left corner)

          \item[x]

          vertical position (upper-left corner)

          \item[w]

          width in coord units

          \item[h]

          height in coord units

          \item[label]

          text label of timer

        \end{Ventry}

      \end{quote}

      \textbf{Return Value}
    \vspace{-1ex}

      \begin{quote}
      pObject

      \end{quote}

\textbf{Status:} Tested + NoDoc + Demo = OK



    \end{boxedminipage}

    \label{xformslib:library:fl_set_timer}
    \index{xformslib \textit{(package)}!xformslib.library \textit{(module)}!xformslib.library.fl\_set\_timer \textit{(function)}}

    \vspace{0.5ex}

\hspace{.8\funcindent}\begin{boxedminipage}{\funcwidth}

    \raggedright \textbf{fl\_set\_timer}(\textit{pObject}, \textit{total})

    \vspace{-1.5ex}

    \rule{\textwidth}{0.5\fboxrule}
\setlength{\parskip}{2ex}
\setlength{\parskip}{1ex}
      \textbf{Parameters}
      \vspace{-1ex}

      \begin{quote}
        \begin{Ventry}{xxxxxxx}

          \item[pObject]

          pointer to object

        \end{Ventry}

      \end{quote}

\textbf{Status:} Tested + NoDoc + Demo = OK



    \end{boxedminipage}

    \label{xformslib:library:fl_get_timer}
    \index{xformslib \textit{(package)}!xformslib.library \textit{(module)}!xformslib.library.fl\_get\_timer \textit{(function)}}

    \vspace{0.5ex}

\hspace{.8\funcindent}\begin{boxedminipage}{\funcwidth}

    \raggedright \textbf{fl\_get\_timer}(\textit{pObject})

    \vspace{-1.5ex}

    \rule{\textwidth}{0.5\fboxrule}
\setlength{\parskip}{2ex}
\setlength{\parskip}{1ex}
      \textbf{Parameters}
      \vspace{-1ex}

      \begin{quote}
        \begin{Ventry}{xxxxxxx}

          \item[pObject]

          pointer to object

        \end{Ventry}

      \end{quote}

      \textbf{Return Value}
    \vspace{-1ex}

      \begin{quote}
      num

      \end{quote}

\textbf{Status:} Untested + NoDoc + NoDemo = NOT OK



    \end{boxedminipage}

    \label{xformslib:library:fl_set_timer_countup}
    \index{xformslib \textit{(package)}!xformslib.library \textit{(module)}!xformslib.library.fl\_set\_timer\_countup \textit{(function)}}

    \vspace{0.5ex}

\hspace{.8\funcindent}\begin{boxedminipage}{\funcwidth}

    \raggedright \textbf{fl\_set\_timer\_countup}(\textit{pObject}, \textit{yes})

    \vspace{-1.5ex}

    \rule{\textwidth}{0.5\fboxrule}
\setlength{\parskip}{2ex}
\setlength{\parskip}{1ex}
      \textbf{Parameters}
      \vspace{-1ex}

      \begin{quote}
        \begin{Ventry}{xxxxxxx}

          \item[pObject]

          pointer to object

        \end{Ventry}

      \end{quote}

\textbf{Status:} Tested + NoDoc + Demo = OK



    \end{boxedminipage}

    \label{xformslib:library:fl_set_timer_filter}
    \index{xformslib \textit{(package)}!xformslib.library \textit{(module)}!xformslib.library.fl\_set\_timer\_filter \textit{(function)}}

    \vspace{0.5ex}

\hspace{.8\funcindent}\begin{boxedminipage}{\funcwidth}

    \raggedright \textbf{fl\_set\_timer\_filter}(\textit{pObject}, \textit{py\_TimerFilter})

    \vspace{-1.5ex}

    \rule{\textwidth}{0.5\fboxrule}
\setlength{\parskip}{2ex}
\setlength{\parskip}{1ex}
      \textbf{Parameters}
      \vspace{-1ex}

      \begin{quote}
        \begin{Ventry}{xxxxxxxxxxxxxx}

          \item[pObject]

          pointer to object

          \item[py\_TimerFilter]

          python function, fn(pObject, valfloat) -{\textgreater} string

        \end{Ventry}

      \end{quote}

      \textbf{Return Value}
    \vspace{-1ex}

      \begin{quote}
      timer\_filter func

      \end{quote}

\textbf{Status:} Untested + NoDoc + NoDemo = NOT OK



    \end{boxedminipage}

    \label{xformslib:library:fl_suspend_timer}
    \index{xformslib \textit{(package)}!xformslib.library \textit{(module)}!xformslib.library.fl\_suspend\_timer \textit{(function)}}

    \vspace{0.5ex}

\hspace{.8\funcindent}\begin{boxedminipage}{\funcwidth}

    \raggedright \textbf{fl\_suspend\_timer}(\textit{pObject})

    \vspace{-1.5ex}

    \rule{\textwidth}{0.5\fboxrule}
\setlength{\parskip}{2ex}
\setlength{\parskip}{1ex}
      \textbf{Parameters}
      \vspace{-1ex}

      \begin{quote}
        \begin{Ventry}{xxxxxxx}

          \item[pObject]

          pointer to object

        \end{Ventry}

      \end{quote}

\textbf{Status:} Tested + NoDoc + Demo = OK



    \end{boxedminipage}

    \label{xformslib:library:fl_resume_timer}
    \index{xformslib \textit{(package)}!xformslib.library \textit{(module)}!xformslib.library.fl\_resume\_timer \textit{(function)}}

    \vspace{0.5ex}

\hspace{.8\funcindent}\begin{boxedminipage}{\funcwidth}

    \raggedright \textbf{fl\_resume\_timer}(\textit{pObject})

    \vspace{-1.5ex}

    \rule{\textwidth}{0.5\fboxrule}
\setlength{\parskip}{2ex}
    Resume timer previously paused.

\setlength{\parskip}{1ex}
      \textbf{Parameters}
      \vspace{-1ex}

      \begin{quote}
        \begin{Ventry}{xxxxxxx}

          \item[pObject]

          pointer to timer object

        \end{Ventry}

      \end{quote}

\textbf{Status:} Tested + NoDoc + Demo = OK



    \end{boxedminipage}

    \label{xformslib:library:fl_add_xyplot}
    \index{xformslib \textit{(package)}!xformslib.library \textit{(module)}!xformslib.library.fl\_add\_xyplot \textit{(function)}}

    \vspace{0.5ex}

\hspace{.8\funcindent}\begin{boxedminipage}{\funcwidth}

    \raggedright \textbf{fl\_add\_xyplot}(\textit{plottype}, \textit{x}, \textit{y}, \textit{w}, \textit{h}, \textit{label})

    \vspace{-1.5ex}

    \rule{\textwidth}{0.5\fboxrule}
\setlength{\parskip}{2ex}
    Adds an xyplot object.

\setlength{\parskip}{1ex}
      \textbf{Parameters}
      \vspace{-1ex}

      \begin{quote}
        \begin{Ventry}{xxxxxxxx}

          \item[plottype]

          type of xyplot to be added

          \item[x]

          horizontal position (upper-left corner)

          \item[x]

          vertical position (upper-left corner)

          \item[w]

          width in coord units

          \item[h]

          height in coord units

          \item[label]

          text label of xyplot

        \end{Ventry}

      \end{quote}

      \textbf{Return Value}
    \vspace{-1ex}

      \begin{quote}
      pObject

      \end{quote}

\textbf{Status:} Untested + NoDoc + NoDemo = NOT OK



    \end{boxedminipage}

    \label{xformslib:library:fl_set_xyplot_data}
    \index{xformslib \textit{(package)}!xformslib.library \textit{(module)}!xformslib.library.fl\_set\_xyplot\_data \textit{(function)}}

    \vspace{0.5ex}

\hspace{.8\funcindent}\begin{boxedminipage}{\funcwidth}

    \raggedright \textbf{fl\_set\_xyplot\_data}(\textit{pObject}, \textit{xlist}, \textit{ylist}, \textit{n}, \textit{title}, \textit{xlabel}, \textit{ylabel})

    \vspace{-1.5ex}

    \rule{\textwidth}{0.5\fboxrule}
\setlength{\parskip}{2ex}
\setlength{\parskip}{1ex}
      \textbf{Parameters}
      \vspace{-1ex}

      \begin{quote}
        \begin{Ventry}{xxxxxxx}

          \item[pObject]

          pointer to object

        \end{Ventry}

      \end{quote}

\textbf{Status:} Untested + NoDoc + NoDemo = NOT OK



    \end{boxedminipage}

    \label{xformslib:library:fl_set_xyplot_data_double}
    \index{xformslib \textit{(package)}!xformslib.library \textit{(module)}!xformslib.library.fl\_set\_xyplot\_data\_double \textit{(function)}}

    \vspace{0.5ex}

\hspace{.8\funcindent}\begin{boxedminipage}{\funcwidth}

    \raggedright \textbf{fl\_set\_xyplot\_data\_double}(\textit{pObject}, \textit{x}, \textit{y}, \textit{n}, \textit{title}, \textit{xlabel}, \textit{ylabel})

    \vspace{-1.5ex}

    \rule{\textwidth}{0.5\fboxrule}
\setlength{\parskip}{2ex}
\setlength{\parskip}{1ex}
      \textbf{Parameters}
      \vspace{-1ex}

      \begin{quote}
        \begin{Ventry}{xxxxxxx}

          \item[pObject]

          pointer to object

        \end{Ventry}

      \end{quote}

\textbf{Status:} Untested + NoDoc + NoDemo = NOT OK



    \end{boxedminipage}

    \label{xformslib:library:fl_set_xyplot_file}
    \index{xformslib \textit{(package)}!xformslib.library \textit{(module)}!xformslib.library.fl\_set\_xyplot\_file \textit{(function)}}

    \vspace{0.5ex}

\hspace{.8\funcindent}\begin{boxedminipage}{\funcwidth}

    \raggedright \textbf{fl\_set\_xyplot\_file}(\textit{pObject}, \textit{fname}, \textit{title}, \textit{xl}, \textit{yl})

    \vspace{-1.5ex}

    \rule{\textwidth}{0.5\fboxrule}
\setlength{\parskip}{2ex}
\setlength{\parskip}{1ex}
      \textbf{Parameters}
      \vspace{-1ex}

      \begin{quote}
        \begin{Ventry}{xxxxxxx}

          \item[pObject]

          pointer to object

        \end{Ventry}

      \end{quote}

      \textbf{Return Value}
    \vspace{-1ex}

      \begin{quote}
      num

      \end{quote}

\textbf{Status:} Untested + NoDoc + NoDemo = NOT OK



    \end{boxedminipage}

    \label{xformslib:library:fl_insert_xyplot_data}
    \index{xformslib \textit{(package)}!xformslib.library \textit{(module)}!xformslib.library.fl\_insert\_xyplot\_data \textit{(function)}}

    \vspace{0.5ex}

\hspace{.8\funcindent}\begin{boxedminipage}{\funcwidth}

    \raggedright \textbf{fl\_insert\_xyplot\_data}(\textit{pObject}, \textit{idnum}, \textit{n}, \textit{valx}, \textit{valy})

    \vspace{-1.5ex}

    \rule{\textwidth}{0.5\fboxrule}
\setlength{\parskip}{2ex}
\setlength{\parskip}{1ex}
      \textbf{Parameters}
      \vspace{-1ex}

      \begin{quote}
        \begin{Ventry}{xxxxxxx}

          \item[pObject]

          pointer to object

        \end{Ventry}

      \end{quote}

\textbf{Status:} Untested + NoDoc + NoDemo = NOT OK



    \end{boxedminipage}

    \label{xformslib:library:fl_add_xyplot_text}
    \index{xformslib \textit{(package)}!xformslib.library \textit{(module)}!xformslib.library.fl\_add\_xyplot\_text \textit{(function)}}

    \vspace{0.5ex}

\hspace{.8\funcindent}\begin{boxedminipage}{\funcwidth}

    \raggedright \textbf{fl\_add\_xyplot\_text}(\textit{pObject}, \textit{valx}, \textit{valy}, \textit{text}, \textit{al}, \textit{colr})

    \vspace{-1.5ex}

    \rule{\textwidth}{0.5\fboxrule}
\setlength{\parskip}{2ex}
\setlength{\parskip}{1ex}
      \textbf{Parameters}
      \vspace{-1ex}

      \begin{quote}
        \begin{Ventry}{xxxxxxx}

          \item[pObject]

          pointer to object

        \end{Ventry}

      \end{quote}

\textbf{Status:} Untested + NoDoc + NoDemo = NOT OK



    \end{boxedminipage}

    \label{xformslib:library:fl_delete_xyplot_text}
    \index{xformslib \textit{(package)}!xformslib.library \textit{(module)}!xformslib.library.fl\_delete\_xyplot\_text \textit{(function)}}

    \vspace{0.5ex}

\hspace{.8\funcindent}\begin{boxedminipage}{\funcwidth}

    \raggedright \textbf{fl\_delete\_xyplot\_text}(\textit{pObject}, \textit{text})

    \vspace{-1.5ex}

    \rule{\textwidth}{0.5\fboxrule}
\setlength{\parskip}{2ex}
\setlength{\parskip}{1ex}
      \textbf{Parameters}
      \vspace{-1ex}

      \begin{quote}
        \begin{Ventry}{xxxxxxx}

          \item[pObject]

          pointer to object

        \end{Ventry}

      \end{quote}

\textbf{Status:} Untested + NoDoc + NoDemo = NOT OK



    \end{boxedminipage}

    \label{xformslib:library:fl_set_xyplot_maxoverlays}
    \index{xformslib \textit{(package)}!xformslib.library \textit{(module)}!xformslib.library.fl\_set\_xyplot\_maxoverlays \textit{(function)}}

    \vspace{0.5ex}

\hspace{.8\funcindent}\begin{boxedminipage}{\funcwidth}

    \raggedright \textbf{fl\_set\_xyplot\_maxoverlays}(\textit{pObject}, \textit{maxover})

    \vspace{-1.5ex}

    \rule{\textwidth}{0.5\fboxrule}
\setlength{\parskip}{2ex}
\setlength{\parskip}{1ex}
      \textbf{Parameters}
      \vspace{-1ex}

      \begin{quote}
        \begin{Ventry}{xxxxxxx}

          \item[pObject]

          pointer to object

        \end{Ventry}

      \end{quote}

      \textbf{Return Value}
    \vspace{-1ex}

      \begin{quote}
      num

      \end{quote}

\textbf{Status:} Untested + NoDoc + NoDemo = NOT OK



    \end{boxedminipage}

    \label{xformslib:library:fl_add_xyplot_overlay}
    \index{xformslib \textit{(package)}!xformslib.library \textit{(module)}!xformslib.library.fl\_add\_xyplot\_overlay \textit{(function)}}

    \vspace{0.5ex}

\hspace{.8\funcindent}\begin{boxedminipage}{\funcwidth}

    \raggedright \textbf{fl\_add\_xyplot\_overlay}(\textit{pObject}, \textit{idnum}, \textit{x}, \textit{y}, \textit{n}, \textit{colr})

    \vspace{-1.5ex}

    \rule{\textwidth}{0.5\fboxrule}
\setlength{\parskip}{2ex}
\setlength{\parskip}{1ex}
      \textbf{Parameters}
      \vspace{-1ex}

      \begin{quote}
        \begin{Ventry}{xxxxxxx}

          \item[pObject]

          pointer to object

        \end{Ventry}

      \end{quote}

\textbf{Status:} Untested + NoDoc + NoDemo = NOT OK



    \end{boxedminipage}

    \label{xformslib:library:fl_add_xyplot_overlay_file}
    \index{xformslib \textit{(package)}!xformslib.library \textit{(module)}!xformslib.library.fl\_add\_xyplot\_overlay\_file \textit{(function)}}

    \vspace{0.5ex}

\hspace{.8\funcindent}\begin{boxedminipage}{\funcwidth}

    \raggedright \textbf{fl\_add\_xyplot\_overlay\_file}(\textit{pObject}, \textit{idnum}, \textit{fname}, \textit{colr})

    \vspace{-1.5ex}

    \rule{\textwidth}{0.5\fboxrule}
\setlength{\parskip}{2ex}
\setlength{\parskip}{1ex}
      \textbf{Parameters}
      \vspace{-1ex}

      \begin{quote}
        \begin{Ventry}{xxxxxxx}

          \item[pObject]

          pointer to object

        \end{Ventry}

      \end{quote}

      \textbf{Return Value}
    \vspace{-1ex}

      \begin{quote}
      num

      \end{quote}

\textbf{Status:} Untested + NoDoc + NoDemo = NOT OK



    \end{boxedminipage}

    \label{xformslib:library:fl_set_xyplot_return}
    \index{xformslib \textit{(package)}!xformslib.library \textit{(module)}!xformslib.library.fl\_set\_xyplot\_return \textit{(function)}}

    \vspace{0.5ex}

\hspace{.8\funcindent}\begin{boxedminipage}{\funcwidth}

    \raggedright \textbf{fl\_set\_xyplot\_return}(\textit{pObject}, \textit{when})

    \vspace{-1.5ex}

    \rule{\textwidth}{0.5\fboxrule}
\setlength{\parskip}{2ex}
\setlength{\parskip}{1ex}
      \textbf{Parameters}
      \vspace{-1ex}

      \begin{quote}
        \begin{Ventry}{xxxxxxx}

          \item[pObject]

          pointer to object

          \item[when]

          return type

        \end{Ventry}

      \end{quote}

\textbf{Status:} Untested + NoDoc + NoDemo = NOT OK



    \end{boxedminipage}

    \label{xformslib:library:fl_set_xyplot_xtics}
    \index{xformslib \textit{(package)}!xformslib.library \textit{(module)}!xformslib.library.fl\_set\_xyplot\_xtics \textit{(function)}}

    \vspace{0.5ex}

\hspace{.8\funcindent}\begin{boxedminipage}{\funcwidth}

    \raggedright \textbf{fl\_set\_xyplot\_xtics}(\textit{pObject}, \textit{major}, \textit{minor})

    \vspace{-1.5ex}

    \rule{\textwidth}{0.5\fboxrule}
\setlength{\parskip}{2ex}
\setlength{\parskip}{1ex}
      \textbf{Parameters}
      \vspace{-1ex}

      \begin{quote}
        \begin{Ventry}{xxxxxxx}

          \item[pObject]

          pointer to object

        \end{Ventry}

      \end{quote}

\textbf{Status:} Untested + NoDoc + NoDemo = NOT OK



    \end{boxedminipage}

    \label{xformslib:library:fl_set_xyplot_ytics}
    \index{xformslib \textit{(package)}!xformslib.library \textit{(module)}!xformslib.library.fl\_set\_xyplot\_ytics \textit{(function)}}

    \vspace{0.5ex}

\hspace{.8\funcindent}\begin{boxedminipage}{\funcwidth}

    \raggedright \textbf{fl\_set\_xyplot\_ytics}(\textit{pObject}, \textit{major}, \textit{minor})

    \vspace{-1.5ex}

    \rule{\textwidth}{0.5\fboxrule}
\setlength{\parskip}{2ex}
\setlength{\parskip}{1ex}
      \textbf{Parameters}
      \vspace{-1ex}

      \begin{quote}
        \begin{Ventry}{xxxxxxx}

          \item[pObject]

          pointer to object

        \end{Ventry}

      \end{quote}

\textbf{Status:} Untested + NoDoc + NoDemo = NOT OK



    \end{boxedminipage}

    \label{xformslib:library:fl_set_xyplot_xbounds}
    \index{xformslib \textit{(package)}!xformslib.library \textit{(module)}!xformslib.library.fl\_set\_xyplot\_xbounds \textit{(function)}}

    \vspace{0.5ex}

\hspace{.8\funcindent}\begin{boxedminipage}{\funcwidth}

    \raggedright \textbf{fl\_set\_xyplot\_xbounds}(\textit{pObject}, \textit{minbound}, \textit{maxbound})

    \vspace{-1.5ex}

    \rule{\textwidth}{0.5\fboxrule}
\setlength{\parskip}{2ex}
\setlength{\parskip}{1ex}
      \textbf{Parameters}
      \vspace{-1ex}

      \begin{quote}
        \begin{Ventry}{xxxxxxx}

          \item[pObject]

          pointer to object

        \end{Ventry}

      \end{quote}

\textbf{Status:} Untested + NoDoc + NoDemo = NOT OK



    \end{boxedminipage}

    \label{xformslib:library:fl_set_xyplot_ybounds}
    \index{xformslib \textit{(package)}!xformslib.library \textit{(module)}!xformslib.library.fl\_set\_xyplot\_ybounds \textit{(function)}}

    \vspace{0.5ex}

\hspace{.8\funcindent}\begin{boxedminipage}{\funcwidth}

    \raggedright \textbf{fl\_set\_xyplot\_ybounds}(\textit{pObject}, \textit{minbound}, \textit{maxbound})

    \vspace{-1.5ex}

    \rule{\textwidth}{0.5\fboxrule}
\setlength{\parskip}{2ex}
\setlength{\parskip}{1ex}
      \textbf{Parameters}
      \vspace{-1ex}

      \begin{quote}
        \begin{Ventry}{xxxxxxx}

          \item[pObject]

          pointer to object

        \end{Ventry}

      \end{quote}

\textbf{Status:} Untested + NoDoc + NoDemo = NOT OK



    \end{boxedminipage}

    \label{xformslib:library:fl_get_xyplot_xbounds}
    \index{xformslib \textit{(package)}!xformslib.library \textit{(module)}!xformslib.library.fl\_get\_xyplot\_xbounds \textit{(function)}}

    \vspace{0.5ex}

\hspace{.8\funcindent}\begin{boxedminipage}{\funcwidth}

    \raggedright \textbf{fl\_get\_xyplot\_xbounds}(\textit{pObject})

    \vspace{-1.5ex}

    \rule{\textwidth}{0.5\fboxrule}
\setlength{\parskip}{2ex}
\setlength{\parskip}{1ex}
      \textbf{Parameters}
      \vspace{-1ex}

      \begin{quote}
        \begin{Ventry}{xxxxxxx}

          \item[pObject]

          pointer to object

        \end{Ventry}

      \end{quote}

      \textbf{Return Value}
    \vspace{-1ex}

      \begin{quote}
      minbound, maxbound

      \end{quote}

\textbf{Attention:} API change from XForms - upstream was fl\_get\_xyplot\_xbounds(pObject, 
minbound, maxbound)



\textbf{Status:} Untested + NoDoc + NoDemo = NOT OK



    \end{boxedminipage}

    \label{xformslib:library:fl_get_xyplot_ybounds}
    \index{xformslib \textit{(package)}!xformslib.library \textit{(module)}!xformslib.library.fl\_get\_xyplot\_ybounds \textit{(function)}}

    \vspace{0.5ex}

\hspace{.8\funcindent}\begin{boxedminipage}{\funcwidth}

    \raggedright \textbf{fl\_get\_xyplot\_ybounds}(\textit{pObject})

    \vspace{-1.5ex}

    \rule{\textwidth}{0.5\fboxrule}
\setlength{\parskip}{2ex}
\setlength{\parskip}{1ex}
      \textbf{Parameters}
      \vspace{-1ex}

      \begin{quote}
        \begin{Ventry}{xxxxxxx}

          \item[pObject]

          pointer to object

        \end{Ventry}

      \end{quote}

      \textbf{Return Value}
    \vspace{-1ex}

      \begin{quote}
      minbound, maxbound

      \end{quote}

\textbf{Attention:} API change from XForms - upstream was fl\_get\_xyplot\_ybounds(pObject, 
minbound, maxbound)



\textbf{Status:} Untested + NoDoc + NoDemo = NOT OK



    \end{boxedminipage}

    \label{xformslib:library:fl_get_xyplot}
    \index{xformslib \textit{(package)}!xformslib.library \textit{(module)}!xformslib.library.fl\_get\_xyplot \textit{(function)}}

    \vspace{0.5ex}

\hspace{.8\funcindent}\begin{boxedminipage}{\funcwidth}

    \raggedright \textbf{fl\_get\_xyplot}(\textit{pObject})

    \vspace{-1.5ex}

    \rule{\textwidth}{0.5\fboxrule}
\setlength{\parskip}{2ex}
\setlength{\parskip}{1ex}
      \textbf{Parameters}
      \vspace{-1ex}

      \begin{quote}
        \begin{Ventry}{xxxxxxx}

          \item[pObject]

          pointer to object

        \end{Ventry}

      \end{quote}

      \textbf{Return Value}
    \vspace{-1ex}

      \begin{quote}
      x, y, i

      \end{quote}

\textbf{Attention:} API change from XForms - upstream was fl\_get\_xyplot(pObject, x, y, i)



\textbf{Status:} Untested + NoDoc + NoDemo = NOT OK



    \end{boxedminipage}

    \label{xformslib:library:fl_get_xyplot_data}
    \index{xformslib \textit{(package)}!xformslib.library \textit{(module)}!xformslib.library.fl\_get\_xyplot\_data \textit{(function)}}

    \vspace{0.5ex}

\hspace{.8\funcindent}\begin{boxedminipage}{\funcwidth}

    \raggedright \textbf{fl\_get\_xyplot\_data}(\textit{pObject})

    \vspace{-1.5ex}

    \rule{\textwidth}{0.5\fboxrule}
\setlength{\parskip}{2ex}
\setlength{\parskip}{1ex}
      \textbf{Parameters}
      \vspace{-1ex}

      \begin{quote}
        \begin{Ventry}{xxxxxxx}

          \item[pObject]

          pointer to object

        \end{Ventry}

      \end{quote}

      \textbf{Return Value}
    \vspace{-1ex}

      \begin{quote}
      x, y, n

      \end{quote}

\textbf{Attention:} API change from XForms - upstream was fl\_get\_xyplot\_data(pObject, x, y, 
n)



\textbf{Status:} Untested + NoDoc + NoDemo = NOT OK



    \end{boxedminipage}

    \label{xformslib:library:fl_get_xyplot_data_pointer}
    \index{xformslib \textit{(package)}!xformslib.library \textit{(module)}!xformslib.library.fl\_get\_xyplot\_data\_pointer \textit{(function)}}

    \vspace{0.5ex}

\hspace{.8\funcindent}\begin{boxedminipage}{\funcwidth}

    \raggedright \textbf{fl\_get\_xyplot\_data\_pointer}(\textit{pObject}, \textit{idnum})

    \vspace{-1.5ex}

    \rule{\textwidth}{0.5\fboxrule}
\setlength{\parskip}{2ex}
\setlength{\parskip}{1ex}
      \textbf{Parameters}
      \vspace{-1ex}

      \begin{quote}
        \begin{Ventry}{xxxxxxx}

          \item[pObject]

          pointer to object

        \end{Ventry}

      \end{quote}

      \textbf{Return Value}
    \vspace{-1ex}

      \begin{quote}
      x, y, n

      \end{quote}

\textbf{Attention:} API change from XForms - upstream was 
fl\_get\_xyplot\_data\_pointer(pObject, idnum, x, y, n)



\textbf{Status:} Untested + NoDoc + NoDemo = NOT OK



    \end{boxedminipage}

    \label{xformslib:library:fl_get_xyplot_overlay_data}
    \index{xformslib \textit{(package)}!xformslib.library \textit{(module)}!xformslib.library.fl\_get\_xyplot\_overlay\_data \textit{(function)}}

    \vspace{0.5ex}

\hspace{.8\funcindent}\begin{boxedminipage}{\funcwidth}

    \raggedright \textbf{fl\_get\_xyplot\_overlay\_data}(\textit{pObject}, \textit{idnum})

    \vspace{-1.5ex}

    \rule{\textwidth}{0.5\fboxrule}
\setlength{\parskip}{2ex}
\setlength{\parskip}{1ex}
      \textbf{Parameters}
      \vspace{-1ex}

      \begin{quote}
        \begin{Ventry}{xxxxxxx}

          \item[pObject]

          pointer to object

        \end{Ventry}

      \end{quote}

      \textbf{Return Value}
    \vspace{-1ex}

      \begin{quote}
      x, y, n

      \end{quote}

\textbf{Attention:} API change from XForms - upstream was 
fl\_get\_xyplot\_overlay\_data(pObject, idnum, x, y, n)



\textbf{Status:} Untested + NoDoc + NoDemo = NOT OK



    \end{boxedminipage}

    \label{xformslib:library:fl_set_xyplot_overlay_type}
    \index{xformslib \textit{(package)}!xformslib.library \textit{(module)}!xformslib.library.fl\_set\_xyplot\_overlay\_type \textit{(function)}}

    \vspace{0.5ex}

\hspace{.8\funcindent}\begin{boxedminipage}{\funcwidth}

    \raggedright \textbf{fl\_set\_xyplot\_overlay\_type}(\textit{pObject}, \textit{idnum}, \textit{plottype})

    \vspace{-1.5ex}

    \rule{\textwidth}{0.5\fboxrule}
\setlength{\parskip}{2ex}
\setlength{\parskip}{1ex}
      \textbf{Parameters}
      \vspace{-1ex}

      \begin{quote}
        \begin{Ventry}{xxxxxxx}

          \item[pObject]

          pointer to object

        \end{Ventry}

      \end{quote}

\textbf{Status:} Untested + NoDoc + NoDemo = NOT OK



    \end{boxedminipage}

    \label{xformslib:library:fl_delete_xyplot_overlay}
    \index{xformslib \textit{(package)}!xformslib.library \textit{(module)}!xformslib.library.fl\_delete\_xyplot\_overlay \textit{(function)}}

    \vspace{0.5ex}

\hspace{.8\funcindent}\begin{boxedminipage}{\funcwidth}

    \raggedright \textbf{fl\_delete\_xyplot\_overlay}(\textit{pObject}, \textit{idnum})

    \vspace{-1.5ex}

    \rule{\textwidth}{0.5\fboxrule}
\setlength{\parskip}{2ex}
\setlength{\parskip}{1ex}
      \textbf{Parameters}
      \vspace{-1ex}

      \begin{quote}
        \begin{Ventry}{xxxxxxx}

          \item[pObject]

          pointer to object

        \end{Ventry}

      \end{quote}

\textbf{Status:} Untested + NoDoc + NoDemo = NOT OK



    \end{boxedminipage}

    \label{xformslib:library:fl_set_xyplot_interpolate}
    \index{xformslib \textit{(package)}!xformslib.library \textit{(module)}!xformslib.library.fl\_set\_xyplot\_interpolate \textit{(function)}}

    \vspace{0.5ex}

\hspace{.8\funcindent}\begin{boxedminipage}{\funcwidth}

    \raggedright \textbf{fl\_set\_xyplot\_interpolate}(\textit{pObject}, \textit{idnum}, \textit{deg}, \textit{grid})

    \vspace{-1.5ex}

    \rule{\textwidth}{0.5\fboxrule}
\setlength{\parskip}{2ex}
\setlength{\parskip}{1ex}
      \textbf{Parameters}
      \vspace{-1ex}

      \begin{quote}
        \begin{Ventry}{xxxxxxx}

          \item[pObject]

          pointer to object

        \end{Ventry}

      \end{quote}

\textbf{Status:} Untested + NoDoc + NoDemo = NOT OK



    \end{boxedminipage}

    \label{xformslib:library:fl_set_xyplot_inspect}
    \index{xformslib \textit{(package)}!xformslib.library \textit{(module)}!xformslib.library.fl\_set\_xyplot\_inspect \textit{(function)}}

    \vspace{0.5ex}

\hspace{.8\funcindent}\begin{boxedminipage}{\funcwidth}

    \raggedright \textbf{fl\_set\_xyplot\_inspect}(\textit{pObject}, \textit{yes})

    \vspace{-1.5ex}

    \rule{\textwidth}{0.5\fboxrule}
\setlength{\parskip}{2ex}
\setlength{\parskip}{1ex}
      \textbf{Parameters}
      \vspace{-1ex}

      \begin{quote}
        \begin{Ventry}{xxxxxxx}

          \item[pObject]

          pointer to object

        \end{Ventry}

      \end{quote}

\textbf{Status:} Untested + NoDoc + NoDemo = NOT OK



    \end{boxedminipage}

    \label{xformslib:library:fl_set_xyplot_symbolsize}
    \index{xformslib \textit{(package)}!xformslib.library \textit{(module)}!xformslib.library.fl\_set\_xyplot\_symbolsize \textit{(function)}}

    \vspace{0.5ex}

\hspace{.8\funcindent}\begin{boxedminipage}{\funcwidth}

    \raggedright \textbf{fl\_set\_xyplot\_symbolsize}(\textit{pObject}, \textit{n})

    \vspace{-1.5ex}

    \rule{\textwidth}{0.5\fboxrule}
\setlength{\parskip}{2ex}
\setlength{\parskip}{1ex}
      \textbf{Parameters}
      \vspace{-1ex}

      \begin{quote}
        \begin{Ventry}{xxxxxxx}

          \item[pObject]

          pointer to object

        \end{Ventry}

      \end{quote}

\textbf{Status:} Untested + NoDoc + NoDemo = NOT OK



    \end{boxedminipage}

    \label{xformslib:library:fl_replace_xyplot_point}
    \index{xformslib \textit{(package)}!xformslib.library \textit{(module)}!xformslib.library.fl\_replace\_xyplot\_point \textit{(function)}}

    \vspace{0.5ex}

\hspace{.8\funcindent}\begin{boxedminipage}{\funcwidth}

    \raggedright \textbf{fl\_replace\_xyplot\_point}(\textit{pObject}, \textit{i}, \textit{valx}, \textit{valy})

    \vspace{-1.5ex}

    \rule{\textwidth}{0.5\fboxrule}
\setlength{\parskip}{2ex}
\setlength{\parskip}{1ex}
      \textbf{Parameters}
      \vspace{-1ex}

      \begin{quote}
        \begin{Ventry}{xxxxxxx}

          \item[pObject]

          pointer to object

        \end{Ventry}

      \end{quote}

\textbf{Status:} Untested + NoDoc + NoDemo = NOT OK



    \end{boxedminipage}

    \label{xformslib:library:fl_replace_xyplot_point_in_overlay}
    \index{xformslib \textit{(package)}!xformslib.library \textit{(module)}!xformslib.library.fl\_replace\_xyplot\_point\_in\_overlay \textit{(function)}}

    \vspace{0.5ex}

\hspace{.8\funcindent}\begin{boxedminipage}{\funcwidth}

    \raggedright \textbf{fl\_replace\_xyplot\_point\_in\_overlay}(\textit{pObject}, \textit{i}, \textit{setID}, \textit{valx}, \textit{valy})

    \vspace{-1.5ex}

    \rule{\textwidth}{0.5\fboxrule}
\setlength{\parskip}{2ex}
\setlength{\parskip}{1ex}
      \textbf{Parameters}
      \vspace{-1ex}

      \begin{quote}
        \begin{Ventry}{xxxxxxx}

          \item[pObject]

          pointer to object

        \end{Ventry}

      \end{quote}

\textbf{Status:} Untested + NoDoc + NoDemo = NOT OK



    \end{boxedminipage}

    \label{xformslib:library:fl_get_xyplot_xmapping}
    \index{xformslib \textit{(package)}!xformslib.library \textit{(module)}!xformslib.library.fl\_get\_xyplot\_xmapping \textit{(function)}}

    \vspace{0.5ex}

\hspace{.8\funcindent}\begin{boxedminipage}{\funcwidth}

    \raggedright \textbf{fl\_get\_xyplot\_xmapping}(\textit{pObject})

    \vspace{-1.5ex}

    \rule{\textwidth}{0.5\fboxrule}
\setlength{\parskip}{2ex}
\setlength{\parskip}{1ex}
      \textbf{Parameters}
      \vspace{-1ex}

      \begin{quote}
        \begin{Ventry}{xxxxxxx}

          \item[pObject]

          pointer to object

        \end{Ventry}

      \end{quote}

      \textbf{Return Value}
    \vspace{-1ex}

      \begin{quote}
      a, b

      \end{quote}

\textbf{Attention:} API change from XForms - upstream was fl\_get\_xyplot\_xmapping(pObject, a,
b)



\textbf{Status:} Untested + NoDoc + NoDemo = NOT OK



    \end{boxedminipage}

    \label{xformslib:library:fl_get_xyplot_ymapping}
    \index{xformslib \textit{(package)}!xformslib.library \textit{(module)}!xformslib.library.fl\_get\_xyplot\_ymapping \textit{(function)}}

    \vspace{0.5ex}

\hspace{.8\funcindent}\begin{boxedminipage}{\funcwidth}

    \raggedright \textbf{fl\_get\_xyplot\_ymapping}(\textit{pObject})

    \vspace{-1.5ex}

    \rule{\textwidth}{0.5\fboxrule}
\setlength{\parskip}{2ex}
\setlength{\parskip}{1ex}
      \textbf{Parameters}
      \vspace{-1ex}

      \begin{quote}
        \begin{Ventry}{xxxxxxx}

          \item[pObject]

          pointer to object

        \end{Ventry}

      \end{quote}

      \textbf{Return Value}
    \vspace{-1ex}

      \begin{quote}
      a, b

      \end{quote}

\textbf{Attention:} API change from XForms - upstream was fl\_get\_xyplot\_ymapping(pObject, a,
b)



\textbf{Status:} Untested + NoDoc + NoDemo = NOT OK



    \end{boxedminipage}

    \label{xformslib:library:fl_set_xyplot_keys}
    \index{xformslib \textit{(package)}!xformslib.library \textit{(module)}!xformslib.library.fl\_set\_xyplot\_keys \textit{(function)}}

    \vspace{0.5ex}

\hspace{.8\funcindent}\begin{boxedminipage}{\funcwidth}

    \raggedright \textbf{fl\_set\_xyplot\_keys}(\textit{pObject}, \textit{keys}, \textit{valx}, \textit{valy}, \textit{align})

    \vspace{-1.5ex}

    \rule{\textwidth}{0.5\fboxrule}
\setlength{\parskip}{2ex}
\setlength{\parskip}{1ex}
      \textbf{Parameters}
      \vspace{-1ex}

      \begin{quote}
        \begin{Ventry}{xxxxxxx}

          \item[pObject]

          pointer to object

        \end{Ventry}

      \end{quote}

\textbf{Status:} Untested + NoDoc + NoDemo = NOT OK



    \end{boxedminipage}

    \label{xformslib:library:fl_set_xyplot_key}
    \index{xformslib \textit{(package)}!xformslib.library \textit{(module)}!xformslib.library.fl\_set\_xyplot\_key \textit{(function)}}

    \vspace{0.5ex}

\hspace{.8\funcindent}\begin{boxedminipage}{\funcwidth}

    \raggedright \textbf{fl\_set\_xyplot\_key}(\textit{pObject}, \textit{idnum}, \textit{keytxt})

    \vspace{-1.5ex}

    \rule{\textwidth}{0.5\fboxrule}
\setlength{\parskip}{2ex}
\setlength{\parskip}{1ex}
      \textbf{Parameters}
      \vspace{-1ex}

      \begin{quote}
        \begin{Ventry}{xxxxxxx}

          \item[pObject]

          pointer to object

        \end{Ventry}

      \end{quote}

\textbf{Status:} Untested + NoDoc + NoDemo = NOT OK



    \end{boxedminipage}

    \label{xformslib:library:fl_set_xyplot_key_position}
    \index{xformslib \textit{(package)}!xformslib.library \textit{(module)}!xformslib.library.fl\_set\_xyplot\_key\_position \textit{(function)}}

    \vspace{0.5ex}

\hspace{.8\funcindent}\begin{boxedminipage}{\funcwidth}

    \raggedright \textbf{fl\_set\_xyplot\_key\_position}(\textit{pObject}, \textit{valx}, \textit{valy}, \textit{align})

    \vspace{-1.5ex}

    \rule{\textwidth}{0.5\fboxrule}
\setlength{\parskip}{2ex}
\setlength{\parskip}{1ex}
      \textbf{Parameters}
      \vspace{-1ex}

      \begin{quote}
        \begin{Ventry}{xxxxxxx}

          \item[pObject]

          pointer to object

        \end{Ventry}

      \end{quote}

\textbf{Status:} Untested + NoDoc + NoDemo = NOT OK



    \end{boxedminipage}

    \label{xformslib:library:fl_set_xyplot_key_font}
    \index{xformslib \textit{(package)}!xformslib.library \textit{(module)}!xformslib.library.fl\_set\_xyplot\_key\_font \textit{(function)}}

    \vspace{0.5ex}

\hspace{.8\funcindent}\begin{boxedminipage}{\funcwidth}

    \raggedright \textbf{fl\_set\_xyplot\_key\_font}(\textit{pObject}, \textit{style}, \textit{size})

    \vspace{-1.5ex}

    \rule{\textwidth}{0.5\fboxrule}
\setlength{\parskip}{2ex}
\setlength{\parskip}{1ex}
      \textbf{Parameters}
      \vspace{-1ex}

      \begin{quote}
        \begin{Ventry}{xxxxxxx}

          \item[pObject]

          pointer to object

        \end{Ventry}

      \end{quote}

\textbf{Status:} Untested + NoDoc + NoDemo = NOT OK



    \end{boxedminipage}

    \label{xformslib:library:fl_get_xyplot_numdata}
    \index{xformslib \textit{(package)}!xformslib.library \textit{(module)}!xformslib.library.fl\_get\_xyplot\_numdata \textit{(function)}}

    \vspace{0.5ex}

\hspace{.8\funcindent}\begin{boxedminipage}{\funcwidth}

    \raggedright \textbf{fl\_get\_xyplot\_numdata}(\textit{pObject}, \textit{idnum})

    \vspace{-1.5ex}

    \rule{\textwidth}{0.5\fboxrule}
\setlength{\parskip}{2ex}
\setlength{\parskip}{1ex}
      \textbf{Parameters}
      \vspace{-1ex}

      \begin{quote}
        \begin{Ventry}{xxxxxxx}

          \item[pObject]

          pointer to object

        \end{Ventry}

      \end{quote}

      \textbf{Return Value}
    \vspace{-1ex}

      \begin{quote}
      num

      \end{quote}

\textbf{Status:} Untested + NoDoc + NoDemo = NOT OK



    \end{boxedminipage}

    \label{xformslib:library:fl_xyplot_s2w}
    \index{xformslib \textit{(package)}!xformslib.library \textit{(module)}!xformslib.library.fl\_xyplot\_s2w \textit{(function)}}

    \vspace{0.5ex}

\hspace{.8\funcindent}\begin{boxedminipage}{\funcwidth}

    \raggedright \textbf{fl\_xyplot\_s2w}(\textit{pObject}, \textit{sx}, \textit{sy}, \textit{wx}, \textit{wy})

    \vspace{-1.5ex}

    \rule{\textwidth}{0.5\fboxrule}
\setlength{\parskip}{2ex}
\setlength{\parskip}{1ex}
      \textbf{Parameters}
      \vspace{-1ex}

      \begin{quote}
        \begin{Ventry}{xxxxxxx}

          \item[pObject]

          pointer to object

        \end{Ventry}

      \end{quote}

\textbf{Status:} Untested + NoDoc + NoDemo = NOT OK



    \end{boxedminipage}

    \label{xformslib:library:fl_xyplot_w2s}
    \index{xformslib \textit{(package)}!xformslib.library \textit{(module)}!xformslib.library.fl\_xyplot\_w2s \textit{(function)}}

    \vspace{0.5ex}

\hspace{.8\funcindent}\begin{boxedminipage}{\funcwidth}

    \raggedright \textbf{fl\_xyplot\_w2s}(\textit{pObject}, \textit{wx}, \textit{wy}, \textit{sx}, \textit{sy})

    \vspace{-1.5ex}

    \rule{\textwidth}{0.5\fboxrule}
\setlength{\parskip}{2ex}
\setlength{\parskip}{1ex}
      \textbf{Parameters}
      \vspace{-1ex}

      \begin{quote}
        \begin{Ventry}{xxxxxxx}

          \item[pObject]

          pointer to object

        \end{Ventry}

      \end{quote}

\textbf{Status:} Untested + NoDoc + NoDemo = NOT OK



    \end{boxedminipage}

    \label{xformslib:library:fl_set_xyplot_xscale}
    \index{xformslib \textit{(package)}!xformslib.library \textit{(module)}!xformslib.library.fl\_set\_xyplot\_xscale \textit{(function)}}

    \vspace{0.5ex}

\hspace{.8\funcindent}\begin{boxedminipage}{\funcwidth}

    \raggedright \textbf{fl\_set\_xyplot\_xscale}(\textit{pObject}, \textit{scale}, \textit{base})

    \vspace{-1.5ex}

    \rule{\textwidth}{0.5\fboxrule}
\setlength{\parskip}{2ex}
\setlength{\parskip}{1ex}
      \textbf{Parameters}
      \vspace{-1ex}

      \begin{quote}
        \begin{Ventry}{xxxxxxx}

          \item[pObject]

          pointer to object

        \end{Ventry}

      \end{quote}

\textbf{Status:} Untested + NoDoc + NoDemo = NOT OK



    \end{boxedminipage}

    \label{xformslib:library:fl_set_xyplot_yscale}
    \index{xformslib \textit{(package)}!xformslib.library \textit{(module)}!xformslib.library.fl\_set\_xyplot\_yscale \textit{(function)}}

    \vspace{0.5ex}

\hspace{.8\funcindent}\begin{boxedminipage}{\funcwidth}

    \raggedright \textbf{fl\_set\_xyplot\_yscale}(\textit{pObject}, \textit{scale}, \textit{base})

    \vspace{-1.5ex}

    \rule{\textwidth}{0.5\fboxrule}
\setlength{\parskip}{2ex}
\setlength{\parskip}{1ex}
      \textbf{Parameters}
      \vspace{-1ex}

      \begin{quote}
        \begin{Ventry}{xxxxxxx}

          \item[pObject]

          pointer to object

        \end{Ventry}

      \end{quote}

\textbf{Status:} Untested + NoDoc + NoDemo = NOT OK



    \end{boxedminipage}

    \label{xformslib:library:fl_clear_xyplot}
    \index{xformslib \textit{(package)}!xformslib.library \textit{(module)}!xformslib.library.fl\_clear\_xyplot \textit{(function)}}

    \vspace{0.5ex}

\hspace{.8\funcindent}\begin{boxedminipage}{\funcwidth}

    \raggedright \textbf{fl\_clear\_xyplot}(\textit{pObject})

    \vspace{-1.5ex}

    \rule{\textwidth}{0.5\fboxrule}
\setlength{\parskip}{2ex}
\setlength{\parskip}{1ex}
      \textbf{Parameters}
      \vspace{-1ex}

      \begin{quote}
        \begin{Ventry}{xxxxxxx}

          \item[pObject]

          pointer to object

        \end{Ventry}

      \end{quote}

\textbf{Status:} Untested + NoDoc + NoDemo = NOT OK



    \end{boxedminipage}

    \label{xformslib:library:fl_set_xyplot_linewidth}
    \index{xformslib \textit{(package)}!xformslib.library \textit{(module)}!xformslib.library.fl\_set\_xyplot\_linewidth \textit{(function)}}

    \vspace{0.5ex}

\hspace{.8\funcindent}\begin{boxedminipage}{\funcwidth}

    \raggedright \textbf{fl\_set\_xyplot\_linewidth}(\textit{pObject}, \textit{idnum}, \textit{lw})

    \vspace{-1.5ex}

    \rule{\textwidth}{0.5\fboxrule}
\setlength{\parskip}{2ex}
\setlength{\parskip}{1ex}
      \textbf{Parameters}
      \vspace{-1ex}

      \begin{quote}
        \begin{Ventry}{xxxxxxx}

          \item[pObject]

          pointer to object

        \end{Ventry}

      \end{quote}

\textbf{Status:} Untested + NoDoc + NoDemo = NOT OK



    \end{boxedminipage}

    \label{xformslib:library:fl_set_xyplot_xgrid}
    \index{xformslib \textit{(package)}!xformslib.library \textit{(module)}!xformslib.library.fl\_set\_xyplot\_xgrid \textit{(function)}}

    \vspace{0.5ex}

\hspace{.8\funcindent}\begin{boxedminipage}{\funcwidth}

    \raggedright \textbf{fl\_set\_xyplot\_xgrid}(\textit{pObject}, \textit{xgrid})

    \vspace{-1.5ex}

    \rule{\textwidth}{0.5\fboxrule}
\setlength{\parskip}{2ex}
\setlength{\parskip}{1ex}
      \textbf{Parameters}
      \vspace{-1ex}

      \begin{quote}
        \begin{Ventry}{xxxxxxx}

          \item[pObject]

          pointer to object

        \end{Ventry}

      \end{quote}

\textbf{Status:} Untested + NoDoc + NoDemo = NOT OK



    \end{boxedminipage}

    \label{xformslib:library:fl_set_xyplot_ygrid}
    \index{xformslib \textit{(package)}!xformslib.library \textit{(module)}!xformslib.library.fl\_set\_xyplot\_ygrid \textit{(function)}}

    \vspace{0.5ex}

\hspace{.8\funcindent}\begin{boxedminipage}{\funcwidth}

    \raggedright \textbf{fl\_set\_xyplot\_ygrid}(\textit{pObject}, \textit{ygrid})

    \vspace{-1.5ex}

    \rule{\textwidth}{0.5\fboxrule}
\setlength{\parskip}{2ex}
\setlength{\parskip}{1ex}
      \textbf{Parameters}
      \vspace{-1ex}

      \begin{quote}
        \begin{Ventry}{xxxxxxx}

          \item[pObject]

          pointer to object

        \end{Ventry}

      \end{quote}

\textbf{Status:} Untested + NoDoc + NoDemo = NOT OK



    \end{boxedminipage}

    \label{xformslib:library:fl_set_xyplot_grid_linestyle}
    \index{xformslib \textit{(package)}!xformslib.library \textit{(module)}!xformslib.library.fl\_set\_xyplot\_grid\_linestyle \textit{(function)}}

    \vspace{0.5ex}

\hspace{.8\funcindent}\begin{boxedminipage}{\funcwidth}

    \raggedright \textbf{fl\_set\_xyplot\_grid\_linestyle}(\textit{pObject}, \textit{linestyle})

    \vspace{-1.5ex}

    \rule{\textwidth}{0.5\fboxrule}
\setlength{\parskip}{2ex}
\setlength{\parskip}{1ex}
      \textbf{Parameters}
      \vspace{-1ex}

      \begin{quote}
        \begin{Ventry}{xxxxxxxxx}

          \item[pObject]

          pointer to object

          \item[linestyle]

          style of the line to draw

            {\it (type=[num./int] from xfdata module FL\_SOLID, FL\_USERDASH, FL\_USERDOUBLEDASH, 
FL\_DOT, FL\_DOTDASH, FL\_DASH, FL\_LONGDASH)}

        \end{Ventry}

      \end{quote}

      \textbf{Return Value}
    \vspace{-1ex}

      \begin{quote}
      num

      \end{quote}

\textbf{Status:} Untested + NoDoc + NoDemo = NOT OK



    \end{boxedminipage}

    \label{xformslib:library:fl_set_xyplot_alphaxtics}
    \index{xformslib \textit{(package)}!xformslib.library \textit{(module)}!xformslib.library.fl\_set\_xyplot\_alphaxtics \textit{(function)}}

    \vspace{0.5ex}

\hspace{.8\funcindent}\begin{boxedminipage}{\funcwidth}

    \raggedright \textbf{fl\_set\_xyplot\_alphaxtics}(\textit{pObject}, \textit{m}, \textit{s})

    \vspace{-1.5ex}

    \rule{\textwidth}{0.5\fboxrule}
\setlength{\parskip}{2ex}
\setlength{\parskip}{1ex}
      \textbf{Parameters}
      \vspace{-1ex}

      \begin{quote}
        \begin{Ventry}{xxxxxxx}

          \item[pObject]

          pointer to object

        \end{Ventry}

      \end{quote}

\textbf{Status:} Untested + NoDoc + NoDemo = NOT OK



    \end{boxedminipage}

    \label{xformslib:library:fl_set_xyplot_alphaytics}
    \index{xformslib \textit{(package)}!xformslib.library \textit{(module)}!xformslib.library.fl\_set\_xyplot\_alphaytics \textit{(function)}}

    \vspace{0.5ex}

\hspace{.8\funcindent}\begin{boxedminipage}{\funcwidth}

    \raggedright \textbf{fl\_set\_xyplot\_alphaytics}(\textit{pObject}, \textit{m}, \textit{s})

    \vspace{-1.5ex}

    \rule{\textwidth}{0.5\fboxrule}
\setlength{\parskip}{2ex}
\setlength{\parskip}{1ex}
      \textbf{Parameters}
      \vspace{-1ex}

      \begin{quote}
        \begin{Ventry}{xxxxxxx}

          \item[pObject]

          pointer to object

        \end{Ventry}

      \end{quote}

\textbf{Status:} Untested + NoDoc + NoDemo = NOT OK



    \end{boxedminipage}

    \label{xformslib:library:fl_set_xyplot_fixed_xaxis}
    \index{xformslib \textit{(package)}!xformslib.library \textit{(module)}!xformslib.library.fl\_set\_xyplot\_fixed\_xaxis \textit{(function)}}

    \vspace{0.5ex}

\hspace{.8\funcindent}\begin{boxedminipage}{\funcwidth}

    \raggedright \textbf{fl\_set\_xyplot\_fixed\_xaxis}(\textit{pObject}, \textit{lm}, \textit{rm})

    \vspace{-1.5ex}

    \rule{\textwidth}{0.5\fboxrule}
\setlength{\parskip}{2ex}
\setlength{\parskip}{1ex}
      \textbf{Parameters}
      \vspace{-1ex}

      \begin{quote}
        \begin{Ventry}{xxxxxxx}

          \item[pObject]

          pointer to object

        \end{Ventry}

      \end{quote}

\textbf{Status:} Untested + NoDoc + NoDemo = NOT OK



    \end{boxedminipage}

    \label{xformslib:library:fl_set_xyplot_fixed_yaxis}
    \index{xformslib \textit{(package)}!xformslib.library \textit{(module)}!xformslib.library.fl\_set\_xyplot\_fixed\_yaxis \textit{(function)}}

    \vspace{0.5ex}

\hspace{.8\funcindent}\begin{boxedminipage}{\funcwidth}

    \raggedright \textbf{fl\_set\_xyplot\_fixed\_yaxis}(\textit{pObject}, \textit{bm}, \textit{tm})

    \vspace{-1.5ex}

    \rule{\textwidth}{0.5\fboxrule}
\setlength{\parskip}{2ex}
\setlength{\parskip}{1ex}
      \textbf{Parameters}
      \vspace{-1ex}

      \begin{quote}
        \begin{Ventry}{xxxxxxx}

          \item[pObject]

          pointer to object

        \end{Ventry}

      \end{quote}

\textbf{Status:} Untested + NoDoc + NoDemo = NOT OK



    \end{boxedminipage}

    \label{xformslib:library:fl_interpolate}
    \index{xformslib \textit{(package)}!xformslib.library \textit{(module)}!xformslib.library.fl\_interpolate \textit{(function)}}

    \vspace{0.5ex}

\hspace{.8\funcindent}\begin{boxedminipage}{\funcwidth}

    \raggedright \textbf{fl\_interpolate}(\textit{wx}, \textit{wy}, \textit{nin}, \textit{x}, \textit{y}, \textit{grid}, \textit{ndeg})

    \vspace{-1.5ex}

    \rule{\textwidth}{0.5\fboxrule}
\setlength{\parskip}{2ex}
\setlength{\parskip}{1ex}
      \textbf{Return Value}
    \vspace{-1ex}

      \begin{quote}
      num

      \end{quote}

\textbf{Status:} Untested + NoDoc + NoDemo = NOT OK



    \end{boxedminipage}

    \label{xformslib:library:fl_spline_interpolate}
    \index{xformslib \textit{(package)}!xformslib.library \textit{(module)}!xformslib.library.fl\_spline\_interpolate \textit{(function)}}

    \vspace{0.5ex}

\hspace{.8\funcindent}\begin{boxedminipage}{\funcwidth}

    \raggedright \textbf{fl\_spline\_interpolate}(\textit{wx}, \textit{wy}, \textit{nin}, \textit{x}, \textit{y}, \textit{grid})

    \vspace{-1.5ex}

    \rule{\textwidth}{0.5\fboxrule}
\setlength{\parskip}{2ex}
\setlength{\parskip}{1ex}
      \textbf{Return Value}
    \vspace{-1ex}

      \begin{quote}
      num

      \end{quote}

\textbf{Status:} Untested + NoDoc + NoDemo = NOT OK



    \end{boxedminipage}

    \label{xformslib:library:fl_set_xyplot_symbol}
    \index{xformslib \textit{(package)}!xformslib.library \textit{(module)}!xformslib.library.fl\_set\_xyplot\_symbol \textit{(function)}}

    \vspace{0.5ex}

\hspace{.8\funcindent}\begin{boxedminipage}{\funcwidth}

    \raggedright \textbf{fl\_set\_xyplot\_symbol}(\textit{pObject}, \textit{idnum}, \textit{py\_XyPlotSymbol})

    \vspace{-1.5ex}

    \rule{\textwidth}{0.5\fboxrule}
\setlength{\parskip}{2ex}
\setlength{\parskip}{1ex}
      \textbf{Parameters}
      \vspace{-1ex}

      \begin{quote}
        \begin{Ventry}{xxxxxxx}

          \item[pObject]

          pointer to object

        \end{Ventry}

      \end{quote}

      \textbf{Return Value}
    \vspace{-1ex}

      \begin{quote}
      xyplot\_symbol func

      \end{quote}

\textbf{Status:} Untested + NoDoc + NoDemo = NOT OK



    \end{boxedminipage}

    \label{xformslib:library:fl_set_xyplot_mark_active}
    \index{xformslib \textit{(package)}!xformslib.library \textit{(module)}!xformslib.library.fl\_set\_xyplot\_mark\_active \textit{(function)}}

    \vspace{0.5ex}

\hspace{.8\funcindent}\begin{boxedminipage}{\funcwidth}

    \raggedright \textbf{fl\_set\_xyplot\_mark\_active}(\textit{pObject}, \textit{y})

    \vspace{-1.5ex}

    \rule{\textwidth}{0.5\fboxrule}
\setlength{\parskip}{2ex}
\setlength{\parskip}{1ex}
      \textbf{Parameters}
      \vspace{-1ex}

      \begin{quote}
        \begin{Ventry}{xxxxxxx}

          \item[pObject]

          pointer to object

        \end{Ventry}

      \end{quote}

      \textbf{Return Value}
    \vspace{-1ex}

      \begin{quote}
      num

      \end{quote}

\textbf{Status:} Untested + NoDoc + NoDemo = NOT OK



    \end{boxedminipage}

    \label{xformslib:library:fl_gc_}
    \index{xformslib \textit{(package)}!xformslib.library \textit{(module)}!xformslib.library.fl\_gc\_ \textit{(function)}}

    \vspace{0.5ex}

\hspace{.8\funcindent}\begin{boxedminipage}{\funcwidth}

    \raggedright \textbf{fl\_gc\_}()

    \vspace{-1.5ex}

    \rule{\textwidth}{0.5\fboxrule}
\setlength{\parskip}{2ex}
\setlength{\parskip}{1ex}
      \textbf{Return Value}
    \vspace{-1ex}

      \begin{quote}
      gc

      \end{quote}

\textbf{Status:} Untested + NoDoc + NoDemo = NOT OK



    \end{boxedminipage}

    \label{xformslib:library:fl_gc_}
    \index{xformslib \textit{(package)}!xformslib.library \textit{(module)}!xformslib.library.fl\_gc\_ \textit{(function)}}

    \vspace{0.5ex}

\hspace{.8\funcindent}\begin{boxedminipage}{\funcwidth}

    \raggedright \textbf{fl\_gc}()

    \vspace{-1.5ex}

    \rule{\textwidth}{0.5\fboxrule}
\setlength{\parskip}{2ex}
\setlength{\parskip}{1ex}
      \textbf{Return Value}
    \vspace{-1ex}

      \begin{quote}
      gc

      \end{quote}

\textbf{Status:} Untested + NoDoc + NoDemo = NOT OK



    \end{boxedminipage}

    \label{xformslib:library:fl_textgc_}
    \index{xformslib \textit{(package)}!xformslib.library \textit{(module)}!xformslib.library.fl\_textgc\_ \textit{(function)}}

    \vspace{0.5ex}

\hspace{.8\funcindent}\begin{boxedminipage}{\funcwidth}

    \raggedright \textbf{fl\_textgc\_}()

    \vspace{-1.5ex}

    \rule{\textwidth}{0.5\fboxrule}
\setlength{\parskip}{2ex}
\setlength{\parskip}{1ex}
      \textbf{Return Value}
    \vspace{-1ex}

      \begin{quote}
      gc

      \end{quote}

\textbf{Status:} Untested + NoDoc + NoDemo = NOT OK



    \end{boxedminipage}

    \label{xformslib:library:fl_textgc_}
    \index{xformslib \textit{(package)}!xformslib.library \textit{(module)}!xformslib.library.fl\_textgc\_ \textit{(function)}}

    \vspace{0.5ex}

\hspace{.8\funcindent}\begin{boxedminipage}{\funcwidth}

    \raggedright \textbf{fl\_textgc}()

    \vspace{-1.5ex}

    \rule{\textwidth}{0.5\fboxrule}
\setlength{\parskip}{2ex}
\setlength{\parskip}{1ex}
      \textbf{Return Value}
    \vspace{-1ex}

      \begin{quote}
      gc

      \end{quote}

\textbf{Status:} Untested + NoDoc + NoDemo = NOT OK



    \end{boxedminipage}

    \label{xformslib:library:fl_fheight_}
    \index{xformslib \textit{(package)}!xformslib.library \textit{(module)}!xformslib.library.fl\_fheight\_ \textit{(function)}}

    \vspace{0.5ex}

\hspace{.8\funcindent}\begin{boxedminipage}{\funcwidth}

    \raggedright \textbf{fl\_fheight\_}()

    \vspace{-1.5ex}

    \rule{\textwidth}{0.5\fboxrule}
\setlength{\parskip}{2ex}
\setlength{\parskip}{1ex}
      \textbf{Return Value}
    \vspace{-1ex}

      \begin{quote}
      num

      \end{quote}

\textbf{Status:} Untested + NoDoc + NoDemo = NOT OK



    \end{boxedminipage}

    \label{xformslib:library:fl_fheight_}
    \index{xformslib \textit{(package)}!xformslib.library \textit{(module)}!xformslib.library.fl\_fheight\_ \textit{(function)}}

    \vspace{0.5ex}

\hspace{.8\funcindent}\begin{boxedminipage}{\funcwidth}

    \raggedright \textbf{fl\_fheight}()

    \vspace{-1.5ex}

    \rule{\textwidth}{0.5\fboxrule}
\setlength{\parskip}{2ex}
\setlength{\parskip}{1ex}
      \textbf{Return Value}
    \vspace{-1ex}

      \begin{quote}
      num

      \end{quote}

\textbf{Status:} Untested + NoDoc + NoDemo = NOT OK



    \end{boxedminipage}

    \label{xformslib:library:fl_fdesc_}
    \index{xformslib \textit{(package)}!xformslib.library \textit{(module)}!xformslib.library.fl\_fdesc\_ \textit{(function)}}

    \vspace{0.5ex}

\hspace{.8\funcindent}\begin{boxedminipage}{\funcwidth}

    \raggedright \textbf{fl\_fdesc\_}()

    \vspace{-1.5ex}

    \rule{\textwidth}{0.5\fboxrule}
\setlength{\parskip}{2ex}
\setlength{\parskip}{1ex}
      \textbf{Return Value}
    \vspace{-1ex}

      \begin{quote}
      num

      \end{quote}

\textbf{Status:} Untested + NoDoc + NoDemo = NOT OK



    \end{boxedminipage}

    \label{xformslib:library:fl_fdesc_}
    \index{xformslib \textit{(package)}!xformslib.library \textit{(module)}!xformslib.library.fl\_fdesc\_ \textit{(function)}}

    \vspace{0.5ex}

\hspace{.8\funcindent}\begin{boxedminipage}{\funcwidth}

    \raggedright \textbf{fl\_fdesc}()

    \vspace{-1.5ex}

    \rule{\textwidth}{0.5\fboxrule}
\setlength{\parskip}{2ex}
\setlength{\parskip}{1ex}
      \textbf{Return Value}
    \vspace{-1ex}

      \begin{quote}
      num

      \end{quote}

\textbf{Status:} Untested + NoDoc + NoDemo = NOT OK



    \end{boxedminipage}

    \label{xformslib:library:fl_cur_win_}
    \index{xformslib \textit{(package)}!xformslib.library \textit{(module)}!xformslib.library.fl\_cur\_win\_ \textit{(function)}}

    \vspace{0.5ex}

\hspace{.8\funcindent}\begin{boxedminipage}{\funcwidth}

    \raggedright \textbf{fl\_cur\_win\_}()

    \vspace{-1.5ex}

    \rule{\textwidth}{0.5\fboxrule}
\setlength{\parskip}{2ex}
\setlength{\parskip}{1ex}
      \textbf{Return Value}
    \vspace{-1ex}

      \begin{quote}
      window

      \end{quote}

\textbf{Status:} Untested + NoDoc + NoDemo = NOT OK



    \end{boxedminipage}

    \label{xformslib:library:fl_cur_win_}
    \index{xformslib \textit{(package)}!xformslib.library \textit{(module)}!xformslib.library.fl\_cur\_win\_ \textit{(function)}}

    \vspace{0.5ex}

\hspace{.8\funcindent}\begin{boxedminipage}{\funcwidth}

    \raggedright \textbf{fl\_cur\_win}()

    \vspace{-1.5ex}

    \rule{\textwidth}{0.5\fboxrule}
\setlength{\parskip}{2ex}
\setlength{\parskip}{1ex}
      \textbf{Return Value}
    \vspace{-1ex}

      \begin{quote}
      window

      \end{quote}

\textbf{Status:} Untested + NoDoc + NoDemo = NOT OK



    \end{boxedminipage}

    \label{xformslib:library:fl_cur_fs_}
    \index{xformslib \textit{(package)}!xformslib.library \textit{(module)}!xformslib.library.fl\_cur\_fs\_ \textit{(function)}}

    \vspace{0.5ex}

\hspace{.8\funcindent}\begin{boxedminipage}{\funcwidth}

    \raggedright \textbf{fl\_cur\_fs\_}()

\setlength{\parskip}{2ex}
\setlength{\parskip}{1ex}
      \textbf{Return Value}
    \vspace{-1ex}

      \begin{quote}
      XFontStruct class

      \end{quote}

    \end{boxedminipage}

    \label{xformslib:library:fl_cur_fs_}
    \index{xformslib \textit{(package)}!xformslib.library \textit{(module)}!xformslib.library.fl\_cur\_fs\_ \textit{(function)}}

    \vspace{0.5ex}

\hspace{.8\funcindent}\begin{boxedminipage}{\funcwidth}

    \raggedright \textbf{fl\_cur\_fs}()

\setlength{\parskip}{2ex}
\setlength{\parskip}{1ex}
      \textbf{Return Value}
    \vspace{-1ex}

      \begin{quote}
      XFontStruct class

      \end{quote}

    \end{boxedminipage}

    \label{xformslib:library:fl_display_}
    \index{xformslib \textit{(package)}!xformslib.library \textit{(module)}!xformslib.library.fl\_display\_ \textit{(function)}}

    \vspace{0.5ex}

\hspace{.8\funcindent}\begin{boxedminipage}{\funcwidth}

    \raggedright \textbf{fl\_display\_}()

    \vspace{-1.5ex}

    \rule{\textwidth}{0.5\fboxrule}
\setlength{\parskip}{2ex}
\setlength{\parskip}{1ex}
      \textbf{Return Value}
    \vspace{-1ex}

      \begin{quote}
      pDisplay

      \end{quote}

\textbf{Status:} Untested + NoDoc + NoDemo = NOT OK



    \end{boxedminipage}

    \label{xformslib:library:FL_RGB2GRAY}
    \index{xformslib \textit{(package)}!xformslib.library \textit{(module)}!xformslib.library.FL\_RGB2GRAY \textit{(function)}}

    \vspace{0.5ex}

\hspace{.8\funcindent}\begin{boxedminipage}{\funcwidth}

    \raggedright \textbf{FL\_RGB2GRAY}(\textit{r}, \textit{g}, \textit{b})

\setlength{\parskip}{2ex}
\setlength{\parskip}{1ex}
    \end{boxedminipage}

    \label{xformslib:library:FL_IsRGB}
    \index{xformslib \textit{(package)}!xformslib.library \textit{(module)}!xformslib.library.FL\_IsRGB \textit{(function)}}

    \vspace{0.5ex}

\hspace{.8\funcindent}\begin{boxedminipage}{\funcwidth}

    \raggedright \textbf{FL\_IsRGB}(\textit{pImage})

\setlength{\parskip}{2ex}
\setlength{\parskip}{1ex}
    \end{boxedminipage}

    \label{xformslib:library:FL_IsPacked}
    \index{xformslib \textit{(package)}!xformslib.library \textit{(module)}!xformslib.library.FL\_IsPacked \textit{(function)}}

    \vspace{0.5ex}

\hspace{.8\funcindent}\begin{boxedminipage}{\funcwidth}

    \raggedright \textbf{FL\_IsPacked}(\textit{pImage})

\setlength{\parskip}{2ex}
\setlength{\parskip}{1ex}
    \end{boxedminipage}

    \label{xformslib:library:flimage_setup}
    \index{xformslib \textit{(package)}!xformslib.library \textit{(module)}!xformslib.library.flimage\_setup \textit{(function)}}

    \vspace{0.5ex}

\hspace{.8\funcindent}\begin{boxedminipage}{\funcwidth}

    \raggedright \textbf{flimage\_setup}(\textit{pImageSetup})

    \vspace{-1.5ex}

    \rule{\textwidth}{0.5\fboxrule}
\setlength{\parskip}{2ex}
\setlength{\parskip}{1ex}
      \textbf{Parameters}
      \vspace{-1ex}

      \begin{quote}
        \begin{Ventry}{xxxxxxxxxxx}

          \item[pImageSetup]

          pointer to imagesetup struct

        \end{Ventry}

      \end{quote}

\textbf{Status:} Untested + NoDoc + NoDemo = NOT OK



    \end{boxedminipage}

    \label{xformslib:library:flimage_load}
    \index{xformslib \textit{(package)}!xformslib.library \textit{(module)}!xformslib.library.flimage\_load \textit{(function)}}

    \vspace{0.5ex}

\hspace{.8\funcindent}\begin{boxedminipage}{\funcwidth}

    \raggedright \textbf{flimage\_load}(\textit{filename})

    \vspace{-1.5ex}

    \rule{\textwidth}{0.5\fboxrule}
\setlength{\parskip}{2ex}
\setlength{\parskip}{1ex}
      \textbf{Parameters}
      \vspace{-1ex}

      \begin{quote}
        \begin{Ventry}{xxxxxxxx}

          \item[filename]

          name of file to load

        \end{Ventry}

      \end{quote}

      \textbf{Return Value}
    \vspace{-1ex}

      \begin{quote}
      pImage

      \end{quote}

\textbf{Status:} Untested + NoDoc + NoDemo = NOT OK



    \end{boxedminipage}

    \label{xformslib:library:flimage_read}
    \index{xformslib \textit{(package)}!xformslib.library \textit{(module)}!xformslib.library.flimage\_read \textit{(function)}}

    \vspace{0.5ex}

\hspace{.8\funcindent}\begin{boxedminipage}{\funcwidth}

    \raggedright \textbf{flimage\_read}(\textit{pImage})

    \vspace{-1.5ex}

    \rule{\textwidth}{0.5\fboxrule}
\setlength{\parskip}{2ex}
\setlength{\parskip}{1ex}
      \textbf{Parameters}
      \vspace{-1ex}

      \begin{quote}
        \begin{Ventry}{xxxxxx}

          \item[pImage]

          pointer to image

        \end{Ventry}

      \end{quote}

      \textbf{Return Value}
    \vspace{-1ex}

      \begin{quote}
      pImage

      \end{quote}

\textbf{Status:} Untested + NoDoc + NoDemo = NOT OK



    \end{boxedminipage}

    \label{xformslib:library:flimage_dump}
    \index{xformslib \textit{(package)}!xformslib.library \textit{(module)}!xformslib.library.flimage\_dump \textit{(function)}}

    \vspace{0.5ex}

\hspace{.8\funcindent}\begin{boxedminipage}{\funcwidth}

    \raggedright \textbf{flimage\_dump}(\textit{pImage}, \textit{p2}, \textit{p3})

    \vspace{-1.5ex}

    \rule{\textwidth}{0.5\fboxrule}
\setlength{\parskip}{2ex}
\setlength{\parskip}{1ex}
      \textbf{Parameters}
      \vspace{-1ex}

      \begin{quote}
        \begin{Ventry}{xxxxxx}

          \item[pImage]

          pointer to image

        \end{Ventry}

      \end{quote}

      \textbf{Return Value}
    \vspace{-1ex}

      \begin{quote}
      num

      \end{quote}

\textbf{Status:} Untested + NoDoc + NoDemo = NOT OK



    \end{boxedminipage}

    \label{xformslib:library:flimage_close}
    \index{xformslib \textit{(package)}!xformslib.library \textit{(module)}!xformslib.library.flimage\_close \textit{(function)}}

    \vspace{0.5ex}

\hspace{.8\funcindent}\begin{boxedminipage}{\funcwidth}

    \raggedright \textbf{flimage\_close}(\textit{pImage})

    \vspace{-1.5ex}

    \rule{\textwidth}{0.5\fboxrule}
\setlength{\parskip}{2ex}
\setlength{\parskip}{1ex}
      \textbf{Parameters}
      \vspace{-1ex}

      \begin{quote}
        \begin{Ventry}{xxxxxx}

          \item[pImage]

          pointer to image

        \end{Ventry}

      \end{quote}

      \textbf{Return Value}
    \vspace{-1ex}

      \begin{quote}
      num

      \end{quote}

\textbf{Status:} Untested + NoDoc + NoDemo = NOT OK



    \end{boxedminipage}

    \label{xformslib:library:flimage_alloc}
    \index{xformslib \textit{(package)}!xformslib.library \textit{(module)}!xformslib.library.flimage\_alloc \textit{(function)}}

    \vspace{0.5ex}

\hspace{.8\funcindent}\begin{boxedminipage}{\funcwidth}

    \raggedright \textbf{flimage\_alloc}()

    \vspace{-1.5ex}

    \rule{\textwidth}{0.5\fboxrule}
\setlength{\parskip}{2ex}
\setlength{\parskip}{1ex}
      \textbf{Return Value}
    \vspace{-1ex}

      \begin{quote}
      pImage

      \end{quote}

\textbf{Status:} Untested + NoDoc + NoDemo = NOT OK



    \end{boxedminipage}

    \label{xformslib:library:flimage_getmem}
    \index{xformslib \textit{(package)}!xformslib.library \textit{(module)}!xformslib.library.flimage\_getmem \textit{(function)}}

    \vspace{0.5ex}

\hspace{.8\funcindent}\begin{boxedminipage}{\funcwidth}

    \raggedright \textbf{flimage\_getmem}(\textit{pImage})

    \vspace{-1.5ex}

    \rule{\textwidth}{0.5\fboxrule}
\setlength{\parskip}{2ex}
\setlength{\parskip}{1ex}
      \textbf{Parameters}
      \vspace{-1ex}

      \begin{quote}
        \begin{Ventry}{xxxxxx}

          \item[pImage]

          pointer to image

        \end{Ventry}

      \end{quote}

      \textbf{Return Value}
    \vspace{-1ex}

      \begin{quote}
      num

      \end{quote}

\textbf{Status:} Untested + NoDoc + NoDemo = NOT OK



    \end{boxedminipage}

    \label{xformslib:library:flimage_is_supported}
    \index{xformslib \textit{(package)}!xformslib.library \textit{(module)}!xformslib.library.flimage\_is\_supported \textit{(function)}}

    \vspace{0.5ex}

\hspace{.8\funcindent}\begin{boxedminipage}{\funcwidth}

    \raggedright \textbf{flimage\_is\_supported}(\textit{fname})

    \vspace{-1.5ex}

    \rule{\textwidth}{0.5\fboxrule}
\setlength{\parskip}{2ex}
\setlength{\parskip}{1ex}
      \textbf{Parameters}
      \vspace{-1ex}

      \begin{quote}
        \begin{Ventry}{xxxxx}

          \item[fname]

          filename

        \end{Ventry}

      \end{quote}

      \textbf{Return Value}
    \vspace{-1ex}

      \begin{quote}
      num

      \end{quote}

\textbf{Status:} Untested + NoDoc + NoDemo = NOT OK



    \end{boxedminipage}

    \label{xformslib:library:flimage_description_via_filter}
    \index{xformslib \textit{(package)}!xformslib.library \textit{(module)}!xformslib.library.flimage\_description\_via\_filter \textit{(function)}}

    \vspace{0.5ex}

\hspace{.8\funcindent}\begin{boxedminipage}{\funcwidth}

    \raggedright \textbf{flimage\_description\_via\_filter}(\textit{pImage}, \textit{p2}, \textit{p3}, \textit{p4})

    \vspace{-1.5ex}

    \rule{\textwidth}{0.5\fboxrule}
\setlength{\parskip}{2ex}
\setlength{\parskip}{1ex}
      \textbf{Parameters}
      \vspace{-1ex}

      \begin{quote}
        \begin{Ventry}{xxxxxx}

          \item[pImage]

          pointer to image

        \end{Ventry}

      \end{quote}

      \textbf{Return Value}
    \vspace{-1ex}

      \begin{quote}
      num

      \end{quote}

\textbf{Status:} Untested + NoDoc + NoDemo = NOT OK



    \end{boxedminipage}

    \label{xformslib:library:flimage_write_via_filter}
    \index{xformslib \textit{(package)}!xformslib.library \textit{(module)}!xformslib.library.flimage\_write\_via\_filter \textit{(function)}}

    \vspace{0.5ex}

\hspace{.8\funcindent}\begin{boxedminipage}{\funcwidth}

    \raggedright \textbf{flimage\_write\_via\_filter}(\textit{pImage}, \textit{p2}, \textit{p3}, \textit{p4})

    \vspace{-1.5ex}

    \rule{\textwidth}{0.5\fboxrule}
\setlength{\parskip}{2ex}
\setlength{\parskip}{1ex}
      \textbf{Parameters}
      \vspace{-1ex}

      \begin{quote}
        \begin{Ventry}{xxxxxx}

          \item[pImage]

          pointer to image

        \end{Ventry}

      \end{quote}

      \textbf{Return Value}
    \vspace{-1ex}

      \begin{quote}
      num

      \end{quote}

\textbf{Status:} Untested + NoDoc + NoDemo = NOT OK



    \end{boxedminipage}

    \label{xformslib:library:flimage_free}
    \index{xformslib \textit{(package)}!xformslib.library \textit{(module)}!xformslib.library.flimage\_free \textit{(function)}}

    \vspace{0.5ex}

\hspace{.8\funcindent}\begin{boxedminipage}{\funcwidth}

    \raggedright \textbf{flimage\_free}(\textit{pImage})

    \vspace{-1.5ex}

    \rule{\textwidth}{0.5\fboxrule}
\setlength{\parskip}{2ex}
\setlength{\parskip}{1ex}
      \textbf{Parameters}
      \vspace{-1ex}

      \begin{quote}
        \begin{Ventry}{xxxxxx}

          \item[pImage]

          pointer to image

        \end{Ventry}

      \end{quote}

      \textbf{Return Value}
    \vspace{-1ex}

      \begin{quote}
      num

      \end{quote}

\textbf{Status:} Untested + NoDoc + NoDemo = NOT OK



    \end{boxedminipage}

    \label{xformslib:library:flimage_display}
    \index{xformslib \textit{(package)}!xformslib.library \textit{(module)}!xformslib.library.flimage\_display \textit{(function)}}

    \vspace{0.5ex}

\hspace{.8\funcindent}\begin{boxedminipage}{\funcwidth}

    \raggedright \textbf{flimage\_display}(\textit{pImage}, \textit{win})

    \vspace{-1.5ex}

    \rule{\textwidth}{0.5\fboxrule}
\setlength{\parskip}{2ex}
\setlength{\parskip}{1ex}
      \textbf{Parameters}
      \vspace{-1ex}

      \begin{quote}
        \begin{Ventry}{xxxxxx}

          \item[pImage]

          pointer to image

          \item[win]

          window

        \end{Ventry}

      \end{quote}

      \textbf{Return Value}
    \vspace{-1ex}

      \begin{quote}
      num

      \end{quote}

\textbf{Status:} Untested + NoDoc + NoDemo = NOT OK



    \end{boxedminipage}

    \label{xformslib:library:flimage_sdisplay}
    \index{xformslib \textit{(package)}!xformslib.library \textit{(module)}!xformslib.library.flimage\_sdisplay \textit{(function)}}

    \vspace{0.5ex}

\hspace{.8\funcindent}\begin{boxedminipage}{\funcwidth}

    \raggedright \textbf{flimage\_sdisplay}(\textit{pImage}, \textit{win})

    \vspace{-1.5ex}

    \rule{\textwidth}{0.5\fboxrule}
\setlength{\parskip}{2ex}
\setlength{\parskip}{1ex}
      \textbf{Parameters}
      \vspace{-1ex}

      \begin{quote}
        \begin{Ventry}{xxxxxx}

          \item[pImage]

          pointer to image

          \item[win]

          window

        \end{Ventry}

      \end{quote}

      \textbf{Return Value}
    \vspace{-1ex}

      \begin{quote}
      num

      \end{quote}

\textbf{Status:} Untested + NoDoc + NoDemo = NOT OK



    \end{boxedminipage}

    \label{xformslib:library:flimage_convert}
    \index{xformslib \textit{(package)}!xformslib.library \textit{(module)}!xformslib.library.flimage\_convert \textit{(function)}}

    \vspace{0.5ex}

\hspace{.8\funcindent}\begin{boxedminipage}{\funcwidth}

    \raggedright \textbf{flimage\_convert}(\textit{pImage}, \textit{newtype}, \textit{ncolors})

    \vspace{-1.5ex}

    \rule{\textwidth}{0.5\fboxrule}
\setlength{\parskip}{2ex}
    Convert an image to a new type.

\setlength{\parskip}{1ex}
      \textbf{Parameters}
      \vspace{-1ex}

      \begin{quote}
        \begin{Ventry}{xxxxxxx}

          \item[pImage]

          pointer to image

          \item[newtype]

          new type of flimage to convert to

          \item[ncolors]

          number of colors

        \end{Ventry}

      \end{quote}

      \textbf{Return Value}
    \vspace{-1ex}

      \begin{quote}
      num

      \end{quote}

\textbf{Status:} Untested + NoDoc + NoDemo = NOT OK



    \end{boxedminipage}

    \label{xformslib:library:flimage_type_name}
    \index{xformslib \textit{(package)}!xformslib.library \textit{(module)}!xformslib.library.flimage\_type\_name \textit{(function)}}

    \vspace{0.5ex}

\hspace{.8\funcindent}\begin{boxedminipage}{\funcwidth}

    \raggedright \textbf{flimage\_type\_name}(\textit{flimagetype})

    \vspace{-1.5ex}

    \rule{\textwidth}{0.5\fboxrule}
\setlength{\parskip}{2ex}
\setlength{\parskip}{1ex}
      \textbf{Parameters}
      \vspace{-1ex}

      \begin{quote}
        \begin{Ventry}{xxxxxxxxxxx}

          \item[flimagetype]

          type of flimage

        \end{Ventry}

      \end{quote}

      \textbf{Return Value}
    \vspace{-1ex}

      \begin{quote}
      name string

      \end{quote}

\textbf{Status:} Untested + NoDoc + NoDemo = NOT OK



    \end{boxedminipage}

    \label{xformslib:library:flimage_add_text}
    \index{xformslib \textit{(package)}!xformslib.library \textit{(module)}!xformslib.library.flimage\_add\_text \textit{(function)}}

    \vspace{0.5ex}

\hspace{.8\funcindent}\begin{boxedminipage}{\funcwidth}

    \raggedright \textbf{flimage\_add\_text}(\textit{pImage}, \textit{text}, \textit{length}, \textit{style}, \textit{size}, \textit{txtcolr}, \textit{bgcolr}, \textit{tran}, \textit{tx}, \textit{ty}, \textit{rot})

    \vspace{-1.5ex}

    \rule{\textwidth}{0.5\fboxrule}
\setlength{\parskip}{2ex}
\setlength{\parskip}{1ex}
      \textbf{Parameters}
      \vspace{-1ex}

      \begin{quote}
        \begin{Ventry}{xxxxxx}

          \item[pImage]

          pointer to image

        \end{Ventry}

      \end{quote}

      \textbf{Return Value}
    \vspace{-1ex}

      \begin{quote}
      num

      \end{quote}

\textbf{Status:} Untested + NoDoc + NoDemo = NOT OK



    \end{boxedminipage}

    \label{xformslib:library:flimage_add_text_struct}
    \index{xformslib \textit{(package)}!xformslib.library \textit{(module)}!xformslib.library.flimage\_add\_text\_struct \textit{(function)}}

    \vspace{0.5ex}

\hspace{.8\funcindent}\begin{boxedminipage}{\funcwidth}

    \raggedright \textbf{flimage\_add\_text\_struct}(\textit{pImage}, \textit{pImageText})

    \vspace{-1.5ex}

    \rule{\textwidth}{0.5\fboxrule}
\setlength{\parskip}{2ex}
\setlength{\parskip}{1ex}
      \textbf{Parameters}
      \vspace{-1ex}

      \begin{quote}
        \begin{Ventry}{xxxxxx}

          \item[pImage]

          pointer to image

        \end{Ventry}

      \end{quote}

      \textbf{Return Value}
    \vspace{-1ex}

      \begin{quote}
      num

      \end{quote}

\textbf{Status:} Untested + NoDoc + NoDemo = NOT OK



    \end{boxedminipage}

    \label{xformslib:library:flimage_delete_all_text}
    \index{xformslib \textit{(package)}!xformslib.library \textit{(module)}!xformslib.library.flimage\_delete\_all\_text \textit{(function)}}

    \vspace{0.5ex}

\hspace{.8\funcindent}\begin{boxedminipage}{\funcwidth}

    \raggedright \textbf{flimage\_delete\_all\_text}(\textit{pImage})

    \vspace{-1.5ex}

    \rule{\textwidth}{0.5\fboxrule}
\setlength{\parskip}{2ex}
\setlength{\parskip}{1ex}
      \textbf{Parameters}
      \vspace{-1ex}

      \begin{quote}
        \begin{Ventry}{xxxxxx}

          \item[pImage]

          pointer to image

        \end{Ventry}

      \end{quote}

\textbf{Status:} Untested + NoDoc + NoDemo = NOT OK



    \end{boxedminipage}

    \label{xformslib:library:flimage_add_marker}
    \index{xformslib \textit{(package)}!xformslib.library \textit{(module)}!xformslib.library.flimage\_add\_marker \textit{(function)}}

    \vspace{0.5ex}

\hspace{.8\funcindent}\begin{boxedminipage}{\funcwidth}

    \raggedright \textbf{flimage\_add\_marker}(\textit{pImage}, \textit{text}, \textit{p3}, \textit{p4}, \textit{p5}, \textit{p6}, \textit{p7}, \textit{p8}, \textit{p9}, \textit{colr}, \textit{bcolr})

    \vspace{-1.5ex}

    \rule{\textwidth}{0.5\fboxrule}
\setlength{\parskip}{2ex}
\setlength{\parskip}{1ex}
      \textbf{Parameters}
      \vspace{-1ex}

      \begin{quote}
        \begin{Ventry}{xxxxxx}

          \item[pImage]

          pointer to image

        \end{Ventry}

      \end{quote}

      \textbf{Return Value}
    \vspace{-1ex}

      \begin{quote}
      num

      \end{quote}

\textbf{Status:} Untested + NoDoc + NoDemo = NOT OK



    \end{boxedminipage}

    \label{xformslib:library:flimage_add_marker_struct}
    \index{xformslib \textit{(package)}!xformslib.library \textit{(module)}!xformslib.library.flimage\_add\_marker\_struct \textit{(function)}}

    \vspace{0.5ex}

\hspace{.8\funcindent}\begin{boxedminipage}{\funcwidth}

    \raggedright \textbf{flimage\_add\_marker\_struct}(\textit{pImage}, \textit{pImageMarker})

    \vspace{-1.5ex}

    \rule{\textwidth}{0.5\fboxrule}
\setlength{\parskip}{2ex}
\setlength{\parskip}{1ex}
      \textbf{Parameters}
      \vspace{-1ex}

      \begin{quote}
        \begin{Ventry}{xxxxxx}

          \item[pImage]

          pointer to image

        \end{Ventry}

      \end{quote}

      \textbf{Return Value}
    \vspace{-1ex}

      \begin{quote}
      num

      \end{quote}

\textbf{Status:} Untested + NoDoc + NoDemo = NOT OK



    \end{boxedminipage}

    \label{xformslib:library:flimage_define_marker}
    \index{xformslib \textit{(package)}!xformslib.library \textit{(module)}!xformslib.library.flimage\_define\_marker \textit{(function)}}

    \vspace{0.5ex}

\hspace{.8\funcindent}\begin{boxedminipage}{\funcwidth}

    \raggedright \textbf{flimage\_define\_marker}(\textit{text1}, \textit{pImageMarker}, \textit{text2})

    \vspace{-1.5ex}

    \rule{\textwidth}{0.5\fboxrule}
\setlength{\parskip}{2ex}
\setlength{\parskip}{1ex}
      \textbf{Return Value}
    \vspace{-1ex}

      \begin{quote}
      num

      \end{quote}

\textbf{Status:} Untested + NoDoc + NoDemo = NOT OK



    \end{boxedminipage}

    \label{xformslib:library:flimage_delete_all_markers}
    \index{xformslib \textit{(package)}!xformslib.library \textit{(module)}!xformslib.library.flimage\_delete\_all\_markers \textit{(function)}}

    \vspace{0.5ex}

\hspace{.8\funcindent}\begin{boxedminipage}{\funcwidth}

    \raggedright \textbf{flimage\_delete\_all\_markers}(\textit{pImage})

    \vspace{-1.5ex}

    \rule{\textwidth}{0.5\fboxrule}
\setlength{\parskip}{2ex}
\setlength{\parskip}{1ex}
      \textbf{Parameters}
      \vspace{-1ex}

      \begin{quote}
        \begin{Ventry}{xxxxxx}

          \item[pImage]

          pointer to image

        \end{Ventry}

      \end{quote}

\textbf{Status:} Untested + NoDoc + NoDemo = NOT OK



    \end{boxedminipage}

    \label{xformslib:library:flimage_render_annotation}
    \index{xformslib \textit{(package)}!xformslib.library \textit{(module)}!xformslib.library.flimage\_render\_annotation \textit{(function)}}

    \vspace{0.5ex}

\hspace{.8\funcindent}\begin{boxedminipage}{\funcwidth}

    \raggedright \textbf{flimage\_render\_annotation}(\textit{pImage}, \textit{win})

    \vspace{-1.5ex}

    \rule{\textwidth}{0.5\fboxrule}
\setlength{\parskip}{2ex}
\setlength{\parskip}{1ex}
      \textbf{Parameters}
      \vspace{-1ex}

      \begin{quote}
        \begin{Ventry}{xxxxxx}

          \item[pImage]

          pointer to image

          \item[win]

          window

        \end{Ventry}

      \end{quote}

      \textbf{Return Value}
    \vspace{-1ex}

      \begin{quote}
      num

      \end{quote}

\textbf{Status:} Untested + NoDoc + NoDemo = NOT OK



    \end{boxedminipage}

    \label{xformslib:library:flimage_error}
    \index{xformslib \textit{(package)}!xformslib.library \textit{(module)}!xformslib.library.flimage\_error \textit{(function)}}

    \vspace{0.5ex}

\hspace{.8\funcindent}\begin{boxedminipage}{\funcwidth}

    \raggedright \textbf{flimage\_error}(\textit{pImage}, \textit{text})

    \vspace{-1.5ex}

    \rule{\textwidth}{0.5\fboxrule}
\setlength{\parskip}{2ex}
\setlength{\parskip}{1ex}
      \textbf{Parameters}
      \vspace{-1ex}

      \begin{quote}
        \begin{Ventry}{xxxxxx}

          \item[pImage]

          pointer to image

        \end{Ventry}

      \end{quote}

\textbf{Status:} Untested + NoDoc + NoDemo = NOT OK



    \end{boxedminipage}

    \label{xformslib:library:flimage_enable_pnm}
    \index{xformslib \textit{(package)}!xformslib.library \textit{(module)}!xformslib.library.flimage\_enable\_pnm \textit{(function)}}

    \vspace{0.5ex}

\hspace{.8\funcindent}\begin{boxedminipage}{\funcwidth}

    \raggedright \textbf{flimage\_enable\_pnm}()

    \vspace{-1.5ex}

    \rule{\textwidth}{0.5\fboxrule}
\setlength{\parskip}{2ex}
    Enables use of PNM (Portable anymap) image format

\setlength{\parskip}{1ex}
\textbf{Example:} flimage\_enable\_pnm()



\textbf{Status:} Tested + Doc + NoDemo = OK



    \end{boxedminipage}

    \label{xformslib:library:flimage_set_fits_bits}
    \index{xformslib \textit{(package)}!xformslib.library \textit{(module)}!xformslib.library.flimage\_set\_fits\_bits \textit{(function)}}

    \vspace{0.5ex}

\hspace{.8\funcindent}\begin{boxedminipage}{\funcwidth}

    \raggedright \textbf{flimage\_set\_fits\_bits}(\textit{p1})

    \vspace{-1.5ex}

    \rule{\textwidth}{0.5\fboxrule}
\setlength{\parskip}{2ex}
\setlength{\parskip}{1ex}
      \textbf{Return Value}
    \vspace{-1ex}

      \begin{quote}
      num

      \end{quote}

\textbf{Status:} Untested + NoDoc + NoDemo = NOT OK



    \end{boxedminipage}

    \label{xformslib:library:flimage_jpeg_output_options}
    \index{xformslib \textit{(package)}!xformslib.library \textit{(module)}!xformslib.library.flimage\_jpeg\_output\_options \textit{(function)}}

    \vspace{0.5ex}

\hspace{.8\funcindent}\begin{boxedminipage}{\funcwidth}

    \raggedright \textbf{flimage\_jpeg\_output\_options}(\textit{pImageJpegOption})

    \vspace{-1.5ex}

    \rule{\textwidth}{0.5\fboxrule}
\setlength{\parskip}{2ex}
\setlength{\parskip}{1ex}
\textbf{Status:} Untested + NoDoc + NoDemo = NOT OK



    \end{boxedminipage}

    \label{xformslib:library:flimage_pnm_output_options}
    \index{xformslib \textit{(package)}!xformslib.library \textit{(module)}!xformslib.library.flimage\_pnm\_output\_options \textit{(function)}}

    \vspace{0.5ex}

\hspace{.8\funcindent}\begin{boxedminipage}{\funcwidth}

    \raggedright \textbf{flimage\_pnm\_output\_options}(\textit{p1})

    \vspace{-1.5ex}

    \rule{\textwidth}{0.5\fboxrule}
\setlength{\parskip}{2ex}
\setlength{\parskip}{1ex}
\textbf{Status:} Untested + NoDoc + NoDemo = NOT OK



    \end{boxedminipage}

    \label{xformslib:library:flimage_gif_output_options}
    \index{xformslib \textit{(package)}!xformslib.library \textit{(module)}!xformslib.library.flimage\_gif\_output\_options \textit{(function)}}

    \vspace{0.5ex}

\hspace{.8\funcindent}\begin{boxedminipage}{\funcwidth}

    \raggedright \textbf{flimage\_gif\_output\_options}(\textit{p1})

    \vspace{-1.5ex}

    \rule{\textwidth}{0.5\fboxrule}
\setlength{\parskip}{2ex}
\setlength{\parskip}{1ex}
\textbf{Status:} Untested + NoDoc + NoDemo = NOT OK



    \end{boxedminipage}

    \label{xformslib:library:flimage_ps_options}
    \index{xformslib \textit{(package)}!xformslib.library \textit{(module)}!xformslib.library.flimage\_ps\_options \textit{(function)}}

    \vspace{0.5ex}

\hspace{.8\funcindent}\begin{boxedminipage}{\funcwidth}

    \raggedright \textbf{flimage\_ps\_options}()

    \vspace{-1.5ex}

    \rule{\textwidth}{0.5\fboxrule}
\setlength{\parskip}{2ex}
\setlength{\parskip}{1ex}
      \textbf{Return Value}
    \vspace{-1ex}

      \begin{quote}
      pFlpsControl

      \end{quote}

\textbf{Status:} Untested + NoDoc + NoDemo = NOT OK



    \end{boxedminipage}

    \label{xformslib:library:flimage_jpeg_output_options}
    \index{xformslib \textit{(package)}!xformslib.library \textit{(module)}!xformslib.library.flimage\_jpeg\_output\_options \textit{(function)}}

    \vspace{0.5ex}

\hspace{.8\funcindent}\begin{boxedminipage}{\funcwidth}

    \raggedright \textbf{flimage\_jpeg\_options}(\textit{pImageJpegOption})

    \vspace{-1.5ex}

    \rule{\textwidth}{0.5\fboxrule}
\setlength{\parskip}{2ex}
\setlength{\parskip}{1ex}
\textbf{Status:} Untested + NoDoc + NoDemo = NOT OK



    \end{boxedminipage}

    \label{xformslib:library:flimage_pnm_output_options}
    \index{xformslib \textit{(package)}!xformslib.library \textit{(module)}!xformslib.library.flimage\_pnm\_output\_options \textit{(function)}}

    \vspace{0.5ex}

\hspace{.8\funcindent}\begin{boxedminipage}{\funcwidth}

    \raggedright \textbf{flimage\_pnm\_options}(\textit{p1})

    \vspace{-1.5ex}

    \rule{\textwidth}{0.5\fboxrule}
\setlength{\parskip}{2ex}
\setlength{\parskip}{1ex}
\textbf{Status:} Untested + NoDoc + NoDemo = NOT OK



    \end{boxedminipage}

    \label{xformslib:library:flimage_gif_output_options}
    \index{xformslib \textit{(package)}!xformslib.library \textit{(module)}!xformslib.library.flimage\_gif\_output\_options \textit{(function)}}

    \vspace{0.5ex}

\hspace{.8\funcindent}\begin{boxedminipage}{\funcwidth}

    \raggedright \textbf{flimage\_gif\_options}(\textit{p1})

    \vspace{-1.5ex}

    \rule{\textwidth}{0.5\fboxrule}
\setlength{\parskip}{2ex}
\setlength{\parskip}{1ex}
\textbf{Status:} Untested + NoDoc + NoDemo = NOT OK



    \end{boxedminipage}

    \label{xformslib:library:flimage_get_number_of_formats}
    \index{xformslib \textit{(package)}!xformslib.library \textit{(module)}!xformslib.library.flimage\_get\_number\_of\_formats \textit{(function)}}

    \vspace{0.5ex}

\hspace{.8\funcindent}\begin{boxedminipage}{\funcwidth}

    \raggedright \textbf{flimage\_get\_number\_of\_formats}()

    \vspace{-1.5ex}

    \rule{\textwidth}{0.5\fboxrule}
\setlength{\parskip}{2ex}
\setlength{\parskip}{1ex}
      \textbf{Return Value}
    \vspace{-1ex}

      \begin{quote}
      num

      \end{quote}

\textbf{Status:} Untested + NoDoc + NoDemo = NOT OK



    \end{boxedminipage}

    \label{xformslib:library:flimage_get_format_info}
    \index{xformslib \textit{(package)}!xformslib.library \textit{(module)}!xformslib.library.flimage\_get\_format\_info \textit{(function)}}

    \vspace{0.5ex}

\hspace{.8\funcindent}\begin{boxedminipage}{\funcwidth}

    \raggedright \textbf{flimage\_get\_format\_info}(\textit{p1})

    \vspace{-1.5ex}

    \rule{\textwidth}{0.5\fboxrule}
\setlength{\parskip}{2ex}
\setlength{\parskip}{1ex}
      \textbf{Return Value}
    \vspace{-1ex}

      \begin{quote}
      ImageFormatInfo class instance

      \end{quote}

\textbf{Status:} Untested + NoDoc + NoDemo = NOT OK



    \end{boxedminipage}

    \label{xformslib:library:fl_get_matrix}
    \index{xformslib \textit{(package)}!xformslib.library \textit{(module)}!xformslib.library.fl\_get\_matrix \textit{(function)}}

    \vspace{0.5ex}

\hspace{.8\funcindent}\begin{boxedminipage}{\funcwidth}

    \raggedright \textbf{fl\_get\_matrix}(\textit{nrows}, \textit{ncols}, \textit{esize})

    \vspace{-1.5ex}

    \rule{\textwidth}{0.5\fboxrule}
\setlength{\parskip}{2ex}
\setlength{\parskip}{1ex}
      \textbf{Parameters}
      \vspace{-1ex}

      \begin{quote}
        \begin{Ventry}{xxxxx}

          \item[nrows]

          number of rows

          \item[ncols]

          number of columns

          \item[esize]

          size of matrix

        \end{Ventry}

      \end{quote}

      \textbf{Return Value}
    \vspace{-1ex}

      \begin{quote}
      ?

      \end{quote}

\textbf{Status:} Untested + NoDoc + NoDemo = NOT OK



    \end{boxedminipage}

    \label{xformslib:library:fl_make_matrix}
    \index{xformslib \textit{(package)}!xformslib.library \textit{(module)}!xformslib.library.fl\_make\_matrix \textit{(function)}}

    \vspace{0.5ex}

\hspace{.8\funcindent}\begin{boxedminipage}{\funcwidth}

    \raggedright \textbf{fl\_make\_matrix}(\textit{nrows}, \textit{ncols}, \textit{esize}, \textit{mem})

    \vspace{-1.5ex}

    \rule{\textwidth}{0.5\fboxrule}
\setlength{\parskip}{2ex}
    Makes a matrix out of a given piece of memory.

\setlength{\parskip}{1ex}
      \textbf{Parameters}
      \vspace{-1ex}

      \begin{quote}
        \begin{Ventry}{xxxxx}

          \item[nrows]

          number of rows

          \item[ncols]

          number of columns

          \item[esize]

          size of matrix

          \item[mem]

          memory

        \end{Ventry}

      \end{quote}

      \textbf{Return Value}
    \vspace{-1ex}

      \begin{quote}
      ?

      \end{quote}

\textbf{Status:} Untested + NoDoc + NoDemo = NOT OK



    \end{boxedminipage}

    \label{xformslib:library:fl_free_matrix}
    \index{xformslib \textit{(package)}!xformslib.library \textit{(module)}!xformslib.library.fl\_free\_matrix \textit{(function)}}

    \vspace{0.5ex}

\hspace{.8\funcindent}\begin{boxedminipage}{\funcwidth}

    \raggedright \textbf{fl\_free\_matrix}(\textit{mtrx})

    \vspace{-1.5ex}

    \rule{\textwidth}{0.5\fboxrule}
\setlength{\parskip}{2ex}
\setlength{\parskip}{1ex}
\textbf{Status:} Untested + NoDoc + NoDemo = NOT OK



    \end{boxedminipage}

    \label{xformslib:library:fl_lookup_RGBcolor}
    \index{xformslib \textit{(package)}!xformslib.library \textit{(module)}!xformslib.library.fl\_lookup\_RGBcolor \textit{(function)}}

    \vspace{0.5ex}

\hspace{.8\funcindent}\begin{boxedminipage}{\funcwidth}

    \raggedright \textbf{fl\_lookup\_RGBcolor}(\textit{text}, \textit{p2}, \textit{p3}, \textit{p4})

    \vspace{-1.5ex}

    \rule{\textwidth}{0.5\fboxrule}
\setlength{\parskip}{2ex}
\setlength{\parskip}{1ex}
      \textbf{Return Value}
    \vspace{-1ex}

      \begin{quote}
      num

      \end{quote}

\textbf{Status:} Untested + NoDoc + NoDemo = NOT OK



    \end{boxedminipage}

    \label{xformslib:library:flimage_add_format}
    \index{xformslib \textit{(package)}!xformslib.library \textit{(module)}!xformslib.library.flimage\_add\_format \textit{(function)}}

    \vspace{0.5ex}

\hspace{.8\funcindent}\begin{boxedminipage}{\funcwidth}

    \raggedright \textbf{flimage\_add\_format}(\textit{formalname}, \textit{shortname}, \textit{extension}, \textit{flimagetype}, \textit{py\_ImageIdentify}, \textit{py\_ImageDescription}, \textit{py\_ImageReadPixels}, \textit{py\_ImageWriteImage})

    \vspace{-1.5ex}

    \rule{\textwidth}{0.5\fboxrule}
\setlength{\parskip}{2ex}
\setlength{\parskip}{1ex}
      \textbf{Return Value}
    \vspace{-1ex}

      \begin{quote}
      num

      \end{quote}

\textbf{Status:} Untested + NoDoc + NoDemo = NOT OK



    \end{boxedminipage}

    \label{xformslib:library:flimage_set_annotation_support}
    \index{xformslib \textit{(package)}!xformslib.library \textit{(module)}!xformslib.library.flimage\_set\_annotation\_support \textit{(function)}}

    \vspace{0.5ex}

\hspace{.8\funcindent}\begin{boxedminipage}{\funcwidth}

    \raggedright \textbf{flimage\_set\_annotation\_support}(\textit{p1}, \textit{p2})

    \vspace{-1.5ex}

    \rule{\textwidth}{0.5\fboxrule}
\setlength{\parskip}{2ex}
\setlength{\parskip}{1ex}
\textbf{Status:} Untested + NoDoc + NoDemo = NOT OK



    \end{boxedminipage}

    \label{xformslib:library:flimage_getcolormap}
    \index{xformslib \textit{(package)}!xformslib.library \textit{(module)}!xformslib.library.flimage\_getcolormap \textit{(function)}}

    \vspace{0.5ex}

\hspace{.8\funcindent}\begin{boxedminipage}{\funcwidth}

    \raggedright \textbf{flimage\_getcolormap}(\textit{pImage})

    \vspace{-1.5ex}

    \rule{\textwidth}{0.5\fboxrule}
\setlength{\parskip}{2ex}
\setlength{\parskip}{1ex}
      \textbf{Parameters}
      \vspace{-1ex}

      \begin{quote}
        \begin{Ventry}{xxxxxx}

          \item[pImage]

          pointer to image

        \end{Ventry}

      \end{quote}

      \textbf{Return Value}
    \vspace{-1ex}

      \begin{quote}
      num

      \end{quote}

\textbf{Status:} Untested + NoDoc + NoDemo = NOT OK



    \end{boxedminipage}

    \label{xformslib:library:fl_select_mediancut_quantizer}
    \index{xformslib \textit{(package)}!xformslib.library \textit{(module)}!xformslib.library.fl\_select\_mediancut\_quantizer \textit{(function)}}

    \vspace{0.5ex}

\hspace{.8\funcindent}\begin{boxedminipage}{\funcwidth}

    \raggedright \textbf{fl\_select\_mediancut\_quantizer}()

    \vspace{-1.5ex}

    \rule{\textwidth}{0.5\fboxrule}
\setlength{\parskip}{2ex}
\setlength{\parskip}{1ex}
\textbf{Status:} Untested + NoDoc + NoDemo = NOT OK



    \end{boxedminipage}

    \label{xformslib:library:flimage_convolve}
    \index{xformslib \textit{(package)}!xformslib.library \textit{(module)}!xformslib.library.flimage\_convolve \textit{(function)}}

    \vspace{0.5ex}

\hspace{.8\funcindent}\begin{boxedminipage}{\funcwidth}

    \raggedright \textbf{flimage\_convolve}(\textit{pImage}, \textit{p2}, \textit{p3}, \textit{p4})

    \vspace{-1.5ex}

    \rule{\textwidth}{0.5\fboxrule}
\setlength{\parskip}{2ex}
\setlength{\parskip}{1ex}
      \textbf{Parameters}
      \vspace{-1ex}

      \begin{quote}
        \begin{Ventry}{xxxxxx}

          \item[pImage]

          pointer to image

        \end{Ventry}

      \end{quote}

      \textbf{Return Value}
    \vspace{-1ex}

      \begin{quote}
      num

      \end{quote}

\textbf{Status:} Untested + NoDoc + NoDemo = NOT OK



    \end{boxedminipage}

    \label{xformslib:library:flimage_convolvea}
    \index{xformslib \textit{(package)}!xformslib.library \textit{(module)}!xformslib.library.flimage\_convolvea \textit{(function)}}

    \vspace{0.5ex}

\hspace{.8\funcindent}\begin{boxedminipage}{\funcwidth}

    \raggedright \textbf{flimage\_convolvea}(\textit{pImage}, \textit{p2}, \textit{p3}, \textit{p4})

    \vspace{-1.5ex}

    \rule{\textwidth}{0.5\fboxrule}
\setlength{\parskip}{2ex}
\setlength{\parskip}{1ex}
      \textbf{Parameters}
      \vspace{-1ex}

      \begin{quote}
        \begin{Ventry}{xxxxxx}

          \item[pImage]

          pointer to image

        \end{Ventry}

      \end{quote}

      \textbf{Return Value}
    \vspace{-1ex}

      \begin{quote}
      num

      \end{quote}

\textbf{Status:} Untested + NoDoc + NoDemo = NOT OK



    \end{boxedminipage}

    \label{xformslib:library:flimage_tint}
    \index{xformslib \textit{(package)}!xformslib.library \textit{(module)}!xformslib.library.flimage\_tint \textit{(function)}}

    \vspace{0.5ex}

\hspace{.8\funcindent}\begin{boxedminipage}{\funcwidth}

    \raggedright \textbf{flimage\_tint}(\textit{pImage}, \textit{p2}, \textit{p3})

    \vspace{-1.5ex}

    \rule{\textwidth}{0.5\fboxrule}
\setlength{\parskip}{2ex}
\setlength{\parskip}{1ex}
      \textbf{Parameters}
      \vspace{-1ex}

      \begin{quote}
        \begin{Ventry}{xxxxxx}

          \item[pImage]

          pointer to image

        \end{Ventry}

      \end{quote}

      \textbf{Return Value}
    \vspace{-1ex}

      \begin{quote}
      num

      \end{quote}

\textbf{Status:} Untested + NoDoc + NoDemo = NOT OK



    \end{boxedminipage}

    \label{xformslib:library:flimage_rotate}
    \index{xformslib \textit{(package)}!xformslib.library \textit{(module)}!xformslib.library.flimage\_rotate \textit{(function)}}

    \vspace{0.5ex}

\hspace{.8\funcindent}\begin{boxedminipage}{\funcwidth}

    \raggedright \textbf{flimage\_rotate}(\textit{pImage}, \textit{p2}, \textit{p3})

    \vspace{-1.5ex}

    \rule{\textwidth}{0.5\fboxrule}
\setlength{\parskip}{2ex}
\setlength{\parskip}{1ex}
      \textbf{Parameters}
      \vspace{-1ex}

      \begin{quote}
        \begin{Ventry}{xxxxxx}

          \item[pImage]

          pointer to image

        \end{Ventry}

      \end{quote}

      \textbf{Return Value}
    \vspace{-1ex}

      \begin{quote}
      num

      \end{quote}

\textbf{Status:} Untested + NoDoc + NoDemo = NOT OK



    \end{boxedminipage}

    \label{xformslib:library:flimage_flip}
    \index{xformslib \textit{(package)}!xformslib.library \textit{(module)}!xformslib.library.flimage\_flip \textit{(function)}}

    \vspace{0.5ex}

\hspace{.8\funcindent}\begin{boxedminipage}{\funcwidth}

    \raggedright \textbf{flimage\_flip}(\textit{pImage}, \textit{p2})

    \vspace{-1.5ex}

    \rule{\textwidth}{0.5\fboxrule}
\setlength{\parskip}{2ex}
\setlength{\parskip}{1ex}
      \textbf{Parameters}
      \vspace{-1ex}

      \begin{quote}
        \begin{Ventry}{xxxxxx}

          \item[pImage]

          pointer to image

        \end{Ventry}

      \end{quote}

      \textbf{Return Value}
    \vspace{-1ex}

      \begin{quote}
      num

      \end{quote}

\textbf{Status:} Untested + NoDoc + NoDemo = NOT OK



    \end{boxedminipage}

    \label{xformslib:library:flimage_scale}
    \index{xformslib \textit{(package)}!xformslib.library \textit{(module)}!xformslib.library.flimage\_scale \textit{(function)}}

    \vspace{0.5ex}

\hspace{.8\funcindent}\begin{boxedminipage}{\funcwidth}

    \raggedright \textbf{flimage\_scale}(\textit{pImage}, \textit{p2}, \textit{p3}, \textit{p4})

    \vspace{-1.5ex}

    \rule{\textwidth}{0.5\fboxrule}
\setlength{\parskip}{2ex}
\setlength{\parskip}{1ex}
      \textbf{Parameters}
      \vspace{-1ex}

      \begin{quote}
        \begin{Ventry}{xxxxxx}

          \item[pImage]

          pointer to image

        \end{Ventry}

      \end{quote}

      \textbf{Return Value}
    \vspace{-1ex}

      \begin{quote}
      num

      \end{quote}

\textbf{Status:} Untested + NoDoc + NoDemo = NOT OK



    \end{boxedminipage}

    \label{xformslib:library:flimage_warp}
    \index{xformslib \textit{(package)}!xformslib.library \textit{(module)}!xformslib.library.flimage\_warp \textit{(function)}}

    \vspace{0.5ex}

\hspace{.8\funcindent}\begin{boxedminipage}{\funcwidth}

    \raggedright \textbf{flimage\_warp}(\textit{pImage}, \textit{p2}, \textit{p3}, \textit{p4}, \textit{p5})

    \vspace{-1.5ex}

    \rule{\textwidth}{0.5\fboxrule}
\setlength{\parskip}{2ex}
\setlength{\parskip}{1ex}
      \textbf{Parameters}
      \vspace{-1ex}

      \begin{quote}
        \begin{Ventry}{xxxxxx}

          \item[pImage]

          pointer to image

        \end{Ventry}

      \end{quote}

      \textbf{Return Value}
    \vspace{-1ex}

      \begin{quote}
      num

      \end{quote}

\textbf{Status:} Untested + NoDoc + NoDemo = NOT OK



    \end{boxedminipage}

    \label{xformslib:library:flimage_autocrop}
    \index{xformslib \textit{(package)}!xformslib.library \textit{(module)}!xformslib.library.flimage\_autocrop \textit{(function)}}

    \vspace{0.5ex}

\hspace{.8\funcindent}\begin{boxedminipage}{\funcwidth}

    \raggedright \textbf{flimage\_autocrop}(\textit{pImage}, \textit{p2})

    \vspace{-1.5ex}

    \rule{\textwidth}{0.5\fboxrule}
\setlength{\parskip}{2ex}
\setlength{\parskip}{1ex}
      \textbf{Parameters}
      \vspace{-1ex}

      \begin{quote}
        \begin{Ventry}{xxxxxx}

          \item[pImage]

          pointer to image

        \end{Ventry}

      \end{quote}

      \textbf{Return Value}
    \vspace{-1ex}

      \begin{quote}
      num

      \end{quote}

\textbf{Status:} Untested + NoDoc + NoDemo = NOT OK



    \end{boxedminipage}

    \label{xformslib:library:flimage_get_autocrop}
    \index{xformslib \textit{(package)}!xformslib.library \textit{(module)}!xformslib.library.flimage\_get\_autocrop \textit{(function)}}

    \vspace{0.5ex}

\hspace{.8\funcindent}\begin{boxedminipage}{\funcwidth}

    \raggedright \textbf{flimage\_get\_autocrop}(\textit{pImage}, \textit{bk})

    \vspace{-1.5ex}

    \rule{\textwidth}{0.5\fboxrule}
\setlength{\parskip}{2ex}
\setlength{\parskip}{1ex}
      \textbf{Parameters}
      \vspace{-1ex}

      \begin{quote}
        \begin{Ventry}{xxxxxx}

          \item[pImage]

          pointer to image

        \end{Ventry}

      \end{quote}

      \textbf{Return Value}
    \vspace{-1ex}

      \begin{quote}
      num., xl, yt, xr, yb

      \end{quote}

\textbf{Attention:} API change from XForms - upstream was flimage\_get\_autocrop(pImage, bk, 
xl, yt, xr, yb)



\textbf{Status:} Untested + NoDoc + NoDemo = NOT OK



    \end{boxedminipage}

    \label{xformslib:library:flimage_crop}
    \index{xformslib \textit{(package)}!xformslib.library \textit{(module)}!xformslib.library.flimage\_crop \textit{(function)}}

    \vspace{0.5ex}

\hspace{.8\funcindent}\begin{boxedminipage}{\funcwidth}

    \raggedright \textbf{flimage\_crop}(\textit{pImage}, \textit{p2}, \textit{p3}, \textit{p4}, \textit{p5})

    \vspace{-1.5ex}

    \rule{\textwidth}{0.5\fboxrule}
\setlength{\parskip}{2ex}
\setlength{\parskip}{1ex}
      \textbf{Parameters}
      \vspace{-1ex}

      \begin{quote}
        \begin{Ventry}{xxxxxx}

          \item[pImage]

          pointer to image

        \end{Ventry}

      \end{quote}

      \textbf{Return Value}
    \vspace{-1ex}

      \begin{quote}
      num

      \end{quote}

\textbf{Status:} Untested + NoDoc + NoDemo = NOT OK



    \end{boxedminipage}

    \label{xformslib:library:flimage_replace_pixel}
    \index{xformslib \textit{(package)}!xformslib.library \textit{(module)}!xformslib.library.flimage\_replace\_pixel \textit{(function)}}

    \vspace{0.5ex}

\hspace{.8\funcindent}\begin{boxedminipage}{\funcwidth}

    \raggedright \textbf{flimage\_replace\_pixel}(\textit{pImage}, \textit{p2}, \textit{p3})

    \vspace{-1.5ex}

    \rule{\textwidth}{0.5\fboxrule}
\setlength{\parskip}{2ex}
\setlength{\parskip}{1ex}
      \textbf{Parameters}
      \vspace{-1ex}

      \begin{quote}
        \begin{Ventry}{xxxxxx}

          \item[pImage]

          pointer to image

        \end{Ventry}

      \end{quote}

      \textbf{Return Value}
    \vspace{-1ex}

      \begin{quote}
      num

      \end{quote}

\textbf{Status:} Untested + NoDoc + NoDemo = NOT OK



    \end{boxedminipage}

    \label{xformslib:library:flimage_transform_pixels}
    \index{xformslib \textit{(package)}!xformslib.library \textit{(module)}!xformslib.library.flimage\_transform\_pixels \textit{(function)}}

    \vspace{0.5ex}

\hspace{.8\funcindent}\begin{boxedminipage}{\funcwidth}

    \raggedright \textbf{flimage\_transform\_pixels}(\textit{pImage}, \textit{red}, \textit{green}, \textit{blue})

    \vspace{-1.5ex}

    \rule{\textwidth}{0.5\fboxrule}
\setlength{\parskip}{2ex}
\setlength{\parskip}{1ex}
      \textbf{Parameters}
      \vspace{-1ex}

      \begin{quote}
        \begin{Ventry}{xxxxxx}

          \item[pImage]

          pointer to image

        \end{Ventry}

      \end{quote}

      \textbf{Return Value}
    \vspace{-1ex}

      \begin{quote}
      num

      \end{quote}

\textbf{Status:} Untested + NoDoc + NoDemo = NOT OK



    \end{boxedminipage}

    \label{xformslib:library:flimage_windowlevel}
    \index{xformslib \textit{(package)}!xformslib.library \textit{(module)}!xformslib.library.flimage\_windowlevel \textit{(function)}}

    \vspace{0.5ex}

\hspace{.8\funcindent}\begin{boxedminipage}{\funcwidth}

    \raggedright \textbf{flimage\_windowlevel}(\textit{pImage}, \textit{p2}, \textit{p3})

    \vspace{-1.5ex}

    \rule{\textwidth}{0.5\fboxrule}
\setlength{\parskip}{2ex}
\setlength{\parskip}{1ex}
      \textbf{Parameters}
      \vspace{-1ex}

      \begin{quote}
        \begin{Ventry}{xxxxxx}

          \item[pImage]

          pointer to image

        \end{Ventry}

      \end{quote}

      \textbf{Return Value}
    \vspace{-1ex}

      \begin{quote}
      num

      \end{quote}

\textbf{Status:} Untested + NoDoc + NoDemo = NOT OK



    \end{boxedminipage}

    \label{xformslib:library:flimage_enhance}
    \index{xformslib \textit{(package)}!xformslib.library \textit{(module)}!xformslib.library.flimage\_enhance \textit{(function)}}

    \vspace{0.5ex}

\hspace{.8\funcindent}\begin{boxedminipage}{\funcwidth}

    \raggedright \textbf{flimage\_enhance}(\textit{pImage}, \textit{p2})

    \vspace{-1.5ex}

    \rule{\textwidth}{0.5\fboxrule}
\setlength{\parskip}{2ex}
\setlength{\parskip}{1ex}
      \textbf{Parameters}
      \vspace{-1ex}

      \begin{quote}
        \begin{Ventry}{xxxxxx}

          \item[pImage]

          pointer to image

        \end{Ventry}

      \end{quote}

      \textbf{Return Value}
    \vspace{-1ex}

      \begin{quote}
      num

      \end{quote}

\textbf{Status:} Untested + NoDoc + NoDemo = NOT OK



    \end{boxedminipage}

    \label{xformslib:library:flimage_from_pixmap}
    \index{xformslib \textit{(package)}!xformslib.library \textit{(module)}!xformslib.library.flimage\_from\_pixmap \textit{(function)}}

    \vspace{0.5ex}

\hspace{.8\funcindent}\begin{boxedminipage}{\funcwidth}

    \raggedright \textbf{flimage\_from\_pixmap}(\textit{pImage}, \textit{pixmap})

    \vspace{-1.5ex}

    \rule{\textwidth}{0.5\fboxrule}
\setlength{\parskip}{2ex}
\setlength{\parskip}{1ex}
      \textbf{Parameters}
      \vspace{-1ex}

      \begin{quote}
        \begin{Ventry}{xxxxxx}

          \item[pImage]

          pointer to image

          \item[pixmap]

          pixmap value

        \end{Ventry}

      \end{quote}

      \textbf{Return Value}
    \vspace{-1ex}

      \begin{quote}
      num

      \end{quote}

\textbf{Status:} Untested + NoDoc + NoDemo = NOT OK



    \end{boxedminipage}

    \label{xformslib:library:flimage_to_pixmap}
    \index{xformslib \textit{(package)}!xformslib.library \textit{(module)}!xformslib.library.flimage\_to\_pixmap \textit{(function)}}

    \vspace{0.5ex}

\hspace{.8\funcindent}\begin{boxedminipage}{\funcwidth}

    \raggedright \textbf{flimage\_to\_pixmap}(\textit{pImage}, \textit{win})

    \vspace{-1.5ex}

    \rule{\textwidth}{0.5\fboxrule}
\setlength{\parskip}{2ex}
\setlength{\parskip}{1ex}
      \textbf{Parameters}
      \vspace{-1ex}

      \begin{quote}
        \begin{Ventry}{xxxxxx}

          \item[pImage]

          pointer to image

          \item[win]

          window id

        \end{Ventry}

      \end{quote}

      \textbf{Return Value}
    \vspace{-1ex}

      \begin{quote}
      pixmap

      \end{quote}

\textbf{Status:} Untested + NoDoc + NoDemo = NOT OK



    \end{boxedminipage}

    \label{xformslib:library:flimage_dup}
    \index{xformslib \textit{(package)}!xformslib.library \textit{(module)}!xformslib.library.flimage\_dup \textit{(function)}}

    \vspace{0.5ex}

\hspace{.8\funcindent}\begin{boxedminipage}{\funcwidth}

    \raggedright \textbf{flimage\_dup}(\textit{pImage})

    \vspace{-1.5ex}

    \rule{\textwidth}{0.5\fboxrule}
\setlength{\parskip}{2ex}
\setlength{\parskip}{1ex}
      \textbf{Parameters}
      \vspace{-1ex}

      \begin{quote}
        \begin{Ventry}{xxxxxx}

          \item[pImage]

          pointer to image

        \end{Ventry}

      \end{quote}

      \textbf{Return Value}
    \vspace{-1ex}

      \begin{quote}
      pImage

      \end{quote}

\textbf{Status:} Untested + NoDoc + NoDemo = NOT OK



    \end{boxedminipage}

    \label{xformslib:library:fl_get_submatrix}
    \index{xformslib \textit{(package)}!xformslib.library \textit{(module)}!xformslib.library.fl\_get\_submatrix \textit{(function)}}

    \vspace{0.5ex}

\hspace{.8\funcindent}\begin{boxedminipage}{\funcwidth}

    \raggedright \textbf{fl\_get\_submatrix}(\textit{inmtrx}, \textit{rows}, \textit{cols}, \textit{r1}, \textit{c1}, \textit{rs}, \textit{cs}, \textit{esize})

    \vspace{-1.5ex}

    \rule{\textwidth}{0.5\fboxrule}
\setlength{\parskip}{2ex}
\setlength{\parskip}{1ex}
      \textbf{Return Value}
    \vspace{-1ex}

      \begin{quote}
      ?

      \end{quote}

\textbf{Status:} Untested + NoDoc + NoDemo = NOT OK



    \end{boxedminipage}

    \label{xformslib:library:fl_j2pass_quantize_packed}
    \index{xformslib \textit{(package)}!xformslib.library \textit{(module)}!xformslib.library.fl\_j2pass\_quantize\_packed \textit{(function)}}

    \vspace{0.5ex}

\hspace{.8\funcindent}\begin{boxedminipage}{\funcwidth}

    \raggedright \textbf{fl\_j2pass\_quantize\_packed}(\textit{p1}, \textit{p2}, \textit{p3}, \textit{p4}, \textit{p5}, \textit{p6}, \textit{p7}, \textit{p8}, \textit{p9}, \textit{pImage})

    \vspace{-1.5ex}

    \rule{\textwidth}{0.5\fboxrule}
\setlength{\parskip}{2ex}
\setlength{\parskip}{1ex}
      \textbf{Return Value}
    \vspace{-1ex}

      \begin{quote}
      num

      \end{quote}

\textbf{Status:} Untested + NoDoc + NoDemo = NOT OK



    \end{boxedminipage}

    \label{xformslib:library:fl_j2pass_quantize_rgb}
    \index{xformslib \textit{(package)}!xformslib.library \textit{(module)}!xformslib.library.fl\_j2pass\_quantize\_rgb \textit{(function)}}

    \vspace{0.5ex}

\hspace{.8\funcindent}\begin{boxedminipage}{\funcwidth}

    \raggedright \textbf{fl\_j2pass\_quantize\_rgb}(\textit{p1}, \textit{p2}, \textit{p3}, \textit{p4}, \textit{p5}, \textit{p6}, \textit{p7}, \textit{p8}, \textit{p9}, \textit{p10}, \textit{p11}, \textit{pImage})

    \vspace{-1.5ex}

    \rule{\textwidth}{0.5\fboxrule}
\setlength{\parskip}{2ex}
\setlength{\parskip}{1ex}
      \textbf{Return Value}
    \vspace{-1ex}

      \begin{quote}
      num

      \end{quote}

\textbf{Status:} Untested + NoDoc + NoDemo = NOT OK



    \end{boxedminipage}

    \label{xformslib:library:fl_make_submatrix}
    \index{xformslib \textit{(package)}!xformslib.library \textit{(module)}!xformslib.library.fl\_make\_submatrix \textit{(function)}}

    \vspace{0.5ex}

\hspace{.8\funcindent}\begin{boxedminipage}{\funcwidth}

    \raggedright \textbf{fl\_make\_submatrix}(\textit{in\_}, \textit{rows}, \textit{cols}, \textit{r1}, \textit{c1}, \textit{rs}, \textit{cs}, \textit{esize})

    \vspace{-1.5ex}

    \rule{\textwidth}{0.5\fboxrule}
\setlength{\parskip}{2ex}
\setlength{\parskip}{1ex}
      \textbf{Return Value}
    \vspace{-1ex}

      \begin{quote}
      ?

      \end{quote}

\textbf{Status:} Untested + NoDoc + NoDemo = NOT OK



    \end{boxedminipage}

    \label{xformslib:library:fl_pack_bits}
    \index{xformslib \textit{(package)}!xformslib.library \textit{(module)}!xformslib.library.fl\_pack\_bits \textit{(function)}}

    \vspace{0.5ex}

\hspace{.8\funcindent}\begin{boxedminipage}{\funcwidth}

    \raggedright \textbf{fl\_pack\_bits}(\textit{p1}, \textit{p2}, \textit{p3})

    \vspace{-1.5ex}

    \rule{\textwidth}{0.5\fboxrule}
\setlength{\parskip}{2ex}
\setlength{\parskip}{1ex}
\textbf{Status:} Untested + NoDoc + NoDemo = NOT OK



    \end{boxedminipage}

    \label{xformslib:library:fl_unpack_bits}
    \index{xformslib \textit{(package)}!xformslib.library \textit{(module)}!xformslib.library.fl\_unpack\_bits \textit{(function)}}

    \vspace{0.5ex}

\hspace{.8\funcindent}\begin{boxedminipage}{\funcwidth}

    \raggedright \textbf{fl\_unpack\_bits}(\textit{p1}, \textit{p2}, \textit{p3})

    \vspace{-1.5ex}

    \rule{\textwidth}{0.5\fboxrule}
\setlength{\parskip}{2ex}
\setlength{\parskip}{1ex}
\textbf{Status:} Untested + NoDoc + NoDemo = NOT OK



    \end{boxedminipage}

    \label{xformslib:library:fl_value_to_bits}
    \index{xformslib \textit{(package)}!xformslib.library \textit{(module)}!xformslib.library.fl\_value\_to\_bits \textit{(function)}}

    \vspace{0.5ex}

\hspace{.8\funcindent}\begin{boxedminipage}{\funcwidth}

    \raggedright \textbf{fl\_value\_to\_bits}(\textit{val})

    \vspace{-1.5ex}

    \rule{\textwidth}{0.5\fboxrule}
\setlength{\parskip}{2ex}
\setlength{\parskip}{1ex}
      \textbf{Parameters}
      \vspace{-1ex}

      \begin{quote}
        \begin{Ventry}{xxx}

          \item[val]

          value to convert to bits

        \end{Ventry}

      \end{quote}

      \textbf{Return Value}
    \vspace{-1ex}

      \begin{quote}
      num

      \end{quote}

\textbf{Status:} Untested + NoDoc + NoDemo = NOT OK



    \end{boxedminipage}

    \label{xformslib:library:flimage_add_comments}
    \index{xformslib \textit{(package)}!xformslib.library \textit{(module)}!xformslib.library.flimage\_add\_comments \textit{(function)}}

    \vspace{0.5ex}

\hspace{.8\funcindent}\begin{boxedminipage}{\funcwidth}

    \raggedright \textbf{flimage\_add\_comments}(\textit{pImage}, \textit{p2}, \textit{p3})

    \vspace{-1.5ex}

    \rule{\textwidth}{0.5\fboxrule}
\setlength{\parskip}{2ex}
\setlength{\parskip}{1ex}
      \textbf{Parameters}
      \vspace{-1ex}

      \begin{quote}
        \begin{Ventry}{xxxxxx}

          \item[pImage]

          pointer to image

        \end{Ventry}

      \end{quote}

\textbf{Status:} Untested + NoDoc + NoDemo = NOT OK



    \end{boxedminipage}

    \label{xformslib:library:flimage_color_to_pixel}
    \index{xformslib \textit{(package)}!xformslib.library \textit{(module)}!xformslib.library.flimage\_color\_to\_pixel \textit{(function)}}

    \vspace{0.5ex}

\hspace{.8\funcindent}\begin{boxedminipage}{\funcwidth}

    \raggedright \textbf{flimage\_color\_to\_pixel}(\textit{pImage}, \textit{p2}, \textit{p3}, \textit{p4}, \textit{p5})

    \vspace{-1.5ex}

    \rule{\textwidth}{0.5\fboxrule}
\setlength{\parskip}{2ex}
\setlength{\parskip}{1ex}
      \textbf{Parameters}
      \vspace{-1ex}

      \begin{quote}
        \begin{Ventry}{xxxxxx}

          \item[pImage]

          pointer to image

        \end{Ventry}

      \end{quote}

      \textbf{Return Value}
    \vspace{-1ex}

      \begin{quote}
      num

      \end{quote}

\textbf{Status:} Untested + NoDoc + NoDemo = NOT OK



    \end{boxedminipage}

    \label{xformslib:library:flimage_combine}
    \index{xformslib \textit{(package)}!xformslib.library \textit{(module)}!xformslib.library.flimage\_combine \textit{(function)}}

    \vspace{0.5ex}

\hspace{.8\funcindent}\begin{boxedminipage}{\funcwidth}

    \raggedright \textbf{flimage\_combine}(\textit{pImage1}, \textit{pImage2}, \textit{alpha})

    \vspace{-1.5ex}

    \rule{\textwidth}{0.5\fboxrule}
\setlength{\parskip}{2ex}
\setlength{\parskip}{1ex}
      \textbf{Parameters}
      \vspace{-1ex}

      \begin{quote}
        \begin{Ventry}{xxxxxxx}

          \item[pImage1]

          pointer to first image to combine

          \item[pImage2]

          pointer to second image to combine

          \item[alpha]

          alpha level?

        \end{Ventry}

      \end{quote}

      \textbf{Return Value}
    \vspace{-1ex}

      \begin{quote}
      pImage

      \end{quote}

\textbf{Status:} Untested + NoDoc + NoDemo = NOT OK



    \end{boxedminipage}

    \label{xformslib:library:flimage_display_markers}
    \index{xformslib \textit{(package)}!xformslib.library \textit{(module)}!xformslib.library.flimage\_display\_markers \textit{(function)}}

    \vspace{0.5ex}

\hspace{.8\funcindent}\begin{boxedminipage}{\funcwidth}

    \raggedright \textbf{flimage\_display\_markers}(\textit{pImage})

    \vspace{-1.5ex}

    \rule{\textwidth}{0.5\fboxrule}
\setlength{\parskip}{2ex}
\setlength{\parskip}{1ex}
      \textbf{Parameters}
      \vspace{-1ex}

      \begin{quote}
        \begin{Ventry}{xxxxxx}

          \item[pImage]

          pointer to image

        \end{Ventry}

      \end{quote}

\textbf{Status:} Untested + NoDoc + NoDemo = NOT OK



    \end{boxedminipage}

    \label{xformslib:library:flimage_dup_}
    \index{xformslib \textit{(package)}!xformslib.library \textit{(module)}!xformslib.library.flimage\_dup\_ \textit{(function)}}

    \vspace{0.5ex}

\hspace{.8\funcindent}\begin{boxedminipage}{\funcwidth}

    \raggedright \textbf{flimage\_dup\_}(\textit{pImage}, \textit{p2})

    \vspace{-1.5ex}

    \rule{\textwidth}{0.5\fboxrule}
\setlength{\parskip}{2ex}
\setlength{\parskip}{1ex}
      \textbf{Parameters}
      \vspace{-1ex}

      \begin{quote}
        \begin{Ventry}{xxxxxx}

          \item[pImage]

          pointer to image

        \end{Ventry}

      \end{quote}

      \textbf{Return Value}
    \vspace{-1ex}

      \begin{quote}
      pImage

      \end{quote}

\textbf{Status:} Untested + NoDoc + NoDemo = NOT OK



    \end{boxedminipage}

    \label{xformslib:library:flimage_enable_bmp}
    \index{xformslib \textit{(package)}!xformslib.library \textit{(module)}!xformslib.library.flimage\_enable\_bmp \textit{(function)}}

    \vspace{0.5ex}

\hspace{.8\funcindent}\begin{boxedminipage}{\funcwidth}

    \raggedright \textbf{flimage\_enable\_bmp}()

    \vspace{-1.5ex}

    \rule{\textwidth}{0.5\fboxrule}
\setlength{\parskip}{2ex}
    Enables use of BMP (Windows/OS2 Bitmap) image format.

\setlength{\parskip}{1ex}
\textbf{Example:} flimage\_enable\_bmp()



\textbf{Status:} Tested + Doc + NoDemo = OK



    \end{boxedminipage}

    \label{xformslib:library:flimage_enable_fits}
    \index{xformslib \textit{(package)}!xformslib.library \textit{(module)}!xformslib.library.flimage\_enable\_fits \textit{(function)}}

    \vspace{0.5ex}

\hspace{.8\funcindent}\begin{boxedminipage}{\funcwidth}

    \raggedright \textbf{flimage\_enable\_fits}()

    \vspace{-1.5ex}

    \rule{\textwidth}{0.5\fboxrule}
\setlength{\parskip}{2ex}
    Enables use of NASA/NOTS standard FITS image format.

\setlength{\parskip}{1ex}
\textbf{Example:} flimage\_enable\_fits()



\textbf{Status:} Tested + Doc + NoDemo = OK



    \end{boxedminipage}

    \label{xformslib:library:flimage_enable_genesis}
    \index{xformslib \textit{(package)}!xformslib.library \textit{(module)}!xformslib.library.flimage\_enable\_genesis \textit{(function)}}

    \vspace{0.5ex}

\hspace{.8\funcindent}\begin{boxedminipage}{\funcwidth}

    \raggedright \textbf{flimage\_enable\_genesis}()

    \vspace{-1.5ex}

    \rule{\textwidth}{0.5\fboxrule}
\setlength{\parskip}{2ex}
    Enables use of Genesis image format.

\setlength{\parskip}{1ex}
\textbf{Example:} flimage\_enable\_genesis()



\textbf{Status:} Tested + Doc + NoDemo = OK



    \end{boxedminipage}

    \label{xformslib:library:flimage_enable_gif}
    \index{xformslib \textit{(package)}!xformslib.library \textit{(module)}!xformslib.library.flimage\_enable\_gif \textit{(function)}}

    \vspace{0.5ex}

\hspace{.8\funcindent}\begin{boxedminipage}{\funcwidth}

    \raggedright \textbf{flimage\_enable\_gif}()

    \vspace{-1.5ex}

    \rule{\textwidth}{0.5\fboxrule}
\setlength{\parskip}{2ex}
    Enables use of GIF (Compuserve Graphics Interchange format) image 
    format.

\setlength{\parskip}{1ex}
\textbf{Example:} flimage\_enable\_gif()



\textbf{Status:} Tested + Doc + NoDemo = OK



    \end{boxedminipage}

    \label{xformslib:library:flimage_enable_gzip}
    \index{xformslib \textit{(package)}!xformslib.library \textit{(module)}!xformslib.library.flimage\_enable\_gzip \textit{(function)}}

    \vspace{0.5ex}

\hspace{.8\funcindent}\begin{boxedminipage}{\funcwidth}

    \raggedright \textbf{flimage\_enable\_gzip}()

    \vspace{-1.5ex}

    \rule{\textwidth}{0.5\fboxrule}
\setlength{\parskip}{2ex}
    Enables use of gzip compression filter for images.

\setlength{\parskip}{1ex}
\textbf{Example:} flimage\_enable\_gzip()



\textbf{Status:} Tested + Doc + NoDemo = OK



    \end{boxedminipage}

    \label{xformslib:library:flimage_enable_jpeg}
    \index{xformslib \textit{(package)}!xformslib.library \textit{(module)}!xformslib.library.flimage\_enable\_jpeg \textit{(function)}}

    \vspace{0.5ex}

\hspace{.8\funcindent}\begin{boxedminipage}{\funcwidth}

    \raggedright \textbf{flimage\_enable\_jpeg}()

    \vspace{-1.5ex}

    \rule{\textwidth}{0.5\fboxrule}
\setlength{\parskip}{2ex}
    Enables use of JPEG/JFIF (Joint Photographic Experts Group) image 
    format.

\setlength{\parskip}{1ex}
\textbf{Example:} flimage\_enable\_jpeg()



\textbf{Status:} Tested + Doc + NoDemo = OK



    \end{boxedminipage}

    \label{xformslib:library:flimage_enable_png}
    \index{xformslib \textit{(package)}!xformslib.library \textit{(module)}!xformslib.library.flimage\_enable\_png \textit{(function)}}

    \vspace{0.5ex}

\hspace{.8\funcindent}\begin{boxedminipage}{\funcwidth}

    \raggedright \textbf{flimage\_enable\_png}()

    \vspace{-1.5ex}

    \rule{\textwidth}{0.5\fboxrule}
\setlength{\parskip}{2ex}
    Enables use of PNG (Portable Network Graphics) image format. It 
    requires netpbm library to be installed

\setlength{\parskip}{1ex}
\textbf{Example:} flimage\_enable\_png()



\textbf{Status:} Tested + Doc + NoDemo = OK



    \end{boxedminipage}

    \label{xformslib:library:flimage_enable_ps}
    \index{xformslib \textit{(package)}!xformslib.library \textit{(module)}!xformslib.library.flimage\_enable\_ps \textit{(function)}}

    \vspace{0.5ex}

\hspace{.8\funcindent}\begin{boxedminipage}{\funcwidth}

    \raggedright \textbf{flimage\_enable\_ps}()

    \vspace{-1.5ex}

    \rule{\textwidth}{0.5\fboxrule}
\setlength{\parskip}{2ex}
    Enables use of PS (Adobe PostScript) image format. It needs gs for 
    reading.

\setlength{\parskip}{1ex}
\textbf{Example:} flimage\_enable\_ps()



\textbf{Status:} Tested + Doc + NoDemo = OK



    \end{boxedminipage}

    \label{xformslib:library:flimage_enable_sgi}
    \index{xformslib \textit{(package)}!xformslib.library \textit{(module)}!xformslib.library.flimage\_enable\_sgi \textit{(function)}}

    \vspace{0.5ex}

\hspace{.8\funcindent}\begin{boxedminipage}{\funcwidth}

    \raggedright \textbf{flimage\_enable\_sgi}()

    \vspace{-1.5ex}

    \rule{\textwidth}{0.5\fboxrule}
\setlength{\parskip}{2ex}
    Enables use of SGI (Silicon Graphics-Iris) image format. It requires 
    pbmplus/netpbm library to be installed

\setlength{\parskip}{1ex}
\textbf{Example:} flimage\_enable\_sgi()



\textbf{Status:} Tested + Doc + NoDemo = OK



    \end{boxedminipage}

    \label{xformslib:library:flimage_enable_tiff}
    \index{xformslib \textit{(package)}!xformslib.library \textit{(module)}!xformslib.library.flimage\_enable\_tiff \textit{(function)}}

    \vspace{0.5ex}

\hspace{.8\funcindent}\begin{boxedminipage}{\funcwidth}

    \raggedright \textbf{flimage\_enable\_tiff}()

    \vspace{-1.5ex}

    \rule{\textwidth}{0.5\fboxrule}
\setlength{\parskip}{2ex}
    Enables use of TIFF (Tagged Image file, with no compression) image 
    format.

\setlength{\parskip}{1ex}
\textbf{Example:} flimage\_enable\_tiff()



\textbf{Status:} Tested + Doc + NoDemo = OK



    \end{boxedminipage}

    \label{xformslib:library:flimage_enable_xbm}
    \index{xformslib \textit{(package)}!xformslib.library \textit{(module)}!xformslib.library.flimage\_enable\_xbm \textit{(function)}}

    \vspace{0.5ex}

\hspace{.8\funcindent}\begin{boxedminipage}{\funcwidth}

    \raggedright \textbf{flimage\_enable\_xbm}()

    \vspace{-1.5ex}

    \rule{\textwidth}{0.5\fboxrule}
\setlength{\parskip}{2ex}
    Enables use of XBM (X Window Bitmap) image format.

\setlength{\parskip}{1ex}
\textbf{Example:} flimage\_enable\_xbm()



\textbf{Status:} Tested + Doc + NoDemo = OK



    \end{boxedminipage}

    \label{xformslib:library:flimage_enable_xpm}
    \index{xformslib \textit{(package)}!xformslib.library \textit{(module)}!xformslib.library.flimage\_enable\_xpm \textit{(function)}}

    \vspace{0.5ex}

\hspace{.8\funcindent}\begin{boxedminipage}{\funcwidth}

    \raggedright \textbf{flimage\_enable\_xpm}()

    \vspace{-1.5ex}

    \rule{\textwidth}{0.5\fboxrule}
\setlength{\parskip}{2ex}
    Enables use of XPM3 (X Window PixMap) image format.

\setlength{\parskip}{1ex}
\textbf{Example:} flimage\_enable\_xpm()



\textbf{Status:} Tested + Doc + NoDemo = OK



    \end{boxedminipage}

    \label{xformslib:library:flimage_enable_xwd}
    \index{xformslib \textit{(package)}!xformslib.library \textit{(module)}!xformslib.library.flimage\_enable\_xwd \textit{(function)}}

    \vspace{0.5ex}

\hspace{.8\funcindent}\begin{boxedminipage}{\funcwidth}

    \raggedright \textbf{flimage\_enable\_xwd}()

    \vspace{-1.5ex}

    \rule{\textwidth}{0.5\fboxrule}
\setlength{\parskip}{2ex}
    Enables use of XWD (X Window Dump) image format.

\setlength{\parskip}{1ex}
\textbf{Example:} flimage\_enable\_xwd()



\textbf{Status:} Tested + Doc + NoDemo = OK



    \end{boxedminipage}

    \label{xformslib:library:flimage_free_ci}
    \index{xformslib \textit{(package)}!xformslib.library \textit{(module)}!xformslib.library.flimage\_free\_ci \textit{(function)}}

    \vspace{0.5ex}

\hspace{.8\funcindent}\begin{boxedminipage}{\funcwidth}

    \raggedright \textbf{flimage\_free\_ci}(\textit{pImage})

    \vspace{-1.5ex}

    \rule{\textwidth}{0.5\fboxrule}
\setlength{\parskip}{2ex}
\setlength{\parskip}{1ex}
      \textbf{Parameters}
      \vspace{-1ex}

      \begin{quote}
        \begin{Ventry}{xxxxxx}

          \item[pImage]

          pointer to image

        \end{Ventry}

      \end{quote}

\textbf{Status:} Untested + NoDoc + NoDemo = NOT OK



    \end{boxedminipage}

    \label{xformslib:library:flimage_free_gray}
    \index{xformslib \textit{(package)}!xformslib.library \textit{(module)}!xformslib.library.flimage\_free\_gray \textit{(function)}}

    \vspace{0.5ex}

\hspace{.8\funcindent}\begin{boxedminipage}{\funcwidth}

    \raggedright \textbf{flimage\_free\_gray}(\textit{pImage})

    \vspace{-1.5ex}

    \rule{\textwidth}{0.5\fboxrule}
\setlength{\parskip}{2ex}
\setlength{\parskip}{1ex}
      \textbf{Parameters}
      \vspace{-1ex}

      \begin{quote}
        \begin{Ventry}{xxxxxx}

          \item[pImage]

          image ({\textless}pointer to xfdata.FL\_IMAGE{\textgreater})

        \end{Ventry}

      \end{quote}

\textbf{Status:} Untested + NoDoc + NoDemo = NOT OK



    \end{boxedminipage}

    \label{xformslib:library:flimage_free_linearlut}
    \index{xformslib \textit{(package)}!xformslib.library \textit{(module)}!xformslib.library.flimage\_free\_linearlut \textit{(function)}}

    \vspace{0.5ex}

\hspace{.8\funcindent}\begin{boxedminipage}{\funcwidth}

    \raggedright \textbf{flimage\_free\_linearlut}(\textit{pImage})

    \vspace{-1.5ex}

    \rule{\textwidth}{0.5\fboxrule}
\setlength{\parskip}{2ex}
\setlength{\parskip}{1ex}
      \textbf{Parameters}
      \vspace{-1ex}

      \begin{quote}
        \begin{Ventry}{xxxxxx}

          \item[pImage]

          image ({\textless}pointer to xfdata.FL\_IMAGE{\textgreater})

        \end{Ventry}

      \end{quote}

\textbf{Status:} Untested + NoDoc + NoDemo = NOT OK



    \end{boxedminipage}

    \label{xformslib:library:flimage_free_rgb}
    \index{xformslib \textit{(package)}!xformslib.library \textit{(module)}!xformslib.library.flimage\_free\_rgb \textit{(function)}}

    \vspace{0.5ex}

\hspace{.8\funcindent}\begin{boxedminipage}{\funcwidth}

    \raggedright \textbf{flimage\_free\_rgb}(\textit{pImage})

    \vspace{-1.5ex}

    \rule{\textwidth}{0.5\fboxrule}
\setlength{\parskip}{2ex}
\setlength{\parskip}{1ex}
      \textbf{Parameters}
      \vspace{-1ex}

      \begin{quote}
        \begin{Ventry}{xxxxxx}

          \item[pImage]

          image ({\textless}pointer to xfdata.FL\_IMAGE{\textgreater})

        \end{Ventry}

      \end{quote}

\textbf{Status:} Untested + NoDoc + NoDemo = NOT OK



    \end{boxedminipage}

    \label{xformslib:library:flimage_freemem}
    \index{xformslib \textit{(package)}!xformslib.library \textit{(module)}!xformslib.library.flimage\_freemem \textit{(function)}}

    \vspace{0.5ex}

\hspace{.8\funcindent}\begin{boxedminipage}{\funcwidth}

    \raggedright \textbf{flimage\_freemem}(\textit{pImage})

    \vspace{-1.5ex}

    \rule{\textwidth}{0.5\fboxrule}
\setlength{\parskip}{2ex}
\setlength{\parskip}{1ex}
      \textbf{Parameters}
      \vspace{-1ex}

      \begin{quote}
        \begin{Ventry}{xxxxxx}

          \item[pImage]

          image ({\textless}pointer to xfdata.FL\_IMAGE{\textgreater})

        \end{Ventry}

      \end{quote}

\textbf{Status:} Untested + NoDoc + NoDemo = NOT OK



    \end{boxedminipage}

    \label{xformslib:library:flimage_get_closest_color_from_map}
    \index{xformslib \textit{(package)}!xformslib.library \textit{(module)}!xformslib.library.flimage\_get\_closest\_color\_from\_map \textit{(function)}}

    \vspace{0.5ex}

\hspace{.8\funcindent}\begin{boxedminipage}{\funcwidth}

    \raggedright \textbf{flimage\_get\_closest\_color\_from\_map}(\textit{pImage}, \textit{p2})

    \vspace{-1.5ex}

    \rule{\textwidth}{0.5\fboxrule}
\setlength{\parskip}{2ex}
\setlength{\parskip}{1ex}
      \textbf{Parameters}
      \vspace{-1ex}

      \begin{quote}
        \begin{Ventry}{xxxxxx}

          \item[pImage]

          image ({\textless}pointer to xfdata.FL\_IMAGE{\textgreater})

        \end{Ventry}

      \end{quote}

      \textbf{Return Value}
    \vspace{-1ex}

      \begin{quote}
      num

      \end{quote}

\textbf{Status:} Untested + NoDoc + NoDemo = NOT OK



    \end{boxedminipage}

    \label{xformslib:library:flimage_get_linearlut}
    \index{xformslib \textit{(package)}!xformslib.library \textit{(module)}!xformslib.library.flimage\_get\_linearlut \textit{(function)}}

    \vspace{0.5ex}

\hspace{.8\funcindent}\begin{boxedminipage}{\funcwidth}

    \raggedright \textbf{flimage\_get\_linearlut}(\textit{pImage})

    \vspace{-1.5ex}

    \rule{\textwidth}{0.5\fboxrule}
\setlength{\parskip}{2ex}
\setlength{\parskip}{1ex}
      \textbf{Parameters}
      \vspace{-1ex}

      \begin{quote}
        \begin{Ventry}{xxxxxx}

          \item[pImage]

          image ({\textless}pointer to xfdata.FL\_IMAGE{\textgreater})

        \end{Ventry}

      \end{quote}

      \textbf{Return Value}
    \vspace{-1ex}

      \begin{quote}
      num

      \end{quote}

\textbf{Status:} Untested + NoDoc + NoDemo = NOT OK



    \end{boxedminipage}

    \label{xformslib:library:flimage_invalidate_pixels}
    \index{xformslib \textit{(package)}!xformslib.library \textit{(module)}!xformslib.library.flimage\_invalidate\_pixels \textit{(function)}}

    \vspace{0.5ex}

\hspace{.8\funcindent}\begin{boxedminipage}{\funcwidth}

    \raggedright \textbf{flimage\_invalidate\_pixels}(\textit{pImage})

    \vspace{-1.5ex}

    \rule{\textwidth}{0.5\fboxrule}
\setlength{\parskip}{2ex}
\setlength{\parskip}{1ex}
      \textbf{Parameters}
      \vspace{-1ex}

      \begin{quote}
        \begin{Ventry}{xxxxxx}

          \item[pImage]

          image ({\textless}pointer to xfdata.FL\_IMAGE{\textgreater})

        \end{Ventry}

      \end{quote}

\textbf{Status:} Untested + NoDoc + NoDemo = NOT OK



    \end{boxedminipage}

    \label{xformslib:library:flimage_open}
    \index{xformslib \textit{(package)}!xformslib.library \textit{(module)}!xformslib.library.flimage\_open \textit{(function)}}

    \vspace{0.5ex}

\hspace{.8\funcindent}\begin{boxedminipage}{\funcwidth}

    \raggedright \textbf{flimage\_open}(\textit{filename})

    \vspace{-1.5ex}

    \rule{\textwidth}{0.5\fboxrule}
\setlength{\parskip}{2ex}
\setlength{\parskip}{1ex}
      \textbf{Parameters}
      \vspace{-1ex}

      \begin{quote}
        \begin{Ventry}{xxxxxxxx}

          \item[filename]

          name of file to open

        \end{Ventry}

      \end{quote}

      \textbf{Return Value}
    \vspace{-1ex}

      \begin{quote}
      image ({\textless}pointer to xfdata.FL\_IMAGE{\textgreater})

      {\it (type=pImage)}

      \end{quote}

\textbf{Status:} Untested + NoDoc + NoDemo = NOT OK



    \end{boxedminipage}

    \label{xformslib:library:flimage_read_annotation}
    \index{xformslib \textit{(package)}!xformslib.library \textit{(module)}!xformslib.library.flimage\_read\_annotation \textit{(function)}}

    \vspace{0.5ex}

\hspace{.8\funcindent}\begin{boxedminipage}{\funcwidth}

    \raggedright \textbf{flimage\_read\_annotation}(\textit{pImage})

    \vspace{-1.5ex}

    \rule{\textwidth}{0.5\fboxrule}
\setlength{\parskip}{2ex}
\setlength{\parskip}{1ex}
      \textbf{Parameters}
      \vspace{-1ex}

      \begin{quote}
        \begin{Ventry}{xxxxxx}

          \item[pImage]

          pointer to image

        \end{Ventry}

      \end{quote}

      \textbf{Return Value}
    \vspace{-1ex}

      \begin{quote}
      num

      \end{quote}

\textbf{Status:} Untested + NoDoc + NoDemo = NOT OK



    \end{boxedminipage}

    \label{xformslib:library:flimage_replace_image}
    \index{xformslib \textit{(package)}!xformslib.library \textit{(module)}!xformslib.library.flimage\_replace\_image \textit{(function)}}

    \vspace{0.5ex}

\hspace{.8\funcindent}\begin{boxedminipage}{\funcwidth}

    \raggedright \textbf{flimage\_replace\_image}(\textit{pImage}, \textit{w}, \textit{h}, \textit{r}, \textit{g}, \textit{b})

    \vspace{-1.5ex}

    \rule{\textwidth}{0.5\fboxrule}
\setlength{\parskip}{2ex}
\setlength{\parskip}{1ex}
      \textbf{Parameters}
      \vspace{-1ex}

      \begin{quote}
        \begin{Ventry}{xxxxxx}

          \item[pImage]

          pointer to image

        \end{Ventry}

      \end{quote}

\textbf{Status:} Untested + NoDoc + NoDemo = NOT OK



    \end{boxedminipage}

    \label{xformslib:library:flimage_swapbuffer}
    \index{xformslib \textit{(package)}!xformslib.library \textit{(module)}!xformslib.library.flimage\_swapbuffer \textit{(function)}}

    \vspace{0.5ex}

\hspace{.8\funcindent}\begin{boxedminipage}{\funcwidth}

    \raggedright \textbf{flimage\_swapbuffer}(\textit{pImage})

    \vspace{-1.5ex}

    \rule{\textwidth}{0.5\fboxrule}
\setlength{\parskip}{2ex}
\setlength{\parskip}{1ex}
      \textbf{Parameters}
      \vspace{-1ex}

      \begin{quote}
        \begin{Ventry}{xxxxxx}

          \item[pImage]

          pointer to image

        \end{Ventry}

      \end{quote}

      \textbf{Return Value}
    \vspace{-1ex}

      \begin{quote}
      num

      \end{quote}

\textbf{Status:} Untested + NoDoc + NoDemo = NOT OK



    \end{boxedminipage}

    \label{xformslib:library:flimage_to_ximage}
    \index{xformslib \textit{(package)}!xformslib.library \textit{(module)}!xformslib.library.flimage\_to\_ximage \textit{(function)}}

    \vspace{0.5ex}

\hspace{.8\funcindent}\begin{boxedminipage}{\funcwidth}

    \raggedright \textbf{flimage\_to\_ximage}(\textit{pImage}, \textit{win}, \textit{pXWindowAttributes})

    \vspace{-1.5ex}

    \rule{\textwidth}{0.5\fboxrule}
\setlength{\parskip}{2ex}
\setlength{\parskip}{1ex}
      \textbf{Parameters}
      \vspace{-1ex}

      \begin{quote}
        \begin{Ventry}{xxxxxxxxxxxxxxxxxx}

          \item[pImage]

          pointer to image

          \item[win]

          window id

          \item[pXWindowAttributes]

          pointer to XWindowAttributes class instance

        \end{Ventry}

      \end{quote}

      \textbf{Return Value}
    \vspace{-1ex}

      \begin{quote}
      num

      \end{quote}

\textbf{Status:} Untested + NoDoc + NoDemo = NOT OK



    \end{boxedminipage}

    \label{xformslib:library:flimage_write_annotation}
    \index{xformslib \textit{(package)}!xformslib.library \textit{(module)}!xformslib.library.flimage\_write\_annotation \textit{(function)}}

    \vspace{0.5ex}

\hspace{.8\funcindent}\begin{boxedminipage}{\funcwidth}

    \raggedright \textbf{flimage\_write\_annotation}(\textit{pImage})

    \vspace{-1.5ex}

    \rule{\textwidth}{0.5\fboxrule}
\setlength{\parskip}{2ex}
\setlength{\parskip}{1ex}
      \textbf{Parameters}
      \vspace{-1ex}

      \begin{quote}
        \begin{Ventry}{xxxxxx}

          \item[pImage]

          pointer to image

        \end{Ventry}

      \end{quote}

      \textbf{Return Value}
    \vspace{-1ex}

      \begin{quote}
      num

      \end{quote}

\textbf{Status:} Untested + NoDoc + NoDemo = NOT OK



    \end{boxedminipage}


%%%%%%%%%%%%%%%%%%%%%%%%%%%%%%%%%%%%%%%%%%%%%%%%%%%%%%%%%%%%%%%%%%%%%%%%%%%
%%                               Variables                               %%
%%%%%%%%%%%%%%%%%%%%%%%%%%%%%%%%%%%%%%%%%%%%%%%%%%%%%%%%%%%%%%%%%%%%%%%%%%%

  \subsection{Variables}

    \vspace{-1cm}
\hspace{\varindent}\begin{longtable}{|p{\varnamewidth}|p{\vardescrwidth}|l}
\cline{1-2}
\cline{1-2} \centering \textbf{Name} & \centering \textbf{Description}& \\
\cline{1-2}
\endhead\cline{1-2}\multicolumn{3}{r}{\small\textit{continued on next page}}\\\endfoot\cline{1-2}
\endlastfoot\raggedright \_\-\_\-m\-a\-i\-n\-v\-e\-r\-s\-i\-o\-n\-\_\-\_\- & \raggedright \textbf{Value:} 
{\tt \texttt{'}\texttt{0.3.5}\texttt{'}}&\\
\cline{1-2}
\raggedright \_\-\_\-v\-e\-r\-s\-\_\-a\-g\-a\-i\-n\-s\-t\-\_\-x\-f\-o\-r\-m\-s\-\_\-\_\- & \raggedright \textbf{Value:} 
{\tt \texttt{'}\texttt{1.0.93pre3}\texttt{'}}&\\
\cline{1-2}
\raggedright h\-e\-a\-d\-e\-r\-\_\-f\-i\-l\-e\-n\-a\-m\-e\- & \raggedright \textbf{Value:} 
{\tt \texttt{'}\texttt{/usr/include/forms.h}\texttt{'}}&\\
\cline{1-2}
\raggedright l\-o\-a\-d\-e\-d\-\_\-x\-l\-i\-b\-r\-a\-r\-i\-e\-s\- & \raggedright \textbf{Value:} 
{\tt \texttt{\{}\texttt{'}\texttt{libflimage}\texttt{'}\texttt{: }None\texttt{, }\texttt{'}\texttt{libforms}\texttt{'}\texttt{: }{\textless}CDLL 'libforms.so.2', h\texttt{...}}&\\
\cline{1-2}
\raggedright f\-l\-i\-n\-i\-t\-i\-a\-l\-i\-z\-e\-d\- & \raggedright \textbf{Value:} 
{\tt False}&\\
\cline{1-2}
\raggedright F\-L\-\_\-E\-V\-E\-N\-T\- & \raggedright \textbf{Value:} 
{\tt cty.POINTER(xfc.FL\_OBJECT).in\_dll(load\_so\_libforms(), 'FL\texttt{...}}&\\
\cline{1-2}
\raggedright f\-l\-\_\-c\-u\-r\-r\-e\-n\-t\-\_\-f\-o\-r\-m\- & \raggedright \textbf{Value:} 
{\tt cty.POINTER(xfc.FL\_FORM).in\_dll(load\_so\_libforms(), 'fl\_c\texttt{...}}&\\
\cline{1-2}
\raggedright f\-l\-\_\-d\-i\-s\-p\-l\-a\-y\- & \raggedright \textbf{Value:} 
{\tt cty.POINTER(xfc.Display).in\_dll(load\_so\_libforms(), 'fl\_d\texttt{...}}&\\
\cline{1-2}
\raggedright f\-l\-\_\-s\-c\-r\-e\-e\-n\- & \raggedright \textbf{Value:} 
{\tt c\_int(0)}&\\
\cline{1-2}
\raggedright f\-l\-\_\-r\-o\-o\-t\- & \raggedright \textbf{Value:} 
{\tt c\_ulong(0L)}&\\
\cline{1-2}
\raggedright f\-l\-\_\-v\-r\-o\-o\-t\- & \raggedright \textbf{Value:} 
{\tt c\_ulong(0L)}&\\
\cline{1-2}
\raggedright f\-l\-\_\-s\-c\-r\-h\- & \raggedright \textbf{Value:} 
{\tt c\_int(0)}&\\
\cline{1-2}
\raggedright f\-l\-\_\-s\-c\-r\-w\- & \raggedright \textbf{Value:} 
{\tt c\_int(0)}&\\
\cline{1-2}
\raggedright f\-l\-\_\-v\-m\-o\-d\-e\- & \raggedright \textbf{Value:} 
{\tt c\_int(-1)}&\\
\cline{1-2}
\raggedright f\-l\-\_\-s\-t\-a\-t\-e\- & \raggedright \textbf{Value:} 
{\tt cty.POINTER(xfc.FL\_State).in\_dll(load\_so\_libforms(), 'fl\_\texttt{...}}&\\
\cline{1-2}
\raggedright f\-l\-\_\-u\-l\-\_\-m\-a\-g\-i\-c\-\_\-c\-h\-a\-r\- & \raggedright \textbf{Value:} 
{\tt c\_char\_p(None)}&\\
\cline{1-2}
\raggedright \_\-\_\-p\-a\-c\-k\-a\-g\-e\-\_\-\_\- & \raggedright \textbf{Value:} 
{\tt \texttt{'}\texttt{xformslib}\texttt{'}}&\\
\cline{1-2}
\end{longtable}


%%%%%%%%%%%%%%%%%%%%%%%%%%%%%%%%%%%%%%%%%%%%%%%%%%%%%%%%%%%%%%%%%%%%%%%%%%%
%%                           Class Description                           %%
%%%%%%%%%%%%%%%%%%%%%%%%%%%%%%%%%%%%%%%%%%%%%%%%%%%%%%%%%%%%%%%%%%%%%%%%%%%

    \index{xformslib \textit{(package)}!xformslib.library \textit{(module)}!xformslib.library.XFormsLoadError \textit{(class)}|(}
\subsection{Class XFormsLoadError}

    \label{xformslib:library:XFormsLoadError}
\begin{tabular}{cccccccccccccccc}
% Line for object, linespec=[False, False, False, False, False, False]
\multicolumn{2}{r}{\settowidth{\BCL}{object}\multirow{2}{\BCL}{object}}
&&
&&
&&
&&
&&
&&
  \\\cline{3-3}
  &&\multicolumn{1}{c|}{}
&&
&&
&&
&&
&&
&&
  \\
% Line for exceptions.BaseException, linespec=[False, False, False, False, False]
\multicolumn{4}{r}{\settowidth{\BCL}{exceptions.BaseException}\multirow{2}{\BCL}{exceptions.BaseException}}
&&
&&
&&
&&
&&
  \\\cline{5-5}
  &&&&\multicolumn{1}{c|}{}
&&
&&
&&
&&
&&
  \\
% Line for exceptions.Exception, linespec=[False, False, False, False]
\multicolumn{6}{r}{\settowidth{\BCL}{exceptions.Exception}\multirow{2}{\BCL}{exceptions.Exception}}
&&
&&
&&
&&
  \\\cline{7-7}
  &&&&&&\multicolumn{1}{c|}{}
&&
&&
&&
&&
  \\
% Line for exceptions.StandardError, linespec=[False, False, False]
\multicolumn{8}{r}{\settowidth{\BCL}{exceptions.StandardError}\multirow{2}{\BCL}{exceptions.StandardError}}
&&
&&
&&
  \\\cline{9-9}
  &&&&&&&&\multicolumn{1}{c|}{}
&&
&&
&&
  \\
% Line for exceptions.EnvironmentError, linespec=[False, False]
\multicolumn{10}{r}{\settowidth{\BCL}{exceptions.EnvironmentError}\multirow{2}{\BCL}{exceptions.EnvironmentError}}
&&
&&
  \\\cline{11-11}
  &&&&&&&&&&\multicolumn{1}{c|}{}
&&
&&
  \\
% Line for exceptions.OSError, linespec=[False]
\multicolumn{12}{r}{\settowidth{\BCL}{exceptions.OSError}\multirow{2}{\BCL}{exceptions.OSError}}
&&
  \\\cline{13-13}
  &&&&&&&&&&&&\multicolumn{1}{c|}{}
&&
  \\
&&&&&&&&&&&&\multicolumn{2}{l}{\textbf{xformslib.library.XFormsLoadError}}
\end{tabular}

Error in loading shared object library


%%%%%%%%%%%%%%%%%%%%%%%%%%%%%%%%%%%%%%%%%%%%%%%%%%%%%%%%%%%%%%%%%%%%%%%%%%%
%%                                Methods                                %%
%%%%%%%%%%%%%%%%%%%%%%%%%%%%%%%%%%%%%%%%%%%%%%%%%%%%%%%%%%%%%%%%%%%%%%%%%%%

  \subsubsection{Methods}


\large{\textbf{\textit{Inherited from exceptions.OSError}}}

\begin{quote}
\_\_init\_\_(), \_\_new\_\_()
\end{quote}

\large{\textbf{\textit{Inherited from exceptions.EnvironmentError}}}

\begin{quote}
\_\_reduce\_\_(), \_\_str\_\_()
\end{quote}

\large{\textbf{\textit{Inherited from exceptions.BaseException}}}

\begin{quote}
\_\_delattr\_\_(), \_\_getattribute\_\_(), \_\_getitem\_\_(), \_\_getslice\_\_(), \_\_repr\_\_(), \_\_setattr\_\_(), \_\_setstate\_\_(), \_\_unicode\_\_()
\end{quote}

\large{\textbf{\textit{Inherited from object}}}

\begin{quote}
\_\_format\_\_(), \_\_hash\_\_(), \_\_reduce\_ex\_\_(), \_\_sizeof\_\_(), \_\_subclasshook\_\_()
\end{quote}

%%%%%%%%%%%%%%%%%%%%%%%%%%%%%%%%%%%%%%%%%%%%%%%%%%%%%%%%%%%%%%%%%%%%%%%%%%%
%%                              Properties                               %%
%%%%%%%%%%%%%%%%%%%%%%%%%%%%%%%%%%%%%%%%%%%%%%%%%%%%%%%%%%%%%%%%%%%%%%%%%%%

  \subsubsection{Properties}

    \vspace{-1cm}
\hspace{\varindent}\begin{longtable}{|p{\varnamewidth}|p{\vardescrwidth}|l}
\cline{1-2}
\cline{1-2} \centering \textbf{Name} & \centering \textbf{Description}& \\
\cline{1-2}
\endhead\cline{1-2}\multicolumn{3}{r}{\small\textit{continued on next page}}\\\endfoot\cline{1-2}
\endlastfoot\multicolumn{2}{|l|}{\textit{Inherited from exceptions.EnvironmentError}}\\
\multicolumn{2}{|p{\varwidth}|}{\raggedright errno, filename, strerror}\\
\cline{1-2}
\multicolumn{2}{|l|}{\textit{Inherited from exceptions.BaseException}}\\
\multicolumn{2}{|p{\varwidth}|}{\raggedright args, message}\\
\cline{1-2}
\multicolumn{2}{|l|}{\textit{Inherited from object}}\\
\multicolumn{2}{|p{\varwidth}|}{\raggedright \_\_class\_\_}\\
\cline{1-2}
\end{longtable}

    \index{xformslib \textit{(package)}!xformslib.library \textit{(module)}!xformslib.library.XFormsLoadError \textit{(class)}|)}

%%%%%%%%%%%%%%%%%%%%%%%%%%%%%%%%%%%%%%%%%%%%%%%%%%%%%%%%%%%%%%%%%%%%%%%%%%%
%%                           Class Description                           %%
%%%%%%%%%%%%%%%%%%%%%%%%%%%%%%%%%%%%%%%%%%%%%%%%%%%%%%%%%%%%%%%%%%%%%%%%%%%

    \index{xformslib \textit{(package)}!xformslib.library \textit{(module)}!xformslib.library.XFormsTypeError \textit{(class)}|(}
\subsection{Class XFormsTypeError}

    \label{xformslib:library:XFormsTypeError}
\begin{tabular}{cccccccccccccc}
% Line for object, linespec=[False, False, False, False, False]
\multicolumn{2}{r}{\settowidth{\BCL}{object}\multirow{2}{\BCL}{object}}
&&
&&
&&
&&
&&
  \\\cline{3-3}
  &&\multicolumn{1}{c|}{}
&&
&&
&&
&&
&&
  \\
% Line for exceptions.BaseException, linespec=[False, False, False, False]
\multicolumn{4}{r}{\settowidth{\BCL}{exceptions.BaseException}\multirow{2}{\BCL}{exceptions.BaseException}}
&&
&&
&&
&&
  \\\cline{5-5}
  &&&&\multicolumn{1}{c|}{}
&&
&&
&&
&&
  \\
% Line for exceptions.Exception, linespec=[False, False, False]
\multicolumn{6}{r}{\settowidth{\BCL}{exceptions.Exception}\multirow{2}{\BCL}{exceptions.Exception}}
&&
&&
&&
  \\\cline{7-7}
  &&&&&&\multicolumn{1}{c|}{}
&&
&&
&&
  \\
% Line for exceptions.StandardError, linespec=[False, False]
\multicolumn{8}{r}{\settowidth{\BCL}{exceptions.StandardError}\multirow{2}{\BCL}{exceptions.StandardError}}
&&
&&
  \\\cline{9-9}
  &&&&&&&&\multicolumn{1}{c|}{}
&&
&&
  \\
% Line for exceptions.TypeError, linespec=[False]
\multicolumn{10}{r}{\settowidth{\BCL}{exceptions.TypeError}\multirow{2}{\BCL}{exceptions.TypeError}}
&&
  \\\cline{11-11}
  &&&&&&&&&&\multicolumn{1}{c|}{}
&&
  \\
&&&&&&&&&&\multicolumn{2}{l}{\textbf{xformslib.library.XFormsTypeError}}
\end{tabular}

Generic error for type mismatch


%%%%%%%%%%%%%%%%%%%%%%%%%%%%%%%%%%%%%%%%%%%%%%%%%%%%%%%%%%%%%%%%%%%%%%%%%%%
%%                                Methods                                %%
%%%%%%%%%%%%%%%%%%%%%%%%%%%%%%%%%%%%%%%%%%%%%%%%%%%%%%%%%%%%%%%%%%%%%%%%%%%

  \subsubsection{Methods}


\large{\textbf{\textit{Inherited from exceptions.TypeError}}}

\begin{quote}
\_\_init\_\_(), \_\_new\_\_()
\end{quote}

\large{\textbf{\textit{Inherited from exceptions.BaseException}}}

\begin{quote}
\_\_delattr\_\_(), \_\_getattribute\_\_(), \_\_getitem\_\_(), \_\_getslice\_\_(), \_\_reduce\_\_(), \_\_repr\_\_(), \_\_setattr\_\_(), \_\_setstate\_\_(), \_\_str\_\_(), \_\_unicode\_\_()
\end{quote}

\large{\textbf{\textit{Inherited from object}}}

\begin{quote}
\_\_format\_\_(), \_\_hash\_\_(), \_\_reduce\_ex\_\_(), \_\_sizeof\_\_(), \_\_subclasshook\_\_()
\end{quote}

%%%%%%%%%%%%%%%%%%%%%%%%%%%%%%%%%%%%%%%%%%%%%%%%%%%%%%%%%%%%%%%%%%%%%%%%%%%
%%                              Properties                               %%
%%%%%%%%%%%%%%%%%%%%%%%%%%%%%%%%%%%%%%%%%%%%%%%%%%%%%%%%%%%%%%%%%%%%%%%%%%%

  \subsubsection{Properties}

    \vspace{-1cm}
\hspace{\varindent}\begin{longtable}{|p{\varnamewidth}|p{\vardescrwidth}|l}
\cline{1-2}
\cline{1-2} \centering \textbf{Name} & \centering \textbf{Description}& \\
\cline{1-2}
\endhead\cline{1-2}\multicolumn{3}{r}{\small\textit{continued on next page}}\\\endfoot\cline{1-2}
\endlastfoot\multicolumn{2}{|l|}{\textit{Inherited from exceptions.BaseException}}\\
\multicolumn{2}{|p{\varwidth}|}{\raggedright args, message}\\
\cline{1-2}
\multicolumn{2}{|l|}{\textit{Inherited from object}}\\
\multicolumn{2}{|p{\varwidth}|}{\raggedright \_\_class\_\_}\\
\cline{1-2}
\end{longtable}

    \index{xformslib \textit{(package)}!xformslib.library \textit{(module)}!xformslib.library.XFormsTypeError \textit{(class)}|)}

%%%%%%%%%%%%%%%%%%%%%%%%%%%%%%%%%%%%%%%%%%%%%%%%%%%%%%%%%%%%%%%%%%%%%%%%%%%
%%                           Class Description                           %%
%%%%%%%%%%%%%%%%%%%%%%%%%%%%%%%%%%%%%%%%%%%%%%%%%%%%%%%%%%%%%%%%%%%%%%%%%%%

    \index{xformslib \textit{(package)}!xformslib.library \textit{(module)}!xformslib.library.XFormsInitError \textit{(class)}|(}
\subsection{Class XFormsInitError}

    \label{xformslib:library:XFormsInitError}
\begin{tabular}{cccccccccccccccc}
% Line for object, linespec=[False, False, False, False, False, False]
\multicolumn{2}{r}{\settowidth{\BCL}{object}\multirow{2}{\BCL}{object}}
&&
&&
&&
&&
&&
&&
  \\\cline{3-3}
  &&\multicolumn{1}{c|}{}
&&
&&
&&
&&
&&
&&
  \\
% Line for exceptions.BaseException, linespec=[False, False, False, False, False]
\multicolumn{4}{r}{\settowidth{\BCL}{exceptions.BaseException}\multirow{2}{\BCL}{exceptions.BaseException}}
&&
&&
&&
&&
&&
  \\\cline{5-5}
  &&&&\multicolumn{1}{c|}{}
&&
&&
&&
&&
&&
  \\
% Line for exceptions.Exception, linespec=[False, False, False, False]
\multicolumn{6}{r}{\settowidth{\BCL}{exceptions.Exception}\multirow{2}{\BCL}{exceptions.Exception}}
&&
&&
&&
&&
  \\\cline{7-7}
  &&&&&&\multicolumn{1}{c|}{}
&&
&&
&&
&&
  \\
% Line for exceptions.StandardError, linespec=[False, False, False]
\multicolumn{8}{r}{\settowidth{\BCL}{exceptions.StandardError}\multirow{2}{\BCL}{exceptions.StandardError}}
&&
&&
&&
  \\\cline{9-9}
  &&&&&&&&\multicolumn{1}{c|}{}
&&
&&
&&
  \\
% Line for exceptions.EnvironmentError, linespec=[False, False]
\multicolumn{10}{r}{\settowidth{\BCL}{exceptions.EnvironmentError}\multirow{2}{\BCL}{exceptions.EnvironmentError}}
&&
&&
  \\\cline{11-11}
  &&&&&&&&&&\multicolumn{1}{c|}{}
&&
&&
  \\
% Line for exceptions.OSError, linespec=[False]
\multicolumn{12}{r}{\settowidth{\BCL}{exceptions.OSError}\multirow{2}{\BCL}{exceptions.OSError}}
&&
  \\\cline{13-13}
  &&&&&&&&&&&&\multicolumn{1}{c|}{}
&&
  \\
&&&&&&&&&&&&\multicolumn{2}{l}{\textbf{xformslib.library.XFormsInitError}}
\end{tabular}

Error in initializing library, not using fl\_initialize() before functions 
who require it.


%%%%%%%%%%%%%%%%%%%%%%%%%%%%%%%%%%%%%%%%%%%%%%%%%%%%%%%%%%%%%%%%%%%%%%%%%%%
%%                                Methods                                %%
%%%%%%%%%%%%%%%%%%%%%%%%%%%%%%%%%%%%%%%%%%%%%%%%%%%%%%%%%%%%%%%%%%%%%%%%%%%

  \subsubsection{Methods}


\large{\textbf{\textit{Inherited from exceptions.OSError}}}

\begin{quote}
\_\_init\_\_(), \_\_new\_\_()
\end{quote}

\large{\textbf{\textit{Inherited from exceptions.EnvironmentError}}}

\begin{quote}
\_\_reduce\_\_(), \_\_str\_\_()
\end{quote}

\large{\textbf{\textit{Inherited from exceptions.BaseException}}}

\begin{quote}
\_\_delattr\_\_(), \_\_getattribute\_\_(), \_\_getitem\_\_(), \_\_getslice\_\_(), \_\_repr\_\_(), \_\_setattr\_\_(), \_\_setstate\_\_(), \_\_unicode\_\_()
\end{quote}

\large{\textbf{\textit{Inherited from object}}}

\begin{quote}
\_\_format\_\_(), \_\_hash\_\_(), \_\_reduce\_ex\_\_(), \_\_sizeof\_\_(), \_\_subclasshook\_\_()
\end{quote}

%%%%%%%%%%%%%%%%%%%%%%%%%%%%%%%%%%%%%%%%%%%%%%%%%%%%%%%%%%%%%%%%%%%%%%%%%%%
%%                              Properties                               %%
%%%%%%%%%%%%%%%%%%%%%%%%%%%%%%%%%%%%%%%%%%%%%%%%%%%%%%%%%%%%%%%%%%%%%%%%%%%

  \subsubsection{Properties}

    \vspace{-1cm}
\hspace{\varindent}\begin{longtable}{|p{\varnamewidth}|p{\vardescrwidth}|l}
\cline{1-2}
\cline{1-2} \centering \textbf{Name} & \centering \textbf{Description}& \\
\cline{1-2}
\endhead\cline{1-2}\multicolumn{3}{r}{\small\textit{continued on next page}}\\\endfoot\cline{1-2}
\endlastfoot\multicolumn{2}{|l|}{\textit{Inherited from exceptions.EnvironmentError}}\\
\multicolumn{2}{|p{\varwidth}|}{\raggedright errno, filename, strerror}\\
\cline{1-2}
\multicolumn{2}{|l|}{\textit{Inherited from exceptions.BaseException}}\\
\multicolumn{2}{|p{\varwidth}|}{\raggedright args, message}\\
\cline{1-2}
\multicolumn{2}{|l|}{\textit{Inherited from object}}\\
\multicolumn{2}{|p{\varwidth}|}{\raggedright \_\_class\_\_}\\
\cline{1-2}
\end{longtable}

    \index{xformslib \textit{(package)}!xformslib.library \textit{(module)}!xformslib.library.XFormsInitError \textit{(class)}|)}
    \index{xformslib \textit{(package)}!xformslib.library \textit{(module)}|)}
