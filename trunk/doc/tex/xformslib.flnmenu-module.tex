%
% API Documentation for API Documentation
% Module xformslib.flnmenu
%
% Generated by epydoc 3.0.1
% [Fri May 21 15:38:49 2010]
%

%%%%%%%%%%%%%%%%%%%%%%%%%%%%%%%%%%%%%%%%%%%%%%%%%%%%%%%%%%%%%%%%%%%%%%%%%%%
%%                          Module Description                           %%
%%%%%%%%%%%%%%%%%%%%%%%%%%%%%%%%%%%%%%%%%%%%%%%%%%%%%%%%%%%%%%%%%%%%%%%%%%%

    \index{xformslib \textit{(package)}!xformslib.flnmenu \textit{(module)}|(}
\section{Module xformslib.flnmenu}

    \label{xformslib:flnmenu}

xforms-python's functions to manage nmenu objects.

Copyright (C) 2009, 2010  Luca Lazzaroni ``LukenShiro''
e-mail: <\href{mailto:lukenshiro@ngi.it}{lukenshiro@ngi.it}>

This program is free software: you can redistribute it and/or modify
it under the terms of the GNU Lesser General Public License as
published by the Free Software Foundation, version 2.1 of the License.

This program is distributed in the hope that it will be useful,
but WITHOUT ANY WARRANTY; without even the implied warranty of
MERCHANTABILITY or FITNESS FOR A PARTICULAR PURPOSE. See the
GNU Lesser General Public License for more details.

You should have received a copy of the GNU LGPL along with this
program. If not, see <\href{http://www.gnu.org/licenses/}{http://www.gnu.org/licenses/}>.

See CREDITS file to read acknowledgements and thanks to XForms,
ctypes and other developers.

%%%%%%%%%%%%%%%%%%%%%%%%%%%%%%%%%%%%%%%%%%%%%%%%%%%%%%%%%%%%%%%%%%%%%%%%%%%
%%                               Functions                               %%
%%%%%%%%%%%%%%%%%%%%%%%%%%%%%%%%%%%%%%%%%%%%%%%%%%%%%%%%%%%%%%%%%%%%%%%%%%%

  \subsection{Functions}

    \label{xformslib:flnmenu:fl_add_nmenu}
    \index{xformslib \textit{(package)}!xformslib.flnmenu \textit{(module)}!xformslib.flnmenu.fl\_add\_nmenu \textit{(function)}}

    \vspace{0.5ex}

\hspace{.8\funcindent}\begin{boxedminipage}{\funcwidth}

    \raggedright \textbf{fl\_add\_nmenu}(\textit{nmenutype}, \textit{x}, \textit{y}, \textit{w}, \textit{h}, \textit{label})

    \vspace{-1.5ex}

    \rule{\textwidth}{0.5\fboxrule}
\setlength{\parskip}{2ex}

Adds a new generation menu (nmenu) object. It heavily depends on
popups.

-{}-
\setlength{\parskip}{1ex}
      \textbf{Parameters}
      \vspace{-1ex}

      \begin{quote}
        \begin{Ventry}{xxxxxxxxx}

          \item[nmenutype]


FL\_NORMAL\_NMENU, FL\_NORMAL\_TOUCH\_NMENU, FL\_BUTTON\_NMENU,
FL\_BUTTON\_TOUCH\_NMENU
            {\it (type=type of nmenu to be. Values (from xfdata.py))}

          \item[x]


horizontal position (upper-left corner)
            {\it (type=int)}

          \item[y]


vertical position (upper-left corner)
            {\it (type=int)}

          \item[w]


width in coord units
            {\it (type=int)}

          \item[h]


height in coord units
            {\it (type=int)}

          \item[label]


text label of nmenu object
            {\it (type=str)}

        \end{Ventry}

      \end{quote}

      \textbf{Return Value}
    \vspace{-1ex}

      \begin{quote}

nmenu object added (pFlObject)
      {\it (type=pointer to xfdata.FL\_OBJECT)}

      \end{quote}

\textbf{Note:} 
e.g. \emph{todo}


\textbf{Status:} 
Tested + NoDoc + Demo = OK


    \end{boxedminipage}

    \label{xformslib:flnmenu:fl_clear_nmenu}
    \index{xformslib \textit{(package)}!xformslib.flnmenu \textit{(module)}!xformslib.flnmenu.fl\_clear\_nmenu \textit{(function)}}

    \vspace{0.5ex}

\hspace{.8\funcindent}\begin{boxedminipage}{\funcwidth}

    \raggedright \textbf{fl\_clear\_nmenu}(\textit{pFlObject})

    \vspace{-1.5ex}

    \rule{\textwidth}{0.5\fboxrule}
\setlength{\parskip}{2ex}

Removes all items from a nmenu object at once.

-{}-
\setlength{\parskip}{1ex}
      \textbf{Parameters}
      \vspace{-1ex}

      \begin{quote}
        \begin{Ventry}{xxxxxxxxx}

          \item[pFlObject]


nmenu object
            {\it (type=pointer to xfdata.FL\_OBJECT)}

        \end{Ventry}

      \end{quote}

      \textbf{Return Value}
    \vspace{-1ex}

      \begin{quote}

num.
      {\it (type=int)}

      \end{quote}

\textbf{Note:} 
e.g. \emph{todo}


\textbf{Status:} 
Untested + NoDoc + NoDemo = NOT OK


    \end{boxedminipage}

    \label{xformslib:flnmenu:fl_add_nmenu_items}
    \index{xformslib \textit{(package)}!xformslib.flnmenu \textit{(module)}!xformslib.flnmenu.fl\_add\_nmenu\_items \textit{(function)}}

    \vspace{0.5ex}

\hspace{.8\funcindent}\begin{boxedminipage}{\funcwidth}

    \raggedright \textbf{fl\_add\_nmenu\_items}(\textit{pFlObject}, \textit{itemstr})

    \vspace{-1.5ex}

    \rule{\textwidth}{0.5\fboxrule}
\setlength{\parskip}{2ex}

Adds items to an existing nmenu object.

-{}-
\setlength{\parskip}{1ex}
      \textbf{Parameters}
      \vspace{-1ex}

      \begin{quote}
        \begin{Ventry}{xxxxxxxxx}

          \item[pFlObject]


nmenu object
            {\it (type=pointer to xfdata.FL\_OBJECT)}

          \item[itemstr]


text of the item (among special sequences only \%S is supported)
            {\it (type=str)}

        \end{Ventry}

      \end{quote}

      \textbf{Return Value}
    \vspace{-1ex}

      \begin{quote}

popup entry
      {\it (type=pointer to xfdata.FL\_POPUP\_ENTRY)}

      \end{quote}

\textbf{Note:} 
e.g. \emph{todo}


\textbf{Status:} 
HalfTested + NoDoc + Demo = NOT OK (sequence param.)


    \end{boxedminipage}

    \label{xformslib:flnmenu:fl_insert_nmenu_items}
    \index{xformslib \textit{(package)}!xformslib.flnmenu \textit{(module)}!xformslib.flnmenu.fl\_insert\_nmenu\_items \textit{(function)}}

    \vspace{0.5ex}

\hspace{.8\funcindent}\begin{boxedminipage}{\funcwidth}

    \raggedright \textbf{fl\_insert\_nmenu\_items}(\textit{pFlObject}, \textit{pPopupEntry}, \textit{itemstr})

    \vspace{-1.5ex}

    \rule{\textwidth}{0.5\fboxrule}
\setlength{\parskip}{2ex}

Inserts additional items in nmenu object.

-{}-
\setlength{\parskip}{1ex}
      \textbf{Parameters}
      \vspace{-1ex}

      \begin{quote}
        \begin{Ventry}{xxxxxxxxxxx}

          \item[pFlObject]


nmenu object
            {\it (type=pointer to xfdata.FL\_OBJECT)}

          \item[pPopupEntry]


existing popup entry, after which the new items are to be inserted.
If it's 'None', it inserts items at the very start.
            {\it (type=pointer to xfdata.FL\_POPUP\_ENTRY)}

          \item[itemstr]


text of the item (among special sequences only \%S is supported)
            {\it (type=str)}

        \end{Ventry}

      \end{quote}

      \textbf{Return Value}
    \vspace{-1ex}

      \begin{quote}

popup entry
      {\it (type=pointer to xfdata.FL\_POPUP\_ENTRY)}

      \end{quote}

\textbf{Note:} 
e.g. \emph{todo}


\textbf{Status:} 
HalfTested + NoDoc + Demo = NOT OK (special sequences)


    \end{boxedminipage}

    \label{xformslib:flnmenu:fl_replace_nmenu_item}
    \index{xformslib \textit{(package)}!xformslib.flnmenu \textit{(module)}!xformslib.flnmenu.fl\_replace\_nmenu\_item \textit{(function)}}

    \vspace{0.5ex}

\hspace{.8\funcindent}\begin{boxedminipage}{\funcwidth}

    \raggedright \textbf{fl\_replace\_nmenu\_item}(\textit{pFlObject}, \textit{pPopupEntry}, \textit{itemstr})

    \vspace{-1.5ex}

    \rule{\textwidth}{0.5\fboxrule}
\setlength{\parskip}{2ex}

-{}-
\setlength{\parskip}{1ex}
      \textbf{Parameters}
      \vspace{-1ex}

      \begin{quote}
        \begin{Ventry}{xxxxxxxxxxx}

          \item[pFlObject]


nmenu object
            {\it (type=pointer to xfdata.FL\_OBJECT)}

          \item[pPopupEntry]


old popup entry to be replaced
            {\it (type=pointer to xfdata.FL\_POPUP\_ENTRY)}

          \item[itemstr]


text of the item (among special sequences only \%S is supported)
            {\it (type=str)}

        \end{Ventry}

      \end{quote}

      \textbf{Return Value}
    \vspace{-1ex}

      \begin{quote}

popup entry
      {\it (type=pointer to xfdata.FL\_POPUP\_ENTRY)}

      \end{quote}

\textbf{Note:} 
e.g. \emph{todo}


\textbf{Status:} 
Untested + NoDoc + NoDemo = NOT OK


    \end{boxedminipage}

    \label{xformslib:flnmenu:fl_delete_nmenu_item}
    \index{xformslib \textit{(package)}!xformslib.flnmenu \textit{(module)}!xformslib.flnmenu.fl\_delete\_nmenu\_item \textit{(function)}}

    \vspace{0.5ex}

\hspace{.8\funcindent}\begin{boxedminipage}{\funcwidth}

    \raggedright \textbf{fl\_delete\_nmenu\_item}(\textit{pFlObject}, \textit{pPopupEntry})

    \vspace{-1.5ex}

    \rule{\textwidth}{0.5\fboxrule}
\setlength{\parskip}{2ex}

Deletes an item froma nmenu object.

-{}-
\setlength{\parskip}{1ex}
      \textbf{Parameters}
      \vspace{-1ex}

      \begin{quote}
        \begin{Ventry}{xxxxxxxxxxx}

          \item[pFlObject]


nmenu object
            {\it (type=pointer to xfdata.FL\_OBJECT)}

          \item[pPopupEntry]


existing popup entry to delete
            {\it (type=pointer to xfdata.FL\_POPUP\_ENTRY)}

        \end{Ventry}

      \end{quote}

      \textbf{Return Value}
    \vspace{-1ex}

      \begin{quote}

num.
      {\it (type=int)}

      \end{quote}

\textbf{Note:} 
e.g. \emph{todo}


\textbf{Status:} 
Untested + NoDoc + NoDemo = NOT OK


    \end{boxedminipage}

    \label{xformslib:flnmenu:fl_set_nmenu_items}
    \index{xformslib \textit{(package)}!xformslib.flnmenu \textit{(module)}!xformslib.flnmenu.fl\_set\_nmenu\_items \textit{(function)}}

    \vspace{0.5ex}

\hspace{.8\funcindent}\begin{boxedminipage}{\funcwidth}

    \raggedright \textbf{fl\_set\_nmenu\_items}(\textit{pFlObject}, \textit{pPopupItem})

    \vspace{-1.5ex}

    \rule{\textwidth}{0.5\fboxrule}
\setlength{\parskip}{2ex}

Sets a popup nmenu item.

-{}-
\setlength{\parskip}{1ex}
      \textbf{Parameters}
      \vspace{-1ex}

      \begin{quote}
        \begin{Ventry}{xxxxxxxxxx}

          \item[pFlObject]


nmenu object
            {\it (type=pointer to xfdata.FL\_OBJECT)}

          \item[pPopupItem]


popup item to be set
            {\it (type=pointer to xfdata.FL\_POPUP\_ITEM)}

        \end{Ventry}

      \end{quote}

      \textbf{Return Value}
    \vspace{-1ex}

      \begin{quote}

first nmenu item, or None (on failure)
      {\it (type=pointer to xfdata.FL\_POPUP\_ENTRY)}

      \end{quote}

\textbf{Note:} 
e.g. \emph{todo}


\textbf{Status:} 
Untested + NoDoc + NoDemo = NOT OK


    \end{boxedminipage}

    \label{xformslib:flnmenu:fl_add_nmenu_items2}
    \index{xformslib \textit{(package)}!xformslib.flnmenu \textit{(module)}!xformslib.flnmenu.fl\_add\_nmenu\_items2 \textit{(function)}}

    \vspace{0.5ex}

\hspace{.8\funcindent}\begin{boxedminipage}{\funcwidth}

    \raggedright \textbf{fl\_add\_nmenu\_items2}(\textit{pFlObject}, \textit{pPopupItem})

    \vspace{-1.5ex}

    \rule{\textwidth}{0.5\fboxrule}
\setlength{\parskip}{2ex}

Adds items to a nmenu object (alternative).

-{}-
\setlength{\parskip}{1ex}
      \textbf{Parameters}
      \vspace{-1ex}

      \begin{quote}
        \begin{Ventry}{xxxxxxxxxx}

          \item[pFlObject]


nmenu object
            {\it (type=pointer to xfdata.FL\_OBJECT)}

          \item[pPopupItem]


popup item to be set. It needs to be prepared beforehand with
libr.create\_pPopupItem\_from\_list(..) function for single or multiple
lists, or with libr.create\_pPopupItem\_from\_dict(..) for a single dict.
            {\it (type=pointer to xfdata.FL\_POPUP\_ITEM)}

        \end{Ventry}

      \end{quote}

      \textbf{Return Value}
    \vspace{-1ex}

      \begin{quote}

first nmenu item, or None (on failure)
      {\it (type=pointer to xfdata.FL\_POPUP\_ENTRY)}

      \end{quote}

\textbf{Note:} 
e.g. \emph{todo}


\textbf{Status:} 
Tested + NoDoc + Demo = OK


    \end{boxedminipage}

    \label{xformslib:flnmenu:fl_insert_nmenu_items2}
    \index{xformslib \textit{(package)}!xformslib.flnmenu \textit{(module)}!xformslib.flnmenu.fl\_insert\_nmenu\_items2 \textit{(function)}}

    \vspace{0.5ex}

\hspace{.8\funcindent}\begin{boxedminipage}{\funcwidth}

    \raggedright \textbf{fl\_insert\_nmenu\_items2}(\textit{pFlObject}, \textit{pPopupEntry}, \textit{pPopupItem})

    \vspace{-1.5ex}

    \rule{\textwidth}{0.5\fboxrule}
\setlength{\parskip}{2ex}

Inserts items in a nmenu object (alternative).

-{}-
\setlength{\parskip}{1ex}
      \textbf{Parameters}
      \vspace{-1ex}

      \begin{quote}
        \begin{Ventry}{xxxxxxxxxxx}

          \item[pFlObject]


nmenu object
            {\it (type=pointer to xfdata.FL\_OBJECT)}

          \item[pPopupEntry]


existing popup entry, after which the new items are to be inserted.
If it is 'None', it inserts items at the very start.
            {\it (type=pointer to xfdata.FL\_POPUP\_ENTRY)}

          \item[pPopupItem]


popup item to be set. It needs to be prepared beforehand with
libr.create\_pPopupItem\_from\_list(..) function for single or multiple
lists, or with libr.create\_pPopupItem\_from\_dict(..) for a single dict.
            {\it (type=pointer to xfdata.FL\_POPUP\_ITEM)}

        \end{Ventry}

      \end{quote}

      \textbf{Return Value}
    \vspace{-1ex}

      \begin{quote}

first nmenu item, or None (on failure)
      {\it (type=pointer to xfdata.FL\_POPUP\_ENTRY)}

      \end{quote}

\textbf{Note:} 
e.g. \emph{todo}


\textbf{Status:} 
Tested + NoDoc + Demo = OK


    \end{boxedminipage}

    \label{xformslib:flnmenu:fl_replace_nmenu_items2}
    \index{xformslib \textit{(package)}!xformslib.flnmenu \textit{(module)}!xformslib.flnmenu.fl\_replace\_nmenu\_items2 \textit{(function)}}

    \vspace{0.5ex}

\hspace{.8\funcindent}\begin{boxedminipage}{\funcwidth}

    \raggedright \textbf{fl\_replace\_nmenu\_items2}(\textit{pFlObject}, \textit{pPopupEntry}, \textit{pPopupItem})

    \vspace{-1.5ex}

    \rule{\textwidth}{0.5\fboxrule}
\setlength{\parskip}{2ex}

Replaces an item of a nmenu object (alternative).

-{}-
\setlength{\parskip}{1ex}
      \textbf{Parameters}
      \vspace{-1ex}

      \begin{quote}
        \begin{Ventry}{xxxxxxxxxxx}

          \item[pFlObject]


nmenu object
            {\it (type=pointer to xfdata.FL\_OBJECT)}

          \item[pPopupEntry]


old popup entry to be replaced
            {\it (type=pointer to xfdata.FL\_POPUP\_ENTRY)}

          \item[pPopupItem]


new popup item. It needs to be prepared beforehand with
libr.create\_pPopupItem\_from\_list(..) function for single or multiple
lists, or with libr.create\_pPopupItem\_from\_dict(..) for a single dict.
            {\it (type=pointer to xfdata.FL\_POPUP\_ITEM)}

        \end{Ventry}

      \end{quote}

      \textbf{Return Value}
    \vspace{-1ex}

      \begin{quote}

first nmenu item, or None (on failure)
      {\it (type=pointer to xfdata.FL\_POPUP\_ENTRY)}

      \end{quote}

\textbf{Note:} 
e.g. \emph{todo}


\textbf{Status:} 
Tested + NoDoc + Demo = OK


    \end{boxedminipage}

    \label{xformslib:flnmenu:fl_get_nmenu_popup}
    \index{xformslib \textit{(package)}!xformslib.flnmenu \textit{(module)}!xformslib.flnmenu.fl\_get\_nmenu\_popup \textit{(function)}}

    \vspace{0.5ex}

\hspace{.8\funcindent}\begin{boxedminipage}{\funcwidth}

    \raggedright \textbf{fl\_get\_nmenu\_popup}(\textit{pFlObject})

    \vspace{-1.5ex}

    \rule{\textwidth}{0.5\fboxrule}
\setlength{\parskip}{2ex}

Determines which popup is associated with the nmenu object.

-{}-
\setlength{\parskip}{1ex}
      \textbf{Parameters}
      \vspace{-1ex}

      \begin{quote}
        \begin{Ventry}{xxxxxxxxx}

          \item[pFlObject]


nmenu object
            {\it (type=pointer to xfdata.FL\_OBJECT)}

        \end{Ventry}

      \end{quote}

      \textbf{Return Value}
    \vspace{-1ex}

      \begin{quote}

popup class instance (pPopup)
      {\it (type=pointer to xfdata.FL\_POPUP)}

      \end{quote}

\textbf{Note:} 
e.g. \emph{todo}


\textbf{Status:} 
Untested + NoDoc + NoDemo = NOT OK


    \end{boxedminipage}

    \label{xformslib:flnmenu:fl_set_nmenu_popup}
    \index{xformslib \textit{(package)}!xformslib.flnmenu \textit{(module)}!xformslib.flnmenu.fl\_set\_nmenu\_popup \textit{(function)}}

    \vspace{0.5ex}

\hspace{.8\funcindent}\begin{boxedminipage}{\funcwidth}

    \raggedright \textbf{fl\_set\_nmenu\_popup}(\textit{pFlObject}, \textit{pPopup})

    \vspace{-1.5ex}

    \rule{\textwidth}{0.5\fboxrule}
\setlength{\parskip}{2ex}

Sets an existing popup as the nmenu's popup. The popup you associate
with the nmenu object in this way can't be a sub-popup.

-{}-
\setlength{\parskip}{1ex}
      \textbf{Parameters}
      \vspace{-1ex}

      \begin{quote}
        \begin{Ventry}{xxxxxxxxx}

          \item[pFlObject]


nmenu object
            {\it (type=pointer to xfdata.FL\_OBJECT)}

          \item[pPopup]


popup class instance
            {\it (type=pointer to xfdata.FL\_POPUP)}

        \end{Ventry}

      \end{quote}

      \textbf{Return Value}
    \vspace{-1ex}

      \begin{quote}

popup class instance
      {\it (type=pointer to xfdata.FL\_POPUP)}

      \end{quote}

\textbf{Note:} 
e.g. \emph{todo}


\textbf{Status:} 
Untested + NoDoc + NoDemo = NOT OK


    \end{boxedminipage}

    \label{xformslib:flnmenu:fl_get_nmenu_item}
    \index{xformslib \textit{(package)}!xformslib.flnmenu \textit{(module)}!xformslib.flnmenu.fl\_get\_nmenu\_item \textit{(function)}}

    \vspace{0.5ex}

\hspace{.8\funcindent}\begin{boxedminipage}{\funcwidth}

    \raggedright \textbf{fl\_get\_nmenu\_item}(\textit{pFlObject})

    \vspace{-1.5ex}

    \rule{\textwidth}{0.5\fboxrule}
\setlength{\parskip}{2ex}

Finds out which item of a nmenu object was selected.

-{}-
\setlength{\parskip}{1ex}
      \textbf{Parameters}
      \vspace{-1ex}

      \begin{quote}
        \begin{Ventry}{xxxxxxxxx}

          \item[pFlObject]


nmenu object
            {\it (type=pointer to xfdata.FL\_OBJECT)}

        \end{Ventry}

      \end{quote}

      \textbf{Return Value}
    \vspace{-1ex}

      \begin{quote}

popup return class instance (pPopupReturn)), or None (if no
selection was made the last time the nmenu object was used)
      {\it (type=pointer to xfdata.FL\_POPUP\_RETURN)}

      \end{quote}

\textbf{Note:} 
e.g. \emph{todo}


\textbf{Status:} 
Tested + NoDoc + Demo = OK


    \end{boxedminipage}

    \label{xformslib:flnmenu:fl_get_nmenu_item_by_value}
    \index{xformslib \textit{(package)}!xformslib.flnmenu \textit{(module)}!xformslib.flnmenu.fl\_get\_nmenu\_item\_by\_value \textit{(function)}}

    \vspace{0.5ex}

\hspace{.8\funcindent}\begin{boxedminipage}{\funcwidth}

    \raggedright \textbf{fl\_get\_nmenu\_item\_by\_value}(\textit{pFlObject}, \textit{value})

    \vspace{-1.5ex}

    \rule{\textwidth}{0.5\fboxrule}
\setlength{\parskip}{2ex}

Searches through the list of all items (including items in sub-popups)
and returns the first one with the val associated with the item

-{}-
\setlength{\parskip}{1ex}
      \textbf{Parameters}
      \vspace{-1ex}

      \begin{quote}
        \begin{Ventry}{xxxxxxxxx}

          \item[pFlObject]


nmenu object
            {\it (type=pointer to xfdata.FL\_OBJECT)}

          \item[value]


value corresponding to an item to be searched.
            {\it (type=long)}

        \end{Ventry}

      \end{quote}

      \textbf{Return Value}
    \vspace{-1ex}

      \begin{quote}

first item associated, or None (if none is found)
      {\it (type=pointer to xfdata.FL\_POPUP\_ENTRY)}

      \end{quote}

\textbf{Note:} 
e.g. \emph{todo}


\textbf{Status:} 
Tested + NoDoc + Demo = OK


    \end{boxedminipage}

    \label{xformslib:flnmenu:fl_get_nmenu_item_by_label}
    \index{xformslib \textit{(package)}!xformslib.flnmenu \textit{(module)}!xformslib.flnmenu.fl\_get\_nmenu\_item\_by\_label \textit{(function)}}

    \vspace{0.5ex}

\hspace{.8\funcindent}\begin{boxedminipage}{\funcwidth}

    \raggedright \textbf{fl\_get\_nmenu\_item\_by\_label}(\textit{pFlObject}, \textit{label})

    \vspace{-1.5ex}

    \rule{\textwidth}{0.5\fboxrule}
\setlength{\parskip}{2ex}

Searches for a certain label as displayed for the item in the nmenu's
popup.

-{}-
\setlength{\parskip}{1ex}
      \textbf{Parameters}
      \vspace{-1ex}

      \begin{quote}
        \begin{Ventry}{xxxxxxxxx}

          \item[pFlObject]


nmenu object
            {\it (type=pointer to xfdata.FL\_OBJECT)}

          \item[label]


text associated with an item.
            {\it (type=str)}

        \end{Ventry}

      \end{quote}

      \textbf{Return Value}
    \vspace{-1ex}

      \begin{quote}

first item associated, or None (if none is found)
      {\it (type=pointer to xfdata.FL\_POPUP\_ENTRY)}

      \end{quote}

\textbf{Note:} 
e.g. \emph{todo}


\textbf{Status:} 
Untested + NoDoc + NoDemo = NOT OK


    \end{boxedminipage}

    \label{xformslib:flnmenu:fl_get_nmenu_item_by_text}
    \index{xformslib \textit{(package)}!xformslib.flnmenu \textit{(module)}!xformslib.flnmenu.fl\_get\_nmenu\_item\_by\_text \textit{(function)}}

    \vspace{0.5ex}

\hspace{.8\funcindent}\begin{boxedminipage}{\funcwidth}

    \raggedright \textbf{fl\_get\_nmenu\_item\_by\_text}(\textit{pFlObject}, \textit{text})

    \vspace{-1.5ex}

    \rule{\textwidth}{0.5\fboxrule}
\setlength{\parskip}{2ex}

Searches for the text the item in nmenu object was created by (that
might be the same as the label text in simple cases).

-{}-
\setlength{\parskip}{1ex}
      \textbf{Parameters}
      \vspace{-1ex}

      \begin{quote}
        \begin{Ventry}{xxxxxxxxx}

          \item[pFlObject]


nmenu object
            {\it (type=pointer to xfdata.FL\_OBJECT)}

          \item[text]


text associated with an item.
            {\it (type=str)}

        \end{Ventry}

      \end{quote}

      \textbf{Return Value}
    \vspace{-1ex}

      \begin{quote}

first item associated, or None (if none is found)
      {\it (type=pointer to xfdata.FL\_POPUP\_ENTRY)}

      \end{quote}

\textbf{Note:} 
e.g. \emph{todo}


\textbf{Status:} 
Untested + NoDoc + NoDemo = NOT OK


    \end{boxedminipage}

    \label{xformslib:flnmenu:fl_set_nmenu_policy}
    \index{xformslib \textit{(package)}!xformslib.flnmenu \textit{(module)}!xformslib.flnmenu.fl\_set\_nmenu\_policy \textit{(function)}}

    \vspace{0.5ex}

\hspace{.8\funcindent}\begin{boxedminipage}{\funcwidth}

    \raggedright \textbf{fl\_set\_nmenu\_policy}(\textit{pFlObject}, \textit{policy})

    \vspace{-1.5ex}

    \rule{\textwidth}{0.5\fboxrule}
\setlength{\parskip}{2ex}

Changes nmenu policy about closing, so that the popup also gets closed
(without a selection) when the mouse button is clicked or released on a
non-selectable item (giving the impression of a ``pull-down'' menu). By
default, the popup is closed when an item is selected or (without a
selection) when the user clicks somehwere outside of the popups area.

-{}-
\setlength{\parskip}{1ex}
      \textbf{Parameters}
      \vspace{-1ex}

      \begin{quote}
        \begin{Ventry}{xxxxxxxxx}

          \item[pFlObject]


nmenu object
            {\it (type=pointer to xfdata.FL\_OBJECT)}

          \item[policy]


under which conditions the nmenu's popup gets closed. Values (from
xfdata.py) FL\_POPUP\_NORMAL\_SELECT (default) or FL\_POPUP\_DRAG\_SELECT
            {\it (type=int)}

        \end{Ventry}

      \end{quote}

      \textbf{Return Value}
    \vspace{-1ex}

      \begin{quote}

old policy settings, or -1 (on failure)
      {\it (type=int)}

      \end{quote}

\textbf{Note:} 
e.g. \emph{todo}


\textbf{Status:} 
Untested + NoDoc + NoDemo = NOT OK


    \end{boxedminipage}

    \label{xformslib:flnmenu:fl_set_nmenu_hl_text_color}
    \index{xformslib \textit{(package)}!xformslib.flnmenu \textit{(module)}!xformslib.flnmenu.fl\_set\_nmenu\_hl\_text\_color \textit{(function)}}

    \vspace{0.5ex}

\hspace{.8\funcindent}\begin{boxedminipage}{\funcwidth}

    \raggedright \textbf{fl\_set\_nmenu\_hl\_text\_color}(\textit{pFlObject}, \textit{colr})

    \vspace{-1.5ex}

    \rule{\textwidth}{0.5\fboxrule}
\setlength{\parskip}{2ex}

Sets the color of label when it is in ``active'' state (i.e. while the
popup is shown). In ``inactive'' state this is set by fl\_set\_object\_lcol().
By default, this color is xfdata.FL\_BLACK for nmenus that are shown as a
button while being ``active'', while for normal nmenus it?s the same color
that is used items in the popup when the mouse is hovering over them.

-{}-
\setlength{\parskip}{1ex}
      \textbf{Parameters}
      \vspace{-1ex}

      \begin{quote}
        \begin{Ventry}{xxxxxxxxx}

          \item[pFlObject]


nmenu object
            {\it (type=pointer to xfdata.FL\_OBJECT)}

          \item[colr]


color to be set
            {\it (type=long\_pos)}

        \end{Ventry}

      \end{quote}

      \textbf{Return Value}
    \vspace{-1ex}

      \begin{quote}

old color, or xfdata.FL\_MAX\_COLORS (on failure)
      {\it (type=long\_pos)}

      \end{quote}

\textbf{Note:} 
e.g. \emph{todo}


\textbf{Status:} 
Untested + NoDoc + NoDemo = NOT OK


    \end{boxedminipage}

    \index{xformslib \textit{(package)}!xformslib.flnmenu \textit{(module)}|)}
