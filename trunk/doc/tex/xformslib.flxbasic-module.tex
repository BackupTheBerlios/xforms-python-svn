%
% API Documentation for API Documentation
% Module xformslib.flxbasic
%
% Generated by epydoc 3.0.1
% [Fri May 21 19:29:33 2010]
%

%%%%%%%%%%%%%%%%%%%%%%%%%%%%%%%%%%%%%%%%%%%%%%%%%%%%%%%%%%%%%%%%%%%%%%%%%%%
%%                          Module Description                           %%
%%%%%%%%%%%%%%%%%%%%%%%%%%%%%%%%%%%%%%%%%%%%%%%%%%%%%%%%%%%%%%%%%%%%%%%%%%%

    \index{xformslib \textit{(package)}!xformslib.flxbasic \textit{(module)}|(}
\section{Module xformslib.flxbasic}

    \label{xformslib:flxbasic}

xforms-python's functions to handle X Window basic and drawing stuff.

Copyright (C) 2009, 2010  Luca Lazzaroni ``LukenShiro''
e-mail: <\href{mailto:lukenshiro@ngi.it}{lukenshiro@ngi.it}>

This program is free software: you can redistribute it and/or modify
it under the terms of the GNU Lesser General Public License as
published by the Free Software Foundation, version 2.1 of the License.

This program is distributed in the hope that it will be useful,
but WITHOUT ANY WARRANTY; without even the implied warranty of
MERCHANTABILITY or FITNESS FOR A PARTICULAR PURPOSE. See the
GNU Lesser General Public License for more details.

You should have received a copy of the GNU LGPL along with this
program. If not, see <\href{http://www.gnu.org/licenses/}{http://www.gnu.org/licenses/}>.

See CREDITS file to read acknowledgements and thanks to XForms,
ctypes and other developers.

%%%%%%%%%%%%%%%%%%%%%%%%%%%%%%%%%%%%%%%%%%%%%%%%%%%%%%%%%%%%%%%%%%%%%%%%%%%
%%                               Functions                               %%
%%%%%%%%%%%%%%%%%%%%%%%%%%%%%%%%%%%%%%%%%%%%%%%%%%%%%%%%%%%%%%%%%%%%%%%%%%%

  \subsection{Functions}

    \label{xformslib:flxbasic:FL_is_gray}
    \index{xformslib \textit{(package)}!xformslib.flxbasic \textit{(module)}!xformslib.flxbasic.FL\_is\_gray \textit{(function)}}

    \vspace{0.5ex}

\hspace{.8\funcindent}\begin{boxedminipage}{\funcwidth}

    \raggedright \textbf{FL\_is\_gray}(\textit{v})

\setlength{\parskip}{2ex}
\setlength{\parskip}{1ex}
    \end{boxedminipage}

    \label{xformslib:flxbasic:FL_is_rgb}
    \index{xformslib \textit{(package)}!xformslib.flxbasic \textit{(module)}!xformslib.flxbasic.FL\_is\_rgb \textit{(function)}}

    \vspace{0.5ex}

\hspace{.8\funcindent}\begin{boxedminipage}{\funcwidth}

    \raggedright \textbf{FL\_is\_rgb}(\textit{v})

\setlength{\parskip}{2ex}
\setlength{\parskip}{1ex}
    \end{boxedminipage}

    \label{xformslib:flxbasic:fl_get_vclass}
    \index{xformslib \textit{(package)}!xformslib.flxbasic \textit{(module)}!xformslib.flxbasic.fl\_get\_vclass \textit{(function)}}

    \vspace{0.5ex}

\hspace{.8\funcindent}\begin{boxedminipage}{\funcwidth}

    \raggedright \textbf{fl\_get\_vclass}()

\setlength{\parskip}{2ex}
\setlength{\parskip}{1ex}
    \end{boxedminipage}

    \label{xformslib:flxbasic:fl_get_form_vclass}
    \index{xformslib \textit{(package)}!xformslib.flxbasic \textit{(module)}!xformslib.flxbasic.fl\_get\_form\_vclass \textit{(function)}}

    \vspace{0.5ex}

\hspace{.8\funcindent}\begin{boxedminipage}{\funcwidth}

    \raggedright \textbf{fl\_get\_form\_vclass}(\textit{a})

\setlength{\parskip}{2ex}
\setlength{\parskip}{1ex}
    \end{boxedminipage}

    \label{xformslib:flxbasic:fl_get_gc}
    \index{xformslib \textit{(package)}!xformslib.flxbasic \textit{(module)}!xformslib.flxbasic.fl\_get\_gc \textit{(function)}}

    \vspace{0.5ex}

\hspace{.8\funcindent}\begin{boxedminipage}{\funcwidth}

    \raggedright \textbf{fl\_get\_gc}()

\setlength{\parskip}{2ex}
\setlength{\parskip}{1ex}
    \end{boxedminipage}

    \label{xformslib:flxbasic:fl_mode_capable}
    \index{xformslib \textit{(package)}!xformslib.flxbasic \textit{(module)}!xformslib.flxbasic.fl\_mode\_capable \textit{(function)}}

    \vspace{0.5ex}

\hspace{.8\funcindent}\begin{boxedminipage}{\funcwidth}

    \raggedright \textbf{fl\_mode\_capable}(\textit{mode}, \textit{warn})

    \vspace{-1.5ex}

    \rule{\textwidth}{0.5\fboxrule}
\setlength{\parskip}{2ex}

Returns if the system is capable of displaying in the specified visual
class, or not.

-{}-
\setlength{\parskip}{1ex}
      \textbf{Parameters}
      \vspace{-1ex}

      \begin{quote}
        \begin{Ventry}{xxxx}

          \item[mode]


visual mode. Values (from xfdata module) GrayScale, StaticGray,
PseudoColor, StaticColor, DirectColor or TrueColor
            {\it (type=int)}

          \item[warn]


if set a warning is printed out in case the capability asked for
isn't available. Values 0 (don't print warning) or 1 (print warning)
            {\it (type=int)}

        \end{Ventry}

      \end{quote}

      \textbf{Return Value}
    \vspace{-1ex}

      \begin{quote}

flag 1 (if capable) or 0 otherwise
      {\it (type=int)}

      \end{quote}

\textbf{Note:} 
e.g. capable = fl\_mode\_capable(xfdata.GrayScale, 1)


\textbf{Status:} 
Tested + Doc + NoDemo = OK


    \end{boxedminipage}

    \label{xformslib:flxbasic:fl_default_win}
    \index{xformslib \textit{(package)}!xformslib.flxbasic \textit{(module)}!xformslib.flxbasic.fl\_default\_win \textit{(function)}}

    \vspace{0.5ex}

\hspace{.8\funcindent}\begin{boxedminipage}{\funcwidth}

    \raggedright \textbf{fl\_default\_win}()

\setlength{\parskip}{2ex}
\setlength{\parskip}{1ex}
    \end{boxedminipage}

    \label{xformslib:flxbasic:fl_default_window}
    \index{xformslib \textit{(package)}!xformslib.flxbasic \textit{(module)}!xformslib.flxbasic.fl\_default\_window \textit{(function)}}

    \vspace{0.5ex}

\hspace{.8\funcindent}\begin{boxedminipage}{\funcwidth}

    \raggedright \textbf{fl\_default\_window}()

\setlength{\parskip}{2ex}
\setlength{\parskip}{1ex}
    \end{boxedminipage}

    \label{xformslib:flxbasic:fl_rectangle}
    \index{xformslib \textit{(package)}!xformslib.flxbasic \textit{(module)}!xformslib.flxbasic.fl\_rectangle \textit{(function)}}

    \vspace{0.5ex}

\hspace{.8\funcindent}\begin{boxedminipage}{\funcwidth}

    \raggedright \textbf{fl\_rectangle}(\textit{fill}, \textit{x}, \textit{y}, \textit{w}, \textit{h}, \textit{colr})

    \vspace{-1.5ex}

    \rule{\textwidth}{0.5\fboxrule}
\setlength{\parskip}{2ex}

Draws a rectangle.

-{}-
\setlength{\parskip}{1ex}
      \textbf{Parameters}
      \vspace{-1ex}

      \begin{quote}
        \begin{Ventry}{xxxx}

          \item[fill]


flag if the rectangle has to be filled or just the outline is needed.
Values 0 (the outline only) or 1 (filled).
            {\it (type=int)}

          \item[x]


horizontal position (upper-left corner)
            {\it (type=int)}

          \item[y]


vertical position (upper-left corner)
            {\it (type=int)}

          \item[w]


width in coord units
            {\it (type=int)}

          \item[h]


height in coord units
            {\it (type=int)}

          \item[colr]


color value
            {\it (type=long\_pos)}

        \end{Ventry}

      \end{quote}

\textbf{Note:} 
e.g. fl\_rectangle(1, 100, 200, 300, 200, xfdata.FL\_BEIGE)


\textbf{Status:} 
Tested + Doc + Demo = OK


    \end{boxedminipage}

    \label{xformslib:flxbasic:fl_rectbound}
    \index{xformslib \textit{(package)}!xformslib.flxbasic \textit{(module)}!xformslib.flxbasic.fl\_rectbound \textit{(function)}}

    \vspace{0.5ex}

\hspace{.8\funcindent}\begin{boxedminipage}{\funcwidth}

    \raggedright \textbf{fl\_rectbound}(\textit{x}, \textit{y}, \textit{w}, \textit{h}, \textit{colr})

    \vspace{-1.5ex}

    \rule{\textwidth}{0.5\fboxrule}
\setlength{\parskip}{2ex}

Draws a filled rectangle with a black border.

-{}-
\setlength{\parskip}{1ex}
      \textbf{Parameters}
      \vspace{-1ex}

      \begin{quote}
        \begin{Ventry}{xxxx}

          \item[x]


horizontal position (upper-left corner)
            {\it (type=int)}

          \item[y]


vertical position (upper-left corner)
            {\it (type=int)}

          \item[w]


width in coord units
            {\it (type=int)}

          \item[h]


height in coord units
            {\it (type=int)}

          \item[colr]


color value
            {\it (type=long\_pos)}

        \end{Ventry}

      \end{quote}

\textbf{Note:} 
e.g. fl\_rectbound(100, 200, 300, 200, xfdata.FL\_PINK)


\textbf{Status:} 
Tested + Doc + NoDemo = OK


    \end{boxedminipage}

    \label{xformslib:flxbasic:fl_rectf}
    \index{xformslib \textit{(package)}!xformslib.flxbasic \textit{(module)}!xformslib.flxbasic.fl\_rectf \textit{(function)}}

    \vspace{0.5ex}

\hspace{.8\funcindent}\begin{boxedminipage}{\funcwidth}

    \raggedright \textbf{fl\_rectf}(\textit{x}, \textit{y}, \textit{w}, \textit{h}, \textit{colr})

    \vspace{-1.5ex}

    \rule{\textwidth}{0.5\fboxrule}
\setlength{\parskip}{2ex}

Draws a filled rectangle on the screen.

-{}-
\setlength{\parskip}{1ex}
      \textbf{Parameters}
      \vspace{-1ex}

      \begin{quote}
        \begin{Ventry}{xxxx}

          \item[x]


horizontal position (upper-left corner)
            {\it (type=int)}

          \item[y]


vertical position (upper-left corner)
            {\it (type=int)}

          \item[w]


width in coord units
            {\it (type=int)}

          \item[h]


height in coord units
            {\it (type=int)}

          \item[colr]


color value
            {\it (type=long\_pos)}

        \end{Ventry}

      \end{quote}

\textbf{Note:} 
e.g. fl\_rectf(150, 220, 300, 200, xfdata.FL\_TOMATO)


\textbf{Status:} 
Tested + Doc + Demo = OK


    \end{boxedminipage}

    \label{xformslib:flxbasic:fl_rect}
    \index{xformslib \textit{(package)}!xformslib.flxbasic \textit{(module)}!xformslib.flxbasic.fl\_rect \textit{(function)}}

    \vspace{0.5ex}

\hspace{.8\funcindent}\begin{boxedminipage}{\funcwidth}

    \raggedright \textbf{fl\_rect}(\textit{x}, \textit{y}, \textit{w}, \textit{h}, \textit{colr})

    \vspace{-1.5ex}

    \rule{\textwidth}{0.5\fboxrule}
\setlength{\parskip}{2ex}

Draws a rectangle's outline on the screen.

-{}-
\setlength{\parskip}{1ex}
      \textbf{Parameters}
      \vspace{-1ex}

      \begin{quote}
        \begin{Ventry}{xxxx}

          \item[x]


horizontal position (upper-left corner)
            {\it (type=int)}

          \item[y]


vertical position (upper-left corner)
            {\it (type=int)}

          \item[w]


width in coord units
            {\it (type=int)}

          \item[h]


height in coord units
            {\it (type=int)}

          \item[colr]


color value
            {\it (type=long\_pos)}

        \end{Ventry}

      \end{quote}

\textbf{Note:} 
e.g. fl\_rect(100, 200, 300, 200, xfdata.FL\_SLATEBLUE)


\textbf{Status:} 
Tested + Doc + Demo = OK


    \end{boxedminipage}

    \label{xformslib:flxbasic:fl_roundrectangle}
    \index{xformslib \textit{(package)}!xformslib.flxbasic \textit{(module)}!xformslib.flxbasic.fl\_roundrectangle \textit{(function)}}

    \vspace{0.5ex}

\hspace{.8\funcindent}\begin{boxedminipage}{\funcwidth}

    \raggedright \textbf{fl\_roundrectangle}(\textit{fill}, \textit{x}, \textit{y}, \textit{w}, \textit{h}, \textit{colr})

    \vspace{-1.5ex}

    \rule{\textwidth}{0.5\fboxrule}
\setlength{\parskip}{2ex}

Draws a rectangle with rounded corners (filled or just the outline).

-{}-
\setlength{\parskip}{1ex}
      \textbf{Parameters}
      \vspace{-1ex}

      \begin{quote}
        \begin{Ventry}{xxxx}

          \item[fill]


flag if the rectangle has to be filled or just the outline is
needed. Values 0 (the outline only) or 1 (filled)
            {\it (type=int)}

          \item[x]


horizontal position (upper-left corner)
            {\it (type=int)}

          \item[y]


vertical position (upper-left corner)
            {\it (type=int)}

          \item[w]


width in coord units
            {\it (type=int)}

          \item[h]


height in coord units
            {\it (type=int)}

          \item[colr]


color value
            {\it (type=long\_pos)}

        \end{Ventry}

      \end{quote}

\textbf{Note:} 
e.g. fl\_roundrectangle(1, 100, 200, 300, 200, xfdata.FL\_MAGENTA)


\textbf{Status:} 
Tested + Doc + NoDemo = OK


    \end{boxedminipage}

    \label{xformslib:flxbasic:fl_roundrectf}
    \index{xformslib \textit{(package)}!xformslib.flxbasic \textit{(module)}!xformslib.flxbasic.fl\_roundrectf \textit{(function)}}

    \vspace{0.5ex}

\hspace{.8\funcindent}\begin{boxedminipage}{\funcwidth}

    \raggedright \textbf{fl\_roundrectf}(\textit{x}, \textit{y}, \textit{w}, \textit{h}, \textit{colr})

    \vspace{-1.5ex}

    \rule{\textwidth}{0.5\fboxrule}
\setlength{\parskip}{2ex}

Draws a filled rectangle with rounded corners.

-{}-
\setlength{\parskip}{1ex}
      \textbf{Parameters}
      \vspace{-1ex}

      \begin{quote}
        \begin{Ventry}{xxxx}

          \item[x]


horizontal position (upper-left corner)
            {\it (type=int)}

          \item[y]


vertical position (upper-left corner)
            {\it (type=int)}

          \item[w]


width in coord units
            {\it (type=int)}

          \item[h]


height in coord units
            {\it (type=int)}

          \item[colr]


color value
            {\it (type=long\_pos)}

        \end{Ventry}

      \end{quote}

\textbf{Note:} 
e.g. fl\_roundrectf(100, 200, 300, 200, xfdata.FL\_CYAN)


\textbf{Status:} 
Tested + Doc + NoDemo = OK


    \end{boxedminipage}

    \label{xformslib:flxbasic:fl_roundrect}
    \index{xformslib \textit{(package)}!xformslib.flxbasic \textit{(module)}!xformslib.flxbasic.fl\_roundrect \textit{(function)}}

    \vspace{0.5ex}

\hspace{.8\funcindent}\begin{boxedminipage}{\funcwidth}

    \raggedright \textbf{fl\_roundrect}(\textit{x}, \textit{y}, \textit{w}, \textit{h}, \textit{colr})

    \vspace{-1.5ex}

    \rule{\textwidth}{0.5\fboxrule}
\setlength{\parskip}{2ex}

Draws a rectangle's outline with rounded corners.

-{}-
\setlength{\parskip}{1ex}
      \textbf{Parameters}
      \vspace{-1ex}

      \begin{quote}
        \begin{Ventry}{xxxx}

          \item[x]


horizontal position (upper-left corner)
            {\it (type=int)}

          \item[y]


vertical position (upper-left corner)
            {\it (type=int)}

          \item[w]


width in coord units
            {\it (type=int)}

          \item[h]


height in coord units
            {\it (type=int)}

          \item[colr]


color value
            {\it (type=long\_pos)}

        \end{Ventry}

      \end{quote}

\textbf{Note:} 
e.g. fl\_roundrect(100, 200, 300, 200, xfdata.Fl\_INDIANRED)


\textbf{Status:} 
Tested + Doc + NoDemo = OK


    \end{boxedminipage}

    \label{xformslib:flxbasic:fl_polygon}
    \index{xformslib \textit{(package)}!xformslib.flxbasic \textit{(module)}!xformslib.flxbasic.fl\_polygon \textit{(function)}}

    \vspace{0.5ex}

\hspace{.8\funcindent}\begin{boxedminipage}{\funcwidth}

    \raggedright \textbf{fl\_polygon}(\textit{fill}, \textit{Point}, \textit{numpt}, \textit{colr})

    \vspace{-1.5ex}

    \rule{\textwidth}{0.5\fboxrule}
\setlength{\parskip}{2ex}

Draws a generic polygon on the screen (filled or just an outline).

-{}-
\setlength{\parskip}{1ex}
      \textbf{Parameters}
      \vspace{-1ex}

      \begin{quote}
        \begin{Ventry}{xxxxx}

          \item[fill]


if polygon has to be filled or just an outline is needed. Values 1
(if filled) or 0 (an outline only)
            {\it (type=int)}

          \item[Point]


an array of point class instances
            {\it (type=array of xfdata.FL\_POINT)}

          \item[numpt]


number of points
            {\it (type=int)}

          \item[colr]


value of color to be set
            {\it (type=long\_pos)}

        \end{Ventry}

      \end{quote}

\textbf{Notes:}
\begin{quote}
  \begin{itemize}

  \item
    \setlength{\parskip}{0.6ex}

e.g. pointmap = (FL\_POINT * 4)()


  \item 
e.g. pointmap{[}0{]}.x = 12 ; pointmap{[}0{]}.y = 32


  \item 
e.g. pointmap{[}1{]}.x = 24 ; pointmap{[}1{]}.y = 100


  \item 
e.g. pointmap{[}2{]}.x = 87 ; pointmap{[}0{]}.y = 132


  \item 
e.g. fl\_polygon(1, pointmap, 3, xfdata.FL\_PALEGREEN)


\end{itemize}

\end{quote}

\textbf{Status:} 
Tested + Doc + Demo = OK


    \end{boxedminipage}

    \label{xformslib:flxbasic:fl_polyf}
    \index{xformslib \textit{(package)}!xformslib.flxbasic \textit{(module)}!xformslib.flxbasic.fl\_polyf \textit{(function)}}

    \vspace{0.5ex}

\hspace{.8\funcindent}\begin{boxedminipage}{\funcwidth}

    \raggedright \textbf{fl\_polyf}(\textit{Point}, \textit{numpt}, \textit{colr})

    \vspace{-1.5ex}

    \rule{\textwidth}{0.5\fboxrule}
\setlength{\parskip}{2ex}

Draws a generic filled polygon on the screen.

-{}-
\setlength{\parskip}{1ex}
      \textbf{Parameters}
      \vspace{-1ex}

      \begin{quote}
        \begin{Ventry}{xxxxx}

          \item[Point]


an array of point class instances
            {\it (type=array of xfdata.FL\_POINT)}

          \item[numpt]


number of points
            {\it (type=int)}

          \item[colr]


value of color to be set
            {\it (type=long\_pos)}

        \end{Ventry}

      \end{quote}

\textbf{Notes:}
\begin{quote}
  \begin{itemize}

  \item
    \setlength{\parskip}{0.6ex}

e.g. pointmap = (FL\_POINT * 4)() ;


  \item 
e.g. pointmap{[}0{]}.x = 12 ; pointmap{[}0{]}.y = 32 ;


  \item 
e.g. pointmap{[}1{]}.x = 24 ; pointmap{[}1{]}.y = 100 ;


  \item 
e.g. pointmap{[}2{]}.x = 87 ; pointmap{[}0{]}.y = 132 ;


  \item 
e.g. fl\_polyf(pointmap, 3, xfdata.FL\_PALEGREEN)


\end{itemize}

\end{quote}

\textbf{Status:} 
Tested + Doc + Demo = OK


    \end{boxedminipage}

    \label{xformslib:flxbasic:fl_polyl}
    \index{xformslib \textit{(package)}!xformslib.flxbasic \textit{(module)}!xformslib.flxbasic.fl\_polyl \textit{(function)}}

    \vspace{0.5ex}

\hspace{.8\funcindent}\begin{boxedminipage}{\funcwidth}

    \raggedright \textbf{fl\_polyl}(\textit{Point}, \textit{numpt}, \textit{colr})

    \vspace{-1.5ex}

    \rule{\textwidth}{0.5\fboxrule}
\setlength{\parskip}{2ex}

Draws a generic polygon's outline on the screen.

-{}-
\setlength{\parskip}{1ex}
      \textbf{Parameters}
      \vspace{-1ex}

      \begin{quote}
        \begin{Ventry}{xxxxx}

          \item[Point]


an array of point class instances
            {\it (type=array of xfdata.FL\_POINT)}

          \item[numpt]


number of points
            {\it (type=int)}

          \item[colr]


value of color to be set
            {\it (type=long\_pos)}

        \end{Ventry}

      \end{quote}

\textbf{Notes:}
\begin{quote}
  \begin{itemize}

  \item
    \setlength{\parskip}{0.6ex}

e.g. pointmap = (FL\_POINT * 4)() ;


  \item 
e.g. pointmap{[}0{]}.x = 12 ; pointmap{[}0{]}.y = 32 ;


  \item 
e.g. pointmap{[}1{]}.x = 24 ; pointmap{[}1{]}.y = 100 ;


  \item 
e.g. pointmap{[}2{]}.x = 87 ; pointmap{[}0{]}.y = 132 ;


  \item 
e.g. fl\_polyl(pointmap, 3, xfdata.FL\_ORCHID)


\end{itemize}

\end{quote}

\textbf{Status:} 
Tested + Doc + NoDemo = OK


    \end{boxedminipage}

    \label{xformslib:flxbasic:fl_polybound}
    \index{xformslib \textit{(package)}!xformslib.flxbasic \textit{(module)}!xformslib.flxbasic.fl\_polybound \textit{(function)}}

    \vspace{0.5ex}

\hspace{.8\funcindent}\begin{boxedminipage}{\funcwidth}

    \raggedright \textbf{fl\_polybound}(\textit{Point}, \textit{numpt}, \textit{colr})

    \vspace{-1.5ex}

    \rule{\textwidth}{0.5\fboxrule}
\setlength{\parskip}{2ex}

Draws a generic filled polygon with a black border in the screen.

-{}-
\setlength{\parskip}{1ex}
      \textbf{Parameters}
      \vspace{-1ex}

      \begin{quote}
        \begin{Ventry}{xxxxx}

          \item[Point]


an array of point class instances
            {\it (type=array of xfdata.FL\_POINT)}

          \item[numpt]


number of points
            {\it (type=int)}

          \item[colr]


value of color to be set
            {\it (type=long\_pos)}

        \end{Ventry}

      \end{quote}

\textbf{Notes:}
\begin{quote}
  \begin{itemize}

  \item
    \setlength{\parskip}{0.6ex}

e.g. pointmap = (FL\_POINT * 4)() ;


  \item 
e.g. pointmap{[}0{]}.x = 12 ; pointmap{[}0{]}.y = 32 ;


  \item 
e.g. pointmap{[}1{]}.x = 24 ; pointmap{[}1{]}.y = 100 ;


  \item 
e.g. pointmap{[}2{]}.x = 87 ; pointmap{[}0{]}.y = 132 ;


  \item 
e.g. fl\_polybound(pointmap, 3, xfdata.FL\_DARKGOLD)


\end{itemize}

\end{quote}

\textbf{Status:} 
Tested + Doc + NoDemo = OK


    \end{boxedminipage}

    \label{xformslib:flxbasic:fl_lines}
    \index{xformslib \textit{(package)}!xformslib.flxbasic \textit{(module)}!xformslib.flxbasic.fl\_lines \textit{(function)}}

    \vspace{0.5ex}

\hspace{.8\funcindent}\begin{boxedminipage}{\funcwidth}

    \raggedright \textbf{fl\_lines}(\textit{Point}, \textit{numpt}, \textit{colr})

    \vspace{-1.5ex}

    \rule{\textwidth}{0.5\fboxrule}
\setlength{\parskip}{2ex}

Draws connected line segments between a number of points.

-{}-
\setlength{\parskip}{1ex}
      \textbf{Parameters}
      \vspace{-1ex}

      \begin{quote}
        \begin{Ventry}{xxxxx}

          \item[Point]


an array of point class instances
            {\it (type=array of xfdata.FL\_POINT)}

          \item[numpt]


number of points
            {\it (type=int)}

          \item[colr]


value of color to be set
            {\it (type=long\_pos)}

        \end{Ventry}

      \end{quote}

\textbf{Notes:}
\begin{quote}
  \begin{itemize}

  \item
    \setlength{\parskip}{0.6ex}

e.g. pointmap = (FL\_POINT * 4)() ;


  \item 
e.g. pointmap{[}0{]}.x = 23 ; pointmap{[}0{]}.y = 12 ;


  \item 
e.g. pointmap{[}1{]}.x = 56 ; pointmap{[}1{]}.y = 34 ;


  \item 
e.g. pointmap{[}2{]}.x = 102 ; pointmap{[}2{]}.y = 250 ;


  \item 
e.g. fl\_lines(pointmap, 3, xfdata.FL\_DODGERBLUE)


\end{itemize}

\end{quote}

\textbf{Status:} 
Tested + Doc + NoDemo = OK


    \end{boxedminipage}

    \label{xformslib:flxbasic:fl_line}
    \index{xformslib \textit{(package)}!xformslib.flxbasic \textit{(module)}!xformslib.flxbasic.fl\_line \textit{(function)}}

    \vspace{0.5ex}

\hspace{.8\funcindent}\begin{boxedminipage}{\funcwidth}

    \raggedright \textbf{fl\_line}(\textit{xi}, \textit{yi}, \textit{xf}, \textit{yf}, \textit{colr})

    \vspace{-1.5ex}

    \rule{\textwidth}{0.5\fboxrule}
\setlength{\parskip}{2ex}

Connects two points with a straight line.

-{}-
\setlength{\parskip}{1ex}
      \textbf{Parameters}
      \vspace{-1ex}

      \begin{quote}
        \begin{Ventry}{xxxx}

          \item[xi]


initial horizontal position (upper-left corner)
            {\it (type=int)}

          \item[yi]


initial vertical position (upper-left corner)
            {\it (type=int)}

          \item[xf]


final horizontal position (upper-left corner)
            {\it (type=int)}

          \item[yf]


final vertical position (upper-left corner)
            {\it (type=int)}

          \item[colr]


color value
            {\it (type=long\_pos)}

        \end{Ventry}

      \end{quote}

\textbf{Note:} 
e.g. fl\_line(100, 100, 200, 200, xfdata.FL\_ANTIQUEWHITE)


\textbf{Status:} 
Tested + Doc + NoDemo = OK


    \end{boxedminipage}

    \label{xformslib:flxbasic:fl_line}
    \index{xformslib \textit{(package)}!xformslib.flxbasic \textit{(module)}!xformslib.flxbasic.fl\_line \textit{(function)}}

    \vspace{0.5ex}

\hspace{.8\funcindent}\begin{boxedminipage}{\funcwidth}

    \raggedright \textbf{fl\_simple\_line}(\textit{xi}, \textit{yi}, \textit{xf}, \textit{yf}, \textit{colr})

    \vspace{-1.5ex}

    \rule{\textwidth}{0.5\fboxrule}
\setlength{\parskip}{2ex}

Connects two points with a straight line.

-{}-
\setlength{\parskip}{1ex}
      \textbf{Parameters}
      \vspace{-1ex}

      \begin{quote}
        \begin{Ventry}{xxxx}

          \item[xi]


initial horizontal position (upper-left corner)
            {\it (type=int)}

          \item[yi]


initial vertical position (upper-left corner)
            {\it (type=int)}

          \item[xf]


final horizontal position (upper-left corner)
            {\it (type=int)}

          \item[yf]


final vertical position (upper-left corner)
            {\it (type=int)}

          \item[colr]


color value
            {\it (type=long\_pos)}

        \end{Ventry}

      \end{quote}

\textbf{Note:} 
e.g. fl\_line(100, 100, 200, 200, xfdata.FL\_ANTIQUEWHITE)


\textbf{Status:} 
Tested + Doc + NoDemo = OK


    \end{boxedminipage}

    \label{xformslib:flxbasic:fl_point}
    \index{xformslib \textit{(package)}!xformslib.flxbasic \textit{(module)}!xformslib.flxbasic.fl\_point \textit{(function)}}

    \vspace{0.5ex}

\hspace{.8\funcindent}\begin{boxedminipage}{\funcwidth}

    \raggedright \textbf{fl\_point}(\textit{x}, \textit{y}, \textit{colr})

    \vspace{-1.5ex}

    \rule{\textwidth}{0.5\fboxrule}
\setlength{\parskip}{2ex}

Draws one point on the screen.

-{}-
\setlength{\parskip}{1ex}
      \textbf{Parameters}
      \vspace{-1ex}

      \begin{quote}
        \begin{Ventry}{xxxx}

          \item[x]


horizontal position (upper-left corner)
            {\it (type=int)}

          \item[y]


vertical position (upper-left corner)
            {\it (type=int)}

          \item[colr]


color value
            {\it (type=long\_pos)}

        \end{Ventry}

      \end{quote}

\textbf{Note:} 
e.g. fl\_point(75, 452, xfdata.FL\_CHARTREUSE)


\textbf{Status:} 
Tested + Doc + NoDemo = OK


    \end{boxedminipage}

    \label{xformslib:flxbasic:fl_points}
    \index{xformslib \textit{(package)}!xformslib.flxbasic \textit{(module)}!xformslib.flxbasic.fl\_points \textit{(function)}}

    \vspace{0.5ex}

\hspace{.8\funcindent}\begin{boxedminipage}{\funcwidth}

    \raggedright \textbf{fl\_points}(\textit{Point}, \textit{numpt}, \textit{colr})

    \vspace{-1.5ex}

    \rule{\textwidth}{0.5\fboxrule}
\setlength{\parskip}{2ex}

Draws more than one points.

-{}-
\setlength{\parskip}{1ex}
      \textbf{Parameters}
      \vspace{-1ex}

      \begin{quote}
        \begin{Ventry}{xxxxx}

          \item[Point]


an array of point class instances
            {\it (type=array of xfdata.FL\_POINT)}

          \item[numpt]


number of points
            {\it (type=int)}

          \item[colr]


value of color to be set
            {\it (type=long\_pos)}

        \end{Ventry}

      \end{quote}

\textbf{Notes:}
\begin{quote}
  \begin{itemize}

  \item
    \setlength{\parskip}{0.6ex}

e.g. pointmap = (FL\_POINT * 3)() ;


  \item 
e.g. pointmap{[}0{]}.x = 23 ; pointmap{[}0{]}.y = 12 ;


  \item 
e.g. pointmap{[}1{]}.x = 56 ; pointmap{[}1{]}.y = 34 ;


  \item 
e.g. pointmap{[}2{]}.x = 102 ; pointmap{[}2{]}.y = 250 ;


  \item 
e.g. fl\_points(pointmap, 3, xfdata.FL\_AZURE)


\end{itemize}

\end{quote}

\textbf{Status:} 
Tested + Doc + NoDemo = OK


    \end{boxedminipage}

    \label{xformslib:flxbasic:fl_dashedlinestyle}
    \index{xformslib \textit{(package)}!xformslib.flxbasic \textit{(module)}!xformslib.flxbasic.fl\_dashedlinestyle \textit{(function)}}

    \vspace{0.5ex}

\hspace{.8\funcindent}\begin{boxedminipage}{\funcwidth}

    \raggedright \textbf{fl\_dashedlinestyle}(\textit{dash}, \textit{ndash})

    \vspace{-1.5ex}

    \rule{\textwidth}{0.5\fboxrule}
\setlength{\parskip}{2ex}

Changes the dash pattern for xfdata.FL\_USERDASH and
xfdata.FL USERDOUBLEDASH. Each element of the array dash is the length of
a segment of the pattern in pixels. Dashed lines are drawn as alternating
segments, each with the length of an element in dash. Thus the overall
length of the dash pattern, in pixels, is the sum of all elements of dash.
When the pattern is used up but the line to draw is longer it used from the
start again. You have to call this one whenever xfdata.FL\_USERDASH is used
to set the dash pattern, otherwise whatever the last pattern was, it will
be used. After the sequence, the pattern repeats.

-{}-
\setlength{\parskip}{1ex}
      \textbf{Parameters}
      \vspace{-1ex}

      \begin{quote}
        \begin{Ventry}{xxxxx}

          \item[dash]


sequence list of dashes to use. If it's 'None', use default dash
pattern.
            {\it (type=list\_of\_int)}

          \item[ndash]


length of dashes list
            {\it (type=int)}

        \end{Ventry}

      \end{quote}

\textbf{Notes:}
\begin{quote}
  \begin{itemize}

  \item
    \setlength{\parskip}{0.6ex}

e.g. dashlist = {[}9, 3, 2, 3{]}


  \item 
e.g. fl\_dashedlinestyle(dashlist, 4)


\end{itemize}

\end{quote}

\textbf{Status:} 
Tested + Doc + NoDemo = OK


    \end{boxedminipage}

    \label{xformslib:flxbasic:fl_update_display}
    \index{xformslib \textit{(package)}!xformslib.flxbasic \textit{(module)}!xformslib.flxbasic.fl\_update\_display \textit{(function)}}

    \vspace{0.5ex}

\hspace{.8\funcindent}\begin{boxedminipage}{\funcwidth}

    \raggedright \textbf{fl\_update\_display}(\textit{block})

    \vspace{-1.5ex}

    \rule{\textwidth}{0.5\fboxrule}
\setlength{\parskip}{2ex}

Flushes properly the output buffer. It resolves the problem of the
form being only partially redrawn, due to the two way buffering mechanism
of Xlib, if fl\_show\_form() is followed by something that blocks (e.g.,
waiting for a device other than X devices to come online). For typical
programs that use fl\_do\_forms() or fl\_check\_forms() after fl\_show\_form(),
flushing is not necessary as the output buffer is flushed automatically.
Excessive call to fl\_update\_display() degrades performance.

-{}-
\setlength{\parskip}{1ex}
      \textbf{Parameters}
      \vspace{-1ex}

      \begin{quote}
        \begin{Ventry}{xxxxx}

          \item[block]


mode of X buffer flushing. Values 0 (it's flushed so the drawing
requests are on their way to the server) or 1 (it's flushed and
waits until all the events are received and processed by the server)
            {\it (type=int)}

        \end{Ventry}

      \end{quote}

\textbf{Note:} 
e.g. fl\_update\_display()


\textbf{Postcondition:} 
to be used after fl\_show\_form()


\textbf{Status:} 
Tested + Doc + NoDemo = OK


    \end{boxedminipage}

    \label{xformslib:flxbasic:fl_diagline}
    \index{xformslib \textit{(package)}!xformslib.flxbasic \textit{(module)}!xformslib.flxbasic.fl\_diagline \textit{(function)}}

    \vspace{0.5ex}

\hspace{.8\funcindent}\begin{boxedminipage}{\funcwidth}

    \raggedright \textbf{fl\_diagline}(\textit{x}, \textit{y}, \textit{w}, \textit{h}, \textit{colr})

    \vspace{-1.5ex}

    \rule{\textwidth}{0.5\fboxrule}
\setlength{\parskip}{2ex}

Draws a line along the diagonal of a box (to draw a horizontal line
set h to 1, not to 0).

-{}-
\setlength{\parskip}{1ex}
      \textbf{Parameters}
      \vspace{-1ex}

      \begin{quote}
        \begin{Ventry}{xxxx}

          \item[x]


horizontal position (upper-left corner)
            {\it (type=int)}

          \item[y]


vertical position (upper-left corner)
            {\it (type=int)}

          \item[w]


width in coord units
            {\it (type=int)}

          \item[h]


height in coord units
            {\it (type=int)}

          \item[colr]


color value
            {\it (type=long\_pos)}

        \end{Ventry}

      \end{quote}

\textbf{Note:} 
e.g. fl\_diagline(180, 90, 5, 2, xfdata.FL\_BISQUE)


\textbf{Status:} 
Tested + Doc + NoDemo = OK


    \end{boxedminipage}

    \label{xformslib:flxbasic:fl_linewidth}
    \index{xformslib \textit{(package)}!xformslib.flxbasic \textit{(module)}!xformslib.flxbasic.fl\_linewidth \textit{(function)}}

    \vspace{0.5ex}

\hspace{.8\funcindent}\begin{boxedminipage}{\funcwidth}

    \raggedright \textbf{fl\_linewidth}(\textit{lw})

    \vspace{-1.5ex}

    \rule{\textwidth}{0.5\fboxrule}
\setlength{\parskip}{2ex}

Changes the line width.

-{}-
\setlength{\parskip}{1ex}
      \textbf{Parameters}
      \vspace{-1ex}

      \begin{quote}
        \begin{Ventry}{xx}

          \item[lw]


width of line in coord units. If it's 0, reset to the server's
default
            {\it (type=int)}

        \end{Ventry}

      \end{quote}

\textbf{Note:} 
e.g. fl\_linewidth(2)


\textbf{Status:} 
Tested + Doc + NoDemo = OK


    \end{boxedminipage}

    \label{xformslib:flxbasic:fl_linewidth}
    \index{xformslib \textit{(package)}!xformslib.flxbasic \textit{(module)}!xformslib.flxbasic.fl\_linewidth \textit{(function)}}

    \vspace{0.5ex}

\hspace{.8\funcindent}\begin{boxedminipage}{\funcwidth}

    \raggedright \textbf{fl\_set\_linewidth}(\textit{lw})

    \vspace{-1.5ex}

    \rule{\textwidth}{0.5\fboxrule}
\setlength{\parskip}{2ex}

Changes the line width.

-{}-
\setlength{\parskip}{1ex}
      \textbf{Parameters}
      \vspace{-1ex}

      \begin{quote}
        \begin{Ventry}{xx}

          \item[lw]


width of line in coord units. If it's 0, reset to the server's
default
            {\it (type=int)}

        \end{Ventry}

      \end{quote}

\textbf{Note:} 
e.g. fl\_linewidth(2)


\textbf{Status:} 
Tested + Doc + NoDemo = OK


    \end{boxedminipage}

    \label{xformslib:flxbasic:fl_linestyle}
    \index{xformslib \textit{(package)}!xformslib.flxbasic \textit{(module)}!xformslib.flxbasic.fl\_linestyle \textit{(function)}}

    \vspace{0.5ex}

\hspace{.8\funcindent}\begin{boxedminipage}{\funcwidth}

    \raggedright \textbf{fl\_linestyle}(\textit{linestyle})

    \vspace{-1.5ex}

    \rule{\textwidth}{0.5\fboxrule}
\setlength{\parskip}{2ex}

Changes the line style.

-{}-
\setlength{\parskip}{1ex}
      \textbf{Parameters}
      \vspace{-1ex}

      \begin{quote}
        \begin{Ventry}{xxxxxxxxx}

          \item[linestyle]


style of the line to draw. Values (from xfdata.py) FL\_SOLID,
FL\_USERDASH, FL\_USERDOUBLEDASH, FL\_DOT, FL\_DOTDASH, FL\_DASH,
FL\_LONGDASH
            {\it (type=int)}

        \end{Ventry}

      \end{quote}

\textbf{Note:} 
e.g. fl\_linestyle(xfdata.FL\_DOT)


\textbf{Status:} 
Tested + Doc + NoDemo = OK


    \end{boxedminipage}

    \label{xformslib:flxbasic:fl_linestyle}
    \index{xformslib \textit{(package)}!xformslib.flxbasic \textit{(module)}!xformslib.flxbasic.fl\_linestyle \textit{(function)}}

    \vspace{0.5ex}

\hspace{.8\funcindent}\begin{boxedminipage}{\funcwidth}

    \raggedright \textbf{fl\_set\_linestyle}(\textit{linestyle})

    \vspace{-1.5ex}

    \rule{\textwidth}{0.5\fboxrule}
\setlength{\parskip}{2ex}

Changes the line style.

-{}-
\setlength{\parskip}{1ex}
      \textbf{Parameters}
      \vspace{-1ex}

      \begin{quote}
        \begin{Ventry}{xxxxxxxxx}

          \item[linestyle]


style of the line to draw. Values (from xfdata.py) FL\_SOLID,
FL\_USERDASH, FL\_USERDOUBLEDASH, FL\_DOT, FL\_DOTDASH, FL\_DASH,
FL\_LONGDASH
            {\it (type=int)}

        \end{Ventry}

      \end{quote}

\textbf{Note:} 
e.g. fl\_linestyle(xfdata.FL\_DOT)


\textbf{Status:} 
Tested + Doc + NoDemo = OK


    \end{boxedminipage}

    \label{xformslib:flxbasic:fl_drawmode}
    \index{xformslib \textit{(package)}!xformslib.flxbasic \textit{(module)}!xformslib.flxbasic.fl\_drawmode \textit{(function)}}

    \vspace{0.5ex}

\hspace{.8\funcindent}\begin{boxedminipage}{\funcwidth}

    \raggedright \textbf{fl\_drawmode}(\textit{mode})

    \vspace{-1.5ex}

    \rule{\textwidth}{0.5\fboxrule}
\setlength{\parskip}{2ex}

Changes the drawing mode so the destination pixel values play a role
in the final pixel value. By default, all lines are drawn so they
overwrite the destination pixel values.

-{}-
\setlength{\parskip}{1ex}
      \textbf{Parameters}
      \vspace{-1ex}

      \begin{quote}
        \begin{Ventry}{xxxx}

          \item[mode]


requested mode setting. Values (from xfdata module) FL\_XOR, FL\_COPY,
FL\_AND
            {\it (type=int)}

        \end{Ventry}

      \end{quote}

\textbf{Note:} 
e.g. fl\_drawmode(xfdata.FL\_AND)


\textbf{Status:} 
Tested + Doc + NoDemo = OK


    \end{boxedminipage}

    \label{xformslib:flxbasic:fl_get_linewidth}
    \index{xformslib \textit{(package)}!xformslib.flxbasic \textit{(module)}!xformslib.flxbasic.fl\_get\_linewidth \textit{(function)}}

    \vspace{0.5ex}

\hspace{.8\funcindent}\begin{boxedminipage}{\funcwidth}

    \raggedright \textbf{fl\_get\_linewidth}()

    \vspace{-1.5ex}

    \rule{\textwidth}{0.5\fboxrule}
\setlength{\parskip}{2ex}

Returns the width of line.

-{}-
\setlength{\parskip}{1ex}
      \textbf{Return Value}
    \vspace{-1ex}

      \begin{quote}

line width (lw)
      {\it (type=int)}

      \end{quote}

\textbf{Note:} 
e.g. wid = fl\_get\_linewidth()


\textbf{Status:} 
Tested + Doc + NoDemo = OK


    \end{boxedminipage}

    \label{xformslib:flxbasic:fl_get_linestyle}
    \index{xformslib \textit{(package)}!xformslib.flxbasic \textit{(module)}!xformslib.flxbasic.fl\_get\_linestyle \textit{(function)}}

    \vspace{0.5ex}

\hspace{.8\funcindent}\begin{boxedminipage}{\funcwidth}

    \raggedright \textbf{fl\_get\_linestyle}()

    \vspace{-1.5ex}

    \rule{\textwidth}{0.5\fboxrule}
\setlength{\parskip}{2ex}

Returns the style of line (from xfdata.py, e.g. FL\_SOLID, FL\_DOT,
etc..).

-{}-
\setlength{\parskip}{1ex}
      \textbf{Return Value}
    \vspace{-1ex}

      \begin{quote}

line style
      {\it (type=int)}

      \end{quote}

\textbf{Note:} 
e.g. currstl = fl\_get\_linestyle()


\textbf{Status:} 
Tested + Doc + NoDemo = OK


    \end{boxedminipage}

    \label{xformslib:flxbasic:fl_get_drawmode}
    \index{xformslib \textit{(package)}!xformslib.flxbasic \textit{(module)}!xformslib.flxbasic.fl\_get\_drawmode \textit{(function)}}

    \vspace{0.5ex}

\hspace{.8\funcindent}\begin{boxedminipage}{\funcwidth}

    \raggedright \textbf{fl\_get\_drawmode}()

    \vspace{-1.5ex}

    \rule{\textwidth}{0.5\fboxrule}
\setlength{\parskip}{2ex}

Obtains the drawing mode of lines (from xfdata.py, e.g. FL\_AND, FL\_XOR
etc..).

-{}-
\setlength{\parskip}{1ex}
      \textbf{Return Value}
    \vspace{-1ex}

      \begin{quote}

drawing mode
      {\it (type=int)}

      \end{quote}

\textbf{Note:} 
e.g. currdrw = fl\_get\_draw\_mode()


\textbf{Status:} 
Tested + Doc + NoDemo = OK


    \end{boxedminipage}

    \label{xformslib:flxbasic:fl_drawmode}
    \index{xformslib \textit{(package)}!xformslib.flxbasic \textit{(module)}!xformslib.flxbasic.fl\_drawmode \textit{(function)}}

    \vspace{0.5ex}

\hspace{.8\funcindent}\begin{boxedminipage}{\funcwidth}

    \raggedright \textbf{fl\_set\_drawmode}(\textit{mode})

    \vspace{-1.5ex}

    \rule{\textwidth}{0.5\fboxrule}
\setlength{\parskip}{2ex}

Changes the drawing mode so the destination pixel values play a role
in the final pixel value. By default, all lines are drawn so they
overwrite the destination pixel values.

-{}-
\setlength{\parskip}{1ex}
      \textbf{Parameters}
      \vspace{-1ex}

      \begin{quote}
        \begin{Ventry}{xxxx}

          \item[mode]


requested mode setting. Values (from xfdata module) FL\_XOR, FL\_COPY,
FL\_AND
            {\it (type=int)}

        \end{Ventry}

      \end{quote}

\textbf{Note:} 
e.g. fl\_drawmode(xfdata.FL\_AND)


\textbf{Status:} 
Tested + Doc + NoDemo = OK


    \end{boxedminipage}

    \label{xformslib:flxbasic:fl_oval}
    \index{xformslib \textit{(package)}!xformslib.flxbasic \textit{(module)}!xformslib.flxbasic.fl\_oval \textit{(function)}}

    \vspace{0.5ex}

\hspace{.8\funcindent}\begin{boxedminipage}{\funcwidth}

    \raggedright \textbf{fl\_oval}(\textit{fill}, \textit{x}, \textit{y}, \textit{w}, \textit{h}, \textit{colr})

    \vspace{-1.5ex}

    \rule{\textwidth}{0.5\fboxrule}
\setlength{\parskip}{2ex}

Draws an ellipse, either filled or open. Use w equal to h to get a
circle.

-{}-
\setlength{\parskip}{1ex}
      \textbf{Parameters}
      \vspace{-1ex}

      \begin{quote}
        \begin{Ventry}{xxxx}

          \item[fill]


flag if filled or open ellipse. Values 1 (if filled ellipse) or 0
(if open)
            {\it (type=int)}

          \item[x]


horizontal position (upper-left corner)
            {\it (type=int)}

          \item[y]


vertical position (upper-left corner)
            {\it (type=int)}

          \item[w]


width in coord units
            {\it (type=int)}

          \item[h]


height in coord units
            {\it (type=int)}

          \item[colr]


color value
            {\it (type=long\_pos)}

        \end{Ventry}

      \end{quote}

\textbf{Note:} 
e.g. fl\_oval(1, 125, 256, 145, 320, xfdata.FL\_BURLYWOOD)


\textbf{Status:} 
Tested + Doc + NoDemo = OK


    \end{boxedminipage}

    \label{xformslib:flxbasic:fl_ovalbound}
    \index{xformslib \textit{(package)}!xformslib.flxbasic \textit{(module)}!xformslib.flxbasic.fl\_ovalbound \textit{(function)}}

    \vspace{0.5ex}

\hspace{.8\funcindent}\begin{boxedminipage}{\funcwidth}

    \raggedright \textbf{fl\_ovalbound}(\textit{x}, \textit{y}, \textit{w}, \textit{h}, \textit{colr})

    \vspace{-1.5ex}

    \rule{\textwidth}{0.5\fboxrule}
\setlength{\parskip}{2ex}

Draws a filled ellipse with a black outline. Use w equal to h to get a
circle.

-{}-
\setlength{\parskip}{1ex}
      \textbf{Parameters}
      \vspace{-1ex}

      \begin{quote}
        \begin{Ventry}{xxxx}

          \item[x]


horizontal position (upper-left corner)
            {\it (type=int)}

          \item[y]


vertical position (upper-left corner)
            {\it (type=int)}

          \item[w]


width in coord units
            {\it (type=int)}

          \item[h]


height in coord units
            {\it (type=int)}

          \item[colr]


color value
            {\it (type=long\_pos)}

        \end{Ventry}

      \end{quote}

\textbf{Note:} 
e.g. fl\_ovalbound(1, 125, 256, 145, 320, xfdata.FL\_BLANCHEDALMOND)


\textbf{Status:} 
Tested + Doc + NoDemo = OK


    \end{boxedminipage}

    \label{xformslib:flxbasic:fl_ovalarc}
    \index{xformslib \textit{(package)}!xformslib.flxbasic \textit{(module)}!xformslib.flxbasic.fl\_ovalarc \textit{(function)}}

    \vspace{0.5ex}

\hspace{.8\funcindent}\begin{boxedminipage}{\funcwidth}

    \raggedright \textbf{fl\_ovalarc}(\textit{fill}, \textit{x}, \textit{y}, \textit{w}, \textit{h}, \textit{stheta}, \textit{dtheta}, \textit{colr})

    \vspace{-1.5ex}

    \rule{\textwidth}{0.5\fboxrule}
\setlength{\parskip}{2ex}

Draws an elliptical arc, either filled or open.

-{}-
\setlength{\parskip}{1ex}
      \textbf{Parameters}
      \vspace{-1ex}

      \begin{quote}
        \begin{Ventry}{xxxxxx}

          \item[fill]


flag if filled or open. Values 1 (if filled) or 0 (if open)
            {\it (type=int)}

          \item[x]


horizontal position (upper-left corner)
            {\it (type=int)}

          \item[y]


vertical position (upper-left corner)
            {\it (type=int)}

          \item[w]


width in coord units
            {\it (type=int)}

          \item[h]


height in coord units
            {\it (type=int)}

          \item[stheta]


starting angle, measured in tenth of a degree and with 0 at 3
o'clock position
            {\it (type=int)}

          \item[dtheta]


the directione and the extent of the arc. If positive the arc is
drawn in counter-clockwise direction from the starting point,
otherwise in clockwise direction. If it is larger than 3600 it is
truncated to 3600.
            {\it (type=int)}

          \item[colr]


color value
            {\it (type=long\_pos)}

        \end{Ventry}

      \end{quote}

\textbf{Note:} 
e.g. fl\_ovalarc(1, 275, 256, 145, 320, 200, 900,
xfdata.FL\_DARKSALMON)


\textbf{Status:} 
Tested + Doc + NoDemo = OK


    \end{boxedminipage}

    \label{xformslib:flxbasic:fl_ovalf}
    \index{xformslib \textit{(package)}!xformslib.flxbasic \textit{(module)}!xformslib.flxbasic.fl\_ovalf \textit{(function)}}

    \vspace{0.5ex}

\hspace{.8\funcindent}\begin{boxedminipage}{\funcwidth}

    \raggedright \textbf{fl\_ovalf}(\textit{x}, \textit{y}, \textit{w}, \textit{h}, \textit{colr})

    \vspace{-1.5ex}

    \rule{\textwidth}{0.5\fboxrule}
\setlength{\parskip}{2ex}

Draws a filled ellipse. Use w equal to h to get a circle.

-{}-
\setlength{\parskip}{1ex}
      \textbf{Parameters}
      \vspace{-1ex}

      \begin{quote}
        \begin{Ventry}{xxxx}

          \item[x]


horizontal position (upper-left corner)
            {\it (type=int)}

          \item[y]


vertical position (upper-left corner)
            {\it (type=int)}

          \item[w]


width in coord units
            {\it (type=int)}

          \item[h]


height in coord units
            {\it (type=int)}

          \item[colr]


color value
            {\it (type=long\_pos)}

        \end{Ventry}

      \end{quote}

\textbf{Note:} 
e.g. fl\_ovalf(125, 256, 145, 320, xfdata.FL\_CORNFLOWERBLUE)


\textbf{Status:} 
Tested + Doc + NoDemo = OK


    \end{boxedminipage}

    \label{xformslib:flxbasic:fl_ovall}
    \index{xformslib \textit{(package)}!xformslib.flxbasic \textit{(module)}!xformslib.flxbasic.fl\_ovall \textit{(function)}}

    \vspace{0.5ex}

\hspace{.8\funcindent}\begin{boxedminipage}{\funcwidth}

    \raggedright \textbf{fl\_ovall}(\textit{x}, \textit{y}, \textit{w}, \textit{h}, \textit{colr})

    \vspace{-1.5ex}

    \rule{\textwidth}{0.5\fboxrule}
\setlength{\parskip}{2ex}

Draws an open ellipse. Use w equal to h to get a circle.

-{}-
\setlength{\parskip}{1ex}
      \textbf{Parameters}
      \vspace{-1ex}

      \begin{quote}
        \begin{Ventry}{xxxx}

          \item[x]


horizontal position (upper-left corner)
            {\it (type=int)}

          \item[y]


vertical position (upper-left corner)
            {\it (type=int)}

          \item[w]


width in coord units
            {\it (type=int)}

          \item[h]


height in coord units
            {\it (type=int)}

          \item[colr]


color value
            {\it (type=long\_pos)}

        \end{Ventry}

      \end{quote}

\textbf{Note:} 
e.g. fl\_ovall(125, 256, 145, 320, xfdata.FL\_DARKERED)


\textbf{Status:} 
Tested + Doc + NoDemo = OK


    \end{boxedminipage}

    \label{xformslib:flxbasic:fl_ovalbound}
    \index{xformslib \textit{(package)}!xformslib.flxbasic \textit{(module)}!xformslib.flxbasic.fl\_ovalbound \textit{(function)}}

    \vspace{0.5ex}

\hspace{.8\funcindent}\begin{boxedminipage}{\funcwidth}

    \raggedright \textbf{fl\_oval\_bound}(\textit{x}, \textit{y}, \textit{w}, \textit{h}, \textit{colr})

    \vspace{-1.5ex}

    \rule{\textwidth}{0.5\fboxrule}
\setlength{\parskip}{2ex}

Draws a filled ellipse with a black outline. Use w equal to h to get a
circle.

-{}-
\setlength{\parskip}{1ex}
      \textbf{Parameters}
      \vspace{-1ex}

      \begin{quote}
        \begin{Ventry}{xxxx}

          \item[x]


horizontal position (upper-left corner)
            {\it (type=int)}

          \item[y]


vertical position (upper-left corner)
            {\it (type=int)}

          \item[w]


width in coord units
            {\it (type=int)}

          \item[h]


height in coord units
            {\it (type=int)}

          \item[colr]


color value
            {\it (type=long\_pos)}

        \end{Ventry}

      \end{quote}

\textbf{Note:} 
e.g. fl\_ovalbound(1, 125, 256, 145, 320, xfdata.FL\_BLANCHEDALMOND)


\textbf{Status:} 
Tested + Doc + NoDemo = OK


    \end{boxedminipage}

    \label{xformslib:flxbasic:fl_circf}
    \index{xformslib \textit{(package)}!xformslib.flxbasic \textit{(module)}!xformslib.flxbasic.fl\_circf \textit{(function)}}

    \vspace{0.5ex}

\hspace{.8\funcindent}\begin{boxedminipage}{\funcwidth}

    \raggedright \textbf{fl\_circf}(\textit{x}, \textit{y}, \textit{r}, \textit{colr})

    \vspace{-1.5ex}

    \rule{\textwidth}{0.5\fboxrule}
\setlength{\parskip}{2ex}

Draws a filled circle.

-{}-
\setlength{\parskip}{1ex}
      \textbf{Parameters}
      \vspace{-1ex}

      \begin{quote}
        \begin{Ventry}{xxxx}

          \item[x]


horizontal position of the center of the arc
            {\it (type=int)}

          \item[y]


vertical position of the center of the arc
            {\it (type=int)}

          \item[r]


radius of the arc
            {\it (type=int)}

          \item[colr]


color value
            {\it (type=long\_pos)}

        \end{Ventry}

      \end{quote}

\textbf{Note:} 
e.g. fl\_circf(200, 250, 69, xfdata.FL\_FUCHSIA)


\textbf{Status:} 
Tested + NoDoc + Demo = OK


    \end{boxedminipage}

    \label{xformslib:flxbasic:fl_circ}
    \index{xformslib \textit{(package)}!xformslib.flxbasic \textit{(module)}!xformslib.flxbasic.fl\_circ \textit{(function)}}

    \vspace{0.5ex}

\hspace{.8\funcindent}\begin{boxedminipage}{\funcwidth}

    \raggedright \textbf{fl\_circ}(\textit{x}, \textit{y}, \textit{r}, \textit{colr})

    \vspace{-1.5ex}

    \rule{\textwidth}{0.5\fboxrule}
\setlength{\parskip}{2ex}

Draws an open circle.

-{}-
\setlength{\parskip}{1ex}
      \textbf{Parameters}
      \vspace{-1ex}

      \begin{quote}
        \begin{Ventry}{x}

          \item[x]


horizontal position of the center of the arc
            {\it (type=int)}

          \item[y]


vertical position of the center of the arc
            {\it (type=int)}

          \item[r]


radius of the arc
            {\it (type=int)}

        \end{Ventry}

      \end{quote}

\textbf{Note:} 
e.g. fl\_circ(200, 250, 69, xfdata.FL\_GAINSBORO)


\textbf{Status:} 
Tested + Doc + NoDemo = OK


    \end{boxedminipage}

    \label{xformslib:flxbasic:fl_pieslice}
    \index{xformslib \textit{(package)}!xformslib.flxbasic \textit{(module)}!xformslib.flxbasic.fl\_pieslice \textit{(function)}}

    \vspace{0.5ex}

\hspace{.8\funcindent}\begin{boxedminipage}{\funcwidth}

    \raggedright \textbf{fl\_pieslice}(\textit{fill}, \textit{x}, \textit{y}, \textit{w}, \textit{h}, \textit{stheta}, \textit{etheta}, \textit{colr})

    \vspace{-1.5ex}

    \rule{\textwidth}{0.5\fboxrule}
\setlength{\parskip}{2ex}

Draws an elliptical arc, either filled or open.

-{}-
\setlength{\parskip}{1ex}
      \textbf{Parameters}
      \vspace{-1ex}

      \begin{quote}
        \begin{Ventry}{xxxxxx}

          \item[fill]


if the arc is filled or open. Values 1 (if filled) or 0 (if open)
            {\it (type=int)}

          \item[x]


horizontal position of the bounding box
            {\it (type=int)}

          \item[y]


vertical position of the bounding box
            {\it (type=int)}

          \item[h]


horizontal axe of the ellipse
            {\it (type=int)}

          \item[w]


vertical axe of the ellipse
            {\it (type=int)}

          \item[stheta]


starting angle of the arc in units of tenths of a degree (where 0
stands for a direction of 3 o'clock, i.e. the right-most point of a
circle)
            {\it (type=int)}

          \item[etheta]


ending angle of the arc in units of tenths of a degree (where 0
stands for a direction of 3 o'clock, i.e. the right-most point of a
circle)
            {\it (type=int)}

          \item[colr]


color value
            {\it (type=long\_pos)}

        \end{Ventry}

      \end{quote}

\textbf{Note:} 
e.g. fl\_pieslice(1, 120, 253, 400, 100, 60, 70, xfdata.FL\_GOLD)


\textbf{Status:} 
Tested + Doc + NoDemo = OK


    \end{boxedminipage}

    \label{xformslib:flxbasic:fl_arcf}
    \index{xformslib \textit{(package)}!xformslib.flxbasic \textit{(module)}!xformslib.flxbasic.fl\_arcf \textit{(function)}}

    \vspace{0.5ex}

\hspace{.8\funcindent}\begin{boxedminipage}{\funcwidth}

    \raggedright \textbf{fl\_arcf}(\textit{x}, \textit{y}, \textit{r}, \textit{stheta}, \textit{etheta}, \textit{colr})

    \vspace{-1.5ex}

    \rule{\textwidth}{0.5\fboxrule}
\setlength{\parskip}{2ex}

Draws a filled circular arc. If the difference between theta end and
theta start is larger than 3600 (360 degrees), drawing is truncated to
360 degrees.

-{}-
\setlength{\parskip}{1ex}
      \textbf{Parameters}
      \vspace{-1ex}

      \begin{quote}
        \begin{Ventry}{xxxxxx}

          \item[x]


horizontal position of the center of the arc
            {\it (type=int)}

          \item[y]


vertical position of the center of the arc
            {\it (type=int)}

          \item[r]


radius of the arc
            {\it (type=int)}

          \item[stheta]


starting angle of the arc in units of tenths of a degree (where 0
stands for a direction of 3 o'clock, i.e. the right-most point of a
circle)
            {\it (type=int)}

          \item[etheta]


ending angle of the arc in units of tenths of a degree (where 0
stands for a direction of 3 o'clock, i.e. the right-most point of a
circle)
            {\it (type=int)}

          \item[colr]


color value
            {\it (type=long\_pos)}

        \end{Ventry}

      \end{quote}

\textbf{Note:} 
e.g. fl\_arcf(120, 253, 40, 10, 60, xfdata.FL\_FIREBRICK)


\textbf{Status:} 
Tested + Doc + NoDemo = OK


    \end{boxedminipage}

    \label{xformslib:flxbasic:fl_arc}
    \index{xformslib \textit{(package)}!xformslib.flxbasic \textit{(module)}!xformslib.flxbasic.fl\_arc \textit{(function)}}

    \vspace{0.5ex}

\hspace{.8\funcindent}\begin{boxedminipage}{\funcwidth}

    \raggedright \textbf{fl\_arc}(\textit{x}, \textit{y}, \textit{r}, \textit{stheta}, \textit{etheta}, \textit{colr})

    \vspace{-1.5ex}

    \rule{\textwidth}{0.5\fboxrule}
\setlength{\parskip}{2ex}

Draws an open circular arc. If the difference between theta end and
theta start is larger than 3600 (360 degrees), drawing is truncated to
360 degrees.

-{}-
\setlength{\parskip}{1ex}
      \textbf{Parameters}
      \vspace{-1ex}

      \begin{quote}
        \begin{Ventry}{xxxxxx}

          \item[x]


horizontal position of the center of the arc
            {\it (type=int)}

          \item[y]


vertical position of the center of the arc
            {\it (type=int)}

          \item[r]


radius of the arc
            {\it (type=int)}

          \item[stheta]


(where 0 stands for a direction of 3 o'clock, i.e. the right-most
point of a circle)
            {\it (type=starting angle of the arc in units of tenths of a degree)}

          \item[etheta]


(where 0 stands for a direction of 3 o'clock, i.e. the right-most
point of a circle)
            {\it (type=ending angle of the arc in units of tenths of a degree)}

          \item[colr]


color value
            {\it (type=long\_pos)}

        \end{Ventry}

      \end{quote}

\textbf{Note:} 
e.g. fl\_arc(120, 253, 40, 10, 60, xfdata.FL\_FORESTGREEN)


\textbf{Status:} 
Tested + Doc + NoDemo = OK


    \end{boxedminipage}

    \label{xformslib:flxbasic:fl_drw_frame}
    \index{xformslib \textit{(package)}!xformslib.flxbasic \textit{(module)}!xformslib.flxbasic.fl\_drw\_frame \textit{(function)}}

    \vspace{0.5ex}

\hspace{.8\funcindent}\begin{boxedminipage}{\funcwidth}

    \raggedright \textbf{fl\_drw\_frame}(\textit{boxtype}, \textit{x}, \textit{y}, \textit{w}, \textit{h}, \textit{colr}, \textit{bw})

    \vspace{-1.5ex}

    \rule{\textwidth}{0.5\fboxrule}
\setlength{\parskip}{2ex}

Draws a frame outside of the bounding box specified.

-{}-
\setlength{\parskip}{1ex}
      \textbf{Parameters}
      \vspace{-1ex}

      \begin{quote}
        \begin{Ventry}{xxxxxxx}

          \item[boxtype]


type of frame box. Values (from xfdata.py) FL\_NO\_BOX, FL\_UP\_BOX,
FL\_DOWN\_BOX, FL\_BORDER\_BOX, FL\_SHADOW\_BOX, FL\_FRAME\_BOX,
FL\_ROUNDED\_BOX, FL\_EMBOSSED\_BOX, FL\_FLAT\_BOX, FL\_RFLAT\_BOX,
FL\_RSHADOW\_BOX, FL\_OVAL\_BOX, FL\_ROUNDED3D\_UPBOX, FL\_ROUNDED3D\_DOWNBOX,
FL\_OVAL3D\_UPBOX, FL\_OVAL3D\_DOWNBOX, FL\_OVAL3D\_FRAMEBOX,
FL\_OVAL3D\_EMBOSSEDBOX
            {\it (type=int)}

          \item[x]


horizontal position (upper-left corner)
            {\it (type=int)}

          \item[y]


vertical position (upper-left corner)
            {\it (type=int)}

          \item[w]


width in coord units
            {\it (type=int)}

          \item[h]


height in coord units
            {\it (type=int)}

          \item[colr]


color value
            {\it (type=long\_pos)}

          \item[bw]


width of boundary
            {\it (type=int)}

        \end{Ventry}

      \end{quote}

\textbf{Note:} 
e.g. fl\_drw\_frame(xfdata.FL\_UP\_BOX, 470, 560, 170, 280,
xfdata.FL\_DIMGRAY, 2)


\textbf{Status:} 
Tested + Doc + NoDemo = OK


    \end{boxedminipage}

    \label{xformslib:flxbasic:fl_drw_checkbox}
    \index{xformslib \textit{(package)}!xformslib.flxbasic \textit{(module)}!xformslib.flxbasic.fl\_drw\_checkbox \textit{(function)}}

    \vspace{0.5ex}

\hspace{.8\funcindent}\begin{boxedminipage}{\funcwidth}

    \raggedright \textbf{fl\_drw\_checkbox}(\textit{boxtype}, \textit{x}, \textit{y}, \textit{w}, \textit{h}, \textit{colr}, \textit{bw})

    \vspace{-1.5ex}

    \rule{\textwidth}{0.5\fboxrule}
\setlength{\parskip}{2ex}

Draws a box rotated 45 degrees.

-{}-
\setlength{\parskip}{1ex}
      \textbf{Parameters}
      \vspace{-1ex}

      \begin{quote}
        \begin{Ventry}{xxxxxxx}

          \item[boxtype]


type of checkbox to draw. Values (from xfdata.py) FL\_NO\_BOX,
FL\_UP\_BOX, FL\_DOWN\_BOX, FL\_BORDER\_BOX, FL\_SHADOW\_BOX, FL\_FRAME\_BOX,
FL\_ROUNDED\_BOX, FL\_EMBOSSED\_BOX, FL\_FLAT\_BOX, FL\_RFLAT\_BOX,
FL\_RSHADOW\_BOX, FL\_OVAL\_BOX, FL\_ROUNDED3D\_UPBOX,
FL\_ROUNDED3D\_DOWNBOX, FL\_OVAL3D\_UPBOX, FL\_OVAL3D\_DOWNBOX,
FL\_OVAL3D\_FRAMEBOX, FL\_OVAL3D\_EMBOSSEDBOX
            {\it (type=int)}

          \item[x]


horizontal position (upper-left corner)
            {\it (type=int)}

          \item[y]


vertical position (upper-left corner)
            {\it (type=int)}

          \item[w]


width in coord units
            {\it (type=int)}

          \item[h]


height in coord units
            {\it (type=int)}

          \item[colr]


color value
            {\it (type=long\_pos)}

          \item[bw]


width of boundary
            {\it (type=int)}

        \end{Ventry}

      \end{quote}

\textbf{Note:} 
e.g. fl\_drw\_checkbox(xfdata.FL\_ROUNDED3D\_UPBOX, 470, 560, 170,
280, xfdata.FL\_LEMONCHIFFON, -2)


\textbf{Status:} 
Tested + Doc + NoDemo = OK


    \end{boxedminipage}

    \label{xformslib:flxbasic:fl_get_fontstruct}
    \index{xformslib \textit{(package)}!xformslib.flxbasic \textit{(module)}!xformslib.flxbasic.fl\_get\_fontstruct \textit{(function)}}

    \vspace{0.5ex}

\hspace{.8\funcindent}\begin{boxedminipage}{\funcwidth}

    \raggedright \textbf{fl\_get\_fontstruct}(\textit{style}, \textit{size})

    \vspace{-1.5ex}

    \rule{\textwidth}{0.5\fboxrule}
\setlength{\parskip}{2ex}

Returns the X font structure for a particular size and style as used
in XForms library.

-{}-
\setlength{\parskip}{1ex}
      \textbf{Parameters}
      \vspace{-1ex}

      \begin{quote}
        \begin{Ventry}{xxxxx}

          \item[style]


font style. Values (from xfdata.py) FL\_NORMAL\_STYLE, FL\_BOLD\_STYLE,
FL\_ITALIC\_STYLE, FL\_BOLDITALIC\_STYLE, FL\_FIXED\_STYLE,
FL\_FIXEDBOLD\_STYLE, FL\_FIXEDITALIC\_STYLE, FL\_FIXEDBOLDITALIC\_STYLE,
FL\_TIMES\_STYLE, FL\_TIMESBOLD\_STYLE, FL\_TIMESITALIC\_STYLE,
FL\_TIMESBOLDITALIC\_STYLE, FL\_MISC\_STYLE, FL\_MISCBOLD\_STYLE,
FL\_MISCITALIC\_STYLE, FL\_SYMBOL\_STYLE, FL\_SHADOW\_STYLE,
FL\_ENGRAVED\_STYLE, FL\_EMBOSSED\_STYLE
            {\it (type=int)}

          \item[size]


font size. Values (from xfdata.py) FL\_TINY\_SIZE, FL\_SMALL\_SIZE,
FL\_NORMAL\_SIZE, FL\_MEDIUM\_SIZE, FL\_LARGE\_SIZE, FL\_HUGE\_SIZE,
FL\_DEFAULT\_SIZE
            {\it (type=int)}

        \end{Ventry}

      \end{quote}

      \textbf{Return Value}
    \vspace{-1ex}

      \begin{quote}

XFontStruct class instance
      {\it (type=pointer to xfdata.XFontStruct)}

      \end{quote}

\textbf{Note:} 
e.g. pfstruc = fl\_get\_fontstruct(FL\_ITALIC\_STYLE, FL\_NORMAL\_STYLE)


\textbf{Status:} 
Tested + Doc + NoDemo = OK


    \end{boxedminipage}

    \label{xformslib:flxbasic:fl_get_fontstruct}
    \index{xformslib \textit{(package)}!xformslib.flxbasic \textit{(module)}!xformslib.flxbasic.fl\_get\_fontstruct \textit{(function)}}

    \vspace{0.5ex}

\hspace{.8\funcindent}\begin{boxedminipage}{\funcwidth}

    \raggedright \textbf{fl\_get\_font\_struct}(\textit{style}, \textit{size})

    \vspace{-1.5ex}

    \rule{\textwidth}{0.5\fboxrule}
\setlength{\parskip}{2ex}

Returns the X font structure for a particular size and style as used
in XForms library.

-{}-
\setlength{\parskip}{1ex}
      \textbf{Parameters}
      \vspace{-1ex}

      \begin{quote}
        \begin{Ventry}{xxxxx}

          \item[style]


font style. Values (from xfdata.py) FL\_NORMAL\_STYLE, FL\_BOLD\_STYLE,
FL\_ITALIC\_STYLE, FL\_BOLDITALIC\_STYLE, FL\_FIXED\_STYLE,
FL\_FIXEDBOLD\_STYLE, FL\_FIXEDITALIC\_STYLE, FL\_FIXEDBOLDITALIC\_STYLE,
FL\_TIMES\_STYLE, FL\_TIMESBOLD\_STYLE, FL\_TIMESITALIC\_STYLE,
FL\_TIMESBOLDITALIC\_STYLE, FL\_MISC\_STYLE, FL\_MISCBOLD\_STYLE,
FL\_MISCITALIC\_STYLE, FL\_SYMBOL\_STYLE, FL\_SHADOW\_STYLE,
FL\_ENGRAVED\_STYLE, FL\_EMBOSSED\_STYLE
            {\it (type=int)}

          \item[size]


font size. Values (from xfdata.py) FL\_TINY\_SIZE, FL\_SMALL\_SIZE,
FL\_NORMAL\_SIZE, FL\_MEDIUM\_SIZE, FL\_LARGE\_SIZE, FL\_HUGE\_SIZE,
FL\_DEFAULT\_SIZE
            {\it (type=int)}

        \end{Ventry}

      \end{quote}

      \textbf{Return Value}
    \vspace{-1ex}

      \begin{quote}

XFontStruct class instance
      {\it (type=pointer to xfdata.XFontStruct)}

      \end{quote}

\textbf{Note:} 
e.g. pfstruc = fl\_get\_fontstruct(FL\_ITALIC\_STYLE, FL\_NORMAL\_STYLE)


\textbf{Status:} 
Tested + Doc + NoDemo = OK


    \end{boxedminipage}

    \label{xformslib:flxbasic:fl_get_fontstruct}
    \index{xformslib \textit{(package)}!xformslib.flxbasic \textit{(module)}!xformslib.flxbasic.fl\_get\_fontstruct \textit{(function)}}

    \vspace{0.5ex}

\hspace{.8\funcindent}\begin{boxedminipage}{\funcwidth}

    \raggedright \textbf{fl\_get\_fntstruct}(\textit{style}, \textit{size})

    \vspace{-1.5ex}

    \rule{\textwidth}{0.5\fboxrule}
\setlength{\parskip}{2ex}

Returns the X font structure for a particular size and style as used
in XForms library.

-{}-
\setlength{\parskip}{1ex}
      \textbf{Parameters}
      \vspace{-1ex}

      \begin{quote}
        \begin{Ventry}{xxxxx}

          \item[style]


font style. Values (from xfdata.py) FL\_NORMAL\_STYLE, FL\_BOLD\_STYLE,
FL\_ITALIC\_STYLE, FL\_BOLDITALIC\_STYLE, FL\_FIXED\_STYLE,
FL\_FIXEDBOLD\_STYLE, FL\_FIXEDITALIC\_STYLE, FL\_FIXEDBOLDITALIC\_STYLE,
FL\_TIMES\_STYLE, FL\_TIMESBOLD\_STYLE, FL\_TIMESITALIC\_STYLE,
FL\_TIMESBOLDITALIC\_STYLE, FL\_MISC\_STYLE, FL\_MISCBOLD\_STYLE,
FL\_MISCITALIC\_STYLE, FL\_SYMBOL\_STYLE, FL\_SHADOW\_STYLE,
FL\_ENGRAVED\_STYLE, FL\_EMBOSSED\_STYLE
            {\it (type=int)}

          \item[size]


font size. Values (from xfdata.py) FL\_TINY\_SIZE, FL\_SMALL\_SIZE,
FL\_NORMAL\_SIZE, FL\_MEDIUM\_SIZE, FL\_LARGE\_SIZE, FL\_HUGE\_SIZE,
FL\_DEFAULT\_SIZE
            {\it (type=int)}

        \end{Ventry}

      \end{quote}

      \textbf{Return Value}
    \vspace{-1ex}

      \begin{quote}

XFontStruct class instance
      {\it (type=pointer to xfdata.XFontStruct)}

      \end{quote}

\textbf{Note:} 
e.g. pfstruc = fl\_get\_fontstruct(FL\_ITALIC\_STYLE, FL\_NORMAL\_STYLE)


\textbf{Status:} 
Tested + Doc + NoDemo = OK


    \end{boxedminipage}

    \label{xformslib:flxbasic:fl_get_mouse}
    \index{xformslib \textit{(package)}!xformslib.flxbasic \textit{(module)}!xformslib.flxbasic.fl\_get\_mouse \textit{(function)}}

    \vspace{0.5ex}

\hspace{.8\funcindent}\begin{boxedminipage}{\funcwidth}

    \raggedright \textbf{fl\_get\_mouse}()

    \vspace{-1.5ex}

    \rule{\textwidth}{0.5\fboxrule}
\setlength{\parskip}{2ex}

Obtains the current mouse position relative to the root window, and
the current state of the modifier keys and pointer buttons.

-{}-
\setlength{\parskip}{1ex}
      \textbf{Return Value}
    \vspace{-1ex}

      \begin{quote}

window the mouse is in (win), horizontal (x) and vertical
position (y), keymask
      {\it (type=long\_pos, int, int, int\_pos)}

      \end{quote}

\textbf{Note:} 
e.g. win, x, y, kmsk = fl\_get\_mouse()


\textbf{Attention:} 
API change from XForms - upstream was
fl\_get\_mouse(x, y, keymask)


\textbf{Status:} 
Tested + Doc + NoDemo = OK


    \end{boxedminipage}

    \label{xformslib:flxbasic:fl_set_mouse}
    \index{xformslib \textit{(package)}!xformslib.flxbasic \textit{(module)}!xformslib.flxbasic.fl\_set\_mouse \textit{(function)}}

    \vspace{0.5ex}

\hspace{.8\funcindent}\begin{boxedminipage}{\funcwidth}

    \raggedright \textbf{fl\_set\_mouse}(\textit{x}, \textit{y})

    \vspace{-1.5ex}

    \rule{\textwidth}{0.5\fboxrule}
\setlength{\parskip}{2ex}

Moves the mouse to a specific location relative to the root window.
Use this function sparingly, it can be extremely annoying for the user
if the mouse position is changed by a program.

-{}-
\setlength{\parskip}{1ex}
      \textbf{Parameters}
      \vspace{-1ex}

      \begin{quote}
        \begin{Ventry}{x}

          \item[x]


horizontal position
            {\it (type=int)}

          \item[y]


vertical position
            {\it (type=int)}

        \end{Ventry}

      \end{quote}

\textbf{Note:} 
e.g. fl\_set\_mouse(200, 120)


\textbf{Status:} 
Tested + Doc + NoDemo = OK


    \end{boxedminipage}

    \label{xformslib:flxbasic:fl_get_win_mouse}
    \index{xformslib \textit{(package)}!xformslib.flxbasic \textit{(module)}!xformslib.flxbasic.fl\_get\_win\_mouse \textit{(function)}}

    \vspace{0.5ex}

\hspace{.8\funcindent}\begin{boxedminipage}{\funcwidth}

    \raggedright \textbf{fl\_get\_win\_mouse}(\textit{win})

    \vspace{-1.5ex}

    \rule{\textwidth}{0.5\fboxrule}
\setlength{\parskip}{2ex}

Obtains the position of the mouse relative to a certain window, and
the current state of the modifier keys and pointer buttons.

-{}-
\setlength{\parskip}{1ex}
      \textbf{Parameters}
      \vspace{-1ex}

      \begin{quote}
        \begin{Ventry}{xxx}

          \item[win]


window id
            {\it (type=long\_pos)}

        \end{Ventry}

      \end{quote}

      \textbf{Return Value}
    \vspace{-1ex}

      \begin{quote}

window the mouse is in (win), horizontal (x) and vertical
position (y), keymask
      {\it (type=long\_pos, int, int, int\_pos)}

      \end{quote}

\textbf{Note:} 
e.g. win, x, y, keym = fl\_get\_win\_mouse()


\textbf{Attention:} 
API change from XForms - upstream was
fl\_get\_win\_mouse(win, x, y, keymask)


\textbf{Status:} 
Tested + Doc + Demo = OK


    \end{boxedminipage}

    \label{xformslib:flxbasic:fl_get_form_mouse}
    \index{xformslib \textit{(package)}!xformslib.flxbasic \textit{(module)}!xformslib.flxbasic.fl\_get\_form\_mouse \textit{(function)}}

    \vspace{0.5ex}

\hspace{.8\funcindent}\begin{boxedminipage}{\funcwidth}

    \raggedright \textbf{fl\_get\_form\_mouse}(\textit{pFlForm})

    \vspace{-1.5ex}

    \rule{\textwidth}{0.5\fboxrule}
\setlength{\parskip}{2ex}

Obtains the position of the mouse relative to a certain form, and
the current state of the modifier keys and pointer buttons.

-{}-
\setlength{\parskip}{1ex}
      \textbf{Parameters}
      \vspace{-1ex}

      \begin{quote}
        \begin{Ventry}{xxxxxxx}

          \item[pFlForm]


form
            {\it (type=pointer to xfdata.FL\_FORM)}

        \end{Ventry}

      \end{quote}

      \textbf{Return Value}
    \vspace{-1ex}

      \begin{quote}

window the mouse is in (win), horizontal (x), vertical
position (y), keymask
      {\it (type=long\_pos, int, int, int\_pos)}

      \end{quote}

\textbf{Note:} 
e.g. win, x, y, keym = fl\_get\_form\_mouse()


\textbf{Attention:} 
API change from XForms - upstream was
fl\_get\_form\_mouse(fm, x, y, keymask)


\textbf{Status:} 
Tested + Doc + NoDemo = OK


    \end{boxedminipage}

    \label{xformslib:flxbasic:fl_win_to_form}
    \index{xformslib \textit{(package)}!xformslib.flxbasic \textit{(module)}!xformslib.flxbasic.fl\_win\_to\_form \textit{(function)}}

    \vspace{0.5ex}

\hspace{.8\funcindent}\begin{boxedminipage}{\funcwidth}

    \raggedright \textbf{fl\_win\_to\_form}(\textit{win})

    \vspace{-1.5ex}

    \rule{\textwidth}{0.5\fboxrule}
\setlength{\parskip}{2ex}

Returns the form the specified window belongs to.

-{}-
\setlength{\parskip}{1ex}
      \textbf{Parameters}
      \vspace{-1ex}

      \begin{quote}
        \begin{Ventry}{xxx}

          \item[win]


window id
            {\it (type=long\_pos)}

        \end{Ventry}

      \end{quote}

      \textbf{Return Value}
    \vspace{-1ex}

      \begin{quote}

form (pFlForm) or None (on failure)
      {\it (type=pointer to xfdata.FL\_FORM)}

      \end{quote}

\textbf{Note:} 
e.g. pform2 = fl\_win\_to\_form(win1)


\textbf{Status:} 
Tested + Doc + NoDemo = OK


    \end{boxedminipage}

    \label{xformslib:flxbasic:fl_set_form_icon}
    \index{xformslib \textit{(package)}!xformslib.flxbasic \textit{(module)}!xformslib.flxbasic.fl\_set\_form\_icon \textit{(function)}}

    \vspace{0.5ex}

\hspace{.8\funcindent}\begin{boxedminipage}{\funcwidth}

    \raggedright \textbf{fl\_set\_form\_icon}(\textit{pFlForm}, \textit{icon}, \textit{mask})

    \vspace{-1.5ex}

    \rule{\textwidth}{0.5\fboxrule}
\setlength{\parskip}{2ex}

Sets or changes the icon shown when a form is iconified.

-{}-
\setlength{\parskip}{1ex}
      \textbf{Parameters}
      \vspace{-1ex}

      \begin{quote}
        \begin{Ventry}{xxxxxxx}

          \item[pFlForm]


form
            {\it (type=pointer to xfdata.FL\_FORM)}

          \item[icon]


icon pixmap id
            {\it (type=long\_pos)}

          \item[mask]


mask pixmap id
            {\it (type=long\_pos)}

        \end{Ventry}

      \end{quote}

\textbf{Note:} 
e.g. \emph{todo}


\textbf{Status:} 
Tested + Doc + Demo = OK


    \end{boxedminipage}

    \label{xformslib:flxbasic:fl_get_decoration_sizes}
    \index{xformslib \textit{(package)}!xformslib.flxbasic \textit{(module)}!xformslib.flxbasic.fl\_get\_decoration\_sizes \textit{(function)}}

    \vspace{0.5ex}

\hspace{.8\funcindent}\begin{boxedminipage}{\funcwidth}

    \raggedright \textbf{fl\_get\_decoration\_sizes}(\textit{pFlForm})

    \vspace{-1.5ex}

    \rule{\textwidth}{0.5\fboxrule}
\setlength{\parskip}{2ex}

Returns the sizes of the ``decorations'' the window manager puts around
a form's window.

-{}-
\setlength{\parskip}{1ex}
      \textbf{Parameters}
      \vspace{-1ex}

      \begin{quote}
        \begin{Ventry}{xxxxxxx}

          \item[pFlForm]


form
            {\it (type=pointer to xfdata.FL\_FORM)}

        \end{Ventry}

      \end{quote}

      \textbf{Return Value}
    \vspace{-1ex}

      \begin{quote}

0 (on success) or -1 (if the form isn't visible or it's a form
embedded into another form), top size, right size, bottom size, left
size
      {\it (type=int, int, int, int)}

      \end{quote}

\textbf{Note:} 
e.g. dsize = fl\_get\_decoration\_sizes(pform)


\textbf{Attention:} 
API change from XForms - upstream was
fl\_get\_decoration\_sizes(pFlForm, top, right, bottom, left)


\textbf{Status:} 
Tested + Doc + NoDemo = OK


    \end{boxedminipage}

    \label{xformslib:flxbasic:fl_raise_form}
    \index{xformslib \textit{(package)}!xformslib.flxbasic \textit{(module)}!xformslib.flxbasic.fl\_raise\_form \textit{(function)}}

    \vspace{0.5ex}

\hspace{.8\funcindent}\begin{boxedminipage}{\funcwidth}

    \raggedright \textbf{fl\_raise\_form}(\textit{pFlForm})

    \vspace{-1.5ex}

    \rule{\textwidth}{0.5\fboxrule}
\setlength{\parskip}{2ex}

Raises a form to the top of the screen so no other forms obscure it.

-{}-
\setlength{\parskip}{1ex}
      \textbf{Parameters}
      \vspace{-1ex}

      \begin{quote}
        \begin{Ventry}{xxxxxxx}

          \item[pFlForm]


form to be raised
            {\it (type=pointer to xfdata.FL\_FORM)}

        \end{Ventry}

      \end{quote}

\textbf{Note:} 
e.g. fl\_raise\_form(pform2)


\textbf{Status:} 
Tested + Doc + NoDemo = OK


    \end{boxedminipage}

    \label{xformslib:flxbasic:fl_lower_form}
    \index{xformslib \textit{(package)}!xformslib.flxbasic \textit{(module)}!xformslib.flxbasic.fl\_lower\_form \textit{(function)}}

    \vspace{0.5ex}

\hspace{.8\funcindent}\begin{boxedminipage}{\funcwidth}

    \raggedright \textbf{fl\_lower\_form}(\textit{pFlForm})

    \vspace{-1.5ex}

    \rule{\textwidth}{0.5\fboxrule}
\setlength{\parskip}{2ex}

Lowers a form to the bottom of the stack.

-{}-
\setlength{\parskip}{1ex}
      \textbf{Parameters}
      \vspace{-1ex}

      \begin{quote}
        \begin{Ventry}{xxxxxxx}

          \item[pFlForm]


form to be lowered
            {\it (type=pointer to xfdata.FL\_FORM)}

        \end{Ventry}

      \end{quote}

\textbf{Note:} 
e.g. fl\_lower\_form(pform2)


\textbf{Status:} 
Tested + Doc + NoDemo = OK


    \end{boxedminipage}

    \label{xformslib:flxbasic:fl_set_foreground}
    \index{xformslib \textit{(package)}!xformslib.flxbasic \textit{(module)}!xformslib.flxbasic.fl\_set\_foreground \textit{(function)}}

    \vspace{0.5ex}

\hspace{.8\funcindent}\begin{boxedminipage}{\funcwidth}

    \raggedright \textbf{fl\_set\_foreground}(\textit{gc}, \textit{colr})

    \vspace{-1.5ex}

    \rule{\textwidth}{0.5\fboxrule}
\setlength{\parskip}{2ex}

Sets foreground color in Graphics Contexts (GCs) other than the XForms
library's default.

-{}-
\setlength{\parskip}{1ex}
      \textbf{Parameters}
      \vspace{-1ex}

      \begin{quote}
        \begin{Ventry}{xxxx}

          \item[gc]


Graphics context number
            {\it (type=pointer to xfdata.GC?)}

          \item[colr]


color value to be set as foreground
            {\it (type=long\_pos)}

        \end{Ventry}

      \end{quote}

\textbf{Notes:}
\begin{quote}
  \begin{itemize}

  \item
    \setlength{\parskip}{0.6ex}

e.g. gc = fl\_state{[}fl\_get\_vclass(){]}.gc{[}0{]} ??


  \item 
e.g. fl\_set\_foreground(gc, xfdata.FL\_LAWNGREEN)


\end{itemize}

\end{quote}

\textbf{Status:} 
Untested + NoDoc + NoDemo = NOT OK (NULL pointer access)


    \end{boxedminipage}

    \label{xformslib:flxbasic:fl_set_background}
    \index{xformslib \textit{(package)}!xformslib.flxbasic \textit{(module)}!xformslib.flxbasic.fl\_set\_background \textit{(function)}}

    \vspace{0.5ex}

\hspace{.8\funcindent}\begin{boxedminipage}{\funcwidth}

    \raggedright \textbf{fl\_set\_background}(\textit{gc}, \textit{colr})

    \vspace{-1.5ex}

    \rule{\textwidth}{0.5\fboxrule}
\setlength{\parskip}{2ex}

Sets background color in Graphics contexts (GCs) other than the XForms
library's default.

-{}-
\setlength{\parskip}{1ex}
      \textbf{Parameters}
      \vspace{-1ex}

      \begin{quote}
        \begin{Ventry}{xxxx}

          \item[gc]


Graphics context number
            {\it (type=pointer to xfdata.GC?)}

          \item[colr]


color value to be set as background
            {\it (type=long\_pos)}

        \end{Ventry}

      \end{quote}

\textbf{Note:} 
e.g. gc = fl\_state{[}fl\_get\_vclass(){]}.gc{[}0{]} ??
fl\_set\_foreground(gc, xfdata.FL\_HONEYDEW)


\textbf{Status:} 
Untested + NoDoc + NoDemo = NOT OK (NULL pointer access)


    \end{boxedminipage}

    \label{xformslib:flxbasic:fl_wincreate}
    \index{xformslib \textit{(package)}!xformslib.flxbasic \textit{(module)}!xformslib.flxbasic.fl\_wincreate \textit{(function)}}

    \vspace{0.5ex}

\hspace{.8\funcindent}\begin{boxedminipage}{\funcwidth}

    \raggedright \textbf{fl\_wincreate}(\textit{title})

    \vspace{-1.5ex}

    \rule{\textwidth}{0.5\fboxrule}
\setlength{\parskip}{2ex}

Creates a window with a specified title.

-{}-
\setlength{\parskip}{1ex}
      \textbf{Parameters}
      \vspace{-1ex}

      \begin{quote}
        \begin{Ventry}{xxxxx}

          \item[title]


title of the window
            {\it (type=str)}

        \end{Ventry}

      \end{quote}

      \textbf{Return Value}
    \vspace{-1ex}

      \begin{quote}

created window id (win)
      {\it (type=long\_pos)}

      \end{quote}

\textbf{Note:} 
e.g. win2 = fl\_wincreate(``My long title'')


\textbf{Status:} 
Tested + Doc + NoDemo = OK


    \end{boxedminipage}

    \label{xformslib:flxbasic:fl_winshow}
    \index{xformslib \textit{(package)}!xformslib.flxbasic \textit{(module)}!xformslib.flxbasic.fl\_winshow \textit{(function)}}

    \vspace{0.5ex}

\hspace{.8\funcindent}\begin{boxedminipage}{\funcwidth}

    \raggedright \textbf{fl\_winshow}(\textit{win})

    \vspace{-1.5ex}

    \rule{\textwidth}{0.5\fboxrule}
\setlength{\parskip}{2ex}

Shows the window (created with fl\_wincreate).

-{}-
\setlength{\parskip}{1ex}
      \textbf{Parameters}
      \vspace{-1ex}

      \begin{quote}
        \begin{Ventry}{xxx}

          \item[win]


window id to show
            {\it (type=long\_pos)}

        \end{Ventry}

      \end{quote}

      \textbf{Return Value}
    \vspace{-1ex}

      \begin{quote}

window id shown (win)
      {\it (type=long\_pos)}

      \end{quote}

\textbf{Note:} 
e.g. winw = fl\_winshow(win2)


\textbf{Status:} 
Tested + Doc + NoDemo = OK


    \end{boxedminipage}

    \label{xformslib:flxbasic:fl_winopen}
    \index{xformslib \textit{(package)}!xformslib.flxbasic \textit{(module)}!xformslib.flxbasic.fl\_winopen \textit{(function)}}

    \vspace{0.5ex}

\hspace{.8\funcindent}\begin{boxedminipage}{\funcwidth}

    \raggedright \textbf{fl\_winopen}(\textit{title})

    \vspace{-1.5ex}

    \rule{\textwidth}{0.5\fboxrule}
\setlength{\parskip}{2ex}

Opens (creates and shows) a toplevel window with the specified title.

-{}-
\setlength{\parskip}{1ex}
      \textbf{Parameters}
      \vspace{-1ex}

      \begin{quote}
        \begin{Ventry}{xxxxx}

          \item[title]


title of the window
            {\it (type=str)}

        \end{Ventry}

      \end{quote}

      \textbf{Return Value}
    \vspace{-1ex}

      \begin{quote}

created window id (win)
      {\it (type=long\_pos)}

      \end{quote}

\textbf{Note:} 
e.g. win2 = fl\_winopen(``My long title'')


\textbf{Status:} 
Tested + Doc + Demo = OK


    \end{boxedminipage}

    \label{xformslib:flxbasic:fl_winhide}
    \index{xformslib \textit{(package)}!xformslib.flxbasic \textit{(module)}!xformslib.flxbasic.fl\_winhide \textit{(function)}}

    \vspace{0.5ex}

\hspace{.8\funcindent}\begin{boxedminipage}{\funcwidth}

    \raggedright \textbf{fl\_winhide}(\textit{win})

    \vspace{-1.5ex}

    \rule{\textwidth}{0.5\fboxrule}
\setlength{\parskip}{2ex}

Hides a shown window.

-{}-
\setlength{\parskip}{1ex}
      \textbf{Parameters}
      \vspace{-1ex}

      \begin{quote}
        \begin{Ventry}{xxx}

          \item[win]


window id to hide
            {\it (type=long\_pos)}

        \end{Ventry}

      \end{quote}

\textbf{Note:} 
e.g. fl\_winhide(win2)


\textbf{Status:} 
Tested + Doc + Demo = OK


    \end{boxedminipage}

    \label{xformslib:flxbasic:fl_winclose}
    \index{xformslib \textit{(package)}!xformslib.flxbasic \textit{(module)}!xformslib.flxbasic.fl\_winclose \textit{(function)}}

    \vspace{0.5ex}

\hspace{.8\funcindent}\begin{boxedminipage}{\funcwidth}

    \raggedright \textbf{fl\_winclose}(\textit{win})

    \vspace{-1.5ex}

    \rule{\textwidth}{0.5\fboxrule}
\setlength{\parskip}{2ex}

Closes (hides and destroys) the specified window.

-{}-
\setlength{\parskip}{1ex}
      \textbf{Parameters}
      \vspace{-1ex}

      \begin{quote}
        \begin{Ventry}{xxx}

          \item[win]


window id to close
            {\it (type=long\_pos)}

        \end{Ventry}

      \end{quote}

\textbf{Note:} 
e.g. fl\_winclose(win2)


\textbf{Status:} 
Tested + Doc + NoDemo = OK


    \end{boxedminipage}

    \label{xformslib:flxbasic:fl_winset}
    \index{xformslib \textit{(package)}!xformslib.flxbasic \textit{(module)}!xformslib.flxbasic.fl\_winset \textit{(function)}}

    \vspace{0.5ex}

\hspace{.8\funcindent}\begin{boxedminipage}{\funcwidth}

    \raggedright \textbf{fl\_winset}(\textit{win})

    \vspace{-1.5ex}

    \rule{\textwidth}{0.5\fboxrule}
\setlength{\parskip}{2ex}

Sets the ``current window'', defined as the window the object that uses
the drawing routine belongs to.

-{}-
\setlength{\parskip}{1ex}
      \textbf{Parameters}
      \vspace{-1ex}

      \begin{quote}
        \begin{Ventry}{xxx}

          \item[win]


window id to be set
            {\it (type=long\_pos)}

        \end{Ventry}

      \end{quote}

\textbf{Note:} 
e.g. fl\_winset(win3)


\textbf{Status:} 
Tested + Doc + Demo = OK


    \end{boxedminipage}

    \label{xformslib:flxbasic:fl_winreparent}
    \index{xformslib \textit{(package)}!xformslib.flxbasic \textit{(module)}!xformslib.flxbasic.fl\_winreparent \textit{(function)}}

    \vspace{0.5ex}

\hspace{.8\funcindent}\begin{boxedminipage}{\funcwidth}

    \raggedright \textbf{fl\_winreparent}(\textit{win}, \textit{winnewparent})

    \vspace{-1.5ex}

    \rule{\textwidth}{0.5\fboxrule}
\setlength{\parskip}{2ex}

Makes a toplevel window a subwindow of another (new parent) window;
both the window and the parent window must be valid ones.

-{}-
\setlength{\parskip}{1ex}
      \textbf{Parameters}
      \vspace{-1ex}

      \begin{quote}
        \begin{Ventry}{xxxxxxxxxxxx}

          \item[win]


window id to be made a subwindow
            {\it (type=long\_pos)}

          \item[winnewparent]


window id to become its new parent window
            {\it (type=long\_pos)}

        \end{Ventry}

      \end{quote}

      \textbf{Return Value}
    \vspace{-1ex}

      \begin{quote}

num., or -1 (on failure)
      {\it (type=int)}

      \end{quote}

\textbf{Note:} 
e.g. exitval = fl\_winreparent(win1, win3)


\textbf{Status:} 
Tested + Doc + NoDemo = OK


    \end{boxedminipage}

    \label{xformslib:flxbasic:fl_winfocus}
    \index{xformslib \textit{(package)}!xformslib.flxbasic \textit{(module)}!xformslib.flxbasic.fl\_winfocus \textit{(function)}}

    \vspace{0.5ex}

\hspace{.8\funcindent}\begin{boxedminipage}{\funcwidth}

    \raggedright \textbf{fl\_winfocus}(\textit{win})

    \vspace{-1.5ex}

    \rule{\textwidth}{0.5\fboxrule}
\setlength{\parskip}{2ex}

Keyboard input is directed to the specified window, overriding the
keyboard focus assignment.

-{}-
\setlength{\parskip}{1ex}
      \textbf{Parameters}
      \vspace{-1ex}

      \begin{quote}
        \begin{Ventry}{xxx}

          \item[win]


window id
            {\it (type=long\_pos)}

        \end{Ventry}

      \end{quote}

\textbf{Note:} 
e.g. fl\_winfocus(win3)


\textbf{Status:} 
Tested + Doc + NoDemo = OK


    \end{boxedminipage}

    \label{xformslib:flxbasic:fl_winget}
    \index{xformslib \textit{(package)}!xformslib.flxbasic \textit{(module)}!xformslib.flxbasic.fl\_winget \textit{(function)}}

    \vspace{0.5ex}

\hspace{.8\funcindent}\begin{boxedminipage}{\funcwidth}

    \raggedright \textbf{fl\_winget}()

    \vspace{-1.5ex}

    \rule{\textwidth}{0.5\fboxrule}
\setlength{\parskip}{2ex}

Queries the current window. One caveat about fl\_winget() is that it
can return None if called outside of an object's event handler, depending
on where the mouse is. Thus, the return value of this function should be
checked when called outside of an object handler.

-{}-
\setlength{\parskip}{1ex}
      \textbf{Return Value}
    \vspace{-1ex}

      \begin{quote}

window id (win)
      {\it (type=long\_pos)}

      \end{quote}

\textbf{Note:} 
e.g. currwin = fl\_winget()


\textbf{Status:} 
Tested + Doc + NoDemo = OK


    \end{boxedminipage}

    \label{xformslib:flxbasic:fl_iconify}
    \index{xformslib \textit{(package)}!xformslib.flxbasic \textit{(module)}!xformslib.flxbasic.fl\_iconify \textit{(function)}}

    \vspace{0.5ex}

\hspace{.8\funcindent}\begin{boxedminipage}{\funcwidth}

    \raggedright \textbf{fl\_iconify}(\textit{win})

    \vspace{-1.5ex}

    \rule{\textwidth}{0.5\fboxrule}
\setlength{\parskip}{2ex}

Iconifies the specified window.

-{}-
\setlength{\parskip}{1ex}
      \textbf{Parameters}
      \vspace{-1ex}

      \begin{quote}
        \begin{Ventry}{xxx}

          \item[win]


window id
            {\it (type=long\_pos)}

        \end{Ventry}

      \end{quote}

      \textbf{Return Value}
    \vspace{-1ex}

      \begin{quote}

num.
      {\it (type=int)}

      \end{quote}

\textbf{Note:} 
e.g. fl\_iconify(win2)


\textbf{Status:} 
Tested + Doc + NoDemo = OK


    \end{boxedminipage}

    \label{xformslib:flxbasic:fl_winresize}
    \index{xformslib \textit{(package)}!xformslib.flxbasic \textit{(module)}!xformslib.flxbasic.fl\_winresize \textit{(function)}}

    \vspace{0.5ex}

\hspace{.8\funcindent}\begin{boxedminipage}{\funcwidth}

    \raggedright \textbf{fl\_winresize}(\textit{win}, \textit{w}, \textit{h})

    \vspace{-1.5ex}

    \rule{\textwidth}{0.5\fboxrule}
\setlength{\parskip}{2ex}

Resizes a window.

-{}-
\setlength{\parskip}{1ex}
      \textbf{Parameters}
      \vspace{-1ex}

      \begin{quote}
        \begin{Ventry}{xxx}

          \item[win]


window id to resize
            {\it (type=long\_pos)}

          \item[w]


new width in coord units
            {\it (type=int)}

          \item[h]


new height in coord units
            {\it (type=int)}

        \end{Ventry}

      \end{quote}

\textbf{Note:} 
e.g. fl\_winresize(win6, 547, 624)


\textbf{Status:} 
Tested + Doc + NoDemo = OK


    \end{boxedminipage}

    \label{xformslib:flxbasic:fl_winmove}
    \index{xformslib \textit{(package)}!xformslib.flxbasic \textit{(module)}!xformslib.flxbasic.fl\_winmove \textit{(function)}}

    \vspace{0.5ex}

\hspace{.8\funcindent}\begin{boxedminipage}{\funcwidth}

    \raggedright \textbf{fl\_winmove}(\textit{win}, \textit{x}, \textit{y})

    \vspace{-1.5ex}

    \rule{\textwidth}{0.5\fboxrule}
\setlength{\parskip}{2ex}

Moves the specified window to a new position.

-{}-
\setlength{\parskip}{1ex}
      \textbf{Parameters}
      \vspace{-1ex}

      \begin{quote}
        \begin{Ventry}{xxx}

          \item[win]


window id to move to a new position
            {\it (type=long\_pos)}

          \item[x]


new horizontal position (upper-left corner)
            {\it (type=int)}

          \item[y]


new vertical position (upper-left corner)
            {\it (type=int)}

        \end{Ventry}

      \end{quote}

\textbf{Note:} 
e.g. fl\_winmove(win5, 116, 331)


\textbf{Status:} 
Tested + Doc + NoDemo = OK


    \end{boxedminipage}

    \label{xformslib:flxbasic:fl_winreshape}
    \index{xformslib \textit{(package)}!xformslib.flxbasic \textit{(module)}!xformslib.flxbasic.fl\_winreshape \textit{(function)}}

    \vspace{0.5ex}

\hspace{.8\funcindent}\begin{boxedminipage}{\funcwidth}

    \raggedright \textbf{fl\_winreshape}(\textit{win}, \textit{x}, \textit{y}, \textit{w}, \textit{h})

    \vspace{-1.5ex}

    \rule{\textwidth}{0.5\fboxrule}
\setlength{\parskip}{2ex}

Reshapes (resizes and moves) a window.

-{}-
\setlength{\parskip}{1ex}
      \textbf{Parameters}
      \vspace{-1ex}

      \begin{quote}
        \begin{Ventry}{xxx}

          \item[win]


window id to reshape
            {\it (type=long\_pos)}

          \item[x]


new horizontal position (upper-left corner)
            {\it (type=int)}

          \item[y]


new vertical position (upper-left corner)
            {\it (type=int)}

          \item[w]


new width in coord units
            {\it (type=int)}

          \item[h]


new height in coord units
            {\it (type=int)}

        \end{Ventry}

      \end{quote}

\textbf{Note:} 
e.g. fl\_winreshape(win5, 116, 331, 144, 182)


\textbf{Status:} 
Tested + Doc + NoDemo = OK


    \end{boxedminipage}

    \label{xformslib:flxbasic:fl_winicon}
    \index{xformslib \textit{(package)}!xformslib.flxbasic \textit{(module)}!xformslib.flxbasic.fl\_winicon \textit{(function)}}

    \vspace{0.5ex}

\hspace{.8\funcindent}\begin{boxedminipage}{\funcwidth}

    \raggedright \textbf{fl\_winicon}(\textit{win}, \textit{icon}, \textit{mask})

    \vspace{-1.5ex}

    \rule{\textwidth}{0.5\fboxrule}
\setlength{\parskip}{2ex}

Installs an icon for the window.

-{}-
\setlength{\parskip}{1ex}
      \textbf{Parameters}
      \vspace{-1ex}

      \begin{quote}
        \begin{Ventry}{xxxx}

          \item[win]


window id
            {\it (type=long\_pos)}

          \item[icon]


pixmap icon id to be installed in window
            {\it (type=long\_pos)}

          \item[mask]


pixmap mask id
            {\it (type=long\_pos)}

        \end{Ventry}

      \end{quote}

\textbf{Note:} 
e.g. \emph{todo}


\textbf{Status:} 
Untested + NoDoc + NoDemo = NOT OK


    \end{boxedminipage}

    \label{xformslib:flxbasic:fl_winbackground}
    \index{xformslib \textit{(package)}!xformslib.flxbasic \textit{(module)}!xformslib.flxbasic.fl\_winbackground \textit{(function)}}

    \vspace{0.5ex}

\hspace{.8\funcindent}\begin{boxedminipage}{\funcwidth}

    \raggedright \textbf{fl\_winbackground}(\textit{win}, \textit{bgcolr})

    \vspace{-1.5ex}

    \rule{\textwidth}{0.5\fboxrule}
\setlength{\parskip}{2ex}

Sets the background of window to a certain color.

-{}-
\setlength{\parskip}{1ex}
      \textbf{Parameters}
      \vspace{-1ex}

      \begin{quote}
        \begin{Ventry}{xxxxxx}

          \item[win]


window id
            {\it (type=long\_pos)}

          \item[bgcolr]


background color to be set
            {\it (type=long\_pos)}

        \end{Ventry}

      \end{quote}

\textbf{Note:} 
e.g. fl\_winbackground(win1, xfdata.FL\_GHOSTWHITE)


\textbf{Status:} 
Tested + NoDoc + Demo = OK


    \end{boxedminipage}

    \label{xformslib:flxbasic:fl_winbackground}
    \index{xformslib \textit{(package)}!xformslib.flxbasic \textit{(module)}!xformslib.flxbasic.fl\_winbackground \textit{(function)}}

    \vspace{0.5ex}

\hspace{.8\funcindent}\begin{boxedminipage}{\funcwidth}

    \raggedright \textbf{fl\_win\_background}(\textit{win}, \textit{bgcolr})

    \vspace{-1.5ex}

    \rule{\textwidth}{0.5\fboxrule}
\setlength{\parskip}{2ex}

Sets the background of window to a certain color.

-{}-
\setlength{\parskip}{1ex}
      \textbf{Parameters}
      \vspace{-1ex}

      \begin{quote}
        \begin{Ventry}{xxxxxx}

          \item[win]


window id
            {\it (type=long\_pos)}

          \item[bgcolr]


background color to be set
            {\it (type=long\_pos)}

        \end{Ventry}

      \end{quote}

\textbf{Note:} 
e.g. fl\_winbackground(win1, xfdata.FL\_GHOSTWHITE)


\textbf{Status:} 
Tested + NoDoc + Demo = OK


    \end{boxedminipage}

    \label{xformslib:flxbasic:fl_winstepsize}
    \index{xformslib \textit{(package)}!xformslib.flxbasic \textit{(module)}!xformslib.flxbasic.fl\_winstepsize \textit{(function)}}

    \vspace{0.5ex}

\hspace{.8\funcindent}\begin{boxedminipage}{\funcwidth}

    \raggedright \textbf{fl\_winstepsize}(\textit{win}, \textit{xunit}, \textit{yunit})

    \vspace{-1.5ex}

    \rule{\textwidth}{0.5\fboxrule}
\setlength{\parskip}{2ex}

Sets the steps by which the size of a window can be changed. Changes
to the window size will be multiples of specified units after this
call. Note that this only applies to interactive resizing.

-{}-
\setlength{\parskip}{1ex}
      \textbf{Parameters}
      \vspace{-1ex}

      \begin{quote}
        \begin{Ventry}{xxxxx}

          \item[win]


window id
            {\it (type=long\_pos)}

          \item[xunit]


number of pixels of changes per unit in horizontal direction
            {\it (type=int)}

          \item[yunit]


number of pixels of changes per unit in vertical direction
            {\it (type=int)}

        \end{Ventry}

      \end{quote}

\textbf{Note:} 
e.g. fl\_winstepsize(win0, 10, 10)


\textbf{Status:} 
Tested + Doc + NoDemo = OK


    \end{boxedminipage}

    \label{xformslib:flxbasic:fl_winstepsize}
    \index{xformslib \textit{(package)}!xformslib.flxbasic \textit{(module)}!xformslib.flxbasic.fl\_winstepsize \textit{(function)}}

    \vspace{0.5ex}

\hspace{.8\funcindent}\begin{boxedminipage}{\funcwidth}

    \raggedright \textbf{fl\_winstepunit}(\textit{win}, \textit{xunit}, \textit{yunit})

    \vspace{-1.5ex}

    \rule{\textwidth}{0.5\fboxrule}
\setlength{\parskip}{2ex}

Sets the steps by which the size of a window can be changed. Changes
to the window size will be multiples of specified units after this
call. Note that this only applies to interactive resizing.

-{}-
\setlength{\parskip}{1ex}
      \textbf{Parameters}
      \vspace{-1ex}

      \begin{quote}
        \begin{Ventry}{xxxxx}

          \item[win]


window id
            {\it (type=long\_pos)}

          \item[xunit]


number of pixels of changes per unit in horizontal direction
            {\it (type=int)}

          \item[yunit]


number of pixels of changes per unit in vertical direction
            {\it (type=int)}

        \end{Ventry}

      \end{quote}

\textbf{Note:} 
e.g. fl\_winstepsize(win0, 10, 10)


\textbf{Status:} 
Tested + Doc + NoDemo = OK


    \end{boxedminipage}

    \label{xformslib:flxbasic:fl_winstepsize}
    \index{xformslib \textit{(package)}!xformslib.flxbasic \textit{(module)}!xformslib.flxbasic.fl\_winstepsize \textit{(function)}}

    \vspace{0.5ex}

\hspace{.8\funcindent}\begin{boxedminipage}{\funcwidth}

    \raggedright \textbf{fl\_set\_winstepunit}(\textit{win}, \textit{xunit}, \textit{yunit})

    \vspace{-1.5ex}

    \rule{\textwidth}{0.5\fboxrule}
\setlength{\parskip}{2ex}

Sets the steps by which the size of a window can be changed. Changes
to the window size will be multiples of specified units after this
call. Note that this only applies to interactive resizing.

-{}-
\setlength{\parskip}{1ex}
      \textbf{Parameters}
      \vspace{-1ex}

      \begin{quote}
        \begin{Ventry}{xxxxx}

          \item[win]


window id
            {\it (type=long\_pos)}

          \item[xunit]


number of pixels of changes per unit in horizontal direction
            {\it (type=int)}

          \item[yunit]


number of pixels of changes per unit in vertical direction
            {\it (type=int)}

        \end{Ventry}

      \end{quote}

\textbf{Note:} 
e.g. fl\_winstepsize(win0, 10, 10)


\textbf{Status:} 
Tested + Doc + NoDemo = OK


    \end{boxedminipage}

    \label{xformslib:flxbasic:fl_winisvalid}
    \index{xformslib \textit{(package)}!xformslib.flxbasic \textit{(module)}!xformslib.flxbasic.fl\_winisvalid \textit{(function)}}

    \vspace{0.5ex}

\hspace{.8\funcindent}\begin{boxedminipage}{\funcwidth}

    \raggedright \textbf{fl\_winisvalid}(\textit{win})

    \vspace{-1.5ex}

    \rule{\textwidth}{0.5\fboxrule}
\setlength{\parskip}{2ex}

Checks if a window id is valid or not. Note that excessive use of
this function may negatively impact performance.

-{}-
\setlength{\parskip}{1ex}
      \textbf{Parameters}
      \vspace{-1ex}

      \begin{quote}
        \begin{Ventry}{xxx}

          \item[win]


window id to evaluate
            {\it (type=long\_pos)}

        \end{Ventry}

      \end{quote}

      \textbf{Return Value}
    \vspace{-1ex}

      \begin{quote}

num.
      {\it (type=int)}

      \end{quote}

\textbf{Note:} 
e.g. if fl\_winisvalid(win3): ...


\textbf{Status:} 
Tested + Doc + NoDemo = OK


    \end{boxedminipage}

    \label{xformslib:flxbasic:fl_wintitle}
    \index{xformslib \textit{(package)}!xformslib.flxbasic \textit{(module)}!xformslib.flxbasic.fl\_wintitle \textit{(function)}}

    \vspace{0.5ex}

\hspace{.8\funcindent}\begin{boxedminipage}{\funcwidth}

    \raggedright \textbf{fl\_wintitle}(\textit{win}, \textit{title})

    \vspace{-1.5ex}

    \rule{\textwidth}{0.5\fboxrule}
\setlength{\parskip}{2ex}

Changes the window title (and its associated icon title).

-{}-
\setlength{\parskip}{1ex}
      \textbf{Parameters}
      \vspace{-1ex}

      \begin{quote}
        \begin{Ventry}{xxxxx}

          \item[win]


window id
            {\it (type=long\_pos)}

          \item[title]


window title to be set
            {\it (type=str)}

        \end{Ventry}

      \end{quote}

\textbf{Note:} 
e.g. fl\_wintitle(``My brand new title'')


\textbf{Status:} 
Tested + Doc + NoDemo = OK


    \end{boxedminipage}

    \label{xformslib:flxbasic:fl_winicontitle}
    \index{xformslib \textit{(package)}!xformslib.flxbasic \textit{(module)}!xformslib.flxbasic.fl\_winicontitle \textit{(function)}}

    \vspace{0.5ex}

\hspace{.8\funcindent}\begin{boxedminipage}{\funcwidth}

    \raggedright \textbf{fl\_winicontitle}(\textit{win}, \textit{title})

    \vspace{-1.5ex}

    \rule{\textwidth}{0.5\fboxrule}
\setlength{\parskip}{2ex}

Changes only the icon title for the window.

-{}-
\setlength{\parskip}{1ex}
      \textbf{Parameters}
      \vspace{-1ex}

      \begin{quote}
        \begin{Ventry}{xxxxx}

          \item[win]


window id
            {\it (type=long\_pos)}

          \item[title]


icon title to be set
            {\it (type=str)}

        \end{Ventry}

      \end{quote}

\textbf{Note:} 
e.g. fl\_winicontitle(win0, ``My icon label'')


\textbf{Status:} 
Tested + Doc + NoDemo = OK


    \end{boxedminipage}

    \label{xformslib:flxbasic:fl_winposition}
    \index{xformslib \textit{(package)}!xformslib.flxbasic \textit{(module)}!xformslib.flxbasic.fl\_winposition \textit{(function)}}

    \vspace{0.5ex}

\hspace{.8\funcindent}\begin{boxedminipage}{\funcwidth}

    \raggedright \textbf{fl\_winposition}(\textit{x}, \textit{y})

    \vspace{-1.5ex}

    \rule{\textwidth}{0.5\fboxrule}
\setlength{\parskip}{2ex}

Sets the position of a window to be opened.

-{}-
\setlength{\parskip}{1ex}
      \textbf{Parameters}
      \vspace{-1ex}

      \begin{quote}
        \begin{Ventry}{x}

          \item[x]


horizontal position of window (upper-left corner)
            {\it (type=int)}

          \item[y]


vertical position of window (upper-left corner)
            {\it (type=int)}

        \end{Ventry}

      \end{quote}

\textbf{Note:} 
e.g. fl\_winposition(140, 123)


\textbf{Status:} 
Tested + Doc + NoDemo = OK


    \end{boxedminipage}

    \label{xformslib:flxbasic:fl_winposition}
    \index{xformslib \textit{(package)}!xformslib.flxbasic \textit{(module)}!xformslib.flxbasic.fl\_winposition \textit{(function)}}

    \vspace{0.5ex}

\hspace{.8\funcindent}\begin{boxedminipage}{\funcwidth}

    \raggedright \textbf{fl\_pref\_winposition}(\textit{x}, \textit{y})

    \vspace{-1.5ex}

    \rule{\textwidth}{0.5\fboxrule}
\setlength{\parskip}{2ex}

Sets the position of a window to be opened.

-{}-
\setlength{\parskip}{1ex}
      \textbf{Parameters}
      \vspace{-1ex}

      \begin{quote}
        \begin{Ventry}{x}

          \item[x]


horizontal position of window (upper-left corner)
            {\it (type=int)}

          \item[y]


vertical position of window (upper-left corner)
            {\it (type=int)}

        \end{Ventry}

      \end{quote}

\textbf{Note:} 
e.g. fl\_winposition(140, 123)


\textbf{Status:} 
Tested + Doc + NoDemo = OK


    \end{boxedminipage}

    \label{xformslib:flxbasic:fl_winminsize}
    \index{xformslib \textit{(package)}!xformslib.flxbasic \textit{(module)}!xformslib.flxbasic.fl\_winminsize \textit{(function)}}

    \vspace{0.5ex}

\hspace{.8\funcindent}\begin{boxedminipage}{\funcwidth}

    \raggedright \textbf{fl\_winminsize}(\textit{win}, \textit{w}, \textit{h})

    \vspace{-1.5ex}

    \rule{\textwidth}{0.5\fboxrule}
\setlength{\parskip}{2ex}

Sets a constraint for a resizable window whose size will be within a
range not less than minumum (to be used before calling fl\_winopen).

-{}-
\setlength{\parskip}{1ex}
      \textbf{Parameters}
      \vspace{-1ex}

      \begin{quote}
        \begin{Ventry}{xxx}

          \item[win]


window id to be set
            {\it (type=long\_pos)}

          \item[w]


minimum width of window in coord units
            {\it (type=int)}

          \item[h]


minimum height of window in coord units
            {\it (type=int)}

        \end{Ventry}

      \end{quote}

\textbf{Note:} 
e.g. fl\_winminsize(win1, 500, 500)


\textbf{Status:} 
Tested + Doc + NoDemo = OK


    \end{boxedminipage}

    \label{xformslib:flxbasic:fl_winmaxsize}
    \index{xformslib \textit{(package)}!xformslib.flxbasic \textit{(module)}!xformslib.flxbasic.fl\_winmaxsize \textit{(function)}}

    \vspace{0.5ex}

\hspace{.8\funcindent}\begin{boxedminipage}{\funcwidth}

    \raggedright \textbf{fl\_winmaxsize}(\textit{win}, \textit{w}, \textit{h})

    \vspace{-1.5ex}

    \rule{\textwidth}{0.5\fboxrule}
\setlength{\parskip}{2ex}

Sets a constraint for a resizable window whose size will be within a
range not bigger than maximum (before calling fl\_winopen).

-{}-
\setlength{\parskip}{1ex}
      \textbf{Parameters}
      \vspace{-1ex}

      \begin{quote}
        \begin{Ventry}{xxx}

          \item[win]


window id to be set
            {\it (type=long\_pos)}

          \item[w]


maximum width of window in coord units
            {\it (type=int)}

          \item[h]


maximum height of window in coord units
            {\it (type=int)}

        \end{Ventry}

      \end{quote}

\textbf{Note:} 
e.g. fl\_winmaxsize(win1, 500, 500)


\textbf{Status:} 
Tested + Doc + NoDemo = OK


    \end{boxedminipage}

    \label{xformslib:flxbasic:fl_winaspect}
    \index{xformslib \textit{(package)}!xformslib.flxbasic \textit{(module)}!xformslib.flxbasic.fl\_winaspect \textit{(function)}}

    \vspace{0.5ex}

\hspace{.8\funcindent}\begin{boxedminipage}{\funcwidth}

    \raggedright \textbf{fl\_winaspect}(\textit{win}, \textit{x}, \textit{y})

    \vspace{-1.5ex}

    \rule{\textwidth}{0.5\fboxrule}
\setlength{\parskip}{2ex}

Sets the aspect ratio of the window for later interactive resizing.

-{}-
\setlength{\parskip}{1ex}
      \textbf{Parameters}
      \vspace{-1ex}

      \begin{quote}
        \begin{Ventry}{xxx}

          \item[win]


window id to be set
            {\it (type=long\_pos)}

          \item[x]


horizontal aspect ratio in coord units
            {\it (type=int)}

          \item[y]


vertical aspect ratio in coord units
            {\it (type=int)}

        \end{Ventry}

      \end{quote}

\textbf{Note:} 
e.g. fl\_winaspect(win0, 2, 4)


\textbf{Status:} 
Tested + Doc + NoDemo = OK


    \end{boxedminipage}

    \label{xformslib:flxbasic:fl_reset_winconstraints}
    \index{xformslib \textit{(package)}!xformslib.flxbasic \textit{(module)}!xformslib.flxbasic.fl\_reset\_winconstraints \textit{(function)}}

    \vspace{0.5ex}

\hspace{.8\funcindent}\begin{boxedminipage}{\funcwidth}

    \raggedright \textbf{fl\_reset\_winconstraints}(\textit{win})

    \vspace{-1.5ex}

    \rule{\textwidth}{0.5\fboxrule}
\setlength{\parskip}{2ex}

Changes constraints (size and aspect ratio) on an active window.

-{}-
\setlength{\parskip}{1ex}
      \textbf{Parameters}
      \vspace{-1ex}

      \begin{quote}
        \begin{Ventry}{xxx}

          \item[win]


window to be reset
            {\it (type=long\_pos)}

        \end{Ventry}

      \end{quote}

\textbf{Note:} 
e.g. fl\_reset\_constraints(win0)


\textbf{Status:} 
Tested + Doc + NoDemo = OK


    \end{boxedminipage}

    \label{xformslib:flxbasic:fl_winsize}
    \index{xformslib \textit{(package)}!xformslib.flxbasic \textit{(module)}!xformslib.flxbasic.fl\_winsize \textit{(function)}}

    \vspace{0.5ex}

\hspace{.8\funcindent}\begin{boxedminipage}{\funcwidth}

    \raggedright \textbf{fl\_winsize}(\textit{w}, \textit{h})

    \vspace{-1.5ex}

    \rule{\textwidth}{0.5\fboxrule}
\setlength{\parskip}{2ex}

Sets the preferred window size (before calling fl\_winopen), and
makes the window non-resizeable.

-{}-
\setlength{\parskip}{1ex}
      \textbf{Parameters}
      \vspace{-1ex}

      \begin{quote}
        \begin{Ventry}{x}

          \item[w]


width in coord units
            {\it (type=int)}

          \item[h]


height in coord units
            {\it (type=int)}

        \end{Ventry}

      \end{quote}

\textbf{Note:} 
e.g. fl\_winsize(700, 600)


\textbf{Status:} 
Tested + Doc + NoDemo = OK


    \end{boxedminipage}

    \label{xformslib:flxbasic:fl_winsize}
    \index{xformslib \textit{(package)}!xformslib.flxbasic \textit{(module)}!xformslib.flxbasic.fl\_winsize \textit{(function)}}

    \vspace{0.5ex}

\hspace{.8\funcindent}\begin{boxedminipage}{\funcwidth}

    \raggedright \textbf{fl\_pref\_winsize}(\textit{w}, \textit{h})

    \vspace{-1.5ex}

    \rule{\textwidth}{0.5\fboxrule}
\setlength{\parskip}{2ex}

Sets the preferred window size (before calling fl\_winopen), and
makes the window non-resizeable.

-{}-
\setlength{\parskip}{1ex}
      \textbf{Parameters}
      \vspace{-1ex}

      \begin{quote}
        \begin{Ventry}{x}

          \item[w]


width in coord units
            {\it (type=int)}

          \item[h]


height in coord units
            {\it (type=int)}

        \end{Ventry}

      \end{quote}

\textbf{Note:} 
e.g. fl\_winsize(700, 600)


\textbf{Status:} 
Tested + Doc + NoDemo = OK


    \end{boxedminipage}

    \label{xformslib:flxbasic:fl_initial_winsize}
    \index{xformslib \textit{(package)}!xformslib.flxbasic \textit{(module)}!xformslib.flxbasic.fl\_initial\_winsize \textit{(function)}}

    \vspace{0.5ex}

\hspace{.8\funcindent}\begin{boxedminipage}{\funcwidth}

    \raggedright \textbf{fl\_initial\_winsize}(\textit{w}, \textit{h})

    \vspace{-1.5ex}

    \rule{\textwidth}{0.5\fboxrule}
\setlength{\parskip}{2ex}

Sets the preferred window size (before calling fl\_winopen).

-{}-
\setlength{\parskip}{1ex}
      \textbf{Parameters}
      \vspace{-1ex}

      \begin{quote}
        \begin{Ventry}{x}

          \item[w]


width in coord units
            {\it (type=int)}

          \item[h]


height in coord units
            {\it (type=int)}

        \end{Ventry}

      \end{quote}

\textbf{Note:} 
e.g. fl\_initial\_winsize(700, 600)


\textbf{Status:} 
Tested + Doc + Demo = OK


    \end{boxedminipage}

    \label{xformslib:flxbasic:fl_initial_winstate}
    \index{xformslib \textit{(package)}!xformslib.flxbasic \textit{(module)}!xformslib.flxbasic.fl\_initial\_winstate \textit{(function)}}

    \vspace{0.5ex}

\hspace{.8\funcindent}\begin{boxedminipage}{\funcwidth}

    \raggedright \textbf{fl\_initial\_winstate}(\textit{state})

    \vspace{-1.5ex}

    \rule{\textwidth}{0.5\fboxrule}
\setlength{\parskip}{2ex}

Sets initial state, normal or iconic, of the window.

-{}-
\setlength{\parskip}{1ex}
      \textbf{Parameters}
      \vspace{-1ex}

      \begin{quote}
        \begin{Ventry}{xxxxx}

          \item[state]


window state to be set. Values (from xfdata.py) NormalState,
IconicState
            {\it (type=int)}

        \end{Ventry}

      \end{quote}

\textbf{Note:} 
e.g. fl\_initial\_winstate(xfdata.IconicState)


\textbf{Status:} 
Tested + Doc + NoDemo = OK


    \end{boxedminipage}

    \label{xformslib:flxbasic:fl_create_colormap}
    \index{xformslib \textit{(package)}!xformslib.flxbasic \textit{(module)}!xformslib.flxbasic.fl\_create\_colormap \textit{(function)}}

    \vspace{0.5ex}

\hspace{.8\funcindent}\begin{boxedminipage}{\funcwidth}

    \raggedright \textbf{fl\_create\_colormap}(\textit{pXVisualInfo}, \textit{nfill})

    \vspace{-1.5ex}

    \rule{\textwidth}{0.5\fboxrule}
\setlength{\parskip}{2ex}

Creates a colormap appropriate for a given visual to be used with
a canvas.

-{}-
\setlength{\parskip}{1ex}
      \textbf{Parameters}
      \vspace{-1ex}

      \begin{quote}
        \begin{Ventry}{xxxxxxxxxxxx}

          \item[pXVisualInfo]


XVisualInfo class instance
            {\it (type=pointer to xfdata.XVisualInfo)}

          \item[nfill]


how many colors in the newly created colormap should be filled with
XForms' default colors (to avoid flashing effects)
            {\it (type=int)}

        \end{Ventry}

      \end{quote}

      \textbf{Return Value}
    \vspace{-1ex}

      \begin{quote}

created colormap
      {\it (type=long\_pos)}

      \end{quote}

\textbf{Note:} 
e.g. \emph{todo}


\textbf{Status:} 
Untested + Doc + NoDemo = NOT OK


    \end{boxedminipage}

    \label{xformslib:flxbasic:fl_wingeometry}
    \index{xformslib \textit{(package)}!xformslib.flxbasic \textit{(module)}!xformslib.flxbasic.fl\_wingeometry \textit{(function)}}

    \vspace{0.5ex}

\hspace{.8\funcindent}\begin{boxedminipage}{\funcwidth}

    \raggedright \textbf{fl\_wingeometry}(\textit{x}, \textit{y}, \textit{w}, \textit{h})

    \vspace{-1.5ex}

    \rule{\textwidth}{0.5\fboxrule}
\setlength{\parskip}{2ex}

Sets the initial geometry (position and size) of the window to be
opened; the window will not be resizable.

-{}-
\setlength{\parskip}{1ex}
      \textbf{Parameters}
      \vspace{-1ex}

      \begin{quote}
        \begin{Ventry}{x}

          \item[x]


horizontal position (upper-left corner)
            {\it (type=int)}

          \item[y]


vertical position (upper-left corner)
            {\it (type=int)}

          \item[w]


width in coord units
            {\it (type=int)}

          \item[h]


height in coord units
            {\it (type=int)}

        \end{Ventry}

      \end{quote}

\textbf{Note:} 
e.g. fl\_wingeometry(192, 231, 450, 550)


\textbf{Status:} 
Tested + Doc + Demo = OK


    \end{boxedminipage}

    \label{xformslib:flxbasic:fl_wingeometry}
    \index{xformslib \textit{(package)}!xformslib.flxbasic \textit{(module)}!xformslib.flxbasic.fl\_wingeometry \textit{(function)}}

    \vspace{0.5ex}

\hspace{.8\funcindent}\begin{boxedminipage}{\funcwidth}

    \raggedright \textbf{fl\_pref\_wingeometry}(\textit{x}, \textit{y}, \textit{w}, \textit{h})

    \vspace{-1.5ex}

    \rule{\textwidth}{0.5\fboxrule}
\setlength{\parskip}{2ex}

Sets the initial geometry (position and size) of the window to be
opened; the window will not be resizable.

-{}-
\setlength{\parskip}{1ex}
      \textbf{Parameters}
      \vspace{-1ex}

      \begin{quote}
        \begin{Ventry}{x}

          \item[x]


horizontal position (upper-left corner)
            {\it (type=int)}

          \item[y]


vertical position (upper-left corner)
            {\it (type=int)}

          \item[w]


width in coord units
            {\it (type=int)}

          \item[h]


height in coord units
            {\it (type=int)}

        \end{Ventry}

      \end{quote}

\textbf{Note:} 
e.g. fl\_wingeometry(192, 231, 450, 550)


\textbf{Status:} 
Tested + Doc + Demo = OK


    \end{boxedminipage}

    \label{xformslib:flxbasic:fl_initial_wingeometry}
    \index{xformslib \textit{(package)}!xformslib.flxbasic \textit{(module)}!xformslib.flxbasic.fl\_initial\_wingeometry \textit{(function)}}

    \vspace{0.5ex}

\hspace{.8\funcindent}\begin{boxedminipage}{\funcwidth}

    \raggedright \textbf{fl\_initial\_wingeometry}(\textit{x}, \textit{y}, \textit{w}, \textit{h})

    \vspace{-1.5ex}

    \rule{\textwidth}{0.5\fboxrule}
\setlength{\parskip}{2ex}

Sets the initial geometry (position and size) of the window to be
opened.

-{}-
\setlength{\parskip}{1ex}
      \textbf{Parameters}
      \vspace{-1ex}

      \begin{quote}
        \begin{Ventry}{x}

          \item[x]


horizontal position (upper-left corner)
            {\it (type=int)}

          \item[y]


vertical position (upper-left corner)
            {\it (type=int)}

          \item[w]


width in coord units
            {\it (type=int)}

          \item[h]


height in coord units
            {\it (type=int)}

        \end{Ventry}

      \end{quote}

\textbf{Note:} 
e.g. fl\_initial\_wingeometry(192, 231, 450, 550)


\textbf{Status:} 
Tested + Doc + NoDemo = OK


    \end{boxedminipage}

    \label{xformslib:flxbasic:fl_noborder}
    \index{xformslib \textit{(package)}!xformslib.flxbasic \textit{(module)}!xformslib.flxbasic.fl\_noborder \textit{(function)}}

    \vspace{0.5ex}

\hspace{.8\funcindent}\begin{boxedminipage}{\funcwidth}

    \raggedright \textbf{fl\_noborder}()

    \vspace{-1.5ex}

    \rule{\textwidth}{0.5\fboxrule}
\setlength{\parskip}{2ex}

Suppresses the window manager's decoration (before creating the
window).

-{}-
\setlength{\parskip}{1ex}
\textbf{Note:} 
e.g. fl\_noborder()


\textbf{Status:} 
Tested + Doc + NoDemo = OK


    \end{boxedminipage}

    \label{xformslib:flxbasic:fl_transient}
    \index{xformslib \textit{(package)}!xformslib.flxbasic \textit{(module)}!xformslib.flxbasic.fl\_transient \textit{(function)}}

    \vspace{0.5ex}

\hspace{.8\funcindent}\begin{boxedminipage}{\funcwidth}

    \raggedright \textbf{fl\_transient}()

    \vspace{-1.5ex}

    \rule{\textwidth}{0.5\fboxrule}
\setlength{\parskip}{2ex}

Makes a window a transient one (before creating the window).

-{}-
\setlength{\parskip}{1ex}
\textbf{Note:} 
e.g. fl\_transient()


\textbf{Status:} 
Tested + Doc + NoDemo = OK


    \end{boxedminipage}

    \label{xformslib:flxbasic:fl_get_winsize}
    \index{xformslib \textit{(package)}!xformslib.flxbasic \textit{(module)}!xformslib.flxbasic.fl\_get\_winsize \textit{(function)}}

    \vspace{0.5ex}

\hspace{.8\funcindent}\begin{boxedminipage}{\funcwidth}

    \raggedright \textbf{fl\_get\_winsize}(\textit{win})

    \vspace{-1.5ex}

    \rule{\textwidth}{0.5\fboxrule}
\setlength{\parskip}{2ex}

Obtains the size of the specified window.

-{}-
\setlength{\parskip}{1ex}
      \textbf{Parameters}
      \vspace{-1ex}

      \begin{quote}
        \begin{Ventry}{xxx}

          \item[win]


window id to evaluate
            {\it (type=long\_pos)}

        \end{Ventry}

      \end{quote}

      \textbf{Return Value}
    \vspace{-1ex}

      \begin{quote}

width (w), height (h) of window
      {\it (type=int, int)}

      \end{quote}

\textbf{Note:} 
e.g. wid, hei = fl\_get\_winsize(win0)


\textbf{Attention:} 
API change from XForms - upstream was
fl\_get\_winsize(win, w, h)


\textbf{Status:} 
Tested + Doc + NoDemo = OK


    \end{boxedminipage}

    \label{xformslib:flxbasic:fl_get_winorigin}
    \index{xformslib \textit{(package)}!xformslib.flxbasic \textit{(module)}!xformslib.flxbasic.fl\_get\_winorigin \textit{(function)}}

    \vspace{0.5ex}

\hspace{.8\funcindent}\begin{boxedminipage}{\funcwidth}

    \raggedright \textbf{fl\_get\_winorigin}(\textit{win})

    \vspace{-1.5ex}

    \rule{\textwidth}{0.5\fboxrule}
\setlength{\parskip}{2ex}

Obtains the origin (position) of the specified window.

-{}-
\setlength{\parskip}{1ex}
      \textbf{Parameters}
      \vspace{-1ex}

      \begin{quote}
        \begin{Ventry}{xxx}

          \item[win]


window id to evaluate
            {\it (type=long\_pos)}

        \end{Ventry}

      \end{quote}

      \textbf{Return Value}
    \vspace{-1ex}

      \begin{quote}

horizontal (x) and vertical position (y) of window
      {\it (type=int, int)}

      \end{quote}

\textbf{Note:} 
e.g. xpos, ypos = fl\_get\_winorigin(win0)


\textbf{Attention:} 
API change from XForms - upstream was
fl\_get\_winorigin(win, x, y)


\textbf{Status:} 
Tested + Doc + NoDemo = OK


    \end{boxedminipage}

    \label{xformslib:flxbasic:fl_get_wingeometry}
    \index{xformslib \textit{(package)}!xformslib.flxbasic \textit{(module)}!xformslib.flxbasic.fl\_get\_wingeometry \textit{(function)}}

    \vspace{0.5ex}

\hspace{.8\funcindent}\begin{boxedminipage}{\funcwidth}

    \raggedright \textbf{fl\_get\_wingeometry}(\textit{win})

    \vspace{-1.5ex}

    \rule{\textwidth}{0.5\fboxrule}
\setlength{\parskip}{2ex}

Obtains geometry (position and size) of a window.

-{}-
\setlength{\parskip}{1ex}
      \textbf{Parameters}
      \vspace{-1ex}

      \begin{quote}
        \begin{Ventry}{xxx}

          \item[win]


window id to evaluate
            {\it (type=long\_pos)}

        \end{Ventry}

      \end{quote}

      \textbf{Return Value}
    \vspace{-1ex}

      \begin{quote}

horizontal (x), vertical position (y), width (w) and height (h)
of window
      {\it (type=int, int, int, int)}

      \end{quote}

\textbf{Note:} 
e.g. xpos, ypos, wid, hei = fl\_get\_wingeometry(win0)


\textbf{Attention:} 
API change from XForms - upstream was
fl\_get\_wingeometry(win, x, y, w, h)


\textbf{Status:} 
Tested + Doc + NoDemo = OK


    \end{boxedminipage}

    \label{xformslib:flxbasic:fl_get_display}
    \index{xformslib \textit{(package)}!xformslib.flxbasic \textit{(module)}!xformslib.flxbasic.fl\_get\_display \textit{(function)}}

    \vspace{0.5ex}

\hspace{.8\funcindent}\begin{boxedminipage}{\funcwidth}

    \raggedright \textbf{fl\_get\_display}()

\setlength{\parskip}{2ex}
\setlength{\parskip}{1ex}
    \end{boxedminipage}

    \label{xformslib:flxbasic:FL_FormDisplay}
    \index{xformslib \textit{(package)}!xformslib.flxbasic \textit{(module)}!xformslib.flxbasic.FL\_FormDisplay \textit{(function)}}

    \vspace{0.5ex}

\hspace{.8\funcindent}\begin{boxedminipage}{\funcwidth}

    \raggedright \textbf{FL\_FormDisplay}(\textit{pFlForm})

\setlength{\parskip}{2ex}
\setlength{\parskip}{1ex}
    \end{boxedminipage}

    \label{xformslib:flxbasic:FL_ObjectDisplay}
    \index{xformslib \textit{(package)}!xformslib.flxbasic \textit{(module)}!xformslib.flxbasic.FL\_ObjectDisplay \textit{(function)}}

    \vspace{0.5ex}

\hspace{.8\funcindent}\begin{boxedminipage}{\funcwidth}

    \raggedright \textbf{FL\_ObjectDisplay}(\textit{pFlObject})

\setlength{\parskip}{2ex}
\setlength{\parskip}{1ex}
    \end{boxedminipage}

    \label{xformslib:flxbasic:FL_IS_CANVAS}
    \index{xformslib \textit{(package)}!xformslib.flxbasic \textit{(module)}!xformslib.flxbasic.FL\_IS\_CANVAS \textit{(function)}}

    \vspace{0.5ex}

\hspace{.8\funcindent}\begin{boxedminipage}{\funcwidth}

    \raggedright \textbf{FL\_IS\_CANVAS}(\textit{pFlObject})

\setlength{\parskip}{2ex}
\setlength{\parskip}{1ex}
    \end{boxedminipage}

    \label{xformslib:flxbasic:FL_ObjWin}
    \index{xformslib \textit{(package)}!xformslib.flxbasic \textit{(module)}!xformslib.flxbasic.FL\_ObjWin \textit{(function)}}

    \vspace{0.5ex}

\hspace{.8\funcindent}\begin{boxedminipage}{\funcwidth}

    \raggedright \textbf{FL\_ObjWin}(\textit{pFlObject})

    \vspace{-1.5ex}

    \rule{\textwidth}{0.5\fboxrule}
\setlength{\parskip}{2ex}

Obtains the window id an object belongs to (for general use).

-{}-
\setlength{\parskip}{1ex}
      \textbf{Parameters}
      \vspace{-1ex}

      \begin{quote}
        \begin{Ventry}{xxxxxxxxx}

          \item[pFlObject]


object
            {\it (type=pointer to xfdata.FL\_OBJECT)}

        \end{Ventry}

      \end{quote}

      \textbf{Return Value}
    \vspace{-1ex}

      \begin{quote}

window id (win)
      {\it (type=long\_pos)}

      \end{quote}

\textbf{Note:} 
e.g. wind = FL\_ObjWin(pobj)


\textbf{Status:} 
Tested + Doc + Demo = OK


    \end{boxedminipage}

    \label{xformslib:flxbasic:fl_get_real_object_window}
    \index{xformslib \textit{(package)}!xformslib.flxbasic \textit{(module)}!xformslib.flxbasic.fl\_get\_real\_object\_window \textit{(function)}}

    \vspace{0.5ex}

\hspace{.8\funcindent}\begin{boxedminipage}{\funcwidth}

    \raggedright \textbf{fl\_get\_real\_object\_window}(\textit{pFlObject})

    \vspace{-1.5ex}

    \rule{\textwidth}{0.5\fboxrule}
\setlength{\parskip}{2ex}

Obtains the real window id an object belongs to (to be used for
cursor or pointer routines).

-{}-
\setlength{\parskip}{1ex}
      \textbf{Parameters}
      \vspace{-1ex}

      \begin{quote}
        \begin{Ventry}{xxxxxxxxx}

          \item[pFlObject]


object
            {\it (type=pointer to xfdata.FL\_OBJECT)}

        \end{Ventry}

      \end{quote}

      \textbf{Return Value}
    \vspace{-1ex}

      \begin{quote}

window id (win)
      {\it (type=long\_pos)}

      \end{quote}

\textbf{Note:} 
e.g. wind = fl\_get\_real\_object\_window(pobj)


\textbf{Status:} 
Tested + Doc + NoDemo = OK


    \end{boxedminipage}

    \label{xformslib:flxbasic:FL_ObjWin}
    \index{xformslib \textit{(package)}!xformslib.flxbasic \textit{(module)}!xformslib.flxbasic.FL\_ObjWin \textit{(function)}}

    \vspace{0.5ex}

\hspace{.8\funcindent}\begin{boxedminipage}{\funcwidth}

    \raggedright \textbf{FL\_OBJECT\_WID}(\textit{pFlObject})

    \vspace{-1.5ex}

    \rule{\textwidth}{0.5\fboxrule}
\setlength{\parskip}{2ex}

Obtains the window id an object belongs to (for general use).

-{}-
\setlength{\parskip}{1ex}
      \textbf{Parameters}
      \vspace{-1ex}

      \begin{quote}
        \begin{Ventry}{xxxxxxxxx}

          \item[pFlObject]


object
            {\it (type=pointer to xfdata.FL\_OBJECT)}

        \end{Ventry}

      \end{quote}

      \textbf{Return Value}
    \vspace{-1ex}

      \begin{quote}

window id (win)
      {\it (type=long\_pos)}

      \end{quote}

\textbf{Note:} 
e.g. wind = FL\_ObjWin(pobj)


\textbf{Status:} 
Tested + Doc + Demo = OK


    \end{boxedminipage}

    \label{xformslib:flxbasic:fl_XNextEvent}
    \index{xformslib \textit{(package)}!xformslib.flxbasic \textit{(module)}!xformslib.flxbasic.fl\_XNextEvent \textit{(function)}}

    \vspace{0.5ex}

\hspace{.8\funcindent}\begin{boxedminipage}{\funcwidth}

    \raggedright \textbf{fl\_XNextEvent}(\textit{pXEvent})

    \vspace{-1.5ex}

    \rule{\textwidth}{0.5\fboxrule}
\setlength{\parskip}{2ex}

X11 XNextEvent equivalent function.

-{}-
\setlength{\parskip}{1ex}
      \textbf{Parameters}
      \vspace{-1ex}

      \begin{quote}
        \begin{Ventry}{xxxxxxx}

          \item[pXEvent]


XEvent class instance
            {\it (type=pointer to xfdata.XEvent)}

        \end{Ventry}

      \end{quote}

      \textbf{Return Value}
    \vspace{-1ex}

      \begin{quote}

event num.
      {\it (type=int)}

      \end{quote}

\textbf{Note:} 
e.g. \emph{todo}


\textbf{Status:} 
Untested + NoDoc + NoDemo = NOT OK


    \end{boxedminipage}

    \label{xformslib:flxbasic:fl_XPeekEvent}
    \index{xformslib \textit{(package)}!xformslib.flxbasic \textit{(module)}!xformslib.flxbasic.fl\_XPeekEvent \textit{(function)}}

    \vspace{0.5ex}

\hspace{.8\funcindent}\begin{boxedminipage}{\funcwidth}

    \raggedright \textbf{fl\_XPeekEvent}(\textit{pXEvent})

    \vspace{-1.5ex}

    \rule{\textwidth}{0.5\fboxrule}
\setlength{\parskip}{2ex}

X11 XPeekEvent equivalent function.

-{}-
\setlength{\parskip}{1ex}
      \textbf{Parameters}
      \vspace{-1ex}

      \begin{quote}
        \begin{Ventry}{xxxxxxx}

          \item[pXEvent]


XEvent class instance
            {\it (type=pointer to xfdata.XEvent)}

        \end{Ventry}

      \end{quote}

      \textbf{Return Value}
    \vspace{-1ex}

      \begin{quote}

event num.
      {\it (type=int)}

      \end{quote}

\textbf{Note:} 
e.g. \emph{todo}


\textbf{Status:} 
Untested + NoDoc + NoDemo = NOT OK


    \end{boxedminipage}

    \label{xformslib:flxbasic:fl_XEventsQueued}
    \index{xformslib \textit{(package)}!xformslib.flxbasic \textit{(module)}!xformslib.flxbasic.fl\_XEventsQueued \textit{(function)}}

    \vspace{0.5ex}

\hspace{.8\funcindent}\begin{boxedminipage}{\funcwidth}

    \raggedright \textbf{fl\_XEventsQueued}(\textit{mode})

    \vspace{-1.5ex}

    \rule{\textwidth}{0.5\fboxrule}
\setlength{\parskip}{2ex}

X11 XEventsQueued equivalent function.

-{}-
\setlength{\parskip}{1ex}
      \textbf{Parameters}
      \vspace{-1ex}

      \begin{quote}
        \begin{Ventry}{xxxx}

          \item[mode]


mode
            {\it (type=int)}

        \end{Ventry}

      \end{quote}

      \textbf{Return Value}
    \vspace{-1ex}

      \begin{quote}

event num.
      {\it (type=int)}

      \end{quote}

\textbf{Note:} 
e.g. \emph{todo}


\textbf{Status:} 
Untested + NoDoc + NoDemo = NOT OK


    \end{boxedminipage}

    \label{xformslib:flxbasic:fl_XPutBackEvent}
    \index{xformslib \textit{(package)}!xformslib.flxbasic \textit{(module)}!xformslib.flxbasic.fl\_XPutBackEvent \textit{(function)}}

    \vspace{0.5ex}

\hspace{.8\funcindent}\begin{boxedminipage}{\funcwidth}

    \raggedright \textbf{fl\_XPutBackEvent}(\textit{pXEvent})

    \vspace{-1.5ex}

    \rule{\textwidth}{0.5\fboxrule}
\setlength{\parskip}{2ex}

X11 XPutBackEvent equivalent function.

-{}-
\setlength{\parskip}{1ex}
      \textbf{Parameters}
      \vspace{-1ex}

      \begin{quote}
        \begin{Ventry}{xxxxxxx}

          \item[pXEvent]


XEvent class instance
            {\it (type=pointer to xfdata.XEvent)}

        \end{Ventry}

      \end{quote}

\textbf{Note:} 
e.g. \emph{todo}


\textbf{Status:} 
Untested + NoDoc + NoDemo = NOT OK


    \end{boxedminipage}

    \label{xformslib:flxbasic:fl_last_event}
    \index{xformslib \textit{(package)}!xformslib.flxbasic \textit{(module)}!xformslib.flxbasic.fl\_last\_event \textit{(function)}}

    \vspace{0.5ex}

\hspace{.8\funcindent}\begin{boxedminipage}{\funcwidth}

    \raggedright \textbf{fl\_last\_event}()

    \vspace{-1.5ex}

    \rule{\textwidth}{0.5\fboxrule}
\setlength{\parskip}{2ex}

Obtains the last X event. If this routine is used outside of a callback
function, the value returned may not be the real ``last event'' if the
program was idling and, in this case, it returns a synthetic
xfdata.MotionNotify event.

-{}-
\setlength{\parskip}{1ex}
      \textbf{Return Value}
    \vspace{-1ex}

      \begin{quote}

XEvent class instance (pXEvent)
      {\it (type=pointer to xfdata.XEvent)}

      \end{quote}

\textbf{Note:} 
e.g. pxev = flk\_last\_event()


\textbf{Status:} 
Untested + NoDoc + NoDemo = NOT OK


    \end{boxedminipage}

    \label{xformslib:flxbasic:fl_set_event_callback}
    \index{xformslib \textit{(package)}!xformslib.flxbasic \textit{(module)}!xformslib.flxbasic.fl\_set\_event\_callback \textit{(function)}}

    \vspace{0.5ex}

\hspace{.8\funcindent}\begin{boxedminipage}{\funcwidth}

    \raggedright \textbf{fl\_set\_event\_callback}(\textit{py\_AppEventCb}, \textit{vdata})

    \vspace{-1.5ex}

    \rule{\textwidth}{0.5\fboxrule}
\setlength{\parskip}{2ex}

Sets up an event callback routine. Whenever an event happens, the
callback function is invoked with the event as the first argument.
This assumes the application program solicits the events and further,
the callback routine should be prepared to handle all X Event for all
non-form windows. The callback function normally should return 0
unless the event isn't for one of the applcation-managed windows.
This routine will be called whenever an X Event is pending for the
application's own window.

-{}-
\setlength{\parskip}{1ex}
      \textbf{Parameters}
      \vspace{-1ex}

      \begin{quote}
        \begin{Ventry}{xxxxxxxxxxxxx}

          \item[py\_AppEventCb]


name referring to function(pXEvent, vdata) -> num.
            {\it (type=python function callback, returning value)}

          \item[vdata]


user data to be passed to function; callback has to take care of
type check
            {\it (type=any type (e.g. 'None', int, str, etc..))}

        \end{Ventry}

      \end{quote}

      \textbf{Return Value}
    \vspace{-1ex}

      \begin{quote}

old event callback
      {\it (type=xfdata.FL\_APPEVENT\_CB)}

      \end{quote}

\textbf{Notes:}
\begin{quote}
  \begin{itemize}

  \item
    \setlength{\parskip}{0.6ex}

e.g. def eventcb(pxev, vdata): > ... ; return 0


  \item 
e.g. fl\_set\_event\_callback(eventcb, None)


\end{itemize}

\end{quote}

\textbf{Status:} 
Tested + Doc + Demo = OK


    \end{boxedminipage}

    \label{xformslib:flxbasic:fl_set_idle_callback}
    \index{xformslib \textit{(package)}!xformslib.flxbasic \textit{(module)}!xformslib.flxbasic.fl\_set\_idle\_callback \textit{(function)}}

    \vspace{0.5ex}

\hspace{.8\funcindent}\begin{boxedminipage}{\funcwidth}

    \raggedright \textbf{fl\_set\_idle\_callback}(\textit{py\_AppEventCb}, \textit{vdata})

    \vspace{-1.5ex}

    \rule{\textwidth}{0.5\fboxrule}
\setlength{\parskip}{2ex}

Registers an idle callback. Interaction with it  can used for periodic
tasks, e.g. rotating an image, checking the status of some external
device or application state etc. An idle callback is an application
function that is registered with the system and is called whenever there
are no events pending for forms (or application windows). If called with
a function as callback who does nothing, it removes idle callback.
The time interval between invocations of the idle callback can vary
considerably depending on interface activity and other factors. A
range between 50 and 300 msec should be expected.

-{}-
\setlength{\parskip}{1ex}
      \textbf{Parameters}
      \vspace{-1ex}

      \begin{quote}
        \begin{Ventry}{xxxxxxxxxxxxx}

          \item[py\_AppEventCb]


name referring to function(pXEvent, vdata) -> num.
            {\it (type=python function callback, returning unused value)}

          \item[vdata]


user data to be passed to function; callback has to take care of
type check
            {\it (type=any type (e.g. 'None', int, str, etc..))}

        \end{Ventry}

      \end{quote}

      \textbf{Return Value}
    \vspace{-1ex}

      \begin{quote}

old event callback function
      {\it (type=xfdata.FL\_APPEVENT\_CB)}

      \end{quote}

\textbf{Notes:}
\begin{quote}
  \begin{itemize}

  \item
    \setlength{\parskip}{0.6ex}

e.g. def idlecb(xev, userdata): > ... ; return 0


  \item 
e.g. appevtcb = fl\_set\_idle\_callback(idlecb, None)


  \item 
e.g. def donothing\_idlecb(xev, userdata): > pass


  \item 
e.g. removedcb = fl\_set\_idle\_callback(donothing\_idlecb, None)


\end{itemize}

\end{quote}

\textbf{Status:} 
Tested + Doc + NoDemo = OK


    \end{boxedminipage}

    \label{xformslib:flxbasic:fl_addto_selected_xevent}
    \index{xformslib \textit{(package)}!xformslib.flxbasic \textit{(module)}!xformslib.flxbasic.fl\_addto\_selected\_xevent \textit{(function)}}

    \vspace{0.5ex}

\hspace{.8\funcindent}\begin{boxedminipage}{\funcwidth}

    \raggedright \textbf{fl\_addto\_selected\_xevent}(\textit{win}, \textit{mask})

    \vspace{-1.5ex}

    \rule{\textwidth}{0.5\fboxrule}
\setlength{\parskip}{2ex}

Adds solicited event masks on the fly without altering other masks
already selected.

-{}-
\setlength{\parskip}{1ex}
      \textbf{Parameters}
      \vspace{-1ex}

      \begin{quote}
        \begin{Ventry}{xxxx}

          \item[win]


window id
            {\it (type=long\_pos)}

          \item[mask]


event mask
            {\it (type=long)}

        \end{Ventry}

      \end{quote}

      \textbf{Return Value}
    \vspace{-1ex}

      \begin{quote}

num.
      {\it (type=long)}

      \end{quote}

\textbf{Note:} 
e.g. lnum = fl\_addto\_selected\_xevent(win7,         xfdata.ButtonMotionMask)


\textbf{Status:} 
Tested + Doc + NoDemo = OK


    \end{boxedminipage}

    \label{xformslib:flxbasic:fl_remove_selected_xevent}
    \index{xformslib \textit{(package)}!xformslib.flxbasic \textit{(module)}!xformslib.flxbasic.fl\_remove\_selected\_xevent \textit{(function)}}

    \vspace{0.5ex}

\hspace{.8\funcindent}\begin{boxedminipage}{\funcwidth}

    \raggedright \textbf{fl\_remove\_selected\_xevent}(\textit{win}, \textit{mask})

    \vspace{-1.5ex}

    \rule{\textwidth}{0.5\fboxrule}
\setlength{\parskip}{2ex}

Removes solicited event masks on the fly without altering other masks
already selected.

-{}-
\setlength{\parskip}{1ex}
      \textbf{Parameters}
      \vspace{-1ex}

      \begin{quote}
        \begin{Ventry}{xxxx}

          \item[win]


window id
            {\it (type=long\_pos)}

          \item[mask]


event mask
            {\it (type=long)}

        \end{Ventry}

      \end{quote}

      \textbf{Return Value}
    \vspace{-1ex}

      \begin{quote}

num.
      {\it (type=long)}

      \end{quote}

\textbf{Note:} 
e.g. lnum = fl\_remove\_selected\_xevent(win7,         xfdata.ButtonMotionMask)


\textbf{Status:} 
Tested + Doc + NoDemo = OK


    \end{boxedminipage}

    \label{xformslib:flxbasic:fl_addto_selected_xevent}
    \index{xformslib \textit{(package)}!xformslib.flxbasic \textit{(module)}!xformslib.flxbasic.fl\_addto\_selected\_xevent \textit{(function)}}

    \vspace{0.5ex}

\hspace{.8\funcindent}\begin{boxedminipage}{\funcwidth}

    \raggedright \textbf{fl\_add\_selected\_xevent}(\textit{win}, \textit{mask})

    \vspace{-1.5ex}

    \rule{\textwidth}{0.5\fboxrule}
\setlength{\parskip}{2ex}

Adds solicited event masks on the fly without altering other masks
already selected.

-{}-
\setlength{\parskip}{1ex}
      \textbf{Parameters}
      \vspace{-1ex}

      \begin{quote}
        \begin{Ventry}{xxxx}

          \item[win]


window id
            {\it (type=long\_pos)}

          \item[mask]


event mask
            {\it (type=long)}

        \end{Ventry}

      \end{quote}

      \textbf{Return Value}
    \vspace{-1ex}

      \begin{quote}

num.
      {\it (type=long)}

      \end{quote}

\textbf{Note:} 
e.g. lnum = fl\_addto\_selected\_xevent(win7,         xfdata.ButtonMotionMask)


\textbf{Status:} 
Tested + Doc + NoDemo = OK


    \end{boxedminipage}

    \label{xformslib:flxbasic:fl_set_idle_delta}
    \index{xformslib \textit{(package)}!xformslib.flxbasic \textit{(module)}!xformslib.flxbasic.fl\_set\_idle\_delta \textit{(function)}}

    \vspace{0.5ex}

\hspace{.8\funcindent}\begin{boxedminipage}{\funcwidth}

    \raggedright \textbf{fl\_set\_idle\_delta}(\textit{msec})

    \vspace{-1.5ex}

    \rule{\textwidth}{0.5\fboxrule}
\setlength{\parskip}{2ex}

Changes what the library considers to be ``idle''. Be aware that under
some conditions ad idle callback can be called sooner than the minimum
interval; if the timing of the idle callback is of concerned, timeouts
should be used.

-{}-
\setlength{\parskip}{1ex}
      \textbf{Parameters}
      \vspace{-1ex}

      \begin{quote}
        \begin{Ventry}{xxxx}

          \item[msec]


minimum time interval of inactivity, after which the main loop is
considered to be in idle state
            {\it (type=long)}

        \end{Ventry}

      \end{quote}

\textbf{Note:} 
e.g. fl\_set\_idle\_delta(800)


\textbf{Status:} 
Tested + Doc + NoDemo = OK


    \end{boxedminipage}

    \label{xformslib:flxbasic:fl_add_event_callback}
    \index{xformslib \textit{(package)}!xformslib.flxbasic \textit{(module)}!xformslib.flxbasic.fl\_add\_event\_callback \textit{(function)}}

    \vspace{0.5ex}

\hspace{.8\funcindent}\begin{boxedminipage}{\funcwidth}

    \raggedright \textbf{fl\_add\_event\_callback}(\textit{win}, \textit{evttype}, \textit{py\_AppEventCb}, \textit{vdata})

    \vspace{-1.5ex}

    \rule{\textwidth}{0.5\fboxrule}
\setlength{\parskip}{2ex}

Adds an event handler for a window. Manipulates the event callback
functions for the window specified, which will be called when an
event of specified type is pending for the window. It does not
solicit any event for the caller, i.e. the XForms library assumes
the caller opens the window and solicits all events before calling
these routines.

-{}-
\setlength{\parskip}{1ex}
      \textbf{Parameters}
      \vspace{-1ex}

      \begin{quote}
        \begin{Ventry}{xxxxxxxxxxxxx}

          \item[win]


window id to add event handler to
            {\it (type=long\_pos)}

          \item[evttype]


event type number. If it's 0, the callback is for all events for the
window
            {\it (type=int)}

          \item[py\_AppEventCb]


name referring function(pXEvent, vdata) -> num.
            {\it (type=python function callback, returning value)}

          \item[vdata]


user data to be passed to function; callback has to take care of
type check
            {\it (type=any type (e.g. 'None', int, str, etc..))}

        \end{Ventry}

      \end{quote}

      \textbf{Return Value}
    \vspace{-1ex}

      \begin{quote}

old event callback
      {\it (type=pointer to xfdata.FL\_APPEVENT\_CB)}

      \end{quote}

\textbf{Notes:}
\begin{quote}
  \begin{itemize}

  \item
    \setlength{\parskip}{0.6ex}

e.g. def eventcb(pxev, vdata): > ... ; return 0


  \item 
e.g. fl\_add\_event\_callback(win2, 0, eventcb, None)


\end{itemize}

\end{quote}

\textbf{Status:} 
Tested + Doc + NoDemo = OK


    \end{boxedminipage}

    \label{xformslib:flxbasic:fl_remove_event_callback}
    \index{xformslib \textit{(package)}!xformslib.flxbasic \textit{(module)}!xformslib.flxbasic.fl\_remove\_event\_callback \textit{(function)}}

    \vspace{0.5ex}

\hspace{.8\funcindent}\begin{boxedminipage}{\funcwidth}

    \raggedright \textbf{fl\_remove\_event\_callback}(\textit{win}, \textit{evttype})

    \vspace{-1.5ex}

    \rule{\textwidth}{0.5\fboxrule}
\setlength{\parskip}{2ex}

Removes one or all event callbacks for a window and for an event of
specified type. May be called with for a window for which no event
callbacks have been set.

-{}-
\setlength{\parskip}{1ex}
      \textbf{Parameters}
      \vspace{-1ex}

      \begin{quote}
        \begin{Ventry}{xxxxxxx}

          \item[win]


window id
            {\it (type=long\_pos)}

          \item[evttype]


event type number
            {\it (type=int)}

        \end{Ventry}

      \end{quote}

\textbf{Note:} 
e.g. fl\_remove\_event\_callback(win2, 0)


\textbf{Status:} 
Tested + Doc + NoDemo = OK


    \end{boxedminipage}

    \label{xformslib:flxbasic:fl_activate_event_callbacks}
    \index{xformslib \textit{(package)}!xformslib.flxbasic \textit{(module)}!xformslib.flxbasic.fl\_activate\_event\_callbacks \textit{(function)}}

    \vspace{0.5ex}

\hspace{.8\funcindent}\begin{boxedminipage}{\funcwidth}

    \raggedright \textbf{fl\_activate\_event\_callbacks}(\textit{win})

    \vspace{-1.5ex}

    \rule{\textwidth}{0.5\fboxrule}
\setlength{\parskip}{2ex}

Handles event solicitation. Activates the default mapping of events
to event masks built-in in the XForms Library, and causes the system
to solicit the events for you. Note however, the mapping of events to
masks are not unique and depending on applications, the default mapping
may or may not be the one you want.

-{}-
\setlength{\parskip}{1ex}
      \textbf{Parameters}
      \vspace{-1ex}

      \begin{quote}
        \begin{Ventry}{xxx}

          \item[win]


window whose events are referred to
            {\it (type=long\_pos)}

        \end{Ventry}

      \end{quote}

\textbf{Note:} 
e.g. fl\_activate\_event\_callback(win3)


\textbf{Status:} 
Tested + Doc + NoDemo = OK


    \end{boxedminipage}

    \label{xformslib:flxbasic:fl_print_xevent_name}
    \index{xformslib \textit{(package)}!xformslib.flxbasic \textit{(module)}!xformslib.flxbasic.fl\_print\_xevent\_name \textit{(function)}}

    \vspace{0.5ex}

\hspace{.8\funcindent}\begin{boxedminipage}{\funcwidth}

    \raggedright \textbf{fl\_print\_xevent\_name}(\textit{where}, \textit{pXEvent})

    \vspace{-1.5ex}

    \rule{\textwidth}{0.5\fboxrule}
\setlength{\parskip}{2ex}

Print the name of an XEvent and some other infos.

-{}-
\setlength{\parskip}{1ex}
      \textbf{Parameters}
      \vspace{-1ex}

      \begin{quote}
        \begin{Ventry}{xxxxxxx}

          \item[where]


text, it can indicate where this function is called.
            {\it (type=str)}

          \item[pXEvent]


XEvent class instance
            {\it (type=pointer to xfdata.XEvent)}

        \end{Ventry}

      \end{quote}

      \textbf{Return Value}
    \vspace{-1ex}

      \begin{quote}

event (pXEvent)
      {\it (type=pointer to xfdata.XEvent)}

      \end{quote}

\textbf{Note:} 
e.g. pxev = fl\_print\_xevent\_name(``from whatever.py'', pxev)


\textbf{Status:} 
Tested + Doc + NoDemo = OK


    \end{boxedminipage}

    \label{xformslib:flxbasic:fl_XFlush}
    \index{xformslib \textit{(package)}!xformslib.flxbasic \textit{(module)}!xformslib.flxbasic.fl\_XFlush \textit{(function)}}

    \vspace{0.5ex}

\hspace{.8\funcindent}\begin{boxedminipage}{\funcwidth}

    \raggedright \textbf{fl\_XFlush}()

    \vspace{-1.5ex}

    \rule{\textwidth}{0.5\fboxrule}
\setlength{\parskip}{2ex}

Flushes the output buffer. Convenience replacement for X11 XFlush()
function.

-{}-
\setlength{\parskip}{1ex}
\textbf{Note:} 
e.g. fl\_XFlush()


\textbf{Status:} 
Tested + Doc + Demo = OK


    \end{boxedminipage}

    \label{xformslib:flxbasic:metakey_down}
    \index{xformslib \textit{(package)}!xformslib.flxbasic \textit{(module)}!xformslib.flxbasic.metakey\_down \textit{(function)}}

    \vspace{0.5ex}

\hspace{.8\funcindent}\begin{boxedminipage}{\funcwidth}

    \raggedright \textbf{metakey\_down}(\textit{mask})

\setlength{\parskip}{2ex}
\setlength{\parskip}{1ex}
    \end{boxedminipage}

    \label{xformslib:flxbasic:shiftkey_down}
    \index{xformslib \textit{(package)}!xformslib.flxbasic \textit{(module)}!xformslib.flxbasic.shiftkey\_down \textit{(function)}}

    \vspace{0.5ex}

\hspace{.8\funcindent}\begin{boxedminipage}{\funcwidth}

    \raggedright \textbf{shiftkey\_down}(\textit{mask})

\setlength{\parskip}{2ex}
\setlength{\parskip}{1ex}
    \end{boxedminipage}

    \label{xformslib:flxbasic:controlkey_down}
    \index{xformslib \textit{(package)}!xformslib.flxbasic \textit{(module)}!xformslib.flxbasic.controlkey\_down \textit{(function)}}

    \vspace{0.5ex}

\hspace{.8\funcindent}\begin{boxedminipage}{\funcwidth}

    \raggedright \textbf{controlkey\_down}(\textit{mask})

\setlength{\parskip}{2ex}
\setlength{\parskip}{1ex}
    \end{boxedminipage}

    \label{xformslib:flxbasic:button_down}
    \index{xformslib \textit{(package)}!xformslib.flxbasic \textit{(module)}!xformslib.flxbasic.button\_down \textit{(function)}}

    \vspace{0.5ex}

\hspace{.8\funcindent}\begin{boxedminipage}{\funcwidth}

    \raggedright \textbf{button\_down}(\textit{mask})

\setlength{\parskip}{2ex}
\setlength{\parskip}{1ex}
    \end{boxedminipage}

    \label{xformslib:flxbasic:fl_initialize}
    \index{xformslib \textit{(package)}!xformslib.flxbasic \textit{(module)}!xformslib.flxbasic.fl\_initialize \textit{(function)}}

    \vspace{0.5ex}

\hspace{.8\funcindent}\begin{boxedminipage}{\funcwidth}

    \raggedright \textbf{fl\_initialize}(\textit{numargs}, \textit{argslist}, \textit{appname}={\tt ""}, \textit{appoptions}={\tt 0}, \textit{nappopts}={\tt 0})

    \vspace{-1.5ex}

    \rule{\textwidth}{0.5\fboxrule}
\setlength{\parskip}{2ex}

Initializes XForms libr. It should always be called before any
other calls to the XForms Library (except fl\_set\_defaults() and a few
other functions that alter some of the defaults of the library.
Command line arguments are NOT supported here, but you can always set
most of parameters with relative functions.

-{}-
\setlength{\parskip}{1ex}
      \textbf{Parameters}
      \vspace{-1ex}

      \begin{quote}
        \begin{Ventry}{xxxxxxxxxx}

          \item[numargs]


number of arguments passed to command line, unused in python
            {\it (type=int)}

          \item[argslist]


arguments passed to command line, unused in python
            {\it (type=list\_of\_str)}

          \item[appname]


application class name
            {\it (type=str)}

          \item[appoptions]


options passed as a flcmdopt class instance
            {\it (type=instance of xfdata.FL\_CMD\_OPT)}

          \item[nappopts]


number of options
            {\it (type=int)}

        \end{Ventry}

      \end{quote}

      \textbf{Return Value}
    \vspace{-1ex}

      \begin{quote}

display (pDisplay) or None (on failure, if a connection
couldn't be made)
      {\it (type=pointer to xfdata.Display)}

      \end{quote}

\textbf{Notes:}
\begin{quote}
  \begin{itemize}

  \item
    \setlength{\parskip}{0.6ex}

e.g. import sys


  \item 
e.g. fl\_initialize(len(sys.argv), sys.argv, ``MyFormDemo'', 0, 0)


\end{itemize}

\end{quote}

\textbf{Status:} 
HalfTested + Doc + Demo = HALF OK (not for command line args)


    \end{boxedminipage}

    \label{xformslib:flxbasic:fl_finish}
    \index{xformslib \textit{(package)}!xformslib.flxbasic \textit{(module)}!xformslib.flxbasic.fl\_finish \textit{(function)}}

    \vspace{0.5ex}

\hspace{.8\funcindent}\begin{boxedminipage}{\funcwidth}

    \raggedright \textbf{fl\_finish}()

    \vspace{-1.5ex}

    \rule{\textwidth}{0.5\fboxrule}
\setlength{\parskip}{2ex}

It is a final cleanup routine, restores all X server defaults, shuts
down the connection and frees dynamically allocated memory.

-{}-
\setlength{\parskip}{1ex}
\textbf{Note:} 
e.g. fl\_finish()


\textbf{Status:} 
Tested + Doc + Demo = OK


    \end{boxedminipage}

    \label{xformslib:flxbasic:fl_get_resource}
    \index{xformslib \textit{(package)}!xformslib.flxbasic \textit{(module)}!xformslib.flxbasic.fl\_get\_resource \textit{(function)}}

    \vspace{0.5ex}

\hspace{.8\funcindent}\begin{boxedminipage}{\funcwidth}

    \raggedright \textbf{fl\_get\_resource}(\textit{resname}, \textit{resclass}, \textit{dtype}, \textit{defval}, \textit{val}, \textit{size})

    \vspace{-1.5ex}

    \rule{\textwidth}{0.5\fboxrule}
\setlength{\parskip}{2ex}

Obtains resource data at the lowest level. It may be useful to e.g.
retrieve arbitrary strings and values and to pass data around.

-{}-
\setlength{\parskip}{1ex}
      \textbf{Parameters}
      \vspace{-1ex}

      \begin{quote}
        \begin{Ventry}{xxxxxxxx}

          \item[resname]


complete resource name specification (minus the application name)
and should not contain wildcards of any kind
            {\it (type=str)}

          \item[resclass]


complete resource class specification (minus the application name)
and should not contain wildcards of any kind
            {\it (type=str)}

          \item[dtype]


type of resource. Values (from xfdata.py) FL\_NONE, FL\_SHORT, FL\_BOOL,
FL\_INT, FL\_LONG, FL\_FLOAT, FL\_STRING
            {\it (type=int)}

          \item[defval]


default value
            {\it (type=str)}

          \item[size]


number of bytes, used only if dtype is FL\_STRING
            {\it (type=int)}

        \end{Ventry}

      \end{quote}

      \textbf{Return Value}
    \vspace{-1ex}

      \begin{quote}

text representation of the resource value or None (on failure),
variable value (val)
      {\it (type=str, any type)}

      \end{quote}

\textbf{Note:} 
e.g. \emph{todo}


\textbf{Attention:} 
API change from XForms - upstream was
fl\_get\_resource(resname, resclass, dtype, defval, val, size)


\textbf{Status:} 
Untested + NoDoc + NoDemo = NOT OK


    \end{boxedminipage}

    \label{xformslib:flxbasic:fl_set_resource}
    \index{xformslib \textit{(package)}!xformslib.flxbasic \textit{(module)}!xformslib.flxbasic.fl\_set\_resource \textit{(function)}}

    \vspace{0.5ex}

\hspace{.8\funcindent}\begin{boxedminipage}{\funcwidth}

    \raggedright \textbf{fl\_set\_resource}(\textit{resnamecls}, \textit{txtval})

    \vspace{-1.5ex}

    \rule{\textwidth}{0.5\fboxrule}
\setlength{\parskip}{2ex}

Sets a resource, associating a value to it. It may be useful to e.g.
change a built-in button labels with proper resource names, or to store
arbitrary strings and values and to pass data around.

-{}-
\setlength{\parskip}{1ex}
      \textbf{Parameters}
      \vspace{-1ex}

      \begin{quote}
        \begin{Ventry}{xxxxxxxxxx}

          \item[resnamecls]


a fully qualified resource name (minus the application name) or a
resource class
            {\it (type=str)}

          \item[txtval]


new text value for resource
            {\it (type=str)}

        \end{Ventry}

      \end{quote}

\textbf{Note:} 
e.g. \emph{todo}


\textbf{Status:} 
Tested + NoDoc + Demo = OK


    \end{boxedminipage}

    \label{xformslib:flxbasic:fl_get_app_resources}
    \index{xformslib \textit{(package)}!xformslib.flxbasic \textit{(module)}!xformslib.flxbasic.fl\_get\_app\_resources \textit{(function)}}

    \vspace{0.5ex}

\hspace{.8\funcindent}\begin{boxedminipage}{\funcwidth}

    \raggedright \textbf{fl\_get\_app\_resources}(\textit{pResource}, \textit{nresources})

    \vspace{-1.5ex}

    \rule{\textwidth}{0.5\fboxrule}
\setlength{\parskip}{2ex}

\emph{todo}

-{}-
\setlength{\parskip}{1ex}
      \textbf{Parameters}
      \vspace{-1ex}

      \begin{quote}
        \begin{Ventry}{xxxxxxxxxx}

          \item[pResource]


an array of resource class instances
            {\it (type=pointer to xfdata.FL\_RESOURCE)}

          \item[nresources]


number of resources (starting from 1) passed with pResource array
            {\it (type=int)}

        \end{Ventry}

      \end{quote}

\textbf{Note:} 
e.g. \emph{todo}


\textbf{Status:} 
Untested + NoDoc + NoDemo = NOT OK


    \end{boxedminipage}

    \label{xformslib:flxbasic:fl_set_graphics_mode}
    \index{xformslib \textit{(package)}!xformslib.flxbasic \textit{(module)}!xformslib.flxbasic.fl\_set\_graphics\_mode \textit{(function)}}

    \vspace{0.5ex}

\hspace{.8\funcindent}\begin{boxedminipage}{\funcwidth}

    \raggedright \textbf{fl\_set\_graphics\_mode}(\textit{mode}, \textit{doublebuf})

    \vspace{-1.5ex}

    \rule{\textwidth}{0.5\fboxrule}
\setlength{\parskip}{2ex}

\emph{todo}

-{}-
\setlength{\parskip}{1ex}
      \textbf{Parameters}
      \vspace{-1ex}

      \begin{quote}
        \begin{Ventry}{xxxxxxxxx}

          \item[mode]


graphics mode to be set
            {\it (type=int)}

          \item[doublebuf]


\emph{todo}
            {\it (type=int)}

        \end{Ventry}

      \end{quote}

      \textbf{Return Value}
    \vspace{-1ex}

      \begin{quote}

num.
      {\it (type=int)}

      \end{quote}

\textbf{Note:} 
e.g. \emph{todo}


\textbf{Status:} 
Untested + NoDoc + NoDemo = NOT OK


    \end{boxedminipage}

    \label{xformslib:flxbasic:fl_set_visualID}
    \index{xformslib \textit{(package)}!xformslib.flxbasic \textit{(module)}!xformslib.flxbasic.fl\_set\_visualID \textit{(function)}}

    \vspace{0.5ex}

\hspace{.8\funcindent}\begin{boxedminipage}{\funcwidth}

    \raggedright \textbf{fl\_set\_visualID}(\textit{idnum})

    \vspace{-1.5ex}

    \rule{\textwidth}{0.5\fboxrule}
\setlength{\parskip}{2ex}

Sets visual and depth. By default, X Server's visual and depth values
are used.

-{}-
\setlength{\parskip}{1ex}
      \textbf{Parameters}
      \vspace{-1ex}

      \begin{quote}
        \begin{Ventry}{xxxxx}

          \item[idnum]


visual id. Values (from xfdata.py) TrueColor, PseudoColor, etc..
            {\it (type=int)}

        \end{Ventry}

      \end{quote}

\textbf{Note:} 
e.g. \emph{todo}


\textbf{Precondition:} 
to be called before fl\_initialize()


\textbf{Status:} 
Untested + NoDoc + NoDemo = NOT OK


    \end{boxedminipage}

    \label{xformslib:flxbasic:fl_keysym_pressed}
    \index{xformslib \textit{(package)}!xformslib.flxbasic \textit{(module)}!xformslib.flxbasic.fl\_keysym\_pressed \textit{(function)}}

    \vspace{0.5ex}

\hspace{.8\funcindent}\begin{boxedminipage}{\funcwidth}

    \raggedright \textbf{fl\_keysym\_pressed}(\textit{keysym})

    \vspace{-1.5ex}

    \rule{\textwidth}{0.5\fboxrule}
\setlength{\parskip}{2ex}

\emph{todo}

-{}-
\setlength{\parskip}{1ex}
      \textbf{Parameters}
      \vspace{-1ex}

      \begin{quote}
        \begin{Ventry}{xxxxxx}

          \item[keysym]


\emph{todo}
            {\it (type=long\_pos)}

        \end{Ventry}

      \end{quote}

      \textbf{Return Value}
    \vspace{-1ex}

      \begin{quote}

num., or 0 (on failure)
      {\it (type=int)}

      \end{quote}

\textbf{Note:} 
e.g. \emph{todo}


\textbf{Status:} 
Untested + NoDoc + NoDemo = NOT OK


    \end{boxedminipage}

    \label{xformslib:flxbasic:fl_keysym_pressed}
    \index{xformslib \textit{(package)}!xformslib.flxbasic \textit{(module)}!xformslib.flxbasic.fl\_keysym\_pressed \textit{(function)}}

    \vspace{0.5ex}

\hspace{.8\funcindent}\begin{boxedminipage}{\funcwidth}

    \raggedright \textbf{fl\_keypressed}(\textit{keysym})

    \vspace{-1.5ex}

    \rule{\textwidth}{0.5\fboxrule}
\setlength{\parskip}{2ex}

\emph{todo}

-{}-
\setlength{\parskip}{1ex}
      \textbf{Parameters}
      \vspace{-1ex}

      \begin{quote}
        \begin{Ventry}{xxxxxx}

          \item[keysym]


\emph{todo}
            {\it (type=long\_pos)}

        \end{Ventry}

      \end{quote}

      \textbf{Return Value}
    \vspace{-1ex}

      \begin{quote}

num., or 0 (on failure)
      {\it (type=int)}

      \end{quote}

\textbf{Note:} 
e.g. \emph{todo}


\textbf{Status:} 
Untested + NoDoc + NoDemo = NOT OK


    \end{boxedminipage}

    \label{xformslib:flxbasic:fl_set_defaults}
    \index{xformslib \textit{(package)}!xformslib.flxbasic \textit{(module)}!xformslib.flxbasic.fl\_set\_defaults \textit{(function)}}

    \vspace{0.5ex}

\hspace{.8\funcindent}\begin{boxedminipage}{\funcwidth}

    \raggedright \textbf{fl\_set\_defaults}(\textit{mask}, \textit{pIopt})

    \vspace{-1.5ex}

    \rule{\textwidth}{0.5\fboxrule}
\setlength{\parskip}{2ex}

-{}-
\setlength{\parskip}{1ex}
      \textbf{Parameters}
      \vspace{-1ex}

      \begin{quote}
        \begin{Ventry}{xxxxx}

          \item[mask]


Mask of program defaults. Values (from xfdata.py) FL\_PDDepth,
FL\_PDClass, FL\_PDDouble, FL\_PDSync, FL\_PDPrivateMap,
FL\_PDScrollbarType, FL\_PDPupFontSize, FL\_PDButtonFontSize,
FL\_PDInputFontSize, FL\_PDSliderFontSize, FL\_PDVisual,
FL\_PDULThickness, FL\_PDULPropWidth, FL\_PDBS, FL\_PDCoordUnit,
FL\_PDDebug, FL\_PDSharedMap, FL\_PDStandardMap, FL\_PDBorderWidth,
FL\_PDSafe, FL\_PDMenuFontSize, FL\_PDBrowserFontSize,
FL\_PDChoiceFontSize, FL\_PDLabelFontSize, FL\_PDButtonLabelSize,
FL\_PDSliderLabelSize, FL\_PDInputLabelSize, FL\_PDButtonLabel
            {\it (type=long\_pos)}

          \item[pIopt]


an array of program defaults class instances
            {\it (type=pointer to xfdata.FL\_IOPT array)}

        \end{Ventry}

      \end{quote}

\textbf{Note:} 
e.g. \emph{todo}


\textbf{Status:} 
Untested + NoDoc + NoDemo = NOT OK


    \end{boxedminipage}

    \label{xformslib:flxbasic:fl_set_tabstop}
    \index{xformslib \textit{(package)}!xformslib.flxbasic \textit{(module)}!xformslib.flxbasic.fl\_set\_tabstop \textit{(function)}}

    \vspace{0.5ex}

\hspace{.8\funcindent}\begin{boxedminipage}{\funcwidth}

    \raggedright \textbf{fl\_set\_tabstop}(\textit{tabtext})

    \vspace{-1.5ex}

    \rule{\textwidth}{0.5\fboxrule}
\setlength{\parskip}{2ex}

Adjusts the distance by setting the tab stops. For proportional font,
substituting tabs with spaces is not always appropriate because this
most likely will fail to align text properly. Instead, a tab is
treated as an absolute measure of distance, in pixels, and a tab
stop will always end at multiples of this distance. The default is
``aaaaaaaa'', i.e. eight 'a's.

-{}-
\setlength{\parskip}{1ex}
      \textbf{Parameters}
      \vspace{-1ex}

      \begin{quote}
        \begin{Ventry}{xxxxxxx}

          \item[tabtext]


text string whose width in pixel is to be used as the tab length.
The font used to calculate the width is the same font that is used
to render the string in which the tab is embedded.
            {\it (type=str)}

        \end{Ventry}

      \end{quote}

\textbf{Note:} 
e.g. fl\_set\_tabstop(``aaaa'')


\textbf{Status:} 
Tested + Doc + NoDemo = OK


    \end{boxedminipage}

    \label{xformslib:flxbasic:fl_get_defaults}
    \index{xformslib \textit{(package)}!xformslib.flxbasic \textit{(module)}!xformslib.flxbasic.fl\_get\_defaults \textit{(function)}}

    \vspace{0.5ex}

\hspace{.8\funcindent}\begin{boxedminipage}{\funcwidth}

    \raggedright \textbf{fl\_get\_defaults}()

    \vspace{-1.5ex}

    \rule{\textwidth}{0.5\fboxrule}
\setlength{\parskip}{2ex}

Obtains program defaults from the resource database.

-{}-
\setlength{\parskip}{1ex}
      \textbf{Return Value}
    \vspace{-1ex}

      \begin{quote}

program defaults class instance
      {\it (type=instance of xfdata.FL\_IOPT)}

      \end{quote}

\textbf{Note:} 
e.g. defprgres = fl\_get\_defaults()


\textbf{Attention:} 
API change from XForms - upstream was
fl\_get\_defaults(pIopt)


\textbf{Status:} 
Tested + Doc + NoDemo = OK


    \end{boxedminipage}

    \label{xformslib:flxbasic:fl_get_visual_depth}
    \index{xformslib \textit{(package)}!xformslib.flxbasic \textit{(module)}!xformslib.flxbasic.fl\_get\_visual\_depth \textit{(function)}}

    \vspace{0.5ex}

\hspace{.8\funcindent}\begin{boxedminipage}{\funcwidth}

    \raggedright \textbf{fl\_get\_visual\_depth}()

    \vspace{-1.5ex}

    \rule{\textwidth}{0.5\fboxrule}
\setlength{\parskip}{2ex}

Obtains the visual depth.

-{}-
\setlength{\parskip}{1ex}
      \textbf{Return Value}
    \vspace{-1ex}

      \begin{quote}

visual depth for current mode
      {\it (type=int)}

      \end{quote}

\textbf{Note:} 
e.g. curdepth = fl\_get\_visual\_depth()


\textbf{Status:} 
Tested + Doc + Demo = OK


    \end{boxedminipage}

    \label{xformslib:flxbasic:fl_vclass_name}
    \index{xformslib \textit{(package)}!xformslib.flxbasic \textit{(module)}!xformslib.flxbasic.fl\_vclass\_name \textit{(function)}}

    \vspace{0.5ex}

\hspace{.8\funcindent}\begin{boxedminipage}{\funcwidth}

    \raggedright \textbf{fl\_vclass\_name}(\textit{mode})

    \vspace{-1.5ex}

    \rule{\textwidth}{0.5\fboxrule}
\setlength{\parskip}{2ex}

Obtains name corresponding to a visual mode.

-{}-
\setlength{\parskip}{1ex}
      \textbf{Parameters}
      \vspace{-1ex}

      \begin{quote}
        \begin{Ventry}{xxxx}

          \item[mode]


visual mode.
            {\it (type=int)}

        \end{Ventry}

      \end{quote}

      \textbf{Return Value}
    \vspace{-1ex}

      \begin{quote}

vclass name
      {\it (type=str)}

      \end{quote}

\textbf{Note:} 
e.g. \emph{todo}


\textbf{Status:} 
Untested + NoDoc + NoDemo = NOT OK


    \end{boxedminipage}

    \label{xformslib:flxbasic:fl_vclass_val}
    \index{xformslib \textit{(package)}!xformslib.flxbasic \textit{(module)}!xformslib.flxbasic.fl\_vclass\_val \textit{(function)}}

    \vspace{0.5ex}

\hspace{.8\funcindent}\begin{boxedminipage}{\funcwidth}

    \raggedright \textbf{fl\_vclass\_val}(\textit{name})

    \vspace{-1.5ex}

    \rule{\textwidth}{0.5\fboxrule}
\setlength{\parskip}{2ex}

Obtains value of visual mode.

-{}-
\setlength{\parskip}{1ex}
      \textbf{Parameters}
      \vspace{-1ex}

      \begin{quote}
        \begin{Ventry}{xxxx}

          \item[name]


name of visual mode
            {\it (type=str)}

        \end{Ventry}

      \end{quote}

      \textbf{Return Value}
    \vspace{-1ex}

      \begin{quote}

vclass num.
      {\it (type=int)}

      \end{quote}

\textbf{Note:} 
e.g. \emph{todo}


\textbf{Status:} 
Untested + NoDoc + NoDemo = NOT OK


    \end{boxedminipage}

    \label{xformslib:flxbasic:fl_set_ul_property}
    \index{xformslib \textit{(package)}!xformslib.flxbasic \textit{(module)}!xformslib.flxbasic.fl\_set\_ul\_property \textit{(function)}}

    \vspace{0.5ex}

\hspace{.8\funcindent}\begin{boxedminipage}{\funcwidth}

    \raggedright \textbf{fl\_set\_ul\_property}(\textit{proportional}, \textit{thickness})

    \vspace{-1.5ex}

    \rule{\textwidth}{0.5\fboxrule}
\setlength{\parskip}{2ex}

Sets property of an underlined text.

-{}-
\setlength{\parskip}{1ex}
      \textbf{Parameters}
      \vspace{-1ex}

      \begin{quote}
        \begin{Ventry}{xxxxxxxxxxxx}

          \item[proportional]


if width is proportional or not. Values 0 (if fixed) or 1 (if
proportional)
            {\it (type=int)}

          \item[thickness]


thickness of underline.
            {\it (type=int)}

        \end{Ventry}

      \end{quote}

\textbf{Note:} 
e.g. \emph{todo}


\textbf{Status:} 
Untested + NoDoc + NoDemo = NOT OK


    \end{boxedminipage}

    \label{xformslib:flxbasic:fl_set_clipping}
    \index{xformslib \textit{(package)}!xformslib.flxbasic \textit{(module)}!xformslib.flxbasic.fl\_set\_clipping \textit{(function)}}

    \vspace{0.5ex}

\hspace{.8\funcindent}\begin{boxedminipage}{\funcwidth}

    \raggedright \textbf{fl\_set\_clipping}(\textit{x}, \textit{y}, \textit{w}, \textit{h})

    \vspace{-1.5ex}

    \rule{\textwidth}{0.5\fboxrule}
\setlength{\parskip}{2ex}

Sets a clipping region in the Forms Library's default GC (gc{[}0{]}).
This defines the area (delimited by arguments passed) drawing is to
restrict to and are relative to the window/form that will be drawn to.
In this way you can prevent drawing over other object and outside the
box.

-{}-
\setlength{\parskip}{1ex}
      \textbf{Parameters}
      \vspace{-1ex}

      \begin{quote}
        \begin{Ventry}{x}

          \item[x]


horizontal position (upper-left corner)
            {\it (type=int)}

          \item[y]


vertical position (upper-left corner)
            {\it (type=int)}

          \item[w]


width in coord units
            {\it (type=int)}

          \item[h]


height in coord units
            {\it (type=int)}

        \end{Ventry}

      \end{quote}

\textbf{Note:} 
e.g. fl\_set\_clipping(250, 200, 100, 80)


\textbf{Status:} 
Tested + Doc + NoDemo = OK


    \end{boxedminipage}

    \label{xformslib:flxbasic:fl_set_clippings}
    \index{xformslib \textit{(package)}!xformslib.flxbasic \textit{(module)}!xformslib.flxbasic.fl\_set\_clippings \textit{(function)}}

    \vspace{0.5ex}

\hspace{.8\funcindent}\begin{boxedminipage}{\funcwidth}

    \raggedright \textbf{fl\_set\_clippings}(\textit{pRect}, \textit{nrect})

    \vspace{-1.5ex}

    \rule{\textwidth}{0.5\fboxrule}
\setlength{\parskip}{2ex}

\emph{todo}

-{}-
\setlength{\parskip}{1ex}
      \textbf{Parameters}
      \vspace{-1ex}

      \begin{quote}
        \begin{Ventry}{xxxxx}

          \item[pRect]


rectangle class instance
            {\it (type=pointer to xfdata.FL\_RECT)}

          \item[nrect]


number of rectangles
            {\it (type=int)}

        \end{Ventry}

      \end{quote}

\textbf{Note:} 
e.g. \emph{todo}


\textbf{Status:} 
Untested + NoDoc + NoDemo = NOT OK


    \end{boxedminipage}

    \label{xformslib:flxbasic:fl_unset_clipping}
    \index{xformslib \textit{(package)}!xformslib.flxbasic \textit{(module)}!xformslib.flxbasic.fl\_unset\_clipping \textit{(function)}}

    \vspace{0.5ex}

\hspace{.8\funcindent}\begin{boxedminipage}{\funcwidth}

    \raggedright \textbf{fl\_unset\_clipping}()

    \vspace{-1.5ex}

    \rule{\textwidth}{0.5\fboxrule}
\setlength{\parskip}{2ex}

Stops clipping and removes clipping area defined with
fl\_set\_clipping()

-{}-
\setlength{\parskip}{1ex}
\textbf{Note:} 
e.g. fl\_unset\_clipping()


\textbf{Status:} 
Tested + Doc + NoDemo = OK


    \end{boxedminipage}

    \label{xformslib:flxbasic:fl_set_text_clipping}
    \index{xformslib \textit{(package)}!xformslib.flxbasic \textit{(module)}!xformslib.flxbasic.fl\_set\_text\_clipping \textit{(function)}}

    \vspace{0.5ex}

\hspace{.8\funcindent}\begin{boxedminipage}{\funcwidth}

    \raggedright \textbf{fl\_set\_text\_clipping}(\textit{x}, \textit{y}, \textit{w}, \textit{h})

    \vspace{-1.5ex}

    \rule{\textwidth}{0.5\fboxrule}
\setlength{\parskip}{2ex}

Sets a clipping region for text in the Forms Library's default GC
(gc{[}0{]}). This defines the area (delimited by arguments passed) drawing
is to restrict to and are relative to the window/form that will be
drawn to. In this way you can prevent drawing over other object and
outside the box.

-{}-
\setlength{\parskip}{1ex}
      \textbf{Parameters}
      \vspace{-1ex}

      \begin{quote}
        \begin{Ventry}{x}

          \item[x]


horizontal position (upper-left corner)
            {\it (type=int)}

          \item[y]


vertical position (upper-left corner)
            {\it (type=int)}

          \item[w]


width in coord units
            {\it (type=int)}

          \item[h]


height in coord units
            {\it (type=int)}

        \end{Ventry}

      \end{quote}

\textbf{Note:} 
e.g. fl\_set\_text\_clipping(200, 200, 300, 50)


\textbf{Status:} 
Tested + Doc + NoDemo = OK


    \end{boxedminipage}

    \label{xformslib:flxbasic:fl_unset_text_clipping}
    \index{xformslib \textit{(package)}!xformslib.flxbasic \textit{(module)}!xformslib.flxbasic.fl\_unset\_text\_clipping \textit{(function)}}

    \vspace{0.5ex}

\hspace{.8\funcindent}\begin{boxedminipage}{\funcwidth}

    \raggedright \textbf{fl\_unset\_text\_clipping}()

    \vspace{-1.5ex}

    \rule{\textwidth}{0.5\fboxrule}
\setlength{\parskip}{2ex}

Stops clipping for text and removes clipping area defined with
fl\_set\_text\_clipping()

-{}-
\setlength{\parskip}{1ex}
\textbf{Note:} 
e.g. fl\_unset\_text\_clipping()


\textbf{Status:} 
Tested + Doc + NoDemo = OK


    \end{boxedminipage}

    \label{xformslib:flxbasic:FL_PCCLAMP}
    \index{xformslib \textit{(package)}!xformslib.flxbasic \textit{(module)}!xformslib.flxbasic.FL\_PCCLAMP \textit{(function)}}

    \vspace{0.5ex}

\hspace{.8\funcindent}\begin{boxedminipage}{\funcwidth}

    \raggedright \textbf{FL\_PCCLAMP}(\textit{a})

\setlength{\parskip}{2ex}
\setlength{\parskip}{1ex}
    \end{boxedminipage}

    \label{xformslib:flxbasic:FL_GETR}
    \index{xformslib \textit{(package)}!xformslib.flxbasic \textit{(module)}!xformslib.flxbasic.FL\_GETR \textit{(function)}}

    \vspace{0.5ex}

\hspace{.8\funcindent}\begin{boxedminipage}{\funcwidth}

    \raggedright \textbf{FL\_GETR}(\textit{packed})

\setlength{\parskip}{2ex}
\setlength{\parskip}{1ex}
    \end{boxedminipage}

    \label{xformslib:flxbasic:FL_GETG}
    \index{xformslib \textit{(package)}!xformslib.flxbasic \textit{(module)}!xformslib.flxbasic.FL\_GETG \textit{(function)}}

    \vspace{0.5ex}

\hspace{.8\funcindent}\begin{boxedminipage}{\funcwidth}

    \raggedright \textbf{FL\_GETG}(\textit{packed})

\setlength{\parskip}{2ex}
\setlength{\parskip}{1ex}
    \end{boxedminipage}

    \label{xformslib:flxbasic:FL_GETB}
    \index{xformslib \textit{(package)}!xformslib.flxbasic \textit{(module)}!xformslib.flxbasic.FL\_GETB \textit{(function)}}

    \vspace{0.5ex}

\hspace{.8\funcindent}\begin{boxedminipage}{\funcwidth}

    \raggedright \textbf{FL\_GETB}(\textit{packed})

\setlength{\parskip}{2ex}
\setlength{\parskip}{1ex}
    \end{boxedminipage}

    \label{xformslib:flxbasic:FL_GETA}
    \index{xformslib \textit{(package)}!xformslib.flxbasic \textit{(module)}!xformslib.flxbasic.FL\_GETA \textit{(function)}}

    \vspace{0.5ex}

\hspace{.8\funcindent}\begin{boxedminipage}{\funcwidth}

    \raggedright \textbf{FL\_GETA}(\textit{packed})

\setlength{\parskip}{2ex}
\setlength{\parskip}{1ex}
    \end{boxedminipage}

    \label{xformslib:flxbasic:FL_PACK3}
    \index{xformslib \textit{(package)}!xformslib.flxbasic \textit{(module)}!xformslib.flxbasic.FL\_PACK3 \textit{(function)}}

    \vspace{0.5ex}

\hspace{.8\funcindent}\begin{boxedminipage}{\funcwidth}

    \raggedright \textbf{FL\_PACK3}(\textit{r}, \textit{g}, \textit{b})

\setlength{\parskip}{2ex}
\setlength{\parskip}{1ex}
    \end{boxedminipage}

    \label{xformslib:flxbasic:FL_PACK3}
    \index{xformslib \textit{(package)}!xformslib.flxbasic \textit{(module)}!xformslib.flxbasic.FL\_PACK3 \textit{(function)}}

    \vspace{0.5ex}

\hspace{.8\funcindent}\begin{boxedminipage}{\funcwidth}

    \raggedright \textbf{FL\_PACK}(\textit{r}, \textit{g}, \textit{b})

\setlength{\parskip}{2ex}
\setlength{\parskip}{1ex}
    \end{boxedminipage}

    \label{xformslib:flxbasic:FL_PACK4}
    \index{xformslib \textit{(package)}!xformslib.flxbasic \textit{(module)}!xformslib.flxbasic.FL\_PACK4 \textit{(function)}}

    \vspace{0.5ex}

\hspace{.8\funcindent}\begin{boxedminipage}{\funcwidth}

    \raggedright \textbf{FL\_PACK4}(\textit{r}, \textit{g}, \textit{b}, \textit{a})

\setlength{\parskip}{2ex}
\setlength{\parskip}{1ex}
    \end{boxedminipage}

    \label{xformslib:flxbasic:FL_UNPACK}
    \index{xformslib \textit{(package)}!xformslib.flxbasic \textit{(module)}!xformslib.flxbasic.FL\_UNPACK \textit{(function)}}

    \vspace{0.5ex}

\hspace{.8\funcindent}\begin{boxedminipage}{\funcwidth}

    \raggedright \textbf{FL\_UNPACK}(\textit{packed})

\setlength{\parskip}{2ex}
\setlength{\parskip}{1ex}
    \end{boxedminipage}

    \label{xformslib:flxbasic:FL_UNPACK}
    \index{xformslib \textit{(package)}!xformslib.flxbasic \textit{(module)}!xformslib.flxbasic.FL\_UNPACK \textit{(function)}}

    \vspace{0.5ex}

\hspace{.8\funcindent}\begin{boxedminipage}{\funcwidth}

    \raggedright \textbf{FL\_UNPACK3}(\textit{packed})

\setlength{\parskip}{2ex}
\setlength{\parskip}{1ex}
    \end{boxedminipage}

    \label{xformslib:flxbasic:FL_UNPACK4}
    \index{xformslib \textit{(package)}!xformslib.flxbasic \textit{(module)}!xformslib.flxbasic.FL\_UNPACK4 \textit{(function)}}

    \vspace{0.5ex}

\hspace{.8\funcindent}\begin{boxedminipage}{\funcwidth}

    \raggedright \textbf{FL\_UNPACK4}(\textit{p}, \textit{r}, \textit{g}, \textit{b}, \textit{a})

\setlength{\parskip}{2ex}
\setlength{\parskip}{1ex}
    \end{boxedminipage}


%%%%%%%%%%%%%%%%%%%%%%%%%%%%%%%%%%%%%%%%%%%%%%%%%%%%%%%%%%%%%%%%%%%%%%%%%%%
%%                               Variables                               %%
%%%%%%%%%%%%%%%%%%%%%%%%%%%%%%%%%%%%%%%%%%%%%%%%%%%%%%%%%%%%%%%%%%%%%%%%%%%

  \subsection{Variables}

    \vspace{-1cm}
\hspace{\varindent}\begin{longtable}{|p{\varnamewidth}|p{\vardescrwidth}|l}
\cline{1-2}
\cline{1-2} \centering \textbf{Name} & \centering \textbf{Description}& \\
\cline{1-2}
\endhead\cline{1-2}\multicolumn{3}{r}{\small\textit{continued on next page}}\\\endfoot\cline{1-2}
\endlastfoot\raggedright f\-l\-\_\-c\-u\-r\-r\-e\-n\-t\-\_\-f\-o\-r\-m\- & \raggedright \textbf{Value:} 
{\tt cty.POINTER(xfdata.FL\_FORM).in\_dll(libr.load\_so\_libforms(\texttt{...}}&\\
\cline{1-2}
\raggedright f\-l\-\_\-d\-i\-s\-p\-l\-a\-y\- & \raggedright \textbf{Value:} 
{\tt cty.POINTER(xfdata.Display).in\_dll(libr.load\_so\_libforms(\texttt{...}}&\\
\cline{1-2}
\raggedright f\-l\-\_\-s\-c\-r\-e\-e\-n\- & \raggedright \textbf{Value:} 
{\tt cty.c\_int.in\_dll(libr.load\_so\_libforms(), 'fl\_screen')}&\\
\cline{1-2}
\raggedright f\-l\-\_\-r\-o\-o\-t\- & \raggedright \textbf{Value:} 
{\tt xfdata.Window.in\_dll(libr.load\_so\_libforms(), 'fl\_root')}&\\
\cline{1-2}
\raggedright f\-l\-\_\-v\-r\-o\-o\-t\- & \raggedright \textbf{Value:} 
{\tt xfdata.Window.in\_dll(libr.load\_so\_libforms(), 'fl\_vroot')}&\\
\cline{1-2}
\raggedright f\-l\-\_\-s\-c\-r\-h\- & \raggedright \textbf{Value:} 
{\tt cty.c\_int.in\_dll(libr.load\_so\_libforms(), 'fl\_scrh')}&\\
\cline{1-2}
\raggedright f\-l\-\_\-s\-c\-r\-w\- & \raggedright \textbf{Value:} 
{\tt cty.c\_int.in\_dll(libr.load\_so\_libforms(), 'fl\_scrw')}&\\
\cline{1-2}
\raggedright f\-l\-\_\-v\-m\-o\-d\-e\- & \raggedright \textbf{Value:} 
{\tt cty.c\_int.in\_dll(libr.load\_so\_libforms(), 'fl\_vmode')}&\\
\cline{1-2}
\raggedright f\-l\-\_\-s\-t\-a\-t\-e\- & \raggedright \textbf{Value:} 
{\tt cty.POINTER(xfdata.FL\_State).in\_dll(libr.load\_so\_libforms\texttt{...}}&\\
\cline{1-2}
\raggedright f\-l\-\_\-u\-l\-\_\-m\-a\-g\-i\-c\-\_\-c\-h\-a\-r\- & \raggedright \textbf{Value:} 
{\tt xfdata.STRING.in\_dll(libr.load\_so\_libforms(), 'fl\_state')}&\\
\cline{1-2}
\end{longtable}

    \index{xformslib \textit{(package)}!xformslib.flxbasic \textit{(module)}|)}
