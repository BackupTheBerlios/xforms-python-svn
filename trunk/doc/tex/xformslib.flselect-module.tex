%
% API Documentation for API Documentation
% Module xformslib.flselect
%
% Generated by epydoc 3.0.1
% [Tue May 18 00:48:13 2010]
%

%%%%%%%%%%%%%%%%%%%%%%%%%%%%%%%%%%%%%%%%%%%%%%%%%%%%%%%%%%%%%%%%%%%%%%%%%%%
%%                          Module Description                           %%
%%%%%%%%%%%%%%%%%%%%%%%%%%%%%%%%%%%%%%%%%%%%%%%%%%%%%%%%%%%%%%%%%%%%%%%%%%%

    \index{xformslib \textit{(package)}!xformslib.flselect \textit{(module)}|(}
\section{Module xformslib.flselect}

    \label{xformslib:flselect}

xforms-python's functions to manage select objects.

Copyright (C) 2009, 2010  Luca Lazzaroni ``LukenShiro''
e-mail: <\href{mailto:lukenshiro@ngi.it}{lukenshiro@ngi.it}>

This program is free software: you can redistribute it and/or modify
it under the terms of the GNU Lesser General Public License as
published by the Free Software Foundation, version 2.1 of the License.

This program is distributed in the hope that it will be useful,
but WITHOUT ANY WARRANTY; without even the implied warranty of
MERCHANTABILITY or FITNESS FOR A PARTICULAR PURPOSE. See the
GNU Lesser General Public License for more details.

You should have received a copy of the GNU LGPL along with this
program. If not, see <\href{http://www.gnu.org/licenses/}{http://www.gnu.org/licenses/}>.

See CREDITS file to read acknowledgements and thanks to XForms,
ctypes and other developers.

%%%%%%%%%%%%%%%%%%%%%%%%%%%%%%%%%%%%%%%%%%%%%%%%%%%%%%%%%%%%%%%%%%%%%%%%%%%
%%                               Functions                               %%
%%%%%%%%%%%%%%%%%%%%%%%%%%%%%%%%%%%%%%%%%%%%%%%%%%%%%%%%%%%%%%%%%%%%%%%%%%%

  \subsection{Functions}

    \label{xformslib:flselect:fl_add_select}
    \index{xformslib \textit{(package)}!xformslib.flselect \textit{(module)}!xformslib.flselect.fl\_add\_select \textit{(function)}}

    \vspace{0.5ex}

\hspace{.8\funcindent}\begin{boxedminipage}{\funcwidth}

    \raggedright \textbf{fl\_add\_select}(\textit{selecttype}, \textit{x}, \textit{y}, \textit{w}, \textit{h}, \textit{label})

    \vspace{-1.5ex}

    \rule{\textwidth}{0.5\fboxrule}
\setlength{\parskip}{2ex}

Adds a select (new generation choice) object to the form.

-{}-
\setlength{\parskip}{1ex}
      \textbf{Parameters}
      \vspace{-1ex}

      \begin{quote}
        \begin{Ventry}{xxxxxxxxxx}

          \item[selecttype]


type of select to be added. Values (from xfdata.py) FL\_NORMAL\_SELECT,
FL\_MENU\_SELECT, FL\_DROPLIST\_SELECT
            {\it (type=int)}

          \item[x]


horizontal position (upper-left corner)
            {\it (type=int)}

          \item[y]


vertical position (upper-left corner)
            {\it (type=int)}

          \item[w]


width in coord units
            {\it (type=int)}

          \item[h]


height in coord units
            {\it (type=int)}

          \item[label]


text label of select
            {\it (type=str)}

        \end{Ventry}

      \end{quote}

      \textbf{Return Value}
    \vspace{-1ex}

      \begin{quote}

select object added (pFlObject)
      {\it (type=pointer to xfdata.FL\_OBJECT)}

      \end{quote}

\textbf{Note:} 
e.g. \emph{todo}


\textbf{Status:} 
Tested + NoDoc + Demo = OK


    \end{boxedminipage}

    \label{xformslib:flselect:fl_clear_select}
    \index{xformslib \textit{(package)}!xformslib.flselect \textit{(module)}!xformslib.flselect.fl\_clear\_select \textit{(function)}}

    \vspace{0.5ex}

\hspace{.8\funcindent}\begin{boxedminipage}{\funcwidth}

    \raggedright \textbf{fl\_clear\_select}(\textit{pFlObject})

    \vspace{-1.5ex}

    \rule{\textwidth}{0.5\fboxrule}
\setlength{\parskip}{2ex}

\emph{todo}

-{}-
\setlength{\parskip}{1ex}
      \textbf{Parameters}
      \vspace{-1ex}

      \begin{quote}
        \begin{Ventry}{xxxxxxxxx}

          \item[pFlObject]


select object
            {\it (type=pointer to xfdata.FL\_OBJECT)}

        \end{Ventry}

      \end{quote}

      \textbf{Return Value}
    \vspace{-1ex}

      \begin{quote}

num.
      {\it (type=int)}

      \end{quote}

\textbf{Note:} 
e.g. \emph{todo}


\textbf{Status:} 
Untested + NoDoc + NoDemo = NOT OK


    \end{boxedminipage}

    \label{xformslib:flselect:fl_add_select_items}
    \index{xformslib \textit{(package)}!xformslib.flselect \textit{(module)}!xformslib.flselect.fl\_add\_select\_items \textit{(function)}}

    \vspace{0.5ex}

\hspace{.8\funcindent}\begin{boxedminipage}{\funcwidth}

    \raggedright \textbf{fl\_add\_select\_items}(\textit{pFlObject}, \textit{itemstr})

    \vspace{-1.5ex}

    \rule{\textwidth}{0.5\fboxrule}
\setlength{\parskip}{2ex}

Adds one or more items to a select object.

-{}-
\setlength{\parskip}{1ex}
      \textbf{Parameters}
      \vspace{-1ex}

      \begin{quote}
        \begin{Ventry}{xxxxxxxxx}

          \item[pFlObject]


select object
            {\it (type=pointer to xfdata.FL\_OBJECT)}

          \item[itemstr]


text for the items to add, separated by the | character. Some special
sequences are allowed just after the item ('\%d' marks the item as
disabled, i.e. it can't be selected and its text is per default drawn
in a different color / '\%h' marks the item as hidden, i.e. it is not
shown while in this state / '\%S' can split the items text into two
parts, the first one (before it) being drawn flushed left and the
second part flushed right, anyway you still need to set a shortcut key
/ '\%s' sets one or more shortcut keys for an item, it requires a
string with the shortcuts in the argument following the items string;
the character in the label identical to the shortcut character is only
shown as underlined if \%S isn't used.
            {\it (type=str)}

        \end{Ventry}

      \end{quote}

      \textbf{Return Value}
    \vspace{-1ex}

      \begin{quote}

popup entry (pPopupEntry)
      {\it (type=pointer to xfdata.FL\_POPUP\_ENTRY)}

      \end{quote}

\textbf{Note:} 
e.g. \emph{todo}


\textbf{Status:} 
HalfTested + NoDoc + Demo = NOT OK (sequence param.)


    \end{boxedminipage}

    \label{xformslib:flselect:fl_insert_select_items}
    \index{xformslib \textit{(package)}!xformslib.flselect \textit{(module)}!xformslib.flselect.fl\_insert\_select\_items \textit{(function)}}

    \vspace{0.5ex}

\hspace{.8\funcindent}\begin{boxedminipage}{\funcwidth}

    \raggedright \textbf{fl\_insert\_select\_items}(\textit{pFlObject}, \textit{pPopupEntry}, \textit{itemstr})

    \vspace{-1.5ex}

    \rule{\textwidth}{0.5\fboxrule}
\setlength{\parskip}{2ex}

\emph{todo}

-{}-
\setlength{\parskip}{1ex}
      \textbf{Parameters}
      \vspace{-1ex}

      \begin{quote}
        \begin{Ventry}{xxxxxxxxxxx}

          \item[pFlObject]


select object
            {\it (type=pointer to xfdata.FL\_OBJECT)}

          \item[pPopupEntry]


popup entry
            {\it (type=pointer to xfdata.FL\_POPUP\_ENTRY)}

          \item[itemstr]


text of the item (among special sequences only \%S is supported)
            {\it (type=str)}

        \end{Ventry}

      \end{quote}

      \textbf{Return Value}
    \vspace{-1ex}

      \begin{quote}

popup entry
      {\it (type=pointer to xfdata.FL\_POPUP\_ENTRY)}

      \end{quote}

\textbf{Note:} 
e.g. \emph{todo}


\textbf{Status:} 
HalfTested + NoDoc + Demo = NOT OK (special sequence)


    \end{boxedminipage}

    \label{xformslib:flselect:fl_replace_select_item}
    \index{xformslib \textit{(package)}!xformslib.flselect \textit{(module)}!xformslib.flselect.fl\_replace\_select\_item \textit{(function)}}

    \vspace{0.5ex}

\hspace{.8\funcindent}\begin{boxedminipage}{\funcwidth}

    \raggedright \textbf{fl\_replace\_select\_item}(\textit{pFlObject}, \textit{pPopupEntry}, \textit{itemstr})

    \vspace{-1.5ex}

    \rule{\textwidth}{0.5\fboxrule}
\setlength{\parskip}{2ex}

\emph{todo}

-{}-
\setlength{\parskip}{1ex}
      \textbf{Parameters}
      \vspace{-1ex}

      \begin{quote}
        \begin{Ventry}{xxxxxxxxxxx}

          \item[pFlObject]


select object
            {\it (type=pointer to xfdata.FL\_OBJECT)}

          \item[pPopupEntry]


popup entry
            {\it (type=pointer to xfdata.FL\_POPUP\_ENTRY)}

          \item[itemstr]


text of the item (among special sequences only \%S is supported)
            {\it (type=str)}

        \end{Ventry}

      \end{quote}

      \textbf{Return Value}
    \vspace{-1ex}

      \begin{quote}

popup entry
      {\it (type=pointer to xfdata.FL\_POPUP\_ENTRY)}

      \end{quote}

\textbf{Note:} 
e.g. \emph{todo}


\textbf{Status:} 
Untested + NoDoc + NoDemo = NOT OK


    \end{boxedminipage}

    \label{xformslib:flselect:fl_delete_select_item}
    \index{xformslib \textit{(package)}!xformslib.flselect \textit{(module)}!xformslib.flselect.fl\_delete\_select\_item \textit{(function)}}

    \vspace{0.5ex}

\hspace{.8\funcindent}\begin{boxedminipage}{\funcwidth}

    \raggedright \textbf{fl\_delete\_select\_item}(\textit{pFlObject}, \textit{pPopupEntry})

    \vspace{-1.5ex}

    \rule{\textwidth}{0.5\fboxrule}
\setlength{\parskip}{2ex}

\emph{todo}

-{}-
\setlength{\parskip}{1ex}
      \textbf{Parameters}
      \vspace{-1ex}

      \begin{quote}
        \begin{Ventry}{xxxxxxxxxxx}

          \item[pFlObject]


select object
            {\it (type=pointer to xfdata.FL\_OBJECT)}

          \item[pPopupEntry]


popup entry
            {\it (type=pointer to xfdata.FL\_POPUP\_ENTRY)}

        \end{Ventry}

      \end{quote}

      \textbf{Return Value}
    \vspace{-1ex}

      \begin{quote}

num.
      {\it (type=int)}

      \end{quote}

\textbf{Note:} 
e.g. \emph{todo}


\textbf{Status:} 
Untested + NoDoc + NoDemo = NOT OK


    \end{boxedminipage}

    \label{xformslib:flselect:fl_set_select_items}
    \index{xformslib \textit{(package)}!xformslib.flselect \textit{(module)}!xformslib.flselect.fl\_set\_select\_items \textit{(function)}}

    \vspace{0.5ex}

\hspace{.8\funcindent}\begin{boxedminipage}{\funcwidth}

    \raggedright \textbf{fl\_set\_select\_items}(\textit{pFlObject}, \textit{pPopupItem})

    \vspace{-1.5ex}

    \rule{\textwidth}{0.5\fboxrule}
\setlength{\parskip}{2ex}

(Re)populates a select object.

-{}-
\setlength{\parskip}{1ex}
      \textbf{Parameters}
      \vspace{-1ex}

      \begin{quote}
        \begin{Ventry}{xxxxxxxxxx}

          \item[pFlObject]


select object
            {\it (type=pointer to xfdata.FL\_OBJECT)}

          \item[pPopupItem]


popup item class instance (array of it)
            {\it (type=pointer to xfdata.FL\_POPUP\_ITEM)}

        \end{Ventry}

      \end{quote}

      \textbf{Return Value}
    \vspace{-1ex}

      \begin{quote}

num.
      {\it (type=int)}

      \end{quote}

\textbf{Note:} 
e.g. \emph{todo}


\textbf{Status:} 
Untested + NoDoc + NoDemo = NOT OK


    \end{boxedminipage}

    \label{xformslib:flselect:fl_get_select_popup}
    \index{xformslib \textit{(package)}!xformslib.flselect \textit{(module)}!xformslib.flselect.fl\_get\_select\_popup \textit{(function)}}

    \vspace{0.5ex}

\hspace{.8\funcindent}\begin{boxedminipage}{\funcwidth}

    \raggedright \textbf{fl\_get\_select\_popup}(\textit{pFlObject})

    \vspace{-1.5ex}

    \rule{\textwidth}{0.5\fboxrule}
\setlength{\parskip}{2ex}

\emph{todo}

-{}-
\setlength{\parskip}{1ex}
      \textbf{Parameters}
      \vspace{-1ex}

      \begin{quote}
        \begin{Ventry}{xxxxxxxxx}

          \item[pFlObject]


select object
            {\it (type=pointer to xfdata.FL\_OBJECT)}

        \end{Ventry}

      \end{quote}

      \textbf{Return Value}
    \vspace{-1ex}

      \begin{quote}

popup class instance
      {\it (type=pointer to xfdata.FL\_POPUP)}

      \end{quote}

\textbf{Note:} 
e.g. \emph{todo}


\textbf{Status:} 
Tested + NoDoc + Demo = OK


    \end{boxedminipage}

    \label{xformslib:flselect:fl_set_select_popup}
    \index{xformslib \textit{(package)}!xformslib.flselect \textit{(module)}!xformslib.flselect.fl\_set\_select\_popup \textit{(function)}}

    \vspace{0.5ex}

\hspace{.8\funcindent}\begin{boxedminipage}{\funcwidth}

    \raggedright \textbf{fl\_set\_select\_popup}(\textit{pFlObject}, \textit{pPopup})

    \vspace{-1.5ex}

    \rule{\textwidth}{0.5\fboxrule}
\setlength{\parskip}{2ex}

\emph{todo}

-{}-
\setlength{\parskip}{1ex}
      \textbf{Parameters}
      \vspace{-1ex}

      \begin{quote}
        \begin{Ventry}{xxxxxxxxx}

          \item[pFlObject]


select object
            {\it (type=pointer to xfdata.FL\_OBJECT)}

          \item[pPopup]


popup class instance
            {\it (type=pointer to xfdata.FL\_POPUP)}

        \end{Ventry}

      \end{quote}

      \textbf{Return Value}
    \vspace{-1ex}

      \begin{quote}

num.
      {\it (type=int)}

      \end{quote}

\textbf{Note:} 
e.g. \emph{todo}


\textbf{Status:} 
Untested + NoDoc + NoDemo = NOT OK


    \end{boxedminipage}

    \label{xformslib:flselect:fl_get_select_item}
    \index{xformslib \textit{(package)}!xformslib.flselect \textit{(module)}!xformslib.flselect.fl\_get\_select\_item \textit{(function)}}

    \vspace{0.5ex}

\hspace{.8\funcindent}\begin{boxedminipage}{\funcwidth}

    \raggedright \textbf{fl\_get\_select\_item}(\textit{pFlObject})

    \vspace{-1.5ex}

    \rule{\textwidth}{0.5\fboxrule}
\setlength{\parskip}{2ex}

Obtains currently selected item of a select object.

-{}-
\setlength{\parskip}{1ex}
      \textbf{Parameters}
      \vspace{-1ex}

      \begin{quote}
        \begin{Ventry}{xxxxxxxxx}

          \item[pFlObject]


select object
            {\it (type=pointer to xfdata.FL\_OBJECT)}

        \end{Ventry}

      \end{quote}

      \textbf{Return Value}
    \vspace{-1ex}

      \begin{quote}

popup return
      {\it (type=pointer to xfdata.FL\_POPUP\_RETURN)}

      \end{quote}

\textbf{Note:} 
e.g. \emph{todo}


\textbf{Status:} 
Tested + NoDoc + Demo = OK


    \end{boxedminipage}

    \label{xformslib:flselect:fl_set_select_item}
    \index{xformslib \textit{(package)}!xformslib.flselect \textit{(module)}!xformslib.flselect.fl\_set\_select\_item \textit{(function)}}

    \vspace{0.5ex}

\hspace{.8\funcindent}\begin{boxedminipage}{\funcwidth}

    \raggedright \textbf{fl\_set\_select\_item}(\textit{pFlObject}, \textit{pPopupEntry})

    \vspace{-1.5ex}

    \rule{\textwidth}{0.5\fboxrule}
\setlength{\parskip}{2ex}

Sets a new item of a select object as currently selected.

-{}-
\setlength{\parskip}{1ex}
      \textbf{Parameters}
      \vspace{-1ex}

      \begin{quote}
        \begin{Ventry}{xxxxxxxxxxx}

          \item[pFlObject]


select object
            {\it (type=pointer to xfdata.FL\_OBJECT)}

          \item[pPopupEntry]


popup entry class instance
            {\it (type=pointer to xfdata.FL\_POPUP\_ENTRY)}

        \end{Ventry}

      \end{quote}

      \textbf{Return Value}
    \vspace{-1ex}

      \begin{quote}

popup return
      {\it (type=pointer to xfdata.FL\_POPUP\_RETURN)}

      \end{quote}

\textbf{Note:} 
e.g. \emph{todo}


\textbf{Status:} 
HalfTested + NoDoc + Demo = NOT OK (FL\_POPUP\_ENTRY not prepared)


    \end{boxedminipage}

    \label{xformslib:flselect:fl_get_select_item_by_value}
    \index{xformslib \textit{(package)}!xformslib.flselect \textit{(module)}!xformslib.flselect.fl\_get\_select\_item\_by\_value \textit{(function)}}

    \vspace{0.5ex}

\hspace{.8\funcindent}\begin{boxedminipage}{\funcwidth}

    \raggedright \textbf{fl\_get\_select\_item\_by\_value}(\textit{pFlObject}, \textit{value})

    \vspace{-1.5ex}

    \rule{\textwidth}{0.5\fboxrule}
\setlength{\parskip}{2ex}

\emph{todo}

-{}-
\setlength{\parskip}{1ex}
      \textbf{Parameters}
      \vspace{-1ex}

      \begin{quote}
        \begin{Ventry}{xxxxxxxxx}

          \item[pFlObject]


select object
            {\it (type=pointer to xfdata.FL\_OBJECT)}

          \item[value]


value?
            {\it (type=long)}

        \end{Ventry}

      \end{quote}

      \textbf{Return Value}
    \vspace{-1ex}

      \begin{quote}

popup entry class instance
      {\it (type=pointer to xfdata.FL\_POPUP\_ENTRY)}

      \end{quote}

\textbf{Note:} 
e.g. \emph{todo}


\textbf{Status:} 
Untested + NoDoc + NoDemo = NOT OK


    \end{boxedminipage}

    \label{xformslib:flselect:fl_get_select_item_by_label}
    \index{xformslib \textit{(package)}!xformslib.flselect \textit{(module)}!xformslib.flselect.fl\_get\_select\_item\_by\_label \textit{(function)}}

    \vspace{0.5ex}

\hspace{.8\funcindent}\begin{boxedminipage}{\funcwidth}

    \raggedright \textbf{fl\_get\_select\_item\_by\_label}(\textit{pFlObject}, \textit{label})

    \vspace{-1.5ex}

    \rule{\textwidth}{0.5\fboxrule}
\setlength{\parskip}{2ex}

\emph{todo}

-{}-
\setlength{\parskip}{1ex}
      \textbf{Parameters}
      \vspace{-1ex}

      \begin{quote}
        \begin{Ventry}{xxxxxxxxx}

          \item[pFlObject]


select object
            {\it (type=pointer to xfdata.FL\_OBJECT)}

          \item[label]


label?
            {\it (type=str)}

        \end{Ventry}

      \end{quote}

      \textbf{Return Value}
    \vspace{-1ex}

      \begin{quote}

popup entry class instance
      {\it (type=pointer to xfdata.FL\_POPUP\_ENTRY)}

      \end{quote}

\textbf{Note:} 
e.g. \emph{todo}


\textbf{Status:} 
Tested + NoDoc + Demo = OK


    \end{boxedminipage}

    \label{xformslib:flselect:fl_get_select_item_by_text}
    \index{xformslib \textit{(package)}!xformslib.flselect \textit{(module)}!xformslib.flselect.fl\_get\_select\_item\_by\_text \textit{(function)}}

    \vspace{0.5ex}

\hspace{.8\funcindent}\begin{boxedminipage}{\funcwidth}

    \raggedright \textbf{fl\_get\_select\_item\_by\_text}(\textit{pFlObject}, \textit{txtstr})

    \vspace{-1.5ex}

    \rule{\textwidth}{0.5\fboxrule}
\setlength{\parskip}{2ex}

\emph{todo}

-{}-
\setlength{\parskip}{1ex}
      \textbf{Parameters}
      \vspace{-1ex}

      \begin{quote}
        \begin{Ventry}{xxxxxxxxx}

          \item[pFlObject]


select object
            {\it (type=pointer to xfdata.FL\_OBJECT)}

          \item[txtstr]


text?
            {\it (type=str)}

        \end{Ventry}

      \end{quote}

      \textbf{Return Value}
    \vspace{-1ex}

      \begin{quote}

popup entry class instance
      {\it (type=pointer to xfdata.FL\_POPUP\_ENTRY)}

      \end{quote}

\textbf{Note:} 
e.g. \emph{todo}


\textbf{Status:} 
Untested + NoDoc + NoDemo = NOT OK


    \end{boxedminipage}

    \label{xformslib:flselect:fl_get_select_text_color}
    \index{xformslib \textit{(package)}!xformslib.flselect \textit{(module)}!xformslib.flselect.fl\_get\_select\_text\_color \textit{(function)}}

    \vspace{0.5ex}

\hspace{.8\funcindent}\begin{boxedminipage}{\funcwidth}

    \raggedright \textbf{fl\_get\_select\_text\_color}(\textit{pFlObject})

    \vspace{-1.5ex}

    \rule{\textwidth}{0.5\fboxrule}
\setlength{\parskip}{2ex}

\emph{todo}

-{}-
\setlength{\parskip}{1ex}
      \textbf{Parameters}
      \vspace{-1ex}

      \begin{quote}
        \begin{Ventry}{xxxxxxxxx}

          \item[pFlObject]


select object
            {\it (type=pointer to xfdata.FL\_OBJECT)}

        \end{Ventry}

      \end{quote}

      \textbf{Return Value}
    \vspace{-1ex}

      \begin{quote}

color
      {\it (type=long\_pos)}

      \end{quote}

\textbf{Note:} 
e.g. \emph{todo}


\textbf{Status:} 
Untested + NoDoc + NoDemo = NOT OK


    \end{boxedminipage}

    \label{xformslib:flselect:fl_set_select_text_color}
    \index{xformslib \textit{(package)}!xformslib.flselect \textit{(module)}!xformslib.flselect.fl\_set\_select\_text\_color \textit{(function)}}

    \vspace{0.5ex}

\hspace{.8\funcindent}\begin{boxedminipage}{\funcwidth}

    \raggedright \textbf{fl\_set\_select\_text\_color}(\textit{pFlObject}, \textit{colr})

    \vspace{-1.5ex}

    \rule{\textwidth}{0.5\fboxrule}
\setlength{\parskip}{2ex}

\emph{todo}

-{}-
\setlength{\parskip}{1ex}
      \textbf{Parameters}
      \vspace{-1ex}

      \begin{quote}
        \begin{Ventry}{xxxxxxxxx}

          \item[pFlObject]


select object
            {\it (type=pointer to xfdata.FL\_OBJECT)}

          \item[colr]


color value
            {\it (type=long\_pos)}

        \end{Ventry}

      \end{quote}

      \textbf{Return Value}
    \vspace{-1ex}

      \begin{quote}

old color?
      {\it (type=long\_pos)}

      \end{quote}

\textbf{Note:} 
e.g. \emph{todo}


\textbf{Status:} 
Untested + NoDoc + NoDemo = NOT OK


    \end{boxedminipage}

    \label{xformslib:flselect:fl_get_select_text_font}
    \index{xformslib \textit{(package)}!xformslib.flselect \textit{(module)}!xformslib.flselect.fl\_get\_select\_text\_font \textit{(function)}}

    \vspace{0.5ex}

\hspace{.8\funcindent}\begin{boxedminipage}{\funcwidth}

    \raggedright \textbf{fl\_get\_select\_text\_font}(\textit{pFlObject})

    \vspace{-1.5ex}

    \rule{\textwidth}{0.5\fboxrule}
\setlength{\parskip}{2ex}

\emph{todo}

-{}-
\setlength{\parskip}{1ex}
      \textbf{Parameters}
      \vspace{-1ex}

      \begin{quote}
        \begin{Ventry}{xxxxxxxxx}

          \item[pFlObject]


select object
            {\it (type=pointer to xfdata.FL\_OBJECT)}

        \end{Ventry}

      \end{quote}

      \textbf{Return Value}
    \vspace{-1ex}

      \begin{quote}

0 or -1 (on failure), style, size
      {\it (type=int, int, int)}

      \end{quote}

\textbf{Note:} 
e.g. \emph{todo}


\textbf{Attention:} 
API change from XForms - upstream was
fl\_get\_select\_text\_font(pFlObject, p2, p3)


\textbf{Status:} 
Untested + NoDoc + NoDemo = NOT OK


    \end{boxedminipage}

    \label{xformslib:flselect:fl_set_select_text_font}
    \index{xformslib \textit{(package)}!xformslib.flselect \textit{(module)}!xformslib.flselect.fl\_set\_select\_text\_font \textit{(function)}}

    \vspace{0.5ex}

\hspace{.8\funcindent}\begin{boxedminipage}{\funcwidth}

    \raggedright \textbf{fl\_set\_select\_text\_font}(\textit{pFlObject}, \textit{style}, \textit{size})

    \vspace{-1.5ex}

    \rule{\textwidth}{0.5\fboxrule}
\setlength{\parskip}{2ex}

\emph{todo}

-{}-
\setlength{\parskip}{1ex}
      \textbf{Parameters}
      \vspace{-1ex}

      \begin{quote}
        \begin{Ventry}{xxxxxxxxx}

          \item[pFlObject]


select object
            {\it (type=pointer to xfdata.FL\_OBJECT)}

          \item[style]


text style. Values (from xfdata.py) FL\_NORMAL\_STYLE,
FL\_BOLD\_STYLE, FL\_ITALIC\_STYLE, FL\_BOLDITALIC\_STYLE, FL\_FIXED\_STYLE,
FL\_FIXEDBOLD\_STYLE, FL\_FIXEDITALIC\_STYLE, FL\_FIXEDBOLDITALIC\_STYLE,
FL\_TIMES\_STYLE, FL\_TIMESBOLD\_STYLE, FL\_TIMESITALIC\_STYLE,
FL\_TIMESBOLDITALIC\_STYLE, FL\_MISC\_STYLE, FL\_MISCBOLD\_STYLE,
FL\_MISCITALIC\_STYLE, FL\_SYMBOL\_STYLE, FL\_SHADOW\_STYLE,
FL\_ENGRAVED\_STYLE, FL\_EMBOSSED\_STYLE
            {\it (type=int)}

          \item[size]


text size. Values (from xfdata.py) FL\_TINY\_SIZE, FL\_SMALL\_SIZE,
FL\_NORMAL\_SIZE, FL\_MEDIUM\_SIZE, FL\_LARGE\_SIZE, FL\_HUGE\_SIZE,
FL\_DEFAULT\_SIZE
            {\it (type=int)}

        \end{Ventry}

      \end{quote}

      \textbf{Return Value}
    \vspace{-1ex}

      \begin{quote}

font num.
      {\it (type=int)}

      \end{quote}

\textbf{Note:} 
e.g. \emph{todo}


\textbf{Status:} 
Untested + NoDoc + NoDemo = NOT OK


    \end{boxedminipage}

    \label{xformslib:flselect:fl_get_select_text_align}
    \index{xformslib \textit{(package)}!xformslib.flselect \textit{(module)}!xformslib.flselect.fl\_get\_select\_text\_align \textit{(function)}}

    \vspace{0.5ex}

\hspace{.8\funcindent}\begin{boxedminipage}{\funcwidth}

    \raggedright \textbf{fl\_get\_select\_text\_align}(\textit{pFlObject})

    \vspace{-1.5ex}

    \rule{\textwidth}{0.5\fboxrule}
\setlength{\parskip}{2ex}

\emph{todo}

-{}-
\setlength{\parskip}{1ex}
      \textbf{Parameters}
      \vspace{-1ex}

      \begin{quote}
        \begin{Ventry}{xxxxxxxxx}

          \item[pFlObject]


select object
            {\it (type=pointer to xfdata.FL\_OBJECT)}

        \end{Ventry}

      \end{quote}

      \textbf{Return Value}
    \vspace{-1ex}

      \begin{quote}

num.
      {\it (type=int)}

      \end{quote}

\textbf{Note:} 
e.g. \emph{todo}


\textbf{Status:} 
Untested + NoDoc + NoDemo = NOT OK


    \end{boxedminipage}

    \label{xformslib:flselect:fl_set_select_text_align}
    \index{xformslib \textit{(package)}!xformslib.flselect \textit{(module)}!xformslib.flselect.fl\_set\_select\_text\_align \textit{(function)}}

    \vspace{0.5ex}

\hspace{.8\funcindent}\begin{boxedminipage}{\funcwidth}

    \raggedright \textbf{fl\_set\_select\_text\_align}(\textit{pFlObject}, \textit{align})

    \vspace{-1.5ex}

    \rule{\textwidth}{0.5\fboxrule}
\setlength{\parskip}{2ex}

\emph{todo}

-{}-
\setlength{\parskip}{1ex}
      \textbf{Parameters}
      \vspace{-1ex}

      \begin{quote}
        \begin{Ventry}{xxxxxxxxx}

          \item[pFlObject]


select object
            {\it (type=pointer to xfdata.FL\_OBJECT)}

          \item[align]


alignment of text. Values (from xfdata.py) FL\_ALIGN\_CENTER,
FL\_ALIGN\_TOP, FL\_ALIGN\_BOTTOM, FL\_ALIGN\_LEFT, FL\_ALIGN\_RIGHT,
FL\_ALIGN\_LEFT\_TOP, FL\_ALIGN\_RIGHT\_TOP, FL\_ALIGN\_LEFT\_BOTTOM,
FL\_ALIGN\_RIGHT\_BOTTOM, FL\_ALIGN\_INSIDE, FL\_ALIGN\_VERT.
Bitwise OR with FL\_ALIGN\_INSIDE is allowed.
            {\it (type=int)}

        \end{Ventry}

      \end{quote}

      \textbf{Return Value}
    \vspace{-1ex}

      \begin{quote}

num.
      {\it (type=int)}

      \end{quote}

\textbf{Note:} 
e.g. \emph{todo}


\textbf{Status:} 
Untested + NoDoc + NoDemo = NOT OK


    \end{boxedminipage}

    \label{xformslib:flselect:fl_set_select_policy}
    \index{xformslib \textit{(package)}!xformslib.flselect \textit{(module)}!xformslib.flselect.fl\_set\_select\_policy \textit{(function)}}

    \vspace{0.5ex}

\hspace{.8\funcindent}\begin{boxedminipage}{\funcwidth}

    \raggedright \textbf{fl\_set\_select\_policy}(\textit{pFlObject}, \textit{policy})

    \vspace{-1.5ex}

    \rule{\textwidth}{0.5\fboxrule}
\setlength{\parskip}{2ex}

\emph{todo}

-{}-
\setlength{\parskip}{1ex}
      \textbf{Parameters}
      \vspace{-1ex}

      \begin{quote}
        \begin{Ventry}{xxxxxxxxx}

          \item[pFlObject]


select object
            {\it (type=pointer to xfdata.FL\_OBJECT)}

          \item[policy]


popup policy to be set. Values (from xfdata.py) FL\_POPUP\_NORMAL\_SELECT,
FL\_POPUP\_DRAG\_SELECT.
            {\it (type=int)}

        \end{Ventry}

      \end{quote}

      \textbf{Return Value}
    \vspace{-1ex}

      \begin{quote}

num.
      {\it (type=int)}

      \end{quote}

\textbf{Note:} 
e.g. \emph{todo}


\textbf{Status:} 
Tested + NoDoc + Demo = OK


    \end{boxedminipage}


%%%%%%%%%%%%%%%%%%%%%%%%%%%%%%%%%%%%%%%%%%%%%%%%%%%%%%%%%%%%%%%%%%%%%%%%%%%
%%                               Variables                               %%
%%%%%%%%%%%%%%%%%%%%%%%%%%%%%%%%%%%%%%%%%%%%%%%%%%%%%%%%%%%%%%%%%%%%%%%%%%%

  \subsection{Variables}

    \vspace{-1cm}
\hspace{\varindent}\begin{longtable}{|p{\varnamewidth}|p{\vardescrwidth}|l}
\cline{1-2}
\cline{1-2} \centering \textbf{Name} & \centering \textbf{Description}& \\
\cline{1-2}
\endhead\cline{1-2}\multicolumn{3}{r}{\small\textit{continued on next page}}\\\endfoot\cline{1-2}
\endlastfoot\raggedright \_\-\_\-p\-a\-c\-k\-a\-g\-e\-\_\-\_\- & \raggedright \textbf{Value:} 
{\tt \texttt{'}\texttt{xformslib}\texttt{'}}&\\
\cline{1-2}
\end{longtable}

    \index{xformslib \textit{(package)}!xformslib.flselect \textit{(module)}|)}
