%
% API Documentation for xforms-python
% Module xformslib.flxyplot
%
% Generated by epydoc 3.0.1
% [Mon May 24 11:23:25 2010]
%

%%%%%%%%%%%%%%%%%%%%%%%%%%%%%%%%%%%%%%%%%%%%%%%%%%%%%%%%%%%%%%%%%%%%%%%%%%%
%%                          Module Description                           %%
%%%%%%%%%%%%%%%%%%%%%%%%%%%%%%%%%%%%%%%%%%%%%%%%%%%%%%%%%%%%%%%%%%%%%%%%%%%

    \index{xformslib \textit{(package)}!xformslib.flxyplot \textit{(module)}|(}
\section{Module xformslib.flxyplot}

    \label{xformslib:flxyplot}

xforms-python's functions to manage xyplot objects.

Copyright (C) 2009, 2010  Luca Lazzaroni ``LukenShiro''
e-mail: <\href{mailto:lukenshiro@ngi.it}{lukenshiro@ngi.it}>

This program is free software: you can redistribute it and/or modify
it under the terms of the GNU Lesser General Public License as
published by the Free Software Foundation, version 2.1 of the License.

This program is distributed in the hope that it will be useful,
but WITHOUT ANY WARRANTY; without even the implied warranty of
MERCHANTABILITY or FITNESS FOR A PARTICULAR PURPOSE. See the
GNU Lesser General Public License for more details.

You should have received a copy of the GNU LGPL along with this
program. If not, see <\href{http://www.gnu.org/licenses/}{http://www.gnu.org/licenses/}>.

See CREDITS file to read acknowledgements and thanks to XForms,
ctypes and other developers.

%%%%%%%%%%%%%%%%%%%%%%%%%%%%%%%%%%%%%%%%%%%%%%%%%%%%%%%%%%%%%%%%%%%%%%%%%%%
%%                               Functions                               %%
%%%%%%%%%%%%%%%%%%%%%%%%%%%%%%%%%%%%%%%%%%%%%%%%%%%%%%%%%%%%%%%%%%%%%%%%%%%

  \subsection{Functions}

    \label{xformslib:flxyplot:fl_add_xyplot}
    \index{xformslib \textit{(package)}!xformslib.flxyplot \textit{(module)}!xformslib.flxyplot.fl\_add\_xyplot \textit{(function)}}

    \vspace{0.5ex}

\hspace{.8\funcindent}\begin{boxedminipage}{\funcwidth}

    \raggedright \textbf{fl\_add\_xyplot}(\textit{plottype}, \textit{x}, \textit{y}, \textit{w}, \textit{h}, \textit{label})

    \vspace{-1.5ex}

    \rule{\textwidth}{0.5\fboxrule}
\setlength{\parskip}{2ex}

Adds an xyplot object. It gives an easy way to display a tabulated
function generated on the fly or from an existing data file.

-{}-
\setlength{\parskip}{1ex}
      \textbf{Parameters}
      \vspace{-1ex}

      \begin{quote}
        \begin{Ventry}{xxxxxxxx}

          \item[plottype]


type of xyplot to be added. Values (from xfdata.py)
FL\_NORMAL\_XYPLOT, FL\_SQUARE\_XYPLOT, FL\_CIRCLE\_XYPLOT,
FL\_FILL\_XYPLOT, FL\_POINTS\_XYPLOT, FL\_DASHED\_XYPLOT,
FL\_IMPULSE\_XYPLOT, FL\_ACTIVE\_XYPLOT, FL\_EMPTY\_XYPLOT,
FL\_DOTTED\_XYPLOT, FL\_DOTDASHED\_XYPLOT, FL\_LONGDASHED\_XYPLOT,
FL\_LINEPOINTS\_XYPLOT
            {\it (type=int)}

          \item[x]


horizontal position (upper-left corner)
            {\it (type=int)}

          \item[y]


vertical position (upper-left corner)
            {\it (type=int)}

          \item[w]


width in coord units
            {\it (type=int)}

          \item[h]


height in coord units
            {\it (type=int)}

          \item[label]


text label of xyplot
            {\it (type=str)}

        \end{Ventry}

      \end{quote}

      \textbf{Return Value}
    \vspace{-1ex}

      \begin{quote}

xyplot object added (pFlObject)
      {\it (type=pointer to xfdata.FL\_OBJECT)}

      \end{quote}

\textbf{Note:} 
e.g. \emph{todo}


\textbf{Status:} 
Untested + NoDoc + NoDemo = NOT OK


    \end{boxedminipage}

    \label{xformslib:flxyplot:fl_set_xyplot_data}
    \index{xformslib \textit{(package)}!xformslib.flxyplot \textit{(module)}!xformslib.flxyplot.fl\_set\_xyplot\_data \textit{(function)}}

    \vspace{0.5ex}

\hspace{.8\funcindent}\begin{boxedminipage}{\funcwidth}

    \raggedright \textbf{fl\_set\_xyplot\_data}(\textit{pFlObject}, \textit{xlist}, \textit{ylist}, \textit{npoints}, \textit{title}, \textit{xlabel}, \textit{ylabel})

    \vspace{-1.5ex}

    \rule{\textwidth}{0.5\fboxrule}
\setlength{\parskip}{2ex}

Set or replaces data for a xyplot object, using supplied values. If
the xyplot object being set exists already, old data will be cleared.

-{}-
\setlength{\parskip}{1ex}
      \textbf{Parameters}
      \vspace{-1ex}

      \begin{quote}
        \begin{Ventry}{xxxxxxxxx}

          \item[pFlObject]


xyplot object
            {\it (type=pointer to xfdata.FL\_OBJECT)}

          \item[xlist]


list of values for the tabulated function on the x-axis
            {\it (type=list\_of\_float)}

          \item[ylist]


list of values for the tabulated function on the y-axis
            {\it (type=list\_of\_float)}

          \item[npoints]


number of data points
            {\it (type=int)}

          \item[title]


title drawn above the xyplot
            {\it (type=str)}

          \item[xlabel]


label for values on x-axis
            {\it (type=str)}

          \item[ylabel]


label for values on y-axis
            {\it (type=str)}

        \end{Ventry}

      \end{quote}

\textbf{Note:} 
e.g. \emph{todo}


\textbf{Status:} 
Untested + NoDoc + NoDemo = NOT OK


    \end{boxedminipage}

    \label{xformslib:flxyplot:fl_set_xyplot_data_double}
    \index{xformslib \textit{(package)}!xformslib.flxyplot \textit{(module)}!xformslib.flxyplot.fl\_set\_xyplot\_data\_double \textit{(function)}}

    \vspace{0.5ex}

\hspace{.8\funcindent}\begin{boxedminipage}{\funcwidth}

    \raggedright \textbf{fl\_set\_xyplot\_data\_double}(\textit{pFlObject}, \textit{xlist}, \textit{ylist}, \textit{npoints}, \textit{title}, \textit{xlabel}, \textit{ylabel})

    \vspace{-1.5ex}

    \rule{\textwidth}{0.5\fboxrule}
\setlength{\parskip}{2ex}
%
\begin{description}
\item[{Set or replaces data for a xyplot object, using supplied values. If}] \leavevmode 
the xyplot object being set exists already, old data will be cleared.
It's about the same as fl\_set\_xyplot\_data\_double(), but uses double type
internally.

\end{description}

-{}-
%
\begin{quote}

\end{quote}
\setlength{\parskip}{1ex}
      \textbf{Parameters}
      \vspace{-1ex}

      \begin{quote}
        \begin{Ventry}{xxxxxxxxx}

          \item[pFlObject]


xyplot object
            {\it (type=pointer to xfdata.FL\_OBJECT)}

          \item[xlist]


list of values for the tabulated function on the x-axis
            {\it (type=list\_of\_float)}

          \item[ylist]


list of values for the tabulated function on the y-axis
            {\it (type=list\_of\_float)}

          \item[npoints]


number of data points
            {\it (type=int)}

          \item[title]


title drawn above the xyplot
            {\it (type=str)}

          \item[xlabel]


label for values on x-axis
            {\it (type=str)}

          \item[ylabel]


label for values on y-axis
            {\it (type=str)}

        \end{Ventry}

      \end{quote}

\textbf{Note:} 
e.g. \emph{todo}


\textbf{Status:} 
Untested + NoDoc + NoDemo = NOT OK


    \end{boxedminipage}

    \label{xformslib:flxyplot:fl_set_xyplot_file}
    \index{xformslib \textit{(package)}!xformslib.flxyplot \textit{(module)}!xformslib.flxyplot.fl\_set\_xyplot\_file \textit{(function)}}

    \vspace{0.5ex}

\hspace{.8\funcindent}\begin{boxedminipage}{\funcwidth}

    \raggedright \textbf{fl\_set\_xyplot\_file}(\textit{pFlObject}, \textit{fname}, \textit{title}, \textit{xlabel}, \textit{ylabel})

    \vspace{-1.5ex}

    \rule{\textwidth}{0.5\fboxrule}
\setlength{\parskip}{2ex}

Sets or replaces data for a xyplot object, by loading a tabulated
function from a file. The data file should be an ASCII file consisting
of data lines. Each data line must have two columns, indicating the (x,y)
pair with a space, tab or comma (,) separating the two columns. Lines
that start with any of !, ; or \# are considered to be comments and are
ignored.

-{}-
\setlength{\parskip}{1ex}
      \textbf{Parameters}
      \vspace{-1ex}

      \begin{quote}
        \begin{Ventry}{xxxxxxxxx}

          \item[pFlObject]


xyplot object
            {\it (type=pointer to xfdata.FL\_OBJECT)}

          \item[fname]


name of file.
            {\it (type=str)}

          \item[title]


title of xyplot
            {\it (type=str)}

          \item[xlabel]


label for horizontal values
            {\it (type=str)}

          \item[ylabel]


label for vertical values
            {\it (type=str)}

        \end{Ventry}

      \end{quote}

      \textbf{Return Value}
    \vspace{-1ex}

      \begin{quote}

number of data points successfully read, or 0 (if the file can?t
be opened)
      {\it (type=int)}

      \end{quote}

\textbf{Note:} 
e.g. \emph{todo}


\textbf{Status:} 
Untested + NoDoc + NoDemo = NOT OK


    \end{boxedminipage}

    \label{xformslib:flxyplot:fl_insert_xyplot_data}
    \index{xformslib \textit{(package)}!xformslib.flxyplot \textit{(module)}!xformslib.flxyplot.fl\_insert\_xyplot\_data \textit{(function)}}

    \vspace{0.5ex}

\hspace{.8\funcindent}\begin{boxedminipage}{\funcwidth}

    \raggedright \textbf{fl\_insert\_xyplot\_data}(\textit{pFlObject}, \textit{ovlid}, \textit{idxpt}, \textit{x}, \textit{y})

    \vspace{-1.5ex}

    \rule{\textwidth}{0.5\fboxrule}
\setlength{\parskip}{2ex}

Inserts a point after a supplied index position in a xyplot object.

-{}-
\setlength{\parskip}{1ex}
      \textbf{Parameters}
      \vspace{-1ex}

      \begin{quote}
        \begin{Ventry}{xxxxxxxxx}

          \item[pFlObject]


xyplot object
            {\it (type=pointer to xfdata.FL\_OBJECT)}

          \item[ovlid]


overlay id. Values between 1 and xfdata.FL\_MAX\_XYPLOTOVERLAY or the
number set via fl\_set\_xyplot\_maxoverlays()
            {\it (type=int)}

          \item[idxpt]


the index of the point after which the data new point is to be
inserted.If it's -1 inserts the point in front. To append to the data,
set it to be equal or larger than the return value of
fl\_get\_xyplot\_numdata().
            {\it (type=int)}

          \item[x]


horizontal position of the point
            {\it (type=float)}

          \item[y]


vertical position of the point
            {\it (type=float)}

        \end{Ventry}

      \end{quote}

\textbf{Note:} 
e.g. \emph{todo}


\textbf{Status:} 
Untested + NoDoc + NoDemo = NOT OK


    \end{boxedminipage}

    \label{xformslib:flxyplot:fl_add_xyplot_text}
    \index{xformslib \textit{(package)}!xformslib.flxyplot \textit{(module)}!xformslib.flxyplot.fl\_add\_xyplot\_text \textit{(function)}}

    \vspace{0.5ex}

\hspace{.8\funcindent}\begin{boxedminipage}{\funcwidth}

    \raggedright \textbf{fl\_add\_xyplot\_text}(\textit{pFlObject}, \textit{x}, \textit{y}, \textit{text}, \textit{align}, \textit{colr})

    \vspace{-1.5ex}

    \rule{\textwidth}{0.5\fboxrule}
\setlength{\parskip}{2ex}

Places an inset text on an xyplot object (up to
xfdata.FL\_MAX\_XYPLOTOVERLAY or the value set via
fl\_set\_xyplot\_maxoverlays() of such insets can be accommodated).

-{}-
\setlength{\parskip}{1ex}
      \textbf{Parameters}
      \vspace{-1ex}

      \begin{quote}
        \begin{Ventry}{xxxxxxxxx}

          \item[pFlObject]


xyplot object
            {\it (type=pointer to xfdata.FL\_OBJECT)}

          \item[x]


horizontal coordinates where text is to be placed
            {\it (type=float)}

          \item[y]


vertical coordinates where text is to be placed
            {\it (type=float)}

          \item[text]


text to be added to xyplot. If it starts with '@', a symbol is drawn.
            {\it (type=str)}

          \item[align]


alignment of text. Values (from xfdata.py) FL\_ALIGN\_CENTER,
FL\_ALIGN\_TOP, FL\_ALIGN\_BOTTOM, FL\_ALIGN\_LEFT, FL\_ALIGN\_RIGHT,
FL\_ALIGN\_LEFT\_TOP, FL\_ALIGN\_RIGHT\_TOP, FL\_ALIGN\_LEFT\_BOTTOM,
FL\_ALIGN\_RIGHT\_BOTTOM, FL\_ALIGN\_INSIDE, FL\_ALIGN\_VERT.
Bitwise OR with FL\_ALIGN\_INSIDE is allowed.
            {\it (type=int)}

          \item[colr]


color value
            {\it (type=long\_pos)}

        \end{Ventry}

      \end{quote}

\textbf{Note:} 
e.g. \emph{todo}


\textbf{Status:} 
Untested + NoDoc + NoDemo = NOT OK


    \end{boxedminipage}

    \label{xformslib:flxyplot:fl_delete_xyplot_text}
    \index{xformslib \textit{(package)}!xformslib.flxyplot \textit{(module)}!xformslib.flxyplot.fl\_delete\_xyplot\_text \textit{(function)}}

    \vspace{0.5ex}

\hspace{.8\funcindent}\begin{boxedminipage}{\funcwidth}

    \raggedright \textbf{fl\_delete\_xyplot\_text}(\textit{pFlObject}, \textit{text})

    \vspace{-1.5ex}

    \rule{\textwidth}{0.5\fboxrule}
\setlength{\parskip}{2ex}

Removes an inset text from a xyplot object.

-{}-
\setlength{\parskip}{1ex}
      \textbf{Parameters}
      \vspace{-1ex}

      \begin{quote}
        \begin{Ventry}{xxxxxxxxx}

          \item[pFlObject]


xyplot object
            {\it (type=pointer to xfdata.FL\_OBJECT)}

          \item[text]


text to be deleted from xyplot
            {\it (type=str)}

        \end{Ventry}

      \end{quote}

\textbf{Note:} 
e.g. \emph{todo}


\textbf{Status:} 
Untested + NoDoc + NoDemo = NOT OK


    \end{boxedminipage}

    \label{xformslib:flxyplot:fl_set_xyplot_maxoverlays}
    \index{xformslib \textit{(package)}!xformslib.flxyplot \textit{(module)}!xformslib.flxyplot.fl\_set\_xyplot\_maxoverlays \textit{(function)}}

    \vspace{0.5ex}

\hspace{.8\funcindent}\begin{boxedminipage}{\funcwidth}

    \raggedright \textbf{fl\_set\_xyplot\_maxoverlays}(\textit{pFlObject}, \textit{numovl})

    \vspace{-1.5ex}

    \rule{\textwidth}{0.5\fboxrule}
\setlength{\parskip}{2ex}

Changes the maximum number of overlays an object can have. By default,
it is 32.

-{}-
\setlength{\parskip}{1ex}
      \textbf{Parameters}
      \vspace{-1ex}

      \begin{quote}
        \begin{Ventry}{xxxxxxxxx}

          \item[pFlObject]


xyplot object
            {\it (type=pointer to xfdata.FL\_OBJECT)}

          \item[numovl]


maximum number of overlays.
            {\it (type=int)}

        \end{Ventry}

      \end{quote}

      \textbf{Return Value}
    \vspace{-1ex}

      \begin{quote}

previous maximum number of overlays
      {\it (type=int)}

      \end{quote}

\textbf{Note:} 
e.g. \emph{todo}


\textbf{Status:} 
Untested + NoDoc + NoDemo = NOT OK


    \end{boxedminipage}

    \label{xformslib:flxyplot:fl_add_xyplot_overlay}
    \index{xformslib \textit{(package)}!xformslib.flxyplot \textit{(module)}!xformslib.flxyplot.fl\_add\_xyplot\_overlay \textit{(function)}}

    \vspace{0.5ex}

\hspace{.8\funcindent}\begin{boxedminipage}{\funcwidth}

    \raggedright \textbf{fl\_add\_xyplot\_overlay}(\textit{pFlObject}, \textit{ovlid}, \textit{x}, \textit{y}, \textit{npoints}, \textit{colr})

    \vspace{-1.5ex}

    \rule{\textwidth}{0.5\fboxrule}
\setlength{\parskip}{2ex}

Overlay several plots together.

-{}-
\setlength{\parskip}{1ex}
      \textbf{Parameters}
      \vspace{-1ex}

      \begin{quote}
        \begin{Ventry}{xxxxxxxxx}

          \item[pFlObject]


xyplot object
            {\it (type=pointer to xfdata.FL\_OBJECT)}

          \item[ovlid]


overlay id. Values between 1 and xfdata.FL\_MAX\_XYPLOTOVERLAY or the
number set via fl\_set\_xyplot\_maxoverlays()
            {\it (type=int)}

          \item[x]


horizontal position
            {\it (type=float)}

          \item[y]


vertical position
            {\it (type=float)}

          \item[npoints]


number of data points.
            {\it (type=int)}

          \item[colr]


color value
            {\it (type=long\_pos)}

        \end{Ventry}

      \end{quote}

\textbf{Note:} 
e.g. \emph{todo}


\textbf{Status:} 
Untested + NoDoc + NoDemo = NOT OK


    \end{boxedminipage}

    \label{xformslib:flxyplot:fl_add_xyplot_overlay_file}
    \index{xformslib \textit{(package)}!xformslib.flxyplot \textit{(module)}!xformslib.flxyplot.fl\_add\_xyplot\_overlay\_file \textit{(function)}}

    \vspace{0.5ex}

\hspace{.8\funcindent}\begin{boxedminipage}{\funcwidth}

    \raggedright \textbf{fl\_add\_xyplot\_overlay\_file}(\textit{pFlObject}, \textit{ovlid}, \textit{fname}, \textit{colr})

    \vspace{-1.5ex}

    \rule{\textwidth}{0.5\fboxrule}
\setlength{\parskip}{2ex}

Adds an overlay, using a data file to specify the (x,y) function for
the base data.

-{}-
\setlength{\parskip}{1ex}
      \textbf{Parameters}
      \vspace{-1ex}

      \begin{quote}
        \begin{Ventry}{xxxxxxxxx}

          \item[pFlObject]


xyplot object
            {\it (type=pointer to xfdata.FL\_OBJECT)}

          \item[ovlid]


overlay id. Values between 1 and xfdata.FL\_MAX\_XYPLOTOVERLAY or the
number set via fl\_set\_xyplot\_maxoverlays()
            {\it (type=int)}

          \item[fname]


name of file
            {\it (type=str)}

          \item[colr]


color value
            {\it (type=long\_pos)}

        \end{Ventry}

      \end{quote}

      \textbf{Return Value}
    \vspace{-1ex}

      \begin{quote}

number of data points successfully read
      {\it (type=int)}

      \end{quote}

\textbf{Note:} 
e.g. \emph{todo}


\textbf{Status:} 
Untested + NoDoc + NoDemo = NOT OK


    \end{boxedminipage}

    \label{xformslib:flxyplot:fl_set_xyplot_xtics}
    \index{xformslib \textit{(package)}!xformslib.flxyplot \textit{(module)}!xformslib.flxyplot.fl\_set\_xyplot\_xtics \textit{(function)}}

    \vspace{0.5ex}

\hspace{.8\funcindent}\begin{boxedminipage}{\funcwidth}

    \raggedright \textbf{fl\_set\_xyplot\_xtics}(\textit{pFlObject}, \textit{major}, \textit{minor})

    \vspace{-1.5ex}

    \rule{\textwidth}{0.5\fboxrule}
\setlength{\parskip}{2ex}

Changes the number of tic marks of a xyplot object on x-axis. The
actual scaling routine may choose a value other than that requested if it
decides that this would make the plot look nicer, thus major and minor
are only taken as a hint to the scaling routine. However, in almost all
cases the scaling routine will not generate a major that differs from the
requested value by more than 3. fl\_set\_xyplot\_alphaxtics can?t be active
at the same time and the one that gets used is the one that was set last.

-{}-
\setlength{\parskip}{1ex}
      \textbf{Parameters}
      \vspace{-1ex}

      \begin{quote}
        \begin{Ventry}{xxxxxxxxx}

          \item[pFlObject]


xyplot object
            {\it (type=pointer to xfdata.FL\_OBJECT)}

          \item[major]


number of tic marks to be placed on the plot. If it's -1 suppresses
the tic marks completely, if it's 0 restores the default settings.
            {\it (type=int)}

          \item[minor]


number of divisions between major tic marks. If it's -1 suppresses
the tic marks completely, if it's 0 restores the default settings.
            {\it (type=int)}

        \end{Ventry}

      \end{quote}

\textbf{Note:} 
e.g. \emph{todo}


\textbf{Status:} 
Untested + NoDoc + NoDemo = NOT OK


    \end{boxedminipage}

    \label{xformslib:flxyplot:fl_set_xyplot_ytics}
    \index{xformslib \textit{(package)}!xformslib.flxyplot \textit{(module)}!xformslib.flxyplot.fl\_set\_xyplot\_ytics \textit{(function)}}

    \vspace{0.5ex}

\hspace{.8\funcindent}\begin{boxedminipage}{\funcwidth}

    \raggedright \textbf{fl\_set\_xyplot\_ytics}(\textit{pFlObject}, \textit{major}, \textit{minor})

    \vspace{-1.5ex}

    \rule{\textwidth}{0.5\fboxrule}
\setlength{\parskip}{2ex}

Changes the number of tic marks of a xyplot object on y-axis. The
actual scaling routine may choose a value other than that requested if it
decides that this would make the plot look nicer, thus major and minor
are only taken as a hint to the scaling routine. However, in almost all
cases the scaling routine will not generate a major that differs from the
requested value by more than 3. fl\_set\_xyplot\_ytics can?t be active at the
same time and the one that gets used is the one that was set last.

-{}-
\setlength{\parskip}{1ex}
      \textbf{Parameters}
      \vspace{-1ex}

      \begin{quote}
        \begin{Ventry}{xxxxxxxxx}

          \item[pFlObject]


xyplot object
            {\it (type=pointer to xfdata.FL\_OBJECT)}

          \item[major]


number of tic marks to be placed on the plot. If it's -1 suppresses
the tic marks completely, if it's 0 restores the default settings.
            {\it (type=int)}

          \item[minor]


number of divisions between major tic marks. If it's -1 suppresses
the tic marks completely, if it's 0 restores the default settings.
            {\it (type=int)}

        \end{Ventry}

      \end{quote}

\textbf{Note:} 
e.g. \emph{todo}


\textbf{Status:} 
Untested + NoDoc + NoDemo = NOT OK


    \end{boxedminipage}

    \label{xformslib:flxyplot:fl_set_xyplot_xbounds}
    \index{xformslib \textit{(package)}!xformslib.flxyplot \textit{(module)}!xformslib.flxyplot.fl\_set\_xyplot\_xbounds \textit{(function)}}

    \vspace{0.5ex}

\hspace{.8\funcindent}\begin{boxedminipage}{\funcwidth}

    \raggedright \textbf{fl\_set\_xyplot\_xbounds}(\textit{pFlObject}, \textit{minbound}, \textit{maxbound})

    \vspace{-1.5ex}

    \rule{\textwidth}{0.5\fboxrule}
\setlength{\parskip}{2ex}

Sets and uses absolute bounds/limits on x-axis of a xyplot object as
opposed to actual bounds in data.

-{}-
\setlength{\parskip}{1ex}
      \textbf{Parameters}
      \vspace{-1ex}

      \begin{quote}
        \begin{Ventry}{xxxxxxxxx}

          \item[pFlObject]


xyplot object
            {\it (type=pointer to xfdata.FL\_OBJECT)}

          \item[minbound]


minimum bound to set
            {\it (type=float)}

          \item[maxbound]


maximum bound to set
            {\it (type=float)}

        \end{Ventry}

      \end{quote}

\textbf{Note:} 
e.g. \emph{todo}


\textbf{Status:} 
Untested + NoDoc + NoDemo = NOT OK


    \end{boxedminipage}

    \label{xformslib:flxyplot:fl_set_xyplot_ybounds}
    \index{xformslib \textit{(package)}!xformslib.flxyplot \textit{(module)}!xformslib.flxyplot.fl\_set\_xyplot\_ybounds \textit{(function)}}

    \vspace{0.5ex}

\hspace{.8\funcindent}\begin{boxedminipage}{\funcwidth}

    \raggedright \textbf{fl\_set\_xyplot\_ybounds}(\textit{pFlObject}, \textit{minbound}, \textit{maxbound})

    \vspace{-1.5ex}

    \rule{\textwidth}{0.5\fboxrule}
\setlength{\parskip}{2ex}

Sets and uses absolute bounds/limits on y-axis of a xyplot object as
opposed to actual bounds in data.

-{}-
\setlength{\parskip}{1ex}
      \textbf{Parameters}
      \vspace{-1ex}

      \begin{quote}
        \begin{Ventry}{xxxxxxxxx}

          \item[pFlObject]


xyplot object
            {\it (type=pointer to xfdata.FL\_OBJECT)}

          \item[minbound]


minimum bound to set
            {\it (type=float)}

          \item[maxbound]


maximum bound to set
            {\it (type=float)}

        \end{Ventry}

      \end{quote}

\textbf{Note:} 
e.g. \emph{todo}


\textbf{Status:} 
Untested + NoDoc + NoDemo = NOT OK


    \end{boxedminipage}

    \label{xformslib:flxyplot:fl_get_xyplot_xbounds}
    \index{xformslib \textit{(package)}!xformslib.flxyplot \textit{(module)}!xformslib.flxyplot.fl\_get\_xyplot\_xbounds \textit{(function)}}

    \vspace{0.5ex}

\hspace{.8\funcindent}\begin{boxedminipage}{\funcwidth}

    \raggedright \textbf{fl\_get\_xyplot\_xbounds}(\textit{pFlObject})

    \vspace{-1.5ex}

    \rule{\textwidth}{0.5\fboxrule}
\setlength{\parskip}{2ex}

Obtains the current bounds/limits for x-axis of a xyplot object. The
bounds returned are the bounds used in clipping the data, which are not
necessarily the bounds used in computing the world/screen mapping due to
tic rounding.

-{}-
\setlength{\parskip}{1ex}
      \textbf{Parameters}
      \vspace{-1ex}

      \begin{quote}
        \begin{Ventry}{xxxxxxxxx}

          \item[pFlObject]


xyplot object
            {\it (type=pointer to xfdata.FL\_OBJECT)}

        \end{Ventry}

      \end{quote}

      \textbf{Return Value}
    \vspace{-1ex}

      \begin{quote}

minimum bound, maximum bound
      {\it (type=float, float)}

      \end{quote}

\textbf{Note:} 
e.g. \emph{todo}


\textbf{Attention:} 
API change from XForms - upstream was
fl\_get\_xyplot\_xbounds(pFlObject, minbound, maxbound)


\textbf{Status:} 
Untested + NoDoc + NoDemo = NOT OK


    \end{boxedminipage}

    \label{xformslib:flxyplot:fl_get_xyplot_ybounds}
    \index{xformslib \textit{(package)}!xformslib.flxyplot \textit{(module)}!xformslib.flxyplot.fl\_get\_xyplot\_ybounds \textit{(function)}}

    \vspace{0.5ex}

\hspace{.8\funcindent}\begin{boxedminipage}{\funcwidth}

    \raggedright \textbf{fl\_get\_xyplot\_ybounds}(\textit{pFlObject})

    \vspace{-1.5ex}

    \rule{\textwidth}{0.5\fboxrule}
\setlength{\parskip}{2ex}

Obtains the current bounds/limits for y-axis of a xyplot object. The
bounds returned are the bounds used in clipping the data, which are not
necessarily the bounds used in computing the world/screen mapping due to
tic rounding.

-{}-
\setlength{\parskip}{1ex}
      \textbf{Parameters}
      \vspace{-1ex}

      \begin{quote}
        \begin{Ventry}{xxxxxxxxx}

          \item[pFlObject]


xyplot object
            {\it (type=pointer to xfdata.FL\_OBJECT)}

        \end{Ventry}

      \end{quote}

      \textbf{Return Value}
    \vspace{-1ex}

      \begin{quote}

minimum bound, maximum bound
      {\it (type=float, float)}

      \end{quote}

\textbf{Note:} 
e.g. \emph{todo}


\textbf{Attention:} 
API change from XForms - upstream was
fl\_get\_xyplot\_ybounds(pFlObject, minbound, maxbound)


\textbf{Status:} 
Untested + NoDoc + NoDemo = NOT OK


    \end{boxedminipage}

    \label{xformslib:flxyplot:fl_get_xyplot}
    \index{xformslib \textit{(package)}!xformslib.flxyplot \textit{(module)}!xformslib.flxyplot.fl\_get\_xyplot \textit{(function)}}

    \vspace{0.5ex}

\hspace{.8\funcindent}\begin{boxedminipage}{\funcwidth}

    \raggedright \textbf{fl\_get\_xyplot}(\textit{pFlObject})

    \vspace{-1.5ex}

    \rule{\textwidth}{0.5\fboxrule}
\setlength{\parskip}{2ex}

Obtains the current value of the point of a xyplot object that has
changed.

-{}-
\setlength{\parskip}{1ex}
      \textbf{Parameters}
      \vspace{-1ex}

      \begin{quote}
        \begin{Ventry}{xxxxxxxxx}

          \item[pFlObject]


xyplot object
            {\it (type=pointer to xfdata.FL\_OBJECT)}

        \end{Ventry}

      \end{quote}

      \textbf{Return Value}
    \vspace{-1ex}

      \begin{quote}

horizontal position of data point (x), vertical position of data
point (y), the data index starting from 0 (i) or -1 (if no point is
changed)
      {\it (type=float, float, int)}

      \end{quote}

\textbf{Note:} 
e.g. \emph{todo}


\textbf{Attention:} 
API change from XForms - upstream was
fl\_get\_xyplot(pFlObject, x, y, i)


\textbf{Status:} 
Untested + NoDoc + NoDemo = NOT OK


    \end{boxedminipage}

    \label{xformslib:flxyplot:fl_get_xyplot_data}
    \index{xformslib \textit{(package)}!xformslib.flxyplot \textit{(module)}!xformslib.flxyplot.fl\_get\_xyplot\_data \textit{(function)}}

    \vspace{0.5ex}

\hspace{.8\funcindent}\begin{boxedminipage}{\funcwidth}

    \raggedright \textbf{fl\_get\_xyplot\_data}(\textit{pFlObject})

    \vspace{-1.5ex}

    \rule{\textwidth}{0.5\fboxrule}
\setlength{\parskip}{2ex}

Obtains a copy of the current xyplot data.

-{}-
\setlength{\parskip}{1ex}
      \textbf{Parameters}
      \vspace{-1ex}

      \begin{quote}
        \begin{Ventry}{xxxxxxxxx}

          \item[pFlObject]


xyplot object
            {\it (type=pointer to xfdata.FL\_OBJECT)}

        \end{Ventry}

      \end{quote}

      \textbf{Return Value}
    \vspace{-1ex}

      \begin{quote}

list of x-axis values?, list of y-axis values?, number of data
points
      {\it (type=float, float, int)}

      \end{quote}

\textbf{Note:} 
e.g. \emph{todo}


\textbf{Attention:} 
API change from XForms - upstream was
fl\_get\_xyplot\_data(pFlObject, x, y, n)


\textbf{Status:} 
Untested + NoDoc + NoDemo = NOT OK


    \end{boxedminipage}

    \label{xformslib:flxyplot:fl_get_xyplot_data_pointer}
    \index{xformslib \textit{(package)}!xformslib.flxyplot \textit{(module)}!xformslib.flxyplot.fl\_get\_xyplot\_data\_pointer \textit{(function)}}

    \vspace{0.5ex}

\hspace{.8\funcindent}\begin{boxedminipage}{\funcwidth}

    \raggedright \textbf{fl\_get\_xyplot\_data\_pointer}(\textit{pFlObject}, \textit{ovlid})

    \vspace{-1.5ex}

    \rule{\textwidth}{0.5\fboxrule}
\setlength{\parskip}{2ex}

Obtains the pointer to the data of xyplot object rather (instead of a
copy of the data).

-{}-
\setlength{\parskip}{1ex}
      \textbf{Parameters}
      \vspace{-1ex}

      \begin{quote}
        \begin{Ventry}{xxxxxxxxx}

          \item[pFlObject]


xyplot object
            {\it (type=pointer to xfdata.FL\_OBJECT)}

          \item[ovlid]


overlay id. Values between 1 and xfdata.FL\_MAX\_XYPLOTOVERLAY or the
number set via fl\_set\_xyplot\_maxoverlays()
            {\it (type=int)}

        \end{Ventry}

      \end{quote}

      \textbf{Return Value}
    \vspace{-1ex}

      \begin{quote}

pointer to list of x-axis values?, pointer to list of y-axis
values?, number of data points
      {\it (type=list\_of\_float?, list\_of\_float?, int)}

      \end{quote}

\textbf{Note:} 
e.g. \emph{todo}


\textbf{Attention:} 
API change from XForms - upstream was
fl\_get\_xyplot\_data\_pointer(pFlObject, ovlid, x, y, n)


\textbf{Status:} 
Untested + NoDoc + NoDemo = NOT OK


    \end{boxedminipage}

    \label{xformslib:flxyplot:fl_get_xyplot_overlay_data}
    \index{xformslib \textit{(package)}!xformslib.flxyplot \textit{(module)}!xformslib.flxyplot.fl\_get\_xyplot\_overlay\_data \textit{(function)}}

    \vspace{0.5ex}

\hspace{.8\funcindent}\begin{boxedminipage}{\funcwidth}

    \raggedright \textbf{fl\_get\_xyplot\_overlay\_data}(\textit{pFlObject}, \textit{ovlid})

    \vspace{-1.5ex}

    \rule{\textwidth}{0.5\fboxrule}
\setlength{\parskip}{2ex}

Obtains the current data of an overlay of a xyplot object.

-{}-
\setlength{\parskip}{1ex}
      \textbf{Parameters}
      \vspace{-1ex}

      \begin{quote}
        \begin{Ventry}{xxxxxxxxx}

          \item[pFlObject]


xyplot object
            {\it (type=pointer to xfdata.FL\_OBJECT)}

          \item[ovlid]


overlay id. Values between 1 and xfdata.FL\_MAX\_XYPLOTOVERLAY or the
number set via fl\_set\_xyplot\_maxoverlays(). If it's 0 uses the base
dataset
            {\it (type=int)}

        \end{Ventry}

      \end{quote}

      \textbf{Return Value}
    \vspace{-1ex}

      \begin{quote}

x-axis value, y-axis value, number of data points (npoints)
      {\it (type=float, float, int)}

      \end{quote}

\textbf{Note:} 
e.g. \emph{todo}


\textbf{Attention:} 
API change from XForms - upstream was
fl\_get\_xyplot\_overlay\_data(pFlObject, ovlid, x, y, n)


\textbf{Status:} 
Untested + NoDoc + NoDemo = NOT OK


    \end{boxedminipage}

    \label{xformslib:flxyplot:fl_set_xyplot_overlay_type}
    \index{xformslib \textit{(package)}!xformslib.flxyplot \textit{(module)}!xformslib.flxyplot.fl\_set\_xyplot\_overlay\_type \textit{(function)}}

    \vspace{0.5ex}

\hspace{.8\funcindent}\begin{boxedminipage}{\funcwidth}

    \raggedright \textbf{fl\_set\_xyplot\_overlay\_type}(\textit{pFlObject}, \textit{ovlid}, \textit{plottype})

    \vspace{-1.5ex}

    \rule{\textwidth}{0.5\fboxrule}
\setlength{\parskip}{2ex}

Changes the type for an overlay of a xyplot object. The type used in
overlay plot is the same as the object itself. Note that although the API
of adding an overlay is similar to adding an object, an xyplot overlay is
not a separate object, it is simply a property of an already existing
xyplot.

-{}-
\setlength{\parskip}{1ex}
      \textbf{Parameters}
      \vspace{-1ex}

      \begin{quote}
        \begin{Ventry}{xxxxxxxxx}

          \item[pFlObject]


xyplot object
            {\it (type=pointer to xfdata.FL\_OBJECT)}

          \item[ovlid]


overlay id. Values between 1 and xfdata.FL\_MAX\_XYPLOTOVERLAY or the
number set via fl\_set\_xyplot\_maxoverlays()
            {\it (type=int)}

          \item[plottype]


type of xyplot.  Values (from xfdata.py) FL\_NORMAL\_XYPLOT,
FL\_SQUARE\_XYPLOT, FL\_CIRCLE\_XYPLOT, FL\_FILL\_XYPLOT,
FL\_POINTS\_XYPLOT, FL\_DASHED\_XYPLOT, FL\_IMPULSE\_XYPLOT,
FL\_ACTIVE\_XYPLOT, FL\_EMPTY\_XYPLOT, FL\_DOTTED\_XYPLOT,
FL\_DOTDASHED\_XYPLOT, FL\_LONGDASHED\_XYPLOT, FL\_LINEPOINTS\_XYPLOT
            {\it (type=int)}

        \end{Ventry}

      \end{quote}

\textbf{Note:} 
e.g. \emph{todo}


\textbf{Status:} 
Untested + NoDoc + NoDemo = NOT OK


    \end{boxedminipage}

    \label{xformslib:flxyplot:fl_delete_xyplot_overlay}
    \index{xformslib \textit{(package)}!xformslib.flxyplot \textit{(module)}!xformslib.flxyplot.fl\_delete\_xyplot\_overlay \textit{(function)}}

    \vspace{0.5ex}

\hspace{.8\funcindent}\begin{boxedminipage}{\funcwidth}

    \raggedright \textbf{fl\_delete\_xyplot\_overlay}(\textit{pFlObject}, \textit{ovlid})

    \vspace{-1.5ex}

    \rule{\textwidth}{0.5\fboxrule}
\setlength{\parskip}{2ex}

Deletes an overlay of a xyplot object.

-{}-
\setlength{\parskip}{1ex}
      \textbf{Parameters}
      \vspace{-1ex}

      \begin{quote}
        \begin{Ventry}{xxxxxxxxx}

          \item[pFlObject]


xyplot object
            {\it (type=pointer to xfdata.FL\_OBJECT)}

          \item[ovlid]


overlay id. Values between 1 and xfdata.FL\_MAX\_XYPLOTOVERLAY or the
number set via fl\_set\_xyplot\_maxoverlays()
            {\it (type=int)}

        \end{Ventry}

      \end{quote}

\textbf{Note:} 
e.g. \emph{todo}


\textbf{Status:} 
Untested + NoDoc + NoDemo = NOT OK


    \end{boxedminipage}

    \label{xformslib:flxyplot:fl_set_xyplot_interpolate}
    \index{xformslib \textit{(package)}!xformslib.flxyplot \textit{(module)}!xformslib.flxyplot.fl\_set\_xyplot\_interpolate \textit{(function)}}

    \vspace{0.5ex}

\hspace{.8\funcindent}\begin{boxedminipage}{\funcwidth}

    \raggedright \textbf{fl\_set\_xyplot\_interpolate}(\textit{pFlObject}, \textit{ovlid}, \textit{deg}, \textit{grid})

    \vspace{-1.5ex}

    \rule{\textwidth}{0.5\fboxrule}
\setlength{\parskip}{2ex}

Interpolates xyplot data using an nth order Lagrangian polynomial.

-{}-
\setlength{\parskip}{1ex}
      \textbf{Parameters}
      \vspace{-1ex}

      \begin{quote}
        \begin{Ventry}{xxxxxxxxx}

          \item[pFlObject]


xyplot object
            {\it (type=pointer to xfdata.FL\_OBJECT)}

          \item[ovlid]


overlay id. Values between 1 and xfdata.FL\_MAX\_XYPLOTOVERLAY or the
number set via fl\_set\_xyplot\_maxoverlays(). If it's 0 uses the base
data set
            {\it (type=int)}

          \item[deg]


the order of the polynomial to use. If it's 0 or 1, restores the
default linear interpolation.
            {\it (type=int)}

          \item[grid]


the working grid onto which the data are to be interpolated.
            {\it (type=float)}

        \end{Ventry}

      \end{quote}

\textbf{Note:} 
e.g. \emph{todo}


\textbf{Status:} 
Untested + NoDoc + NoDemo = NOT OK


    \end{boxedminipage}

    \label{xformslib:flxyplot:fl_set_xyplot_inspect}
    \index{xformslib \textit{(package)}!xformslib.flxyplot \textit{(module)}!xformslib.flxyplot.fl\_set\_xyplot\_inspect \textit{(function)}}

    \vspace{0.5ex}

\hspace{.8\funcindent}\begin{boxedminipage}{\funcwidth}

    \raggedright \textbf{fl\_set\_xyplot\_inspect}(\textit{pFlObject}, \textit{yesno})

    \vspace{-1.5ex}

    \rule{\textwidth}{0.5\fboxrule}
\setlength{\parskip}{2ex}

Makes aware or not xyplot objects of mouse clicks. Once an XYPlot is in
inspect mode, whenever the mouse is released and the mouse position is on
one of the data point, the object is returned to the caller or its
callback is invoked. You can use fl\_get\_xyplot() to find out which point
the mouse was clicked on.

-{}-
\setlength{\parskip}{1ex}
      \textbf{Parameters}
      \vspace{-1ex}

      \begin{quote}
        \begin{Ventry}{xxxxxxxxx}

          \item[pFlObject]


xyplot object
            {\it (type=pointer to xfdata.FL\_OBJECT)}

          \item[yesno]


flag to enable/disable inspect mode. Values 0 (disabled) or 1 (enabled)
            {\it (type=int)}

        \end{Ventry}

      \end{quote}

\textbf{Note:} 
e.g. \emph{todo}


\textbf{Status:} 
Untested + NoDoc + NoDemo = NOT OK


    \end{boxedminipage}

    \label{xformslib:flxyplot:fl_set_xyplot_symbolsize}
    \index{xformslib \textit{(package)}!xformslib.flxyplot \textit{(module)}!xformslib.flxyplot.fl\_set\_xyplot\_symbolsize \textit{(function)}}

    \vspace{0.5ex}

\hspace{.8\funcindent}\begin{boxedminipage}{\funcwidth}

    \raggedright \textbf{fl\_set\_xyplot\_symbolsize}(\textit{pFlObject}, \textit{symsize})

    \vspace{-1.5ex}

    \rule{\textwidth}{0.5\fboxrule}
\setlength{\parskip}{2ex}

Changes the size of the symbols drawn at data points of a xyplot
object. By default it is 4.

-{}-
\setlength{\parskip}{1ex}
      \textbf{Parameters}
      \vspace{-1ex}

      \begin{quote}
        \begin{Ventry}{xxxxxxxxx}

          \item[pFlObject]


xyplot object
            {\it (type=pointer to xfdata.FL\_OBJECT)}

          \item[symsize]


size of symbol in pixel
            {\it (type=int)}

        \end{Ventry}

      \end{quote}

\textbf{Note:} 
e.g. \emph{todo}


\textbf{Status:} 
Untested + NoDoc + NoDemo = NOT OK


    \end{boxedminipage}

    \label{xformslib:flxyplot:fl_replace_xyplot_point}
    \index{xformslib \textit{(package)}!xformslib.flxyplot \textit{(module)}!xformslib.flxyplot.fl\_replace\_xyplot\_point \textit{(function)}}

    \vspace{0.5ex}

\hspace{.8\funcindent}\begin{boxedminipage}{\funcwidth}

    \raggedright \textbf{fl\_replace\_xyplot\_point}(\textit{pFlObject}, \textit{idxpt}, \textit{x}, \textit{y})

    \vspace{-1.5ex}

    \rule{\textwidth}{0.5\fboxrule}
\setlength{\parskip}{2ex}

Replaces the value of a particular point of a xyplot object. It acts
on the first dataset only.

-{}-
\setlength{\parskip}{1ex}
      \textbf{Parameters}
      \vspace{-1ex}

      \begin{quote}
        \begin{Ventry}{xxxxxxxxx}

          \item[pFlObject]


xyplot object
            {\it (type=pointer to xfdata.FL\_OBJECT)}

          \item[idxpt]


index of the value to be replaced. The first value has an index of 0.
            {\it (type=int)}

          \item[x]


new horizontal position
            {\it (type=float)}

          \item[y]


new vertical position
            {\it (type=float)}

        \end{Ventry}

      \end{quote}

\textbf{Note:} 
e.g. \emph{todo}


\textbf{Status:} 
Untested + NoDoc + NoDemo = NOT OK


    \end{boxedminipage}

    \label{xformslib:flxyplot:fl_replace_xyplot_point_in_overlay}
    \index{xformslib \textit{(package)}!xformslib.flxyplot \textit{(module)}!xformslib.flxyplot.fl\_replace\_xyplot\_point\_in\_overlay \textit{(function)}}

    \vspace{0.5ex}

\hspace{.8\funcindent}\begin{boxedminipage}{\funcwidth}

    \raggedright \textbf{fl\_replace\_xyplot\_point\_in\_overlay}(\textit{pFlObject}, \textit{idxpt}, \textit{setID}, \textit{x}, \textit{y})

    \vspace{-1.5ex}

    \rule{\textwidth}{0.5\fboxrule}
\setlength{\parskip}{2ex}

Replaces the value of a particular point in specified dataset. This
routine is an extension of fl\_replace\_xyplot\_point() for more than one
dataset.

-{}-
\setlength{\parskip}{1ex}
      \textbf{Parameters}
      \vspace{-1ex}

      \begin{quote}
        \begin{Ventry}{xxxxxxxxx}

          \item[pFlObject]


xyplot object
            {\it (type=pointer to xfdata.FL\_OBJECT)}

          \item[idxpt]


index of the value to be replaced. The first value has an index of 0.
            {\it (type=int)}

          \item[setID]


dataset the points belongs to. e.g. the first dsata set is 0.
            {\it (type=int)}

          \item[x]


new horizontal position
            {\it (type=float)}

          \item[y]


new vertical position
            {\it (type=float)}

        \end{Ventry}

      \end{quote}

\textbf{Note:} 
e.g. \emph{todo}


\textbf{Status:} 
Untested + NoDoc + NoDemo = NOT OK


    \end{boxedminipage}

    \label{xformslib:flxyplot:fl_get_xyplot_xmapping}
    \index{xformslib \textit{(package)}!xformslib.flxyplot \textit{(module)}!xformslib.flxyplot.fl\_get\_xyplot\_xmapping \textit{(function)}}

    \vspace{0.5ex}

\hspace{.8\funcindent}\begin{boxedminipage}{\funcwidth}

    \raggedright \textbf{fl\_get\_xyplot\_xmapping}(\textit{pFlObject})

    \vspace{-1.5ex}

    \rule{\textwidth}{0.5\fboxrule}
\setlength{\parskip}{2ex}

Obtains the mapping between the screen coordinates and data on x-axis.
Mapping constants are used as follows
screenCoord = a * data + b       (linear scale)
screenCoord = a * math.log(data) / math.log(p) + b (log scale)
where p is the base of the requested logarithm.

-{}-
\setlength{\parskip}{1ex}
      \textbf{Parameters}
      \vspace{-1ex}

      \begin{quote}
        \begin{Ventry}{xxxxxxxxx}

          \item[pFlObject]


xyplot object
            {\it (type=pointer to xfdata.FL\_OBJECT)}

        \end{Ventry}

      \end{quote}

      \textbf{Return Value}
    \vspace{-1ex}

      \begin{quote}

first mapping constant, second mapping constant
      {\it (type=float, float)}

      \end{quote}

\textbf{Note:} 
e.g. \emph{todo}


\textbf{Attention:} 
API change from XForms - upstream was
fl\_get\_xyplot\_xmapping(pFlObject, a, b)


\textbf{Status:} 
Untested + NoDoc + NoDemo = NOT OK


    \end{boxedminipage}

    \label{xformslib:flxyplot:fl_get_xyplot_ymapping}
    \index{xformslib \textit{(package)}!xformslib.flxyplot \textit{(module)}!xformslib.flxyplot.fl\_get\_xyplot\_ymapping \textit{(function)}}

    \vspace{0.5ex}

\hspace{.8\funcindent}\begin{boxedminipage}{\funcwidth}

    \raggedright \textbf{fl\_get\_xyplot\_ymapping}(\textit{pFlObject})

    \vspace{-1.5ex}

    \rule{\textwidth}{0.5\fboxrule}
\setlength{\parskip}{2ex}

Obtains the mapping between the screen coordinates and data on y-axis.
Mapping constants are used as follows
screenCoord = a * data + b       (linear scale)
screenCoord = a * math.log(data) / math.log(p) + b (log scale)
where p is the base of the requested logarithm.

-{}-
\setlength{\parskip}{1ex}
      \textbf{Parameters}
      \vspace{-1ex}

      \begin{quote}
        \begin{Ventry}{xxxxxxxxx}

          \item[pFlObject]


xyplot object
            {\it (type=pointer to xfdata.FL\_OBJECT)}

        \end{Ventry}

      \end{quote}

      \textbf{Return Value}
    \vspace{-1ex}

      \begin{quote}

first mapping constant, second mapping constant
      {\it (type=float, float)}

      \end{quote}

\textbf{Note:} 
e.g. \emph{todo}


\textbf{Attention:} 
API change from XForms - upstream was
fl\_get\_xyplot\_ymapping(pFlObject, a, b)


\textbf{Status:} 
Untested + NoDoc + NoDemo = NOT OK


    \end{boxedminipage}

    \label{xformslib:flxyplot:fl_set_xyplot_keys}
    \index{xformslib \textit{(package)}!xformslib.flxyplot \textit{(module)}!xformslib.flxyplot.fl\_set\_xyplot\_keys \textit{(function)}}

    \vspace{0.5ex}

\hspace{.8\funcindent}\begin{boxedminipage}{\funcwidth}

    \raggedright \textbf{fl\_set\_xyplot\_keys}(\textit{pFlObject}, \textit{keystxt}, \textit{x}, \textit{y}, \textit{align})

    \vspace{-1.5ex}

    \rule{\textwidth}{0.5\fboxrule}
\setlength{\parskip}{2ex}

Adds a series of keys to a particular plot and sets the position for
each key. A key is the combination of drawing a segment of the plot line
style with a piece of text that describes what the corresponding line
represents. Obviously, keys are most useful when you have more than one
plot (i.e. overlays).

-{}-
\setlength{\parskip}{1ex}
      \textbf{Parameters}
      \vspace{-1ex}

      \begin{quote}
        \begin{Ventry}{xxxxxxxxx}

          \item[pFlObject]


xyplot object
            {\it (type=pointer to xfdata.FL\_OBJECT)}

          \item[keystxt]


series of keys for each plot. The last element of the array must be
None to indicate the end. The array index is the plot id, i.e., key{[}0{]}
is the key for the base plot, key{[}1{]} the key for the the first overlay
etc.
            {\it (type=list\_of\_str)}

          \item[x]


series of horizontal positions in world coordinate system
            {\it (type=list\_of\_float)}

          \item[y]


series of horizontal positions in world coordinate system
            {\it (type=list\_of\_float)}

          \item[align]


alignment of the entire key box relative to the given position. Values
(from xfdata.py) FL\_ALIGN\_CENTER, FL\_ALIGN\_TOP, FL\_ALIGN\_BOTTOM,
FL\_ALIGN\_LEFT, FL\_ALIGN\_RIGHT, FL\_ALIGN\_LEFT\_TOP, FL\_ALIGN\_RIGHT\_TOP,
FL\_ALIGN\_LEFT\_BOTTOM, FL\_ALIGN\_RIGHT\_BOTTOM, FL\_ALIGN\_INSIDE,
FL\_ALIGN\_VERT. Bitwise OR with FL\_ALIGN\_INSIDE is allowed.
            {\it (type=int)}

        \end{Ventry}

      \end{quote}

\textbf{Note:} 
e.g. \emph{todo}


\textbf{Status:} 
Untested + NoDoc + NoDemo = NOT OK


    \end{boxedminipage}

    \label{xformslib:flxyplot:fl_set_xyplot_key}
    \index{xformslib \textit{(package)}!xformslib.flxyplot \textit{(module)}!xformslib.flxyplot.fl\_set\_xyplot\_key \textit{(function)}}

    \vspace{0.5ex}

\hspace{.8\funcindent}\begin{boxedminipage}{\funcwidth}

    \raggedright \textbf{fl\_set\_xyplot\_key}(\textit{pFlObject}, \textit{ovlid}, \textit{keytxt})

    \vspace{-1.5ex}

    \rule{\textwidth}{0.5\fboxrule}
\setlength{\parskip}{2ex}

Adds or removes a key to a particular plot. A key is the combination
of drawing a segment of the plot line style with a piece of text that
describes what the corresponding line represents. Obviously, keys are
most useful when you have more than one plot (i.e. overlays). All the
keys will be drawn together inside a box.

-{}-
\setlength{\parskip}{1ex}
      \textbf{Parameters}
      \vspace{-1ex}

      \begin{quote}
        \begin{Ventry}{xxxxxxxxx}

          \item[pFlObject]


xyplot object
            {\it (type=pointer to xfdata.FL\_OBJECT)}

          \item[ovlid]


overlay id. Values between 1 and xfdata.FL\_MAX\_XYPLOTOVERLAY or the
number set via fl\_set\_xyplot\_maxoverlays()
            {\it (type=int)}

          \item[keytxt]


key for the plot. If it's 'None' removes a key.
            {\it (type=str)}

        \end{Ventry}

      \end{quote}

\textbf{Note:} 
e.g. \emph{todo}


\textbf{Status:} 
Untested + NoDoc + NoDemo = NOT OK


    \end{boxedminipage}

    \label{xformslib:flxyplot:fl_set_xyplot_key_position}
    \index{xformslib \textit{(package)}!xformslib.flxyplot \textit{(module)}!xformslib.flxyplot.fl\_set\_xyplot\_key\_position \textit{(function)}}

    \vspace{0.5ex}

\hspace{.8\funcindent}\begin{boxedminipage}{\funcwidth}

    \raggedright \textbf{fl\_set\_xyplot\_key\_position}(\textit{pFlObject}, \textit{x}, \textit{y}, \textit{align})

    \vspace{-1.5ex}

    \rule{\textwidth}{0.5\fboxrule}
\setlength{\parskip}{2ex}

Sets the position of the keys in xyplot object.

-{}-
\setlength{\parskip}{1ex}
      \textbf{Parameters}
      \vspace{-1ex}

      \begin{quote}
        \begin{Ventry}{xxxxxxxxx}

          \item[pFlObject]


xyplot object
            {\it (type=pointer to xfdata.FL\_OBJECT)}

          \item[x]


horizontal position in world coordinate system
            {\it (type=float)}

          \item[y]


horizontal position in world coordinate system
            {\it (type=float)}

          \item[align]


alignment of the entire key box relative to the given position. Values
(from xfdata.py) FL\_ALIGN\_CENTER, FL\_ALIGN\_TOP, FL\_ALIGN\_BOTTOM,
FL\_ALIGN\_LEFT, FL\_ALIGN\_RIGHT, FL\_ALIGN\_LEFT\_TOP, FL\_ALIGN\_RIGHT\_TOP,
FL\_ALIGN\_LEFT\_BOTTOM, FL\_ALIGN\_RIGHT\_BOTTOM, FL\_ALIGN\_INSIDE,
FL\_ALIGN\_VERT. Bitwise OR with FL\_ALIGN\_INSIDE is allowed.
            {\it (type=int)}

        \end{Ventry}

      \end{quote}

\textbf{Note:} 
e.g. \emph{todo}


\textbf{Status:} 
Untested + NoDoc + NoDemo = NOT OK


    \end{boxedminipage}

    \label{xformslib:flxyplot:fl_set_xyplot_key_font}
    \index{xformslib \textit{(package)}!xformslib.flxyplot \textit{(module)}!xformslib.flxyplot.fl\_set\_xyplot\_key\_font \textit{(function)}}

    \vspace{0.5ex}

\hspace{.8\funcindent}\begin{boxedminipage}{\funcwidth}

    \raggedright \textbf{fl\_set\_xyplot\_key\_font}(\textit{pFlObject}, \textit{style}, \textit{size})

    \vspace{-1.5ex}

    \rule{\textwidth}{0.5\fboxrule}
\setlength{\parskip}{2ex}

Changes the font the key text uses in xyplot object.

-{}-
\setlength{\parskip}{1ex}
      \textbf{Parameters}
      \vspace{-1ex}

      \begin{quote}
        \begin{Ventry}{xxxxxxxxx}

          \item[pFlObject]


xyplot object
            {\it (type=pointer to xfdata.FL\_OBJECT)}

          \item[style]


label style. Values (from xfdata.py) FL\_NORMAL\_STYLE,
FL\_BOLD\_STYLE, FL\_ITALIC\_STYLE, FL\_BOLDITALIC\_STYLE, FL\_FIXED\_STYLE,
FL\_FIXEDBOLD\_STYLE, FL\_FIXEDITALIC\_STYLE, FL\_FIXEDBOLDITALIC\_STYLE,
FL\_TIMES\_STYLE, FL\_TIMESBOLD\_STYLE, FL\_TIMESITALIC\_STYLE,
FL\_TIMESBOLDITALIC\_STYLE, FL\_MISC\_STYLE, FL\_MISCBOLD\_STYLE,
FL\_MISCITALIC\_STYLE, FL\_SYMBOL\_STYLE, FL\_SHADOW\_STYLE,
FL\_ENGRAVED\_STYLE, FL\_EMBOSSED\_STYLE
            {\it (type=int)}

          \item[size]


label size. Values (from xfdata.py) FL\_TINY\_SIZE, FL\_SMALL\_SIZE,
FL\_NORMAL\_SIZE, FL\_MEDIUM\_SIZE, FL\_LARGE\_SIZE, FL\_HUGE\_SIZE,
FL\_DEFAULT\_SIZE
            {\it (type=int)}

        \end{Ventry}

      \end{quote}

\textbf{Note:} 
e.g. \emph{todo}


\textbf{Status:} 
Untested + NoDoc + NoDemo = NOT OK


    \end{boxedminipage}

    \label{xformslib:flxyplot:fl_get_xyplot_numdata}
    \index{xformslib \textit{(package)}!xformslib.flxyplot \textit{(module)}!xformslib.flxyplot.fl\_get\_xyplot\_numdata \textit{(function)}}

    \vspace{0.5ex}

\hspace{.8\funcindent}\begin{boxedminipage}{\funcwidth}

    \raggedright \textbf{fl\_get\_xyplot\_numdata}(\textit{pFlObject}, \textit{ovlid})

    \vspace{-1.5ex}

    \rule{\textwidth}{0.5\fboxrule}
\setlength{\parskip}{2ex}

Obtain the number of data points of a xyplot object.

-{}-
\setlength{\parskip}{1ex}
      \textbf{Parameters}
      \vspace{-1ex}

      \begin{quote}
        \begin{Ventry}{xxxxxxxxx}

          \item[pFlObject]


xyplot object
            {\it (type=pointer to xfdata.FL\_OBJECT)}

          \item[ovlid]


overlay id. Values between 1 and xfdata.FL\_MAX\_XYPLOTOVERLAY or the
number set via fl\_set\_xyplot\_maxoverlays(). If it's 0 uses the base
dataset.
            {\it (type=int)}

        \end{Ventry}

      \end{quote}

      \textbf{Return Value}
    \vspace{-1ex}

      \begin{quote}

number of data points
      {\it (type=int)}

      \end{quote}

\textbf{Note:} 
e.g. \emph{todo}


\textbf{Status:} 
Untested + NoDoc + NoDemo = NOT OK


    \end{boxedminipage}

    \label{xformslib:flxyplot:fl_xyplot_s2w}
    \index{xformslib \textit{(package)}!xformslib.flxyplot \textit{(module)}!xformslib.flxyplot.fl\_xyplot\_s2w \textit{(function)}}

    \vspace{0.5ex}

\hspace{.8\funcindent}\begin{boxedminipage}{\funcwidth}

    \raggedright \textbf{fl\_xyplot\_s2w}(\textit{pFlObject}, \textit{sx}, \textit{sy})

    \vspace{-1.5ex}

    \rule{\textwidth}{0.5\fboxrule}
\setlength{\parskip}{2ex}

Obtains, by conversion, the world coordinates from the screen
coordinates of a xyplot object.

-{}-
\setlength{\parskip}{1ex}
      \textbf{Parameters}
      \vspace{-1ex}

      \begin{quote}
        \begin{Ventry}{xxxxxxxxx}

          \item[pFlObject]


xyplot object
            {\it (type=pointer to xfdata.FL\_OBJECT)}

          \item[sx]


horizontal position in screen coordinates
            {\it (type=float)}

          \item[sy]


vertical position in screen coordinates
            {\it (type=float)}

        \end{Ventry}

      \end{quote}

      \textbf{Return Value}
    \vspace{-1ex}

      \begin{quote}

horizontal position in world coordinates (wx), vertical position
in world coordinates (wy)
      {\it (type=float, float)}

      \end{quote}

\textbf{Note:} 
e.g. \emph{todo}


\textbf{Attention:} 
API change from XForms - upstream was
fl\_xyplot\_s2w(pFlObject, sx, sy, wx, wy)


\textbf{Status:} 
Untested + NoDoc + NoDemo = NOT OK


    \end{boxedminipage}

    \label{xformslib:flxyplot:fl_xyplot_w2s}
    \index{xformslib \textit{(package)}!xformslib.flxyplot \textit{(module)}!xformslib.flxyplot.fl\_xyplot\_w2s \textit{(function)}}

    \vspace{0.5ex}

\hspace{.8\funcindent}\begin{boxedminipage}{\funcwidth}

    \raggedright \textbf{fl\_xyplot\_w2s}(\textit{pFlObject}, \textit{wx}, \textit{wy})

    \vspace{-1.5ex}

    \rule{\textwidth}{0.5\fboxrule}
\setlength{\parskip}{2ex}

Obtains, by conversion, the screen coordinates from the world
coordinates of a xyplot object.

-{}-
\setlength{\parskip}{1ex}
      \textbf{Parameters}
      \vspace{-1ex}

      \begin{quote}
        \begin{Ventry}{xxxxxxxxx}

          \item[pFlObject]


xyplot object
            {\it (type=pointer to xfdata.FL\_OBJECT)}

          \item[wx]


horizontal position in world coordinates
            {\it (type=float)}

          \item[wy]


vertical position in world coordinates
            {\it (type=float)}

        \end{Ventry}

      \end{quote}

      \textbf{Return Value}
    \vspace{-1ex}

      \begin{quote}

horizontal position in screen coordinates (sx), vertical position
in screen coordinates (sy)
      {\it (type=float, float)}

      \end{quote}

\textbf{Note:} 
e.g. \emph{todo}


\textbf{Attention:} 
API change from XForms - upstream was
fl\_xyplot\_w2s(pFlObject, wx, wy)


\textbf{Status:} 
Untested + NoDoc + NoDemo = NOT OK


    \end{boxedminipage}

    \label{xformslib:flxyplot:fl_set_xyplot_xscale}
    \index{xformslib \textit{(package)}!xformslib.flxyplot \textit{(module)}!xformslib.flxyplot.fl\_set\_xyplot\_xscale \textit{(function)}}

    \vspace{0.5ex}

\hspace{.8\funcindent}\begin{boxedminipage}{\funcwidth}

    \raggedright \textbf{fl\_set\_xyplot\_xscale}(\textit{pFlObject}, \textit{scale}, \textit{base})

    \vspace{-1.5ex}

    \rule{\textwidth}{0.5\fboxrule}
\setlength{\parskip}{2ex}

Changes the scaling for a xyplot object. By default, a linear scale in
x-direction is used.

-{}-
\setlength{\parskip}{1ex}
      \textbf{Parameters}
      \vspace{-1ex}

      \begin{quote}
        \begin{Ventry}{xxxxxxxxx}

          \item[pFlObject]


xyplot object
            {\it (type=pointer to xfdata.FL\_OBJECT)}

          \item[scale]


scaling to be used. Values (from xfdata.py) FL\_LINEAR (default) or
FL\_LOG
            {\it (type=int)}

          \item[base]


base of the logarithm to be used. Used only if scale is xfdata.FL\_LOG
            {\it (type=float)}

        \end{Ventry}

      \end{quote}

\textbf{Note:} 
e.g. \emph{todo}


\textbf{Status:} 
Untested + NoDoc + NoDemo = NOT OK


    \end{boxedminipage}

    \label{xformslib:flxyplot:fl_set_xyplot_yscale}
    \index{xformslib \textit{(package)}!xformslib.flxyplot \textit{(module)}!xformslib.flxyplot.fl\_set\_xyplot\_yscale \textit{(function)}}

    \vspace{0.5ex}

\hspace{.8\funcindent}\begin{boxedminipage}{\funcwidth}

    \raggedright \textbf{fl\_set\_xyplot\_yscale}(\textit{pFlObject}, \textit{scale}, \textit{base})

    \vspace{-1.5ex}

    \rule{\textwidth}{0.5\fboxrule}
\setlength{\parskip}{2ex}

Changes the scaling for a xyplot object. By default, a linear scale in
y-direction is used.

-{}-
\setlength{\parskip}{1ex}
      \textbf{Parameters}
      \vspace{-1ex}

      \begin{quote}
        \begin{Ventry}{xxxxxxxxx}

          \item[pFlObject]


xyplot object
            {\it (type=pointer to xfdata.FL\_OBJECT)}

          \item[scale]


scaling to be used. Values (from xfdata.py) FL\_LINEAR (default) or
FL\_LOG
            {\it (type=int)}

          \item[base]


base of the logarithm to be used. Used only if scale is xfdata.FL\_LOG
            {\it (type=float)}

        \end{Ventry}

      \end{quote}

\textbf{Note:} 
e.g. \emph{todo}


\textbf{Status:} 
Untested + NoDoc + NoDemo = NOT OK


    \end{boxedminipage}

    \label{xformslib:flxyplot:fl_clear_xyplot}
    \index{xformslib \textit{(package)}!xformslib.flxyplot \textit{(module)}!xformslib.flxyplot.fl\_clear\_xyplot \textit{(function)}}

    \vspace{0.5ex}

\hspace{.8\funcindent}\begin{boxedminipage}{\funcwidth}

    \raggedright \textbf{fl\_clear\_xyplot}(\textit{pFlObject})

    \vspace{-1.5ex}

    \rule{\textwidth}{0.5\fboxrule}
\setlength{\parskip}{2ex}

Clears a xyplot object. It frees all data associated with an xyplot,
including all overlays and all inset texts. It does not reset all
plotting options, such as line thickness, major/minor divisions etc. nor
does it free all memories associated with the xyplot, for this
fl\_free\_object() is needed.

-{}-
\setlength{\parskip}{1ex}
      \textbf{Parameters}
      \vspace{-1ex}

      \begin{quote}
        \begin{Ventry}{xxxxxxxxx}

          \item[pFlObject]


xyplot object
            {\it (type=pointer to xfdata.FL\_OBJECT)}

        \end{Ventry}

      \end{quote}

\textbf{Note:} 
e.g. \emph{todo}


\textbf{Status:} 
Untested + NoDoc + NoDemo = NOT OK


    \end{boxedminipage}

    \label{xformslib:flxyplot:fl_set_xyplot_linewidth}
    \index{xformslib \textit{(package)}!xformslib.flxyplot \textit{(module)}!xformslib.flxyplot.fl\_set\_xyplot\_linewidth \textit{(function)}}

    \vspace{0.5ex}

\hspace{.8\funcindent}\begin{boxedminipage}{\funcwidth}

    \raggedright \textbf{fl\_set\_xyplot\_linewidth}(\textit{pFlObject}, \textit{ovlid}, \textit{lw})

    \vspace{-1.5ex}

    \rule{\textwidth}{0.5\fboxrule}
\setlength{\parskip}{2ex}

Changes the line thickness of an xyplot (base data or overlay).

-{}-
\setlength{\parskip}{1ex}
      \textbf{Parameters}
      \vspace{-1ex}

      \begin{quote}
        \begin{Ventry}{xxxxxxxxx}

          \item[pFlObject]


xyplot object
            {\it (type=pointer to xfdata.FL\_OBJECT)}

          \item[ovlid]


overlay id. Values between 1 and xfdata.FL\_MAX\_XYPLOTOVERLAY or the
number set via fl\_set\_xyplot\_maxoverlays(). If it's 0 uses the base
dataset.
            {\it (type=int)}

          \item[lw]


width of line. If it's 0, restores the server default and typically
is the fastest
            {\it (type=int)}

        \end{Ventry}

      \end{quote}

\textbf{Note:} 
e.g. \emph{todo}


\textbf{Status:} 
Untested + NoDoc + NoDemo = NOT OK


    \end{boxedminipage}

    \label{xformslib:flxyplot:fl_set_xyplot_xgrid}
    \index{xformslib \textit{(package)}!xformslib.flxyplot \textit{(module)}!xformslib.flxyplot.fl\_set\_xyplot\_xgrid \textit{(function)}}

    \vspace{0.5ex}

\hspace{.8\funcindent}\begin{boxedminipage}{\funcwidth}

    \raggedright \textbf{fl\_set\_xyplot\_xgrid}(\textit{pFlObject}, \textit{grid})

    \vspace{-1.5ex}

    \rule{\textwidth}{0.5\fboxrule}
\setlength{\parskip}{2ex}

Sets up the grid level for x-axis of a xyplot object.

-{}-
\setlength{\parskip}{1ex}
      \textbf{Parameters}
      \vspace{-1ex}

      \begin{quote}
        \begin{Ventry}{xxxxxxxxx}

          \item[pFlObject]


xyplot object
            {\it (type=pointer to xfdata.FL\_OBJECT)}

          \item[grid]


level of grid to be set. Values (from xfdata.py) FL\_GRID\_NONE,
FL\_GRID\_MAJOR, FL\_GRID\_MINOR
            {\it (type=int)}

        \end{Ventry}

      \end{quote}

\textbf{Note:} 
e.g. \emph{todo}


\textbf{Status:} 
Untested + NoDoc + NoDemo = NOT OK


    \end{boxedminipage}

    \label{xformslib:flxyplot:fl_set_xyplot_ygrid}
    \index{xformslib \textit{(package)}!xformslib.flxyplot \textit{(module)}!xformslib.flxyplot.fl\_set\_xyplot\_ygrid \textit{(function)}}

    \vspace{0.5ex}

\hspace{.8\funcindent}\begin{boxedminipage}{\funcwidth}

    \raggedright \textbf{fl\_set\_xyplot\_ygrid}(\textit{pFlObject}, \textit{grid})

    \vspace{-1.5ex}

    \rule{\textwidth}{0.5\fboxrule}
\setlength{\parskip}{2ex}

Sets up the grid level for y-axis of a xyplot object.

-{}-
\setlength{\parskip}{1ex}
      \textbf{Parameters}
      \vspace{-1ex}

      \begin{quote}
        \begin{Ventry}{xxxxxxxxx}

          \item[pFlObject]


xyplot object
            {\it (type=pointer to xfdata.FL\_OBJECT)}

          \item[grid]


level of grid to be set. Values (from xfdata.py) FL\_GRID\_NONE,
FL\_GRID\_MAJOR, FL\_GRID\_MINOR
            {\it (type=int)}

        \end{Ventry}

      \end{quote}

\textbf{Note:} 
e.g. \emph{todo}


\textbf{Status:} 
Untested + NoDoc + NoDemo = NOT OK


    \end{boxedminipage}

    \label{xformslib:flxyplot:fl_set_xyplot_grid_linestyle}
    \index{xformslib \textit{(package)}!xformslib.flxyplot \textit{(module)}!xformslib.flxyplot.fl\_set\_xyplot\_grid\_linestyle \textit{(function)}}

    \vspace{0.5ex}

\hspace{.8\funcindent}\begin{boxedminipage}{\funcwidth}

    \raggedright \textbf{fl\_set\_xyplot\_grid\_linestyle}(\textit{pFlObject}, \textit{linestyle})

    \vspace{-1.5ex}

    \rule{\textwidth}{0.5\fboxrule}
\setlength{\parskip}{2ex}

Changes the linestyle used for drawing  the grid line of xyplot. By
default it uses a dotted line

-{}-
\setlength{\parskip}{1ex}
      \textbf{Parameters}
      \vspace{-1ex}

      \begin{quote}
        \begin{Ventry}{xxxxxxxxx}

          \item[pFlObject]


xyplot object
            {\it (type=pointer to xfdata.FL\_OBJECT)}

          \item[linestyle]


style of the line to draw. Values (from xfdata module.py) FL\_SOLID,
FL\_USERDASH, FL\_USERDOUBLEDASH, FL\_DOT, FL\_DOTDASH, FL\_DASH,
FL\_LONGDASH
            {\it (type=int)}

        \end{Ventry}

      \end{quote}

      \textbf{Return Value}
    \vspace{-1ex}

      \begin{quote}

old grid linestyle
      {\it (type=int)}

      \end{quote}

\textbf{Note:} 
e.g. \emph{todo}


\textbf{Status:} 
Untested + NoDoc + NoDemo = NOT OK


    \end{boxedminipage}

    \label{xformslib:flxyplot:fl_set_xyplot_alphaxtics}
    \index{xformslib \textit{(package)}!xformslib.flxyplot \textit{(module)}!xformslib.flxyplot.fl\_set\_xyplot\_alphaxtics \textit{(function)}}

    \vspace{0.5ex}

\hspace{.8\funcindent}\begin{boxedminipage}{\funcwidth}

    \raggedright \textbf{fl\_set\_xyplot\_alphaxtics}(\textit{pFlObject}, \textit{major}, \textit{minor})

    \vspace{-1.5ex}

    \rule{\textwidth}{0.5\fboxrule}
\setlength{\parskip}{2ex}

Labels the major tic marks on x-axis with alphanumerical characters
(instead of numerical values). fl\_set\_xyplot\_xtics can?t be active at the
same time and the one that gets used is the one that was set last. It can
be used to specify non-uniform and arbitary major divisions; to achieve
this, you should embed the major tic location information in the
alphanumerical text; the location information is introduced by the symbol
and followed by a float number specifying the coordinates in world
coordinates; the entire location info should follow the label. E.g.
``\href{mailto:Begin@1.0|3}{Begin@1.0|3}\href{mailto:/4@0.75|1}{/4@0.75|1}.9@1.9'' will produce three major tic marks at 0.75,
1.0, and 1.9 and labeled ``3/4'', ``begin'', and ``1.9''.

-{}-
\setlength{\parskip}{1ex}
      \textbf{Parameters}
      \vspace{-1ex}

      \begin{quote}
        \begin{Ventry}{xxxxxxxxx}

          \item[pFlObject]


xyplot object
            {\it (type=pointer to xfdata.FL\_OBJECT)}

          \item[major]


text specifying the labels with the embedded character | that
describes major divisions. E.g. to label a plot with Monday, Tuesday
etc, major should be given as Monday|Tuesday|....
            {\it (type=str)}

          \item[minor]


currently unused. It is set to 1, i.e, no divisions between major tic
marks.
            {\it (type=str)}

        \end{Ventry}

      \end{quote}

\textbf{Note:} 
e.g. \emph{todo}


\textbf{Status:} 
Untested + NoDoc + NoDemo = NOT OK


    \end{boxedminipage}

    \label{xformslib:flxyplot:fl_set_xyplot_alphaytics}
    \index{xformslib \textit{(package)}!xformslib.flxyplot \textit{(module)}!xformslib.flxyplot.fl\_set\_xyplot\_alphaytics \textit{(function)}}

    \vspace{0.5ex}

\hspace{.8\funcindent}\begin{boxedminipage}{\funcwidth}

    \raggedright \textbf{fl\_set\_xyplot\_alphaytics}(\textit{pFlObject}, \textit{major}, \textit{minor})

    \vspace{-1.5ex}

    \rule{\textwidth}{0.5\fboxrule}
\setlength{\parskip}{2ex}

Labels the major tic marks on y-axis with alphanumerical characters
(instead of numerical values). fl\_set\_xyplot\_ytics can?t be active at the
same time and the one that gets used is the one that was set last. It can
be used to specify non-uniform and arbitary major divisions; to achieve
this, you should embed the major tic location information in the
alphanumerical text; the location information is introduced by the symbol
and followed by a float number specifying the coordinates in world
coordinates; the entire location info should follow the label. E.g.
``\href{mailto:Begin@1.0|3}{Begin@1.0|3}\href{mailto:/4@0.75|1}{/4@0.75|1}.9@1.9'' will produce three major tic marks at 0.75,
1.0, and 1.9 and labeled ``3/4'', ``begin'', and ``1.9''.

-{}-
\setlength{\parskip}{1ex}
      \textbf{Parameters}
      \vspace{-1ex}

      \begin{quote}
        \begin{Ventry}{xxxxxxxxx}

          \item[pFlObject]


xyplot object
            {\it (type=pointer to xfdata.FL\_OBJECT)}

          \item[major]


text specifying the labels with the embedded character | that
describes major divisions. E.g. to label a plot with Monday, Tuesday
etc, major should be given as Monday|Tuesday|....
            {\it (type=str)}

          \item[minor]


currently unused. It is set to 1, i.e, no divisions between major tic
marks.
            {\it (type=str)}

        \end{Ventry}

      \end{quote}

\textbf{Note:} 
e.g. \emph{todo}


\textbf{Status:} 
Untested + NoDoc + NoDemo = NOT OK


    \end{boxedminipage}

    \label{xformslib:flxyplot:fl_set_xyplot_fixed_xaxis}
    \index{xformslib \textit{(package)}!xformslib.flxyplot \textit{(module)}!xformslib.flxyplot.fl\_set\_xyplot\_fixed\_xaxis \textit{(function)}}

    \vspace{0.5ex}

\hspace{.8\funcindent}\begin{boxedminipage}{\funcwidth}

    \raggedright \textbf{fl\_set\_xyplot\_fixed\_xaxis}(\textit{pFlObject}, \textit{leftmrg}, \textit{rightmrg})

    \vspace{-1.5ex}

    \rule{\textwidth}{0.5\fboxrule}
\setlength{\parskip}{2ex}

Controls the plotting area for x-axis of xyplot object. By default,
the plotting area is automatically adjusted for tic labels and titles so
that a maximum plotting area results, but this can be undesirable in
certain situations. The pixel amounts are computed using the current
label font and size.

-{}-
\setlength{\parskip}{1ex}
      \textbf{Parameters}
      \vspace{-1ex}

      \begin{quote}
        \begin{Ventry}{xxxxxxxxx}

          \item[pFlObject]


xyplot object
            {\it (type=pointer to xfdata.FL\_OBJECT)}

          \item[leftmrg]


left margin to be set. If it's 'None' restore automatic margin
computation
            {\it (type=str)}

          \item[rightmrg]


right margin to be set. If it's 'None', restores automatic margin
computation.
            {\it (type=str)}

        \end{Ventry}

      \end{quote}

\textbf{Note:} 
e.g. \emph{todo}


\textbf{Status:} 
Untested + NoDoc + NoDemo = NOT OK


    \end{boxedminipage}

    \label{xformslib:flxyplot:fl_set_xyplot_fixed_yaxis}
    \index{xformslib \textit{(package)}!xformslib.flxyplot \textit{(module)}!xformslib.flxyplot.fl\_set\_xyplot\_fixed\_yaxis \textit{(function)}}

    \vspace{0.5ex}

\hspace{.8\funcindent}\begin{boxedminipage}{\funcwidth}

    \raggedright \textbf{fl\_set\_xyplot\_fixed\_yaxis}(\textit{pFlObject}, \textit{bottommrg}, \textit{topmrg})

    \vspace{-1.5ex}

    \rule{\textwidth}{0.5\fboxrule}
\setlength{\parskip}{2ex}

Controls the plotting area for y-axis of xyplot object. By default,
the plotting area is automatically adjusted for tic labels and titles so
that a maximum plotting area results, but this can be undesirable in
certain situations. The pixel amounts are computed using the current
label font and size. Even for y-axis margins the length of the string,
not the height, is used as the margin, thus to leave space for one line
of text, a single character (say m) or two narrow characters (say ii)
should be used.

-{}-
\setlength{\parskip}{1ex}
      \textbf{Parameters}
      \vspace{-1ex}

      \begin{quote}
        \begin{Ventry}{xxxxxxxxx}

          \item[pFlObject]


xyplot object
            {\it (type=pointer to xfdata.FL\_OBJECT)}

          \item[bottommrg]


bottom margin to be set. If it's 'None' restore automatic margin
computation
            {\it (type=str)}

          \item[topmrg]


top margin to be set. If it's 'None', restores automatic margin
computation.
            {\it (type=str)}

        \end{Ventry}

      \end{quote}

\textbf{Note:} 
e.g. \emph{todo}


\textbf{Status:} 
Untested + NoDoc + NoDemo = NOT OK


    \end{boxedminipage}

    \label{xformslib:flxyplot:fl_interpolate}
    \index{xformslib \textit{(package)}!xformslib.flxyplot \textit{(module)}!xformslib.flxyplot.fl\_interpolate \textit{(function)}}

    \vspace{0.5ex}

\hspace{.8\funcindent}\begin{boxedminipage}{\funcwidth}

    \raggedright \textbf{fl\_interpolate}(\textit{wx}, \textit{wy}, \textit{numin}, \textit{grid}, \textit{degree})

    \vspace{-1.5ex}

    \rule{\textwidth}{0.5\fboxrule}
\setlength{\parskip}{2ex}

Manage polynomial interpolation function and obtains the number of
points in interpolated function ((wx{[}numin - 1{]} - wx{[}0{]}) / grid + 1.01)
and the interpolate values.

-{}-
\setlength{\parskip}{1ex}
      \textbf{Parameters}
      \vspace{-1ex}

      \begin{quote}
        \begin{Ventry}{xxxxxx}

          \item[wx]


horizontal value in world coordinates
            {\it (type=float)}

          \item[wy]


vertical value in world coordinates
            {\it (type=float)}

          \item[numin]


number of points to interpolate
            {\it (type=int)}

          \item[grid]


the working grid onto which the data are to be interpolated.
            {\it (type=float)}

          \item[degree]


the order of the polynomial to use. If it's 0 or 1, restores the
default linear interpolation.
            {\it (type=int)}

        \end{Ventry}

      \end{quote}

      \textbf{Return Value}
    \vspace{-1ex}

      \begin{quote}

number of points in the interpolated function or -1 (on failure),
interpolated value for x-axis (outx), interpolate value for y-axis
(outy)
      {\it (type=int, float, float)}

      \end{quote}

\textbf{Note:} 
e.g. \emph{todo}


\textbf{Attention:} 
API change from XForms - upstream was
fl\_interpolate(inx, iny, num\_in, outx, outy, grid, ndeg)


\textbf{Status:} 
Untested + NoDoc + NoDemo = NOT OK


    \end{boxedminipage}

    \label{xformslib:flxyplot:fl_spline_interpolate}
    \index{xformslib \textit{(package)}!xformslib.flxyplot \textit{(module)}!xformslib.flxyplot.fl\_spline\_interpolate \textit{(function)}}

    \vspace{0.5ex}

\hspace{.8\funcindent}\begin{boxedminipage}{\funcwidth}

    \raggedright \textbf{fl\_spline\_interpolate}(\textit{wx}, \textit{wy}, \textit{numin}, \textit{grid})

    \vspace{-1.5ex}

    \rule{\textwidth}{0.5\fboxrule}
\setlength{\parskip}{2ex}

Manages spline interpolation function. Spline interpolation is a form
of interpolation where the interpolant is a special type of piecewise
polynomial called a spline. Obtain number of points in interpolate
function and the interpolate values.

-{}-
\setlength{\parskip}{1ex}
      \textbf{Parameters}
      \vspace{-1ex}

      \begin{quote}
        \begin{Ventry}{xxxxx}

          \item[wx]


horizontal value in world coordinates
            {\it (type=float)}

          \item[wy]


vertical value in world coordinates
            {\it (type=float)}

          \item[numin]


number of points to interpolate
            {\it (type=int)}

          \item[grid]


the working grid onto which the data are to be interpolated.
            {\it (type=float)}

        \end{Ventry}

      \end{quote}

      \textbf{Return Value}
    \vspace{-1ex}

      \begin{quote}

number of points in the interpolated function or -1 (on failure),
interpolated value for x-axis (outx), interpolate value for y-axis
(outy)
      {\it (type=int, float, float)}

      \end{quote}

\textbf{Note:} 
e.g. \emph{todo}


\textbf{Attention:} 
API change from XForms - upstream was
fl\_spline\_interpolate(inx, iny, num\_in, outx, outy, grid)


\textbf{Status:} 
Untested + NoDoc + NoDemo = NOT OK


    \end{boxedminipage}

    \label{xformslib:flxyplot:fl_set_xyplot_symbol}
    \index{xformslib \textit{(package)}!xformslib.flxyplot \textit{(module)}!xformslib.flxyplot.fl\_set\_xyplot\_symbol \textit{(function)}}

    \vspace{0.5ex}

\hspace{.8\funcindent}\begin{boxedminipage}{\funcwidth}

    \raggedright \textbf{fl\_set\_xyplot\_symbol}(\textit{pFlObject}, \textit{ovlid}, \textit{py\_XyPlotSymbol})

    \vspace{-1.5ex}

    \rule{\textwidth}{0.5\fboxrule}
\setlength{\parskip}{2ex}

Sets a python function to change a symbol, to be invoked for
xfdata.FL\_POINTS\_XYPLOT and xfdata.FL\_LINEPOINTS\_XYPLOT's xyplot types
(main plot or overlay). If the type of xyplot corresponding to ovlid is
not one of them, the function will not be called.

-{}-
\setlength{\parskip}{1ex}
      \textbf{Parameters}
      \vspace{-1ex}

      \begin{quote}
        \begin{Ventry}{xxxxxxxxxxxxxxx}

          \item[pFlObject]


xyplot object
            {\it (type=pointer to xfdata.FL\_OBJECT)}

          \item[ovlid]


overlay id (0 means the main plot, and you can use -1 to indicate all)
            {\it (type=int)}

          \item[py\_XyPlotSymbol]


It will be called to draw the symbols on the data point.
Name referring to function(pFlObject, int ovlid, pPoint, int npoints,
int w, int h). The parameters passed to this function are the object
pointer, the overlay id, the center of the symbol (p.x, p.y), the
number of data points and the preferred symbol size (w, h).
            {\it (type=python function, no return)}

        \end{Ventry}

      \end{quote}

      \textbf{Return Value}
    \vspace{-1ex}

      \begin{quote}

old xyplotsymbol function
      {\it (type=xfdata.FL\_XYPLOT\_SYMBOL)}

      \end{quote}

\textbf{Note:} 
e.g. \emph{todo}


\textbf{Status:} 
Untested + NoDoc + NoDemo = NOT OK


    \end{boxedminipage}

    \label{xformslib:flxyplot:fl_set_xyplot_mark_active}
    \index{xformslib \textit{(package)}!xformslib.flxyplot \textit{(module)}!xformslib.flxyplot.fl\_set\_xyplot\_mark\_active \textit{(function)}}

    \vspace{0.5ex}

\hspace{.8\funcindent}\begin{boxedminipage}{\funcwidth}

    \raggedright \textbf{fl\_set\_xyplot\_mark\_active}(\textit{pFlObject}, \textit{yesno})

    \vspace{-1.5ex}

    \rule{\textwidth}{0.5\fboxrule}
\setlength{\parskip}{2ex}

Draws the squares that mark an active plot or not.

-{}-
\setlength{\parskip}{1ex}
      \textbf{Parameters}
      \vspace{-1ex}

      \begin{quote}
        \begin{Ventry}{xxxxxxxxx}

          \item[pFlObject]


xyplot object
            {\it (type=pointer to xfdata.FL\_OBJECT)}

          \item[yesno]


flag to enable/disable drawing. Values 0 (disabled) or 1 (enabled)
            {\it (type=int)}

        \end{Ventry}

      \end{quote}

      \textbf{Return Value}
    \vspace{-1ex}

      \begin{quote}

old setting
      {\it (type=int)}

      \end{quote}

\textbf{Note:} 
e.g. \emph{todo}


\textbf{Status:} 
Untested + NoDoc + NoDemo = NOT OK


    \end{boxedminipage}


%%%%%%%%%%%%%%%%%%%%%%%%%%%%%%%%%%%%%%%%%%%%%%%%%%%%%%%%%%%%%%%%%%%%%%%%%%%
%%                               Variables                               %%
%%%%%%%%%%%%%%%%%%%%%%%%%%%%%%%%%%%%%%%%%%%%%%%%%%%%%%%%%%%%%%%%%%%%%%%%%%%

  \subsection{Variables}

    \vspace{-1cm}
\hspace{\varindent}\begin{longtable}{|p{\varnamewidth}|p{\vardescrwidth}|l}
\cline{1-2}
\cline{1-2} \centering \textbf{Name} & \centering \textbf{Description}& \\
\cline{1-2}
\endhead\cline{1-2}\multicolumn{3}{r}{\small\textit{continued on next page}}\\\endfoot\cline{1-2}
\endlastfoot\raggedright \_\-\_\-p\-a\-c\-k\-a\-g\-e\-\_\-\_\- & \raggedright \textbf{Value:} 
{\tt \texttt{'}\texttt{xformslib}\texttt{'}}&\\
\cline{1-2}
\end{longtable}

    \index{xformslib \textit{(package)}!xformslib.flxyplot \textit{(module)}|)}
