%
% API Documentation for API Documentation
% Module xformslib.flbasic
%
% Generated by epydoc 3.0.1
% [Sun Jan 31 21:45:30 2010]
%

%%%%%%%%%%%%%%%%%%%%%%%%%%%%%%%%%%%%%%%%%%%%%%%%%%%%%%%%%%%%%%%%%%%%%%%%%%%
%%                          Module Description                           %%
%%%%%%%%%%%%%%%%%%%%%%%%%%%%%%%%%%%%%%%%%%%%%%%%%%%%%%%%%%%%%%%%%%%%%%%%%%%

    \index{xformslib \textit{(package)}!xformslib.flbasic \textit{(module)}|(}
\section{Module xformslib.flbasic}

    \label{xformslib:flbasic}
xforms-python - Python wrapper for XForms (X11) GUI C toolkit library using
ctypes

Copyright (C) 2009, 2010  Luca Lazzaroni "LukenShiro" e-mail: 
{\textless}lukenshiro@ngi.it{\textgreater}

This program is free software: you can redistribute it and/or modify it 
under the terms of the GNU Lesser General Public License as published by 
the Free Software Foundation, version 2.1 of the License.

This program is distributed in the hope that it will be useful, but WITHOUT
ANY WARRANTY; without even the implied warranty of MERCHANTABILITY or 
FITNESS FOR A PARTICULAR PURPOSE. See the GNU Lesser General Public License
for more details.

You should have received a copy of the GNU LGPL along with this program. If
not, see {\textless}http://www.gnu.org/licenses/{\textgreater}.

See CREDITS file to read acknowledgements and thanks to XForms, ctypes and 
other developers.


%%%%%%%%%%%%%%%%%%%%%%%%%%%%%%%%%%%%%%%%%%%%%%%%%%%%%%%%%%%%%%%%%%%%%%%%%%%
%%                               Functions                               %%
%%%%%%%%%%%%%%%%%%%%%%%%%%%%%%%%%%%%%%%%%%%%%%%%%%%%%%%%%%%%%%%%%%%%%%%%%%%

  \subsection{Functions}

    \label{xformslib:flbasic:FL_IS_UPBOX}
    \index{xformslib \textit{(package)}!xformslib.flbasic \textit{(module)}!xformslib.flbasic.FL\_IS\_UPBOX \textit{(function)}}

    \vspace{0.5ex}

\hspace{.8\funcindent}\begin{boxedminipage}{\funcwidth}

    \raggedright \textbf{FL\_IS\_UPBOX}(\textit{boxtype})

\setlength{\parskip}{2ex}
\setlength{\parskip}{1ex}
    \end{boxedminipage}

    \label{xformslib:flbasic:FL_IS_DOWNBOX}
    \index{xformslib \textit{(package)}!xformslib.flbasic \textit{(module)}!xformslib.flbasic.FL\_IS\_DOWNBOX \textit{(function)}}

    \vspace{0.5ex}

\hspace{.8\funcindent}\begin{boxedminipage}{\funcwidth}

    \raggedright \textbf{FL\_IS\_DOWNBOX}(\textit{boxtype})

\setlength{\parskip}{2ex}
\setlength{\parskip}{1ex}
    \end{boxedminipage}

    \label{xformslib:flbasic:FL_TO_DOWNBOX}
    \index{xformslib \textit{(package)}!xformslib.flbasic \textit{(module)}!xformslib.flbasic.FL\_TO\_DOWNBOX \textit{(function)}}

    \vspace{0.5ex}

\hspace{.8\funcindent}\begin{boxedminipage}{\funcwidth}

    \raggedright \textbf{FL\_TO\_DOWNBOX}(\textit{boxtype})

\setlength{\parskip}{2ex}
\setlength{\parskip}{1ex}
    \end{boxedminipage}

    \label{xformslib:flbasic:special_style}
    \index{xformslib \textit{(package)}!xformslib.flbasic \textit{(module)}!xformslib.flbasic.special\_style \textit{(function)}}

    \vspace{0.5ex}

\hspace{.8\funcindent}\begin{boxedminipage}{\funcwidth}

    \raggedright \textbf{special\_style}(\textit{style})

\setlength{\parskip}{2ex}
\setlength{\parskip}{1ex}
    \end{boxedminipage}

    \label{xformslib:flbasic:fl_object_returned}
    \index{xformslib \textit{(package)}!xformslib.flbasic \textit{(module)}!xformslib.flbasic.fl\_object\_returned \textit{(function)}}

    \vspace{0.5ex}

\hspace{.8\funcindent}\begin{boxedminipage}{\funcwidth}

    \raggedright \textbf{fl\_object\_returned}(\textit{pFlObject})

\setlength{\parskip}{2ex}
\setlength{\parskip}{1ex}
    \end{boxedminipage}

    \label{xformslib:flbasic:fl_add_io_callback}
    \index{xformslib \textit{(package)}!xformslib.flbasic \textit{(module)}!xformslib.flbasic.fl\_add\_io\_callback \textit{(function)}}

    \vspace{0.5ex}

\hspace{.8\funcindent}\begin{boxedminipage}{\funcwidth}

    \raggedright \textbf{fl\_add\_io\_callback}(\textit{fd}, \textit{mask}, \textit{py\_IoCallback}, \textit{vdata})

    \vspace{-1.5ex}

    \rule{\textwidth}{0.5\fboxrule}
\setlength{\parskip}{2ex}
    Registers an input callback function when input is available from fd.

\setlength{\parskip}{1ex}
      \textbf{Parameters}
      \vspace{-1ex}

      \begin{quote}
        \begin{Ventry}{xxxxxxxxxxxxx}

          \item[fd]

          a valid file descriptor in a unix system

            {\it (type=int)}

          \item[mask]

          under what circumstance the input callback should be invoked. 
          Values (from xfdata module) i.e. FL\_READ, FL\_WRITE, FL\_EXCEPT

            {\it (type=int)}

          \item[py\_IoCallback]

          python function to be invoked, no return

            {\it (type=\_\_ funcname (num, ptr\_void) \_\_)}

          \item[vdata]

          user data argument to be passed to function

            {\it (type=None or long or pointer to xfdata.FL\_OBJECT)}

        \end{Ventry}

      \end{quote}

\textbf{Example:}
\begin{quote}
  \begin{itemize}

  \item
    \setlength{\parskip}{0.6ex}
def iocb(num, vdata):



  \item {\textbar}-{\textgreater}{\textbar} ...



  \item fdesc = ... function to open file



  \item fl\_add\_io\_callback(fdesc, xfdata.FL\_READ, iocb, None)



\end{itemize}

\end{quote}

\textbf{Status:} Tested + Doc + NoDemo = OK



    \end{boxedminipage}

    \label{xformslib:flbasic:fl_remove_io_callback}
    \index{xformslib \textit{(package)}!xformslib.flbasic \textit{(module)}!xformslib.flbasic.fl\_remove\_io\_callback \textit{(function)}}

    \vspace{0.5ex}

\hspace{.8\funcindent}\begin{boxedminipage}{\funcwidth}

    \raggedright \textbf{fl\_remove\_io\_callback}(\textit{fd}, \textit{mask}, \textit{py\_IoCallback})

    \vspace{-1.5ex}

    \rule{\textwidth}{0.5\fboxrule}
\setlength{\parskip}{2ex}
    Removes the registered callback function when input is available from 
    fd.

\setlength{\parskip}{1ex}
      \textbf{Parameters}
      \vspace{-1ex}

      \begin{quote}
        \begin{Ventry}{xxxxxxxxxxxxx}

          \item[fd]

          a valid file descriptor in a unix system

            {\it (type=int)}

          \item[mask]

          under what circumstance the input callback should be removed. 
          Values (from xfdata module) FL\_READ, FL\_WRITE, FL\_EXCEPT

            {\it (type=int)}

          \item[py\_IoCallback]

          python function to be removed, no return

            {\it (type=\_\_ funcname (num, ptr\_void) \_\_)}

        \end{Ventry}

      \end{quote}

\textbf{Example:}
\begin{quote}
  \begin{itemize}

  \item
    \setlength{\parskip}{0.6ex}
def iocb(num, vdata):



  \item {\textbar}-{\textgreater}{\textbar} ...



  \item fdesc = ... function to open file



  \item fl\_remove\_io\_callback(fdesc, xfdata.FL\_READ, iocb)



\end{itemize}

\end{quote}

\textbf{Status:} Tested + Doc + NoDemo = OK



    \end{boxedminipage}

    \label{xformslib:flbasic:fl_add_signal_callback}
    \index{xformslib \textit{(package)}!xformslib.flbasic \textit{(module)}!xformslib.flbasic.fl\_add\_signal\_callback \textit{(function)}}

    \vspace{0.5ex}

\hspace{.8\funcindent}\begin{boxedminipage}{\funcwidth}

    \raggedright \textbf{fl\_add\_signal\_callback}(\textit{sglnum}, \textit{py\_SignalHandler}, \textit{vdata})

    \vspace{-1.5ex}

    \rule{\textwidth}{0.5\fboxrule}
\setlength{\parskip}{2ex}
    Handles the receipt of a signal by registering a callback function that
    gets called when a signal is caught (only 1 function per signal).

\setlength{\parskip}{1ex}
      \textbf{Parameters}
      \vspace{-1ex}

      \begin{quote}
        \begin{Ventry}{xxxxxxxxxxxxxxxx}

          \item[sglnum]

          signal number. Values (from signal module) e.g. SIGALRM, SIGINT, 
          ...

            {\it (type=int)}

          \item[py\_SignalHandler]

          python function to be invoked after catching the signal, no 
          return

            {\it (type=\_\_ funcname (num, ptr\_void) \_\_)}

          \item[vdata]

          argument to be passed to function

            {\it (type=None or long or pointer to xfdata.FL\_OBJECT)}

        \end{Ventry}

      \end{quote}

\textbf{Example:}
\begin{quote}
  \begin{itemize}

  \item
    \setlength{\parskip}{0.6ex}
def sglhandl(numsgl, vdata):



  \item {\textbar}-{\textgreater}{\textbar} ...



  \item fl\_add\_signal\_callback(signal.SIGALRM, sglhandl, None)



\end{itemize}

\end{quote}

\textbf{Status:} Tested + Doc + NoDemo = OK



    \end{boxedminipage}

    \label{xformslib:flbasic:fl_remove_signal_callback}
    \index{xformslib \textit{(package)}!xformslib.flbasic \textit{(module)}!xformslib.flbasic.fl\_remove\_signal\_callback \textit{(function)}}

    \vspace{0.5ex}

\hspace{.8\funcindent}\begin{boxedminipage}{\funcwidth}

    \raggedright \textbf{fl\_remove\_signal\_callback}(\textit{sglnum})

    \vspace{-1.5ex}

    \rule{\textwidth}{0.5\fboxrule}
\setlength{\parskip}{2ex}
    Removes a previously registered callback function related to a signal.

\setlength{\parskip}{1ex}
      \textbf{Parameters}
      \vspace{-1ex}

      \begin{quote}
        \begin{Ventry}{xxxxxx}

          \item[sglnum]

          signal number. Values (from signal module) e.g. SIGALRM, SIGINT, 
          ...

            {\it (type=int)}

        \end{Ventry}

      \end{quote}

\textbf{Example:} fl\_remove\_signal\_callback(signal.SIGALRM)



\textbf{Status:} Tested + Doc + NoDemo = OK



    \end{boxedminipage}

    \label{xformslib:flbasic:fl_signal_caught}
    \index{xformslib \textit{(package)}!xformslib.flbasic \textit{(module)}!xformslib.flbasic.fl\_signal\_caught \textit{(function)}}

    \vspace{0.5ex}

\hspace{.8\funcindent}\begin{boxedminipage}{\funcwidth}

    \raggedright \textbf{fl\_signal\_caught}(\textit{sglnum})

    \vspace{-1.5ex}

    \rule{\textwidth}{0.5\fboxrule}
\setlength{\parskip}{2ex}
    Informs the main loop of the delivery of the particular signal. The 
    signal is received by the application program.

\setlength{\parskip}{1ex}
      \textbf{Parameters}
      \vspace{-1ex}

      \begin{quote}
        \begin{Ventry}{xxxxxx}

          \item[sglnum]

          signal number. Values (from signal module) e.g. SIGALRM, SIGINT, 
          ...

            {\it (type=int)}

        \end{Ventry}

      \end{quote}

\textbf{Example:} fl\_signal\_caught(signal.SIGALRM)



\textbf{Status:} Tested + Doc + NoDemo = OK



    \end{boxedminipage}

    \label{xformslib:flbasic:fl_app_signal_direct}
    \index{xformslib \textit{(package)}!xformslib.flbasic \textit{(module)}!xformslib.flbasic.fl\_app\_signal\_direct \textit{(function)}}

    \vspace{0.5ex}

\hspace{.8\funcindent}\begin{boxedminipage}{\funcwidth}

    \raggedright \textbf{fl\_app\_signal\_direct}(\textit{flag})

    \vspace{-1.5ex}

    \rule{\textwidth}{0.5\fboxrule}
\setlength{\parskip}{2ex}
    Changes the default behavior of the built-in signal facilities (to be 
    called with a true value for flag prior to any use of 
    fl\_add\_signal\_callback)

\setlength{\parskip}{1ex}
      \textbf{Parameters}
      \vspace{-1ex}

      \begin{quote}
        \begin{Ventry}{xxxx}

          \item[flag]

          flag to disable/enable signal. Values 0 (disabled) or 1 (enabled)

            {\it (type=int)}

        \end{Ventry}

      \end{quote}

\textbf{Example:} fl\_app\_signal\_direct(1)



\textbf{Status:} Tested + Doc + NoDemo = OK



    \end{boxedminipage}

    \label{xformslib:flbasic:fl_add_timeout}
    \index{xformslib \textit{(package)}!xformslib.flbasic \textit{(module)}!xformslib.flbasic.fl\_add\_timeout \textit{(function)}}

    \vspace{0.5ex}

\hspace{.8\funcindent}\begin{boxedminipage}{\funcwidth}

    \raggedright \textbf{fl\_add\_timeout}(\textit{msec}, \textit{py\_TimeoutCallback}, \textit{vdata})

    \vspace{-1.5ex}

    \rule{\textwidth}{0.5\fboxrule}
\setlength{\parskip}{2ex}
    Adds a timeout callback after a specified elapsed time.

\setlength{\parskip}{1ex}
      \textbf{Parameters}
      \vspace{-1ex}

      \begin{quote}
        \begin{Ventry}{xxxxxxxxxxxxxxxxxx}

          \item[msec]

          time elapsed in milliseconds

            {\it (type=long)}

          \item[py\_TimeoutCallback]

          python function to be invoked, no return

            {\it (type=\_\_ funcname (num, ptr\_void) \_\_)}

          \item[vdata]

          user data to be passed to function

            {\it (type=None or long or pointer to xfdata.FL\_OBJECT)}

        \end{Ventry}

      \end{quote}

      \textbf{Return Value}
    \vspace{-1ex}

      \begin{quote}
      timer number id

      {\it (type=int)}

      \end{quote}

\textbf{Example:}
\begin{quote}
  \begin{itemize}

  \item
    \setlength{\parskip}{0.6ex}
def timeoutcb(num, vdata):



  \item {\textbar}-{\textgreater}{\textbar} ...



  \item timnum = fl\_add\_timeout(100, timeoutcb, None)



\end{itemize}

\end{quote}

\textbf{Status:} Tested + Doc + Demo = OK



    \end{boxedminipage}

    \label{xformslib:flbasic:fl_remove_timeout}
    \index{xformslib \textit{(package)}!xformslib.flbasic \textit{(module)}!xformslib.flbasic.fl\_remove\_timeout \textit{(function)}}

    \vspace{0.5ex}

\hspace{.8\funcindent}\begin{boxedminipage}{\funcwidth}

    \raggedright \textbf{fl\_remove\_timeout}(\textit{idnum})

    \vspace{-1.5ex}

    \rule{\textwidth}{0.5\fboxrule}
\setlength{\parskip}{2ex}
    Removes a timeout callback function (created with fl\_add\_timeout).

\setlength{\parskip}{1ex}
      \textbf{Parameters}
      \vspace{-1ex}

      \begin{quote}
        \begin{Ventry}{xxxxx}

          \item[idnum]

          timeout number id

            {\it (type=int)}

        \end{Ventry}

      \end{quote}

\textbf{Example:} fl\_remove\_timeout(timnum)



\textbf{Status:} Tested + Doc + Demo = OK



    \end{boxedminipage}

    \label{xformslib:flbasic:fl_library_version}
    \index{xformslib \textit{(package)}!xformslib.flbasic \textit{(module)}!xformslib.flbasic.fl\_library\_version \textit{(function)}}

    \vspace{0.5ex}

\hspace{.8\funcindent}\begin{boxedminipage}{\funcwidth}

    \raggedright \textbf{fl\_library\_version}()

    \vspace{-1.5ex}

    \rule{\textwidth}{0.5\fboxrule}
\setlength{\parskip}{2ex}
    Returns consolidated, major and minor version informations.

\setlength{\parskip}{1ex}
      \textbf{Return Value}
    \vspace{-1ex}

      \begin{quote}
      version\_rev (computed as 1000 * version + revision), version (e.g. 1
      in 1.x.yy), revision (e.g. 0 in x.0.yy)

      {\it (type=int, int, int)}

      \end{quote}

\textbf{Example:} compver, ver, rev = fl\_library\_version()



\textbf{Attention:} API change from XForms - upstream was fl\_library\_version(ver, rev)



\textbf{Status:} Tested + Doc + NoDemo = OK



    \end{boxedminipage}

    \label{xformslib:flbasic:fl_bgn_form}
    \index{xformslib \textit{(package)}!xformslib.flbasic \textit{(module)}!xformslib.flbasic.fl\_bgn\_form \textit{(function)}}

    \vspace{0.5ex}

\hspace{.8\funcindent}\begin{boxedminipage}{\funcwidth}

    \raggedright \textbf{fl\_bgn\_form}(\textit{formtype}, \textit{w}, \textit{h})

    \vspace{-1.5ex}

    \rule{\textwidth}{0.5\fboxrule}
\setlength{\parskip}{2ex}
    Starts the definition of a form call.

\setlength{\parskip}{1ex}
      \textbf{Parameters}
      \vspace{-1ex}

      \begin{quote}
        \begin{Ventry}{xxxxxxxx}

          \item[formtype]

          type of box that is used as a background. Values (from xfdata 
          module) FL\_NO\_BOX, FL\_UP\_BOX, FL\_DOWN\_BOX, FL\_BORDER\_BOX,
          FL\_SHADOW\_BOX, FL\_FRAME\_BOX, FL\_ROUNDED\_BOX, 
          FL\_EMBOSSED\_BOX, FL\_FLAT\_BOX, FL\_RFLAT\_BOX, 
          FL\_RSHADOW\_BOX, FL\_OVAL\_BOX, FL\_ROUNDED3D\_UPBOX, 
          FL\_ROUNDED3D\_DOWNBOX, FL\_OVAL3D\_UPBOX, FL\_OVAL3D\_DOWNBOX, 
          FL\_OVAL3D\_FRAMEBOX, FL\_OVAL3D\_EMBOSSEDBOX

            {\it (type=int)}

          \item[w]

          width of the new form in coord units

            {\it (type=int)}

          \item[h]

          height of the new form in coord units

            {\it (type=int)}

        \end{Ventry}

      \end{quote}

      \textbf{Return Value}
    \vspace{-1ex}

      \begin{quote}
      form to define (pFlForm)

      {\it (type=pointer to xfdata.FL\_FORM)}

      \end{quote}

\textbf{Example:} pform = fl\_bgn\_form(xfdata.FL\_UP\_BOX, 400, 500)



\textbf{Status:} Tested + Doc + Demo = OK



    \end{boxedminipage}

    \label{xformslib:flbasic:fl_end_form}
    \index{xformslib \textit{(package)}!xformslib.flbasic \textit{(module)}!xformslib.flbasic.fl\_end\_form \textit{(function)}}

    \vspace{0.5ex}

\hspace{.8\funcindent}\begin{boxedminipage}{\funcwidth}

    \raggedright \textbf{fl\_end\_form}()

    \vspace{-1.5ex}

    \rule{\textwidth}{0.5\fboxrule}
\setlength{\parskip}{2ex}
    Ends the definition for a form call, after all required objects have 
    been added to a form call.

\setlength{\parskip}{1ex}
\textbf{Example:} fl\_end\_form()



\textbf{Status:} Tested + Doc + Demo = OK



    \end{boxedminipage}

    \label{xformslib:flbasic:fl_do_forms}
    \index{xformslib \textit{(package)}!xformslib.flbasic \textit{(module)}!xformslib.flbasic.fl\_do\_forms \textit{(function)}}

    \vspace{0.5ex}

\hspace{.8\funcindent}\begin{boxedminipage}{\funcwidth}

    \raggedright \textbf{fl\_do\_forms}()

    \vspace{-1.5ex}

    \rule{\textwidth}{0.5\fboxrule}
\setlength{\parskip}{2ex}
    Starts the main loop of the program and returns only when the state of 
    a FL\_OBJECT (that has no callback bound to it) changes.

\setlength{\parskip}{1ex}
      \textbf{Return Value}
    \vspace{-1ex}

      \begin{quote}
      object changed (pFlObject)

      {\it (type=pointer to xfdata.FL\_OBJECT)}

      \end{quote}

\textbf{Example:} while fl\_do\_forms(): pass



\textbf{Status:} Tested + Doc + Demo = OK



    \end{boxedminipage}

    \label{xformslib:flbasic:fl_check_forms}
    \index{xformslib \textit{(package)}!xformslib.flbasic \textit{(module)}!xformslib.flbasic.fl\_check\_forms \textit{(function)}}

    \vspace{0.5ex}

\hspace{.8\funcindent}\begin{boxedminipage}{\funcwidth}

    \raggedright \textbf{fl\_check\_forms}()

    \vspace{-1.5ex}

    \rule{\textwidth}{0.5\fboxrule}
\setlength{\parskip}{2ex}
    Returns None immediately unless the state of one of xfdata.FL\_OBJECT 
    (without a callback bound to it) changed.

\setlength{\parskip}{1ex}
      \textbf{Return Value}
    \vspace{-1ex}

      \begin{quote}
      object changed (pFlObject)

      {\it (type=pointer to xfdata.FL\_OBJECT)}

      \end{quote}

\textbf{Example:} pobj = fl\_check\_forms()



\textbf{Status:} Tested + Doc + Demo = OK



    \end{boxedminipage}

    \label{xformslib:flbasic:fl_do_only_forms}
    \index{xformslib \textit{(package)}!xformslib.flbasic \textit{(module)}!xformslib.flbasic.fl\_do\_only\_forms \textit{(function)}}

    \vspace{0.5ex}

\hspace{.8\funcindent}\begin{boxedminipage}{\funcwidth}

    \raggedright \textbf{fl\_do\_only\_forms}()

    \vspace{-1.5ex}

    \rule{\textwidth}{0.5\fboxrule}
\setlength{\parskip}{2ex}
    Starts the main loop of the program and returns only when the state of 
    an object changes that has no callback bound to it. It does not handle 
    user events generated by application windows opened via fl\_winopen() 
    or similar routines.

\setlength{\parskip}{1ex}
      \textbf{Return Value}
    \vspace{-1ex}

      \begin{quote}
      object changed (pFlObject)

      {\it (type=pointer to xfdata.FL\_OBJECT)}

      \end{quote}

\textbf{Example:} pobj = fl\_do\_only\_forms()



\textbf{Status:} Tested + Doc + NoDemo = OK



    \end{boxedminipage}

    \label{xformslib:flbasic:fl_check_only_forms}
    \index{xformslib \textit{(package)}!xformslib.flbasic \textit{(module)}!xformslib.flbasic.fl\_check\_only\_forms \textit{(function)}}

    \vspace{0.5ex}

\hspace{.8\funcindent}\begin{boxedminipage}{\funcwidth}

    \raggedright \textbf{fl\_check\_only\_forms}()

    \vspace{-1.5ex}

    \rule{\textwidth}{0.5\fboxrule}
\setlength{\parskip}{2ex}
    Returns None immediately unless the state of one of the object (without
    a callback bound to it) changed. It does not handle user events 
    generated by application windows opened via fl\_winopen() or similar 
    routines.

\setlength{\parskip}{1ex}
      \textbf{Return Value}
    \vspace{-1ex}

      \begin{quote}
      object changed (pFlObject)

      {\it (type=pointer to xfdata.FL\_OBJECT)}

      \end{quote}

\textbf{Example:} pobj = fl\_check\_only\_forms()



\textbf{Status:} Tested + Doc + NoDemo = OK



    \end{boxedminipage}

    \label{xformslib:flbasic:fl_freeze_form}
    \index{xformslib \textit{(package)}!xformslib.flbasic \textit{(module)}!xformslib.flbasic.fl\_freeze\_form \textit{(function)}}

    \vspace{0.5ex}

\hspace{.8\funcindent}\begin{boxedminipage}{\funcwidth}

    \raggedright \textbf{fl\_freeze\_form}(\textit{pFlForm})

    \vspace{-1.5ex}

    \rule{\textwidth}{0.5\fboxrule}
\setlength{\parskip}{2ex}
    Redraw of a form is temporarily suspended, while changes are being 
    made, so all changes made are instead buffered internally.

\setlength{\parskip}{1ex}
      \textbf{Parameters}
      \vspace{-1ex}

      \begin{quote}
        \begin{Ventry}{xxxxxxx}

          \item[pFlForm]

          form not to be re-drawn temporarily

            {\it (type=pointer to xfdata.FL\_FORM)}

        \end{Ventry}

      \end{quote}

\textbf{Example:} fl\_freeze\_form(pform1)



\textbf{Status:} Tested + Doc + Demo = OK



    \end{boxedminipage}

    \label{xformslib:flbasic:fl_set_focus_object}
    \index{xformslib \textit{(package)}!xformslib.flbasic \textit{(module)}!xformslib.flbasic.fl\_set\_focus\_object \textit{(function)}}

    \vspace{0.5ex}

\hspace{.8\funcindent}\begin{boxedminipage}{\funcwidth}

    \raggedright \textbf{fl\_set\_focus\_object}(\textit{pFlForm}, \textit{pFlObject})

    \vspace{-1.5ex}

    \rule{\textwidth}{0.5\fboxrule}
\setlength{\parskip}{2ex}
    Sets the input focus in form to object pFlObject.

\setlength{\parskip}{1ex}
      \textbf{Parameters}
      \vspace{-1ex}

      \begin{quote}
        \begin{Ventry}{xxxxxxxxx}

          \item[pFlForm]

          form whose object has to be focused

            {\it (type=pointer to xfdata.FL\_FORM)}

          \item[pFlObject]

          object to be focused

            {\it (type=pointer to xfdata.FL\_OBJECT)}

        \end{Ventry}

      \end{quote}

\textbf{Example:} fl\_set\_focus\_object(pform, pobj)



\textbf{Status:} Tested + Doc + NoDemo = OK



    \end{boxedminipage}

    \label{xformslib:flbasic:fl_set_focus_object}
    \index{xformslib \textit{(package)}!xformslib.flbasic \textit{(module)}!xformslib.flbasic.fl\_set\_focus\_object \textit{(function)}}

    \vspace{0.5ex}

\hspace{.8\funcindent}\begin{boxedminipage}{\funcwidth}

    \raggedright \textbf{fl\_set\_object\_focus}(\textit{pFlForm}, \textit{pFlObject})

    \vspace{-1.5ex}

    \rule{\textwidth}{0.5\fboxrule}
\setlength{\parskip}{2ex}
    Sets the input focus in form to object pFlObject.

\setlength{\parskip}{1ex}
      \textbf{Parameters}
      \vspace{-1ex}

      \begin{quote}
        \begin{Ventry}{xxxxxxxxx}

          \item[pFlForm]

          form whose object has to be focused

            {\it (type=pointer to xfdata.FL\_FORM)}

          \item[pFlObject]

          object to be focused

            {\it (type=pointer to xfdata.FL\_OBJECT)}

        \end{Ventry}

      \end{quote}

\textbf{Example:} fl\_set\_focus\_object(pform, pobj)



\textbf{Status:} Tested + Doc + NoDemo = OK



    \end{boxedminipage}

    \label{xformslib:flbasic:fl_get_focus_object}
    \index{xformslib \textit{(package)}!xformslib.flbasic \textit{(module)}!xformslib.flbasic.fl\_get\_focus\_object \textit{(function)}}

    \vspace{0.5ex}

\hspace{.8\funcindent}\begin{boxedminipage}{\funcwidth}

    \raggedright \textbf{fl\_get\_focus\_object}(\textit{pFlForm})

    \vspace{-1.5ex}

    \rule{\textwidth}{0.5\fboxrule}
\setlength{\parskip}{2ex}
    Obtains the object that has the focus on a form.

\setlength{\parskip}{1ex}
      \textbf{Parameters}
      \vspace{-1ex}

      \begin{quote}
        \begin{Ventry}{xxxxxxx}

          \item[pFlForm]

          form that has a focused object in

            {\it (type=pointer to xfdata.FL\_FORM)}

        \end{Ventry}

      \end{quote}

      \textbf{Return Value}
    \vspace{-1ex}

      \begin{quote}
      focused object (pFlObject)

      {\it (type=pointer to xfdata.FL\_OBJECT)}

      \end{quote}

\textbf{Example:} pobj2 = fl\_get\_focus\_object(pform1)



\textbf{Status:} Tested + Doc + NoDemo = OK



    \end{boxedminipage}

    \label{xformslib:flbasic:fl_reset_focus_object}
    \index{xformslib \textit{(package)}!xformslib.flbasic \textit{(module)}!xformslib.flbasic.fl\_reset\_focus\_object \textit{(function)}}

    \vspace{0.5ex}

\hspace{.8\funcindent}\begin{boxedminipage}{\funcwidth}

    \raggedright \textbf{fl\_reset\_focus\_object}(\textit{pFlObject})

    \vspace{-1.5ex}

    \rule{\textwidth}{0.5\fboxrule}
\setlength{\parskip}{2ex}
    Resets focus on current object, overriding the xfdata.FL\_UNFOCUS 
    event.

\setlength{\parskip}{1ex}
      \textbf{Parameters}
      \vspace{-1ex}

      \begin{quote}
        \begin{Ventry}{xxxxxxxxx}

          \item[pFlObject]

          object towards applying event

            {\it (type=pointer to xfdata.FL\_OBJECT)}

        \end{Ventry}

      \end{quote}

\textbf{Example:} fl\_reset\_focus\_object(pobj2)



\textbf{Status:} Tested + Doc + NoDemo = OK



    \end{boxedminipage}

    \label{xformslib:flbasic:fl_set_form_atclose}
    \index{xformslib \textit{(package)}!xformslib.flbasic \textit{(module)}!xformslib.flbasic.fl\_set\_form\_atclose \textit{(function)}}

    \vspace{0.5ex}

\hspace{.8\funcindent}\begin{boxedminipage}{\funcwidth}

    \raggedright \textbf{fl\_set\_form\_atclose}(\textit{pFlForm}, \textit{py\_FormAtclose}, \textit{vdata})

    \vspace{-1.5ex}

    \rule{\textwidth}{0.5\fboxrule}
\setlength{\parskip}{2ex}
    Calls a callback function before closing the form.

\setlength{\parskip}{1ex}
      \textbf{Parameters}
      \vspace{-1ex}

      \begin{quote}
        \begin{Ventry}{xxxxxxxxxxxxxx}

          \item[pFlForm]

          form that receives the message

            {\it (type=pointer to xfdata.FL\_FORM)}

          \item[py\_FormAtclose]

          python callback function to be called, with returning value

            {\it (type=\_\_ funcname (pFlForm, ptr\_void) -{\textgreater} num \_\_)}

          \item[vdata]

          user data to be passed to function

            {\it (type=None or long or pointer to xfdata.FL\_OBJECT)}

        \end{Ventry}

      \end{quote}

      \textbf{Return Value}
    \vspace{-1ex}

      \begin{quote}
      old FL\_FORM\_ATCLOSE function

      \end{quote}

\textbf{Example:}
\begin{quote}
  \begin{itemize}

  \item
    \setlength{\parskip}{0.6ex}
def atcolsecb(pform, vdata):



  \item {\textbar}-{\textgreater}{\textbar} ...



  \item {\textbar}-{\textgreater}{\textbar} return 0



  \item oldatclosecb = fl\_set\_form\_atclose(pform1, None)



\end{itemize}

\end{quote}

\textbf{Status:} Tested + Doc + NoDemo = OK



    \end{boxedminipage}

    \label{xformslib:flbasic:fl_set_atclose}
    \index{xformslib \textit{(package)}!xformslib.flbasic \textit{(module)}!xformslib.flbasic.fl\_set\_atclose \textit{(function)}}

    \vspace{0.5ex}

\hspace{.8\funcindent}\begin{boxedminipage}{\funcwidth}

    \raggedright \textbf{fl\_set\_atclose}(\textit{py\_FormAtclose}, \textit{vdata})

    \vspace{-1.5ex}

    \rule{\textwidth}{0.5\fboxrule}
\setlength{\parskip}{2ex}
    Calls a callback function before terminating the application.

\setlength{\parskip}{1ex}
      \textbf{Parameters}
      \vspace{-1ex}

      \begin{quote}
        \begin{Ventry}{xxxxxxxxxxxxxx}

          \item[py\_FormAtclose]

          python callback function to be called, with returning value

            {\it (type=\_\_ funcname (pFlForm, ptr\_void) -{\textgreater} num \_\_)}

          \item[vdata]

          user data to be passed to function

            {\it (type=None or long or pointer to xfdata.FL\_OBJECT)}

        \end{Ventry}

      \end{quote}

      \textbf{Return Value}
    \vspace{-1ex}

      \begin{quote}
      old FL\_FORM\_ATCLOSE function

      \end{quote}

\textbf{Example:}
\begin{quote}
  \begin{itemize}

  \item
    \setlength{\parskip}{0.6ex}
def atclose(pform, vdata):



  \item {\textbar}-{\textgreater}{\textbar} ...



  \item {\textbar}-{\textgreater}{\textbar} return 0



  \item oldatclosefunc = fl\_set\_atclose(atclose, None)



\end{itemize}

\end{quote}

\textbf{Status:} Tested + Doc + NoDemo = OK



    \end{boxedminipage}

    \label{xformslib:flbasic:fl_set_form_atactivate}
    \index{xformslib \textit{(package)}!xformslib.flbasic \textit{(module)}!xformslib.flbasic.fl\_set\_form\_atactivate \textit{(function)}}

    \vspace{0.5ex}

\hspace{.8\funcindent}\begin{boxedminipage}{\funcwidth}

    \raggedright \textbf{fl\_set\_form\_atactivate}(\textit{pFlForm}, \textit{py\_FormAtactivate}, \textit{vdata})

    \vspace{-1.5ex}

    \rule{\textwidth}{0.5\fboxrule}
\setlength{\parskip}{2ex}
    Register a callback that is called when activation status of a forms is
    enabled.

\setlength{\parskip}{1ex}
      \textbf{Parameters}
      \vspace{-1ex}

      \begin{quote}
        \begin{Ventry}{xxxxxxxxxxxxxxxxx}

          \item[pFlForm]

          activated form

            {\it (type=pointer to xfdata.FL\_FORM)}

          \item[py\_FormAtactivate]

          python callback function called, no return

            {\it (type=\_\_ funcname (pFlForm, ptr\_void) \_\_)}

          \item[vdata]

          user data to be passed to function

            {\it (type=None or long or pointer to xfdata.FL\_OBJECT)}

        \end{Ventry}

      \end{quote}

      \textbf{Return Value}
    \vspace{-1ex}

      \begin{quote}
      old FL\_FORM\_ATACTIVATE function

      \end{quote}

\textbf{Example:}
\begin{quote}
  \begin{itemize}

  \item
    \setlength{\parskip}{0.6ex}
def atactcb(pform, vdata):



  \item {\textbar}-{\textgreater}{\textbar} ...



  \item oldactfunc = xf.fl\_set\_form\_atdeactivate(pform, atactcb, None)



\end{itemize}

\end{quote}

\textbf{Status:} Tested + Doc + NoDemo = OK



    \end{boxedminipage}

    \label{xformslib:flbasic:fl_set_form_atdeactivate}
    \index{xformslib \textit{(package)}!xformslib.flbasic \textit{(module)}!xformslib.flbasic.fl\_set\_form\_atdeactivate \textit{(function)}}

    \vspace{0.5ex}

\hspace{.8\funcindent}\begin{boxedminipage}{\funcwidth}

    \raggedright \textbf{fl\_set\_form\_atdeactivate}(\textit{pFlForm}, \textit{py\_FormAtdeactivate}, \textit{vdata})

    \vspace{-1.5ex}

    \rule{\textwidth}{0.5\fboxrule}
\setlength{\parskip}{2ex}
    Register a callback that is called when activation status of a forms is
    disabled.

\setlength{\parskip}{1ex}
      \textbf{Parameters}
      \vspace{-1ex}

      \begin{quote}
        \begin{Ventry}{xxxxxxxxxxxxxxxxxxx}

          \item[pFlForm]

          de-activated form

            {\it (type=pointer to xfdata.FL\_FORM)}

          \item[py\_FormAtdeactivate]

          python callback function called, no return

            {\it (type=\_\_ funcname (pFlForm, ptr\_void) \_\_)}

          \item[vdata]

          user data to be passed to function

            {\it (type=None or long or pointer to xfdata.FL\_OBJECT)}

        \end{Ventry}

      \end{quote}

      \textbf{Return Value}
    \vspace{-1ex}

      \begin{quote}
      old FL\_FORM\_ATDEACTIVATE function

      \end{quote}

\textbf{Example:}
\begin{quote}
  \begin{itemize}

  \item
    \setlength{\parskip}{0.6ex}
def atdeactcb(pform, vdata):



  \item {\textbar}-{\textgreater}{\textbar} ...



  \item oldatdeactfunc = xf.fl\_set\_form\_atdeactivate(pform, atdeactcb, None)



\end{itemize}

\end{quote}

\textbf{Status:} Tested + Doc + NoDemo = OK



    \end{boxedminipage}

    \label{xformslib:flbasic:fl_unfreeze_form}
    \index{xformslib \textit{(package)}!xformslib.flbasic \textit{(module)}!xformslib.flbasic.fl\_unfreeze\_form \textit{(function)}}

    \vspace{0.5ex}

\hspace{.8\funcindent}\begin{boxedminipage}{\funcwidth}

    \raggedright \textbf{fl\_unfreeze\_form}(\textit{pFlForm})

    \vspace{-1.5ex}

    \rule{\textwidth}{0.5\fboxrule}
\setlength{\parskip}{2ex}
    Reverts previous freeze (set with fl\_freeze\_form function), all 
    changes made in the meantime in a form are drawn at once.

\setlength{\parskip}{1ex}
      \textbf{Parameters}
      \vspace{-1ex}

      \begin{quote}
        \begin{Ventry}{xxxxxxx}

          \item[pFlForm]

          form to be re-drawn after freezing

            {\it (type=pointer to xfdata.FL\_FORM)}

        \end{Ventry}

      \end{quote}

\textbf{Example:} fl\_unfreeze\_form(pform)



\textbf{Status:} Tested + Doc + Demo = OK



    \end{boxedminipage}

    \label{xformslib:flbasic:fl_deactivate_form}
    \index{xformslib \textit{(package)}!xformslib.flbasic \textit{(module)}!xformslib.flbasic.fl\_deactivate\_form \textit{(function)}}

    \vspace{0.5ex}

\hspace{.8\funcindent}\begin{boxedminipage}{\funcwidth}

    \raggedright \textbf{fl\_deactivate\_form}(\textit{pFlForm})

    \vspace{-1.5ex}

    \rule{\textwidth}{0.5\fboxrule}
\setlength{\parskip}{2ex}
    Deactivates form temporarily, without hiding it, but not allowing a 
    user to interact with elements contained in form (buttons, etc.).

\setlength{\parskip}{1ex}
      \textbf{Parameters}
      \vspace{-1ex}

      \begin{quote}
        \begin{Ventry}{xxxxxxx}

          \item[pFlForm]

          form to be de-activated

            {\it (type=pointer to xfdata.FL\_FORM)}

        \end{Ventry}

      \end{quote}

\textbf{Example:} fl\_deactivate\_form(pform)



\textbf{Status:} Tested + Doc + Demo = OK



    \end{boxedminipage}

    \label{xformslib:flbasic:fl_activate_form}
    \index{xformslib \textit{(package)}!xformslib.flbasic \textit{(module)}!xformslib.flbasic.fl\_activate\_form \textit{(function)}}

    \vspace{0.5ex}

\hspace{.8\funcindent}\begin{boxedminipage}{\funcwidth}

    \raggedright \textbf{fl\_activate\_form}(\textit{pFlForm})

    \vspace{-1.5ex}

    \rule{\textwidth}{0.5\fboxrule}
\setlength{\parskip}{2ex}
    (Re)activates form (deactivated with fl\_deactivate\_form), allowing 
    the user to interact again with elements contained in form (buttons, 
    etc.).

\setlength{\parskip}{1ex}
      \textbf{Parameters}
      \vspace{-1ex}

      \begin{quote}
        \begin{Ventry}{xxxxxxx}

          \item[pFlForm]

          form to be re-activated

            {\it (type=pointer to xfdata.FL\_FORM)}

        \end{Ventry}

      \end{quote}

\textbf{Example:} fl\_activate\_form(pform)



\textbf{Status:} Tested + Doc + Demo = OK



    \end{boxedminipage}

    \label{xformslib:flbasic:fl_deactivate_all_forms}
    \index{xformslib \textit{(package)}!xformslib.flbasic \textit{(module)}!xformslib.flbasic.fl\_deactivate\_all\_forms \textit{(function)}}

    \vspace{0.5ex}

\hspace{.8\funcindent}\begin{boxedminipage}{\funcwidth}

    \raggedright \textbf{fl\_deactivate\_all\_forms}()

    \vspace{-1.5ex}

    \rule{\textwidth}{0.5\fboxrule}
\setlength{\parskip}{2ex}
    De-activates all current forms, forbidding any event/user interaction.

\setlength{\parskip}{1ex}
\textbf{Example:} fl\_deactivate\_all\_forms()



\textbf{Status:} Tested + Doc + NoDemo = OK



    \end{boxedminipage}

    \label{xformslib:flbasic:fl_activate_all_forms}
    \index{xformslib \textit{(package)}!xformslib.flbasic \textit{(module)}!xformslib.flbasic.fl\_activate\_all\_forms \textit{(function)}}

    \vspace{0.5ex}

\hspace{.8\funcindent}\begin{boxedminipage}{\funcwidth}

    \raggedright \textbf{fl\_activate\_all\_forms}()

    \vspace{-1.5ex}

    \rule{\textwidth}{0.5\fboxrule}
\setlength{\parskip}{2ex}
    (Re)activates all current forms, allowing event/user interaction.

\setlength{\parskip}{1ex}
\textbf{Example:} fl\_activate\_all\_forms()



\textbf{Status:} Tested + Doc + NoDemo = OK



    \end{boxedminipage}

    \label{xformslib:flbasic:fl_freeze_all_forms}
    \index{xformslib \textit{(package)}!xformslib.flbasic \textit{(module)}!xformslib.flbasic.fl\_freeze\_all\_forms \textit{(function)}}

    \vspace{0.5ex}

\hspace{.8\funcindent}\begin{boxedminipage}{\funcwidth}

    \raggedright \textbf{fl\_freeze\_all\_forms}()

    \vspace{-1.5ex}

    \rule{\textwidth}{0.5\fboxrule}
\setlength{\parskip}{2ex}
    All current forms are not temporarily redrawn, while changes are being 
    made and are instead buffered internally.

\setlength{\parskip}{1ex}
\textbf{Example:} fl\_freeze\_all\_forms()



\textbf{Status:} Tested + Doc + NoDemo = OK



    \end{boxedminipage}

    \label{xformslib:flbasic:fl_unfreeze_all_forms}
    \index{xformslib \textit{(package)}!xformslib.flbasic \textit{(module)}!xformslib.flbasic.fl\_unfreeze\_all\_forms \textit{(function)}}

    \vspace{0.5ex}

\hspace{.8\funcindent}\begin{boxedminipage}{\funcwidth}

    \raggedright \textbf{fl\_unfreeze\_all\_forms}()

    \vspace{-1.5ex}

    \rule{\textwidth}{0.5\fboxrule}
\setlength{\parskip}{2ex}
    All changes made in the meantime in all current forms are drawn at 
    once, reverting previous freeze.

\setlength{\parskip}{1ex}
\textbf{Example:} fl\_unfreeze\_all\_forms()



\textbf{Status:} Tested + Doc + NoDemo = OK



    \end{boxedminipage}

    \label{xformslib:flbasic:fl_scale_form}
    \index{xformslib \textit{(package)}!xformslib.flbasic \textit{(module)}!xformslib.flbasic.fl\_scale\_form \textit{(function)}}

    \vspace{0.5ex}

\hspace{.8\funcindent}\begin{boxedminipage}{\funcwidth}

    \raggedright \textbf{fl\_scale\_form}(\textit{pFlForm}, \textit{xsc}, \textit{ysc})

    \vspace{-1.5ex}

    \rule{\textwidth}{0.5\fboxrule}
\setlength{\parskip}{2ex}
    Scales a form and the objects on it in size and position, indicating a 
    scaling factor in x- and y-direction (1.1 = 110 percent, 0.5 = 50, 
    etc.) with respect to the current size, and reshapes the window.

\setlength{\parskip}{1ex}
      \textbf{Parameters}
      \vspace{-1ex}

      \begin{quote}
        \begin{Ventry}{xxxxxxx}

          \item[pFlForm]

          form to be scaled

            {\it (type=pointer to xfdata.FL\_FORM)}

          \item[xsc]

          scaling factor in horizontal direction

            {\it (type=float)}

          \item[ysc]

          scaling factor in vertical direction

            {\it (type=float)}

        \end{Ventry}

      \end{quote}

\textbf{Example:} fl\_scale\_form(pform, 0.8, 1.2)



\textbf{Status:} Tested + Doc + Demo = OK



    \end{boxedminipage}

    \label{xformslib:flbasic:fl_set_form_position}
    \index{xformslib \textit{(package)}!xformslib.flbasic \textit{(module)}!xformslib.flbasic.fl\_set\_form\_position \textit{(function)}}

    \vspace{0.5ex}

\hspace{.8\funcindent}\begin{boxedminipage}{\funcwidth}

    \raggedright \textbf{fl\_set\_form\_position}(\textit{pFlForm}, \textit{x}, \textit{y})

    \vspace{-1.5ex}

    \rule{\textwidth}{0.5\fboxrule}
\setlength{\parskip}{2ex}
    Sets position of form, when placing a form on the screen with 
    xfdata.FL\_PLACE\_GEOMETRY as place argument.

\setlength{\parskip}{1ex}
      \textbf{Parameters}
      \vspace{-1ex}

      \begin{quote}
        \begin{Ventry}{xxxxxxx}

          \item[pFlForm]

          form whose position is to be set

            {\it (type=pointer to xfdata.FL\_FORM)}

          \item[x]

          horizontal position (upper-left corner)

            {\it (type=int)}

          \item[y]

          vertical position (upper-left corner)

            {\it (type=int)}

        \end{Ventry}

      \end{quote}

\textbf{Example:} fl\_set\_form\_position(pform, 125, 250)



\textbf{Status:} Tested + Doc + Demo = OK



    \end{boxedminipage}

    \label{xformslib:flbasic:fl_set_form_title}
    \index{xformslib \textit{(package)}!xformslib.flbasic \textit{(module)}!xformslib.flbasic.fl\_set\_form\_title \textit{(function)}}

    \vspace{0.5ex}

\hspace{.8\funcindent}\begin{boxedminipage}{\funcwidth}

    \raggedright \textbf{fl\_set\_form\_title}(\textit{pFlForm}, \textit{title})

    \vspace{-1.5ex}

    \rule{\textwidth}{0.5\fboxrule}
\setlength{\parskip}{2ex}
    Changes the form title (and the icon name) after it is shown.

\setlength{\parskip}{1ex}
      \textbf{Parameters}
      \vspace{-1ex}

      \begin{quote}
        \begin{Ventry}{xxxxxxx}

          \item[pFlForm]

          form whose title has to be changed

            {\it (type=pointer to xfdata.FL\_FORM)}

          \item[title]

          new title text for the form

            {\it (type=str)}

        \end{Ventry}

      \end{quote}

\textbf{Example:} fl\_set\_form\_title(pform, "My great form")



\textbf{Status:} Tested + Doc + NoDemo = OK



    \end{boxedminipage}

    \label{xformslib:flbasic:fl_set_app_mainform}
    \index{xformslib \textit{(package)}!xformslib.flbasic \textit{(module)}!xformslib.flbasic.fl\_set\_app\_mainform \textit{(function)}}

    \vspace{0.5ex}

\hspace{.8\funcindent}\begin{boxedminipage}{\funcwidth}

    \raggedright \textbf{fl\_set\_app\_mainform}(\textit{pFlForm})

    \vspace{-1.5ex}

    \rule{\textwidth}{0.5\fboxrule}
\setlength{\parskip}{2ex}
    Designates the main form. By default, the main form is set 
    automatically by the library to the first full-bordered form shown.

\setlength{\parskip}{1ex}
      \textbf{Parameters}
      \vspace{-1ex}

      \begin{quote}
        \begin{Ventry}{xxxxxxx}

          \item[pFlForm]

          form to be set as main one

            {\it (type=pointer to xfdata.FL\_FORM)}

        \end{Ventry}

      \end{quote}

\textbf{Example:} fl\_set\_app\_mainform(pform2)



\textbf{Status:} Tested + Doc + Demo = OK



    \end{boxedminipage}

    \label{xformslib:flbasic:fl_get_app_mainform}
    \index{xformslib \textit{(package)}!xformslib.flbasic \textit{(module)}!xformslib.flbasic.fl\_get\_app\_mainform \textit{(function)}}

    \vspace{0.5ex}

\hspace{.8\funcindent}\begin{boxedminipage}{\funcwidth}

    \raggedright \textbf{fl\_get\_app\_mainform}()

    \vspace{-1.5ex}

    \rule{\textwidth}{0.5\fboxrule}
\setlength{\parskip}{2ex}
    Returns the current mainform.

\setlength{\parskip}{1ex}
      \textbf{Return Value}
    \vspace{-1ex}

      \begin{quote}
      main form (pFlForm)

      {\it (type=pointer to xfdata.FL\_FORM)}

      \end{quote}

\textbf{Example:} fl\_get\_app\_mainform()



\textbf{Status:} Tested + Doc + NoDemo = OK



    \end{boxedminipage}

    \label{xformslib:flbasic:fl_set_app_nomainform}
    \index{xformslib \textit{(package)}!xformslib.flbasic \textit{(module)}!xformslib.flbasic.fl\_set\_app\_nomainform \textit{(function)}}

    \vspace{0.5ex}

\hspace{.8\funcindent}\begin{boxedminipage}{\funcwidth}

    \raggedright \textbf{fl\_set\_app\_nomainform}(\textit{flag})

    \vspace{-1.5ex}

    \rule{\textwidth}{0.5\fboxrule}
\setlength{\parskip}{2ex}
    In some situations, either because the concept of an application main 
    form does not apply (for example, an application might have multiple 
    full-bordered windows), or under some (buggy) window managers, the 
    designation of a main form may cause stacking order problems. To 
    workaround these, it can disable the designation of a main form (must 
    be called before any full-bordered form is shown)

\setlength{\parskip}{1ex}
      \textbf{Parameters}
      \vspace{-1ex}

      \begin{quote}
        \begin{Ventry}{xxxx}

          \item[flag]

          flag to disable/enable mainform designation. Values 1 (to 
          disable) or 0 (to enable)

            {\it (type=int)}

        \end{Ventry}

      \end{quote}

\textbf{Example:} fl\_set\_app\_nomainform(1)



\textbf{Status:} Tested + Doc + NoDemo = OK



    \end{boxedminipage}

    \label{xformslib:flbasic:fl_set_form_callback}
    \index{xformslib \textit{(package)}!xformslib.flbasic \textit{(module)}!xformslib.flbasic.fl\_set\_form\_callback \textit{(function)}}

    \vspace{0.5ex}

\hspace{.8\funcindent}\begin{boxedminipage}{\funcwidth}

    \raggedright \textbf{fl\_set\_form\_callback}(\textit{pFlForm}, \textit{py\_FormCallbackPtr}, \textit{vdata})

    \vspace{-1.5ex}

    \rule{\textwidth}{0.5\fboxrule}
\setlength{\parskip}{2ex}
    Sets the callback function bound to an entire form. Whenever 
    fl\_do\_forms() or fl\_check\_forms() would return an object in form 
    they call the routine callback instead, with the object as an argument.
    So with each form you can associate its own callback routine. For 
    objects that have their own callbacks, the object callbacks have 
    priority over the form callback.

\setlength{\parskip}{1ex}
      \textbf{Parameters}
      \vspace{-1ex}

      \begin{quote}
        \begin{Ventry}{xxxxxxxxxxxxxxxxxx}

          \item[pFlForm]

          form whose callback has to be set

            {\it (type=pointer to xfdata.FL\_FORM)}

          \item[py\_FormCallbackPtr]

          python callback to be set, no return

            {\it (type=\_\_ funcname (pFlObject, ptr\_void) \_\_)}

          \item[vdata]

          user data to be passed to function

            {\it (type=None or long or pointer to xfdata.FL\_OBJECT)}

        \end{Ventry}

      \end{quote}

\textbf{Example:}
\begin{quote}
  \begin{itemize}

  \item
    \setlength{\parskip}{0.6ex}
def formcb(pobj, vdata):



  \item {\textbar}-{\textgreater}{\textbar} ...



  \item fl\_set\_form\_callback(pform, formcb, None)



\end{itemize}

\end{quote}

\textbf{Status:} Tested + Doc + Demo = OK



    \end{boxedminipage}

    \label{xformslib:flbasic:fl_set_form_callback}
    \index{xformslib \textit{(package)}!xformslib.flbasic \textit{(module)}!xformslib.flbasic.fl\_set\_form\_callback \textit{(function)}}

    \vspace{0.5ex}

\hspace{.8\funcindent}\begin{boxedminipage}{\funcwidth}

    \raggedright \textbf{fl\_set\_form\_call\_back}(\textit{pFlForm}, \textit{py\_FormCallbackPtr}, \textit{vdata})

    \vspace{-1.5ex}

    \rule{\textwidth}{0.5\fboxrule}
\setlength{\parskip}{2ex}
    Sets the callback function bound to an entire form. Whenever 
    fl\_do\_forms() or fl\_check\_forms() would return an object in form 
    they call the routine callback instead, with the object as an argument.
    So with each form you can associate its own callback routine. For 
    objects that have their own callbacks, the object callbacks have 
    priority over the form callback.

\setlength{\parskip}{1ex}
      \textbf{Parameters}
      \vspace{-1ex}

      \begin{quote}
        \begin{Ventry}{xxxxxxxxxxxxxxxxxx}

          \item[pFlForm]

          form whose callback has to be set

            {\it (type=pointer to xfdata.FL\_FORM)}

          \item[py\_FormCallbackPtr]

          python callback to be set, no return

            {\it (type=\_\_ funcname (pFlObject, ptr\_void) \_\_)}

          \item[vdata]

          user data to be passed to function

            {\it (type=None or long or pointer to xfdata.FL\_OBJECT)}

        \end{Ventry}

      \end{quote}

\textbf{Example:}
\begin{quote}
  \begin{itemize}

  \item
    \setlength{\parskip}{0.6ex}
def formcb(pobj, vdata):



  \item {\textbar}-{\textgreater}{\textbar} ...



  \item fl\_set\_form\_callback(pform, formcb, None)



\end{itemize}

\end{quote}

\textbf{Status:} Tested + Doc + Demo = OK



    \end{boxedminipage}

    \label{xformslib:flbasic:fl_set_form_size}
    \index{xformslib \textit{(package)}!xformslib.flbasic \textit{(module)}!xformslib.flbasic.fl\_set\_form\_size \textit{(function)}}

    \vspace{0.5ex}

\hspace{.8\funcindent}\begin{boxedminipage}{\funcwidth}

    \raggedright \textbf{fl\_set\_form\_size}(\textit{pFlForm}, \textit{w}, \textit{h})

    \vspace{-1.5ex}

    \rule{\textwidth}{0.5\fboxrule}
\setlength{\parskip}{2ex}
    Sets the size of form.

\setlength{\parskip}{1ex}
      \textbf{Parameters}
      \vspace{-1ex}

      \begin{quote}
        \begin{Ventry}{xxxxxxx}

          \item[pFlForm]

          form whose size has to be set

            {\it (type=pointer to xfdata.FL\_FORM)}

          \item[w]

          width of form in coord units

            {\it (type=int)}

          \item[h]

          height of form in coord units

            {\it (type=int)}

        \end{Ventry}

      \end{quote}

\textbf{Example:} fl\_set\_form\_size(pform, 200, 200)



\textbf{Status:} Tested + Doc + Demo = OK



    \end{boxedminipage}

    \label{xformslib:flbasic:fl_set_form_hotspot}
    \index{xformslib \textit{(package)}!xformslib.flbasic \textit{(module)}!xformslib.flbasic.fl\_set\_form\_hotspot \textit{(function)}}

    \vspace{0.5ex}

\hspace{.8\funcindent}\begin{boxedminipage}{\funcwidth}

    \raggedright \textbf{fl\_set\_form\_hotspot}(\textit{pFlForm}, \textit{x}, \textit{y})

    \vspace{-1.5ex}

    \rule{\textwidth}{0.5\fboxrule}
\setlength{\parskip}{2ex}
    Sets the position of the hotspot, for showing a form so that a 
    particular point is under the mouse. You have to use 
    xfdata.FL\_PLACE\_HOTSPOT as place argument in fl\_show\_form().

\setlength{\parskip}{1ex}
      \textbf{Parameters}
      \vspace{-1ex}

      \begin{quote}
        \begin{Ventry}{xxxxxxx}

          \item[pFlForm]

          form to be set

            {\it (type=pointer to xfdata.FL\_FORM)}

          \item[x]

          horizontal position (upper-left corner)

            {\it (type=int)}

          \item[y]

          vertical position (upper-left corner)

            {\it (type=int)}

        \end{Ventry}

      \end{quote}

\textbf{Example:} fl\_set\_form\_hotspot(pform, 300, 50)



\textbf{Status:} Tested + Doc + NoDemo = OK



    \end{boxedminipage}

    \label{xformslib:flbasic:fl_set_form_hotobject}
    \index{xformslib \textit{(package)}!xformslib.flbasic \textit{(module)}!xformslib.flbasic.fl\_set\_form\_hotobject \textit{(function)}}

    \vspace{0.5ex}

\hspace{.8\funcindent}\begin{boxedminipage}{\funcwidth}

    \raggedright \textbf{fl\_set\_form\_hotobject}(\textit{pFlForm}, \textit{pFlObject})

    \vspace{-1.5ex}

    \rule{\textwidth}{0.5\fboxrule}
\setlength{\parskip}{2ex}
    Sets the hotspot for showing a form so that a particular object is 
    under the mouse. You have to use xfdata.FL\_PLACE\_HOTSPOT as place 
    argument in fl\_show\_form().

\setlength{\parskip}{1ex}
      \textbf{Parameters}
      \vspace{-1ex}

      \begin{quote}
        \begin{Ventry}{xxxxxxxxx}

          \item[pFlForm]

          form whose object has to be set

            {\it (type=pointer to xfdata.FL\_FORM)}

          \item[pFlObject]

          object

            {\it (type=pointer to xfdata.FL\_OBJECT)}

        \end{Ventry}

      \end{quote}

\textbf{Example:} fl\_set\_form\_hotobject(pform, pobj)



\textbf{Status:} Tested + Doc + Demo = OK



    \end{boxedminipage}

    \label{xformslib:flbasic:fl_set_form_minsize}
    \index{xformslib \textit{(package)}!xformslib.flbasic \textit{(module)}!xformslib.flbasic.fl\_set\_form\_minsize \textit{(function)}}

    \vspace{0.5ex}

\hspace{.8\funcindent}\begin{boxedminipage}{\funcwidth}

    \raggedright \textbf{fl\_set\_form\_minsize}(\textit{pFlForm}, \textit{w}, \textit{h})

    \vspace{-1.5ex}

    \rule{\textwidth}{0.5\fboxrule}
\setlength{\parskip}{2ex}
    Sets the minimum size a form can have, if interactive resizing is 
    allowed (e.g., by showing the form with xfdata.FL\_PLACE\_POSITION).

\setlength{\parskip}{1ex}
      \textbf{Parameters}
      \vspace{-1ex}

      \begin{quote}
        \begin{Ventry}{xxxxxxx}

          \item[pFlForm]

          form

            {\it (type=pointer to xfdata.FL\_FORM)}

          \item[w]

          width of form in coord units

            {\it (type=int)}

          \item[h]

          height of form in coord units

            {\it (type=int)}

        \end{Ventry}

      \end{quote}

\textbf{Example:} fl\_set\_form\_minsize(pform, 200, 300)



\textbf{Status:} Tested + Doc + NoDemo = OK



    \end{boxedminipage}

    \label{xformslib:flbasic:fl_set_form_maxsize}
    \index{xformslib \textit{(package)}!xformslib.flbasic \textit{(module)}!xformslib.flbasic.fl\_set\_form\_maxsize \textit{(function)}}

    \vspace{0.5ex}

\hspace{.8\funcindent}\begin{boxedminipage}{\funcwidth}

    \raggedright \textbf{fl\_set\_form\_maxsize}(\textit{pFlForm}, \textit{w}, \textit{h})

    \vspace{-1.5ex}

    \rule{\textwidth}{0.5\fboxrule}
\setlength{\parskip}{2ex}
    Sets the maximum size a form can have, if interactive resizing is 
    allowed (e.g. by showing the form with xfdata.FL\_PLACE\_POSITION).

\setlength{\parskip}{1ex}
      \textbf{Parameters}
      \vspace{-1ex}

      \begin{quote}
        \begin{Ventry}{xxxxxxx}

          \item[pFlForm]

          form whose size has to be set

            {\it (type=pointer to xdata.FL\_FORM)}

          \item[w]

          width of form in coord units

            {\it (type=int)}

          \item[h]

          height of form in coord units

            {\it (type=int)}

        \end{Ventry}

      \end{quote}

\textbf{Example:} fl\_set\_form\_maxsize(pform, 400, 450)



\textbf{Status:} Tested + Doc + NoDemo = OK



    \end{boxedminipage}

    \label{xformslib:flbasic:fl_set_form_event_cmask}
    \index{xformslib \textit{(package)}!xformslib.flbasic \textit{(module)}!xformslib.flbasic.fl\_set\_form\_event\_cmask \textit{(function)}}

    \vspace{0.5ex}

\hspace{.8\funcindent}\begin{boxedminipage}{\funcwidth}

    \raggedright \textbf{fl\_set\_form\_event\_cmask}(\textit{pFlForm}, \textit{cmask})

    \vspace{-1.5ex}

    \rule{\textwidth}{0.5\fboxrule}
\setlength{\parskip}{2ex}
    Sets the event compress mask a form can react to.

\setlength{\parskip}{1ex}
      \textbf{Parameters}
      \vspace{-1ex}

      \begin{quote}
        \begin{Ventry}{xxxxxxx}

          \item[pFlForm]

          form

            {\it (type=pointer to xfdata.FL\_FORM)}

          \item[cmask]

          event compress mask for form. Values (from xfdata module) one or 
          more OR-ed between NoEventMask, KeyPressMask, KeyReleaseMask, 
          ButtonPressMask, ButtonReleaseMask, EnterWindowMask, 
          LeaveWindowMask, PointerMotionMask, PointerMotionHintMask, 
          Button1MotionMask, Button2MotionMask, Button3MotionMask, 
          Button4MotionMask, Button5MotionMask, ButtonMotionMask, 
          KeymapStateMask, ExposureMask, VisibilityChangeMask, 
          StructureNotifyMask, ResizeRedirectMask, SubstructureNotifyMask, 
          SubstructureRedirectMask, FocusChangeMask, ColormapChangeMask, 
          OwnerGrabButtonMask, FL\_ALL\_EVENT, ... ?

            {\it (type=long\_pos)}

        \end{Ventry}

      \end{quote}

\textbf{Example:} fl\_set\_form\_event\_cmask(pform, xfdata.FL\_ALL\_EVENT)



\textbf{Status:} Tested + Doc + NoDemo = OK



    \end{boxedminipage}

    \label{xformslib:flbasic:fl_get_form_event_cmask}
    \index{xformslib \textit{(package)}!xformslib.flbasic \textit{(module)}!xformslib.flbasic.fl\_get\_form\_event\_cmask \textit{(function)}}

    \vspace{0.5ex}

\hspace{.8\funcindent}\begin{boxedminipage}{\funcwidth}

    \raggedright \textbf{fl\_get\_form\_event\_cmask}(\textit{pFlForm})

    \vspace{-1.5ex}

    \rule{\textwidth}{0.5\fboxrule}
\setlength{\parskip}{2ex}
    Returns event compress mask a form can react to.

\setlength{\parskip}{1ex}
      \textbf{Parameters}
      \vspace{-1ex}

      \begin{quote}
        \begin{Ventry}{xxxxxxx}

          \item[pFlForm]

          form

            {\it (type=pointer to xfdata.FL\_FORM)}

        \end{Ventry}

      \end{quote}

      \textbf{Return Value}
    \vspace{-1ex}

      \begin{quote}
      event compress mask id

      {\it (type=long\_pos)}

      \end{quote}

\textbf{Example:} cmaskid = fl\_get\_form\_event\_cmask(pform)



\textbf{Status:} Tested + Doc + NoDemo = OK



    \end{boxedminipage}

    \label{xformslib:flbasic:fl_set_form_geometry}
    \index{xformslib \textit{(package)}!xformslib.flbasic \textit{(module)}!xformslib.flbasic.fl\_set\_form\_geometry \textit{(function)}}

    \vspace{0.5ex}

\hspace{.8\funcindent}\begin{boxedminipage}{\funcwidth}

    \raggedright \textbf{fl\_set\_form\_geometry}(\textit{pFlForm}, \textit{x}, \textit{y}, \textit{w}, \textit{h})

    \vspace{-1.5ex}

    \rule{\textwidth}{0.5\fboxrule}
\setlength{\parskip}{2ex}
    Sets the geometry (position and size) of a form.

\setlength{\parskip}{1ex}
      \textbf{Parameters}
      \vspace{-1ex}

      \begin{quote}
        \begin{Ventry}{xxxxxxx}

          \item[pFlForm]

          pointer to form to be set

            {\it (type=pointer to xfdata.FL\_FORM)}

          \item[x]

          horizontal position (upper-left corner)

            {\it (type=int)}

          \item[y]

          vertical position (upper-left corner)

            {\it (type=int)}

          \item[w]

          width of form in coord units

            {\it (type=int)}

          \item[h]

          height of form in coord units

            {\it (type=int)}

        \end{Ventry}

      \end{quote}

\textbf{Example:} fl\_set\_form\_geometry(pform, 300, 400, 150, 150)



\textbf{Status:} Tested + Doc + Demo = OK



    \end{boxedminipage}

    \label{xformslib:flbasic:fl_set_form_geometry}
    \index{xformslib \textit{(package)}!xformslib.flbasic \textit{(module)}!xformslib.flbasic.fl\_set\_form\_geometry \textit{(function)}}

    \vspace{0.5ex}

\hspace{.8\funcindent}\begin{boxedminipage}{\funcwidth}

    \raggedright \textbf{fl\_set\_initial\_placement}(\textit{pFlForm}, \textit{x}, \textit{y}, \textit{w}, \textit{h})

    \vspace{-1.5ex}

    \rule{\textwidth}{0.5\fboxrule}
\setlength{\parskip}{2ex}
    Sets the geometry (position and size) of a form.

\setlength{\parskip}{1ex}
      \textbf{Parameters}
      \vspace{-1ex}

      \begin{quote}
        \begin{Ventry}{xxxxxxx}

          \item[pFlForm]

          pointer to form to be set

            {\it (type=pointer to xfdata.FL\_FORM)}

          \item[x]

          horizontal position (upper-left corner)

            {\it (type=int)}

          \item[y]

          vertical position (upper-left corner)

            {\it (type=int)}

          \item[w]

          width of form in coord units

            {\it (type=int)}

          \item[h]

          height of form in coord units

            {\it (type=int)}

        \end{Ventry}

      \end{quote}

\textbf{Example:} fl\_set\_form\_geometry(pform, 300, 400, 150, 150)



\textbf{Status:} Tested + Doc + Demo = OK



    \end{boxedminipage}

    \label{xformslib:flbasic:fl_show_form}
    \index{xformslib \textit{(package)}!xformslib.flbasic \textit{(module)}!xformslib.flbasic.fl\_show\_form \textit{(function)}}

    \vspace{0.5ex}

\hspace{.8\funcindent}\begin{boxedminipage}{\funcwidth}

    \raggedright \textbf{fl\_show\_form}(\textit{pFlForm}, \textit{place}, \textit{border}, \textit{title})

    \vspace{-1.5ex}

    \rule{\textwidth}{0.5\fboxrule}
\setlength{\parskip}{2ex}
    Shows the form.

\setlength{\parskip}{1ex}
      \textbf{Parameters}
      \vspace{-1ex}

      \begin{quote}
        \begin{Ventry}{xxxxxxx}

          \item[pFlForm]

          form to be shown

            {\it (type=pointer to xfdata.FL\_FORM)}

          \item[place]

          where form has to be placed. Values (from xfdata module) 
          FL\_PLACE\_FREE, FL\_PLACE\_MOUSE, FL\_PLACE\_CENTER, 
          FL\_PLACE\_POSITION, FL\_PLACE\_SIZE, FL\_PLACE\_GEOMETRY, 
          FL\_PLACE\_ASPECT, FL\_PLACE\_FULLSCREEN, FL\_PLACE\_HOTSPOT, 
          FL\_PLACE\_ICONIC, FL\_FREE\_SIZE, FL\_PLACE\_FREE\_CENTER, 
          FL\_PLACE\_CENTERFREE

            {\it (type=int)}

          \item[border]

          window manager decoration. Values (from xfdata module) 
          FL\_FULLBORDER, FL\_TRANSIENT, FL\_NOBORDER

            {\it (type=int)}

          \item[title]

          title of form

            {\it (type=str)}

        \end{Ventry}

      \end{quote}

      \textbf{Return Value}
    \vspace{-1ex}

      \begin{quote}
      window id (win)

      {\it (type=long\_pos)}

      \end{quote}

\textbf{Example:} wind = fl\_show\_form(pform0, FL\_PLACE\_FREE, FL\_FULLBORDER, "MyForm")



\textbf{Status:} Tested + Doc + Demo = OK



    \end{boxedminipage}

    \label{xformslib:flbasic:fl_hide_form}
    \index{xformslib \textit{(package)}!xformslib.flbasic \textit{(module)}!xformslib.flbasic.fl\_hide\_form \textit{(function)}}

    \vspace{0.5ex}

\hspace{.8\funcindent}\begin{boxedminipage}{\funcwidth}

    \raggedright \textbf{fl\_hide\_form}(\textit{pFlForm})

    \vspace{-1.5ex}

    \rule{\textwidth}{0.5\fboxrule}
\setlength{\parskip}{2ex}
    Hides the form.

\setlength{\parskip}{1ex}
      \textbf{Parameters}
      \vspace{-1ex}

      \begin{quote}
        \begin{Ventry}{xxxxxxx}

          \item[pFlForm]

          form to be hidden

            {\it (type=pointer to xfdata.FL\_FORM)}

        \end{Ventry}

      \end{quote}

\textbf{Example:} fl\_hide\_form(pform0)



\textbf{Status:} Tested + Doc + Demo = OK



    \end{boxedminipage}

    \label{xformslib:flbasic:fl_free_form}
    \index{xformslib \textit{(package)}!xformslib.flbasic \textit{(module)}!xformslib.flbasic.fl\_free\_form \textit{(function)}}

    \vspace{0.5ex}

\hspace{.8\funcindent}\begin{boxedminipage}{\funcwidth}

    \raggedright \textbf{fl\_free\_form}(\textit{pFlForm})

    \vspace{-1.5ex}

    \rule{\textwidth}{0.5\fboxrule}
\setlength{\parskip}{2ex}
    Frees the memory used by a form, hiding and deleting it together with 
    all its objects.

\setlength{\parskip}{1ex}
      \textbf{Parameters}
      \vspace{-1ex}

      \begin{quote}
        \begin{Ventry}{xxxxxxx}

          \item[pFlForm]

          form to be freed

            {\it (type=pointer to xfdata.FL\_FORM)}

        \end{Ventry}

      \end{quote}

\textbf{Example:} fl\_free\_form(pform0)



\textbf{Status:} Tested + Doc + Demo = OK



    \end{boxedminipage}

    \label{xformslib:flbasic:fl_redraw_form}
    \index{xformslib \textit{(package)}!xformslib.flbasic \textit{(module)}!xformslib.flbasic.fl\_redraw\_form \textit{(function)}}

    \vspace{0.5ex}

\hspace{.8\funcindent}\begin{boxedminipage}{\funcwidth}

    \raggedright \textbf{fl\_redraw\_form}(\textit{pFlForm})

    \vspace{-1.5ex}

    \rule{\textwidth}{0.5\fboxrule}
\setlength{\parskip}{2ex}
    (Re)draws an entire form.

\setlength{\parskip}{1ex}
      \textbf{Parameters}
      \vspace{-1ex}

      \begin{quote}
        \begin{Ventry}{xxxxxxx}

          \item[pFlForm]

          form to redraw

            {\it (type=pointer to xfdata.FL\_FORM)}

        \end{Ventry}

      \end{quote}

\textbf{Example:} fl\_redraw\_form(pform0)



\textbf{Status:} Tested + Doc + Demo = OK



    \end{boxedminipage}

    \label{xformslib:flbasic:fl_set_form_dblbuffer}
    \index{xformslib \textit{(package)}!xformslib.flbasic \textit{(module)}!xformslib.flbasic.fl\_set\_form\_dblbuffer \textit{(function)}}

    \vspace{0.5ex}

\hspace{.8\funcindent}\begin{boxedminipage}{\funcwidth}

    \raggedright \textbf{fl\_set\_form\_dblbuffer}(\textit{pFlForm}, \textit{flag})

    \vspace{-1.5ex}

    \rule{\textwidth}{0.5\fboxrule}
\setlength{\parskip}{2ex}
    Uses double buffering on a per-form basis. Since Xlib doesn't support 
    double buffering, XForms library simulates this functionality with 
    pixmap bit-bliting. In practice, the effect is hardly distinguishable 
    from double buffering and performance is on par with multi-buffering 
    extensions (it is slower than drawing into a window directly on most 
    workstations however). Bear in mind that a pixmap can be resource 
    hungry, so use this option with discretion.

\setlength{\parskip}{1ex}
      \textbf{Parameters}
      \vspace{-1ex}

      \begin{quote}
        \begin{Ventry}{xxxxxxx}

          \item[pFlForm]

          form to set

            {\it (type=pointer to xfdata.FL\_FORM)}

          \item[flag]

          flag to disable/enable doublebuffer. Values 0 (disabled) or 1 
          (enabled)

            {\it (type=int)}

        \end{Ventry}

      \end{quote}

\textbf{Example:} fl\_set\_form\_dblbuffer(1)



\textbf{Status:} Tested + Doc + Demo = OK



    \end{boxedminipage}

    \label{xformslib:flbasic:fl_prepare_form_window}
    \index{xformslib \textit{(package)}!xformslib.flbasic \textit{(module)}!xformslib.flbasic.fl\_prepare\_form\_window \textit{(function)}}

    \vspace{0.5ex}

\hspace{.8\funcindent}\begin{boxedminipage}{\funcwidth}

    \raggedright \textbf{fl\_prepare\_form\_window}(\textit{pFlForm}, \textit{place}, \textit{border}, \textit{title})

    \vspace{-1.5ex}

    \rule{\textwidth}{0.5\fboxrule}
\setlength{\parskip}{2ex}
    Creates a window that obeys any and all constraints just as 
    fl\_show\_form() does but remains unmapped (not shown), returning its 
    window handle. You need fl\_show\_form\_window() after to show it.

\setlength{\parskip}{1ex}
      \textbf{Parameters}
      \vspace{-1ex}

      \begin{quote}
        \begin{Ventry}{xxxxxxx}

          \item[pFlForm]

          form to display

            {\it (type=pointer to xfdata.FL\_FORM)}

          \item[place]

          where has to be placed. Values (from xfdata module) 
          FL\_PLACE\_FREE, FL\_PLACE\_MOUSE, FL\_PLACE\_CENTER, 
          FL\_PLACE\_POSITION, FL\_PLACE\_SIZE, FL\_PLACE\_GEOMETRY, 
          FL\_PLACE\_ASPECT, FL\_PLACE\_FULLSCREEN, FL\_PLACE\_HOTSPOT, 
          FL\_PLACE\_ICONIC, FL\_FREE\_SIZE, FL\_PLACE\_FREE\_CENTER, 
          FL\_PLACE\_CENTERFREE

            {\it (type=int)}

          \item[border]

          window manager decoration. Values (from xfdata module) 
          FL\_FULLBORDER, FL\_TRANSIENT, FL\_NOBORDER

            {\it (type=int)}

          \item[title]

          text title of form

            {\it (type=str)}

        \end{Ventry}

      \end{quote}

      \textbf{Return Value}
    \vspace{-1ex}

      \begin{quote}
      window id (win)

      {\it (type=long\_pos)}

      \end{quote}

\textbf{Example:} wind = fl\_prepare\_form\_window(pform2, FL\_PLACE\_FREE, FL\_FULLBORDER, 
"MyForm")



\textbf{Status:} Tested + Doc + NoDemo = OK



    \end{boxedminipage}

    \label{xformslib:flbasic:fl_show_form_window}
    \index{xformslib \textit{(package)}!xformslib.flbasic \textit{(module)}!xformslib.flbasic.fl\_show\_form\_window \textit{(function)}}

    \vspace{0.5ex}

\hspace{.8\funcindent}\begin{boxedminipage}{\funcwidth}

    \raggedright \textbf{fl\_show\_form\_window}(\textit{pFlForm})

    \vspace{-1.5ex}

    \rule{\textwidth}{0.5\fboxrule}
\setlength{\parskip}{2ex}
    Maps (shows) a window of form that has been created before with 
    fl\_prepare\_form\_window().

\setlength{\parskip}{1ex}
      \textbf{Parameters}
      \vspace{-1ex}

      \begin{quote}
        \begin{Ventry}{xxxxxxx}

          \item[pFlForm]

          form whose window has to be shown

            {\it (type=pointer to xfdata.FL\_FORM)}

        \end{Ventry}

      \end{quote}

      \textbf{Return Value}
    \vspace{-1ex}

      \begin{quote}
      window id (win)

      {\it (type=long\_pos)}

      \end{quote}

\textbf{Example:} win1 = fl\_show\_form\_window(pform2)



\textbf{Status:} Tested + Doc + Demo = OK



    \end{boxedminipage}

    \label{xformslib:flbasic:fl_adjust_form_size}
    \index{xformslib \textit{(package)}!xformslib.flbasic \textit{(module)}!xformslib.flbasic.fl\_adjust\_form\_size \textit{(function)}}

    \vspace{0.5ex}

\hspace{.8\funcindent}\begin{boxedminipage}{\funcwidth}

    \raggedright \textbf{fl\_adjust\_form\_size}(\textit{pFlForm})

    \vspace{-1.5ex}

    \rule{\textwidth}{0.5\fboxrule}
\setlength{\parskip}{2ex}
    Similar to fl\_fit\_object\_label, but will do it for all objects and 
    has a smaller threshold. Mainly intended for compensation for font size
    variations.

\setlength{\parskip}{1ex}
      \textbf{Parameters}
      \vspace{-1ex}

      \begin{quote}
        \begin{Ventry}{xxxxxxx}

          \item[pFlForm]

          form whose size has to be adjusted

            {\it (type=pointer to xfdata.FL\_FORM)}

        \end{Ventry}

      \end{quote}

      \textbf{Return Value}
    \vspace{-1ex}

      \begin{quote}
      max factor id

      {\it (type=float)}

      \end{quote}

\textbf{Example:} mfactor = fl\_adjust\_form\_size(pform)



\textbf{Status:} Tested + Doc + Demo = OK



    \end{boxedminipage}

    \label{xformslib:flbasic:fl_form_is_visible}
    \index{xformslib \textit{(package)}!xformslib.flbasic \textit{(module)}!xformslib.flbasic.fl\_form\_is\_visible \textit{(function)}}

    \vspace{0.5ex}

\hspace{.8\funcindent}\begin{boxedminipage}{\funcwidth}

    \raggedright \textbf{fl\_form\_is\_visible}(\textit{pFlForm})

    \vspace{-1.5ex}

    \rule{\textwidth}{0.5\fboxrule}
\setlength{\parskip}{2ex}
    Returns if form is visible or not.

\setlength{\parskip}{1ex}
      \textbf{Parameters}
      \vspace{-1ex}

      \begin{quote}
        \begin{Ventry}{xxxxxxx}

          \item[pFlForm]

          form to evaluate

            {\it (type=pointer to xfdata.FL\_FORM)}

        \end{Ventry}

      \end{quote}

      \textbf{Return Value}
    \vspace{-1ex}

      \begin{quote}
      visibility state (0 invisible, non-zero visible)

      {\it (type=int)}

      \end{quote}

\textbf{Example:} visib = fl\_form\_is\_visible(pform)



\textbf{Status:} Tested + Doc + NoDemo = OK



    \end{boxedminipage}

    \label{xformslib:flbasic:fl_form_is_iconified}
    \index{xformslib \textit{(package)}!xformslib.flbasic \textit{(module)}!xformslib.flbasic.fl\_form\_is\_iconified \textit{(function)}}

    \vspace{0.5ex}

\hspace{.8\funcindent}\begin{boxedminipage}{\funcwidth}

    \raggedright \textbf{fl\_form\_is\_iconified}(\textit{pFlForm})

    \vspace{-1.5ex}

    \rule{\textwidth}{0.5\fboxrule}
\setlength{\parskip}{2ex}
    Returns if a form's window is in iconified state or not.

\setlength{\parskip}{1ex}
      \textbf{Parameters}
      \vspace{-1ex}

      \begin{quote}
        \begin{Ventry}{xxxxxxx}

          \item[pFlForm]

          form to evaluate

            {\it (type=pointer to xfdata.FL\_FORM)}

        \end{Ventry}

      \end{quote}

      \textbf{Return Value}
    \vspace{-1ex}

      \begin{quote}
      iconic state (0 not iconified, non-zero iconified)

      {\it (type=int)}

      \end{quote}

\textbf{Example:} iconif = fl\_form\_is\_iconified(pform)



\textbf{Status:} Tested + Doc + NoDemo = OK



    \end{boxedminipage}

    \label{xformslib:flbasic:fl_register_raw_callback}
    \index{xformslib \textit{(package)}!xformslib.flbasic \textit{(module)}!xformslib.flbasic.fl\_register\_raw\_callback \textit{(function)}}

    \vspace{0.5ex}

\hspace{.8\funcindent}\begin{boxedminipage}{\funcwidth}

    \raggedright \textbf{fl\_register\_raw\_callback}(\textit{pFlForm}, \textit{mask}, \textit{py\_RawCallback})

    \vspace{-1.5ex}

    \rule{\textwidth}{0.5\fboxrule}
\setlength{\parskip}{2ex}
    Registers preemptive event handlers. Only one handler is allowed for 
    each event pair.

\setlength{\parskip}{1ex}
      \textbf{Parameters}
      \vspace{-1ex}

      \begin{quote}
        \begin{Ventry}{xxxxxxxxxxxxxx}

          \item[pFlForm]

          form

            {\it (type=pointer to xfdata.FL\_FORM)}

          \item[mask]

          key/button/window event mask (press, release, motion, enter, 
          leave etc..). Values (from xfdata module) i.e. KeyPressMask and 
          KeyReleaseMask, ButtonPressMask and ButtonReleaseMask, 
          EnterWindowMask and LeaveWindowMask, ButtonMotionMask and 
          PointerMotionMask, FL\_ALL\_EVENT

            {\it (type=long\_pos)}

          \item[py\_RawCallback]

          python callback function, with return value

            {\it (type=\_\_ funcname (pFlForm, pXEvent) -{\textgreater} num \_\_)}

        \end{Ventry}

      \end{quote}

      \textbf{Return Value}
    \vspace{-1ex}

      \begin{quote}
      xfdata.FL\_RAW\_CALLBACK old function

      \end{quote}

\textbf{Example:}
\begin{quote}
  \begin{itemize}

  \item
    \setlength{\parskip}{0.6ex}
def rawcb(pform, xev):



  \item {\textbar}-{\textgreater}{\textbar} ...



  \item {\textbar}-{\textgreater}{\textbar} return 0



  \item oldrawcb = fl\_register\_callback(pform3, xfdata.KeyPressMask, rawcb)



\end{itemize}

\end{quote}

\textbf{Status:} Tested + Doc + Demo = OK



    \end{boxedminipage}

    \label{xformslib:flbasic:fl_register_raw_callback}
    \index{xformslib \textit{(package)}!xformslib.flbasic \textit{(module)}!xformslib.flbasic.fl\_register\_raw\_callback \textit{(function)}}

    \vspace{0.5ex}

\hspace{.8\funcindent}\begin{boxedminipage}{\funcwidth}

    \raggedright \textbf{fl\_register\_call\_back}(\textit{pFlForm}, \textit{mask}, \textit{py\_RawCallback})

    \vspace{-1.5ex}

    \rule{\textwidth}{0.5\fboxrule}
\setlength{\parskip}{2ex}
    Registers preemptive event handlers. Only one handler is allowed for 
    each event pair.

\setlength{\parskip}{1ex}
      \textbf{Parameters}
      \vspace{-1ex}

      \begin{quote}
        \begin{Ventry}{xxxxxxxxxxxxxx}

          \item[pFlForm]

          form

            {\it (type=pointer to xfdata.FL\_FORM)}

          \item[mask]

          key/button/window event mask (press, release, motion, enter, 
          leave etc..). Values (from xfdata module) i.e. KeyPressMask and 
          KeyReleaseMask, ButtonPressMask and ButtonReleaseMask, 
          EnterWindowMask and LeaveWindowMask, ButtonMotionMask and 
          PointerMotionMask, FL\_ALL\_EVENT

            {\it (type=long\_pos)}

          \item[py\_RawCallback]

          python callback function, with return value

            {\it (type=\_\_ funcname (pFlForm, pXEvent) -{\textgreater} num \_\_)}

        \end{Ventry}

      \end{quote}

      \textbf{Return Value}
    \vspace{-1ex}

      \begin{quote}
      xfdata.FL\_RAW\_CALLBACK old function

      \end{quote}

\textbf{Example:}
\begin{quote}
  \begin{itemize}

  \item
    \setlength{\parskip}{0.6ex}
def rawcb(pform, xev):



  \item {\textbar}-{\textgreater}{\textbar} ...



  \item {\textbar}-{\textgreater}{\textbar} return 0



  \item oldrawcb = fl\_register\_callback(pform3, xfdata.KeyPressMask, rawcb)



\end{itemize}

\end{quote}

\textbf{Status:} Tested + Doc + Demo = OK



    \end{boxedminipage}

    \label{xformslib:flbasic:fl_bgn_group}
    \index{xformslib \textit{(package)}!xformslib.flbasic \textit{(module)}!xformslib.flbasic.fl\_bgn\_group \textit{(function)}}

    \vspace{0.5ex}

\hspace{.8\funcindent}\begin{boxedminipage}{\funcwidth}

    \raggedright \textbf{fl\_bgn\_group}()

    \vspace{-1.5ex}

    \rule{\textwidth}{0.5\fboxrule}
\setlength{\parskip}{2ex}
    Starts a group of objects definition. It purpose can be e.g. to define 
    a series of objects to be hidden or deactivated or to define a series 
    of radio buttons.

\setlength{\parskip}{1ex}
      \textbf{Return Value}
    \vspace{-1ex}

      \begin{quote}
      group started (pFlObject)

      {\it (type=pointer to xfdata.FL\_OBJECT)}

      \end{quote}

\textbf{Example:} group0 = fl\_bgn\_group()



\textbf{Status:} Tested + Doc + Demo = OK



    \end{boxedminipage}

    \label{xformslib:flbasic:fl_end_group}
    \index{xformslib \textit{(package)}!xformslib.flbasic \textit{(module)}!xformslib.flbasic.fl\_end\_group \textit{(function)}}

    \vspace{0.5ex}

\hspace{.8\funcindent}\begin{boxedminipage}{\funcwidth}

    \raggedright \textbf{fl\_end\_group}()

    \vspace{-1.5ex}

    \rule{\textwidth}{0.5\fboxrule}
\setlength{\parskip}{2ex}
    Ends a group definition.

\setlength{\parskip}{1ex}
\textbf{Example:} fl\_end\_group()



\textbf{Status:} Tested + Doc + Demo = OK



    \end{boxedminipage}

    \label{xformslib:flbasic:fl_addto_group}
    \index{xformslib \textit{(package)}!xformslib.flbasic \textit{(module)}!xformslib.flbasic.fl\_addto\_group \textit{(function)}}

    \vspace{0.5ex}

\hspace{.8\funcindent}\begin{boxedminipage}{\funcwidth}

    \raggedright \textbf{fl\_addto\_group}(\textit{pFlObject})

    \vspace{-1.5ex}

    \rule{\textwidth}{0.5\fboxrule}
\setlength{\parskip}{2ex}
    Reopens a group (after fl\_end\_group) to allow addition of further 
    objects.

\setlength{\parskip}{1ex}
      \textbf{Parameters}
      \vspace{-1ex}

      \begin{quote}
        \begin{Ventry}{xxxxxxxxx}

          \item[pFlObject]

          group object to reopen

            {\it (type=pointer to xfdata.FL\_OBJECT)}

        \end{Ventry}

      \end{quote}

      \textbf{Return Value}
    \vspace{-1ex}

      \begin{quote}
      form (pFlForm) or None (on failure)

      {\it (type=pointer to xfdata.FL\_FORM)}

      \end{quote}

\textbf{Example:} group1 = fl\_addto\_group(closedgroup)



\textbf{Status:} Tested + Doc + NoDemo = OK



    \end{boxedminipage}

    \label{xformslib:flbasic:fl_get_object_objclass}
    \index{xformslib \textit{(package)}!xformslib.flbasic \textit{(module)}!xformslib.flbasic.fl\_get\_object\_objclass \textit{(function)}}

    \vspace{0.5ex}

\hspace{.8\funcindent}\begin{boxedminipage}{\funcwidth}

    \raggedright \textbf{fl\_get\_object\_objclass}(\textit{pFlObject})

    \vspace{-1.5ex}

    \rule{\textwidth}{0.5\fboxrule}
\setlength{\parskip}{2ex}
    Return the object class of an object. (e.g. button, lightbutton, box, 
    nmenu, counter, etc.)

\setlength{\parskip}{1ex}
      \textbf{Parameters}
      \vspace{-1ex}

      \begin{quote}
        \begin{Ventry}{xxxxxxxxx}

          \item[pFlObject]

          object to evaluate

            {\it (type=pointer to xfdata.FL\_OBJECT)}

        \end{Ventry}

      \end{quote}

      \textbf{Return Value}
    \vspace{-1ex}

      \begin{quote}
      objclass id, or -1 (on failure)

      {\it (type=int)}

      \end{quote}

\textbf{Example:} obcls = fl\_get\_object\_objclass(pobj)



\textbf{Status:} Tested + Doc + NoDemo = OK



    \end{boxedminipage}

    \label{xformslib:flbasic:fl_get_object_type}
    \index{xformslib \textit{(package)}!xformslib.flbasic \textit{(module)}!xformslib.flbasic.fl\_get\_object\_type \textit{(function)}}

    \vspace{0.5ex}

\hspace{.8\funcindent}\begin{boxedminipage}{\funcwidth}

    \raggedright \textbf{fl\_get\_object\_type}(\textit{pFlObject})

    \vspace{-1.5ex}

    \rule{\textwidth}{0.5\fboxrule}
\setlength{\parskip}{2ex}
    Return the type of an object (e.g. radio button, multiline input, 
    normal browser, etc..).

\setlength{\parskip}{1ex}
      \textbf{Parameters}
      \vspace{-1ex}

      \begin{quote}
        \begin{Ventry}{xxxxxxxxx}

          \item[pFlObject]

          object to evaluate

            {\it (type=pointer to xfdata.FL\_OBJECT)}

        \end{Ventry}

      \end{quote}

      \textbf{Return Value}
    \vspace{-1ex}

      \begin{quote}
      type id, or -1 (on failure)

      {\it (type=int)}

      \end{quote}

\textbf{Example:} obtype = fl\_get\_object\_type(pobj)



\textbf{Status:} Tested + Doc + NoDemo = OK



    \end{boxedminipage}

    \label{xformslib:flbasic:fl_set_object_boxtype}
    \index{xformslib \textit{(package)}!xformslib.flbasic \textit{(module)}!xformslib.flbasic.fl\_set\_object\_boxtype \textit{(function)}}

    \vspace{0.5ex}

\hspace{.8\funcindent}\begin{boxedminipage}{\funcwidth}

    \raggedright \textbf{fl\_set\_object\_boxtype}(\textit{pFlObject}, \textit{boxtype})

    \vspace{-1.5ex}

    \rule{\textwidth}{0.5\fboxrule}
\setlength{\parskip}{2ex}
    Sets the shape of box of an object. Not all possible boxtypes are 
    suitable for all types of objects.

\setlength{\parskip}{1ex}
      \textbf{Parameters}
      \vspace{-1ex}

      \begin{quote}
        \begin{Ventry}{xxxxxxxxx}

          \item[pFlObject]

          object whose boxtype has to be set

            {\it (type=pointer to xfdata.FL\_OBJECT)}

          \item[boxtype]

          type of the box to be set. Values (from xfdata module) 
          FL\_NO\_BOX, FL\_UP\_BOX, FL\_DOWN\_BOX, FL\_BORDER\_BOX, 
          FL\_SHADOW\_BOX, FL\_FRAME\_BOX, FL\_ROUNDED\_BOX, 
          FL\_EMBOSSED\_BOX, FL\_FLAT\_BOX, FL\_RFLAT\_BOX, 
          FL\_RSHADOW\_BOX, FL\_OVAL\_BOX, FL\_ROUNDED3D\_UPBOX, 
          FL\_ROUNDED3D\_DOWNBOX, FL\_OVAL3D\_UPBOX, FL\_OVAL3D\_DOWNBOX, 
          FL\_OVAL3D\_FRAMEBOX, FL\_OVAL3D\_EMBOSSEDBOX

            {\it (type=int)}

        \end{Ventry}

      \end{quote}

\textbf{Example:} fl\_set\_object\_boxtype(ptextobj, xfdata.FL\_BORDER\_BOX)



\textbf{Status:} Tested + Doc + Demo = OK



    \end{boxedminipage}

    \label{xformslib:flbasic:fl_get_object_boxtype}
    \index{xformslib \textit{(package)}!xformslib.flbasic \textit{(module)}!xformslib.flbasic.fl\_get\_object\_boxtype \textit{(function)}}

    \vspace{0.5ex}

\hspace{.8\funcindent}\begin{boxedminipage}{\funcwidth}

    \raggedright \textbf{fl\_get\_object\_boxtype}(\textit{pFlObject})

    \vspace{-1.5ex}

    \rule{\textwidth}{0.5\fboxrule}
\setlength{\parskip}{2ex}
    Returns the current boxtype of an object (e.g. no box, up box, shadow 
    box, etc..).

\setlength{\parskip}{1ex}
      \textbf{Parameters}
      \vspace{-1ex}

      \begin{quote}
        \begin{Ventry}{xxxxxxxxx}

          \item[pFlObject]

          object to evaluate

            {\it (type=pointer to xfdata.FL\_OBJECT)}

        \end{Ventry}

      \end{quote}

      \textbf{Return Value}
    \vspace{-1ex}

      \begin{quote}
      boxtype id or -1 (on failure)

      {\it (type=int)}

      \end{quote}

\textbf{Example:} boxtp = fl\_get\_object\_boxtype(ptextobj)



\textbf{Status:} Tested + Doc + NoDemo = OK



    \end{boxedminipage}

    \label{xformslib:flbasic:fl_set_object_bw}
    \index{xformslib \textit{(package)}!xformslib.flbasic \textit{(module)}!xformslib.flbasic.fl\_set\_object\_bw \textit{(function)}}

    \vspace{0.5ex}

\hspace{.8\funcindent}\begin{boxedminipage}{\funcwidth}

    \raggedright \textbf{fl\_set\_object\_bw}(\textit{pFlObject}, \textit{bw})

    \vspace{-1.5ex}

    \rule{\textwidth}{0.5\fboxrule}
\setlength{\parskip}{2ex}
    Sets the borderwidth of an object. If requested borderwidth is 0, -1 is
    used. If set to a negative number, all objects appear to have a softer 
    appearance.

\setlength{\parskip}{1ex}
      \textbf{Parameters}
      \vspace{-1ex}

      \begin{quote}
        \begin{Ventry}{xxxxxxxxx}

          \item[pFlObject]

          object

            {\it (type=pointer to xfdata.FL\_OBJECT)}

          \item[bw]

          borderwidth of object to be set

            {\it (type=int)}

        \end{Ventry}

      \end{quote}

\textbf{Example:} fl\_set\_object\_bw(pobj, 2)



\textbf{Status:} Tested + Doc + Demo = OK



    \end{boxedminipage}

    \label{xformslib:flbasic:fl_get_object_bw}
    \index{xformslib \textit{(package)}!xformslib.flbasic \textit{(module)}!xformslib.flbasic.fl\_get\_object\_bw \textit{(function)}}

    \vspace{0.5ex}

\hspace{.8\funcindent}\begin{boxedminipage}{\funcwidth}

    \raggedright \textbf{fl\_get\_object\_bw}(\textit{pFlObject})

    \vspace{-1.5ex}

    \rule{\textwidth}{0.5\fboxrule}
\setlength{\parskip}{2ex}
    Returns the borderwidth of an object.

\setlength{\parskip}{1ex}
      \textbf{Parameters}
      \vspace{-1ex}

      \begin{quote}
        \begin{Ventry}{xxxxxxxxx}

          \item[pFlObject]

          object to evaluate

            {\it (type=pointer to xfdata.FL\_OBJECT)}

        \end{Ventry}

      \end{quote}

      \textbf{Return Value}
    \vspace{-1ex}

      \begin{quote}
      borderwidth (bw)

      {\it (type=int)}

      \end{quote}

\textbf{Example:} currbw = fl\_get\_object\_bw(pobj)



\textbf{Attention:} API change from XForms - upstream was fl\_get\_object\_bw(pFlObject, bw)



\textbf{Status:} Tested + Doc + NoDemo = OK



    \end{boxedminipage}

    \label{xformslib:flbasic:fl_set_object_resize}
    \index{xformslib \textit{(package)}!xformslib.flbasic \textit{(module)}!xformslib.flbasic.fl\_set\_object\_resize \textit{(function)}}

    \vspace{0.5ex}

\hspace{.8\funcindent}\begin{boxedminipage}{\funcwidth}

    \raggedright \textbf{fl\_set\_object\_resize}(\textit{pFlObject}, \textit{what})

    \vspace{-1.5ex}

    \rule{\textwidth}{0.5\fboxrule}
\setlength{\parskip}{2ex}
    Sets the resize property of an object.

\setlength{\parskip}{1ex}
      \textbf{Parameters}
      \vspace{-1ex}

      \begin{quote}
        \begin{Ventry}{xxxxxxxxx}

          \item[pFlObject]

          object to set

            {\it (type=pointer to xfdata.FL\_OBJECT)}

          \item[what]

          resize property. Values (from xfdata module) i.e. 
          FL\_RESIZE\_NONE, FL\_RESIZE\_X, FL\_RESIZE\_Y, FL\_RESIZE\_ALL

            {\it (type=int\_pos)}

        \end{Ventry}

      \end{quote}

\textbf{Example:} fl\_set\_object\_resize(pobj, xfdata.FL\_RESIZE\_ALL)



\textbf{Status:} Tested + Doc + Demo = OK



    \end{boxedminipage}

    \label{xformslib:flbasic:fl_get_object_resize}
    \index{xformslib \textit{(package)}!xformslib.flbasic \textit{(module)}!xformslib.flbasic.fl\_get\_object\_resize \textit{(function)}}

    \vspace{0.5ex}

\hspace{.8\funcindent}\begin{boxedminipage}{\funcwidth}

    \raggedright \textbf{fl\_get\_object\_resize}(\textit{pFlObject})

    \vspace{-1.5ex}

    \rule{\textwidth}{0.5\fboxrule}
\setlength{\parskip}{2ex}
    Returns the resize property of an object (e.g. resize all, resize none,
    etc..).

\setlength{\parskip}{1ex}
      \textbf{Parameters}
      \vspace{-1ex}

      \begin{quote}
        \begin{Ventry}{xxxxxxxxx}

          \item[pFlObject]

          object to evaluate

            {\it (type=pointer to xfdata.FL\_OBJECT)}

        \end{Ventry}

      \end{quote}

      \textbf{Return Value}
    \vspace{-1ex}

      \begin{quote}
      resize property

      {\it (type=int\_pos)}

      \end{quote}

\textbf{Attention:} API change from XForms - upstream was fl\_get\_object\_resize(pFlObject, 
what)



\textbf{Example:} reszprop = fl\_get\_object\_resize(pobj)



\textbf{Status:} Tested + Doc + NoDemo = OK



    \end{boxedminipage}

    \label{xformslib:flbasic:fl_set_object_gravity}
    \index{xformslib \textit{(package)}!xformslib.flbasic \textit{(module)}!xformslib.flbasic.fl\_set\_object\_gravity \textit{(function)}}

    \vspace{0.5ex}

\hspace{.8\funcindent}\begin{boxedminipage}{\funcwidth}

    \raggedright \textbf{fl\_set\_object\_gravity}(\textit{pFlObject}, \textit{nw}, \textit{se})

    \vspace{-1.5ex}

    \rule{\textwidth}{0.5\fboxrule}
\setlength{\parskip}{2ex}
    Sets the gravity properties of an object.

\setlength{\parskip}{1ex}
      \textbf{Parameters}
      \vspace{-1ex}

      \begin{quote}
        \begin{Ventry}{xxxxxxxxx}

          \item[pFlObject]

          object to be set

            {\it (type=pointer to xfdata.FL\_OBJECT)}

          \item[nw]

          gravity property for NorthWest. Values (from xfdata module) 
          FL\_North, FL\_NorthEast, FL\_NorthWest, FL\_South, 
          FL\_SouthEast, FL\_SouthWest, FL\_East, FL\_West, FL\_NoGravity, 
          FL\_ForgetGravity

            {\it (type=int\_pos)}

          \item[se]

          gravity property for SouthEast. Values (from xfdata module) 
          FL\_North, FL\_NorthEast, FL\_NorthWest, FL\_South, 
          FL\_SouthEast, FL\_SouthWest, FL\_East, FL\_West, FL\_NoGravity, 
          FL\_ForgetGravity

            {\it (type=int\_pos)}

        \end{Ventry}

      \end{quote}

\textbf{Example:} fl\_set\_object\_gravity(pobj, xfdata.FL\_North, xfdata.FL\_East)



\textbf{Status:} Tested + Doc + Demo = OK



    \end{boxedminipage}

    \label{xformslib:flbasic:fl_get_object_gravity}
    \index{xformslib \textit{(package)}!xformslib.flbasic \textit{(module)}!xformslib.flbasic.fl\_get\_object\_gravity \textit{(function)}}

    \vspace{0.5ex}

\hspace{.8\funcindent}\begin{boxedminipage}{\funcwidth}

    \raggedright \textbf{fl\_get\_object\_gravity}(\textit{pFlObject})

    \vspace{-1.5ex}

    \rule{\textwidth}{0.5\fboxrule}
\setlength{\parskip}{2ex}
    Returns the gravity properties of an object (e.g. North, SouthWest, 
    etc..).

\setlength{\parskip}{1ex}
      \textbf{Parameters}
      \vspace{-1ex}

      \begin{quote}
        \begin{Ventry}{xxxxxxxxx}

          \item[pFlObject]

          object to set

            {\it (type=pointer to xfdata.FL\_OBJECT)}

        \end{Ventry}

      \end{quote}

      \textbf{Return Value}
    \vspace{-1ex}

      \begin{quote}
      NorthWest and SouthEast gravity

      {\it (type=int\_pos, int\_pos)}

      \end{quote}

\textbf{Attention:} API change from XForms - upstream was fl\_get\_object\_gravity(pFlObject, 
nw, se)



\textbf{Example:} nowe, soea = fl\_get\_object\_gravity(pobj)



\textbf{Status:} Tested + Doc + NoDemo = OK



    \end{boxedminipage}

    \label{xformslib:flbasic:fl_set_object_lsize}
    \index{xformslib \textit{(package)}!xformslib.flbasic \textit{(module)}!xformslib.flbasic.fl\_set\_object\_lsize \textit{(function)}}

    \vspace{0.5ex}

\hspace{.8\funcindent}\begin{boxedminipage}{\funcwidth}

    \raggedright \textbf{fl\_set\_object\_lsize}(\textit{pFlObject}, \textit{size})

    \vspace{-1.5ex}

    \rule{\textwidth}{0.5\fboxrule}
\setlength{\parskip}{2ex}
    Sets the label size of an object.

\setlength{\parskip}{1ex}
      \textbf{Parameters}
      \vspace{-1ex}

      \begin{quote}
        \begin{Ventry}{xxxxxxxxx}

          \item[pFlObject]

          object to be set

            {\it (type=pointer to xfdata.FL\_OBJECT)}

          \item[size]

          label size. Values (from xfdata module) FL\_TINY\_SIZE, 
          FL\_SMALL\_SIZE, FL\_NORMAL\_SIZE, FL\_MEDIUM\_SIZE, 
          FL\_LARGE\_SIZE, FL\_HUGE\_SIZE, FL\_DEFAULT\_SIZE

            {\it (type=int)}

        \end{Ventry}

      \end{quote}

\textbf{Example:} fl\_set\_object\_lsize(pobj, xfdata.FL\_MEDIUM\_SIZE)



\textbf{Status:} Tested + Doc + Demo = OK



    \end{boxedminipage}

    \label{xformslib:flbasic:fl_get_object_lsize}
    \index{xformslib \textit{(package)}!xformslib.flbasic \textit{(module)}!xformslib.flbasic.fl\_get\_object\_lsize \textit{(function)}}

    \vspace{0.5ex}

\hspace{.8\funcindent}\begin{boxedminipage}{\funcwidth}

    \raggedright \textbf{fl\_get\_object\_lsize}(\textit{pFlObject})

    \vspace{-1.5ex}

    \rule{\textwidth}{0.5\fboxrule}
\setlength{\parskip}{2ex}
    Returns the label size of an object.

\setlength{\parskip}{1ex}
      \textbf{Parameters}
      \vspace{-1ex}

      \begin{quote}
        \begin{Ventry}{xxxxxxxxx}

          \item[pFlObject]

          object to evaluate

            {\it (type=pointer to xfdata.FL\_OBJECT)}

        \end{Ventry}

      \end{quote}

      \textbf{Return Value}
    \vspace{-1ex}

      \begin{quote}
      label size

      {\it (type=int)}

      \end{quote}

\textbf{Example:} lsize = fl\_get\_object\_lsize(pobj)



\textbf{Status:} Tested + Doc + Demo = OK



    \end{boxedminipage}

    \label{xformslib:flbasic:fl_set_object_lstyle}
    \index{xformslib \textit{(package)}!xformslib.flbasic \textit{(module)}!xformslib.flbasic.fl\_set\_object\_lstyle \textit{(function)}}

    \vspace{0.5ex}

\hspace{.8\funcindent}\begin{boxedminipage}{\funcwidth}

    \raggedright \textbf{fl\_set\_object\_lstyle}(\textit{pFlObject}, \textit{style})

    \vspace{-1.5ex}

    \rule{\textwidth}{0.5\fboxrule}
\setlength{\parskip}{2ex}
    Sets the label style of an object.

\setlength{\parskip}{1ex}
      \textbf{Parameters}
      \vspace{-1ex}

      \begin{quote}
        \begin{Ventry}{xxxxxxxxx}

          \item[pFlObject]

          object to be set

            {\it (type=pointer to xfdata.FL\_OBJECT)}

          \item[style]

          label style. Values (from xfdata module) FL\_NORMAL\_STYLE, 
          FL\_BOLD\_STYLE, FL\_ITALIC\_STYLE, FL\_BOLDITALIC\_STYLE, 
          FL\_FIXED\_STYLE, FL\_FIXEDBOLD\_STYLE, FL\_FIXEDITALIC\_STYLE, 
          FL\_FIXEDBOLDITALIC\_STYLE, FL\_TIMES\_STYLE, 
          FL\_TIMESBOLD\_STYLE, FL\_TIMESITALIC\_STYLE, 
          FL\_TIMESBOLDITALIC\_STYLE, FL\_MISC\_STYLE, FL\_MISCBOLD\_STYLE,
          FL\_MISCITALIC\_STYLE, FL\_SYMBOL\_STYLE, FL\_SHADOW\_STYLE, 
          FL\_ENGRAVED\_STYLE, FL\_EMBOSSED\_STYLE

            {\it (type=int)}

        \end{Ventry}

      \end{quote}

\textbf{Example:} fl\_set\_object\_lstyle(pobj, xfdata.FL\_TIMESITALIC\_STYLE)



\textbf{Status:} Tested + Doc + Demo = OK



    \end{boxedminipage}

    \label{xformslib:flbasic:fl_get_object_lstyle}
    \index{xformslib \textit{(package)}!xformslib.flbasic \textit{(module)}!xformslib.flbasic.fl\_get\_object\_lstyle \textit{(function)}}

    \vspace{0.5ex}

\hspace{.8\funcindent}\begin{boxedminipage}{\funcwidth}

    \raggedright \textbf{fl\_get\_object\_lstyle}(\textit{pFlObject})

    \vspace{-1.5ex}

    \rule{\textwidth}{0.5\fboxrule}
\setlength{\parskip}{2ex}
    Returns the label style of an object (e.g. xfdata.FL\_BOLD\_STYLE, 
    xfdata.FL\_NORMAL\_STYLE, etc..).

\setlength{\parskip}{1ex}
      \textbf{Parameters}
      \vspace{-1ex}

      \begin{quote}
        \begin{Ventry}{xxxxxxxxx}

          \item[pFlObject]

          object to evaluate

            {\it (type=pointer to xfdata.FL\_OBJECT)}

        \end{Ventry}

      \end{quote}

      \textbf{Return Value}
    \vspace{-1ex}

      \begin{quote}
      label style

      {\it (type=int)}

      \end{quote}

\textbf{Example:} lstyle = fl\_get\_object\_lstyle(pobj)



\textbf{Status:} Tested + NoDoc + Demo = OK



    \end{boxedminipage}

    \label{xformslib:flbasic:fl_set_object_lcol}
    \index{xformslib \textit{(package)}!xformslib.flbasic \textit{(module)}!xformslib.flbasic.fl\_set\_object\_lcol \textit{(function)}}

    \vspace{0.5ex}

\hspace{.8\funcindent}\begin{boxedminipage}{\funcwidth}

    \raggedright \textbf{fl\_set\_object\_lcol}(\textit{pFlObject}, \textit{colr})

    \vspace{-1.5ex}

    \rule{\textwidth}{0.5\fboxrule}
\setlength{\parskip}{2ex}
    Sets the label color of an object.

\setlength{\parskip}{1ex}
      \textbf{Parameters}
      \vspace{-1ex}

      \begin{quote}
        \begin{Ventry}{xxxxxxxxx}

          \item[pFlObject]

          object to be set

            {\it (type=pointer to xfdata.FL\_OBJECT)}

          \item[colr]

          label color. Values (from xfdata module) one of defined colors 
          FL\_BLACK, ... FL\_BLUE, ... FL\_GREEN, ... FL\_RED, ... etc..

            {\it (type=long\_pos)}

        \end{Ventry}

      \end{quote}

\textbf{Example:} fl\_set\_object\_lcol(pobj, xfdata.FL\_BLUE)



\textbf{Status:} Tested + Doc + Demo = OK



    \end{boxedminipage}

    \label{xformslib:flbasic:fl_set_object_lcol}
    \index{xformslib \textit{(package)}!xformslib.flbasic \textit{(module)}!xformslib.flbasic.fl\_set\_object\_lcol \textit{(function)}}

    \vspace{0.5ex}

\hspace{.8\funcindent}\begin{boxedminipage}{\funcwidth}

    \raggedright \textbf{fl\_set\_object\_lcolor}(\textit{pFlObject}, \textit{colr})

    \vspace{-1.5ex}

    \rule{\textwidth}{0.5\fboxrule}
\setlength{\parskip}{2ex}
    Sets the label color of an object.

\setlength{\parskip}{1ex}
      \textbf{Parameters}
      \vspace{-1ex}

      \begin{quote}
        \begin{Ventry}{xxxxxxxxx}

          \item[pFlObject]

          object to be set

            {\it (type=pointer to xfdata.FL\_OBJECT)}

          \item[colr]

          label color. Values (from xfdata module) one of defined colors 
          FL\_BLACK, ... FL\_BLUE, ... FL\_GREEN, ... FL\_RED, ... etc..

            {\it (type=long\_pos)}

        \end{Ventry}

      \end{quote}

\textbf{Example:} fl\_set\_object\_lcol(pobj, xfdata.FL\_BLUE)



\textbf{Status:} Tested + Doc + Demo = OK



    \end{boxedminipage}

    \label{xformslib:flbasic:fl_get_object_lcol}
    \index{xformslib \textit{(package)}!xformslib.flbasic \textit{(module)}!xformslib.flbasic.fl\_get\_object\_lcol \textit{(function)}}

    \vspace{0.5ex}

\hspace{.8\funcindent}\begin{boxedminipage}{\funcwidth}

    \raggedright \textbf{fl\_get\_object\_lcol}(\textit{pFlObject})

    \vspace{-1.5ex}

    \rule{\textwidth}{0.5\fboxrule}
\setlength{\parskip}{2ex}
    Returns the label color of an object (from xfdata, e.g. FL\_WHITE, 
    FL\_LIME, etc..).

\setlength{\parskip}{1ex}
      \textbf{Parameters}
      \vspace{-1ex}

      \begin{quote}
        \begin{Ventry}{xxxxxxxxx}

          \item[pFlObject]

          object to evaluate

            {\it (type=pointer to xfdata.FL\_OBJECT)}

        \end{Ventry}

      \end{quote}

      \textbf{Return Value}
    \vspace{-1ex}

      \begin{quote}
      color value

      {\it (type=long\_pos)}

      \end{quote}

\textbf{Example:} obcolor = fl\_get\_object\_lcol(pobj)



\textbf{Status:} Tested + Doc + NoDemo = OK



    \end{boxedminipage}

    \label{xformslib:flbasic:fl_set_object_return}
    \index{xformslib \textit{(package)}!xformslib.flbasic \textit{(module)}!xformslib.flbasic.fl\_set\_object\_return \textit{(function)}}

    \vspace{0.5ex}

\hspace{.8\funcindent}\begin{boxedminipage}{\funcwidth}

    \raggedright \textbf{fl\_set\_object\_return}(\textit{pFlObject}, \textit{when})

    \vspace{-1.5ex}

    \rule{\textwidth}{0.5\fboxrule}
\setlength{\parskip}{2ex}
    Sets the conditions under which an object gets returned (or its 
    callback invoked). If the object has to do additional work on setting 
    te condition (e.g. it has child objects that also need to be set) it 
    has to set up it's own function that then will called in the end. This 
    should only be called once an object has been created completely! Not 
    all return types make sense for all objects.

\setlength{\parskip}{1ex}
      \textbf{Parameters}
      \vspace{-1ex}

      \begin{quote}
        \begin{Ventry}{xxxxxxxxx}

          \item[pFlObject]

          object

            {\it (type=pointer to xfdata.FL\_OBJECT)}

          \item[when]

          return type (when it returns). Values (from xfdata module) 
          FL\_RETURN\_NONE, FL\_RETURN\_CHANGED, FL\_RETURN\_END, 
          FL\_RETURN\_END\_CHANGED, FL\_RETURN\_SELECTION, 
          FL\_RETURN\_DESELECTION, FL\_RETURN\_TRIGGERED, 
          FL\_RETURN\_ALWAYS

            {\it (type=int\_pos)}

        \end{Ventry}

      \end{quote}

      \textbf{Return Value}
    \vspace{-1ex}

      \begin{quote}
      old return type id

      {\it (type=int)}

      \end{quote}

\textbf{Example:} fl\_set\_object\_return(pobj, xfdata.FL\_RETURN\_CHANGED)



\textbf{Status:} Tested + Doc + Demo = OK



    \end{boxedminipage}

    \label{xformslib:flbasic:fl_notify_object}
    \index{xformslib \textit{(package)}!xformslib.flbasic \textit{(module)}!xformslib.flbasic.fl\_notify\_object \textit{(function)}}

    \vspace{0.5ex}

\hspace{.8\funcindent}\begin{boxedminipage}{\funcwidth}

    \raggedright \textbf{fl\_notify\_object}(\textit{pFlObject}, \textit{cause})

    \vspace{-1.5ex}

    \rule{\textwidth}{0.5\fboxrule}
\setlength{\parskip}{2ex}
    ???

\setlength{\parskip}{1ex}
      \textbf{Parameters}
      \vspace{-1ex}

      \begin{quote}
        \begin{Ventry}{xxxxxxxxx}

          \item[pFlObject]

          pointer to object

            {\it (type=pointer to xfdata.FL\_OBJECT)}

          \item[cause]

          cause for notification. Values (from xfdata module) FL\_ATTRIB, 
          FL\_RESIZED, FL\_MOVEORIGIN

            {\it (type=int)}

        \end{Ventry}

      \end{quote}

\textbf{Example:} fl\_notify\_object(pobj5, xfdata.FL\_RESIZED)



\textbf{Status:} Tested + NoDoc + NoDemo = NOT OK (not clear purpose)



    \end{boxedminipage}

    \label{xformslib:flbasic:fl_set_object_lalign}
    \index{xformslib \textit{(package)}!xformslib.flbasic \textit{(module)}!xformslib.flbasic.fl\_set\_object\_lalign \textit{(function)}}

    \vspace{0.5ex}

\hspace{.8\funcindent}\begin{boxedminipage}{\funcwidth}

    \raggedright \textbf{fl\_set\_object\_lalign}(\textit{pFlObject}, \textit{align})

    \vspace{-1.5ex}

    \rule{\textwidth}{0.5\fboxrule}
\setlength{\parskip}{2ex}
    Sets alignment of an object's label.

\setlength{\parskip}{1ex}
      \textbf{Parameters}
      \vspace{-1ex}

      \begin{quote}
        \begin{Ventry}{xxxxxxxxx}

          \item[pFlObject]

          object to be set

            {\it (type=pointer to xfdata.FL\_OBJECT)}

          \item[align]

          alignment of label. Values (from xfdata module) 
          FL\_ALIGN\_CENTER, FL\_ALIGN\_TOP, FL\_ALIGN\_BOTTOM, 
          FL\_ALIGN\_LEFT, FL\_ALIGN\_RIGHT, FL\_ALIGN\_LEFT\_TOP, 
          FL\_ALIGN\_RIGHT\_TOP, FL\_ALIGN\_LEFT\_BOTTOM, 
          FL\_ALIGN\_RIGHT\_BOTTOM, FL\_ALIGN\_INSIDE, FL\_ALIGN\_VERT

            {\it (type=int)}

        \end{Ventry}

      \end{quote}

\textbf{Example:} fl\_set\_object\_lalign(pobj8, xfdata.FL\_ALIGN\_RIGHT)



\textbf{Status:} Tested + Doc + Demo = OK



    \end{boxedminipage}

    \label{xformslib:flbasic:fl_get_object_lalign}
    \index{xformslib \textit{(package)}!xformslib.flbasic \textit{(module)}!xformslib.flbasic.fl\_get\_object\_lalign \textit{(function)}}

    \vspace{0.5ex}

\hspace{.8\funcindent}\begin{boxedminipage}{\funcwidth}

    \raggedright \textbf{fl\_get\_object\_lalign}(\textit{pFlObject})

    \vspace{-1.5ex}

    \rule{\textwidth}{0.5\fboxrule}
\setlength{\parskip}{2ex}
    Returns alignment of an object's label (from xfdata, e.g. 
    FL\_ALIGN\_LEFT, FL\_ALIGN\_RIGHT\_TOP, etc..).

\setlength{\parskip}{1ex}
      \textbf{Parameters}
      \vspace{-1ex}

      \begin{quote}
        \begin{Ventry}{xxxxxxxxx}

          \item[pFlObject]

          object to be set

            {\it (type=pointer to xfdata.FL\_OBJECT)}

        \end{Ventry}

      \end{quote}

      \textbf{Return Value}
    \vspace{-1ex}

      \begin{quote}
      align num.

      {\it (type=int)}

      \end{quote}

\textbf{Example:} obalign = fl\_get\_object\_lalign(pobj8)



\textbf{Status:} Tested + Doc + Demo = OK



    \end{boxedminipage}

    \label{xformslib:flbasic:fl_set_object_lalign}
    \index{xformslib \textit{(package)}!xformslib.flbasic \textit{(module)}!xformslib.flbasic.fl\_set\_object\_lalign \textit{(function)}}

    \vspace{0.5ex}

\hspace{.8\funcindent}\begin{boxedminipage}{\funcwidth}

    \raggedright \textbf{fl\_set\_object\_align}(\textit{pFlObject}, \textit{align})

    \vspace{-1.5ex}

    \rule{\textwidth}{0.5\fboxrule}
\setlength{\parskip}{2ex}
    Sets alignment of an object's label.

\setlength{\parskip}{1ex}
      \textbf{Parameters}
      \vspace{-1ex}

      \begin{quote}
        \begin{Ventry}{xxxxxxxxx}

          \item[pFlObject]

          object to be set

            {\it (type=pointer to xfdata.FL\_OBJECT)}

          \item[align]

          alignment of label. Values (from xfdata module) 
          FL\_ALIGN\_CENTER, FL\_ALIGN\_TOP, FL\_ALIGN\_BOTTOM, 
          FL\_ALIGN\_LEFT, FL\_ALIGN\_RIGHT, FL\_ALIGN\_LEFT\_TOP, 
          FL\_ALIGN\_RIGHT\_TOP, FL\_ALIGN\_LEFT\_BOTTOM, 
          FL\_ALIGN\_RIGHT\_BOTTOM, FL\_ALIGN\_INSIDE, FL\_ALIGN\_VERT

            {\it (type=int)}

        \end{Ventry}

      \end{quote}

\textbf{Example:} fl\_set\_object\_lalign(pobj8, xfdata.FL\_ALIGN\_RIGHT)



\textbf{Status:} Tested + Doc + Demo = OK



    \end{boxedminipage}

    \label{xformslib:flbasic:fl_set_object_shortcut}
    \index{xformslib \textit{(package)}!xformslib.flbasic \textit{(module)}!xformslib.flbasic.fl\_set\_object\_shortcut \textit{(function)}}

    \vspace{0.5ex}

\hspace{.8\funcindent}\begin{boxedminipage}{\funcwidth}

    \raggedright \textbf{fl\_set\_object\_shortcut}(\textit{pFlObject}, \textit{shctxt}, \textit{showit})

    \vspace{-1.5ex}

    \rule{\textwidth}{0.5\fboxrule}
\setlength{\parskip}{2ex}
    Sets a shortcut, binding a key or a series of keys to an object. It 
    resets any previous defined shortcuts for the object. Using e.g. 
    "acE\#d{\textasciicircum}h" the keys 'a', 'c', 'E', 
    {\textless}Alt{\textgreater}d and {\textless}Ctrl{\textgreater}h are 
    associated with the object. The precise format is as follows: any 
    character in the string is considered as a shortcut, except 
    '{\textasciicircum}' and '\#', which stand for combinations with the 
    {\textless}Ctrl{\textgreater} and {\textless}Alt{\textgreater} keys; 
    the case of the key following '\#' or '{\textasciicircum}' is not 
    important, i.e. no distinction is made between e.g. 
    "{\textasciicircum}C" and "{\textasciicircum}c", both encode the key 
    combination {\textless}Ctrl{\textgreater}C as well as 
    {\textless}Ctrl{\textgreater}C.) The key '{\textasciicircum}' itself 
    can be set as a shortcut key by using 
    "{\textasciicircum}{\textasciicircum}" in the string defining the 
    shortcut. The key '\#' can be obtained as a shortcut by using the 
    string "{\textasciicircum}\#". So, e.g. "\#{\textasciicircum}\#" 
    encodes {\textless}ALT{\textgreater}\#. The 
    {\textless}Esc{\textgreater} key can be given as "{\textasciicircum}[".
    Another special character not mentioned yet is '\&', which indicates 
    function and arrow keys. Use a sequence starting with '\&' and directly
    followed by a number between 1 and 35 to represent one of the function 
    keys. For example, "\&2" stands for the {\textless}F2{\textgreater} 
    function key. The four cursors keys (up, down, right, and left) can be 
    given as "\&A", "\&B", "\&C" and "\&D", respectively. The key '\&' 
    itself can be obtained as a shortcut by prefixing it with 
    '{\textasciicircum}'.

\setlength{\parskip}{1ex}
      \textbf{Parameters}
      \vspace{-1ex}

      \begin{quote}
        \begin{Ventry}{xxxxxxxxx}

          \item[pFlObject]

          object

            {\it (type=pointer to xfdata.FL\_OBJECT)}

          \item[shctxt]

          shortcut text to be set

            {\it (type=str)}

          \item[showit]

          flag if shortcut letter has to be underlined or not if a match 
          exists (only the 1st alphanumeric character is used). Values 0 
          (underline not shown) or 1 (shown)

            {\it (type=int)}

        \end{Ventry}

      \end{quote}

\textbf{Example:} fl\_set\_object\_shortcut(pobj6, "aA\#A{\textasciicircum}A", 1)



\textbf{Status:} Tested + Doc + NoDemo = OK



    \end{boxedminipage}

    \label{xformslib:flbasic:fl_set_object_shortcutkey}
    \index{xformslib \textit{(package)}!xformslib.flbasic \textit{(module)}!xformslib.flbasic.fl\_set\_object\_shortcutkey \textit{(function)}}

    \vspace{0.5ex}

\hspace{.8\funcindent}\begin{boxedminipage}{\funcwidth}

    \raggedright \textbf{fl\_set\_object\_shortcutkey}(\textit{pFlObject}, \textit{keysym})

    \vspace{-1.5ex}

    \rule{\textwidth}{0.5\fboxrule}
\setlength{\parskip}{2ex}
    Uses a special key as a shortcut. It always appends the specified key 
    to the current shortcuts. Special keys can't be underlined.

\setlength{\parskip}{1ex}
      \textbf{Parameters}
      \vspace{-1ex}

      \begin{quote}
        \begin{Ventry}{xxxxxxxxx}

          \item[pFlObject]

          object

            {\it (type=pointer to xfdata.FL\_OBJECT)}

          \item[keysym]

          X key symbolic num. See xfdata module for a (maybe) incomplete 
          list

            {\it (type=int\_pos)}

        \end{Ventry}

      \end{quote}

\textbf{Example:} fl\_set\_object\_shortcutkey(pobj, xfdata.XK\_Home)



\textbf{Status:} Tested + Doc + NoDemo = OK



    \end{boxedminipage}

    \label{xformslib:flbasic:fl_set_object_dblbuffer}
    \index{xformslib \textit{(package)}!xformslib.flbasic \textit{(module)}!xformslib.flbasic.fl\_set\_object\_dblbuffer \textit{(function)}}

    \vspace{0.5ex}

\hspace{.8\funcindent}\begin{boxedminipage}{\funcwidth}

    \raggedright \textbf{fl\_set\_object\_dblbuffer}(\textit{pFlObject}, \textit{flag})

    \vspace{-1.5ex}

    \rule{\textwidth}{0.5\fboxrule}
\setlength{\parskip}{2ex}
    Uses double buffering on a per-object basis. Currently double buffering
    for objects having a non-rectangular box might not work well. A 
    nonrectangular box means that there are regions within the bounding box
    that should not be painted, which is not easily done without complex 
    and expensive clipping and unacceptable inefficiency. Since Xlib 
    doesn't support double buffering, XForms library simulates this 
    functionality with pixmap bit-bliting. In practice, the effect is 
    hardly distinguishable from double buffering and performance is on par 
    with multi-buffering extensions (it is slower than drawing into a 
    window directly on most workstations however). Bear in mind that a 
    pixmap can be resource hungry, so use this option with discretion.

\setlength{\parskip}{1ex}
      \textbf{Parameters}
      \vspace{-1ex}

      \begin{quote}
        \begin{Ventry}{xxxxxxxxx}

          \item[pFlObject]

          object

            {\it (type=pointer to xfdata.FL\_OBJECT)}

          \item[flag]

          flag to disable/enable double buffer. Values 0 (disabled) or 1 
          (enabled)

        \end{Ventry}

      \end{quote}

\textbf{Example:} fl\_set\_object\_dblbuffer(pobj7, 1)



\textbf{Status:} Tested + Doc + Demo = OK



    \end{boxedminipage}

    \label{xformslib:flbasic:fl_set_object_color}
    \index{xformslib \textit{(package)}!xformslib.flbasic \textit{(module)}!xformslib.flbasic.fl\_set\_object\_color \textit{(function)}}

    \vspace{0.5ex}

\hspace{.8\funcindent}\begin{boxedminipage}{\funcwidth}

    \raggedright \textbf{fl\_set\_object\_color}(\textit{pFlObject}, \textit{fgcolr}, \textit{bgcolr})

    \vspace{-1.5ex}

    \rule{\textwidth}{0.5\fboxrule}
\setlength{\parskip}{2ex}
    Sets the color of an object.

\setlength{\parskip}{1ex}
      \textbf{Parameters}
      \vspace{-1ex}

      \begin{quote}
        \begin{Ventry}{xxxxxxxxx}

          \item[pFlObject]

          object

            {\it (type=pointer to xfdata.FL\_OBJECT)}

          \item[fgcolr]

          foreground color value

            {\it (type=long\_pos)}

          \item[bgcolr]

          background color value

            {\it (type=long\_pos)}

        \end{Ventry}

      \end{quote}

\textbf{Example:} fl\_set\_object\_color(pbutob7, xfdata.FL\_AQUA, xfdata.FL\_WHEAT)



\textbf{Status:} Tested + Doc + NoDemo = OK



    \end{boxedminipage}

    \label{xformslib:flbasic:fl_get_object_color}
    \index{xformslib \textit{(package)}!xformslib.flbasic \textit{(module)}!xformslib.flbasic.fl\_get\_object\_color \textit{(function)}}

    \vspace{0.5ex}

\hspace{.8\funcindent}\begin{boxedminipage}{\funcwidth}

    \raggedright \textbf{fl\_get\_object\_color}(\textit{pFlObject})

    \vspace{-1.5ex}

    \rule{\textwidth}{0.5\fboxrule}
\setlength{\parskip}{2ex}
    Returns the foreground and background colors of an object.

\setlength{\parskip}{1ex}
      \textbf{Parameters}
      \vspace{-1ex}

      \begin{quote}
        \begin{Ventry}{xxxxxxxxx}

          \item[pFlObject]

          object

            {\it (type=pointer to xfdata.FL\_OBJECT)}

        \end{Ventry}

      \end{quote}

      \textbf{Return Value}
    \vspace{-1ex}

      \begin{quote}
      foreground and background color values (colr, colr)

      {\it (type=long\_pos, long\_pos)}

      \end{quote}

\textbf{Example:} primcol, secncol = fl\_get\_object\_color(pobj)



\textbf{Attention:} API change from XForms - upstream was fl\_set\_object\_color(pFlObject, 
fgcolr, bgcolr)



\textbf{Status:} Tested + Doc + Demo = OK



    \end{boxedminipage}

    \label{xformslib:flbasic:fl_set_object_label}
    \index{xformslib \textit{(package)}!xformslib.flbasic \textit{(module)}!xformslib.flbasic.fl\_set\_object\_label \textit{(function)}}

    \vspace{0.5ex}

\hspace{.8\funcindent}\begin{boxedminipage}{\funcwidth}

    \raggedright \textbf{fl\_set\_object\_label}(\textit{pFlObject}, \textit{label})

    \vspace{-1.5ex}

    \rule{\textwidth}{0.5\fboxrule}
\setlength{\parskip}{2ex}
    Sets the label of an object.

\setlength{\parskip}{1ex}
      \textbf{Parameters}
      \vspace{-1ex}

      \begin{quote}
        \begin{Ventry}{xxxxxxxxx}

          \item[pFlObject]

          object

            {\it (type=pointer to xfdata.FL\_OBJECT)}

          \item[label]

          text label of object

            {\it (type=str)}

        \end{Ventry}

      \end{quote}

\textbf{Example:} fl\_set\_object\_label(pobj, "My New Label")



\textbf{Status:} Tested + Doc + NoDemo = OK



    \end{boxedminipage}

    \label{xformslib:flbasic:fl_get_object_label}
    \index{xformslib \textit{(package)}!xformslib.flbasic \textit{(module)}!xformslib.flbasic.fl\_get\_object\_label \textit{(function)}}

    \vspace{0.5ex}

\hspace{.8\funcindent}\begin{boxedminipage}{\funcwidth}

    \raggedright \textbf{fl\_get\_object\_label}(\textit{pFlObject})

    \vspace{-1.5ex}

    \rule{\textwidth}{0.5\fboxrule}
\setlength{\parskip}{2ex}
    Returns the label of an object.

\setlength{\parskip}{1ex}
      \textbf{Parameters}
      \vspace{-1ex}

      \begin{quote}
        \begin{Ventry}{xxxxxxxxx}

          \item[pFlObject]

          object

            {\it (type=pointer to xfdata.FL\_OBJECT)}

        \end{Ventry}

      \end{quote}

      \textbf{Return Value}
    \vspace{-1ex}

      \begin{quote}
      text of label

      {\it (type=str)}

      \end{quote}

\textbf{Example:} currlbl = fl\_get\_object\_label(pobj)



\textbf{Status:} Tested + Doc + Demo = OK



    \end{boxedminipage}

    \label{xformslib:flbasic:fl_set_object_helper}
    \index{xformslib \textit{(package)}!xformslib.flbasic \textit{(module)}!xformslib.flbasic.fl\_set\_object\_helper \textit{(function)}}

    \vspace{0.5ex}

\hspace{.8\funcindent}\begin{boxedminipage}{\funcwidth}

    \raggedright \textbf{fl\_set\_object\_helper}(\textit{pFlObject}, \textit{tip})

    \vspace{-1.5ex}

    \rule{\textwidth}{0.5\fboxrule}
\setlength{\parskip}{2ex}
    Sets the tooltip of an object (with possible embedded newlines in it) 
    that will be shown when the mouse hovers over the object for more than 
    about 600 msec.

\setlength{\parskip}{1ex}
      \textbf{Parameters}
      \vspace{-1ex}

      \begin{quote}
        \begin{Ventry}{xxxxxxxxx}

          \item[pFlObject]

          object

            {\it (type=pointer to xfdata.FL\_OBJECT)}

          \item[tip]

          tooltip text for object

            {\it (type=str)}

        \end{Ventry}

      \end{quote}

\textbf{Example:} fl\_set\_object\_helper(pobj, "Button to exit the procedure.")



\textbf{Status:} Tested + Doc + Demo = OK



    \end{boxedminipage}

    \label{xformslib:flbasic:fl_set_object_position}
    \index{xformslib \textit{(package)}!xformslib.flbasic \textit{(module)}!xformslib.flbasic.fl\_set\_object\_position \textit{(function)}}

    \vspace{0.5ex}

\hspace{.8\funcindent}\begin{boxedminipage}{\funcwidth}

    \raggedright \textbf{fl\_set\_object\_position}(\textit{pFlObject}, \textit{x}, \textit{y})

    \vspace{-1.5ex}

    \rule{\textwidth}{0.5\fboxrule}
\setlength{\parskip}{2ex}
    Sets position of an object.

\setlength{\parskip}{1ex}
      \textbf{Parameters}
      \vspace{-1ex}

      \begin{quote}
        \begin{Ventry}{xxxxxxxxx}

          \item[pFlObject]

          object

            {\it (type=pointer to xfdata.FL\_OBJECT)}

          \item[x]

          horizontal position (upper-left corner)

            {\it (type=int)}

          \item[y]

          vertical position (upper-left corner)

            {\it (type=int)}

        \end{Ventry}

      \end{quote}

\textbf{Example:} fl\_set\_object\_position(pobj, 235, 123)



\textbf{Status:} Tested + Doc + Demo = OK



    \end{boxedminipage}

    \label{xformslib:flbasic:fl_get_object_size}
    \index{xformslib \textit{(package)}!xformslib.flbasic \textit{(module)}!xformslib.flbasic.fl\_get\_object\_size \textit{(function)}}

    \vspace{0.5ex}

\hspace{.8\funcindent}\begin{boxedminipage}{\funcwidth}

    \raggedright \textbf{fl\_get\_object\_size}(\textit{pFlObject})

    \vspace{-1.5ex}

    \rule{\textwidth}{0.5\fboxrule}
\setlength{\parskip}{2ex}
    Returns the size of an object.

\setlength{\parskip}{1ex}
      \textbf{Parameters}
      \vspace{-1ex}

      \begin{quote}
        \begin{Ventry}{xxxxxxxxx}

          \item[pFlObject]

          object to evaluate

            {\it (type=pointer to xfdata.FL\_OBJECT)}

        \end{Ventry}

      \end{quote}

      \textbf{Return Value}
    \vspace{-1ex}

      \begin{quote}
      width (w) and height (h) in coord units

      {\it (type=int, int)}

      \end{quote}

\textbf{Example:} wid, hei = fl\_get\_object\_size(pobj)



\textbf{Attention:} API change from XForms - upstream was fl\_get\_object\_size(pFlObject, w, 
h)



\textbf{Status:} Tested + Doc + NoDemo = OK



    \end{boxedminipage}

    \label{xformslib:flbasic:fl_set_object_size}
    \index{xformslib \textit{(package)}!xformslib.flbasic \textit{(module)}!xformslib.flbasic.fl\_set\_object\_size \textit{(function)}}

    \vspace{0.5ex}

\hspace{.8\funcindent}\begin{boxedminipage}{\funcwidth}

    \raggedright \textbf{fl\_set\_object\_size}(\textit{pFlObject}, \textit{w}, \textit{h})

    \vspace{-1.5ex}

    \rule{\textwidth}{0.5\fboxrule}
\setlength{\parskip}{2ex}
    Sets the size of an object.

\setlength{\parskip}{1ex}
      \textbf{Parameters}
      \vspace{-1ex}

      \begin{quote}
        \begin{Ventry}{xxxxxxxxx}

          \item[pFlObject]

          object

            {\it (type=pointer to xfdata.FL\_OBJECT)}

          \item[w]

          width of object in coord units

            {\it (type=int)}

          \item[h]

          height of object in coord units

            {\it (type=int)}

        \end{Ventry}

      \end{quote}

\textbf{Example:} fl\_set\_object\_size(pobj, 90, 35)



\textbf{Status:} Tested + Doc + NoDemo = OK



    \end{boxedminipage}

    \label{xformslib:flbasic:fl_set_object_automatic}
    \index{xformslib \textit{(package)}!xformslib.flbasic \textit{(module)}!xformslib.flbasic.fl\_set\_object\_automatic \textit{(function)}}

    \vspace{0.5ex}

\hspace{.8\funcindent}\begin{boxedminipage}{\funcwidth}

    \raggedright \textbf{fl\_set\_object\_automatic}(\textit{pFlObject}, \textit{flag})

    \vspace{-1.5ex}

    \rule{\textwidth}{0.5\fboxrule}
\setlength{\parskip}{2ex}
    Enables or disables an object to receive a xfdata.FL\_STEP event. This 
    should not be used with built-in objects. An object is automatic if it 
    automatically (without user actions) has to change its contents. 
    Automatic objects get a FL\_STEP event about every 50 msec.

\setlength{\parskip}{1ex}
      \textbf{Parameters}
      \vspace{-1ex}

      \begin{quote}
        \begin{Ventry}{xxxxxxxxx}

          \item[pFlObject]

          object

            {\it (type=pointer to xfdata.FL\_OBJECT)}

          \item[flag]

          flag if automatic or not. Values 0 (not automatic) or 1 
          (automatic)

            {\it (type=int)}

        \end{Ventry}

      \end{quote}

\textbf{Example:} fl\_set\_object\_automatic(pMyobj, 1)



\textbf{Status:} Tested + Doc + NoDemo = OK



    \end{boxedminipage}

    \label{xformslib:flbasic:fl_object_is_automatic}
    \index{xformslib \textit{(package)}!xformslib.flbasic \textit{(module)}!xformslib.flbasic.fl\_object\_is\_automatic \textit{(function)}}

    \vspace{0.5ex}

\hspace{.8\funcindent}\begin{boxedminipage}{\funcwidth}

    \raggedright \textbf{fl\_object\_is\_automatic}(\textit{pFlObject})

    \vspace{-1.5ex}

    \rule{\textwidth}{0.5\fboxrule}
\setlength{\parskip}{2ex}
    Returns if an object receives xfdata.FL\_STEP events. An object is 
    automatic if it automatically (without user actions) has to change its 
    contents. Automatic objects get a FL\_STEP event about every 50 msec.

\setlength{\parskip}{1ex}
      \textbf{Parameters}
      \vspace{-1ex}

      \begin{quote}
        \begin{Ventry}{xxxxxxxxx}

          \item[pFlObject]

          object to evaluate

            {\it (type=pointer to xfdata.FL\_OBJECT)}

        \end{Ventry}

      \end{quote}

      \textbf{Return Value}
    \vspace{-1ex}

      \begin{quote}
      flag if it's automatic (1) or not (0)

      {\it (type=int)}

      \end{quote}

\textbf{Example:} if xf.fl\_object\_is\_automatic(pMyobj): ...



\textbf{Status:} Tested + Doc + NoDemo = OK



    \end{boxedminipage}

    \label{xformslib:flbasic:fl_draw_object_label}
    \index{xformslib \textit{(package)}!xformslib.flbasic \textit{(module)}!xformslib.flbasic.fl\_draw\_object\_label \textit{(function)}}

    \vspace{0.5ex}

\hspace{.8\funcindent}\begin{boxedminipage}{\funcwidth}

    \raggedright \textbf{fl\_draw\_object\_label}(\textit{pFlObject})

    \vspace{-1.5ex}

    \rule{\textwidth}{0.5\fboxrule}
\setlength{\parskip}{2ex}
    Draws the label according to the alignment, which could be inside or 
    outside of the bounding box.

\setlength{\parskip}{1ex}
      \textbf{Parameters}
      \vspace{-1ex}

      \begin{quote}
        \begin{Ventry}{xxxxxxxxx}

          \item[pFlObject]

          object

            {\it (type=pointer to xfdata.FL\_OBJECT)}

        \end{Ventry}

      \end{quote}

\textbf{Example:} fl\_draw\_object\_label(pobj3)



\textbf{Status:} Tested + Doc + NoDemo = OK



    \end{boxedminipage}

    \label{xformslib:flbasic:fl_draw_object_label_outside}
    \index{xformslib \textit{(package)}!xformslib.flbasic \textit{(module)}!xformslib.flbasic.fl\_draw\_object\_label\_outside \textit{(function)}}

    \vspace{0.5ex}

\hspace{.8\funcindent}\begin{boxedminipage}{\funcwidth}

    \raggedright \textbf{fl\_draw\_object\_label\_outside}(\textit{pFlObject})

    \vspace{-1.5ex}

    \rule{\textwidth}{0.5\fboxrule}
\setlength{\parskip}{2ex}
    Draws the label outside of the bounding box.

\setlength{\parskip}{1ex}
      \textbf{Parameters}
      \vspace{-1ex}

      \begin{quote}
        \begin{Ventry}{xxxxxxxxx}

          \item[pFlObject]

          object

            {\it (type=pointer to xfdata.FL\_OBJECT)}

        \end{Ventry}

      \end{quote}

\textbf{Example:} fl\_draw\_object\_label\_outside(pobj3)



\textbf{Status:} Tested + Doc + NoDemo = OK



    \end{boxedminipage}

    \label{xformslib:flbasic:fl_draw_object_label_outside}
    \index{xformslib \textit{(package)}!xformslib.flbasic \textit{(module)}!xformslib.flbasic.fl\_draw\_object\_label\_outside \textit{(function)}}

    \vspace{0.5ex}

\hspace{.8\funcindent}\begin{boxedminipage}{\funcwidth}

    \raggedright \textbf{fl\_draw\_object\_outside\_label}(\textit{pFlObject})

    \vspace{-1.5ex}

    \rule{\textwidth}{0.5\fboxrule}
\setlength{\parskip}{2ex}
    Draws the label outside of the bounding box.

\setlength{\parskip}{1ex}
      \textbf{Parameters}
      \vspace{-1ex}

      \begin{quote}
        \begin{Ventry}{xxxxxxxxx}

          \item[pFlObject]

          object

            {\it (type=pointer to xfdata.FL\_OBJECT)}

        \end{Ventry}

      \end{quote}

\textbf{Example:} fl\_draw\_object\_label\_outside(pobj3)



\textbf{Status:} Tested + Doc + NoDemo = OK



    \end{boxedminipage}

    \label{xformslib:flbasic:fl_get_object_component}
    \index{xformslib \textit{(package)}!xformslib.flbasic \textit{(module)}!xformslib.flbasic.fl\_get\_object\_component \textit{(function)}}

    \vspace{0.5ex}

\hspace{.8\funcindent}\begin{boxedminipage}{\funcwidth}

    \raggedright \textbf{fl\_get\_object\_component}(\textit{pFlObject}, \textit{objclass}, \textit{compontype}, \textit{seqnum})

    \vspace{-1.5ex}

    \rule{\textwidth}{0.5\fboxrule}
\setlength{\parskip}{2ex}
    Returns the object that is a component of a composite object. E.g. the 
    scrollbar object is made of a slider and two scroll buttons.

\setlength{\parskip}{1ex}
      \textbf{Parameters}
      \vspace{-1ex}

      \begin{quote}
        \begin{Ventry}{xxxxxxxxxx}

          \item[pFlObject]

          composite object

            {\it (type=pointer to xfdata.FL\_OBJECT)}

          \item[objclass]

          component object's class id

            {\it (type=int)}

          \item[compontype]

          component object's type id

            {\it (type=int)}

          \item[seqnum]

          the sequence number of the desired object in case the composite 
          has more than one object of the same class and type. You can use 
          -1 to indicate any type of specified class

            {\it (type=int)}

        \end{Ventry}

      \end{quote}

      \textbf{Return Value}
    \vspace{-1ex}

      \begin{quote}
      component object (pFlObject) or None (no object found)

      {\it (type=pointer to xfdata.FL\_OBJECT)}

      \end{quote}

\textbf{Example:} fl\_get\_object\_component(browserobj, xfdata.FL\_SCROLLBAR, 
xfdata.FL\_HOR\_THIN\_SCROLLBAR, 0)



\textbf{Status:} Untested + Doc + NoDemo = NOT OK



    \end{boxedminipage}

    \label{xformslib:flbasic:fl_for_all_objects}
    \index{xformslib \textit{(package)}!xformslib.flbasic \textit{(module)}!xformslib.flbasic.fl\_for\_all\_objects \textit{(function)}}

    \vspace{0.5ex}

\hspace{.8\funcindent}\begin{boxedminipage}{\funcwidth}

    \raggedright \textbf{fl\_for\_all\_objects}(\textit{pFlForm}, \textit{py\_operatecb}, \textit{vdata})

    \vspace{-1.5ex}

    \rule{\textwidth}{0.5\fboxrule}
\setlength{\parskip}{2ex}
    Serves as an iterator to change an attribute for all objects on a 
    particular form. Specified operating function is called for every 
    object of the form form unless it returns nonzero, which terminates the
    iterator.

\setlength{\parskip}{1ex}
      \textbf{Parameters}
      \vspace{-1ex}

      \begin{quote}
        \begin{Ventry}{xxxxxxxxxxxx}

          \item[pFlForm]

          form

            {\it (type=pointer to xfdata.FL\_FORM)}

          \item[py\_operatecb]

          python callback function, returning value

            {\it (type=\_\_ funcname (pFlObject, ptr\_void) -{\textgreater} num \_\_)}

          \item[vdata]

          user data to be passed to function

            {\it (type=None or long or pointer to xfdata.FL\_OBJECT)}

        \end{Ventry}

      \end{quote}

\textbf{Example:}
\begin{quote}
  \begin{itemize}

  \item
    \setlength{\parskip}{0.6ex}
def operatecb(pobj, vdata):



  \item {\textbar}-{\textgreater}{\textbar} ...



  \item {\textbar}-{\textgreater}{\textbar} return 0



  \item fl\_for\_all\_objects(pform5, operatecb, None)



\end{itemize}

\end{quote}

\textbf{Status:} Untested + Doc + NoDemo = NOT OK



    \end{boxedminipage}

    \label{xformslib:flbasic:fl_set_object_dblclick}
    \index{xformslib \textit{(package)}!xformslib.flbasic \textit{(module)}!xformslib.flbasic.fl\_set\_object\_dblclick \textit{(function)}}

    \vspace{0.5ex}

\hspace{.8\funcindent}\begin{boxedminipage}{\funcwidth}

    \raggedright \textbf{fl\_set\_object\_dblclick}(\textit{pFlObject}, \textit{timeout})

    \vspace{-1.5ex}

    \rule{\textwidth}{0.5\fboxrule}
\setlength{\parskip}{2ex}
    Sets double click timeout value of an object, enabling or disabling it 
    to receive the xfdata.FL\_DBLCLICK event.

\setlength{\parskip}{1ex}
      \textbf{Parameters}
      \vspace{-1ex}

      \begin{quote}
        \begin{Ventry}{xxxxxxxxx}

          \item[pFlObject]

          object

            {\it (type=pointer to xfdata.FL\_OBJECT)}

          \item[timeout]

          maximum time interval (in msec) between two clicks for them to be
          considered a double-click (using 0 disables double-click 
          detection)

            {\it (type=long\_pos)}

        \end{Ventry}

      \end{quote}

\textbf{Example:} fl\_set\_object\_dblclick(pobj, 750)



\textbf{Status:} Tested + Doc + NoDemo = OK



    \end{boxedminipage}

    \label{xformslib:flbasic:fl_get_object_dblclick}
    \index{xformslib \textit{(package)}!xformslib.flbasic \textit{(module)}!xformslib.flbasic.fl\_get\_object\_dblclick \textit{(function)}}

    \vspace{0.5ex}

\hspace{.8\funcindent}\begin{boxedminipage}{\funcwidth}

    \raggedright \textbf{fl\_get\_object\_dblclick}(\textit{pFlObject})

    \vspace{-1.5ex}

    \rule{\textwidth}{0.5\fboxrule}
\setlength{\parskip}{2ex}
    Return double click timeout value of an object.

\setlength{\parskip}{1ex}
      \textbf{Parameters}
      \vspace{-1ex}

      \begin{quote}
        \begin{Ventry}{xxxxxxxxx}

          \item[pFlObject]

          object to evaluate

            {\it (type=pointer to xfdata.FL\_OBJECT)}

        \end{Ventry}

      \end{quote}

      \textbf{Return Value}
    \vspace{-1ex}

      \begin{quote}
      timeout value

      {\it (type=long\_pos)}

      \end{quote}

\textbf{Example:} dctim = fl\_get\_object\_dblclick(pobj0)



\textbf{Status:} Tested + Doc + NoDemo = OK



    \end{boxedminipage}

    \label{xformslib:flbasic:fl_set_object_geometry}
    \index{xformslib \textit{(package)}!xformslib.flbasic \textit{(module)}!xformslib.flbasic.fl\_set\_object\_geometry \textit{(function)}}

    \vspace{0.5ex}

\hspace{.8\funcindent}\begin{boxedminipage}{\funcwidth}

    \raggedright \textbf{fl\_set\_object\_geometry}(\textit{pFlObject}, \textit{x}, \textit{y}, \textit{w}, \textit{h})

    \vspace{-1.5ex}

    \rule{\textwidth}{0.5\fboxrule}
\setlength{\parskip}{2ex}
    Sets the geometry (position and size) of an object.

\setlength{\parskip}{1ex}
      \textbf{Parameters}
      \vspace{-1ex}

      \begin{quote}
        \begin{Ventry}{xxxxxxxxx}

          \item[pFlObject]

          object to modify

            {\it (type=pointer to xfdata.FL\_OBJECT)}

          \item[x]

          horizontal position (upper-left corner)

            {\it (type=int)}

          \item[y]

          vertical position (upper-left corner)

            {\it (type=int)}

          \item[w]

          width in coord units

            {\it (type=int)}

          \item[h]

          height in coord units

            {\it (type=int)}

        \end{Ventry}

      \end{quote}

\textbf{Example:} fl\_set\_object\_geometry(pobj, 200, 250, 120, 25)



\textbf{Status:} Tested + Doc + Demo = OK



    \end{boxedminipage}

    \label{xformslib:flbasic:fl_move_object}
    \index{xformslib \textit{(package)}!xformslib.flbasic \textit{(module)}!xformslib.flbasic.fl\_move\_object \textit{(function)}}

    \vspace{0.5ex}

\hspace{.8\funcindent}\begin{boxedminipage}{\funcwidth}

    \raggedright \textbf{fl\_move\_object}(\textit{pFlObject}, \textit{x}, \textit{y})

    \vspace{-1.5ex}

    \rule{\textwidth}{0.5\fboxrule}
\setlength{\parskip}{2ex}
    Moves an object to a new position.

\setlength{\parskip}{1ex}
      \textbf{Parameters}
      \vspace{-1ex}

      \begin{quote}
        \begin{Ventry}{xxxxxxxxx}

          \item[pFlObject]

          object to be moved

            {\it (type=pointer to xfdata.FL\_OBJECT)}

          \item[x]

          new horizontal position (upper-left corner)

            {\it (type=int)}

          \item[y]

          new vertical position (upper-left corner)

            {\it (type=int)}

        \end{Ventry}

      \end{quote}

\textbf{Example:} fl\_move\_object(pobj0, 120, 380)



\textbf{Status:} Tested + Doc + NoDemo = OK



    \end{boxedminipage}

    \label{xformslib:flbasic:fl_fit_object_label}
    \index{xformslib \textit{(package)}!xformslib.flbasic \textit{(module)}!xformslib.flbasic.fl\_fit\_object\_label \textit{(function)}}

    \vspace{0.5ex}

\hspace{.8\funcindent}\begin{boxedminipage}{\funcwidth}

    \raggedright \textbf{fl\_fit\_object\_label}(\textit{pFlObject}, \textit{xmargin}, \textit{ymargin})

    \vspace{-1.5ex}

    \rule{\textwidth}{0.5\fboxrule}
\setlength{\parskip}{2ex}
    Checks if the label of an object fits into it (after x- and y-margin 
    have been added). If not, all objects and the form are enlarged by the 
    necessary factor (but never by more than a factor of 1.5).

\setlength{\parskip}{1ex}
      \textbf{Parameters}
      \vspace{-1ex}

      \begin{quote}
        \begin{Ventry}{xxxxxxxxx}

          \item[pFlObject]

          object

            {\it (type=pointer to xfdata.FL\_OBJECT)}

          \item[xmargin]

          horizontal margin of label in coord units

            {\it (type=int)}

          \item[ymargin]

          vertical margin of label in coord units

            {\it (type=int)}

        \end{Ventry}

      \end{quote}

\textbf{Example:} fl\_fit\_object\_label(pobj2, 10, 10)



\textbf{Status:} Tested + Doc + NoDemo = OK



    \end{boxedminipage}

    \label{xformslib:flbasic:fl_get_object_geometry}
    \index{xformslib \textit{(package)}!xformslib.flbasic \textit{(module)}!xformslib.flbasic.fl\_get\_object\_geometry \textit{(function)}}

    \vspace{0.5ex}

\hspace{.8\funcindent}\begin{boxedminipage}{\funcwidth}

    \raggedright \textbf{fl\_get\_object\_geometry}(\textit{pFlObject})

    \vspace{-1.5ex}

    \rule{\textwidth}{0.5\fboxrule}
\setlength{\parskip}{2ex}
    Returns the geometry (position and size) of an object.

\setlength{\parskip}{1ex}
      \textbf{Parameters}
      \vspace{-1ex}

      \begin{quote}
        \begin{Ventry}{xxxxxxxxx}

          \item[pFlObject]

          object

            {\it (type=pointer to xfdata.FL\_OBJECT)}

        \end{Ventry}

      \end{quote}

      \textbf{Return Value}
    \vspace{-1ex}

      \begin{quote}
      horizontal (x), vertical position (y), width (w), height (h)

      {\it (type=int, int, int, int)}

      \end{quote}

\textbf{Example:} xpos, ypos, wid, hei = fl\_get\_object\_geometry(pobj1)



\textbf{Attention:} API change from XForms - upstream was fl\_get\_object\_geometry(pFlObject, 
x, y, w, h)



\textbf{Status:} Tested + Doc + Demo = OK



    \end{boxedminipage}

    \label{xformslib:flbasic:fl_get_object_position}
    \index{xformslib \textit{(package)}!xformslib.flbasic \textit{(module)}!xformslib.flbasic.fl\_get\_object\_position \textit{(function)}}

    \vspace{0.5ex}

\hspace{.8\funcindent}\begin{boxedminipage}{\funcwidth}

    \raggedright \textbf{fl\_get\_object\_position}(\textit{pFlObject})

    \vspace{-1.5ex}

    \rule{\textwidth}{0.5\fboxrule}
\setlength{\parskip}{2ex}
    Returns the position of an object.

\setlength{\parskip}{1ex}
      \textbf{Parameters}
      \vspace{-1ex}

      \begin{quote}
        \begin{Ventry}{xxxxxxxxx}

          \item[pFlObject]

          object to evaluate

            {\it (type=pointer to xfdata.FL\_OBJECT)}

        \end{Ventry}

      \end{quote}

      \textbf{Return Value}
    \vspace{-1ex}

      \begin{quote}
      horizontal (x) and vertical position (y)

      {\it (type=int, int)}

      \end{quote}

\textbf{Example:} xpos, ypos = fl\_get\_object\_position(pobj2)



\textbf{Attention:} API change from XForms - upstream was fl\_get\_object\_position(pFlObject, 
x, y)



\textbf{Status:} Tested + Doc + NoDemo = OK



    \end{boxedminipage}

    \label{xformslib:flbasic:fl_get_object_bbox}
    \index{xformslib \textit{(package)}!xformslib.flbasic \textit{(module)}!xformslib.flbasic.fl\_get\_object\_bbox \textit{(function)}}

    \vspace{0.5ex}

\hspace{.8\funcindent}\begin{boxedminipage}{\funcwidth}

    \raggedright \textbf{fl\_get\_object\_bbox}(\textit{pFlObject})

    \vspace{-1.5ex}

    \rule{\textwidth}{0.5\fboxrule}
\setlength{\parskip}{2ex}
    Returns the bounding box size that has the label, which could be drawn 
    outside of the object figured in.

\setlength{\parskip}{1ex}
      \textbf{Parameters}
      \vspace{-1ex}

      \begin{quote}
        \begin{Ventry}{xxxxxxxxx}

          \item[pFlObject]

          object to evaluate

            {\it (type=pointer to xfdata.FL\_OBJECT)}

        \end{Ventry}

      \end{quote}

      \textbf{Return Value}
    \vspace{-1ex}

      \begin{quote}
      horizontal (x), vertical position (y), width (w), height (h)

      {\it (type=int, int, int, int)}

      \end{quote}

\textbf{Example:} xpos, ypos, wid, hei = fl\_get\_object\_bbox(pobj4)



\textbf{Attention:} API change from XForms - upstream was fl\_get\_object\_bbox(pFlObject, x, 
y, w, h)



\textbf{Status:} Tested + Doc + NoDemo = OK



    \end{boxedminipage}

    \label{xformslib:flbasic:fl_get_object_bbox}
    \index{xformslib \textit{(package)}!xformslib.flbasic \textit{(module)}!xformslib.flbasic.fl\_get\_object\_bbox \textit{(function)}}

    \vspace{0.5ex}

\hspace{.8\funcindent}\begin{boxedminipage}{\funcwidth}

    \raggedright \textbf{fl\_compute\_object\_geometry}(\textit{pFlObject})

    \vspace{-1.5ex}

    \rule{\textwidth}{0.5\fboxrule}
\setlength{\parskip}{2ex}
    Returns the bounding box size that has the label, which could be drawn 
    outside of the object figured in.

\setlength{\parskip}{1ex}
      \textbf{Parameters}
      \vspace{-1ex}

      \begin{quote}
        \begin{Ventry}{xxxxxxxxx}

          \item[pFlObject]

          object to evaluate

            {\it (type=pointer to xfdata.FL\_OBJECT)}

        \end{Ventry}

      \end{quote}

      \textbf{Return Value}
    \vspace{-1ex}

      \begin{quote}
      horizontal (x), vertical position (y), width (w), height (h)

      {\it (type=int, int, int, int)}

      \end{quote}

\textbf{Example:} xpos, ypos, wid, hei = fl\_get\_object\_bbox(pobj4)



\textbf{Attention:} API change from XForms - upstream was fl\_get\_object\_bbox(pFlObject, x, 
y, w, h)



\textbf{Status:} Tested + Doc + NoDemo = OK



    \end{boxedminipage}

    \label{xformslib:flbasic:fl_call_object_callback}
    \index{xformslib \textit{(package)}!xformslib.flbasic \textit{(module)}!xformslib.flbasic.fl\_call\_object\_callback \textit{(function)}}

    \vspace{0.5ex}

\hspace{.8\funcindent}\begin{boxedminipage}{\funcwidth}

    \raggedright \textbf{fl\_call\_object\_callback}(\textit{pFlObject})

    \vspace{-1.5ex}

    \rule{\textwidth}{0.5\fboxrule}
\setlength{\parskip}{2ex}
    Invokes the callback manually (as opposed to invocation by the main 
    loop). If the object does not have a callback associated with it, this 
    call has not effect.

\setlength{\parskip}{1ex}
      \textbf{Parameters}
      \vspace{-1ex}

      \begin{quote}
        \begin{Ventry}{xxxxxxxxx}

          \item[pFlObject]

          object

            {\it (type=pointer to xfdata.FL\_OBJECT)}

        \end{Ventry}

      \end{quote}

\textbf{Example:} fl\_call\_object\_callback(pobj\_with\_cb)



\textbf{Status:} Tested + Doc + Demo = OK



    \end{boxedminipage}

    \label{xformslib:flbasic:fl_set_object_prehandler}
    \index{xformslib \textit{(package)}!xformslib.flbasic \textit{(module)}!xformslib.flbasic.fl\_set\_object\_prehandler \textit{(function)}}

    \vspace{0.5ex}

\hspace{.8\funcindent}\begin{boxedminipage}{\funcwidth}

    \raggedright \textbf{fl\_set\_object\_prehandler}(\textit{pFlObject}, \textit{py\_HandlerPtr})

    \vspace{-1.5ex}

    \rule{\textwidth}{0.5\fboxrule}
\setlength{\parskip}{2ex}
    By-passes the internal event processing for a particular object. The 
    pre-handler will be called before the built-in object handler. By 
    electing to handle some of the events, a pre-handler can, in effect, 
    replace part of the built-in handler.

\setlength{\parskip}{1ex}
      \textbf{Parameters}
      \vspace{-1ex}

      \begin{quote}
        \begin{Ventry}{xxxxxxxxxxxxx}

          \item[pFlObject]

          object

            {\it (type=pointer to xfdata.FL\_OBJECT)}

          \item[py\_HandlerPtr]

          python callback function, returning value

            {\it (type=\_\_ funcname (pFlObject, num, coord, coord, num, ptr\_void) 
-{\textgreater} num \_\_)}

        \end{Ventry}

      \end{quote}

      \textbf{Return Value}
    \vspace{-1ex}

      \begin{quote}
      old xfdata.FL\_HANDLEPTR function

      \end{quote}

\textbf{Example:}
\begin{quote}
  \begin{itemize}

  \item
    \setlength{\parskip}{0.6ex}
def prehandlecb(pobj, num, crd, crd, num2, vdata):



  \item {\textbar}-{\textgreater}{\textbar} ...



  \item {\textbar}-{\textgreater}{\textbar} return 0



  \item fl\_set\_object\_prehandler(pobj2, prehandlecb)



\end{itemize}

\end{quote}

\textbf{Status:} Tested + Doc + Demo = OK



    \end{boxedminipage}

    \label{xformslib:flbasic:fl_set_object_posthandler}
    \index{xformslib \textit{(package)}!xformslib.flbasic \textit{(module)}!xformslib.flbasic.fl\_set\_object\_posthandler \textit{(function)}}

    \vspace{0.5ex}

\hspace{.8\funcindent}\begin{boxedminipage}{\funcwidth}

    \raggedright \textbf{fl\_set\_object\_posthandler}(\textit{pFlObject}, \textit{py\_HandlerPtr})

    \vspace{-1.5ex}

    \rule{\textwidth}{0.5\fboxrule}
\setlength{\parskip}{2ex}
    By-passes the internal event processing for a particular object. The 
    post-handler will be invoked after the built-in handler finishes. 
    Whenever possible a post-handler should be used instead of a 
    pre-handler.

\setlength{\parskip}{1ex}
      \textbf{Parameters}
      \vspace{-1ex}

      \begin{quote}
        \begin{Ventry}{xxxxxxxxxxxxx}

          \item[pFlObject]

          object

            {\it (type=pointer to xfdata.FL\_OBJECT)}

          \item[py\_HandlerPtr]

          python callback function, returning value

            {\it (type=\_\_ funcname (pFlObject, num, coord, coord, num, ptr\_void) 
-{\textgreater} num \_\_)}

        \end{Ventry}

      \end{quote}

      \textbf{Return Value}
    \vspace{-1ex}

      \begin{quote}
      old xfdata.FL\_HANDLEPTR function

      \end{quote}

\textbf{Example:}
\begin{quote}
  \begin{itemize}

  \item
    \setlength{\parskip}{0.6ex}
def posthandlecb(pobj, num, crd, crd, num2, vdata):



  \item {\textbar}-{\textgreater}{\textbar} ...



  \item {\textbar}-{\textgreater}{\textbar} return 0



  \item fl\_set\_object\_posthandler(pobj2, posthandlecb)



\end{itemize}

\end{quote}

\textbf{Status:} Tested + Doc + Demo = OK



    \end{boxedminipage}

    \label{xformslib:flbasic:fl_set_object_callback}
    \index{xformslib \textit{(package)}!xformslib.flbasic \textit{(module)}!xformslib.flbasic.fl\_set\_object\_callback \textit{(function)}}

    \vspace{0.5ex}

\hspace{.8\funcindent}\begin{boxedminipage}{\funcwidth}

    \raggedright \textbf{fl\_set\_object\_callback}(\textit{pFlObject}, \textit{py\_CallbackPtr}, \textit{data})

    \vspace{-1.5ex}

    \rule{\textwidth}{0.5\fboxrule}
\setlength{\parskip}{2ex}
    Calls a callback function bound to an object, if a condition is met.

\setlength{\parskip}{1ex}
      \textbf{Parameters}
      \vspace{-1ex}

      \begin{quote}
        \begin{Ventry}{xxxxxxxxxxxxxx}

          \item[pFlObject]

          object the callback is bound to

            {\it (type=pointer to xfdata.FL\_OBJECT)}

          \item[py\_CallbackPtr]

          a python function with no () and no args to be used as callback, 
          no return

            {\it (type=\_\_ funcname (pFlObject, longnum) \_\_)}

          \item[data]

          argument being passed to function

            {\it (type=long)}

        \end{Ventry}

      \end{quote}

      \textbf{Return Value}
    \vspace{-1ex}

      \begin{quote}
      old xfdata.FL\_CALLBACKPTR function

      \end{quote}

\textbf{Example:}
\begin{quote}
  \begin{itemize}

  \item
    \setlength{\parskip}{0.6ex}
def myobcb(pobj, longdata):



  \item {\textbar}-{\textgreater}{\textbar} ...



  \item oldcb = fl\_set\_object\_callback(pobj3, myobjcb, 0)



\end{itemize}

\end{quote}

\textbf{Status:} Tested + Doc + Demo = OK



    \end{boxedminipage}

    \label{xformslib:flbasic:fl_set_object_callback}
    \index{xformslib \textit{(package)}!xformslib.flbasic \textit{(module)}!xformslib.flbasic.fl\_set\_object\_callback \textit{(function)}}

    \vspace{0.5ex}

\hspace{.8\funcindent}\begin{boxedminipage}{\funcwidth}

    \raggedright \textbf{fl\_set\_call\_back}(\textit{pFlObject}, \textit{py\_CallbackPtr}, \textit{data})

    \vspace{-1.5ex}

    \rule{\textwidth}{0.5\fboxrule}
\setlength{\parskip}{2ex}
    Calls a callback function bound to an object, if a condition is met.

\setlength{\parskip}{1ex}
      \textbf{Parameters}
      \vspace{-1ex}

      \begin{quote}
        \begin{Ventry}{xxxxxxxxxxxxxx}

          \item[pFlObject]

          object the callback is bound to

            {\it (type=pointer to xfdata.FL\_OBJECT)}

          \item[py\_CallbackPtr]

          a python function with no () and no args to be used as callback, 
          no return

            {\it (type=\_\_ funcname (pFlObject, longnum) \_\_)}

          \item[data]

          argument being passed to function

            {\it (type=long)}

        \end{Ventry}

      \end{quote}

      \textbf{Return Value}
    \vspace{-1ex}

      \begin{quote}
      old xfdata.FL\_CALLBACKPTR function

      \end{quote}

\textbf{Example:}
\begin{quote}
  \begin{itemize}

  \item
    \setlength{\parskip}{0.6ex}
def myobcb(pobj, longdata):



  \item {\textbar}-{\textgreater}{\textbar} ...



  \item oldcb = fl\_set\_object\_callback(pobj3, myobjcb, 0)



\end{itemize}

\end{quote}

\textbf{Status:} Tested + Doc + Demo = OK



    \end{boxedminipage}

    \label{xformslib:flbasic:fl_redraw_object}
    \index{xformslib \textit{(package)}!xformslib.flbasic \textit{(module)}!xformslib.flbasic.fl\_redraw\_object \textit{(function)}}

    \vspace{0.5ex}

\hspace{.8\funcindent}\begin{boxedminipage}{\funcwidth}

    \raggedright \textbf{fl\_redraw\_object}(\textit{pFlObject})

    \vspace{-1.5ex}

    \rule{\textwidth}{0.5\fboxrule}
\setlength{\parskip}{2ex}
    Redraws the particular object. If it is a group it redraws the complete
    group. Normally you should never need this routine because all library 
    routines take care of redrawing objects when necessary, but there might
    be situations in which an explicit redraw is required.

\setlength{\parskip}{1ex}
      \textbf{Parameters}
      \vspace{-1ex}

      \begin{quote}
        \begin{Ventry}{xxxxxxxxx}

          \item[pFlObject]

          object to redraw

            {\it (type=pointer to xfdata.FL\_OBJECT)}

        \end{Ventry}

      \end{quote}

\textbf{Example:} fl\_redraw\_object(pobj)



\textbf{Status:} Tested + Doc + Demo = OK



    \end{boxedminipage}

    \label{xformslib:flbasic:fl_scale_object}
    \index{xformslib \textit{(package)}!xformslib.flbasic \textit{(module)}!xformslib.flbasic.fl\_scale\_object \textit{(function)}}

    \vspace{0.5ex}

\hspace{.8\funcindent}\begin{boxedminipage}{\funcwidth}

    \raggedright \textbf{fl\_scale\_object}(\textit{pFlObject}, \textit{xs}, \textit{ys})

    \vspace{-1.5ex}

    \rule{\textwidth}{0.5\fboxrule}
\setlength{\parskip}{2ex}
    Scales (shrinking or enlarging) an object, indicating a scaling factor 
    in x- and y-direction (1.1 = 110 percent, 0.5 = 50, etc.)

\setlength{\parskip}{1ex}
      \textbf{Parameters}
      \vspace{-1ex}

      \begin{quote}
        \begin{Ventry}{xxxxxxxxx}

          \item[pFlObject]

          object to be scaled

            {\it (type=pointer to xfdata.FL\_OBJECT)}

          \item[xs]

          new horizontal factor

            {\it (type=float)}

          \item[ys]

          new vertical factor

            {\it (type=float)}

        \end{Ventry}

      \end{quote}

\textbf{Example:} fl\_scale\_object(pobj, 0.8, 1.1)



\textbf{Status:} Tested + Doc + NoDemo = OK



    \end{boxedminipage}

    \label{xformslib:flbasic:fl_show_object}
    \index{xformslib \textit{(package)}!xformslib.flbasic \textit{(module)}!xformslib.flbasic.fl\_show\_object \textit{(function)}}

    \vspace{0.5ex}

\hspace{.8\funcindent}\begin{boxedminipage}{\funcwidth}

    \raggedright \textbf{fl\_show\_object}(\textit{pFlObject})

    \vspace{-1.5ex}

    \rule{\textwidth}{0.5\fboxrule}
\setlength{\parskip}{2ex}
    Shows an (hidden) object.

\setlength{\parskip}{1ex}
      \textbf{Parameters}
      \vspace{-1ex}

      \begin{quote}
        \begin{Ventry}{xxxxxxxxx}

          \item[pFlObject]

          object to be shown

            {\it (type=pointer to xfdata.FL\_OBJECT)}

        \end{Ventry}

      \end{quote}

\textbf{Example:} fl\_show\_object(pobj8)



\textbf{Status:} Tested + Doc + Demo = OK



    \end{boxedminipage}

    \label{xformslib:flbasic:fl_hide_object}
    \index{xformslib \textit{(package)}!xformslib.flbasic \textit{(module)}!xformslib.flbasic.fl\_hide\_object \textit{(function)}}

    \vspace{0.5ex}

\hspace{.8\funcindent}\begin{boxedminipage}{\funcwidth}

    \raggedright \textbf{fl\_hide\_object}(\textit{pFlObject})

    \vspace{-1.5ex}

    \rule{\textwidth}{0.5\fboxrule}
\setlength{\parskip}{2ex}
    Hides a shown object.

\setlength{\parskip}{1ex}
      \textbf{Parameters}
      \vspace{-1ex}

      \begin{quote}
        \begin{Ventry}{xxxxxxxxx}

          \item[pFlObject]

          object to be hidden

            {\it (type=pointer to xfdata.FL\_OBJECT)}

        \end{Ventry}

      \end{quote}

\textbf{Example:} fl\_hide\_object(pobj8)



\textbf{Status:} Tested + Doc + Demo = OK



    \end{boxedminipage}

    \label{xformslib:flbasic:fl_object_is_visible}
    \index{xformslib \textit{(package)}!xformslib.flbasic \textit{(module)}!xformslib.flbasic.fl\_object\_is\_visible \textit{(function)}}

    \vspace{0.5ex}

\hspace{.8\funcindent}\begin{boxedminipage}{\funcwidth}

    \raggedright \textbf{fl\_object\_is\_visible}(\textit{pFlObject})

    \vspace{-1.5ex}

    \rule{\textwidth}{0.5\fboxrule}
\setlength{\parskip}{2ex}
    Returns if an object is visible or not.

\setlength{\parskip}{1ex}
      \textbf{Parameters}
      \vspace{-1ex}

      \begin{quote}
        \begin{Ventry}{xxxxxxxxx}

          \item[pFlObject]

          object to evaluate

            {\it (type=pointer to xfdata.FL\_OBJECT)}

        \end{Ventry}

      \end{quote}

      \textbf{Return Value}
    \vspace{-1ex}

      \begin{quote}
      flag 0 (invisible) or non-zero (visible)

      {\it (type=int)}

      \end{quote}

\textbf{Example:} if not fl\_object\_is\_visible(pobj2): ...



\textbf{Status:} Tested + Doc + Demo = OK



    \end{boxedminipage}

    \label{xformslib:flbasic:fl_free_object}
    \index{xformslib \textit{(package)}!xformslib.flbasic \textit{(module)}!xformslib.flbasic.fl\_free\_object \textit{(function)}}

    \vspace{0.5ex}

\hspace{.8\funcindent}\begin{boxedminipage}{\funcwidth}

    \raggedright \textbf{fl\_free\_object}(\textit{pFlObject})

    \vspace{-1.5ex}

    \rule{\textwidth}{0.5\fboxrule}
\setlength{\parskip}{2ex}
    Frees the object and finally destroys it (if necessary deletes the 
    object first).

\setlength{\parskip}{1ex}
      \textbf{Parameters}
      \vspace{-1ex}

      \begin{quote}
        \begin{Ventry}{xxxxxxxxx}

          \item[pFlObject]

          object to free

            {\it (type=pointer to xfdata.FL\_OBJECT)}

        \end{Ventry}

      \end{quote}

\textbf{Example:} fl\_free\_object(pobj)



\textbf{Status:} Tested + Doc + NoDemo = OK



    \end{boxedminipage}

    \label{xformslib:flbasic:fl_delete_object}
    \index{xformslib \textit{(package)}!xformslib.flbasic \textit{(module)}!xformslib.flbasic.fl\_delete\_object \textit{(function)}}

    \vspace{0.5ex}

\hspace{.8\funcindent}\begin{boxedminipage}{\funcwidth}

    \raggedright \textbf{fl\_delete\_object}(\textit{pFlObject})

    \vspace{-1.5ex}

    \rule{\textwidth}{0.5\fboxrule}
\setlength{\parskip}{2ex}
    Deletes an object, breaking its connection to the form, but not 
    destroying it.

\setlength{\parskip}{1ex}
      \textbf{Parameters}
      \vspace{-1ex}

      \begin{quote}
        \begin{Ventry}{xxxxxxxxx}

          \item[pFlObject]

          object to delete

            {\it (type=pointer to xfdata.FL\_OBJECT)}

        \end{Ventry}

      \end{quote}

\textbf{Example:} fl\_delete\_object(pobj)



\textbf{Status:} Tested + Doc + NoDemo = OK



    \end{boxedminipage}

    \label{xformslib:flbasic:fl_get_object_return_state}
    \index{xformslib \textit{(package)}!xformslib.flbasic \textit{(module)}!xformslib.flbasic.fl\_get\_object\_return\_state \textit{(function)}}

    \vspace{0.5ex}

\hspace{.8\funcindent}\begin{boxedminipage}{\funcwidth}

    \raggedright \textbf{fl\_get\_object\_return\_state}(\textit{pFlObject})

    \vspace{-1.5ex}

    \rule{\textwidth}{0.5\fboxrule}
\setlength{\parskip}{2ex}
    Determines the reason an object was returned (or its callback invoked) 
    you. The returned value is logical OR of the conditions that led to the
    object getting returned.

\setlength{\parskip}{1ex}
      \textbf{Parameters}
      \vspace{-1ex}

      \begin{quote}
        \begin{Ventry}{xxxxxxxxx}

          \item[pFlObject]

          object to evaluate

            {\it (type=pointer to xfdata.FL\_OBJECT)}

        \end{Ventry}

      \end{quote}

      \textbf{Return Value}
    \vspace{-1ex}

      \begin{quote}
      current return state

      {\it (type=int)}

      \end{quote}

\textbf{Example:} currstate = fl\_get\_object\_return\_state(pobj5)



\textbf{Status:} Tested + Doc + NoDemo = OK



    \end{boxedminipage}

    \label{xformslib:flbasic:fl_trigger_object}
    \index{xformslib \textit{(package)}!xformslib.flbasic \textit{(module)}!xformslib.flbasic.fl\_trigger\_object \textit{(function)}}

    \vspace{0.5ex}

\hspace{.8\funcindent}\begin{boxedminipage}{\funcwidth}

    \raggedright \textbf{fl\_trigger\_object}(\textit{pFlObject})

    \vspace{-1.5ex}

    \rule{\textwidth}{0.5\fboxrule}
\setlength{\parskip}{2ex}
    Simulates the action of an object being triggered from within the 
    program. Calling this routine on an object obj results in the object 
    returned to the application program or its callback being called if it 
    exists. Note however, there is no visual feedback, i.e. 
    fl\_trigger\_object(button) will not make the button object named 
    button appearing to be pushed.

\setlength{\parskip}{1ex}
      \textbf{Parameters}
      \vspace{-1ex}

      \begin{quote}
        \begin{Ventry}{xxxxxxxxx}

          \item[pFlObject]

          object to trigger

            {\it (type=pointer to xfdata.FL\_OBJECT)}

        \end{Ventry}

      \end{quote}

\textbf{Example:} fl\_trigger\_object(pobj



\textbf{Status:} Tested + Doc + NoDemo = OK



    \end{boxedminipage}

    \label{xformslib:flbasic:fl_activate_object}
    \index{xformslib \textit{(package)}!xformslib.flbasic \textit{(module)}!xformslib.flbasic.fl\_activate\_object \textit{(function)}}

    \vspace{0.5ex}

\hspace{.8\funcindent}\begin{boxedminipage}{\funcwidth}

    \raggedright \textbf{fl\_activate\_object}(\textit{pFlObject})

    \vspace{-1.5ex}

    \rule{\textwidth}{0.5\fboxrule}
\setlength{\parskip}{2ex}
    (Re)activates an object, (re)enabling user interaction.

\setlength{\parskip}{1ex}
      \textbf{Parameters}
      \vspace{-1ex}

      \begin{quote}
        \begin{Ventry}{xxxxxxxxx}

          \item[pFlObject]

          object to activate

            {\it (type=pointer to xfdata.FL\_OBJECT)}

        \end{Ventry}

      \end{quote}

\textbf{Example:} fl\_activate\_object(pobj)



\textbf{Status:} Tested + Doc + Demo = OK



    \end{boxedminipage}

    \label{xformslib:flbasic:fl_deactivate_object}
    \index{xformslib \textit{(package)}!xformslib.flbasic \textit{(module)}!xformslib.flbasic.fl\_deactivate\_object \textit{(function)}}

    \vspace{0.5ex}

\hspace{.8\funcindent}\begin{boxedminipage}{\funcwidth}

    \raggedright \textbf{fl\_deactivate\_object}(\textit{pFlObject})

    \vspace{-1.5ex}

    \rule{\textwidth}{0.5\fboxrule}
\setlength{\parskip}{2ex}
    Makes a particular object to be temporarily inactive, disabling user 
    interaction, e.g., you want to make it impossible for the user to press
    a particular button or to type input in a particular field. When object
    is a group, the whole group is deactivate.

\setlength{\parskip}{1ex}
      \textbf{Parameters}
      \vspace{-1ex}

      \begin{quote}
        \begin{Ventry}{xxxxxxxxx}

          \item[pFlObject]

          object to deactivate

            {\it (type=pointer to xfdata.FL\_OBJECT)}

        \end{Ventry}

      \end{quote}

\textbf{Example:} fl\_deactivate\_object(pactobj)



\textbf{Status:} Tested + Doc + Demo = OK



    \end{boxedminipage}

    \label{xformslib:flbasic:fl_object_is_active}
    \index{xformslib \textit{(package)}!xformslib.flbasic \textit{(module)}!xformslib.flbasic.fl\_object\_is\_active \textit{(function)}}

    \vspace{0.5ex}

\hspace{.8\funcindent}\begin{boxedminipage}{\funcwidth}

    \raggedright \textbf{fl\_object\_is\_active}(\textit{pFlObject})

    \vspace{-1.5ex}

    \rule{\textwidth}{0.5\fboxrule}
\setlength{\parskip}{2ex}
    Returns if object is active and reacting to events, or not.

\setlength{\parskip}{1ex}
      \textbf{Parameters}
      \vspace{-1ex}

      \begin{quote}
        \begin{Ventry}{xxxxxxxxx}

          \item[pFlObject]

          object to evaluate

            {\it (type=pointer to xfdata.FL\_OBJECT)}

        \end{Ventry}

      \end{quote}

      \textbf{Return Value}
    \vspace{-1ex}

      \begin{quote}
      flag 0 (not active) or non-zero (active)

      {\it (type=int)}

      \end{quote}

\textbf{Example:} if not fl\_object\_is\_active(pobj): ...



\textbf{Status:} Tested + Doc + Demo = OK



    \end{boxedminipage}

    \label{xformslib:flbasic:fl_enumerate_fonts}
    \index{xformslib \textit{(package)}!xformslib.flbasic \textit{(module)}!xformslib.flbasic.fl\_enumerate\_fonts \textit{(function)}}

    \vspace{0.5ex}

\hspace{.8\funcindent}\begin{boxedminipage}{\funcwidth}

    \raggedright \textbf{fl\_enumerate\_fonts}(\textit{py\_output}, \textit{shortform})

    \vspace{-1.5ex}

    \rule{\textwidth}{0.5\fboxrule}
\setlength{\parskip}{2ex}
    Lists built-in fonts.

\setlength{\parskip}{1ex}
      \textbf{Parameters}
      \vspace{-1ex}

      \begin{quote}
        \begin{Ventry}{xxxxxxxxx}

          \item[py\_output]

          python callback function, no return

            {\it (type=\_\_ funcname (string) \_\_)}

          \item[shortform]

          flag to use short form or not ({\textless}int{\textgreater})

            {\it (type=0 (long form used) or non-zero (short form used))}

        \end{Ventry}

      \end{quote}

      \textbf{Return Value}
    \vspace{-1ex}

      \begin{quote}
      number of listed fonts

      {\it (type=int)}

      \end{quote}

\textbf{Example:}
\begin{quote}
  \begin{itemize}

  \item
    \setlength{\parskip}{0.6ex}
def pyoutput(strng):



  \item {\textbar}-{\textgreater}{\textbar} print strng



  \item numfonts = fl\_enumerate(pyoutput, 0)



\end{itemize}

\end{quote}

\textbf{Status:} Tested + Doc + Demo = OK



    \end{boxedminipage}

    \label{xformslib:flbasic:fl_set_font_name}
    \index{xformslib \textit{(package)}!xformslib.flbasic \textit{(module)}!xformslib.flbasic.fl\_set\_font\_name \textit{(function)}}

    \vspace{0.5ex}

\hspace{.8\funcindent}\begin{boxedminipage}{\funcwidth}

    \raggedright \textbf{fl\_set\_font\_name}(\textit{fontnum}, \textit{name})

    \vspace{-1.5ex}

    \rule{\textwidth}{0.5\fboxrule}
\setlength{\parskip}{2ex}
    Add a new font (indexed by number) or change an existing font. 
    Preferably the font name contains a '?' in the size position so 
    different sizes can be used. Redraw of all forms is required to 
    actually see the change for visible form.

\setlength{\parskip}{1ex}
      \textbf{Parameters}
      \vspace{-1ex}

      \begin{quote}
        \begin{Ventry}{xxxxxxx}

          \item[fontnum]

          font number. Values between 0 and xfdata.FL\_MAXFONTS-1

            {\it (type=int)}

          \item[name]

          font name

            {\it (type=str)}

        \end{Ventry}

      \end{quote}

      \textbf{Return Value}
    \vspace{-1ex}

      \begin{quote}
      -1 (on errors) or 0 or 1

      {\it (type=int)}

      \end{quote}

\textbf{Example:} fl\_set\_font\_name(40, "symbol-medium-whatever") ?



\textbf{Status:} Tested + Doc + NoDemo = OK



    \end{boxedminipage}

    \label{xformslib:flbasic:fl_set_font}
    \index{xformslib \textit{(package)}!xformslib.flbasic \textit{(module)}!xformslib.flbasic.fl\_set\_font \textit{(function)}}

    \vspace{0.5ex}

\hspace{.8\funcindent}\begin{boxedminipage}{\funcwidth}

    \raggedright \textbf{fl\_set\_font}(\textit{fontnum}, \textit{size})

    \vspace{-1.5ex}

    \rule{\textwidth}{0.5\fboxrule}
\setlength{\parskip}{2ex}
    Makes the specified font as the current.

\setlength{\parskip}{1ex}
      \textbf{Parameters}
      \vspace{-1ex}

      \begin{quote}
        \begin{Ventry}{xxxxxxx}

          \item[fontnum]

          font number

            {\it (type=int)}

          \item[size]

          font size. Values (from xfdata module) FL\_TINY\_SIZE, 
          FL\_SMALL\_SIZE, FL\_NORMAL\_SIZE, FL\_MEDIUM\_SIZE, 
          FL\_LARGE\_SIZE, FL\_HUGE\_SIZE, FL\_DEFAULT\_SIZE

            {\it (type=int)}

        \end{Ventry}

      \end{quote}

\textbf{Example:} fl\_set\_font(5, xfdata.FL\_SMALL\_SIZE)



\textbf{Status:} Tested + Doc + NoDemo = OK



    \end{boxedminipage}

    \label{xformslib:flbasic:fl_get_char_height}
    \index{xformslib \textit{(package)}!xformslib.flbasic \textit{(module)}!xformslib.flbasic.fl\_get\_char\_height \textit{(function)}}

    \vspace{0.5ex}

\hspace{.8\funcindent}\begin{boxedminipage}{\funcwidth}

    \raggedright \textbf{fl\_get\_char\_height}(\textit{style}, \textit{size})

    \vspace{-1.5ex}

    \rule{\textwidth}{0.5\fboxrule}
\setlength{\parskip}{2ex}
    Returns the maximum height of the used font and the height above and 
    below the baseline of the font.

\setlength{\parskip}{1ex}
      \textbf{Parameters}
      \vspace{-1ex}

      \begin{quote}
        \begin{Ventry}{xxxxx}

          \item[style]

          font style. Values (from xfdata module) FL\_NORMAL\_STYLE, 
          FL\_BOLD\_STYLE, FL\_ITALIC\_STYLE, FL\_BOLDITALIC\_STYLE, 
          FL\_FIXED\_STYLE, FL\_FIXEDBOLD\_STYLE, FL\_FIXEDITALIC\_STYLE, 
          FL\_FIXEDBOLDITALIC\_STYLE, FL\_TIMES\_STYLE, 
          FL\_TIMESBOLD\_STYLE, FL\_TIMESITALIC\_STYLE, 
          FL\_TIMESBOLDITALIC\_STYLE, FL\_MISC\_STYLE, FL\_MISCBOLD\_STYLE,
          FL\_MISCITALIC\_STYLE, FL\_SYMBOL\_STYLE, FL\_SHADOW\_STYLE, 
          FL\_ENGRAVED\_STYLE, FL\_EMBOSSED\_STYLE

            {\it (type=int)}

          \item[size]

          font size. Values (from xfdata module) FL\_TINY\_SIZE, 
          FL\_SMALL\_SIZE, FL\_NORMAL\_SIZE, FL\_MEDIUM\_SIZE, 
          FL\_LARGE\_SIZE, FL\_HUGE\_SIZE, FL\_DEFAULT\_SIZE

            {\it (type=int)}

        \end{Ventry}

      \end{quote}

      \textbf{Return Value}
    \vspace{-1ex}

      \begin{quote}
      height (h), ascendent (asc) and descendent (desc)

      {\it (type=int, int, int)}

      \end{quote}

\textbf{Example:} hei, asc, desc = fl\_get\_char\_height(xfdata.FL\_BOLD\_STYLE, 
xfdata.FL\_TINY\_SIZE)



\textbf{Attention:} API change from XForms - upstream was fl\_get\_char\_height(style, size, 
asc, desc)



\textbf{Status:} Tested + Doc + NoDemo = OK



    \end{boxedminipage}

    \label{xformslib:flbasic:fl_get_char_width}
    \index{xformslib \textit{(package)}!xformslib.flbasic \textit{(module)}!xformslib.flbasic.fl\_get\_char\_width \textit{(function)}}

    \vspace{0.5ex}

\hspace{.8\funcindent}\begin{boxedminipage}{\funcwidth}

    \raggedright \textbf{fl\_get\_char\_width}(\textit{style}, \textit{size})

    \vspace{-1.5ex}

    \rule{\textwidth}{0.5\fboxrule}
\setlength{\parskip}{2ex}
    Returns the maximum width of the used font.

\setlength{\parskip}{1ex}
      \textbf{Parameters}
      \vspace{-1ex}

      \begin{quote}
        \begin{Ventry}{xxxxx}

          \item[style]

          font style. Values (from xfdata module) FL\_NORMAL\_STYLE, 
          FL\_BOLD\_STYLE, FL\_ITALIC\_STYLE, FL\_BOLDITALIC\_STYLE, 
          FL\_FIXED\_STYLE, FL\_FIXEDBOLD\_STYLE, FL\_FIXEDITALIC\_STYLE, 
          FL\_FIXEDBOLDITALIC\_STYLE, FL\_TIMES\_STYLE, 
          FL\_TIMESBOLD\_STYLE, FL\_TIMESITALIC\_STYLE, 
          FL\_TIMESBOLDITALIC\_STYLE, FL\_MISC\_STYLE, FL\_MISCBOLD\_STYLE,
          FL\_MISCITALIC\_STYLE, FL\_SYMBOL\_STYLE, FL\_SHADOW\_STYLE, 
          FL\_ENGRAVED\_STYLE, FL\_EMBOSSED\_STYLE

            {\it (type=int)}

          \item[size]

          font size. Values (from xfdata module) FL\_TINY\_SIZE, 
          FL\_SMALL\_SIZE, FL\_NORMAL\_SIZE, FL\_MEDIUM\_SIZE, 
          FL\_LARGE\_SIZE, FL\_HUGE\_SIZE, FL\_DEFAULT\_SIZE

            {\it (type=int)}

        \end{Ventry}

      \end{quote}

      \textbf{Return Value}
    \vspace{-1ex}

      \begin{quote}
      width (w)

      {\it (type=int)}

      \end{quote}

\textbf{Example:} wid = fl\_get\_char\_width(xfdata.FL\_TIMES\_STYLE, xfdata.FL\_HUGE\_SIZE)



\textbf{Status:} Tested + Doc + NoDemo = OK



    \end{boxedminipage}

    \label{xformslib:flbasic:fl_get_string_height}
    \index{xformslib \textit{(package)}!xformslib.flbasic \textit{(module)}!xformslib.flbasic.fl\_get\_string\_height \textit{(function)}}

    \vspace{0.5ex}

\hspace{.8\funcindent}\begin{boxedminipage}{\funcwidth}

    \raggedright \textbf{fl\_get\_string\_height}(\textit{style}, \textit{size}, \textit{txtstr}, \textit{strlng})

    \vspace{-1.5ex}

    \rule{\textwidth}{0.5\fboxrule}
\setlength{\parskip}{2ex}
    Obtains the height information of a specific string and the height 
    above and below the font's baseline.

\setlength{\parskip}{1ex}
      \textbf{Parameters}
      \vspace{-1ex}

      \begin{quote}
        \begin{Ventry}{xxxxxx}

          \item[style]

          font style. Values (from xfdata module) FL\_NORMAL\_STYLE, 
          FL\_BOLD\_STYLE, FL\_ITALIC\_STYLE, FL\_BOLDITALIC\_STYLE, 
          FL\_FIXED\_STYLE, FL\_FIXEDBOLD\_STYLE, FL\_FIXEDITALIC\_STYLE, 
          FL\_FIXEDBOLDITALIC\_STYLE, FL\_TIMES\_STYLE, 
          FL\_TIMESBOLD\_STYLE, FL\_TIMESITALIC\_STYLE, 
          FL\_TIMESBOLDITALIC\_STYLE, FL\_MISC\_STYLE, FL\_MISCBOLD\_STYLE,
          FL\_MISCITALIC\_STYLE, FL\_SYMBOL\_STYLE, FL\_SHADOW\_STYLE, 
          FL\_ENGRAVED\_STYLE, FL\_EMBOSSED\_STYLE

            {\it (type=int)}

          \item[size]

          font size. Values (from xfdata module) i.e. FL\_TINY\_SIZE, 
          FL\_SMALL\_SIZE, FL\_NORMAL\_SIZE, FL\_MEDIUM\_SIZE, 
          FL\_LARGE\_SIZE, FL\_HUGE\_SIZE, FL\_DEFAULT\_SIZE

            {\it (type=int)}

          \item[txtstr]

          text

            {\it (type=str)}

          \item[strlng]

          length of text

            {\it (type=int)}

        \end{Ventry}

      \end{quote}

      \textbf{Return Value}
    \vspace{-1ex}

      \begin{quote}
      height (h), ascendent (asc) and descendent (desc)

      {\it (type=int, int, int)}

      \end{quote}

\textbf{Example:} hei, asc, desc = fl\_get\_string\_height(xfdata.FL\_MISC\_STYLE, 
xfdata.FL\_MEDIUM\_SIZE, "Mystring", 8)



\textbf{Attention:} API change from XForms - upstream was fl\_get\_string\_height(style, size, 
strng, strglen, asc, desc)



\textbf{Status:} Tested + Doc + Demo = OK



    \end{boxedminipage}

    \label{xformslib:flbasic:fl_get_string_width}
    \index{xformslib \textit{(package)}!xformslib.flbasic \textit{(module)}!xformslib.flbasic.fl\_get\_string\_width \textit{(function)}}

    \vspace{0.5ex}

\hspace{.8\funcindent}\begin{boxedminipage}{\funcwidth}

    \raggedright \textbf{fl\_get\_string\_width}(\textit{style}, \textit{size}, \textit{txtstr}, \textit{strlng})

    \vspace{-1.5ex}

    \rule{\textwidth}{0.5\fboxrule}
\setlength{\parskip}{2ex}
    Obtains the width information for a specific string.

\setlength{\parskip}{1ex}
      \textbf{Parameters}
      \vspace{-1ex}

      \begin{quote}
        \begin{Ventry}{xxxxxx}

          \item[style]

          font style. Values (from xfdata module) FL\_NORMAL\_STYLE, 
          FL\_BOLD\_STYLE, FL\_ITALIC\_STYLE, FL\_BOLDITALIC\_STYLE, 
          FL\_FIXED\_STYLE, FL\_FIXEDBOLD\_STYLE, FL\_FIXEDITALIC\_STYLE, 
          FL\_FIXEDBOLDITALIC\_STYLE, FL\_TIMES\_STYLE, 
          FL\_TIMESBOLD\_STYLE, FL\_TIMESITALIC\_STYLE, 
          FL\_TIMESBOLDITALIC\_STYLE, FL\_MISC\_STYLE, FL\_MISCBOLD\_STYLE,
          FL\_MISCITALIC\_STYLE, FL\_SYMBOL\_STYLE, FL\_SHADOW\_STYLE, 
          FL\_ENGRAVED\_STYLE, FL\_EMBOSSED\_STYLE

            {\it (type=int)}

          \item[size]

          font size. Values (from xfdata module) FL\_TINY\_SIZE, 
          FL\_SMALL\_SIZE, FL\_NORMAL\_SIZE, FL\_MEDIUM\_SIZE, 
          FL\_LARGE\_SIZE, FL\_HUGE\_SIZE, FL\_DEFAULT\_SIZE

            {\it (type=int)}

          \item[txtstr]

          text

            {\it (type=str)}

          \item[strlng]

          length of text

            {\it (type=int)}

        \end{Ventry}

      \end{quote}

      \textbf{Return Value}
    \vspace{-1ex}

      \begin{quote}
      width (w)

      {\it (type=int)}

      \end{quote}

\textbf{Example:} wid = fl\_get\_string\_width(xfdata.FL\_MISC\_STYLE, 
xfdata.FL\_MEDIUM\_SIZE, "Mystring", 8)



\textbf{Status:} Tested + Doc + Demo = OK



    \end{boxedminipage}

    \label{xformslib:flbasic:fl_get_string_widthTAB}
    \index{xformslib \textit{(package)}!xformslib.flbasic \textit{(module)}!xformslib.flbasic.fl\_get\_string\_widthTAB \textit{(function)}}

    \vspace{0.5ex}

\hspace{.8\funcindent}\begin{boxedminipage}{\funcwidth}

    \raggedright \textbf{fl\_get\_string\_widthTAB}(\textit{style}, \textit{size}, \textit{txtstr}, \textit{strlng})

    \vspace{-1.5ex}

    \rule{\textwidth}{0.5\fboxrule}
\setlength{\parskip}{2ex}
    ???

\setlength{\parskip}{1ex}
      \textbf{Parameters}
      \vspace{-1ex}

      \begin{quote}
        \begin{Ventry}{xxxxxx}

          \item[style]

          font style. Values (from xfdata module) i.e. FL\_NORMAL\_STYLE, 
          FL\_BOLD\_STYLE, FL\_ITALIC\_STYLE, FL\_BOLDITALIC\_STYLE, 
          FL\_FIXED\_STYLE, FL\_FIXEDBOLD\_STYLE, FL\_FIXEDITALIC\_STYLE, 
          FL\_FIXEDBOLDITALIC\_STYLE, FL\_TIMES\_STYLE, 
          FL\_TIMESBOLD\_STYLE, FL\_TIMESITALIC\_STYLE, 
          FL\_TIMESBOLDITALIC\_STYLE, FL\_MISC\_STYLE, FL\_MISCBOLD\_STYLE,
          FL\_MISCITALIC\_STYLE, FL\_SYMBOL\_STYLE, FL\_SHADOW\_STYLE, 
          FL\_ENGRAVED\_STYLE, FL\_EMBOSSED\_STYLE

            {\it (type=int)}

          \item[size]

          font size. Values (from xfdata module) i.e. FL\_TINY\_SIZE, 
          FL\_SMALL\_SIZE, FL\_NORMAL\_SIZE, FL\_MEDIUM\_SIZE, 
          FL\_LARGE\_SIZE, FL\_HUGE\_SIZE, FL\_DEFAULT\_SIZE

          \item[txtstr]

          text

            {\it (type=str)}

          \item[strlng]

          length of text

            {\it (type=str)}

        \end{Ventry}

      \end{quote}

      \textbf{Return Value}
    \vspace{-1ex}

      \begin{quote}
      width (w)

      {\it (type=int)}

      \end{quote}

\textbf{Example:} wid = fl\_get\_string\_width(xfdata.FL\_MISC\_STYLE, 
xfdata.FL\_MEDIUM\_SIZE, "Mystring", 8)



\textbf{Status:} Untested + NoDoc + NoDemo = NOT OK



    \end{boxedminipage}

    \label{xformslib:flbasic:fl_get_string_dimension}
    \index{xformslib \textit{(package)}!xformslib.flbasic \textit{(module)}!xformslib.flbasic.fl\_get\_string\_dimension \textit{(function)}}

    \vspace{0.5ex}

\hspace{.8\funcindent}\begin{boxedminipage}{\funcwidth}

    \raggedright \textbf{fl\_get\_string\_dimension}(\textit{style}, \textit{size}, \textit{txtstr}, \textit{strlng})

    \vspace{-1.5ex}

    \rule{\textwidth}{0.5\fboxrule}
\setlength{\parskip}{2ex}
    Returns the width and height of a string in one call. In addition, the 
    string passed can contain embedded newline characters and the routine 
    will make proper adjustment so the values returned are (just) large 
    enough to contain the multiple lines of text.

\setlength{\parskip}{1ex}
      \textbf{Parameters}
      \vspace{-1ex}

      \begin{quote}
        \begin{Ventry}{xxxxxx}

          \item[style]

          font style. Values (from xfdata module) FL\_NORMAL\_STYLE, 
          FL\_BOLD\_STYLE, FL\_ITALIC\_STYLE, FL\_BOLDITALIC\_STYLE, 
          FL\_FIXED\_STYLE, FL\_FIXEDBOLD\_STYLE, FL\_FIXEDITALIC\_STYLE, 
          FL\_FIXEDBOLDITALIC\_STYLE, FL\_TIMES\_STYLE, 
          FL\_TIMESBOLD\_STYLE, FL\_TIMESITALIC\_STYLE, 
          FL\_TIMESBOLDITALIC\_STYLE, FL\_MISC\_STYLE, FL\_MISCBOLD\_STYLE,
          FL\_MISCITALIC\_STYLE, FL\_SYMBOL\_STYLE, FL\_SHADOW\_STYLE, 
          FL\_ENGRAVED\_STYLE, FL\_EMBOSSED\_STYLE

            {\it (type=int)}

          \item[size]

          font size. Values (from xfdata module) FL\_TINY\_SIZE, 
          FL\_SMALL\_SIZE, FL\_NORMAL\_SIZE, FL\_MEDIUM\_SIZE, 
          FL\_LARGE\_SIZE, FL\_HUGE\_SIZE, FL\_DEFAULT\_SIZE

            {\it (type=int)}

          \item[txtstr]

          text

            {\it (type=str)}

          \item[strlng]

          length of text

            {\it (type=int)}

        \end{Ventry}

      \end{quote}

      \textbf{Return Value}
    \vspace{-1ex}

      \begin{quote}
      width (w) and height (h)

      {\it (type=int, int)}

      \end{quote}

\textbf{Example:} fl\_get\_string\_dimension(xfdata.FL\_ENGRAVED\_STYLE, 
xfdata.FL\_DEFAULT\_SIZE, "CustomString", 12)



\textbf{Attention:} API change from upstream fl\_get\_string\_dimension(fntstyle, fntsize, 
strng, strglen, w, h)



\textbf{Status:} Tested + Doc + NoDemo = OK



    \end{boxedminipage}

    \label{xformslib:flbasic:fl_get_string_dimension}
    \index{xformslib \textit{(package)}!xformslib.flbasic \textit{(module)}!xformslib.flbasic.fl\_get\_string\_dimension \textit{(function)}}

    \vspace{0.5ex}

\hspace{.8\funcindent}\begin{boxedminipage}{\funcwidth}

    \raggedright \textbf{fl\_get\_string\_size}(\textit{style}, \textit{size}, \textit{txtstr}, \textit{strlng})

    \vspace{-1.5ex}

    \rule{\textwidth}{0.5\fboxrule}
\setlength{\parskip}{2ex}
    Returns the width and height of a string in one call. In addition, the 
    string passed can contain embedded newline characters and the routine 
    will make proper adjustment so the values returned are (just) large 
    enough to contain the multiple lines of text.

\setlength{\parskip}{1ex}
      \textbf{Parameters}
      \vspace{-1ex}

      \begin{quote}
        \begin{Ventry}{xxxxxx}

          \item[style]

          font style. Values (from xfdata module) FL\_NORMAL\_STYLE, 
          FL\_BOLD\_STYLE, FL\_ITALIC\_STYLE, FL\_BOLDITALIC\_STYLE, 
          FL\_FIXED\_STYLE, FL\_FIXEDBOLD\_STYLE, FL\_FIXEDITALIC\_STYLE, 
          FL\_FIXEDBOLDITALIC\_STYLE, FL\_TIMES\_STYLE, 
          FL\_TIMESBOLD\_STYLE, FL\_TIMESITALIC\_STYLE, 
          FL\_TIMESBOLDITALIC\_STYLE, FL\_MISC\_STYLE, FL\_MISCBOLD\_STYLE,
          FL\_MISCITALIC\_STYLE, FL\_SYMBOL\_STYLE, FL\_SHADOW\_STYLE, 
          FL\_ENGRAVED\_STYLE, FL\_EMBOSSED\_STYLE

            {\it (type=int)}

          \item[size]

          font size. Values (from xfdata module) FL\_TINY\_SIZE, 
          FL\_SMALL\_SIZE, FL\_NORMAL\_SIZE, FL\_MEDIUM\_SIZE, 
          FL\_LARGE\_SIZE, FL\_HUGE\_SIZE, FL\_DEFAULT\_SIZE

            {\it (type=int)}

          \item[txtstr]

          text

            {\it (type=str)}

          \item[strlng]

          length of text

            {\it (type=int)}

        \end{Ventry}

      \end{quote}

      \textbf{Return Value}
    \vspace{-1ex}

      \begin{quote}
      width (w) and height (h)

      {\it (type=int, int)}

      \end{quote}

\textbf{Example:} fl\_get\_string\_dimension(xfdata.FL\_ENGRAVED\_STYLE, 
xfdata.FL\_DEFAULT\_SIZE, "CustomString", 12)



\textbf{Attention:} API change from upstream fl\_get\_string\_dimension(fntstyle, fntsize, 
strng, strglen, w, h)



\textbf{Status:} Tested + Doc + NoDemo = OK



    \end{boxedminipage}

    \label{xformslib:flbasic:fl_get_align_xy}
    \index{xformslib \textit{(package)}!xformslib.flbasic \textit{(module)}!xformslib.flbasic.fl\_get\_align\_xy \textit{(function)}}

    \vspace{0.5ex}

\hspace{.8\funcindent}\begin{boxedminipage}{\funcwidth}

    \raggedright \textbf{fl\_get\_align\_xy}(\textit{align}, \textit{x}, \textit{y}, \textit{w}, \textit{h}, \textit{xsize}, \textit{ysize}, \textit{xmargin}, \textit{ymargin})

    \vspace{-1.5ex}

    \rule{\textwidth}{0.5\fboxrule}
\setlength{\parskip}{2ex}
    Obtains the position of where to draw the object with a certain 
    alignment and including padding. It works regardless if it is to be 
    drawn inside or outside of the bounding box

\setlength{\parskip}{1ex}
      \textbf{Parameters}
      \vspace{-1ex}

      \begin{quote}
        \begin{Ventry}{xxxxxxx}

          \item[align]

          alignment. Values (from xfdata module) FL\_ALIGN\_CENTER, 
          FL\_ALIGN\_TOP, FL\_ALIGN\_BOTTOM, FL\_ALIGN\_LEFT, 
          FL\_ALIGN\_RIGHT, FL\_ALIGN\_LEFT\_TOP, FL\_ALIGN\_RIGHT\_TOP, 
          FL\_ALIGN\_LEFT\_BOTTOM, FL\_ALIGN\_RIGHT\_BOTTOM, 
          FL\_ALIGN\_INSIDE, FL\_ALIGN\_VERT

            {\it (type=int)}

          \item[x]

          horizontal position of bounding box (upper-left corner)

            {\it (type=int)}

          \item[y]

          vertical position of bounding box (upper-left corner)

            {\it (type=int)}

          \item[w]

          width of bounding box in coord units

            {\it (type=int)}

          \item[h]

          height of bounding box in coord units

            {\it (type=int)}

          \item[xsize]

          width of the object to be drawn

            {\it (type=int)}

          \item[ysize]

          height of the object to be drawn

            {\it (type=int)}

          \item[xmargin]

          additional horizontal padding to use

            {\it (type=int)}

          \item[ymargin]

          additional vertical padding to use

            {\it (type=int)}

        \end{Ventry}

      \end{quote}

      \textbf{Return Value}
    \vspace{-1ex}

      \begin{quote}
      horizontal (x) and vertical position (y) used for drawing object

      {\it (type=int, int)}

      \end{quote}

\textbf{Example:} xpos, ypos = fl\_get\_align\_xy(xfdata.FL\_ALIGN\_CENTER, 200, 300, 110, 
30, 120, 40, 15, 15)



\textbf{Attention:} API change from XForms - upstream was fl\_get\_align\_xy(align, x, y, w, h,
xsize, ysize, xoff, yoff, xx, yy)



\textbf{Status:} Tested + Doc + NoDemo = OK



    \end{boxedminipage}

    \label{xformslib:flbasic:fl_drw_text}
    \index{xformslib \textit{(package)}!xformslib.flbasic \textit{(module)}!xformslib.flbasic.fl\_drw\_text \textit{(function)}}

    \vspace{0.5ex}

\hspace{.8\funcindent}\begin{boxedminipage}{\funcwidth}

    \raggedright \textbf{fl\_drw\_text}(\textit{align}, \textit{x}, \textit{y}, \textit{w}, \textit{h}, \textit{colr}, \textit{style}, \textit{size}, \textit{txtstr})

    \vspace{-1.5ex}

    \rule{\textwidth}{0.5\fboxrule}
\setlength{\parskip}{2ex}
    Draws the text inside the bounding box according to the alignment 
    requested. It puts a padding of 5 pixels in vertical direction and 4 in
    horizontal around the text. Thus the bounding box should be 10 pixels 
    wider and 8 pixels higher than required for the text to be drawn. It 
    interprets a text string starting with the character @ differently in 
    drawing some symbols instead.

\setlength{\parskip}{1ex}
      \textbf{Parameters}
      \vspace{-1ex}

      \begin{quote}
        \begin{Ventry}{xxxxxx}

          \item[align]

          alignment of text. Values (from xfdata module) FL\_ALIGN\_CENTER,
          FL\_ALIGN\_TOP, FL\_ALIGN\_BOTTOM, FL\_ALIGN\_LEFT, 
          FL\_ALIGN\_RIGHT, FL\_ALIGN\_LEFT\_TOP, FL\_ALIGN\_RIGHT\_TOP, 
          FL\_ALIGN\_LEFT\_BOTTOM, FL\_ALIGN\_RIGHT\_BOTTOM, 
          FL\_ALIGN\_INSIDE, FL\_ALIGN\_VERT

            {\it (type=int)}

          \item[x]

          horizontal position (upper-left corner)

            {\it (type=int)}

          \item[y]

          vertical position (upper-left corner)

            {\it (type=int)}

          \item[w]

          width in coord units

            {\it (type=int)}

          \item[h]

          height in coord units

            {\it (type=int)}

          \item[colr]

          color value

            {\it (type=long\_pos)}

          \item[style]

          font style. Values (from xfdata module) FL\_NORMAL\_STYLE, 
          FL\_BOLD\_STYLE, FL\_ITALIC\_STYLE, FL\_BOLDITALIC\_STYLE, 
          FL\_FIXED\_STYLE, FL\_FIXEDBOLD\_STYLE, FL\_FIXEDITALIC\_STYLE, 
          FL\_FIXEDBOLDITALIC\_STYLE, FL\_TIMES\_STYLE, 
          FL\_TIMESBOLD\_STYLE, FL\_TIMESITALIC\_STYLE, 
          FL\_TIMESBOLDITALIC\_STYLE, FL\_MISC\_STYLE, FL\_MISCBOLD\_STYLE,
          FL\_MISCITALIC\_STYLE, FL\_SYMBOL\_STYLE, FL\_SHADOW\_STYLE, 
          FL\_ENGRAVED\_STYLE, FL\_EMBOSSED\_STYLE

            {\it (type=int)}

          \item[size]

          font size. Values (from xfdata module) FL\_TINY\_SIZE, 
          FL\_SMALL\_SIZE, FL\_NORMAL\_SIZE, FL\_MEDIUM\_SIZE, 
          FL\_LARGE\_SIZE, FL\_HUGE\_SIZE, FL\_DEFAULT\_SIZE

            {\it (type=int)}

          \item[txtstr]

          text to draw

            {\it (type=str)}

        \end{Ventry}

      \end{quote}

\textbf{Example:} fl\_drw\_text(xfdata.FL\_ALIGN\_BOTTOM, 400, 175, 150, 45, 
xfdata.FL\_GREEN, xfdata.FL\_ITALIC\_STYLE, xfdata.FL\_SMALL\_SIZE, "A Good
Old String")



\textbf{Status:} Tested + Doc + NoDemo = OK



    \end{boxedminipage}

    \label{xformslib:flbasic:fl_drw_text_beside}
    \index{xformslib \textit{(package)}!xformslib.flbasic \textit{(module)}!xformslib.flbasic.fl\_drw\_text\_beside \textit{(function)}}

    \vspace{0.5ex}

\hspace{.8\funcindent}\begin{boxedminipage}{\funcwidth}

    \raggedright \textbf{fl\_drw\_text\_beside}(\textit{align}, \textit{x}, \textit{y}, \textit{w}, \textit{h}, \textit{colr}, \textit{style}, \textit{size}, \textit{txtstr})

    \vspace{-1.5ex}

    \rule{\textwidth}{0.5\fboxrule}
\setlength{\parskip}{2ex}
    Draws the text aligned outside of the box. It interprets a text string 
    starting with the character @ differently in drawing some symbols 
    instead.

\setlength{\parskip}{1ex}
      \textbf{Parameters}
      \vspace{-1ex}

      \begin{quote}
        \begin{Ventry}{xxxxxx}

          \item[align]

          alignment of text. Values (from xfdata module) FL\_ALIGN\_CENTER,
          FL\_ALIGN\_TOP, FL\_ALIGN\_BOTTOM, FL\_ALIGN\_LEFT, 
          FL\_ALIGN\_RIGHT, FL\_ALIGN\_LEFT\_TOP, FL\_ALIGN\_RIGHT\_TOP, 
          FL\_ALIGN\_LEFT\_BOTTOM, FL\_ALIGN\_RIGHT\_BOTTOM, 
          FL\_ALIGN\_INSIDE, FL\_ALIGN\_VERT

            {\it (type=int)}

          \item[x]

          horizontal position (upper-left corner)

            {\it (type=int)}

          \item[y]

          vertical position (upper-left corner)

            {\it (type=int)}

          \item[w]

          width in coord units

            {\it (type=int)}

          \item[h]

          height in coord units

            {\it (type=int)}

          \item[colr]

          color value

            {\it (type=long\_pos)}

          \item[style]

          font style. Values (from xfdata module) FL\_NORMAL\_STYLE, 
          FL\_BOLD\_STYLE, FL\_ITALIC\_STYLE, FL\_BOLDITALIC\_STYLE, 
          FL\_FIXED\_STYLE, FL\_FIXEDBOLD\_STYLE, FL\_FIXEDITALIC\_STYLE, 
          FL\_FIXEDBOLDITALIC\_STYLE, FL\_TIMES\_STYLE, 
          FL\_TIMESBOLD\_STYLE, FL\_TIMESITALIC\_STYLE, 
          FL\_TIMESBOLDITALIC\_STYLE, FL\_MISC\_STYLE, FL\_MISCBOLD\_STYLE,
          FL\_MISCITALIC\_STYLE, FL\_SYMBOL\_STYLE, FL\_SHADOW\_STYLE, 
          FL\_ENGRAVED\_STYLE, FL\_EMBOSSED\_STYLE

            {\it (type=int)}

          \item[size]

          font size. Values (from xfdata module) FL\_TINY\_SIZE, 
          FL\_SMALL\_SIZE, FL\_NORMAL\_SIZE, FL\_MEDIUM\_SIZE, 
          FL\_LARGE\_SIZE, FL\_HUGE\_SIZE, FL\_DEFAULT\_SIZE

            {\it (type=int)}

          \item[txtstr]

          text to draw

            {\it (type=str)}

        \end{Ventry}

      \end{quote}

\textbf{Example:} fl\_drw\_text\_beside(xfdata.FL\_ALIGN\_BOTTOM, 400, 175, 150, 45, 
xfdata.FL\_GREEN, xfdata.FL\_ITALIC\_STYLE, xfdata.FL\_SMALL\_SIZE, "A Good
Old String")



\textbf{Status:} Tested + Doc + NoDemo = OK



    \end{boxedminipage}

    \label{xformslib:flbasic:fl_drw_text_cursor}
    \index{xformslib \textit{(package)}!xformslib.flbasic \textit{(module)}!xformslib.flbasic.fl\_drw\_text\_cursor \textit{(function)}}

    \vspace{0.5ex}

\hspace{.8\funcindent}\begin{boxedminipage}{\funcwidth}

    \raggedright \textbf{fl\_drw\_text\_cursor}(\textit{align}, \textit{x}, \textit{y}, \textit{w}, \textit{h}, \textit{colr}, \textit{style}, \textit{size}, \textit{txtstr}, \textit{curscolr}, \textit{pos})

    \vspace{-1.5ex}

    \rule{\textwidth}{0.5\fboxrule}
\setlength{\parskip}{2ex}
    Draw text and, in addition, a cursor can optionally be drawn. It does 
    no interpretation of the special character @ nor does it add padding 
    around the text.

\setlength{\parskip}{1ex}
      \textbf{Parameters}
      \vspace{-1ex}

      \begin{quote}
        \begin{Ventry}{xxxxxxxx}

          \item[align]

          alignment of text. Values (from xfdata module) FL\_ALIGN\_CENTER,
          FL\_ALIGN\_TOP, FL\_ALIGN\_BOTTOM, FL\_ALIGN\_LEFT, 
          FL\_ALIGN\_RIGHT, FL\_ALIGN\_LEFT\_TOP, FL\_ALIGN\_RIGHT\_TOP, 
          FL\_ALIGN\_LEFT\_BOTTOM, FL\_ALIGN\_RIGHT\_BOTTOM, 
          FL\_ALIGN\_INSIDE, FL\_ALIGN\_VERT

            {\it (type=int)}

          \item[x]

          horizontal position (upper-left corner)

            {\it (type=int)}

          \item[y]

          vertical position (upper-left corner)

            {\it (type=int)}

          \item[w]

          width in coord units

            {\it (type=int)}

          \item[h]

          height in coord units

            {\it (type=int)}

          \item[colr]

          color value

            {\it (type=long\_pos)}

          \item[style]

          font style. Values (from xfdata module) FL\_NORMAL\_STYLE, 
          FL\_BOLD\_STYLE, FL\_ITALIC\_STYLE, FL\_BOLDITALIC\_STYLE, 
          FL\_FIXED\_STYLE, FL\_FIXEDBOLD\_STYLE, FL\_FIXEDITALIC\_STYLE, 
          FL\_FIXEDBOLDITALIC\_STYLE, FL\_TIMES\_STYLE, 
          FL\_TIMESBOLD\_STYLE, FL\_TIMESITALIC\_STYLE, 
          FL\_TIMESBOLDITALIC\_STYLE, FL\_MISC\_STYLE, FL\_MISCBOLD\_STYLE,
          FL\_MISCITALIC\_STYLE, FL\_SYMBOL\_STYLE, FL\_SHADOW\_STYLE, 
          FL\_ENGRAVED\_STYLE, FL\_EMBOSSED\_STYLE

            {\it (type=int)}

          \item[size]

          font size. Values (from xfdata module) i.e. FL\_TINY\_SIZE, 
          FL\_SMALL\_SIZE, FL\_NORMAL\_SIZE, FL\_MEDIUM\_SIZE, 
          FL\_LARGE\_SIZE, FL\_HUGE\_SIZE, FL\_DEFAULT\_SIZE

            {\it (type=int)}

          \item[txtstr]

          text to draw

            {\it (type=str)}

          \item[curscolr]

          color of the cursor

            {\it (type=int)}

          \item[pos]

          position which indicates the index of the character before which 
          to draw the cursor (-1 for not showing it)

            {\it (type=int)}

        \end{Ventry}

      \end{quote}

\textbf{Example:} fl\_drw\_text\_cursor(xfdata.FL\_ALIGN\_BOTTOM, 400, 175, 150, 45, 
xfdata.FL\_GREEN, xfdata.FL\_ITALIC\_STYLE, xfdata.FL\_SMALL\_SIZE, "A Good
Old String", xfdata.FL\_YELLOW, 7)



\textbf{Status:} Tested + Doc + NoDemo = OK



    \end{boxedminipage}

    \label{xformslib:flbasic:fl_drw_box}
    \index{xformslib \textit{(package)}!xformslib.flbasic \textit{(module)}!xformslib.flbasic.fl\_drw\_box \textit{(function)}}

    \vspace{0.5ex}

\hspace{.8\funcindent}\begin{boxedminipage}{\funcwidth}

    \raggedright \textbf{fl\_drw\_box}(\textit{boxtype}, \textit{x}, \textit{y}, \textit{w}, \textit{h}, \textit{colr}, \textit{bw})

    \vspace{-1.5ex}

    \rule{\textwidth}{0.5\fboxrule}
\setlength{\parskip}{2ex}
    Draws the bounding box of an object.

\setlength{\parskip}{1ex}
      \textbf{Parameters}
      \vspace{-1ex}

      \begin{quote}
        \begin{Ventry}{xxxxxxx}

          \item[boxtype]

          type of box to draw. Values (from xfdata module) FL\_NO\_BOX, 
          FL\_UP\_BOX, FL\_DOWN\_BOX, FL\_BORDER\_BOX, FL\_SHADOW\_BOX, 
          FL\_FRAME\_BOX, FL\_ROUNDED\_BOX, FL\_EMBOSSED\_BOX, 
          FL\_FLAT\_BOX, FL\_RFLAT\_BOX, FL\_RSHADOW\_BOX, FL\_OVAL\_BOX, 
          FL\_ROUNDED3D\_UPBOX, FL\_ROUNDED3D\_DOWNBOX, FL\_OVAL3D\_UPBOX, 
          FL\_OVAL3D\_DOWNBOX, FL\_OVAL3D\_FRAMEBOX, 
          FL\_OVAL3D\_EMBOSSEDBOX

            {\it (type=int)}

          \item[x]

          horizontal position (upper-left corner)

            {\it (type=int)}

          \item[y]

          vertical position (upper-left corner)

            {\it (type=int)}

          \item[w]

          width in coord units

            {\it (type=int)}

          \item[h]

          height in coord units

            {\it (type=int)}

          \item[colr]

          color value

            {\it (type=long\_pos)}

          \item[bw]

          width of the boundary

            {\it (type=int)}

        \end{Ventry}

      \end{quote}

\textbf{Example:} fl\_drw\_box(xfdata.FL\_DOWN\_BOX, 700, 800, 600, 450, xfdata.FL\_INDIGO, 
3)



\textbf{Status:} Tested + Doc + NoDemo = OK



    \end{boxedminipage}

    \label{xformslib:flbasic:fl_add_symbol}
    \index{xformslib \textit{(package)}!xformslib.flbasic \textit{(module)}!xformslib.flbasic.fl\_add\_symbol \textit{(function)}}

    \vspace{0.5ex}

\hspace{.8\funcindent}\begin{boxedminipage}{\funcwidth}

    \raggedright \textbf{fl\_add\_symbol}(\textit{symbname}, \textit{py\_DrawPtr}, \textit{scalable})

    \vspace{-1.5ex}

    \rule{\textwidth}{0.5\fboxrule}
\setlength{\parskip}{2ex}
    Adds a customly drawn symbol to the system which it can then use to 
    display symbols on objects that are not provided by the library.

\setlength{\parskip}{1ex}
      \textbf{Parameters}
      \vspace{-1ex}

      \begin{quote}
        \begin{Ventry}{xxxxxxxxxx}

          \item[symbname]

          name under which the symbol should be known (at most 15 
          characters), without the leading @

            {\it (type=str)}

          \item[py\_DrawPtr]

          python function to draw symbol, no return

            {\it (type=\_\_ funcname (coord, coord, coord, coord, angle\_degree\_rotation, colr) 
\_\_)}

          \item[scalable]

          not used, a value of 0 will be fine

            {\it (type=int)}

        \end{Ventry}

      \end{quote}

      \textbf{Return Value}
    \vspace{-1ex}

      \begin{quote}
      num.

      {\it (type=int)}

      \end{quote}

\textbf{Example:}
\begin{quote}
  \begin{itemize}

  \item
    \setlength{\parskip}{0.6ex}
def drawsymb(x, y, w, h, angle, col):



  \item {\textbar}-{\textgreater}{\textbar} ...



  \item xf.fl\_add\_symbol("MySymbol", drawsymb, 0)



\end{itemize}

\end{quote}

\textbf{Status:} Tested + Doc + NoDemo = OK



    \end{boxedminipage}

    \label{xformslib:flbasic:fl_draw_symbol}
    \index{xformslib \textit{(package)}!xformslib.flbasic \textit{(module)}!xformslib.flbasic.fl\_draw\_symbol \textit{(function)}}

    \vspace{0.5ex}

\hspace{.8\funcindent}\begin{boxedminipage}{\funcwidth}

    \raggedright \textbf{fl\_draw\_symbol}(\textit{symbname}, \textit{x}, \textit{y}, \textit{w}, \textit{h}, \textit{colr})

    \vspace{-1.5ex}

    \rule{\textwidth}{0.5\fboxrule}
\setlength{\parskip}{2ex}
    Draws directly a symbol on the screen.

\setlength{\parskip}{1ex}
      \textbf{Parameters}
      \vspace{-1ex}

      \begin{quote}
        \begin{Ventry}{xxxxxxxx}

          \item[symbname]

          name given to the symbol

            {\it (type=str)}

          \item[x]

          horizontal position (upper-left corner)

            {\it (type=int)}

          \item[y]

          vertical position (upper-left corner)

            {\it (type=int)}

          \item[w]

          width in coord units

            {\it (type=int)}

          \item[h]

          height in coord units

            {\it (type=int)}

          \item[colr]

          color value

            {\it (type=long\_pos)}

        \end{Ventry}

      \end{quote}

      \textbf{Return Value}
    \vspace{-1ex}

      \begin{quote}
      1 (on success) or 0 (on failure)

      {\it (type=int)}

      \end{quote}

\textbf{Example:} fl\_draw\_symbol("willsym", 120, 120, 15, 20, xfdata.FL\_LIGHTGRAY)



\textbf{Status:} Tested + Doc + NoDemo = OK



    \end{boxedminipage}

    \label{xformslib:flbasic:fl_mapcolor}
    \index{xformslib \textit{(package)}!xformslib.flbasic \textit{(module)}!xformslib.flbasic.fl\_mapcolor \textit{(function)}}

    \vspace{0.5ex}

\hspace{.8\funcindent}\begin{boxedminipage}{\funcwidth}

    \raggedright \textbf{fl\_mapcolor}(\textit{colr}, \textit{r}, \textit{g}, \textit{b})

    \vspace{-1.5ex}

    \rule{\textwidth}{0.5\fboxrule}
\setlength{\parskip}{2ex}
    Changes the colormap and make a color index active so that it can be 
    used in various drawing routines after initialization. It maps a new 
    color using specific values for red, green and blue. In case a request 
    fails, we substitute the closest color. It is recommended that you use 
    an index larger than xfdata.FL\_FREE\_COL1 for your remap request to 
    avoid accidentally free the colors you have not explicitly allocated. 
    Indices larger than 224 are reserved and should not be used.

\setlength{\parskip}{1ex}
      \textbf{Parameters}
      \vspace{-1ex}

      \begin{quote}
        \begin{Ventry}{xxxx}

          \item[colr]

          new color value to be mapped

            {\it (type=long\_pos)}

          \item[r]

          value for red

            {\it (type=int)}

          \item[g]

          value for green

            {\it (type=int)}

          \item[b]

          value for blue

            {\it (type=int)}

        \end{Ventry}

      \end{quote}

      \textbf{Return Value}
    \vspace{-1ex}

      \begin{quote}
      color value, or 0 (on failure)

      {\it (type=long\_pos)}

      \end{quote}

\textbf{Example:} fl\_mapcolor(xfdata.FL\_FREE\_COL1, 100, 200, 300)



\textbf{Status:} Tested + Doc + Demo = OK



    \end{boxedminipage}

    \label{xformslib:flbasic:fl_mapcolorname}
    \index{xformslib \textit{(package)}!xformslib.flbasic \textit{(module)}!xformslib.flbasic.fl\_mapcolorname \textit{(function)}}

    \vspace{0.5ex}

\hspace{.8\funcindent}\begin{boxedminipage}{\funcwidth}

    \raggedright \textbf{fl\_mapcolorname}(\textit{colr}, \textit{rgbcolrname})

    \vspace{-1.5ex}

    \rule{\textwidth}{0.5\fboxrule}
\setlength{\parskip}{2ex}
    Sets the color in the colormap indexed by colr to the specified color 
    name. It associates an index with a color name, which may have been 
    obtained via resources.

\setlength{\parskip}{1ex}
      \textbf{Parameters}
      \vspace{-1ex}

      \begin{quote}
        \begin{Ventry}{xxxxxxxxxxx}

          \item[colr]

          color value to be mapped

            {\it (type=long\_pos)}

          \item[rgbcolrname]

          name of mapped color from the systems color database file 
          "rgb.txt" (see that file for possible values)

            {\it (type=str)}

        \end{Ventry}

      \end{quote}

      \textbf{Return Value}
    \vspace{-1ex}

      \begin{quote}
      color pixel value, or -1 (on failure)

      {\it (type=long)}

      \end{quote}

\textbf{Example:} fl\_mapcolorname(xfdata.FL\_FREE\_COL3, "PowderBlue")



\textbf{Status:} Tested + Doc + NoDemo = OK



    \end{boxedminipage}

    \label{xformslib:flbasic:fl_mapcolorname}
    \index{xformslib \textit{(package)}!xformslib.flbasic \textit{(module)}!xformslib.flbasic.fl\_mapcolorname \textit{(function)}}

    \vspace{0.5ex}

\hspace{.8\funcindent}\begin{boxedminipage}{\funcwidth}

    \raggedright \textbf{fl\_mapcolor\_name}(\textit{colr}, \textit{rgbcolrname})

    \vspace{-1.5ex}

    \rule{\textwidth}{0.5\fboxrule}
\setlength{\parskip}{2ex}
    Sets the color in the colormap indexed by colr to the specified color 
    name. It associates an index with a color name, which may have been 
    obtained via resources.

\setlength{\parskip}{1ex}
      \textbf{Parameters}
      \vspace{-1ex}

      \begin{quote}
        \begin{Ventry}{xxxxxxxxxxx}

          \item[colr]

          color value to be mapped

            {\it (type=long\_pos)}

          \item[rgbcolrname]

          name of mapped color from the systems color database file 
          "rgb.txt" (see that file for possible values)

            {\it (type=str)}

        \end{Ventry}

      \end{quote}

      \textbf{Return Value}
    \vspace{-1ex}

      \begin{quote}
      color pixel value, or -1 (on failure)

      {\it (type=long)}

      \end{quote}

\textbf{Example:} fl\_mapcolorname(xfdata.FL\_FREE\_COL3, "PowderBlue")



\textbf{Status:} Tested + Doc + NoDemo = OK



    \end{boxedminipage}

    \label{xformslib:flbasic:fl_set_color_leak}
    \index{xformslib \textit{(package)}!xformslib.flbasic \textit{(module)}!xformslib.flbasic.fl\_set\_color\_leak \textit{(function)}}

    \vspace{0.5ex}

\hspace{.8\funcindent}\begin{boxedminipage}{\funcwidth}

    \raggedright \textbf{fl\_set\_color\_leak}(\textit{flag})

    \vspace{-1.5ex}

    \rule{\textwidth}{0.5\fboxrule}
\setlength{\parskip}{2ex}
    Enables or disables the leakage of color. ?

\setlength{\parskip}{1ex}
      \textbf{Parameters}
      \vspace{-1ex}

      \begin{quote}
        \begin{Ventry}{xxxx}

          \item[flag]

          flag to enable/disable leakage of color. Values 0 (to disable) or
          1 (to enable)

            {\it (type=int)}

        \end{Ventry}

      \end{quote}

\textbf{Example:} fl\_set\_color\_leak(1)



\textbf{Status:} Tested + Doc + NoDemo = OK



    \end{boxedminipage}

    \label{xformslib:flbasic:fl_getmcolor}
    \index{xformslib \textit{(package)}!xformslib.flbasic \textit{(module)}!xformslib.flbasic.fl\_getmcolor \textit{(function)}}

    \vspace{0.5ex}

\hspace{.8\funcindent}\begin{boxedminipage}{\funcwidth}

    \raggedright \textbf{fl\_getmcolor}(\textit{colr})

    \vspace{-1.5ex}

    \rule{\textwidth}{0.5\fboxrule}
\setlength{\parskip}{2ex}
    Obtains the RGB values of an index, returning the pixel value as known 
    by the X server. If you are interested in the internal colormap of 
    XForms fl\_get\_icm\_color() is more efficient.

\setlength{\parskip}{1ex}
      \textbf{Parameters}
      \vspace{-1ex}

      \begin{quote}
        \begin{Ventry}{xxxx}

          \item[colr]

          color value

            {\it (type=long\_pos)}

        \end{Ventry}

      \end{quote}

      \textbf{Return Value}
    \vspace{-1ex}

      \begin{quote}
      color pixel, red (r), green (r), blue (b)

      {\it (type=long\_pos, int, int, int)}

      \end{quote}

\textbf{Example:} pixl, red, green, blue = fl\_getmcolor(xfdata.FL\_VIOLET)



\textbf{Attention:} API change from XForms - upstream was fl\_getmcolor(colr, r, g, b)



\textbf{Status:} Tested + Doc + NoDemo = OK



    \end{boxedminipage}

    \label{xformslib:flbasic:fl_get_pixel}
    \index{xformslib \textit{(package)}!xformslib.flbasic \textit{(module)}!xformslib.flbasic.fl\_get\_pixel \textit{(function)}}

    \vspace{0.5ex}

\hspace{.8\funcindent}\begin{boxedminipage}{\funcwidth}

    \raggedright \textbf{fl\_get\_pixel}(\textit{colr})

    \vspace{-1.5ex}

    \rule{\textwidth}{0.5\fboxrule}
\setlength{\parskip}{2ex}
    Obtains the actual pixel value the X server understands. XForms library
    keeps an internal colormap, initialized to predefined colors. The 
    predefined colors do not correspond to pixel values the server 
    understands but are indexes into the colormap. Therefore, they can't be
    used in any of the Graphics Context (GC) altering or Xlib routines.

\setlength{\parskip}{1ex}
      \textbf{Parameters}
      \vspace{-1ex}

      \begin{quote}
        \begin{Ventry}{xxxx}

          \item[colr]

          color value

            {\it (type=long\_pos)}

        \end{Ventry}

      \end{quote}

      \textbf{Return Value}
    \vspace{-1ex}

      \begin{quote}
      color pixel

      {\it (type=long\_pos)}

      \end{quote}

\textbf{Example:} pixl = fl\_get\_pixel(xfdata.FL\_PEACHPUFF)



\textbf{Status:} Tested + Doc + Demo = OK



    \end{boxedminipage}

    \label{xformslib:flbasic:fl_get_pixel}
    \index{xformslib \textit{(package)}!xformslib.flbasic \textit{(module)}!xformslib.flbasic.fl\_get\_pixel \textit{(function)}}

    \vspace{0.5ex}

\hspace{.8\funcindent}\begin{boxedminipage}{\funcwidth}

    \raggedright \textbf{fl\_get\_flcolor}(\textit{colr})

    \vspace{-1.5ex}

    \rule{\textwidth}{0.5\fboxrule}
\setlength{\parskip}{2ex}
    Obtains the actual pixel value the X server understands. XForms library
    keeps an internal colormap, initialized to predefined colors. The 
    predefined colors do not correspond to pixel values the server 
    understands but are indexes into the colormap. Therefore, they can't be
    used in any of the Graphics Context (GC) altering or Xlib routines.

\setlength{\parskip}{1ex}
      \textbf{Parameters}
      \vspace{-1ex}

      \begin{quote}
        \begin{Ventry}{xxxx}

          \item[colr]

          color value

            {\it (type=long\_pos)}

        \end{Ventry}

      \end{quote}

      \textbf{Return Value}
    \vspace{-1ex}

      \begin{quote}
      color pixel

      {\it (type=long\_pos)}

      \end{quote}

\textbf{Example:} pixl = fl\_get\_pixel(xfdata.FL\_PEACHPUFF)



\textbf{Status:} Tested + Doc + Demo = OK



    \end{boxedminipage}

    \label{xformslib:flbasic:fl_get_icm_color}
    \index{xformslib \textit{(package)}!xformslib.flbasic \textit{(module)}!xformslib.flbasic.fl\_get\_icm\_color \textit{(function)}}

    \vspace{0.5ex}

\hspace{.8\funcindent}\begin{boxedminipage}{\funcwidth}

    \raggedright \textbf{fl\_get\_icm\_color}(\textit{colr})

    \vspace{-1.5ex}

    \rule{\textwidth}{0.5\fboxrule}
\setlength{\parskip}{2ex}
    Queries the internal colormap handled by XForms, returning red, green 
    and blue values corresponding to color index. Note that it does not 
    communicate with the X server, it only return information about the 
    internal colormap, which is made known to the X server by the 
    initialization routine fl\_initialize().

\setlength{\parskip}{1ex}
      \textbf{Parameters}
      \vspace{-1ex}

      \begin{quote}
        \begin{Ventry}{xxxx}

          \item[colr]

          color value

            {\it (type=long\_pos)}

        \end{Ventry}

      \end{quote}

      \textbf{Return Value}
    \vspace{-1ex}

      \begin{quote}
      red (r), green (g), blue (b)

      {\it (type=int, int, int)}

      \end{quote}

\textbf{Example:} red, green, blue = fl\_get\_icm\_color(xfdata.FL\_OLIVE)



\textbf{Attention:} API change from XForms - upstream was fl\_get\_icm\_color(colr, r, g, b)



\textbf{Status:} Tested + Doc + NoDemo = OK



    \end{boxedminipage}

    \label{xformslib:flbasic:fl_set_icm_color}
    \index{xformslib \textit{(package)}!xformslib.flbasic \textit{(module)}!xformslib.flbasic.fl\_set\_icm\_color \textit{(function)}}

    \vspace{0.5ex}

\hspace{.8\funcindent}\begin{boxedminipage}{\funcwidth}

    \raggedright \textbf{fl\_set\_icm\_color}(\textit{colr}, \textit{r}, \textit{g}, \textit{b})

    \vspace{-1.5ex}

    \rule{\textwidth}{0.5\fboxrule}
\setlength{\parskip}{2ex}
    Changes the internal colormap handled by XForms, setting a color index 
    using a red, green and blue values' combination. You have to call 
    fl\_set\_icm\_color() before fl\_initialize() to change XForms's 
    default colormap. Note that it does not communicate with the X server, 
    it only populate the internal colormap, which is made known to the X 
    server by the initialization routine fl\_initialize().

\setlength{\parskip}{1ex}
      \textbf{Parameters}
      \vspace{-1ex}

      \begin{quote}
        \begin{Ventry}{xxxx}

          \item[colr]

          color value

            {\it (type=long\_pos)}

          \item[r]

          value for red

            {\it (type=int)}

          \item[g]

          value for green

            {\it (type=int)}

          \item[b]

          value for blue

            {\it (type=int)}

        \end{Ventry}

      \end{quote}

\textbf{Example:} fl\_set\_icm\_color(xfdata.FL\_FREE\_COL8, 75, 150, 225)



\textbf{Status:} Tested + Doc + NoDemo = OK



    \end{boxedminipage}

    \label{xformslib:flbasic:fl_color}
    \index{xformslib \textit{(package)}!xformslib.flbasic \textit{(module)}!xformslib.flbasic.fl\_color \textit{(function)}}

    \vspace{0.5ex}

\hspace{.8\funcindent}\begin{boxedminipage}{\funcwidth}

    \raggedright \textbf{fl\_color}(\textit{colr})

    \vspace{-1.5ex}

    \rule{\textwidth}{0.5\fboxrule}
\setlength{\parskip}{2ex}
    Sets the foreground color in the XForms library's default Graphics 
    Context (gc[0]).

\setlength{\parskip}{1ex}
      \textbf{Parameters}
      \vspace{-1ex}

      \begin{quote}
        \begin{Ventry}{xxxx}

          \item[colr]

          color value

            {\it (type=long\_pos)}

        \end{Ventry}

      \end{quote}

\textbf{Example:} fl\_color(xfdata.FL\_ORANGE)



\textbf{Status:} Tested + Doc + NoDemo = OK



    \end{boxedminipage}

    \label{xformslib:flbasic:fl_bk_color}
    \index{xformslib \textit{(package)}!xformslib.flbasic \textit{(module)}!xformslib.flbasic.fl\_bk\_color \textit{(function)}}

    \vspace{0.5ex}

\hspace{.8\funcindent}\begin{boxedminipage}{\funcwidth}

    \raggedright \textbf{fl\_bk\_color}(\textit{colr})

    \vspace{-1.5ex}

    \rule{\textwidth}{0.5\fboxrule}
\setlength{\parskip}{2ex}
    Sets the background color in the default Graphics Context (gc[0]).

\setlength{\parskip}{1ex}
      \textbf{Parameters}
      \vspace{-1ex}

      \begin{quote}
        \begin{Ventry}{xxxx}

          \item[colr]

          color value

            {\it (type=long\_pos)}

        \end{Ventry}

      \end{quote}

\textbf{Example:} fl\_bk\_color(xfdata.FL\_MEDIUMORCHID)



\textbf{Status:} Tested + Doc + NoDemo = OK



    \end{boxedminipage}

    \label{xformslib:flbasic:fl_textcolor}
    \index{xformslib \textit{(package)}!xformslib.flbasic \textit{(module)}!xformslib.flbasic.fl\_textcolor \textit{(function)}}

    \vspace{0.5ex}

\hspace{.8\funcindent}\begin{boxedminipage}{\funcwidth}

    \raggedright \textbf{fl\_textcolor}(\textit{colr})

    \vspace{-1.5ex}

    \rule{\textwidth}{0.5\fboxrule}
\setlength{\parskip}{2ex}
    Sets the foreground color for text in the default Graphics Context 
    (gc[0]).

\setlength{\parskip}{1ex}
      \textbf{Parameters}
      \vspace{-1ex}

      \begin{quote}
        \begin{Ventry}{xxxx}

          \item[colr]

          color value

            {\it (type=long\_pos)}

        \end{Ventry}

      \end{quote}

\textbf{Example:} fl\_textcolor(xfdata.FL\_LIGHTCORAL)



\textbf{Status:} Tested + Doc + NoDemo = OK



    \end{boxedminipage}

    \label{xformslib:flbasic:fl_bk_textcolor}
    \index{xformslib \textit{(package)}!xformslib.flbasic \textit{(module)}!xformslib.flbasic.fl\_bk\_textcolor \textit{(function)}}

    \vspace{0.5ex}

\hspace{.8\funcindent}\begin{boxedminipage}{\funcwidth}

    \raggedright \textbf{fl\_bk\_textcolor}(\textit{colr})

    \vspace{-1.5ex}

    \rule{\textwidth}{0.5\fboxrule}
\setlength{\parskip}{2ex}
    Sets the background color for text in the default Graphics Context 
    (gc[0]).

\setlength{\parskip}{1ex}
      \textbf{Parameters}
      \vspace{-1ex}

      \begin{quote}
        \begin{Ventry}{xxxx}

          \item[colr]

          color value

            {\it (type=long\_pos)}

        \end{Ventry}

      \end{quote}

\textbf{Example:} fl\_bk\_textcolor(xfdata.FL\_IVORY)



\textbf{Status:} Tested + Doc + NoDemo = OK



    \end{boxedminipage}

    \label{xformslib:flbasic:fl_set_gamma}
    \index{xformslib \textit{(package)}!xformslib.flbasic \textit{(module)}!xformslib.flbasic.fl\_set\_gamma \textit{(function)}}

    \vspace{0.5ex}

\hspace{.8\funcindent}\begin{boxedminipage}{\funcwidth}

    \raggedright \textbf{fl\_set\_gamma}(\textit{r}, \textit{g}, \textit{b})

    \vspace{-1.5ex}

    \rule{\textwidth}{0.5\fboxrule}
\setlength{\parskip}{2ex}
    Adjusts the brightness of the builtin colors. Larger the value, 
    brighter the colors. The default gamma is 1.

\setlength{\parskip}{1ex}
      \textbf{Parameters}
      \vspace{-1ex}

      \begin{quote}
        \begin{Ventry}{x}

          \item[r]

          gamma value for red

            {\it (type=float)}

          \item[g]

          gamma value for green

            {\it (type=float)}

          \item[b]

          gamma value for blue

            {\it (type=float)}

        \end{Ventry}

      \end{quote}

\textbf{Example:} fl\_set\_gamma(2.0, 2.0, 2.0)



\textbf{Precondition:} to be called before fl\_initialize()



\textbf{Status:} Tested + Doc + NoDemo = OK



    \end{boxedminipage}

    \label{xformslib:flbasic:FL_max}
    \index{xformslib \textit{(package)}!xformslib.flbasic \textit{(module)}!xformslib.flbasic.FL\_max \textit{(function)}}

    \vspace{0.5ex}

\hspace{.8\funcindent}\begin{boxedminipage}{\funcwidth}

    \raggedright \textbf{FL\_max}(\textit{a}, \textit{b})

\setlength{\parskip}{2ex}
\setlength{\parskip}{1ex}
    \end{boxedminipage}

    \label{xformslib:flbasic:FL_min}
    \index{xformslib \textit{(package)}!xformslib.flbasic \textit{(module)}!xformslib.flbasic.FL\_min \textit{(function)}}

    \vspace{0.5ex}

\hspace{.8\funcindent}\begin{boxedminipage}{\funcwidth}

    \raggedright \textbf{FL\_min}(\textit{a}, \textit{b})

\setlength{\parskip}{2ex}
\setlength{\parskip}{1ex}
    \end{boxedminipage}

    \label{xformslib:flbasic:FL_abs}
    \index{xformslib \textit{(package)}!xformslib.flbasic \textit{(module)}!xformslib.flbasic.FL\_abs \textit{(function)}}

    \vspace{0.5ex}

\hspace{.8\funcindent}\begin{boxedminipage}{\funcwidth}

    \raggedright \textbf{FL\_abs}(\textit{a})

\setlength{\parskip}{2ex}
\setlength{\parskip}{1ex}
    \end{boxedminipage}

    \label{xformslib:flbasic:FL_nint}
    \index{xformslib \textit{(package)}!xformslib.flbasic \textit{(module)}!xformslib.flbasic.FL\_nint \textit{(function)}}

    \vspace{0.5ex}

\hspace{.8\funcindent}\begin{boxedminipage}{\funcwidth}

    \raggedright \textbf{FL\_nint}(\textit{a})

\setlength{\parskip}{2ex}
\setlength{\parskip}{1ex}
    \end{boxedminipage}

    \label{xformslib:flbasic:FL_clamp}
    \index{xformslib \textit{(package)}!xformslib.flbasic \textit{(module)}!xformslib.flbasic.FL\_clamp \textit{(function)}}

    \vspace{0.5ex}

\hspace{.8\funcindent}\begin{boxedminipage}{\funcwidth}

    \raggedright \textbf{FL\_clamp}(\textit{a}, \textit{amin}, \textit{amax})

\setlength{\parskip}{2ex}
\setlength{\parskip}{1ex}
    \end{boxedminipage}

    \label{xformslib:flbasic:FL_crnd}
    \index{xformslib \textit{(package)}!xformslib.flbasic \textit{(module)}!xformslib.flbasic.FL\_crnd \textit{(function)}}

    \vspace{0.5ex}

\hspace{.8\funcindent}\begin{boxedminipage}{\funcwidth}

    \raggedright \textbf{FL\_crnd}(\textit{a})

\setlength{\parskip}{2ex}
\setlength{\parskip}{1ex}
    \end{boxedminipage}

    \label{xformslib:flbasic:fl_add_object}
    \index{xformslib \textit{(package)}!xformslib.flbasic \textit{(module)}!xformslib.flbasic.fl\_add\_object \textit{(function)}}

    \vspace{0.5ex}

\hspace{.8\funcindent}\begin{boxedminipage}{\funcwidth}

    \raggedright \textbf{fl\_add\_object}(\textit{pFlForm}, \textit{pFlObject})

    \vspace{-1.5ex}

    \rule{\textwidth}{0.5\fboxrule}
\setlength{\parskip}{2ex}
    The object remains available (except if it's an object that marks the 
    start or end of a group) and can be added again to the same or another 
    form later. Normally, this function is used in object classes to add a 
    newly created object to a form. It may not be used for objects 
    representing the start or end of a group.

\setlength{\parskip}{1ex}
      \textbf{Parameters}
      \vspace{-1ex}

      \begin{quote}
        \begin{Ventry}{xxxxxxxxx}

          \item[pFlForm]

          form which an object will be added to

            {\it (type=pointer to xfdata.FL\_FORM)}

          \item[pFlObject]

          object to be added

            {\it (type=pointer to xfdata.FL\_OBJECT)}

        \end{Ventry}

      \end{quote}

\textbf{Example:} fl\_add\_object(pform2, pobjnew2)



\textbf{Status:} Tested + Doc + NoDemo = OK



    \end{boxedminipage}

    \label{xformslib:flbasic:fl_addto_form}
    \index{xformslib \textit{(package)}!xformslib.flbasic \textit{(module)}!xformslib.flbasic.fl\_addto\_form \textit{(function)}}

    \vspace{0.5ex}

\hspace{.8\funcindent}\begin{boxedminipage}{\funcwidth}

    \raggedright \textbf{fl\_addto\_form}(\textit{pFlForm})

    \vspace{-1.5ex}

    \rule{\textwidth}{0.5\fboxrule}
\setlength{\parskip}{2ex}
    Reopens a form (after fl\_end\_form) for adding further objects to it.

\setlength{\parskip}{1ex}
      \textbf{Parameters}
      \vspace{-1ex}

      \begin{quote}
        \begin{Ventry}{xxxxxxx}

          \item[pFlForm]

          form

            {\it (type=pointer to xfdata.FL\_FORM)}

        \end{Ventry}

      \end{quote}

      \textbf{Return Value}
    \vspace{-1ex}

      \begin{quote}
      form (pFlForm), or None (on failure)

      {\it (type=pointer to xfdata.FL\_FORM)}

      \end{quote}

\textbf{Example:} form = fl\_addto\_form(closedform)



\textbf{Status:} Tested + Doc + NoDemo = OK



    \end{boxedminipage}

    \label{xformslib:flbasic:fl_make_object}
    \index{xformslib \textit{(package)}!xformslib.flbasic \textit{(module)}!xformslib.flbasic.fl\_make\_object \textit{(function)}}

    \vspace{0.5ex}

\hspace{.8\funcindent}\begin{boxedminipage}{\funcwidth}

    \raggedright \textbf{fl\_make\_object}(\textit{objclass}, \textit{objtype}, \textit{x}, \textit{y}, \textit{w}, \textit{h}, \textit{label}, \textit{py\_HandlePtr})

    \vspace{-1.5ex}

    \rule{\textwidth}{0.5\fboxrule}
\setlength{\parskip}{2ex}
    Makes a custom object.

\setlength{\parskip}{1ex}
      \textbf{Parameters}
      \vspace{-1ex}

      \begin{quote}
        \begin{Ventry}{xxxxxxxxxxxx}

          \item[objclass]

          class type of object to be made

            {\it (type=int)}

          \item[objtype]

          type of the object to be made

            {\it (type=int)}

          \item[x]

          horizontal position of object (upper-left corner)

            {\it (type=int)}

          \item[y]

          vertical position of object (upper-left corner)

            {\it (type=int)}

          \item[w]

          width in coord units

            {\it (type=int)}

          \item[h]

          height coord units

            {\it (type=int)}

          \item[label]

          text label of object

            {\it (type=str)}

          \item[py\_HandlePtr]

          python function for handling object, returning value

            {\it (type=\_\_ funcname (pFlObject, num, coord, coord, num, ptr\_void) 
-{\textgreater} num \_\_)}

        \end{Ventry}

      \end{quote}

      \textbf{Return Value}
    \vspace{-1ex}

      \begin{quote}
      object made (pFlObject)

      {\it (type=pointer to xfdata.FL\_OBJECT)}

      \end{quote}

\textbf{Example:}
\begin{quote}
  \begin{itemize}

  \item
    \setlength{\parskip}{0.6ex}
def handlecb(pobj, num, w, h, num, vdata):



  \item {\textbar}-{\textgreater}{\textbar} ...



  \item {\textbar}-{\textgreater}{\textbar} return 0



  \item fl\_make\_object(...)



\end{itemize}

\end{quote}

\textbf{Status:} Untested + Doc + NoDemo = NOT OK



    \end{boxedminipage}

    \label{xformslib:flbasic:fl_add_child}
    \index{xformslib \textit{(package)}!xformslib.flbasic \textit{(module)}!xformslib.flbasic.fl\_add\_child \textit{(function)}}

    \vspace{0.5ex}

\hspace{.8\funcindent}\begin{boxedminipage}{\funcwidth}

    \raggedright \textbf{fl\_add\_child}(\textit{pFlObject1}, \textit{pFlObject2})

    \vspace{-1.5ex}

    \rule{\textwidth}{0.5\fboxrule}
\setlength{\parskip}{2ex}
    Makes an object a child of another object. An example is the scrollbar 
    object. It has three child objects, a slider and two buttons, which all
    three are childs of the scrollbar object.

\setlength{\parskip}{1ex}
      \textbf{Parameters}
      \vspace{-1ex}

      \begin{quote}
        \begin{Ventry}{xxxxxxxxxx}

          \item[pFlObject1]

          father object

            {\it (type=pointer to xfdata.FL\_OBJECT)}

          \item[pFlObject2]

          child object to add

            {\it (type=pointer to xfdata.FL\_OBJECT)}

        \end{Ventry}

      \end{quote}

\textbf{Example:} fl\_add\_child(pobjf, pobjs)



\textbf{Status:} Tested + Doc + NoDemo = OK



    \end{boxedminipage}

    \label{xformslib:flbasic:fl_set_coordunit}
    \index{xformslib \textit{(package)}!xformslib.flbasic \textit{(module)}!xformslib.flbasic.fl\_set\_coordunit \textit{(function)}}

    \vspace{0.5ex}

\hspace{.8\funcindent}\begin{boxedminipage}{\funcwidth}

    \raggedright \textbf{fl\_set\_coordunit}(\textit{unit})

    \vspace{-1.5ex}

    \rule{\textwidth}{0.5\fboxrule}
\setlength{\parskip}{2ex}
    Sets the unit for screen coordinates, instead of default ones (pixels).

\setlength{\parskip}{1ex}
      \textbf{Parameters}
      \vspace{-1ex}

      \begin{quote}
        \begin{Ventry}{xxxx}

          \item[unit]

          coord unit type to set. Values (from xfdata module) 
          FL\_COORD\_PIXEL, FL\_COORD\_MM, FL\_COORD\_POINT, 
          FL\_COORD\_centiMM, FL\_COORD\_centiPOINT

        \end{Ventry}

      \end{quote}

\textbf{Example:} fl\_set\_coordunit(xfdata.FL\_COORD\_MM)



\textbf{Status:} Tested + Doc + Demo = OK



    \end{boxedminipage}

    \label{xformslib:flbasic:fl_set_border_width}
    \index{xformslib \textit{(package)}!xformslib.flbasic \textit{(module)}!xformslib.flbasic.fl\_set\_border\_width \textit{(function)}}

    \vspace{0.5ex}

\hspace{.8\funcindent}\begin{boxedminipage}{\funcwidth}

    \raggedright \textbf{fl\_set\_border\_width}(\textit{bw})

    \vspace{-1.5ex}

    \rule{\textwidth}{0.5\fboxrule}
\setlength{\parskip}{2ex}
    Sets the width of the border. If set to a negative number, all objects 
    appear to have a softer appearance.

\setlength{\parskip}{1ex}
      \textbf{Parameters}
      \vspace{-1ex}

      \begin{quote}
        \begin{Ventry}{xx}

          \item[bw]

          value of border width

            {\it (type=int)}

        \end{Ventry}

      \end{quote}

\textbf{Example:} fl\_set\_border\_width(-3)



\textbf{Status:} Tested + Doc + Demo = OK



    \end{boxedminipage}

    \label{xformslib:flbasic:fl_set_scrollbar_type}
    \index{xformslib \textit{(package)}!xformslib.flbasic \textit{(module)}!xformslib.flbasic.fl\_set\_scrollbar\_type \textit{(function)}}

    \vspace{0.5ex}

\hspace{.8\funcindent}\begin{boxedminipage}{\funcwidth}

    \raggedright \textbf{fl\_set\_scrollbar\_type}(\textit{sbtype})

    \vspace{-1.5ex}

    \rule{\textwidth}{0.5\fboxrule}
\setlength{\parskip}{2ex}
    Sets the type of a scrollbar.

\setlength{\parskip}{1ex}
      \textbf{Parameters}
      \vspace{-1ex}

      \begin{quote}
        \begin{Ventry}{xxxxxx}

          \item[sbtype]

          type of scrollbar\_var. Values (from xfdata module) 
          FL\_VERT\_SCROLLBAR, FL\_HOR\_SCROLLBAR, 
          FL\_VERT\_THIN\_SCROLLBAR, FL\_HOR\_THIN\_SCROLLBAR, 
          FL\_VERT\_NICE\_SCROLLBAR, FL\_HOR\_NICE\_SCROLLBAR, 
          FL\_VERT\_PLAIN\_SCROLLBAR, FL\_HOR\_PLAIN\_SCROLLBAR, 
          FL\_HOR\_BASIC\_SCROLLBAR, FL\_VERT\_BASIC\_SCROLLBAR, 
          FL\_NORMAL\_SCROLLBAR, FL\_THIN\_SCROLLBAR, FL\_NICE\_SCROLLBAR, 
          FL\_PLAIN\_SCROLLBAR

            {\it (type=int)}

        \end{Ventry}

      \end{quote}

\textbf{Example:} fl\_set\_scrollbar\_type(xfdata.FL\_VERT\_BASIC\_SCROLLBAR)



\textbf{Status:} Tested + Doc + NoDemo = OK



    \end{boxedminipage}

    \label{xformslib:flbasic:fl_set_thinscrollbar}
    \index{xformslib \textit{(package)}!xformslib.flbasic \textit{(module)}!xformslib.flbasic.fl\_set\_thinscrollbar \textit{(function)}}

    \vspace{0.5ex}

\hspace{.8\funcindent}\begin{boxedminipage}{\funcwidth}

    \raggedright \textbf{fl\_set\_thinscrollbar}(\textit{flag})

    \vspace{-1.5ex}

    \rule{\textwidth}{0.5\fboxrule}
\setlength{\parskip}{2ex}
    Sets if scrollbar type is thin or normal.

\setlength{\parskip}{1ex}
      \textbf{Parameters}
      \vspace{-1ex}

      \begin{quote}
        \begin{Ventry}{xxxx}

          \item[flag]

          flag if thin scrollbar or not. Values 1 (for thin) or 0 (for 
          normal)

            {\it (type=int)}

        \end{Ventry}

      \end{quote}

\textbf{Example:} fl\_set\_thinscrollbar(1)



\textbf{Status:} Tested + Doc + NoDemo = OK



    \end{boxedminipage}

    \label{xformslib:flbasic:fl_flip_yorigin}
    \index{xformslib \textit{(package)}!xformslib.flbasic \textit{(module)}!xformslib.flbasic.fl\_flip\_yorigin \textit{(function)}}

    \vspace{0.5ex}

\hspace{.8\funcindent}\begin{boxedminipage}{\funcwidth}

    \raggedright \textbf{fl\_flip\_yorigin}()

    \vspace{-1.5ex}

    \rule{\textwidth}{0.5\fboxrule}
\setlength{\parskip}{2ex}
    Sets the origin of XForms coordinates at the lower-left corner of the 
    form (instead of default upper-left corner).

\setlength{\parskip}{1ex}
\textbf{Example:} fl\_flip\_yorigin()



\textbf{Precondition:} to be called before fl\_initialize()



\textbf{Status:} Tested + Doc + Demo = OK



    \end{boxedminipage}

    \label{xformslib:flbasic:fl_get_coordunit}
    \index{xformslib \textit{(package)}!xformslib.flbasic \textit{(module)}!xformslib.flbasic.fl\_get\_coordunit \textit{(function)}}

    \vspace{0.5ex}

\hspace{.8\funcindent}\begin{boxedminipage}{\funcwidth}

    \raggedright \textbf{fl\_get\_coordunit}()

    \vspace{-1.5ex}

    \rule{\textwidth}{0.5\fboxrule}
\setlength{\parskip}{2ex}
    Returns the unit used for screen coordinates (e.g. 
    xfdata.FL\_COORD\_MM, xfdata.FL\_COORD\_centiPOINT, etc..).

\setlength{\parskip}{1ex}
      \textbf{Return Value}
    \vspace{-1ex}

      \begin{quote}
      current coordinates unit

      {\it (type=int)}

      \end{quote}

\textbf{Example:} cunit = fl\_get\_coordunit()



\textbf{Status:} Tested + Doc + Demo = OK



    \end{boxedminipage}

    \label{xformslib:flbasic:fl_get_border_width}
    \index{xformslib \textit{(package)}!xformslib.flbasic \textit{(module)}!xformslib.flbasic.fl\_get\_border\_width \textit{(function)}}

    \vspace{0.5ex}

\hspace{.8\funcindent}\begin{boxedminipage}{\funcwidth}

    \raggedright \textbf{fl\_get\_border\_width}()

    \vspace{-1.5ex}

    \rule{\textwidth}{0.5\fboxrule}
\setlength{\parskip}{2ex}
    Returns the width of border.

\setlength{\parskip}{1ex}
      \textbf{Return Value}
    \vspace{-1ex}

      \begin{quote}
      borderwidth (bw)

      {\it (type=int)}

      \end{quote}

\textbf{Status:} Tested + Doc + Demo = OK



    \end{boxedminipage}

    \label{xformslib:flbasic:fl_ringbell}
    \index{xformslib \textit{(package)}!xformslib.flbasic \textit{(module)}!xformslib.flbasic.fl\_ringbell \textit{(function)}}

    \vspace{0.5ex}

\hspace{.8\funcindent}\begin{boxedminipage}{\funcwidth}

    \raggedright \textbf{fl\_ringbell}(\textit{percent})

    \vspace{-1.5ex}

    \rule{\textwidth}{0.5\fboxrule}
\setlength{\parskip}{2ex}
    Sounds the keyboard ringbell (if capable). Note that not all keyboards 
    support volume variations.

\setlength{\parskip}{1ex}
      \textbf{Parameters}
      \vspace{-1ex}

      \begin{quote}
        \begin{Ventry}{xxxxxxx}

          \item[percent]

          volume value for the bell. Values from -100 (minimum, off), to 
          100 (max), 0 is default.

            {\it (type=int)}

        \end{Ventry}

      \end{quote}

\textbf{Example:} fl\_ringbell(50)



\textbf{Status:} Tested + Doc + NoDemo = OK



    \end{boxedminipage}

    \label{xformslib:flbasic:fl_gettime}
    \index{xformslib \textit{(package)}!xformslib.flbasic \textit{(module)}!xformslib.flbasic.fl\_gettime \textit{(function)}}

    \vspace{0.5ex}

\hspace{.8\funcindent}\begin{boxedminipage}{\funcwidth}

    \raggedright \textbf{fl\_gettime}()

    \vspace{-1.5ex}

    \rule{\textwidth}{0.5\fboxrule}
\setlength{\parskip}{2ex}
    Returns the current time, expressed in seconds and microseconds since 
    1st January 1970, 00:00 GMT. It is most useful for computing time 
    differences.

\setlength{\parskip}{1ex}
      \textbf{Return Value}
    \vspace{-1ex}

      \begin{quote}
      seconds and microseconds (secs, msecs)

      {\it (type=long, long)}

      \end{quote}

\textbf{Example:} secs, usecs = fl\_gettime()



\textbf{Attention:} API change from XForms - upstream was fl\_gettime(sec, usec)



\textbf{Status:} Tested + Doc + Demo = OK



    \end{boxedminipage}

    \label{xformslib:flbasic:fl_now}
    \index{xformslib \textit{(package)}!xformslib.flbasic \textit{(module)}!xformslib.flbasic.fl\_now \textit{(function)}}

    \vspace{0.5ex}

\hspace{.8\funcindent}\begin{boxedminipage}{\funcwidth}

    \raggedright \textbf{fl\_now}()

    \vspace{-1.5ex}

    \rule{\textwidth}{0.5\fboxrule}
\setlength{\parskip}{2ex}
    Returns a string form of the current date and time. The format of the 
    string is of the form "Wed Jun 30 21:49:08 1993"

\setlength{\parskip}{1ex}
      \textbf{Return Value}
    \vspace{-1ex}

      \begin{quote}
      current date and time (text)

      {\it (type=str)}

      \end{quote}

\textbf{Example:} curdattim = fl\_now()



\textbf{Status:} Tested + Doc + NoDemo = OK



    \end{boxedminipage}

    \label{xformslib:flbasic:fl_whoami}
    \index{xformslib \textit{(package)}!xformslib.flbasic \textit{(module)}!xformslib.flbasic.fl\_whoami \textit{(function)}}

    \vspace{0.5ex}

\hspace{.8\funcindent}\begin{boxedminipage}{\funcwidth}

    \raggedright \textbf{fl\_whoami}()

    \vspace{-1.5ex}

    \rule{\textwidth}{0.5\fboxrule}
\setlength{\parskip}{2ex}
    Returns the user name who is running the application.

\setlength{\parskip}{1ex}
      \textbf{Return Value}
    \vspace{-1ex}

      \begin{quote}
      text of username

      {\it (type=str)}

      \end{quote}

\textbf{Example:} usertxt = fl\_whoami()



\textbf{Status:} Tested + Doc + NoDemo = OK



    \end{boxedminipage}

    \label{xformslib:flbasic:fl_mouse_button}
    \index{xformslib \textit{(package)}!xformslib.flbasic \textit{(module)}!xformslib.flbasic.fl\_mouse\_button \textit{(function)}}

    \vspace{0.5ex}

\hspace{.8\funcindent}\begin{boxedminipage}{\funcwidth}

    \raggedright \textbf{fl\_mouse\_button}()

    \vspace{-1.5ex}

    \rule{\textwidth}{0.5\fboxrule}
\setlength{\parskip}{2ex}
    Returns which mouse button was pushed or released (from xfdata, e.g. 
    FL\_RIGHT\_MOUSE, FL\_MIDDLE\_MOUSE, etc..). Sometimes an application 
    program might need to find out more information about the event that 
    triggered a callback, e.g., to implement mouse button number sensitive 
    functionalities. This function, if needed, should be called from within
    a callback. If the callback is triggered by a shortcut, the function 
    returns the keysym (ascii value if ASCII) of the key plus FL\_SHORTCUT.
    For example, if a button has a shortcut {\textless}Ctrl{\textgreater}C 
    (ASCII value is 3), the button number returned upon activation of the 
    shortcut would be xfdata.FL\_SHORTCUT + 3. You can use 
    xfdata.FL\_SHORTCUT to determine if the callback is triggered by a 
    shortcut or not.

\setlength{\parskip}{1ex}
      \textbf{Return Value}
    \vspace{-1ex}

      \begin{quote}
      which mouse button was pushed or released

      {\it (type=long)}

      \end{quote}

\textbf{Example:} mousebtn = fl\_mouse\_button()



\textbf{Status:} Tested + Doc + Demo = OK



    \end{boxedminipage}

    \label{xformslib:flbasic:fl_mouse_button}
    \index{xformslib \textit{(package)}!xformslib.flbasic \textit{(module)}!xformslib.flbasic.fl\_mouse\_button \textit{(function)}}

    \vspace{0.5ex}

\hspace{.8\funcindent}\begin{boxedminipage}{\funcwidth}

    \raggedright \textbf{fl\_mousebutton}()

    \vspace{-1.5ex}

    \rule{\textwidth}{0.5\fboxrule}
\setlength{\parskip}{2ex}
    Returns which mouse button was pushed or released (from xfdata, e.g. 
    FL\_RIGHT\_MOUSE, FL\_MIDDLE\_MOUSE, etc..). Sometimes an application 
    program might need to find out more information about the event that 
    triggered a callback, e.g., to implement mouse button number sensitive 
    functionalities. This function, if needed, should be called from within
    a callback. If the callback is triggered by a shortcut, the function 
    returns the keysym (ascii value if ASCII) of the key plus FL\_SHORTCUT.
    For example, if a button has a shortcut {\textless}Ctrl{\textgreater}C 
    (ASCII value is 3), the button number returned upon activation of the 
    shortcut would be xfdata.FL\_SHORTCUT + 3. You can use 
    xfdata.FL\_SHORTCUT to determine if the callback is triggered by a 
    shortcut or not.

\setlength{\parskip}{1ex}
      \textbf{Return Value}
    \vspace{-1ex}

      \begin{quote}
      which mouse button was pushed or released

      {\it (type=long)}

      \end{quote}

\textbf{Example:} mousebtn = fl\_mouse\_button()



\textbf{Status:} Tested + Doc + Demo = OK



    \end{boxedminipage}

    \label{xformslib:flbasic:fl_set_err_logfp}
    \index{xformslib \textit{(package)}!xformslib.flbasic \textit{(module)}!xformslib.flbasic.fl\_set\_err\_logfp \textit{(function)}}

    \vspace{0.5ex}

\hspace{.8\funcindent}\begin{boxedminipage}{\funcwidth}

    \raggedright \textbf{fl\_set\_err\_logfp}(\textit{pFile})

    \vspace{-1.5ex}

    \rule{\textwidth}{0.5\fboxrule}
\setlength{\parskip}{2ex}
    Makes the default message handler to log the error to a file instead of
    printing to stderr.

\setlength{\parskip}{1ex}
      \textbf{Parameters}
      \vspace{-1ex}

      \begin{quote}
        \begin{Ventry}{xxxxx}

          \item[pFile]

          file opened in "w" mode by fl\_popen()

            {\it (type=pointer to FILE)}

        \end{Ventry}

      \end{quote}

\textbf{Example:}
\begin{quote}
  \begin{itemize}

  \item
    \setlength{\parskip}{0.6ex}
pfile = fl\_popen("myerrlog", "w")



  \item fl\_set\_err\_logfp(pfile)



\end{itemize}

\end{quote}

\textbf{Status:} Tested + Doc + NoDemo = OK



    \end{boxedminipage}

    \label{xformslib:flbasic:fl_set_error_handler}
    \index{xformslib \textit{(package)}!xformslib.flbasic \textit{(module)}!xformslib.flbasic.fl\_set\_error\_handler \textit{(function)}}

    \vspace{0.5ex}

\hspace{.8\funcindent}\begin{boxedminipage}{\funcwidth}

    \raggedright \textbf{fl\_set\_error\_handler}(\textit{py\_ErrorFunc})

    \vspace{-1.5ex}

    \rule{\textwidth}{0.5\fboxrule}
\setlength{\parskip}{2ex}
    Normally the Forms Library reports errors to stderr. This can be 
    avoided or modified by registering an error handling function. The 
    library will call the user handler function with a string indicating in
    which function an error occured and a formatting string, followed by 
    zero or more arguments. To restore the default handler, call the 
    function again with user handler set to None. You can call this 
    function anytime and as many times as you wish.

\setlength{\parskip}{1ex}
      \textbf{Parameters}
      \vspace{-1ex}

      \begin{quote}
        \begin{Ventry}{xxxxxxxxxxxx}

          \item[py\_ErrorFunc]

          python function for handling error, no return

            {\it (type=\_\_ funcname (strng, strng) \_\_)}

        \end{Ventry}

      \end{quote}

\textbf{Example:}
\begin{quote}
  \begin{itemize}

  \item
    \setlength{\parskip}{0.6ex}
def errhandler(funcnam, errmsg):



  \item {\textbar}-{\textgreater}{\textbar} print "Error caught in \%s: \%s." \% 
(funcnam, errmsg)



  \item fl\_set\_error\_handler(errhandler)



\end{itemize}

\end{quote}

\textbf{Status:} Tested + Doc + NoDemo = OK



    \end{boxedminipage}

    \label{xformslib:flbasic:fl_msleep}
    \index{xformslib \textit{(package)}!xformslib.flbasic \textit{(module)}!xformslib.flbasic.fl\_msleep \textit{(function)}}

    \vspace{0.5ex}

\hspace{.8\funcindent}\begin{boxedminipage}{\funcwidth}

    \raggedright \textbf{fl\_msleep}(\textit{msec})

    \vspace{-1.5ex}

    \rule{\textwidth}{0.5\fboxrule}
\setlength{\parskip}{2ex}
    Waits for a number of milliseconds (with the best resolution possible 
    on your system).

\setlength{\parskip}{1ex}
      \textbf{Parameters}
      \vspace{-1ex}

      \begin{quote}
        \begin{Ventry}{xxxx}

          \item[msec]

          milliseconds to sleep

            {\it (type=long)}

        \end{Ventry}

      \end{quote}

      \textbf{Return Value}
    \vspace{-1ex}

      \begin{quote}
      0 (on success)

      {\it (type=int)}

      \end{quote}

\textbf{Example:} fl\_msleep(200)



\textbf{Status:} Tested + Doc + Demo = OK



    \end{boxedminipage}

    \label{xformslib:flbasic:fl_is_same_object}
    \index{xformslib \textit{(package)}!xformslib.flbasic \textit{(module)}!xformslib.flbasic.fl\_is\_same\_object \textit{(function)}}

    \vspace{0.5ex}

\hspace{.8\funcindent}\begin{boxedminipage}{\funcwidth}

    \raggedright \textbf{fl\_is\_same\_object}(\textit{pFlObject1}, \textit{pFlObject2})

    \vspace{-1.5ex}

    \rule{\textwidth}{0.5\fboxrule}
\setlength{\parskip}{2ex}
    Does a comparison between two objects, if they are the same, or not.

\setlength{\parskip}{1ex}
      \textbf{Parameters}
      \vspace{-1ex}

      \begin{quote}
        \begin{Ventry}{xxxxxxxxxx}

          \item[pFlObject1]

          1st object to compare

            {\it (type=pointer to xfdata.FL\_OBJECT)}

          \item[pFlObject2]

          2nd object to compare

            {\it (type=pointer to xfdata.FL\_OBJECT)}

        \end{Ventry}

      \end{quote}

      \textbf{Return Value}
    \vspace{-1ex}

      \begin{quote}
      0 (if they are not the same) or non-zero (if they are)

      {\it (type=int)}

      \end{quote}

\textbf{Example:} if fl\_is\_same\_object(pobj, pexitobj): ...



\textbf{Status:} Tested + Doc + Demo = OK



    \end{boxedminipage}


%%%%%%%%%%%%%%%%%%%%%%%%%%%%%%%%%%%%%%%%%%%%%%%%%%%%%%%%%%%%%%%%%%%%%%%%%%%
%%                               Variables                               %%
%%%%%%%%%%%%%%%%%%%%%%%%%%%%%%%%%%%%%%%%%%%%%%%%%%%%%%%%%%%%%%%%%%%%%%%%%%%

  \subsection{Variables}

    \vspace{-1cm}
\hspace{\varindent}\begin{longtable}{|p{\varnamewidth}|p{\vardescrwidth}|l}
\cline{1-2}
\cline{1-2} \centering \textbf{Name} & \centering \textbf{Description}& \\
\cline{1-2}
\endhead\cline{1-2}\multicolumn{3}{r}{\small\textit{continued on next page}}\\\endfoot\cline{1-2}
\endlastfoot\raggedright F\-L\-\_\-E\-V\-E\-N\-T\- & \raggedright \textbf{Value:} 
{\tt cty.POINTER(xfdata.FL\_OBJECT).in\_dll(library.load\_so\_libf\texttt{...}}&\\
\cline{1-2}
\raggedright \_\-\_\-p\-a\-c\-k\-a\-g\-e\-\_\-\_\- & \raggedright \textbf{Value:} 
{\tt \texttt{'}\texttt{xformslib}\texttt{'}}&\\
\cline{1-2}
\end{longtable}

    \index{xformslib \textit{(package)}!xformslib.flbasic \textit{(module)}|)}
