%
% API Documentation for API Documentation
% Module xformslib.flslider
%
% Generated by epydoc 3.0.1
% [Fri May 21 15:38:50 2010]
%

%%%%%%%%%%%%%%%%%%%%%%%%%%%%%%%%%%%%%%%%%%%%%%%%%%%%%%%%%%%%%%%%%%%%%%%%%%%
%%                          Module Description                           %%
%%%%%%%%%%%%%%%%%%%%%%%%%%%%%%%%%%%%%%%%%%%%%%%%%%%%%%%%%%%%%%%%%%%%%%%%%%%

    \index{xformslib \textit{(package)}!xformslib.flslider \textit{(module)}|(}
\section{Module xformslib.flslider}

    \label{xformslib:flslider}

xforms-python's functions to manage slider objects.

Copyright (C) 2009, 2010  Luca Lazzaroni ``LukenShiro''
e-mail: <\href{mailto:lukenshiro@ngi.it}{lukenshiro@ngi.it}>

This program is free software: you can redistribute it and/or modify
it under the terms of the GNU Lesser General Public License as
published by the Free Software Foundation, version 2.1 of the License.

This program is distributed in the hope that it will be useful,
but WITHOUT ANY WARRANTY; without even the implied warranty of
MERCHANTABILITY or FITNESS FOR A PARTICULAR PURPOSE. See the
GNU Lesser General Public License for more details.

You should have received a copy of the GNU LGPL along with this
program. If not, see <\href{http://www.gnu.org/licenses/}{http://www.gnu.org/licenses/}>.

See CREDITS file to read acknowledgements and thanks to XForms,
ctypes and other developers.

%%%%%%%%%%%%%%%%%%%%%%%%%%%%%%%%%%%%%%%%%%%%%%%%%%%%%%%%%%%%%%%%%%%%%%%%%%%
%%                               Functions                               %%
%%%%%%%%%%%%%%%%%%%%%%%%%%%%%%%%%%%%%%%%%%%%%%%%%%%%%%%%%%%%%%%%%%%%%%%%%%%

  \subsection{Functions}

    \label{xformslib:flslider:fl_add_slider}
    \index{xformslib \textit{(package)}!xformslib.flslider \textit{(module)}!xformslib.flslider.fl\_add\_slider \textit{(function)}}

    \vspace{0.5ex}

\hspace{.8\funcindent}\begin{boxedminipage}{\funcwidth}

    \raggedright \textbf{fl\_add\_slider}(\textit{slidertype}, \textit{x}, \textit{y}, \textit{w}, \textit{h}, \textit{label})

    \vspace{-1.5ex}

    \rule{\textwidth}{0.5\fboxrule}
\setlength{\parskip}{2ex}

Adds a slider to a form. No value is displayed.

-{}-
\setlength{\parskip}{1ex}
      \textbf{Parameters}
      \vspace{-1ex}

      \begin{quote}
        \begin{Ventry}{xxxxxxxxxx}

          \item[slidertype]


type of slider to be added. Values (from xfdata.py) FL\_VERT\_SLIDER,
FL\_HOR\_SLIDER, FL\_VERT\_FILL\_SLIDER, FL\_HOR\_FILL\_SLIDER,
FL\_VERT\_NICE\_SLIDER, FL\_HOR\_NICE\_SLIDER, FL\_VERT\_BROWSER\_SLIDER,
FL\_HOR\_BROWSER\_SLIDER, FL\_VERT\_BROWSER\_SLIDER2,FL\_HOR\_BROWSER\_SLIDER2,
FL\_VERT\_THIN\_SLIDER, FL\_HOR\_THIN\_SLIDER, FL\_VERT\_THIN\_SLIDER,
FL\_HOR\_THIN\_SLIDER, FL\_VERT\_NICE\_SLIDER2, FL\_HOR\_NICE\_SLIDER2,
FL\_VERT\_BASIC\_SLIDER, FL\_HOR\_BASIC\_SLIDER
            {\it (type=int)}

          \item[x]


horizontal position (upper-left corner)
            {\it (type=int)}

          \item[y]


vertical position (upper-left corner)
            {\it (type=int)}

          \item[w]


width in coord units
            {\it (type=int)}

          \item[h]


height in coord units
            {\it (type=int)}

          \item[label]


label of the slider (placed below it by default)
            {\it (type=int)}

        \end{Ventry}

      \end{quote}

      \textbf{Return Value}
    \vspace{-1ex}

      \begin{quote}

slider object added (pFlObject)
      {\it (type=pointer to xfdata.FL\_OBJECT)}

      \end{quote}

\textbf{Note:} 
e.g. \emph{todo}


\textbf{Status:} 
Tested + NoDoc + Demo = OK


    \end{boxedminipage}

    \label{xformslib:flslider:fl_add_valslider}
    \index{xformslib \textit{(package)}!xformslib.flslider \textit{(module)}!xformslib.flslider.fl\_add\_valslider \textit{(function)}}

    \vspace{0.5ex}

\hspace{.8\funcindent}\begin{boxedminipage}{\funcwidth}

    \raggedright \textbf{fl\_add\_valslider}(\textit{slidertype}, \textit{x}, \textit{y}, \textit{w}, \textit{h}, \textit{label})

    \vspace{-1.5ex}

    \rule{\textwidth}{0.5\fboxrule}
\setlength{\parskip}{2ex}

Adds a slider to a form. Its value is displayed above or to the left
of the slider.

-{}-
\setlength{\parskip}{1ex}
      \textbf{Parameters}
      \vspace{-1ex}

      \begin{quote}
        \begin{Ventry}{xxxxxxxxxx}

          \item[slidertype]


type of the slider. Values (from xfdata.py) FL\_VERT\_SLIDER,
FL\_HOR\_SLIDER, FL\_VERT\_FILL\_SLIDER, FL\_HOR\_FILL\_SLIDER,
FL\_VERT\_NICE\_SLIDER, FL\_HOR\_NICE\_SLIDER, FL\_VERT\_BROWSER\_SLIDER,
FL\_HOR\_BROWSER\_SLIDER, FL\_VERT\_BROWSER\_SLIDER2, FL\_HOR\_BROWSER\_SLIDER2,
FL\_VERT\_THIN\_SLIDER, FL\_HOR\_THIN\_SLIDER, FL\_VERT\_THIN\_SLIDER,
FL\_HOR\_THIN\_SLIDER, FL\_VERT\_NICE\_SLIDER2, FL\_HOR\_NICE\_SLIDER2,
FL\_VERT\_BASIC\_SLIDER, FL\_HOR\_BASIC\_SLIDER
            {\it (type=int)}

          \item[x]


horizontal position (upper-left corner)
            {\it (type=int)}

          \item[y]


vertical position (upper-left corner)
            {\it (type=int)}

          \item[w]


width in coord units
            {\it (type=int)}

          \item[h]


height in coord units
            {\it (type=int)}

          \item[label]


text label of slider
            {\it (type=str)}

        \end{Ventry}

      \end{quote}

      \textbf{Return Value}
    \vspace{-1ex}

      \begin{quote}

slider with value object added (pFlObject)
      {\it (type=pointer to xfdata.FL\_OBJECT)}

      \end{quote}

\textbf{Note:} 
e.g. \emph{todo}


\textbf{Status:} 
Tested + NoDoc + Demo = OK


    \end{boxedminipage}

    \label{xformslib:flslider:fl_set_slider_value}
    \index{xformslib \textit{(package)}!xformslib.flslider \textit{(module)}!xformslib.flslider.fl\_set\_slider\_value \textit{(function)}}

    \vspace{0.5ex}

\hspace{.8\funcindent}\begin{boxedminipage}{\funcwidth}

    \raggedright \textbf{fl\_set\_slider\_value}(\textit{pFlObject}, \textit{val})

    \vspace{-1.5ex}

    \rule{\textwidth}{0.5\fboxrule}
\setlength{\parskip}{2ex}

Changes the value of a slider.

-{}-
\setlength{\parskip}{1ex}
      \textbf{Parameters}
      \vspace{-1ex}

      \begin{quote}
        \begin{Ventry}{xxxxxxxxx}

          \item[pFlObject]


slider object
            {\it (type=pointer to xfdata.FL\_OBJECT)}

          \item[val]


new value of slider
            {\it (type=float)}

        \end{Ventry}

      \end{quote}

\textbf{Note:} 
e.g. \emph{todo}


\textbf{Status:} 
Tested + NoDoc + Demo = OK


    \end{boxedminipage}

    \label{xformslib:flslider:fl_get_slider_value}
    \index{xformslib \textit{(package)}!xformslib.flslider \textit{(module)}!xformslib.flslider.fl\_get\_slider\_value \textit{(function)}}

    \vspace{0.5ex}

\hspace{.8\funcindent}\begin{boxedminipage}{\funcwidth}

    \raggedright \textbf{fl\_get\_slider\_value}(\textit{pFlObject})

    \vspace{-1.5ex}

    \rule{\textwidth}{0.5\fboxrule}
\setlength{\parskip}{2ex}

Obtains value of a slider.

-{}-
\setlength{\parskip}{1ex}
      \textbf{Parameters}
      \vspace{-1ex}

      \begin{quote}
        \begin{Ventry}{xxxxxxxxx}

          \item[pFlObject]


slider object
            {\it (type=pointer to xfdata.FL\_OBJECT)}

        \end{Ventry}

      \end{quote}

      \textbf{Return Value}
    \vspace{-1ex}

      \begin{quote}

value
      {\it (type=float)}

      \end{quote}

\textbf{Note:} 
e.g. \emph{todo}


\textbf{Status:} 
Tested + NoDoc + Demo = OK


    \end{boxedminipage}

    \label{xformslib:flslider:fl_set_slider_bounds}
    \index{xformslib \textit{(package)}!xformslib.flslider \textit{(module)}!xformslib.flslider.fl\_set\_slider\_bounds \textit{(function)}}

    \vspace{0.5ex}

\hspace{.8\funcindent}\begin{boxedminipage}{\funcwidth}

    \raggedright \textbf{fl\_set\_slider\_bounds}(\textit{pFlObject}, \textit{minbound}, \textit{maxbound})

    \vspace{-1.5ex}

    \rule{\textwidth}{0.5\fboxrule}
\setlength{\parskip}{2ex}

Sets minimum and maximum bounds/limits of a slider.

-{}-
\setlength{\parskip}{1ex}
      \textbf{Parameters}
      \vspace{-1ex}

      \begin{quote}
        \begin{Ventry}{xxxxxxxxx}

          \item[pFlObject]


slider object
            {\it (type=pointer to xfdata.FL\_OBJECT)}

          \item[minbound]


minimum bound of slider
            {\it (type=float)}

          \item[maxbound]


maximum bound of slider
            {\it (type=float)}

        \end{Ventry}

      \end{quote}

\textbf{Note:} 
e.g. \emph{todo}


\textbf{Status:} 
Tested + NoDoc + Demo = OK


    \end{boxedminipage}

    \label{xformslib:flslider:fl_get_slider_bounds}
    \index{xformslib \textit{(package)}!xformslib.flslider \textit{(module)}!xformslib.flslider.fl\_get\_slider\_bounds \textit{(function)}}

    \vspace{0.5ex}

\hspace{.8\funcindent}\begin{boxedminipage}{\funcwidth}

    \raggedright \textbf{fl\_get\_slider\_bounds}(\textit{pFlObject})

    \vspace{-1.5ex}

    \rule{\textwidth}{0.5\fboxrule}
\setlength{\parskip}{2ex}

Obtains minimum and maximum bounds/limits of a slider.

-{}-
\setlength{\parskip}{1ex}
      \textbf{Parameters}
      \vspace{-1ex}

      \begin{quote}
        \begin{Ventry}{xxxxxxxxx}

          \item[pFlObject]


slider object
            {\it (type=pointer to xfdata.FL\_OBJECT)}

        \end{Ventry}

      \end{quote}

      \textbf{Return Value}
    \vspace{-1ex}

      \begin{quote}

minimum bound, maximum bound
      {\it (type=float, float)}

      \end{quote}

\textbf{Note:} 
e.g. \emph{todo}


\textbf{Attention:} 
API change from XForms - upstream was
fl\_get\_slider\_bounds(pFlObject, minbound, maxbound)


\textbf{Status:} 
Untested + NoDoc + NoDemo = NOT OK


    \end{boxedminipage}

    \label{xformslib:flslider:fl_set_slider_step}
    \index{xformslib \textit{(package)}!xformslib.flslider \textit{(module)}!xformslib.flslider.fl\_set\_slider\_step \textit{(function)}}

    \vspace{0.5ex}

\hspace{.8\funcindent}\begin{boxedminipage}{\funcwidth}

    \raggedright \textbf{fl\_set\_slider\_step}(\textit{pFlObject}, \textit{value})

    \vspace{-1.5ex}

    \rule{\textwidth}{0.5\fboxrule}
\setlength{\parskip}{2ex}

\emph{todo}

-{}-
\setlength{\parskip}{1ex}
      \textbf{Parameters}
      \vspace{-1ex}

      \begin{quote}
        \begin{Ventry}{xxxxxxxxx}

          \item[pFlObject]


slider object
            {\it (type=pointer to xfdata.FL\_OBJECT)}

          \item[value]


step value
            {\it (type=float)}

        \end{Ventry}

      \end{quote}

\textbf{Note:} 
e.g. \emph{todo}


\textbf{Status:} 
Untested + NoDoc + NoDemo = NOT OK


    \end{boxedminipage}

    \label{xformslib:flslider:fl_set_slider_increment}
    \index{xformslib \textit{(package)}!xformslib.flslider \textit{(module)}!xformslib.flslider.fl\_set\_slider\_increment \textit{(function)}}

    \vspace{0.5ex}

\hspace{.8\funcindent}\begin{boxedminipage}{\funcwidth}

    \raggedright \textbf{fl\_set\_slider\_increment}(\textit{pFlObject}, \textit{leftbtnval}, \textit{midlbtnval})

    \vspace{-1.5ex}

    \rule{\textwidth}{0.5\fboxrule}
\setlength{\parskip}{2ex}

\emph{todo}

-{}-
\setlength{\parskip}{1ex}
      \textbf{Parameters}
      \vspace{-1ex}

      \begin{quote}
        \begin{Ventry}{xxxxxxxxxx}

          \item[pFlObject]


slider object
            {\it (type=pointer to xfdata.FL\_OBJECT)}

          \item[leftbtnval]


value to increment if the left mouse button is pressed
            {\it (type=float)}

          \item[midlbtnval]


value to increment if the middle mouse button is pressed
            {\it (type=float)}

        \end{Ventry}

      \end{quote}

\textbf{Note:} 
e.g. \emph{todo}


\textbf{Status:} 
Untested + NoDoc + NoDemo = NOT OK


    \end{boxedminipage}

    \label{xformslib:flslider:fl_get_slider_increment}
    \index{xformslib \textit{(package)}!xformslib.flslider \textit{(module)}!xformslib.flslider.fl\_get\_slider\_increment \textit{(function)}}

    \vspace{0.5ex}

\hspace{.8\funcindent}\begin{boxedminipage}{\funcwidth}

    \raggedright \textbf{fl\_get\_slider\_increment}(\textit{pFlObject})

    \vspace{-1.5ex}

    \rule{\textwidth}{0.5\fboxrule}
\setlength{\parskip}{2ex}

\emph{todo}

-{}-
\setlength{\parskip}{1ex}
      \textbf{Parameters}
      \vspace{-1ex}

      \begin{quote}
        \begin{Ventry}{xxxxxxxxx}

          \item[pFlObject]


slider object
            {\it (type=pointer to xfdata.FL\_OBJECT)}

        \end{Ventry}

      \end{quote}

      \textbf{Return Value}
    \vspace{-1ex}

      \begin{quote}

left button increment, middle button increment
      {\it (type=float, float)}

      \end{quote}

\textbf{Note:} 
e.g. \emph{todo}


\textbf{Attention:} 
API change from XForms - upstream was
fl\_get\_slider\_increment(pFlObject, leftbtnval, midlbtnval)


\textbf{Status:} 
Untested + NoDoc + NoDemo = NOT OK


    \end{boxedminipage}

    \label{xformslib:flslider:fl_set_slider_size}
    \index{xformslib \textit{(package)}!xformslib.flslider \textit{(module)}!xformslib.flslider.fl\_set\_slider\_size \textit{(function)}}

    \vspace{0.5ex}

\hspace{.8\funcindent}\begin{boxedminipage}{\funcwidth}

    \raggedright \textbf{fl\_set\_slider\_size}(\textit{pFlObject}, \textit{size})

    \vspace{-1.5ex}

    \rule{\textwidth}{0.5\fboxrule}
\setlength{\parskip}{2ex}

Sets the size of a slider.

-{}-
\setlength{\parskip}{1ex}
      \textbf{Parameters}
      \vspace{-1ex}

      \begin{quote}
        \begin{Ventry}{xxxxxxxxx}

          \item[pFlObject]


slider object
            {\it (type=pointer to xfdata.FL\_OBJECT)}

          \item[size]


value of size of the slider
            {\it (type=float)}

        \end{Ventry}

      \end{quote}

\textbf{Note:} 
e.g. \emph{todo}


\textbf{Status:} 
Tested + NoDoc + Demo = OK


    \end{boxedminipage}

    \label{xformslib:flslider:fl_set_slider_precision}
    \index{xformslib \textit{(package)}!xformslib.flslider \textit{(module)}!xformslib.flslider.fl\_set\_slider\_precision \textit{(function)}}

    \vspace{0.5ex}

\hspace{.8\funcindent}\begin{boxedminipage}{\funcwidth}

    \raggedright \textbf{fl\_set\_slider\_precision}(\textit{pFlObject}, \textit{precnum})

    \vspace{-1.5ex}

    \rule{\textwidth}{0.5\fboxrule}
\setlength{\parskip}{2ex}

Sets precision which value a valslider is shown with.

-{}-
\setlength{\parskip}{1ex}
      \textbf{Parameters}
      \vspace{-1ex}

      \begin{quote}
        \begin{Ventry}{xxxxxxxxx}

          \item[pFlObject]


slider object
            {\it (type=pointer to xfdata.FL\_OBJECT)}

          \item[precnum]


precision of shown value
            {\it (type=int)}

        \end{Ventry}

      \end{quote}

\textbf{Note:} 
e.g. \emph{todo}


\textbf{Status:} 
Untested + NoDoc + NoDemo = NOT OK


    \end{boxedminipage}

    \label{xformslib:flslider:fl_set_slider_filter}
    \index{xformslib \textit{(package)}!xformslib.flslider \textit{(module)}!xformslib.flslider.fl\_set\_slider\_filter \textit{(function)}}

    \vspace{0.5ex}

\hspace{.8\funcindent}\begin{boxedminipage}{\funcwidth}

    \raggedright \textbf{fl\_set\_slider\_filter}(\textit{pFlObject}, \textit{py\_ValFilter})

    \vspace{-1.5ex}

    \rule{\textwidth}{0.5\fboxrule}
\setlength{\parskip}{2ex}

Registers a filter function to show alues in a slider object. By
default, slider value shown in floating point format)

-{}-
\setlength{\parskip}{1ex}
      \textbf{Parameters}
      \vspace{-1ex}

      \begin{quote}
        \begin{Ventry}{xxxxxxxxxxxx}

          \item[pFlObject]


slider object
            {\it (type=pointer to xfdata.FL\_OBJECT)}

          \item[py\_ValFilter]


name referring to function(pFlObject, valfloat, intprecis) -> string
            {\it (type=python function to show values in slider, returned value)}

        \end{Ventry}

      \end{quote}

\textbf{Note:} 
e.g. \emph{todo}


\textbf{Status:} 
Untested + NoDoc + NoDemo = NOT OK


    \end{boxedminipage}

    \index{xformslib \textit{(package)}!xformslib.flslider \textit{(module)}|)}
