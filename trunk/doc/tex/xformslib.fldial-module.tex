%
% API Documentation for API Documentation
% Module xformslib.fldial
%
% Generated by epydoc 3.0.1
% [Fri May 14 18:28:31 2010]
%

%%%%%%%%%%%%%%%%%%%%%%%%%%%%%%%%%%%%%%%%%%%%%%%%%%%%%%%%%%%%%%%%%%%%%%%%%%%
%%                          Module Description                           %%
%%%%%%%%%%%%%%%%%%%%%%%%%%%%%%%%%%%%%%%%%%%%%%%%%%%%%%%%%%%%%%%%%%%%%%%%%%%

    \index{xformslib \textit{(package)}!xformslib.fldial \textit{(module)}|(}
\section{Module xformslib.fldial}

    \label{xformslib:fldial}

fldial.py - xforms-python's functions to manage dial objects.

Copyright (C) 2009, 2010  Luca Lazzaroni ``LukenShiro''
e-mail: <\href{mailto:lukenshiro@ngi.it}{lukenshiro@ngi.it}>

This program is free software: you can redistribute it and/or modify
it under the terms of the GNU Lesser General Public License as
published by the Free Software Foundation, version 2.1 of the License.

This program is distributed in the hope that it will be useful,
but WITHOUT ANY WARRANTY; without even the implied warranty of
MERCHANTABILITY or FITNESS FOR A PARTICULAR PURPOSE. See the
GNU Lesser General Public License for more details.

You should have received a copy of the GNU LGPL along with this
program. If not, see <\href{http://www.gnu.org/licenses/}{http://www.gnu.org/licenses/}>.

See CREDITS file to read acknowledgements and thanks to XForms,
ctypes and other developers.

%%%%%%%%%%%%%%%%%%%%%%%%%%%%%%%%%%%%%%%%%%%%%%%%%%%%%%%%%%%%%%%%%%%%%%%%%%%
%%                               Functions                               %%
%%%%%%%%%%%%%%%%%%%%%%%%%%%%%%%%%%%%%%%%%%%%%%%%%%%%%%%%%%%%%%%%%%%%%%%%%%%

  \subsection{Functions}

    \label{xformslib:fldial:fl_add_dial}
    \index{xformslib \textit{(package)}!xformslib.fldial \textit{(module)}!xformslib.fldial.fl\_add\_dial \textit{(function)}}

    \vspace{0.5ex}

\hspace{.8\funcindent}\begin{boxedminipage}{\funcwidth}

    \raggedright \textbf{fl\_add\_dial}(\textit{dialtype}, \textit{x}, \textit{y}, \textit{w}, \textit{h}, \textit{label})

    \vspace{-1.5ex}

    \rule{\textwidth}{0.5\fboxrule}
\setlength{\parskip}{2ex}

Adds a dial object to the form.

-{}-
\setlength{\parskip}{1ex}
      \textbf{Parameters}
      \vspace{-1ex}

      \begin{quote}
        \begin{Ventry}{xxxxxxxx}

          \item[dialtype]


type of dial to be added. Values (from xfdata.py) FL\_NORMAL\_DIAL,
FL\_LINE\_DIAL, FL\_FILL\_DIAL
            {\it (type=int)}

          \item[x]


horizontal position (upper-left corner)
            {\it (type=int)}

          \item[y]


vertical position (upper-left corner)
            {\it (type=int)}

          \item[w]


width in coord units
            {\it (type=int)}

          \item[h]


height in coord units
            {\it (type=int)}

          \item[label]


text label of dial
            {\it (type=str)}

        \end{Ventry}

      \end{quote}

      \textbf{Return Value}
    \vspace{-1ex}

      \begin{quote}

dial object added (pFlObject)
      {\it (type=pointer to xfdata.FL\_OBJECT)}

      \end{quote}

\textbf{Note:} 
e.g. fl\_add\_dial(xfdata.FL\_LINE\_DIAL, 140, 120, 123, 521, ``MyDial'')


\textbf{Status:} 
Tested + Doc + Demo = OK


    \end{boxedminipage}

    \label{xformslib:fldial:fl_set_dial_value}
    \index{xformslib \textit{(package)}!xformslib.fldial \textit{(module)}!xformslib.fldial.fl\_set\_dial\_value \textit{(function)}}

    \vspace{0.5ex}

\hspace{.8\funcindent}\begin{boxedminipage}{\funcwidth}

    \raggedright \textbf{fl\_set\_dial\_value}(\textit{pFlObject}, \textit{val})

    \vspace{-1.5ex}

    \rule{\textwidth}{0.5\fboxrule}
\setlength{\parskip}{2ex}

Sets the value of a dial object. By default the value is 0.

-{}-
\setlength{\parskip}{1ex}
      \textbf{Parameters}
      \vspace{-1ex}

      \begin{quote}
        \begin{Ventry}{xxxxxxxxx}

          \item[pFlObject]


dial object
            {\it (type=pointer to xfdata.FL\_OBJECT)}

          \item[val]


value of dial to be set
            {\it (type=float)}

        \end{Ventry}

      \end{quote}

\textbf{Note:} 
e.g. fl\_set\_dial\_value(dialobj, 155.0)


\textbf{Status:} 
Tested + Doc + Demo = OK


    \end{boxedminipage}

    \label{xformslib:fldial:fl_get_dial_value}
    \index{xformslib \textit{(package)}!xformslib.fldial \textit{(module)}!xformslib.fldial.fl\_get\_dial\_value \textit{(function)}}

    \vspace{0.5ex}

\hspace{.8\funcindent}\begin{boxedminipage}{\funcwidth}

    \raggedright \textbf{fl\_get\_dial\_value}(\textit{pFlObject})

    \vspace{-1.5ex}

    \rule{\textwidth}{0.5\fboxrule}
\setlength{\parskip}{2ex}

Obtains the current value of a dial object.

-{}-
\setlength{\parskip}{1ex}
      \textbf{Parameters}
      \vspace{-1ex}

      \begin{quote}
        \begin{Ventry}{xxxxxxxxx}

          \item[pFlObject]


dial object
            {\it (type=pointer to xfdata.FL\_OBJECT)}

        \end{Ventry}

      \end{quote}

      \textbf{Return Value}
    \vspace{-1ex}

      \begin{quote}

current value of dial
      {\it (type=float)}

      \end{quote}

\textbf{Note:} 
e.g. currval = fl\_get\_dial\_value(dialobj)


\textbf{Status:} 
Tested + Doc + Demo = OK


    \end{boxedminipage}

    \label{xformslib:fldial:fl_set_dial_bounds}
    \index{xformslib \textit{(package)}!xformslib.fldial \textit{(module)}!xformslib.fldial.fl\_set\_dial\_bounds \textit{(function)}}

    \vspace{0.5ex}

\hspace{.8\funcindent}\begin{boxedminipage}{\funcwidth}

    \raggedright \textbf{fl\_set\_dial\_bounds}(\textit{pFlObject}, \textit{minbound}, \textit{maxbound})

    \vspace{-1.5ex}

    \rule{\textwidth}{0.5\fboxrule}
\setlength{\parskip}{2ex}

Sets the minimum and the maximum values of a dial object.

-{}-
\setlength{\parskip}{1ex}
      \textbf{Parameters}
      \vspace{-1ex}

      \begin{quote}
        \begin{Ventry}{xxxxxxxxx}

          \item[pFlObject]


dial object
            {\it (type=pointer to xfdata.FL\_OBJECT)}

          \item[minbound]


minimum value of dial. By default it is 0.0.
            {\it (type=float)}

          \item[maxbound]


maximum value of dial. By default it is 1.0.
            {\it (type=float)}

        \end{Ventry}

      \end{quote}

\textbf{Note:} 
e.g. fl\_set\_dial\_bounds(dialobj, 0, 200)


\textbf{Status:} 
Tested + Doc + Demo = OK


    \end{boxedminipage}

    \label{xformslib:fldial:fl_get_dial_bounds}
    \index{xformslib \textit{(package)}!xformslib.fldial \textit{(module)}!xformslib.fldial.fl\_get\_dial\_bounds \textit{(function)}}

    \vspace{0.5ex}

\hspace{.8\funcindent}\begin{boxedminipage}{\funcwidth}

    \raggedright \textbf{fl\_get\_dial\_bounds}(\textit{pFlObject})

    \vspace{-1.5ex}

    \rule{\textwidth}{0.5\fboxrule}
\setlength{\parskip}{2ex}

Obtains the minimum and maximum values of a dial object.

-{}-
\setlength{\parskip}{1ex}
      \textbf{Parameters}
      \vspace{-1ex}

      \begin{quote}
        \begin{Ventry}{xxxxxxxxx}

          \item[pFlObject]


dial object
            {\it (type=pointer to xfdata.FL\_OBJECT)}

        \end{Ventry}

      \end{quote}

      \textbf{Return Value}
    \vspace{-1ex}

      \begin{quote}

minimum value, maximum value of dial
      {\it (type=float, float)}

      \end{quote}

\textbf{Note:} 
e.g. minb, maxb = fl\_get\_dial\_bounds(dialobj)


\textbf{Attention:} 
API change from XForms - upstream was
fl\_get\_dial\_bounds(pFlObject, minbound, maxbound)


\textbf{Status:} 
Tested + Doc + NoDemo = OK


    \end{boxedminipage}

    \label{xformslib:fldial:fl_set_dial_step}
    \index{xformslib \textit{(package)}!xformslib.fldial \textit{(module)}!xformslib.fldial.fl\_set\_dial\_step \textit{(function)}}

    \vspace{0.5ex}

\hspace{.8\funcindent}\begin{boxedminipage}{\funcwidth}

    \raggedright \textbf{fl\_set\_dial\_step}(\textit{pFlObject}, \textit{step})

    \vspace{-1.5ex}

    \rule{\textwidth}{0.5\fboxrule}
\setlength{\parskip}{2ex}

Sets the dial value to be rounded to a specified step or a
multiple of it.

-{}-
\setlength{\parskip}{1ex}
      \textbf{Parameters}
      \vspace{-1ex}

      \begin{quote}
        \begin{Ventry}{xxxxxxxxx}

          \item[pFlObject]


dial object
            {\it (type=pointer to xfdata.FL\_OBJECT)}

          \item[step]


rounding value to be set. Use 0.0 for step to switch off rounding
            {\it (type=float)}

        \end{Ventry}

      \end{quote}

\textbf{Note:} 
e.g. fl\_set\_dial\_step(dialobj, 2)


\textbf{Status:} 
Tested + Doc + NoDemo = OK


    \end{boxedminipage}

    \label{xformslib:fldial:fl_set_dial_return}
    \index{xformslib \textit{(package)}!xformslib.fldial \textit{(module)}!xformslib.fldial.fl\_set\_dial\_return \textit{(function)}}

    \vspace{0.5ex}

\hspace{.8\funcindent}\begin{boxedminipage}{\funcwidth}

    \raggedright \textbf{fl\_set\_dial\_return}(\textit{pFlObject}, \textit{when})

    \vspace{-1.5ex}

    \rule{\textwidth}{0.5\fboxrule}
\setlength{\parskip}{2ex}

Sets the conditions under which a dial object gets returned (or
its callback invoked).

-{}-
\setlength{\parskip}{1ex}
      \textbf{Parameters}
      \vspace{-1ex}

      \begin{quote}
        \begin{Ventry}{xxxxxxxxx}

          \item[pFlObject]


dial object
            {\it (type=pointer to xfdata.FL\_OBJECT)}

          \item[when]


return type (when it returns). Values (from xfdata.py) FL\_RETURN\_NONE,
FL\_RETURN\_CHANGED, FL\_RETURN\_END, FL\_RETURN\_END\_CHANGED,
FL\_RETURN\_SELECTION, FL\_RETURN\_DESELECTION, FL\_RETURN\_TRIGGERED,
FL\_RETURN\_ALWAYS
            {\it (type=int\_pos)}

        \end{Ventry}

      \end{quote}

\textbf{Note:} 
e.g. fl\_set\_dial\_return(dialobj, xfdata.FL\_RETURN\_END)


\textbf{Status:} 
Tested + Doc + NoDemo = OK


    \end{boxedminipage}

    \label{xformslib:fldial:fl_set_dial_angles}
    \index{xformslib \textit{(package)}!xformslib.fldial \textit{(module)}!xformslib.fldial.fl\_set\_dial\_angles \textit{(function)}}

    \vspace{0.5ex}

\hspace{.8\funcindent}\begin{boxedminipage}{\funcwidth}

    \raggedright \textbf{fl\_set\_dial\_angles}(\textit{pFlObject}, \textit{angmin}, \textit{angmax})

    \vspace{-1.5ex}

    \rule{\textwidth}{0.5\fboxrule}
\setlength{\parskip}{2ex}

Limits the angular range a dial can take or choose an angle other
than 0 to represent the minimum value. The angles are relative to the
origin of the dial, which is by default at 6 o'clock and rotates
clock-wise.

-{}-
\setlength{\parskip}{1ex}
      \textbf{Parameters}
      \vspace{-1ex}

      \begin{quote}
        \begin{Ventry}{xxxxxxxxx}

          \item[pFlObject]


dial object
            {\it (type=pointer to xfdata.FL\_OBJECT)}

          \item[angmin]


minimum value of angle. By default it is 0.
            {\it (type=float)}

          \item[angmax]


maximum value of angle. By default it is 360.
            {\it (type=float)}

        \end{Ventry}

      \end{quote}

\textbf{Note:} 
e.g. fl\_set\_dial\_angles(dialobj, 45, 180)


\textbf{Status:} 
Tested + Doc + Demo = OK


    \end{boxedminipage}

    \label{xformslib:fldial:fl_set_dial_cross}
    \index{xformslib \textit{(package)}!xformslib.fldial \textit{(module)}!xformslib.fldial.fl\_set\_dial\_cross \textit{(function)}}

    \vspace{0.5ex}

\hspace{.8\funcindent}\begin{boxedminipage}{\funcwidth}

    \raggedright \textbf{fl\_set\_dial\_cross}(\textit{pFlObject}, \textit{yesno})

    \vspace{-1.5ex}

    \rule{\textwidth}{0.5\fboxrule}
\setlength{\parskip}{2ex}

Allows crossing over of dial object. By default, crossing from
359.9 to 0 or from 0 to 359.9 is not allowed.

-{}-
\setlength{\parskip}{1ex}
      \textbf{Parameters}
      \vspace{-1ex}

      \begin{quote}
        \begin{Ventry}{xxxxxxxxx}

          \item[pFlObject]


dial object
            {\it (type=pointer to xfdata.FL\_OBJECT)}

          \item[yesno]


flag to enable/disable crossover. Values 1 (enabled) or 0 (disabled)
            {\it (type=int)}

        \end{Ventry}

      \end{quote}

\textbf{Note:} 
e.g. fl\_set\_dial\_cross(dialobj, 1)


\textbf{Status:} 
Tested + Doc + NoDemo = OK


    \end{boxedminipage}

    \label{xformslib:fldial:fl_set_dial_cross}
    \index{xformslib \textit{(package)}!xformslib.fldial \textit{(module)}!xformslib.fldial.fl\_set\_dial\_cross \textit{(function)}}

    \vspace{0.5ex}

\hspace{.8\funcindent}\begin{boxedminipage}{\funcwidth}

    \raggedright \textbf{fl\_set\_dial\_crossover}(\textit{pFlObject}, \textit{yesno})

    \vspace{-1.5ex}

    \rule{\textwidth}{0.5\fboxrule}
\setlength{\parskip}{2ex}

Allows crossing over of dial object. By default, crossing from
359.9 to 0 or from 0 to 359.9 is not allowed.

-{}-
\setlength{\parskip}{1ex}
      \textbf{Parameters}
      \vspace{-1ex}

      \begin{quote}
        \begin{Ventry}{xxxxxxxxx}

          \item[pFlObject]


dial object
            {\it (type=pointer to xfdata.FL\_OBJECT)}

          \item[yesno]


flag to enable/disable crossover. Values 1 (enabled) or 0 (disabled)
            {\it (type=int)}

        \end{Ventry}

      \end{quote}

\textbf{Note:} 
e.g. fl\_set\_dial\_cross(dialobj, 1)


\textbf{Status:} 
Tested + Doc + NoDemo = OK


    \end{boxedminipage}

    \label{xformslib:fldial:fl_set_dial_direction}
    \index{xformslib \textit{(package)}!xformslib.fldial \textit{(module)}!xformslib.fldial.fl\_set\_dial\_direction \textit{(function)}}

    \vspace{0.5ex}

\hspace{.8\funcindent}\begin{boxedminipage}{\funcwidth}

    \raggedright \textbf{fl\_set\_dial\_direction}(\textit{pFlObject}, \textit{directn})

    \vspace{-1.5ex}

    \rule{\textwidth}{0.5\fboxrule}
\setlength{\parskip}{2ex}

Changes what rotation modifies dial value. By default, clock-wise
rotation increases the dial value.

-{}-
\setlength{\parskip}{1ex}
      \textbf{Parameters}
      \vspace{-1ex}

      \begin{quote}
        \begin{Ventry}{xxxxxxxxx}

          \item[pFlObject]


dial object
            {\it (type=pointer to xfdata.FL\_OBJECT)}

          \item[directn]


direction of dial rotation. Values (from xfdata.py) FL\_DIAL\_CCW
(counter-clock-wise) or FL\_DIAL\_CW (clock-wise)
            {\it (type=int)}

        \end{Ventry}

      \end{quote}

\textbf{Note:} 
e.g. fl\_set\_dial\_direction(dialobj, xfdata.FL\_DIAL\_CCW)


\textbf{Status:} 
Tested + Doc + Demo = OK


    \end{boxedminipage}


%%%%%%%%%%%%%%%%%%%%%%%%%%%%%%%%%%%%%%%%%%%%%%%%%%%%%%%%%%%%%%%%%%%%%%%%%%%
%%                               Variables                               %%
%%%%%%%%%%%%%%%%%%%%%%%%%%%%%%%%%%%%%%%%%%%%%%%%%%%%%%%%%%%%%%%%%%%%%%%%%%%

  \subsection{Variables}

    \vspace{-1cm}
\hspace{\varindent}\begin{longtable}{|p{\varnamewidth}|p{\vardescrwidth}|l}
\cline{1-2}
\cline{1-2} \centering \textbf{Name} & \centering \textbf{Description}& \\
\cline{1-2}
\endhead\cline{1-2}\multicolumn{3}{r}{\small\textit{continued on next page}}\\\endfoot\cline{1-2}
\endlastfoot\raggedright \_\-\_\-p\-a\-c\-k\-a\-g\-e\-\_\-\_\- & \raggedright \textbf{Value:} 
{\tt \texttt{'}\texttt{xformslib}\texttt{'}}&\\
\cline{1-2}
\end{longtable}

    \index{xformslib \textit{(package)}!xformslib.fldial \textit{(module)}|)}
