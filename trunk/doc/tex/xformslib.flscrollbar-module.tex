%
% API Documentation for API Documentation
% Module xformslib.flscrollbar
%
% Generated by epydoc 3.0.1
% [Thu May 20 23:16:43 2010]
%

%%%%%%%%%%%%%%%%%%%%%%%%%%%%%%%%%%%%%%%%%%%%%%%%%%%%%%%%%%%%%%%%%%%%%%%%%%%
%%                          Module Description                           %%
%%%%%%%%%%%%%%%%%%%%%%%%%%%%%%%%%%%%%%%%%%%%%%%%%%%%%%%%%%%%%%%%%%%%%%%%%%%

    \index{xformslib \textit{(package)}!xformslib.flscrollbar \textit{(module)}|(}
\section{Module xformslib.flscrollbar}

    \label{xformslib:flscrollbar}

xforms-python's functions to manage scrollbar objects.

Copyright (C) 2009, 2010  Luca Lazzaroni ``LukenShiro''
e-mail: <\href{mailto:lukenshiro@ngi.it}{lukenshiro@ngi.it}>

This program is free software: you can redistribute it and/or modify
it under the terms of the GNU Lesser General Public License as
published by the Free Software Foundation, version 2.1 of the License.

This program is distributed in the hope that it will be useful,
but WITHOUT ANY WARRANTY; without even the implied warranty of
MERCHANTABILITY or FITNESS FOR A PARTICULAR PURPOSE. See the
GNU Lesser General Public License for more details.

You should have received a copy of the GNU LGPL along with this
program. If not, see <\href{http://www.gnu.org/licenses/}{http://www.gnu.org/licenses/}>.

See CREDITS file to read acknowledgements and thanks to XForms,
ctypes and other developers.

%%%%%%%%%%%%%%%%%%%%%%%%%%%%%%%%%%%%%%%%%%%%%%%%%%%%%%%%%%%%%%%%%%%%%%%%%%%
%%                               Functions                               %%
%%%%%%%%%%%%%%%%%%%%%%%%%%%%%%%%%%%%%%%%%%%%%%%%%%%%%%%%%%%%%%%%%%%%%%%%%%%

  \subsection{Functions}

    \label{xformslib:flscrollbar:fl_add_scrollbar}
    \index{xformslib \textit{(package)}!xformslib.flscrollbar \textit{(module)}!xformslib.flscrollbar.fl\_add\_scrollbar \textit{(function)}}

    \vspace{0.5ex}

\hspace{.8\funcindent}\begin{boxedminipage}{\funcwidth}

    \raggedright \textbf{fl\_add\_scrollbar}(\textit{scrolltype}, \textit{x}, \textit{y}, \textit{w}, \textit{h}, \textit{label})

    \vspace{-1.5ex}

    \rule{\textwidth}{0.5\fboxrule}
\setlength{\parskip}{2ex}

Adds a scrollbar object to a form.

-{}-
\setlength{\parskip}{1ex}
      \textbf{Parameters}
      \vspace{-1ex}

      \begin{quote}
        \begin{Ventry}{xxxxxxxxxx}

          \item[scrolltype]


type of scrollbar to be added
            {\it (type=int)}

          \item[x]


horizontal position (upper-left corner)
            {\it (type=int)}

          \item[y]


vertical position (upper-left corner)
            {\it (type=int)}

          \item[w]


width in coord units
            {\it (type=int)}

          \item[h]


height in coord units
            {\it (type=int)}

          \item[label]


label text of scrollbar
            {\it (type=str)}

        \end{Ventry}

      \end{quote}

      \textbf{Return Value}
    \vspace{-1ex}

      \begin{quote}

scrollbar object added (pFlObject)
      {\it (type=pointer to xfdata.FL\_OBJECT)}

      \end{quote}

\textbf{Note:} 
e.g. \emph{todo}


\textbf{Status:} 
Tested + NoDoc + Demo = OK


    \end{boxedminipage}

    \label{xformslib:flscrollbar:fl_get_scrollbar_value}
    \index{xformslib \textit{(package)}!xformslib.flscrollbar \textit{(module)}!xformslib.flscrollbar.fl\_get\_scrollbar\_value \textit{(function)}}

    \vspace{0.5ex}

\hspace{.8\funcindent}\begin{boxedminipage}{\funcwidth}

    \raggedright \textbf{fl\_get\_scrollbar\_value}(\textit{pFlObject})

    \vspace{-1.5ex}

    \rule{\textwidth}{0.5\fboxrule}
\setlength{\parskip}{2ex}

Obtains the current value of a scrollbar object.

-{}-
\setlength{\parskip}{1ex}
      \textbf{Parameters}
      \vspace{-1ex}

      \begin{quote}
        \begin{Ventry}{xxxxxxxxx}

          \item[pFlObject]


scrollbar object
            {\it (type=pointer to xfdata.FL\_OBJECT)}

        \end{Ventry}

      \end{quote}

      \textbf{Return Value}
    \vspace{-1ex}

      \begin{quote}

scrollbar value
      {\it (type=float)}

      \end{quote}

\textbf{Note:} 
e.g. \emph{todo}


\textbf{Status:} 
Untested + NoDoc + NoDemo = NOT OK


    \end{boxedminipage}

    \label{xformslib:flscrollbar:fl_set_scrollbar_value}
    \index{xformslib \textit{(package)}!xformslib.flscrollbar \textit{(module)}!xformslib.flscrollbar.fl\_set\_scrollbar\_value \textit{(function)}}

    \vspace{0.5ex}

\hspace{.8\funcindent}\begin{boxedminipage}{\funcwidth}

    \raggedright \textbf{fl\_set\_scrollbar\_value}(\textit{pFlObject}, \textit{val})

    \vspace{-1.5ex}

    \rule{\textwidth}{0.5\fboxrule}
\setlength{\parskip}{2ex}

Sets the value of a scrollbar object.

-{}-
\setlength{\parskip}{1ex}
      \textbf{Parameters}
      \vspace{-1ex}

      \begin{quote}
        \begin{Ventry}{xxxxxxxxx}

          \item[pFlObject]


scrollbar object
            {\it (type=pointer to xfdata.FL\_OBJECT)}

          \item[val]


value of the scrollbar to be set. By default it's 0.5.
            {\it (type=float)}

        \end{Ventry}

      \end{quote}

\textbf{Note:} 
e.g. \emph{todo}


\textbf{Status:} 
Tested + NoDoc + Demo = OK


    \end{boxedminipage}

    \label{xformslib:flscrollbar:fl_set_scrollbar_size}
    \index{xformslib \textit{(package)}!xformslib.flscrollbar \textit{(module)}!xformslib.flscrollbar.fl\_set\_scrollbar\_size \textit{(function)}}

    \vspace{0.5ex}

\hspace{.8\funcindent}\begin{boxedminipage}{\funcwidth}

    \raggedright \textbf{fl\_set\_scrollbar\_size}(\textit{pFlObject}, \textit{val})

    \vspace{-1.5ex}

    \rule{\textwidth}{0.5\fboxrule}
\setlength{\parskip}{2ex}

Controls the size of the sliding bar inside the box. This function
does nothing if applied to scrollbars of type xfdata.FL\_NICE\_SCROLLBAR.

-{}-
\setlength{\parskip}{1ex}
      \textbf{Parameters}
      \vspace{-1ex}

      \begin{quote}
        \begin{Ventry}{xxxxxxxxx}

          \item[pFlObject]


scrollbar object
            {\it (type=pointer to xfdata.FL\_OBJECT)}

          \item[val]


size should be a value between 0.0 and 1.0. The default is
xfdata.FL\_SLIDER\_WIDTH (which is 0.15 for all scrollbars). If it's 1.0,
the scrollbar covers the box completely and can no longer be moved.
            {\it (type=float)}

        \end{Ventry}

      \end{quote}

\textbf{Note:} 
e.g. \emph{todo}


\textbf{Status:} 
Tested + NoDoc + Demo = OK


    \end{boxedminipage}

    \label{xformslib:flscrollbar:fl_set_scrollbar_increment}
    \index{xformslib \textit{(package)}!xformslib.flscrollbar \textit{(module)}!xformslib.flscrollbar.fl\_set\_scrollbar\_increment \textit{(function)}}

    \vspace{0.5ex}

\hspace{.8\funcindent}\begin{boxedminipage}{\funcwidth}

    \raggedright \textbf{fl\_set\_scrollbar\_increment}(\textit{pFlObject}, \textit{leftbtnval}, \textit{midlbtnval})

    \vspace{-1.5ex}

    \rule{\textwidth}{0.5\fboxrule}
\setlength{\parskip}{2ex}

Sets the size of the steps of a scrollbar jump. By default, if the
mouse is pressed beside the the sliding bar, the bar starts to jumps in
the direction of the mouse position.

-{}-
\setlength{\parskip}{1ex}
      \textbf{Parameters}
      \vspace{-1ex}

      \begin{quote}
        \begin{Ventry}{xxxxxxxxxx}

          \item[pFlObject]


scrollbar object
            {\it (type=pointer to xfdata.FL\_OBJECT)}

          \item[leftbtnval]


value to increment if the left mouse button is pressed
            {\it (type=float)}

          \item[midlbtnval]


value to increment if the middle mouse button is pressed
            {\it (type=float)}

        \end{Ventry}

      \end{quote}

\textbf{Note:} 
e.g. \emph{todo}


\textbf{Status:} 
Untested + NoDoc + NoDemo = NOT OK


    \end{boxedminipage}

    \label{xformslib:flscrollbar:fl_get_scrollbar_increment}
    \index{xformslib \textit{(package)}!xformslib.flscrollbar \textit{(module)}!xformslib.flscrollbar.fl\_get\_scrollbar\_increment \textit{(function)}}

    \vspace{0.5ex}

\hspace{.8\funcindent}\begin{boxedminipage}{\funcwidth}

    \raggedright \textbf{fl\_get\_scrollbar\_increment}(\textit{pFlObject})

    \vspace{-1.5ex}

    \rule{\textwidth}{0.5\fboxrule}
\setlength{\parskip}{2ex}

Obtains the increment of size of a scrollbar for left and middle mouse
buttons.

-{}-
\setlength{\parskip}{1ex}
      \textbf{Parameters}
      \vspace{-1ex}

      \begin{quote}
        \begin{Ventry}{xxxxxxxxx}

          \item[pFlObject]


scrollbar object
            {\it (type=pointer to xfdata.FL\_OBJECT)}

        \end{Ventry}

      \end{quote}

      \textbf{Return Value}
    \vspace{-1ex}

      \begin{quote}

left button increment, middle button increment
      {\it (type=float, float)}

      \end{quote}

\textbf{Note:} 
e.g. \emph{todo}


\textbf{Attention:} 
API change from XForms - upstream was
fl\_get\_scrollbar\_increment(pFlObject, leftbtnval, valmidlbtnval)


\textbf{Status:} 
Untested + NoDoc + NoDemo = NOT OK


    \end{boxedminipage}

    \label{xformslib:flscrollbar:fl_set_scrollbar_bounds}
    \index{xformslib \textit{(package)}!xformslib.flscrollbar \textit{(module)}!xformslib.flscrollbar.fl\_set\_scrollbar\_bounds \textit{(function)}}

    \vspace{0.5ex}

\hspace{.8\funcindent}\begin{boxedminipage}{\funcwidth}

    \raggedright \textbf{fl\_set\_scrollbar\_bounds}(\textit{pFlObject}, \textit{minbound}, \textit{maxbound})

    \vspace{-1.5ex}

    \rule{\textwidth}{0.5\fboxrule}
\setlength{\parskip}{2ex}

Sets minimum and maximum the bounds/limits of a scrollbar object. For
vertical sliders the slider position for the minimum value is at the left,
for horizontal sliders at the top of the slider. By setting min to a
larger value than max in a call of fl\_set\_scrollbar\_bounds() this can be
reversed. When this function is called, if the actual value of a scrollbar
isn't within the range of the new bounds, its value gets adjusted to the
nearest limit.

-{}-
\setlength{\parskip}{1ex}
      \textbf{Parameters}
      \vspace{-1ex}

      \begin{quote}
        \begin{Ventry}{xxxxxxxxx}

          \item[pFlObject]


scrollbar object
            {\it (type=pointer to xfdata.FL\_OBJECT)}

          \item[minbound]


minimum bound of scrollbar. By default, the minimum value
for a slider is 0.0.
            {\it (type=float)}

          \item[maxbound]


maximum bound of scrollbar. By default, the maximum value
for a slider is 1.0.
            {\it (type=float)}

        \end{Ventry}

      \end{quote}

\textbf{Note:} 
e.g. \emph{todo}


\textbf{Status:} 
Untested + NoDoc + NoDemo = NOT OK


    \end{boxedminipage}

    \label{xformslib:flscrollbar:fl_get_scrollbar_bounds}
    \index{xformslib \textit{(package)}!xformslib.flscrollbar \textit{(module)}!xformslib.flscrollbar.fl\_get\_scrollbar\_bounds \textit{(function)}}

    \vspace{0.5ex}

\hspace{.8\funcindent}\begin{boxedminipage}{\funcwidth}

    \raggedright \textbf{fl\_get\_scrollbar\_bounds}(\textit{pFlObject})

    \vspace{-1.5ex}

    \rule{\textwidth}{0.5\fboxrule}
\setlength{\parskip}{2ex}

Returns minimum and maximum bounds/limits of a scrollbar object.

-{}-
\setlength{\parskip}{1ex}
      \textbf{Parameters}
      \vspace{-1ex}

      \begin{quote}
        \begin{Ventry}{xxxxxxxxx}

          \item[pFlObject]


scrollbar object
            {\it (type=pointer to xfdata.FL\_OBJECT)}

        \end{Ventry}

      \end{quote}

      \textbf{Return Value}
    \vspace{-1ex}

      \begin{quote}

minimum bound, maximum bound
      {\it (type=float, float)}

      \end{quote}

\textbf{Note:} 
e.g. \emph{todo}


\textbf{Attention:} 
API change from XForms - upstream was
fl\_get\_scrollbar\_bounds(pFlObject, b1, b2)


\textbf{Status:} 
Untested + NoDoc + NoDemo = NOT OK


    \end{boxedminipage}

    \label{xformslib:flscrollbar:fl_set_scrollbar_step}
    \index{xformslib \textit{(package)}!xformslib.flscrollbar \textit{(module)}!xformslib.flscrollbar.fl\_set\_scrollbar\_step \textit{(function)}}

    \vspace{0.5ex}

\hspace{.8\funcindent}\begin{boxedminipage}{\funcwidth}

    \raggedright \textbf{fl\_set\_scrollbar\_step}(\textit{pFlObject}, \textit{step})

    \vspace{-1.5ex}

    \rule{\textwidth}{0.5\fboxrule}
\setlength{\parskip}{2ex}

Sets the step size of a scrollbar to which values are rounded.

-{}-
\setlength{\parskip}{1ex}
      \textbf{Parameters}
      \vspace{-1ex}

      \begin{quote}
        \begin{Ventry}{xxxxxxxxx}

          \item[pFlObject]


scrollbar object
            {\it (type=pointer to xfdata.FL\_OBJECT)}

          \item[step]


rounded value.
            {\it (type=float)}

        \end{Ventry}

      \end{quote}

\textbf{Note:} 
e.g. \emph{todo}


\textbf{Status:} 
Untested + NoDoc + NoDemo = NOT OK


    \end{boxedminipage}


%%%%%%%%%%%%%%%%%%%%%%%%%%%%%%%%%%%%%%%%%%%%%%%%%%%%%%%%%%%%%%%%%%%%%%%%%%%
%%                               Variables                               %%
%%%%%%%%%%%%%%%%%%%%%%%%%%%%%%%%%%%%%%%%%%%%%%%%%%%%%%%%%%%%%%%%%%%%%%%%%%%

  \subsection{Variables}

    \vspace{-1cm}
\hspace{\varindent}\begin{longtable}{|p{\varnamewidth}|p{\vardescrwidth}|l}
\cline{1-2}
\cline{1-2} \centering \textbf{Name} & \centering \textbf{Description}& \\
\cline{1-2}
\endhead\cline{1-2}\multicolumn{3}{r}{\small\textit{continued on next page}}\\\endfoot\cline{1-2}
\endlastfoot\raggedright \_\-\_\-p\-a\-c\-k\-a\-g\-e\-\_\-\_\- & \raggedright \textbf{Value:} 
{\tt \texttt{'}\texttt{xformslib}\texttt{'}}&\\
\cline{1-2}
\end{longtable}

    \index{xformslib \textit{(package)}!xformslib.flscrollbar \textit{(module)}|)}
