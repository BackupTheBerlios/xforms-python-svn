%
% API Documentation for xforms-python
% Module xformslib.flgoodies
%
% Generated by epydoc 3.0.1
% [Mon May 24 11:23:23 2010]
%

%%%%%%%%%%%%%%%%%%%%%%%%%%%%%%%%%%%%%%%%%%%%%%%%%%%%%%%%%%%%%%%%%%%%%%%%%%%
%%                          Module Description                           %%
%%%%%%%%%%%%%%%%%%%%%%%%%%%%%%%%%%%%%%%%%%%%%%%%%%%%%%%%%%%%%%%%%%%%%%%%%%%

    \index{xformslib \textit{(package)}!xformslib.flgoodies \textit{(module)}|(}
\section{Module xformslib.flgoodies}

    \label{xformslib:flgoodies}

xforms-python's functions to manage goodies objects.

Copyright (C) 2009, 2010  Luca Lazzaroni ``LukenShiro''
e-mail: <\href{mailto:lukenshiro@ngi.it}{lukenshiro@ngi.it}>

This program is free software: you can redistribute it and/or modify
it under the terms of the GNU Lesser General Public License as
published by the Free Software Foundation, version 2.1 of the License.

This program is distributed in the hope that it will be useful,
but WITHOUT ANY WARRANTY; without even the implied warranty of
MERCHANTABILITY or FITNESS FOR A PARTICULAR PURPOSE. See the
GNU Lesser General Public License for more details.

You should have received a copy of the GNU LGPL along with this
program. If not, see <\href{http://www.gnu.org/licenses/}{http://www.gnu.org/licenses/}>.

See CREDITS file to read acknowledgements and thanks to XForms,
ctypes and other developers.

%%%%%%%%%%%%%%%%%%%%%%%%%%%%%%%%%%%%%%%%%%%%%%%%%%%%%%%%%%%%%%%%%%%%%%%%%%%
%%                               Functions                               %%
%%%%%%%%%%%%%%%%%%%%%%%%%%%%%%%%%%%%%%%%%%%%%%%%%%%%%%%%%%%%%%%%%%%%%%%%%%%

  \subsection{Functions}

    \label{xformslib:flgoodies:fl_set_goodies_font}
    \index{xformslib \textit{(package)}!xformslib.flgoodies \textit{(module)}!xformslib.flgoodies.fl\_set\_goodies\_font \textit{(function)}}

    \vspace{0.5ex}

\hspace{.8\funcindent}\begin{boxedminipage}{\funcwidth}

    \raggedright \textbf{fl\_set\_goodies\_font}(\textit{style}, \textit{size})

    \vspace{-1.5ex}

    \rule{\textwidth}{0.5\fboxrule}
\setlength{\parskip}{2ex}

Changes the font used in all messages.

-{}-
\setlength{\parskip}{1ex}
      \textbf{Parameters}
      \vspace{-1ex}

      \begin{quote}
        \begin{Ventry}{xxxxx}

          \item[style]


goodies style. Values (from xfdata.py) FL\_NORMAL\_STYLE, FL\_BOLD\_STYLE,
FL\_ITALIC\_STYLE, FL\_BOLDITALIC\_STYLE, FL\_FIXED\_STYLE,
FL\_FIXEDBOLD\_STYLE, FL\_FIXEDITALIC\_STYLE, FL\_FIXEDBOLDITALIC\_STYLE,
FL\_TIMES\_STYLE, FL\_TIMESBOLD\_STYLE, FL\_TIMESITALIC\_STYLE,
FL\_TIMESBOLDITALIC\_STYLE, FL\_MISC\_STYLE, FL\_MISCBOLD\_STYLE,
FL\_MISCITALIC\_STYLE, FL\_SYMBOL\_STYLE, FL\_SHADOW\_STYLE,
FL\_ENGRAVED\_STYLE, FL\_EMBOSSED\_STYLE
            {\it (type=int)}

          \item[size]


goodies size. Values (from xfdata.py) FL\_TINY\_SIZE, FL\_SMALL\_SIZE,
FL\_NORMAL\_SIZE, FL\_MEDIUM\_SIZE, FL\_LARGE\_SIZE, FL\_HUGE\_SIZE,
FL\_DEFAULT\_SIZE
            {\it (type=int)}

        \end{Ventry}

      \end{quote}

\textbf{Note:} 
e.g. fl\_set\_goodies\_font(xfdata.FL\_BOLD\_STYLE,         xfdata.FL\_MEDIUM\_SIZE)


\textbf{Status:} 
Tested + Doc + Demo = OK


    \end{boxedminipage}

    \label{xformslib:flgoodies:fl_show_message}
    \index{xformslib \textit{(package)}!xformslib.flgoodies \textit{(module)}!xformslib.flgoodies.fl\_show\_message \textit{(function)}}

    \vspace{0.5ex}

\hspace{.8\funcindent}\begin{boxedminipage}{\funcwidth}

    \raggedright \textbf{fl\_show\_message}(\textit{msgtxt1}, \textit{msgtxt2}, \textit{msgtxt3})

    \vspace{-1.5ex}

    \rule{\textwidth}{0.5\fboxrule}
\setlength{\parskip}{2ex}

Shows a simple form with three lines of text and a button labeled
OK on it. The mouse pointer is on the button.

-{}-
\setlength{\parskip}{1ex}
      \textbf{Parameters}
      \vspace{-1ex}

      \begin{quote}
        \begin{Ventry}{xxxxxxx}

          \item[msgtxt1]


first message to show
            {\it (type=str)}

          \item[msgtxt2]


second message to show
            {\it (type=str)}

          \item[msgtxt3]


third message to show
            {\it (type=str)}

        \end{Ventry}

      \end{quote}

\textbf{Note:} 
e.g. fl\_show\_message(``first message'', ``second message'',
``third message'')


\textbf{Status:} 
Tested + Doc + Demo = OK


    \end{boxedminipage}

    \label{xformslib:flgoodies:fl_show_messages}
    \index{xformslib \textit{(package)}!xformslib.flgoodies \textit{(module)}!xformslib.flgoodies.fl\_show\_messages \textit{(function)}}

    \vspace{0.5ex}

\hspace{.8\funcindent}\begin{boxedminipage}{\funcwidth}

    \raggedright \textbf{fl\_show\_messages}(\textit{msgtxt})

    \vspace{-1.5ex}

    \rule{\textwidth}{0.5\fboxrule}
\setlength{\parskip}{2ex}

Shows a message. You can use it with a single line or when you
know the message in advance. To get multi-line messages use embedded
newlines. It blocks execution and does not return immediately (but idle
callback and asynchronous IO continue being run and checked). Execution
continues when the OK button is pressed or <Return> is hit or when the
message form is removed from the screen by fl\_hide\_message().

-{}-
\setlength{\parskip}{1ex}
      \textbf{Parameters}
      \vspace{-1ex}

      \begin{quote}
        \begin{Ventry}{xxxxxx}

          \item[msgtxt]


message to show
            {\it (type=str)}

        \end{Ventry}

      \end{quote}

\textbf{Note:} 
e.g. fl\_show\_messages(``Some messages'')


\textbf{Status:} 
Tested + Doc + Demo = OK


    \end{boxedminipage}

    \label{xformslib:flgoodies:fl_show_msg}
    \index{xformslib \textit{(package)}!xformslib.flgoodies \textit{(module)}!xformslib.flgoodies.fl\_show\_msg \textit{(function)}}

    \vspace{0.5ex}

\hspace{.8\funcindent}\begin{boxedminipage}{\funcwidth}

    \raggedright \textbf{fl\_show\_msg}(\textit{fmttxt})

    \vspace{-1.5ex}

    \rule{\textwidth}{0.5\fboxrule}
\setlength{\parskip}{2ex}

Shows a formatted text message. The string resulting from expansion
of the format string using the remaining arguments can have arbitrary
length and embedded newline characters, producing line breaks. The size
of the message box gets set in a way that the whole text fits into it.
It blocks execution and does not return immediately (but idle callback
and asynchronous IO continue being run and checked). Execution continues
when the OK button is pressed or <Return> is hit or when the message
form is removed from the screen by fl\_hide\_message().

-{}-
\setlength{\parskip}{1ex}
      \textbf{Parameters}
      \vspace{-1ex}

      \begin{quote}
        \begin{Ventry}{xxxxxx}

          \item[fmttxt]


the message to show (with format parameters, e.g. \%s, \%d, \%f etc..)
            {\it (type=str)}

        \end{Ventry}

      \end{quote}

\textbf{Note:} 
e.g. fl\_show\_msg(``formatted text \%s \%d'' \% (mystr, myval))


\textbf{Status:} 
Tested + Doc + NoDemo = OK


    \end{boxedminipage}

    \label{xformslib:flgoodies:fl_hide_message}
    \index{xformslib \textit{(package)}!xformslib.flgoodies \textit{(module)}!xformslib.flgoodies.fl\_hide\_message \textit{(function)}}

    \vspace{0.5ex}

\hspace{.8\funcindent}\begin{boxedminipage}{\funcwidth}

    \raggedright \textbf{fl\_hide\_message}()

    \vspace{-1.5ex}

    \rule{\textwidth}{0.5\fboxrule}
\setlength{\parskip}{2ex}

Hides a text message already shown.

-{}-
\setlength{\parskip}{1ex}
\textbf{Note:} 
e.g. fl\_hide\_message()


\textbf{Status:} 
Tested + Doc + NoDemo = OK


    \end{boxedminipage}

    \label{xformslib:flgoodies:fl_hide_message}
    \index{xformslib \textit{(package)}!xformslib.flgoodies \textit{(module)}!xformslib.flgoodies.fl\_hide\_message \textit{(function)}}

    \vspace{0.5ex}

\hspace{.8\funcindent}\begin{boxedminipage}{\funcwidth}

    \raggedright \textbf{fl\_hide\_msg}()

    \vspace{-1.5ex}

    \rule{\textwidth}{0.5\fboxrule}
\setlength{\parskip}{2ex}

Hides a text message already shown.

-{}-
\setlength{\parskip}{1ex}
\textbf{Note:} 
e.g. fl\_hide\_message()


\textbf{Status:} 
Tested + Doc + NoDemo = OK


    \end{boxedminipage}

    \label{xformslib:flgoodies:fl_hide_message}
    \index{xformslib \textit{(package)}!xformslib.flgoodies \textit{(module)}!xformslib.flgoodies.fl\_hide\_message \textit{(function)}}

    \vspace{0.5ex}

\hspace{.8\funcindent}\begin{boxedminipage}{\funcwidth}

    \raggedright \textbf{fl\_hide\_messages}()

    \vspace{-1.5ex}

    \rule{\textwidth}{0.5\fboxrule}
\setlength{\parskip}{2ex}

Hides a text message already shown.

-{}-
\setlength{\parskip}{1ex}
\textbf{Note:} 
e.g. fl\_hide\_message()


\textbf{Status:} 
Tested + Doc + NoDemo = OK


    \end{boxedminipage}

    \label{xformslib:flgoodies:fl_show_question}
    \index{xformslib \textit{(package)}!xformslib.flgoodies \textit{(module)}!xformslib.flgoodies.fl\_show\_question \textit{(function)}}

    \vspace{0.5ex}

\hspace{.8\funcindent}\begin{boxedminipage}{\funcwidth}

    \raggedright \textbf{fl\_show\_question}(\textit{questmsg}, \textit{defbtn})

    \vspace{-1.5ex}

    \rule{\textwidth}{0.5\fboxrule}
\setlength{\parskip}{2ex}

Shows a message (with possible embedded newlines in it) with a Yes
and a No button. It returns whether the user pushed the Yes button. The
user can also press the <Y> key to mean Yes and the <N> key to mean No.

-{}-
\setlength{\parskip}{1ex}
      \textbf{Parameters}
      \vspace{-1ex}

      \begin{quote}
        \begin{Ventry}{xxxxxxxx}

          \item[questmsg]


text of question message to show
            {\it (type=str)}

          \item[defbtn]


which button the mouse pointer should be on. Values 1  (for Yes) or 0
(for No) and any other value causes the form to be shown so the mouse
pointer is at the center of the form.
            {\it (type=int)}

        \end{Ventry}

      \end{quote}

      \textbf{Return Value}
    \vspace{-1ex}

      \begin{quote}

1 (if Yes button pushed) or 0 otherwise
      {\it (type=int)}

      \end{quote}

\textbf{Note:} 
e.g. qresp = fl\_show\_question(``My question?'', 1)


\textbf{Status:} 
Tested + Doc + Demo = OK


    \end{boxedminipage}

    \label{xformslib:flgoodies:fl_hide_question}
    \index{xformslib \textit{(package)}!xformslib.flgoodies \textit{(module)}!xformslib.flgoodies.fl\_hide\_question \textit{(function)}}

    \vspace{0.5ex}

\hspace{.8\funcindent}\begin{boxedminipage}{\funcwidth}

    \raggedright \textbf{fl\_hide\_question}()

    \vspace{-1.5ex}

    \rule{\textwidth}{0.5\fboxrule}
\setlength{\parskip}{2ex}

Hides a question message already shown.

-{}-
\setlength{\parskip}{1ex}
\textbf{Note:} 
e.g. fl\_hide\_question()


\textbf{Status:} 
Tested + Doc + NoDemo = OK


    \end{boxedminipage}

    \label{xformslib:flgoodies:fl_show_alert}
    \index{xformslib \textit{(package)}!xformslib.flgoodies \textit{(module)}!xformslib.flgoodies.fl\_show\_alert \textit{(function)}}

    \vspace{0.5ex}

\hspace{.8\funcindent}\begin{boxedminipage}{\funcwidth}

    \raggedright \textbf{fl\_show\_alert}(\textit{title}, \textit{msg1}, \textit{msg2}, \textit{centered})

    \vspace{-1.5ex}

    \rule{\textwidth}{0.5\fboxrule}
\setlength{\parskip}{2ex}

Shows an alert message, with an alert icon (!) is added and the
first string is shown bold-faced.

-{}-
\setlength{\parskip}{1ex}
      \textbf{Parameters}
      \vspace{-1ex}

      \begin{quote}
        \begin{Ventry}{xxxxxxxx}

          \item[title]


title of alert
            {\it (type=str)}

          \item[msg1]


first message text
            {\it (type=str)}

          \item[msg2]


other message text
            {\it (type=str)}

          \item[centered]


if alert has to be displayed centered on the screen or not. Values 1
(if centered) or 0 (not centered)
            {\it (type=int)}

        \end{Ventry}

      \end{quote}

\textbf{Note:} 
e.g. fl\_show\_alert(``My title'', ``first text'', ``second text'', 1)


\textbf{Status:} 
Tested + Doc + Demo = OK


    \end{boxedminipage}

    \label{xformslib:flgoodies:fl_show_alert2}
    \index{xformslib \textit{(package)}!xformslib.flgoodies \textit{(module)}!xformslib.flgoodies.fl\_show\_alert2 \textit{(function)}}

    \vspace{0.5ex}

\hspace{.8\funcindent}\begin{boxedminipage}{\funcwidth}

    \raggedright \textbf{fl\_show\_alert2}(\textit{centered}, \textit{fmttxt})

    \vspace{-1.5ex}

    \rule{\textwidth}{0.5\fboxrule}
\setlength{\parskip}{2ex}

Shows a formatted alert message. The string resulting from expansion
of the format string using the rest of the arguments can have arbitrary
length and the first embedded form-feed character (backslash-f) is used
as the separator between the title string and the message of the alert
box. Embedded newline characters produce lines break.

-{}-
\setlength{\parskip}{1ex}
      \textbf{Parameters}
      \vspace{-1ex}

      \begin{quote}
        \begin{Ventry}{xxxxxxxx}

          \item[centered]


if alert has to be displayed centered on the screen or not. Values 1
(if centered) or 0 (not centered)
            {\it (type=int)}

          \item[fmttxt]


formatted message text
            {\it (type=str)}

        \end{Ventry}

      \end{quote}

\textbf{Note:} 
e.g. fl\_show\_alert2(1, ``formatted text \%s \%d'' \% (mystr, myval))


\textbf{Status:} 
Tested + Doc + NoDemo = OK


    \end{boxedminipage}

    \label{xformslib:flgoodies:fl_hide_alert}
    \index{xformslib \textit{(package)}!xformslib.flgoodies \textit{(module)}!xformslib.flgoodies.fl\_hide\_alert \textit{(function)}}

    \vspace{0.5ex}

\hspace{.8\funcindent}\begin{boxedminipage}{\funcwidth}

    \raggedright \textbf{fl\_hide\_alert}()

    \vspace{-1.5ex}

    \rule{\textwidth}{0.5\fboxrule}
\setlength{\parskip}{2ex}

Hides a previously shown alert message.

-{}-
\setlength{\parskip}{1ex}
\textbf{Note:} 
e.g. fl\_hide\_alert()


\textbf{Status:} 
Tested + Doc + Demo = OK


    \end{boxedminipage}

    \label{xformslib:flgoodies:fl_show_input}
    \index{xformslib \textit{(package)}!xformslib.flgoodies \textit{(module)}!xformslib.flgoodies.fl\_show\_input \textit{(function)}}

    \vspace{0.5ex}

\hspace{.8\funcindent}\begin{boxedminipage}{\funcwidth}

    \raggedright \textbf{fl\_show\_input}(\textit{msgtxt}, \textit{defstr})

    \vspace{-1.5ex}

    \rule{\textwidth}{0.5\fboxrule}
\setlength{\parskip}{2ex}

Obtains some text from user, showing a default text. It has OK
and Cancel buttons.

-{}-
\setlength{\parskip}{1ex}
      \textbf{Parameters}
      \vspace{-1ex}

      \begin{quote}
        \begin{Ventry}{xxxxxx}

          \item[msgtxt]


text used to ask for input
            {\it (type=str)}

          \item[defstr]


default user answer to show
            {\it (type=str)}

        \end{Ventry}

      \end{quote}

      \textbf{Return Value}
    \vspace{-1ex}

      \begin{quote}

text inserted by user
      {\it (type=str)}

      \end{quote}

\textbf{Note:} 
e.g. inpstr = fl\_show\_input(``Insert number of eggs: '', ``None'')


\textbf{Status:} 
Tested + Doc + Demo = OK


    \end{boxedminipage}

    \label{xformslib:flgoodies:fl_hide_input}
    \index{xformslib \textit{(package)}!xformslib.flgoodies \textit{(module)}!xformslib.flgoodies.fl\_hide\_input \textit{(function)}}

    \vspace{0.5ex}

\hspace{.8\funcindent}\begin{boxedminipage}{\funcwidth}

    \raggedright \textbf{fl\_hide\_input}()

    \vspace{-1.5ex}

    \rule{\textwidth}{0.5\fboxrule}
\setlength{\parskip}{2ex}

Hides a previously shown input object.

-{}-
\setlength{\parskip}{1ex}
\textbf{Note:} 
e.g. fl\_hide\_input()


\textbf{Status:} 
Tested + Doc + NoDemo = OK


    \end{boxedminipage}

    \label{xformslib:flgoodies:fl_show_simple_input}
    \index{xformslib \textit{(package)}!xformslib.flgoodies \textit{(module)}!xformslib.flgoodies.fl\_show\_simple\_input \textit{(function)}}

    \vspace{0.5ex}

\hspace{.8\funcindent}\begin{boxedminipage}{\funcwidth}

    \raggedright \textbf{fl\_show\_simple\_input}(\textit{msgtxt}, \textit{defstr})

    \vspace{-1.5ex}

    \rule{\textwidth}{0.5\fboxrule}
\setlength{\parskip}{2ex}

Asks the user for textual input. It has an OK button only.

-{}-
\setlength{\parskip}{1ex}
      \textbf{Parameters}
      \vspace{-1ex}

      \begin{quote}
        \begin{Ventry}{xxxxxx}

          \item[msgtxt]


message used to ask for input
            {\it (type=str)}

          \item[defstr]


default user answer in input
            {\it (type=str)}

        \end{Ventry}

      \end{quote}

      \textbf{Return Value}
    \vspace{-1ex}

      \begin{quote}

text inserted by user
      {\it (type=str)}

      \end{quote}

\textbf{Note:} 
e.g. inpstr = fl\_show\_simple\_input(``Insert name and surname:'',
``John Doe'')


\textbf{Status:} 
Tested + Doc + NoDemo = OK


    \end{boxedminipage}

    \label{xformslib:flgoodies:fl_show_colormap}
    \index{xformslib \textit{(package)}!xformslib.flgoodies \textit{(module)}!xformslib.flgoodies.fl\_show\_colormap \textit{(function)}}

    \vspace{0.5ex}

\hspace{.8\funcindent}\begin{boxedminipage}{\funcwidth}

    \raggedright \textbf{fl\_show\_colormap}(\textit{oldcolr})

    \vspace{-1.5ex}

    \rule{\textwidth}{0.5\fboxrule}
\setlength{\parskip}{2ex}

Shows a colormap color selector from which the user can select a
color. The user can decide not to change this color by pressing the
Cancel button in the form. In a number of applications the user has
to select a color from the colormap. For this a goody has been created.
It shows the first 64 entries of the colormap. The user can scroll
through the colormap to see more entries. Once the user presses the
mouse one of the entries the corresponding index is returned and the
colormap is removed from the screen.

-{}-
\setlength{\parskip}{1ex}
      \textbf{Parameters}
      \vspace{-1ex}

      \begin{quote}
        \begin{Ventry}{xxxxxxx}

          \item[oldcolr]


current or default color num. (Not xfdata.FL\_COLOR)
            {\it (type=int)}

        \end{Ventry}

      \end{quote}

      \textbf{Return Value}
    \vspace{-1ex}

      \begin{quote}

index of the color selected (or the index of the old color)
      {\it (type=int)}

      \end{quote}

\textbf{Note:} 
e.g. colridx = fl\_show\_colormap(xfdata.Fl\_YELLOWGREEN)


\textbf{Status:} 
Tested + Doc + Demo = OK


    \end{boxedminipage}

    \label{xformslib:flgoodies:fl_show_choices}
    \index{xformslib \textit{(package)}!xformslib.flgoodies \textit{(module)}!xformslib.flgoodies.fl\_show\_choices \textit{(function)}}

    \vspace{0.5ex}

\hspace{.8\funcindent}\begin{boxedminipage}{\funcwidth}

    \raggedright \textbf{fl\_show\_choices}(\textit{msgtxt}, \textit{numb}, \textit{btn1txt}, \textit{btn2txt}, \textit{btn3txt}, \textit{defcho})

    \vspace{-1.5ex}

    \rule{\textwidth}{0.5\fboxrule}
\setlength{\parskip}{2ex}

Shows a message, as a single string with possible embedded newlines,
with one, two or three buttons. The user can also press the <1>, <2> or
<3> key to indicate the first, second, or third button.

-{}-
\setlength{\parskip}{1ex}
      \textbf{Parameters}
      \vspace{-1ex}

      \begin{quote}
        \begin{Ventry}{xxxxxxx}

          \item[msgtxt]


message text
            {\it (type=str)}

          \item[numb]


number of buttons
            {\it (type=int)}

          \item[btn1txt]


label of first button from the left
            {\it (type=str)}

          \item[btn2txt]


label of second button from the left
            {\it (type=str)}

          \item[btn3txt]


label of first button from the right
            {\it (type=str)}

          \item[defcho]


default choice (1, 2 or 3)
            {\it (type=int)}

        \end{Ventry}

      \end{quote}

      \textbf{Return Value}
    \vspace{-1ex}

      \begin{quote}

number of the button pressed (1, 2 or 3)
      {\it (type=int)}

      \end{quote}

\textbf{Note:} 
e.g. pressbtn = fl\_show\_choices(``some message'', 3, ``1st'', ``2nd'',
``3rd'', 1)


\textbf{Status:} 
Tested + Doc + Demo = OK


    \end{boxedminipage}

    \label{xformslib:flgoodies:fl_show_choice}
    \index{xformslib \textit{(package)}!xformslib.flgoodies \textit{(module)}!xformslib.flgoodies.fl\_show\_choice \textit{(function)}}

    \vspace{0.5ex}

\hspace{.8\funcindent}\begin{boxedminipage}{\funcwidth}

    \raggedright \textbf{fl\_show\_choice}(\textit{msg1txt}, \textit{msg2txt}, \textit{msg3txt}, \textit{numb}, \textit{btn1txt}, \textit{btn2txt}, \textit{btn3txt}, \textit{defcho})

    \vspace{-1.5ex}

    \rule{\textwidth}{0.5\fboxrule}
\setlength{\parskip}{2ex}

Shows a message, up to three lines, with one, two or three buttons.
The user can also press the <1>, <2> or <3> key to indicate the first,
second, or third button.

-{}-
\setlength{\parskip}{1ex}
      \textbf{Parameters}
      \vspace{-1ex}

      \begin{quote}
        \begin{Ventry}{xxxxxxx}

          \item[msg1txt]


first message text
            {\it (type=str)}

          \item[msg2txt]


second message text
            {\it (type=str)}

          \item[msg3txt]


third message text
            {\it (type=str)}

          \item[numb]


number of buttons
            {\it (type=int)}

          \item[btn1txt]


label of first button from the left
            {\it (type=str)}

          \item[btn2txt]


label of second button from the left
            {\it (type=str)}

          \item[btn3txt]


label of first button from the right
            {\it (type=str)}

          \item[defcho]


default choice (1, 2 or 3)
            {\it (type=int)}

        \end{Ventry}

      \end{quote}

      \textbf{Return Value}
    \vspace{-1ex}

      \begin{quote}

number of the button pressed (1, 2 or 3)
      {\it (type=int)}

      \end{quote}

\textbf{Note:} 
e.g. pressbtn = fl\_show\_choices(``some message'', ``some other'',
``the end'', 3, ``1st'', ``2nd'', ``3rd'', 1)


\textbf{Status:} 
Tested + Doc + Demo = OK


    \end{boxedminipage}

    \label{xformslib:flgoodies:fl_hide_choice}
    \index{xformslib \textit{(package)}!xformslib.flgoodies \textit{(module)}!xformslib.flgoodies.fl\_hide\_choice \textit{(function)}}

    \vspace{0.5ex}

\hspace{.8\funcindent}\begin{boxedminipage}{\funcwidth}

    \raggedright \textbf{fl\_hide\_choice}()

    \vspace{-1.5ex}

    \rule{\textwidth}{0.5\fboxrule}
\setlength{\parskip}{2ex}

Hides the choice message.

-{}-
\setlength{\parskip}{1ex}
\textbf{Note:} 
e.g. fl\_hide\_choice()


\textbf{Status:} 
Tested + Doc + NoDemo = OK


    \end{boxedminipage}

    \label{xformslib:flgoodies:fl_set_choices_shortcut}
    \index{xformslib \textit{(package)}!xformslib.flgoodies \textit{(module)}!xformslib.flgoodies.fl\_set\_choices\_shortcut \textit{(function)}}

    \vspace{0.5ex}

\hspace{.8\funcindent}\begin{boxedminipage}{\funcwidth}

    \raggedright \textbf{fl\_set\_choices\_shortcut}(\textit{shc1txt}, \textit{shc2txt}, \textit{shc3txt})

    \vspace{-1.5ex}

    \rule{\textwidth}{0.5\fboxrule}
\setlength{\parskip}{2ex}

Defines more mnemonic hotkeys as shortcut text for choices.
%
\begin{quote}
%
\begin{description}
\item[{\texttt{shc1txt}}] \leavevmode (\textbf{str})

shortcut to bind to first button

\item[{\texttt{shc2txt}}] \leavevmode (\textbf{str})

shortcut to bind to second button

\item[{\texttt{shc3txt}}] \leavevmode (\textbf{str})

shortcut to bind to third button

\end{description}

\end{quote}
\setlength{\parskip}{1ex}
\textbf{Note:} 
e.g. fl\_set\_choices\_shortcut(``a'', ``B'', ``\textasciicircum{}C'')


\textbf{Status:} 
Tested + Doc + NoDemo = OK


    \end{boxedminipage}

    \label{xformslib:flgoodies:fl_set_choices_shortcut}
    \index{xformslib \textit{(package)}!xformslib.flgoodies \textit{(module)}!xformslib.flgoodies.fl\_set\_choices\_shortcut \textit{(function)}}

    \vspace{0.5ex}

\hspace{.8\funcindent}\begin{boxedminipage}{\funcwidth}

    \raggedright \textbf{fl\_set\_choice\_shortcut}(\textit{shc1txt}, \textit{shc2txt}, \textit{shc3txt})

    \vspace{-1.5ex}

    \rule{\textwidth}{0.5\fboxrule}
\setlength{\parskip}{2ex}

Defines more mnemonic hotkeys as shortcut text for choices.
%
\begin{quote}
%
\begin{description}
\item[{\texttt{shc1txt}}] \leavevmode (\textbf{str})

shortcut to bind to first button

\item[{\texttt{shc2txt}}] \leavevmode (\textbf{str})

shortcut to bind to second button

\item[{\texttt{shc3txt}}] \leavevmode (\textbf{str})

shortcut to bind to third button

\end{description}

\end{quote}
\setlength{\parskip}{1ex}
\textbf{Note:} 
e.g. fl\_set\_choices\_shortcut(``a'', ``B'', ``\textasciicircum{}C'')


\textbf{Status:} 
Tested + Doc + NoDemo = OK


    \end{boxedminipage}

    \label{xformslib:flgoodies:fl_show_oneliner}
    \index{xformslib \textit{(package)}!xformslib.flgoodies \textit{(module)}!xformslib.flgoodies.fl\_show\_oneliner \textit{(function)}}

    \vspace{0.5ex}

\hspace{.8\funcindent}\begin{boxedminipage}{\funcwidth}

    \raggedright \textbf{fl\_show\_oneliner}(\textit{text}, \textit{x}, \textit{y})

    \vspace{-1.5ex}

    \rule{\textwidth}{0.5\fboxrule}
\setlength{\parskip}{2ex}

Shows a one-line message that can only be removed programmatically.
Multi-line message is possible by embedding the newline character in text.

-{}-
\setlength{\parskip}{1ex}
      \textbf{Parameters}
      \vspace{-1ex}

      \begin{quote}
        \begin{Ventry}{xxxx}

          \item[text]


oneliner message text
            {\it (type=str)}

          \item[x]


horizontal position (relative to root window)
            {\it (type=int)}

          \item[y]


vertical position (relative to root window)
            {\it (type=int)}

        \end{Ventry}

      \end{quote}

\textbf{Note:} 
e.g. fl\_show\_oneliner(``Button to close window'', 134, 155)


\textbf{Status:} 
Tested + Doc + Demo = OK


    \end{boxedminipage}

    \label{xformslib:flgoodies:fl_hide_oneliner}
    \index{xformslib \textit{(package)}!xformslib.flgoodies \textit{(module)}!xformslib.flgoodies.fl\_hide\_oneliner \textit{(function)}}

    \vspace{0.5ex}

\hspace{.8\funcindent}\begin{boxedminipage}{\funcwidth}

    \raggedright \textbf{fl\_hide\_oneliner}()

    \vspace{-1.5ex}

    \rule{\textwidth}{0.5\fboxrule}
\setlength{\parskip}{2ex}

Hides the oneliner message previously shown.

-{}-
\setlength{\parskip}{1ex}
\textbf{Note:} 
e.g. fl\_hide\_oneliner()


\textbf{Status:} 
Tested + Doc + Demo = OK


    \end{boxedminipage}

    \label{xformslib:flgoodies:fl_set_oneliner_font}
    \index{xformslib \textit{(package)}!xformslib.flgoodies \textit{(module)}!xformslib.flgoodies.fl\_set\_oneliner\_font \textit{(function)}}

    \vspace{0.5ex}

\hspace{.8\funcindent}\begin{boxedminipage}{\funcwidth}

    \raggedright \textbf{fl\_set\_oneliner\_font}(\textit{style}, \textit{size})

    \vspace{-1.5ex}

    \rule{\textwidth}{0.5\fboxrule}
\setlength{\parskip}{2ex}

Sets font style and size to use in a oneliner message.

-{}-
\setlength{\parskip}{1ex}
      \textbf{Parameters}
      \vspace{-1ex}

      \begin{quote}
        \begin{Ventry}{xxxxx}

          \item[style]


label style. Values (from xfdata.py) FL\_NORMAL\_STYLE, FL\_BOLD\_STYLE,
FL\_ITALIC\_STYLE, FL\_BOLDITALIC\_STYLE, FL\_FIXED\_STYLE,
FL\_FIXEDBOLD\_STYLE, FL\_FIXEDITALIC\_STYLE, FL\_FIXEDBOLDITALIC\_STYLE,
FL\_TIMES\_STYLE, FL\_TIMESBOLD\_STYLE, FL\_TIMESITALIC\_STYLE,
FL\_TIMESBOLDITALIC\_STYLE, FL\_MISC\_STYLE, FL\_MISCBOLD\_STYLE,
FL\_MISCITALIC\_STYLE, FL\_SYMBOL\_STYLE, FL\_SHADOW\_STYLE,
FL\_ENGRAVED\_STYLE, FL\_EMBOSSED\_STYLE
            {\it (type=int)}

          \item[size]


label size. Values (from xfdata.py) FL\_TINY\_SIZE, FL\_SMALL\_SIZE,
FL\_NORMAL\_SIZE, FL\_MEDIUM\_SIZE, FL\_LARGE\_SIZE, FL\_HUGE\_SIZE,
FL\_DEFAULT\_SIZE
            {\it (type=int)}

        \end{Ventry}

      \end{quote}

\textbf{Note:} 
e.g. fl\_set\_oneliner\_font(FL\_BOLD\_STYLE, FL\_NORMAL\_SIZE)


\textbf{Status:} 
Tested + Doc + NoDemo = OK


    \end{boxedminipage}

    \label{xformslib:flgoodies:fl_set_oneliner_color}
    \index{xformslib \textit{(package)}!xformslib.flgoodies \textit{(module)}!xformslib.flgoodies.fl\_set\_oneliner\_color \textit{(function)}}

    \vspace{0.5ex}

\hspace{.8\funcindent}\begin{boxedminipage}{\funcwidth}

    \raggedright \textbf{fl\_set\_oneliner\_color}(\textit{fgcolr}, \textit{bgcolr})

    \vspace{-1.5ex}

    \rule{\textwidth}{0.5\fboxrule}
\setlength{\parskip}{2ex}

Sets color to use with oneliner message. By default, the background
of the message is yellow and the text black.

-{}-
\setlength{\parskip}{1ex}
      \textbf{Parameters}
      \vspace{-1ex}

      \begin{quote}
        \begin{Ventry}{xxxxxx}

          \item[fgcolr]


color value for oneliner foreground
            {\it (type=long\_pos)}

          \item[bgcolr]


color value for oneliner background
            {\it (type=long\_pos)}

        \end{Ventry}

      \end{quote}

\textbf{Note:} 
e.g. fl\_set\_oneliner\_color(fgcolr, bgcolr)


\textbf{Status:} 
Tested + Doc + NoDemo = OK


    \end{boxedminipage}

    \label{xformslib:flgoodies:fl_set_tooltip_font}
    \index{xformslib \textit{(package)}!xformslib.flgoodies \textit{(module)}!xformslib.flgoodies.fl\_set\_tooltip\_font \textit{(function)}}

    \vspace{0.5ex}

\hspace{.8\funcindent}\begin{boxedminipage}{\funcwidth}

    \raggedright \textbf{fl\_set\_tooltip\_font}(\textit{style}, \textit{size})

    \vspace{-1.5ex}

    \rule{\textwidth}{0.5\fboxrule}
\setlength{\parskip}{2ex}

Sets the font style and size of the tooltip.

-{}-
\setlength{\parskip}{1ex}
      \textbf{Parameters}
      \vspace{-1ex}

      \begin{quote}
        \begin{Ventry}{xxxxx}

          \item[style]


label style. Values (from xfdata.py) FL\_NORMAL\_STYLE, FL\_BOLD\_STYLE,
FL\_ITALIC\_STYLE, FL\_BOLDITALIC\_STYLE, FL\_FIXED\_STYLE,
FL\_FIXEDBOLD\_STYLE, FL\_FIXEDITALIC\_STYLE, FL\_FIXEDBOLDITALIC\_STYLE,
FL\_TIMES\_STYLE, FL\_TIMESBOLD\_STYLE, FL\_TIMESITALIC\_STYLE,
FL\_TIMESBOLDITALIC\_STYLE, FL\_MISC\_STYLE, FL\_MISCBOLD\_STYLE,
FL\_MISCITALIC\_STYLE, FL\_SYMBOL\_STYLE, FL\_SHADOW\_STYLE,
FL\_ENGRAVED\_STYLE, FL\_EMBOSSED\_STYLE
            {\it (type=int)}

          \item[size]


label size. Values (from xfdata.py) FL\_TINY\_SIZE, FL\_SMALL\_SIZE,
FL\_NORMAL\_SIZE, FL\_MEDIUM\_SIZE, FL\_LARGE\_SIZE, FL\_HUGE\_SIZE,
FL\_DEFAULT\_SIZE
            {\it (type=int)}

        \end{Ventry}

      \end{quote}

\textbf{Note:} 
e.g. fl\_set\_tooltip\_font(xfdata.FL\_SHADOW\_STYLE,
xfdata.FL\_DEFAULT\_SIZE)


\textbf{Status:} 
Tested + Doc + NoDemo = OK


    \end{boxedminipage}

    \label{xformslib:flgoodies:fl_set_tooltip_color}
    \index{xformslib \textit{(package)}!xformslib.flgoodies \textit{(module)}!xformslib.flgoodies.fl\_set\_tooltip\_color \textit{(function)}}

    \vspace{0.5ex}

\hspace{.8\funcindent}\begin{boxedminipage}{\funcwidth}

    \raggedright \textbf{fl\_set\_tooltip\_color}(\textit{fgcolr}, \textit{bgcolr})

    \vspace{-1.5ex}

    \rule{\textwidth}{0.5\fboxrule}
\setlength{\parskip}{2ex}

Sets the foreground and the background colors of the tooltip.

-{}-
\setlength{\parskip}{1ex}
      \textbf{Parameters}
      \vspace{-1ex}

      \begin{quote}
        \begin{Ventry}{xxxxxx}

          \item[fgcolr]


foreground color value
            {\it (type=long\_pos)}

          \item[bgcolr]


background color value
            {\it (type=long\_pos)}

        \end{Ventry}

      \end{quote}

\textbf{Note:} 
e.g. fl\_set\_tooltip\_color(xfdata.FL\_BLUE, xfdata.FL\_VIOLET)


\textbf{Status:} 
Tested + Doc + NoDemo = OK


    \end{boxedminipage}

    \label{xformslib:flgoodies:fl_set_tooltip_boxtype}
    \index{xformslib \textit{(package)}!xformslib.flgoodies \textit{(module)}!xformslib.flgoodies.fl\_set\_tooltip\_boxtype \textit{(function)}}

    \vspace{0.5ex}

\hspace{.8\funcindent}\begin{boxedminipage}{\funcwidth}

    \raggedright \textbf{fl\_set\_tooltip\_boxtype}(\textit{boxtype})

    \vspace{-1.5ex}

    \rule{\textwidth}{0.5\fboxrule}
\setlength{\parskip}{2ex}

Sets the boxtype of the tooltip.

-{}-
\setlength{\parskip}{1ex}
      \textbf{Parameters}
      \vspace{-1ex}

      \begin{quote}
        \begin{Ventry}{xxxxxxx}

          \item[boxtype]


type of the box to be added. Values (from xfdata.py) FL\_NO\_BOX,
FL\_UP\_BOX, FL\_DOWN\_BOX, FL\_BORDER\_BOX, FL\_SHADOW\_BOX, FL\_FRAME\_BOX,
FL\_ROUNDED\_BOX, FL\_EMBOSSED\_BOX, FL\_FLAT\_BOX, FL\_RFLAT\_BOX,
FL\_RSHADOW\_BOX, FL\_OVAL\_BOX, FL\_ROUNDED3D\_UPBOX, FL\_ROUNDED3D\_DOWNBOX,
FL\_OVAL3D\_UPBOX, FL\_OVAL3D\_DOWNBOX, FL\_OVAL3D\_FRAMEBOX,
FL\_OVAL3D\_EMBOSSEDBOX
            {\it (type=int)}

        \end{Ventry}

      \end{quote}

\textbf{Note:} 
e.g. fl\_set\_tooltip\_boxtype(xfdata.FL\_OVAL3D\_DOWNBOX)


\textbf{Status:} 
Tested + Doc + NoDemo = OK


    \end{boxedminipage}

    \label{xformslib:flgoodies:fl_set_tooltip_lalign}
    \index{xformslib \textit{(package)}!xformslib.flgoodies \textit{(module)}!xformslib.flgoodies.fl\_set\_tooltip\_lalign \textit{(function)}}

    \vspace{0.5ex}

\hspace{.8\funcindent}\begin{boxedminipage}{\funcwidth}

    \raggedright \textbf{fl\_set\_tooltip\_lalign}(\textit{align})

    \vspace{-1.5ex}

    \rule{\textwidth}{0.5\fboxrule}
\setlength{\parskip}{2ex}

Sets the alignment of the tooltip.

-{}-
\setlength{\parskip}{1ex}
      \textbf{Parameters}
      \vspace{-1ex}

      \begin{quote}
        \begin{Ventry}{xxxxx}

          \item[align]


alignment of tooltip. Values (from xfdata.py) FL\_ALIGN\_CENTER,
FL\_ALIGN\_TOP, FL\_ALIGN\_BOTTOM, FL\_ALIGN\_LEFT, FL\_ALIGN\_RIGHT,
FL\_ALIGN\_LEFT\_TOP, FL\_ALIGN\_RIGHT\_TOP, FL\_ALIGN\_LEFT\_BOTTOM,
FL\_ALIGN\_RIGHT\_BOTTOM, FL\_ALIGN\_INSIDE, FL\_ALIGN\_VERT.
Bitwise OR with FL\_ALIGN\_INSIDE is allowed.
            {\it (type=int)}

        \end{Ventry}

      \end{quote}

\textbf{Note:} 
e.g. fl\_set\_tooltip\_lalign(xfdata.FL\_ALIGN\_RIGHT\_TOP)


\textbf{Status:} 
Tested + Doc + NoDemo = OK


    \end{boxedminipage}

    \label{xformslib:flgoodies:fl_exe_command}
    \index{xformslib \textit{(package)}!xformslib.flgoodies \textit{(module)}!xformslib.flgoodies.fl\_exe\_command \textit{(function)}}

    \vspace{0.5ex}

\hspace{.8\funcindent}\begin{boxedminipage}{\funcwidth}

    \raggedright \textbf{fl\_exe\_command}(\textit{cmdtxt}, \textit{block})

    \vspace{-1.5ex}

    \rule{\textwidth}{0.5\fboxrule}
\setlength{\parskip}{2ex}

Forks a new process that runs specified command.

-{}-
\setlength{\parskip}{1ex}
      \textbf{Parameters}
      \vspace{-1ex}

      \begin{quote}
        \begin{Ventry}{xxxxxx}

          \item[cmdtxt]


a shell command line
            {\it (type=str)}

          \item[block]


blocking flag indicating if the function should wait for the child
process to finish or not. Values non-zero (for waiting) or 0 (don't
wait).
            {\it (type=int)}

        \end{Ventry}

      \end{quote}

      \textbf{Return Value}
    \vspace{-1ex}

      \begin{quote}

exit status
      {\it (type=long)}

      \end{quote}

\textbf{Note:} 
e.g. \emph{todo}


\textbf{Status:} 
Untested + Doc + NoDemo = NOT OK


    \end{boxedminipage}

    \label{xformslib:flgoodies:fl_exe_command}
    \index{xformslib \textit{(package)}!xformslib.flgoodies \textit{(module)}!xformslib.flgoodies.fl\_exe\_command \textit{(function)}}

    \vspace{0.5ex}

\hspace{.8\funcindent}\begin{boxedminipage}{\funcwidth}

    \raggedright \textbf{fl\_open\_command}(\textit{cmdtxt}, \textit{block})

    \vspace{-1.5ex}

    \rule{\textwidth}{0.5\fboxrule}
\setlength{\parskip}{2ex}

Forks a new process that runs specified command.

-{}-
\setlength{\parskip}{1ex}
      \textbf{Parameters}
      \vspace{-1ex}

      \begin{quote}
        \begin{Ventry}{xxxxxx}

          \item[cmdtxt]


a shell command line
            {\it (type=str)}

          \item[block]


blocking flag indicating if the function should wait for the child
process to finish or not. Values non-zero (for waiting) or 0 (don't
wait).
            {\it (type=int)}

        \end{Ventry}

      \end{quote}

      \textbf{Return Value}
    \vspace{-1ex}

      \begin{quote}

exit status
      {\it (type=long)}

      \end{quote}

\textbf{Note:} 
e.g. \emph{todo}


\textbf{Status:} 
Untested + Doc + NoDemo = NOT OK


    \end{boxedminipage}

    \label{xformslib:flgoodies:fl_end_command}
    \index{xformslib \textit{(package)}!xformslib.flgoodies \textit{(module)}!xformslib.flgoodies.fl\_end\_command \textit{(function)}}

    \vspace{0.5ex}

\hspace{.8\funcindent}\begin{boxedminipage}{\funcwidth}

    \raggedright \textbf{fl\_end\_command}(\textit{pid})

    \vspace{-1.5ex}

    \rule{\textwidth}{0.5\fboxrule}
\setlength{\parskip}{2ex}

Suspends the current process and waits until the child process is
completed.

-{}-
\setlength{\parskip}{1ex}
      \textbf{Parameters}
      \vspace{-1ex}

      \begin{quote}
        \begin{Ventry}{xxx}

          \item[pid]


process id returned by fl\_exe\_command()
            {\it (type=long)}

        \end{Ventry}

      \end{quote}

      \textbf{Return Value}
    \vspace{-1ex}

      \begin{quote}

exit status of child process, or -1 (if an error has occurred)
      {\it (type=long)}

      \end{quote}

\textbf{Note:} 
e.g. fl\_end\_command(1488)


\textbf{Status:} 
Tested + Doc + NoDemo = OK


    \end{boxedminipage}

    \label{xformslib:flgoodies:fl_end_command}
    \index{xformslib \textit{(package)}!xformslib.flgoodies \textit{(module)}!xformslib.flgoodies.fl\_end\_command \textit{(function)}}

    \vspace{0.5ex}

\hspace{.8\funcindent}\begin{boxedminipage}{\funcwidth}

    \raggedright \textbf{fl\_close\_command}(\textit{pid})

    \vspace{-1.5ex}

    \rule{\textwidth}{0.5\fboxrule}
\setlength{\parskip}{2ex}

Suspends the current process and waits until the child process is
completed.

-{}-
\setlength{\parskip}{1ex}
      \textbf{Parameters}
      \vspace{-1ex}

      \begin{quote}
        \begin{Ventry}{xxx}

          \item[pid]


process id returned by fl\_exe\_command()
            {\it (type=long)}

        \end{Ventry}

      \end{quote}

      \textbf{Return Value}
    \vspace{-1ex}

      \begin{quote}

exit status of child process, or -1 (if an error has occurred)
      {\it (type=long)}

      \end{quote}

\textbf{Note:} 
e.g. fl\_end\_command(1488)


\textbf{Status:} 
Tested + Doc + NoDemo = OK


    \end{boxedminipage}

    \label{xformslib:flgoodies:fl_check_command}
    \index{xformslib \textit{(package)}!xformslib.flgoodies \textit{(module)}!xformslib.flgoodies.fl\_check\_command \textit{(function)}}

    \vspace{0.5ex}

\hspace{.8\funcindent}\begin{boxedminipage}{\funcwidth}

    \raggedright \textbf{fl\_check\_command}(\textit{pid})

    \vspace{-1.5ex}

    \rule{\textwidth}{0.5\fboxrule}
\setlength{\parskip}{2ex}

Polls the status of a child process.

-{}-
\setlength{\parskip}{1ex}
      \textbf{Parameters}
      \vspace{-1ex}

      \begin{quote}
        \begin{Ventry}{xxx}

          \item[pid]


process id returned by fl\_exe\_command()
            {\it (type=long)}

        \end{Ventry}

      \end{quote}

      \textbf{Return Value}
    \vspace{-1ex}

      \begin{quote}

0 if the child process is finished, or 1 if the child process
still exists (running or stopped), or -1 if an error has occurred
inside the function
      {\it (type=int)}

      \end{quote}

\textbf{Note:} 
e.g. fl\_check\_command(1488)


\textbf{Status:} 
Tested + Doc + NoDemo = OK


    \end{boxedminipage}

    \label{xformslib:flgoodies:fl_popen}
    \index{xformslib \textit{(package)}!xformslib.flgoodies \textit{(module)}!xformslib.flgoodies.fl\_popen \textit{(function)}}

    \vspace{0.5ex}

\hspace{.8\funcindent}\begin{boxedminipage}{\funcwidth}

    \raggedright \textbf{fl\_popen}(\textit{cmdtxt}, \textit{otype})

    \vspace{-1.5ex}

    \rule{\textwidth}{0.5\fboxrule}
\setlength{\parskip}{2ex}

Executes the command in a child process, and logs the stderr messages
into the command log. If otype is ``w'', stdout will also be logged into
the command browser.

-{}-
\setlength{\parskip}{1ex}
      \textbf{Parameters}
      \vspace{-1ex}

      \begin{quote}
        \begin{Ventry}{xxxxxx}

          \item[cmdtxt]


existing filename to execute
            {\it (type=str)}

          \item[otype]


type of opening (e.g. letter between w, r ..)
            {\it (type=str)}

        \end{Ventry}

      \end{quote}

      \textbf{Return Value}
    \vspace{-1ex}

      \begin{quote}

file opened (pFile)
      {\it (type=pointer to xfdata.FILE)}

      \end{quote}

\textbf{Note:} 
e.g. pfile = fl\_popen(``/usr/bin/somecommand'', ``r'')


\textbf{Status:} 
Tested + Doc + NoDemo = OK


    \end{boxedminipage}

    \label{xformslib:flgoodies:fl_pclose}
    \index{xformslib \textit{(package)}!xformslib.flgoodies \textit{(module)}!xformslib.flgoodies.fl\_pclose \textit{(function)}}

    \vspace{0.5ex}

\hspace{.8\funcindent}\begin{boxedminipage}{\funcwidth}

    \raggedright \textbf{fl\_pclose}(\textit{pFile})

    \vspace{-1.5ex}

    \rule{\textwidth}{0.5\fboxrule}
\setlength{\parskip}{2ex}

Cleans up the child process executed.

-{}-
\setlength{\parskip}{1ex}
      \textbf{Parameters}
      \vspace{-1ex}

      \begin{quote}
        \begin{Ventry}{xxxxx}

          \item[pFile]


opened file stream returned by fl\_popen()
            {\it (type=pointer to xfdata.FILE)}

        \end{Ventry}

      \end{quote}

      \textbf{Return Value}
    \vspace{-1ex}

      \begin{quote}

non-zero, or -1 (on failure)
      {\it (type=int)}

      \end{quote}

\textbf{Note:} 
e.g. if fl\_pclose(pfile) == -1: ...


\textbf{Status:} 
Tested + Doc + NoDemo = OK


    \end{boxedminipage}

    \label{xformslib:flgoodies:fl_end_all_command}
    \index{xformslib \textit{(package)}!xformslib.flgoodies \textit{(module)}!xformslib.flgoodies.fl\_end\_all\_command \textit{(function)}}

    \vspace{0.5ex}

\hspace{.8\funcindent}\begin{boxedminipage}{\funcwidth}

    \raggedright \textbf{fl\_end\_all\_command}()

    \vspace{-1.5ex}

    \rule{\textwidth}{0.5\fboxrule}
\setlength{\parskip}{2ex}

Waits for all the child processes initiated by fl\_exe\_command()
to complete.

-{}-
\setlength{\parskip}{1ex}
      \textbf{Return Value}
    \vspace{-1ex}

      \begin{quote}

exit status of the last child process
      {\it (type=int)}

      \end{quote}

\textbf{Note:} 
e.g. fl\_end\_all\_command()


\textbf{Status:} 
Tested + Doc + NoDemo = OK


    \end{boxedminipage}

    \label{xformslib:flgoodies:fl_show_command_log}
    \index{xformslib \textit{(package)}!xformslib.flgoodies \textit{(module)}!xformslib.flgoodies.fl\_show\_command\_log \textit{(function)}}

    \vspace{0.5ex}

\hspace{.8\funcindent}\begin{boxedminipage}{\funcwidth}

    \raggedright \textbf{fl\_show\_command\_log}(\textit{border})

    \vspace{-1.5ex}

    \rule{\textwidth}{0.5\fboxrule}
\setlength{\parskip}{2ex}

Shows the log of the command output.

-{}-
\setlength{\parskip}{1ex}
      \textbf{Parameters}
      \vspace{-1ex}

      \begin{quote}
        \begin{Ventry}{xxxxxx}

          \item[border]


window manager decoration. Values (from xfdata.py) FL\_FULLBORDER,
FL\_TRANSIENT, FL\_NOBORDER
            {\it (type=int)}

        \end{Ventry}

      \end{quote}

\textbf{Note:} 
e.g. fl\_show\_command\_log(xfdata.FL\_FULLBORDER)


\textbf{Status:} 
Untested + Doc + NoDemo = NOT OK


    \end{boxedminipage}

    \label{xformslib:flgoodies:fl_hide_command_log}
    \index{xformslib \textit{(package)}!xformslib.flgoodies \textit{(module)}!xformslib.flgoodies.fl\_hide\_command\_log \textit{(function)}}

    \vspace{0.5ex}

\hspace{.8\funcindent}\begin{boxedminipage}{\funcwidth}

    \raggedright \textbf{fl\_hide\_command\_log}()

    \vspace{-1.5ex}

    \rule{\textwidth}{0.5\fboxrule}
\setlength{\parskip}{2ex}

Hides the log of the command output.

-{}-
\setlength{\parskip}{1ex}
\textbf{Note:} 
e.g. fl\_hide\_command\_log()


\textbf{Status:} 
Tested + Doc + NoDemo = OK


    \end{boxedminipage}

    \label{xformslib:flgoodies:fl_clear_command_log}
    \index{xformslib \textit{(package)}!xformslib.flgoodies \textit{(module)}!xformslib.flgoodies.fl\_clear\_command\_log \textit{(function)}}

    \vspace{0.5ex}

\hspace{.8\funcindent}\begin{boxedminipage}{\funcwidth}

    \raggedright \textbf{fl\_clear\_command\_log}()

    \vspace{-1.5ex}

    \rule{\textwidth}{0.5\fboxrule}
\setlength{\parskip}{2ex}

Clears the browser and the logging output displayed within it.

-{}-
\setlength{\parskip}{1ex}
\textbf{Note:} 
e.g. fl\_clear\_command\_log()


\textbf{Status:} 
Tested + Doc + NoDemo = OK


    \end{boxedminipage}

    \label{xformslib:flgoodies:fl_addto_command_log}
    \index{xformslib \textit{(package)}!xformslib.flgoodies \textit{(module)}!xformslib.flgoodies.fl\_addto\_command\_log \textit{(function)}}

    \vspace{0.5ex}

\hspace{.8\funcindent}\begin{boxedminipage}{\funcwidth}

    \raggedright \textbf{fl\_addto\_command\_log}(\textit{txtstr})

    \vspace{-1.5ex}

    \rule{\textwidth}{0.5\fboxrule}
\setlength{\parskip}{2ex}

Adds arbitrary text to the command browser.

-{}-
\setlength{\parskip}{1ex}
      \textbf{Parameters}
      \vspace{-1ex}

      \begin{quote}
        \begin{Ventry}{xxxxxx}

          \item[txtstr]


text line to be added (with possible embedded newlines)
            {\it (type=str)}

        \end{Ventry}

      \end{quote}

\textbf{Note:} 
e.g. fl\_addto\_command\_log(``Another line to add to CmdLog'')


\textbf{Status:} 
Tested + Doc + NoDemo = OK


    \end{boxedminipage}

    \label{xformslib:flgoodies:fl_set_command_log_position}
    \index{xformslib \textit{(package)}!xformslib.flgoodies \textit{(module)}!xformslib.flgoodies.fl\_set\_command\_log\_position \textit{(function)}}

    \vspace{0.5ex}

\hspace{.8\funcindent}\begin{boxedminipage}{\funcwidth}

    \raggedright \textbf{fl\_set\_command\_log\_position}(\textit{x}, \textit{y})

    \vspace{-1.5ex}

    \rule{\textwidth}{0.5\fboxrule}
\setlength{\parskip}{2ex}

Changes the default placement of the command log.

-{}-
\setlength{\parskip}{1ex}
      \textbf{Parameters}
      \vspace{-1ex}

      \begin{quote}
        \begin{Ventry}{x}

          \item[x]


horizontal position (upper-left corner)
            {\it (type=int)}

          \item[y]


vertical position (upper-left corner)
            {\it (type=int)}

        \end{Ventry}

      \end{quote}

\textbf{Note:} 
e.g. fl\_set\_command\_log\_position(174, 288)


\textbf{Status:} 
Tested + Doc + NoDemo = OK


    \end{boxedminipage}

    \label{xformslib:flgoodies:fl_get_command_log_fdstruct}
    \index{xformslib \textit{(package)}!xformslib.flgoodies \textit{(module)}!xformslib.flgoodies.fl\_get\_command\_log\_fdstruct \textit{(function)}}

    \vspace{0.5ex}

\hspace{.8\funcindent}\begin{boxedminipage}{\funcwidth}

    \raggedright \textbf{fl\_get\_command\_log\_fdstruct}()

    \vspace{-1.5ex}

    \rule{\textwidth}{0.5\fboxrule}
\setlength{\parskip}{2ex}

Obtains the GUI structure of the command browser. From the information
returned, the application program can change various attributes of the
command browser and its associated objects. Note however, that you should
not hide/show the form or free any member of the returned structure.

-{}-
\setlength{\parskip}{1ex}
      \textbf{Return Value}
    \vspace{-1ex}

      \begin{quote}

command log browser class instance (pCmdlog)
      {\it (type=pointer to xfdata.FD\_CMDLOG)}

      \end{quote}

\textbf{Note:} 
e.g. pcmdlogbr = fl\_get\_command\_log\_fdstruct()


\textbf{Status:} 
Tested + Doc + NoDemo = OK


    \end{boxedminipage}

    \label{xformslib:flgoodies:fl_use_fselector}
    \index{xformslib \textit{(package)}!xformslib.flgoodies \textit{(module)}!xformslib.flgoodies.fl\_use\_fselector \textit{(function)}}

    \vspace{0.5ex}

\hspace{.8\funcindent}\begin{boxedminipage}{\funcwidth}

    \raggedright \textbf{fl\_use\_fselector}(\textit{num})

    \vspace{-1.5ex}

    \rule{\textwidth}{0.5\fboxrule}
\setlength{\parskip}{2ex}

Sets the currently active file selector.

-{}-
\setlength{\parskip}{1ex}
      \textbf{Parameters}
      \vspace{-1ex}

      \begin{quote}
        \begin{Ventry}{xxx}

          \item[num]


fselector number to use. Values between 0 and
xfdata.FL\_MAX\_FSELECTOR - 1
            {\it (type=int)}

        \end{Ventry}

      \end{quote}

      \textbf{Return Value}
    \vspace{-1ex}

      \begin{quote}

old file selector number
      {\it (type=int)}

      \end{quote}

\textbf{Note:} 
e.g. oldfsel = fl\_use\_fselector(1)


\textbf{Status:} 
Tested + Doc + NoDemo = OK


    \end{boxedminipage}

    \label{xformslib:flgoodies:fl_show_fselector}
    \index{xformslib \textit{(package)}!xformslib.flgoodies \textit{(module)}!xformslib.flgoodies.fl\_show\_fselector \textit{(function)}}

    \vspace{0.5ex}

\hspace{.8\funcindent}\begin{boxedminipage}{\funcwidth}

    \raggedright \textbf{fl\_show\_fselector}(\textit{msgtxt}, \textit{dirname}, \textit{pattern}, \textit{deftxt})

    \vspace{-1.5ex}

    \rule{\textwidth}{0.5\fboxrule}
\setlength{\parskip}{2ex}

Show a file selector, providing an easy and interactive way to let
the user select files.

-{}-
\setlength{\parskip}{1ex}
      \textbf{Parameters}
      \vspace{-1ex}

      \begin{quote}
        \begin{Ventry}{xxxxxxx}

          \item[msgtxt]


message text
            {\it (type=str)}

          \item[dirname]


directory name
            {\it (type=str)}

          \item[pattern]


any kind of regular expression, e.g. ``{[}a-f{]}*c'' which would list all
files starting with a letter between a and f and ending with c.
            {\it (type=str)}

          \item[deftxt]


default file name
            {\it (type=str)}

        \end{Ventry}

      \end{quote}

      \textbf{Return Value}
    \vspace{-1ex}

      \begin{quote}

fselector text, or None (if the Cancel button is pressed)
      {\it (type=str)}

      \end{quote}

\textbf{Note:} 
e.g. fstxt = fl\_show\_fselector(``Choose file:'', ``/home/user'',
``\emph{.}'', ``'')


\textbf{Status:} 
Tested + Doc + Demo = OK


    \end{boxedminipage}

    \label{xformslib:flgoodies:fl_show_fselector}
    \index{xformslib \textit{(package)}!xformslib.flgoodies \textit{(module)}!xformslib.flgoodies.fl\_show\_fselector \textit{(function)}}

    \vspace{0.5ex}

\hspace{.8\funcindent}\begin{boxedminipage}{\funcwidth}

    \raggedright \textbf{fl\_show\_file\_selector}(\textit{msgtxt}, \textit{dirname}, \textit{pattern}, \textit{deftxt})

    \vspace{-1.5ex}

    \rule{\textwidth}{0.5\fboxrule}
\setlength{\parskip}{2ex}

Show a file selector, providing an easy and interactive way to let
the user select files.

-{}-
\setlength{\parskip}{1ex}
      \textbf{Parameters}
      \vspace{-1ex}

      \begin{quote}
        \begin{Ventry}{xxxxxxx}

          \item[msgtxt]


message text
            {\it (type=str)}

          \item[dirname]


directory name
            {\it (type=str)}

          \item[pattern]


any kind of regular expression, e.g. ``{[}a-f{]}*c'' which would list all
files starting with a letter between a and f and ending with c.
            {\it (type=str)}

          \item[deftxt]


default file name
            {\it (type=str)}

        \end{Ventry}

      \end{quote}

      \textbf{Return Value}
    \vspace{-1ex}

      \begin{quote}

fselector text, or None (if the Cancel button is pressed)
      {\it (type=str)}

      \end{quote}

\textbf{Note:} 
e.g. fstxt = fl\_show\_fselector(``Choose file:'', ``/home/user'',
``\emph{.}'', ``'')


\textbf{Status:} 
Tested + Doc + Demo = OK


    \end{boxedminipage}

    \label{xformslib:flgoodies:fl_set_fselector_fontsize}
    \index{xformslib \textit{(package)}!xformslib.flgoodies \textit{(module)}!xformslib.flgoodies.fl\_set\_fselector\_fontsize \textit{(function)}}

    \vspace{0.5ex}

\hspace{.8\funcindent}\begin{boxedminipage}{\funcwidth}

    \raggedright \textbf{fl\_set\_fselector\_fontsize}(\textit{size})

    \vspace{-1.5ex}

    \rule{\textwidth}{0.5\fboxrule}
\setlength{\parskip}{2ex}

Changes the font size of a file selector.

-{}-
\setlength{\parskip}{1ex}
      \textbf{Parameters}
      \vspace{-1ex}

      \begin{quote}
        \begin{Ventry}{xxxx}

          \item[size]


label size. Values (from xfdata.py) FL\_TINY\_SIZE, FL\_SMALL\_SIZE,
FL\_NORMAL\_SIZE, FL\_MEDIUM\_SIZE, FL\_LARGE\_SIZE, FL\_HUGE\_SIZE,
FL\_DEFAULT\_SIZE
            {\it (type=int)}

        \end{Ventry}

      \end{quote}

\textbf{Note:} 
e.g. fl\_set\_fselector\_fontsize(xfdata.TINY\_SIZE)


\textbf{Status:} 
Tested + Doc + NoDemo = OK


    \end{boxedminipage}

    \label{xformslib:flgoodies:fl_set_fselector_fontstyle}
    \index{xformslib \textit{(package)}!xformslib.flgoodies \textit{(module)}!xformslib.flgoodies.fl\_set\_fselector\_fontstyle \textit{(function)}}

    \vspace{0.5ex}

\hspace{.8\funcindent}\begin{boxedminipage}{\funcwidth}

    \raggedright \textbf{fl\_set\_fselector\_fontstyle}(\textit{style})

    \vspace{-1.5ex}

    \rule{\textwidth}{0.5\fboxrule}
\setlength{\parskip}{2ex}

Changes the font style of a file selector.

-{}-
\setlength{\parskip}{1ex}
      \textbf{Parameters}
      \vspace{-1ex}

      \begin{quote}
        \begin{Ventry}{xxxxx}

          \item[style]


label style. Values (from xfdata.py) FL\_NORMAL\_STYLE, FL\_BOLD\_STYLE,
FL\_ITALIC\_STYLE, FL\_BOLDITALIC\_STYLE, FL\_FIXED\_STYLE,
FL\_FIXEDBOLD\_STYLE, FL\_FIXEDITALIC\_STYLE, FL\_FIXEDBOLDITALIC\_STYLE,
FL\_TIMES\_STYLE, FL\_TIMESBOLD\_STYLE, FL\_TIMESITALIC\_STYLE,
FL\_TIMESBOLDITALIC\_STYLE, FL\_MISC\_STYLE, FL\_MISCBOLD\_STYLE,
FL\_MISCITALIC\_STYLE, FL\_SYMBOL\_STYLE, FL\_SHADOW\_STYLE,
FL\_ENGRAVED\_STYLE, FL\_EMBOSSED\_STYLE
            {\it (type=int)}

        \end{Ventry}

      \end{quote}

\textbf{Note:} 
e.g. fl\_set\_fselector\_fontstyle(xfdata.FL\_SHADOW\_STYLE)


\textbf{Status:} 
Tested + Doc + NoDemo = OK


    \end{boxedminipage}

    \label{xformslib:flgoodies:fl_set_fselector_placement}
    \index{xformslib \textit{(package)}!xformslib.flgoodies \textit{(module)}!xformslib.flgoodies.fl\_set\_fselector\_placement \textit{(function)}}

    \vspace{0.5ex}

\hspace{.8\funcindent}\begin{boxedminipage}{\funcwidth}

    \raggedright \textbf{fl\_set\_fselector\_placement}(\textit{place})

    \vspace{-1.5ex}

    \rule{\textwidth}{0.5\fboxrule}
\setlength{\parskip}{2ex}

Sets the placement of the file selector. By default it is centered
on the screen (FL\_PLACE\_CENTER|FL\_FREE\_SIZE).

-{}-
\setlength{\parskip}{1ex}
      \textbf{Parameters}
      \vspace{-1ex}

      \begin{quote}
        \begin{Ventry}{xxxxx}

          \item[place]


where to place it. Values (from xfdata.py) FL\_PLACE\_FREE,
FL\_PLACE\_MOUSE, FL\_PLACE\_CENTER, FL\_PLACE\_POSITION, FL\_PLACE\_SIZE,
FL\_PLACE\_GEOMETRY, FL\_PLACE\_ASPECT, FL\_PLACE\_FULLSCREEN,
FL\_PLACE\_HOTSPOT, FL\_PLACE\_ICONIC, FL\_FREE\_SIZE, FL\_PLACE\_FREE\_CENTER,
FL\_PLACE\_CENTERFREE, FL\_PLACE\_MOUSE|FL\_FREE\_SIZE,
FL\_PLACE\_FULLSCREEN|FL\_FREE\_SIZE, FL\_PLACE\_HOTSPOT|FL\_FREE\_SIZE
            {\it (type=int)}

        \end{Ventry}

      \end{quote}

\textbf{Note:} 
e.g. fl\_set\_fselector\_placement(xfdata.FL\_PLACE\_HOTSPOT)


\textbf{Status:} 
Tested + Doc + Demo = OK


    \end{boxedminipage}

    \label{xformslib:flgoodies:fl_set_fselector_border}
    \index{xformslib \textit{(package)}!xformslib.flgoodies \textit{(module)}!xformslib.flgoodies.fl\_set\_fselector\_border \textit{(function)}}

    \vspace{0.5ex}

\hspace{.8\funcindent}\begin{boxedminipage}{\funcwidth}

    \raggedright \textbf{fl\_set\_fselector\_border}(\textit{border})

    \vspace{-1.5ex}

    \rule{\textwidth}{0.5\fboxrule}
\setlength{\parskip}{2ex}

Changes the border of file selector. By default it is displayed with
transient property set (FL\_NOBORDER is ignored).

-{}-
\setlength{\parskip}{1ex}
      \textbf{Parameters}
      \vspace{-1ex}

      \begin{quote}
        \begin{Ventry}{xxxxxx}

          \item[border]


window manager decoration. Values (from xfdata.py) FL\_FULLBORDER,
FL\_TRANSIENT, FL\_NOBORDER
            {\it (type=int)}

        \end{Ventry}

      \end{quote}

\textbf{Note:} 
e.g. fl\_set\_fselector\_border(xfdata.FL\_FULLBORDER)


\textbf{Status:} 
Tested + Doc + NoDemo = OK


    \end{boxedminipage}

    \label{xformslib:flgoodies:fl_set_fselector_transient}
    \index{xformslib \textit{(package)}!xformslib.flgoodies \textit{(module)}!xformslib.flgoodies.fl\_set\_fselector\_transient \textit{(function)}}

    \vspace{0.5ex}

\hspace{.8\funcindent}\begin{boxedminipage}{\funcwidth}

    \raggedright \textbf{fl\_set\_fselector\_transient}(\textit{yesno})

    \vspace{-1.5ex}

    \rule{\textwidth}{0.5\fboxrule}
\setlength{\parskip}{2ex}

Set the property of file selector as transient or fullborder.

-{}-
\setlength{\parskip}{1ex}
      \textbf{Parameters}
      \vspace{-1ex}

      \begin{quote}
        \begin{Ventry}{xxxxx}

          \item[yesno]


flag if transient or not. Values 1 (transient) or 0 (not transient)
            {\it (type=int)}

        \end{Ventry}

      \end{quote}

\textbf{Note:} 
e.g. fl\_set\_fselector\_transient(0)


\textbf{Status:} 
Tested + Doc + NoDemo = OK


    \end{boxedminipage}

    \label{xformslib:flgoodies:fl_set_fselector_callback}
    \index{xformslib \textit{(package)}!xformslib.flgoodies \textit{(module)}!xformslib.flgoodies.fl\_set\_fselector\_callback \textit{(function)}}

    \vspace{0.5ex}

\hspace{.8\funcindent}\begin{boxedminipage}{\funcwidth}

    \raggedright \textbf{fl\_set\_fselector\_callback}(\textit{py\_FSCB}, \textit{vdata})

    \vspace{-1.5ex}

    \rule{\textwidth}{0.5\fboxrule}
\setlength{\parskip}{2ex}

Sets a callback routine so that whenever the user double clicks on a
filename, instead of returning the filename, this routine is invoked with
the filename as the argument. The behavior of the file selector is
slightly different when a callback is present. Without the callback, a
file selector is always modal. Please note that when a file selector has
a callback installed the field for manually entering a file name isn't
shown.

-{}-
\setlength{\parskip}{1ex}
      \textbf{Parameters}
      \vspace{-1ex}

      \begin{quote}
        \begin{Ventry}{xxxxxxx}

          \item[py\_FSCB]


name referring to function(string, vdata) -> num
            {\it (type=python function callback, returning (unused) value)}

          \item[vdata]


user data to be passed to function; callback has to take care of
type check.
            {\it (type=any type (e.g. 'None', int, str, etc..))}

        \end{Ventry}

      \end{quote}

\textbf{Notes:}
\begin{quote}
  \begin{itemize}

  \item
    \setlength{\parskip}{0.6ex}

e.g. def fsel\_cb(fname, cvoidp): > ... ; return UnusedVal


  \item 
e.g. fl\_set\_fselector\_callback(fsel\_cb, None)


\end{itemize}

\end{quote}

\textbf{Status:} 
Tested + Doc + Demo = OK


    \end{boxedminipage}

    \label{xformslib:flgoodies:fl_set_fselector_callback}
    \index{xformslib \textit{(package)}!xformslib.flgoodies \textit{(module)}!xformslib.flgoodies.fl\_set\_fselector\_callback \textit{(function)}}

    \vspace{0.5ex}

\hspace{.8\funcindent}\begin{boxedminipage}{\funcwidth}

    \raggedright \textbf{fl\_set\_fselector\_cb}(\textit{py\_FSCB}, \textit{vdata})

    \vspace{-1.5ex}

    \rule{\textwidth}{0.5\fboxrule}
\setlength{\parskip}{2ex}

Sets a callback routine so that whenever the user double clicks on a
filename, instead of returning the filename, this routine is invoked with
the filename as the argument. The behavior of the file selector is
slightly different when a callback is present. Without the callback, a
file selector is always modal. Please note that when a file selector has
a callback installed the field for manually entering a file name isn't
shown.

-{}-
\setlength{\parskip}{1ex}
      \textbf{Parameters}
      \vspace{-1ex}

      \begin{quote}
        \begin{Ventry}{xxxxxxx}

          \item[py\_FSCB]


name referring to function(string, vdata) -> num
            {\it (type=python function callback, returning (unused) value)}

          \item[vdata]


user data to be passed to function; callback has to take care of
type check.
            {\it (type=any type (e.g. 'None', int, str, etc..))}

        \end{Ventry}

      \end{quote}

\textbf{Notes:}
\begin{quote}
  \begin{itemize}

  \item
    \setlength{\parskip}{0.6ex}

e.g. def fsel\_cb(fname, cvoidp): > ... ; return UnusedVal


  \item 
e.g. fl\_set\_fselector\_callback(fsel\_cb, None)


\end{itemize}

\end{quote}

\textbf{Status:} 
Tested + Doc + Demo = OK


    \end{boxedminipage}

    \label{xformslib:flgoodies:fl_get_filename}
    \index{xformslib \textit{(package)}!xformslib.flgoodies \textit{(module)}!xformslib.flgoodies.fl\_get\_filename \textit{(function)}}

    \vspace{0.5ex}

\hspace{.8\funcindent}\begin{boxedminipage}{\funcwidth}

    \raggedright \textbf{fl\_get\_filename}()

    \vspace{-1.5ex}

    \rule{\textwidth}{0.5\fboxrule}
\setlength{\parskip}{2ex}

Obtains the file name (without the path) after the user changed it.

-{}-
\setlength{\parskip}{1ex}
      \textbf{Return Value}
    \vspace{-1ex}

      \begin{quote}

name of file (fname)
      {\it (type=str)}

      \end{quote}

\textbf{Note:} 
e.g. newfname = fl\_get\_filename()


\textbf{Status:} 
Tested + Doc + NoDemo = OK


    \end{boxedminipage}

    \label{xformslib:flgoodies:fl_get_directory}
    \index{xformslib \textit{(package)}!xformslib.flgoodies \textit{(module)}!xformslib.flgoodies.fl\_get\_directory \textit{(function)}}

    \vspace{0.5ex}

\hspace{.8\funcindent}\begin{boxedminipage}{\funcwidth}

    \raggedright \textbf{fl\_get\_directory}()

    \vspace{-1.5ex}

    \rule{\textwidth}{0.5\fboxrule}
\setlength{\parskip}{2ex}

Obtains the directory name after the user changed it.

-{}-
\setlength{\parskip}{1ex}
      \textbf{Return Value}
    \vspace{-1ex}

      \begin{quote}

name of directory (dirname)
      {\it (type=str)}

      \end{quote}

\textbf{Note:} 
e.g. newdname = fl\_get\_directory()


\textbf{Status:} 
Tested + Doc + NoDemo = OK


    \end{boxedminipage}

    \label{xformslib:flgoodies:fl_get_pattern}
    \index{xformslib \textit{(package)}!xformslib.flgoodies \textit{(module)}!xformslib.flgoodies.fl\_get\_pattern \textit{(function)}}

    \vspace{0.5ex}

\hspace{.8\funcindent}\begin{boxedminipage}{\funcwidth}

    \raggedright \textbf{fl\_get\_pattern}()

    \vspace{-1.5ex}

    \rule{\textwidth}{0.5\fboxrule}
\setlength{\parskip}{2ex}

Obtains the pattern after the user changed it.

-{}-
\setlength{\parskip}{1ex}
      \textbf{Return Value}
    \vspace{-1ex}

      \begin{quote}

pattern text
      {\it (type=str)}

      \end{quote}

\textbf{Status:} 
Tested + Doc + NoDemo = OK


    \end{boxedminipage}

    \label{xformslib:flgoodies:fl_set_directory}
    \index{xformslib \textit{(package)}!xformslib.flgoodies \textit{(module)}!xformslib.flgoodies.fl\_set\_directory \textit{(function)}}

    \vspace{0.5ex}

\hspace{.8\funcindent}\begin{boxedminipage}{\funcwidth}

    \raggedright \textbf{fl\_set\_directory}(\textit{dirname})

    \vspace{-1.5ex}

    \rule{\textwidth}{0.5\fboxrule}
\setlength{\parskip}{2ex}

Sets programmatically new value for the default directory.

-{}-
\setlength{\parskip}{1ex}
      \textbf{Parameters}
      \vspace{-1ex}

      \begin{quote}
        \begin{Ventry}{xxxxxxx}

          \item[dirname]


name of directory to be set
            {\it (type=str)}

        \end{Ventry}

      \end{quote}

      \textbf{Return Value}
    \vspace{-1ex}

      \begin{quote}

0 on success, or 1 (on failure)
      {\it (type=int)}

      \end{quote}

\textbf{Note:} 
e.g. sth = fl\_set\_directory(``/home/user/blabla'')


\textbf{Status:} 
Tested + Doc + NoDemo = OK


    \end{boxedminipage}

    \label{xformslib:flgoodies:fl_set_pattern}
    \index{xformslib \textit{(package)}!xformslib.flgoodies \textit{(module)}!xformslib.flgoodies.fl\_set\_pattern \textit{(function)}}

    \vspace{0.5ex}

\hspace{.8\funcindent}\begin{boxedminipage}{\funcwidth}

    \raggedright \textbf{fl\_set\_pattern}(\textit{pattern})

    \vspace{-1.5ex}

    \rule{\textwidth}{0.5\fboxrule}
\setlength{\parskip}{2ex}

Sets programmatically new value for the default pattern.

-{}-
\setlength{\parskip}{1ex}
      \textbf{Parameters}
      \vspace{-1ex}

      \begin{quote}
        \begin{Ventry}{xxxxxxx}

          \item[pattern]


text to be used for pattern
            {\it (type=str)}

        \end{Ventry}

      \end{quote}

\textbf{Note:} 
e.g. fl\_set\_pattern(``*.txt'')


\textbf{Status:} 
Tested + Doc + NoDemo = OK


    \end{boxedminipage}

    \label{xformslib:flgoodies:fl_refresh_fselector}
    \index{xformslib \textit{(package)}!xformslib.flgoodies \textit{(module)}!xformslib.flgoodies.fl\_refresh\_fselector \textit{(function)}}

    \vspace{0.5ex}

\hspace{.8\funcindent}\begin{boxedminipage}{\funcwidth}

    \raggedright \textbf{fl\_refresh\_fselector}()

    \vspace{-1.5ex}

    \rule{\textwidth}{0.5\fboxrule}
\setlength{\parskip}{2ex}

Refreshes the file selector, re-scanning the current directory and
listing all entries in it.

-{}-
\setlength{\parskip}{1ex}
\textbf{Note:} 
e.g. fl\_refresh\_fselector()


\textbf{Status:} 
Tested + Doc + NoDemo = OK


    \end{boxedminipage}

    \label{xformslib:flgoodies:fl_add_fselector_appbutton}
    \index{xformslib \textit{(package)}!xformslib.flgoodies \textit{(module)}!xformslib.flgoodies.fl\_add\_fselector\_appbutton \textit{(function)}}

    \vspace{0.5ex}

\hspace{.8\funcindent}\begin{boxedminipage}{\funcwidth}

    \raggedright \textbf{fl\_add\_fselector\_appbutton}(\textit{label}, \textit{py\_fn}, \textit{vdata})

    \vspace{-1.5ex}

    \rule{\textwidth}{0.5\fboxrule}
\setlength{\parskip}{2ex}

Adds an application specific button from file selector and a callback
routine for it.

-{}-
\setlength{\parskip}{1ex}
      \textbf{Parameters}
      \vspace{-1ex}

      \begin{quote}
        \begin{Ventry}{xxxxx}

          \item[label]


text of label
            {\it (type=str)}

          \item[py\_fn]


name referring to function(vdata)
            {\it (type=python function callback, no return)}

          \item[vdata]


user data to be passed to function; callback has to take care of
type check
            {\it (type=any type (e.g. 'None', int, str, etc..))}

        \end{Ventry}

      \end{quote}

\textbf{Notes:}
\begin{quote}
  \begin{itemize}

  \item
    \setlength{\parskip}{0.6ex}

e.g. def fsbtn\_cb(cvoidp): > ...


  \item 
e.g. fl\_add\_fselector\_appbutton(``SomeButton'', fsbtn\_cb, None)


\end{itemize}

\end{quote}

\textbf{Status:} 
Tested + Doc + NoDemo = OK


    \end{boxedminipage}

    \label{xformslib:flgoodies:fl_remove_fselector_appbutton}
    \index{xformslib \textit{(package)}!xformslib.flgoodies \textit{(module)}!xformslib.flgoodies.fl\_remove\_fselector\_appbutton \textit{(function)}}

    \vspace{0.5ex}

\hspace{.8\funcindent}\begin{boxedminipage}{\funcwidth}

    \raggedright \textbf{fl\_remove\_fselector\_appbutton}(\textit{label})

    \vspace{-1.5ex}

    \rule{\textwidth}{0.5\fboxrule}
\setlength{\parskip}{2ex}

Removes an application specific button from file selector.

-{}-
\setlength{\parskip}{1ex}
      \textbf{Parameters}
      \vspace{-1ex}

      \begin{quote}
        \begin{Ventry}{xxxxx}

          \item[label]


text of label
            {\it (type=str)}

        \end{Ventry}

      \end{quote}

\textbf{Note:} 
e.g. fl\_remoe\_selector\_appbutton(``SomeButton'')


\textbf{Status:} 
Tested + Doc + NoDemo = OK


    \end{boxedminipage}

    \label{xformslib:flgoodies:fl_disable_fselector_cache}
    \index{xformslib \textit{(package)}!xformslib.flgoodies \textit{(module)}!xformslib.flgoodies.fl\_disable\_fselector\_cache \textit{(function)}}

    \vspace{0.5ex}

\hspace{.8\funcindent}\begin{boxedminipage}{\funcwidth}

    \raggedright \textbf{fl\_disable\_fselector\_cache}(\textit{yesno})

    \vspace{-1.5ex}

    \rule{\textwidth}{0.5\fboxrule}
\setlength{\parskip}{2ex}

Disable file selector caching.

-{}-
\setlength{\parskip}{1ex}
      \textbf{Parameters}
      \vspace{-1ex}

      \begin{quote}
        \begin{Ventry}{xxxxx}

          \item[yesno]


flag. Values 0 (to enable cache) or 1 (to disable cache)
            {\it (type=int)}

        \end{Ventry}

      \end{quote}

\textbf{Note:} 
e.g. fl\_disable\_fselector\_cache(1)


\textbf{Status:} 
Tested + Doc + NoDemo = OK


    \end{boxedminipage}

    \label{xformslib:flgoodies:fl_invalidate_fselector_cache}
    \index{xformslib \textit{(package)}!xformslib.flgoodies \textit{(module)}!xformslib.flgoodies.fl\_invalidate\_fselector\_cache \textit{(function)}}

    \vspace{0.5ex}

\hspace{.8\funcindent}\begin{boxedminipage}{\funcwidth}

    \raggedright \textbf{fl\_invalidate\_fselector\_cache}()

    \vspace{-1.5ex}

    \rule{\textwidth}{0.5\fboxrule}
\setlength{\parskip}{2ex}

Forces an update of file selector caching programmatically. It forces
it only once, and on the directory that is to be browsed.

-{}-
\setlength{\parskip}{1ex}
\textbf{Note:} 
e.g. fl\_invalidate\_fselector\_cache()


\textbf{Status:} 
Tested + Doc + NoDemo = OK


    \end{boxedminipage}

    \label{xformslib:flgoodies:fl_get_fselector_form}
    \index{xformslib \textit{(package)}!xformslib.flgoodies \textit{(module)}!xformslib.flgoodies.fl\_get\_fselector\_form \textit{(function)}}

    \vspace{0.5ex}

\hspace{.8\funcindent}\begin{boxedminipage}{\funcwidth}

    \raggedright \textbf{fl\_get\_fselector\_form}()

    \vspace{-1.5ex}

    \rule{\textwidth}{0.5\fboxrule}
\setlength{\parskip}{2ex}

Obtains the form of file selector.

-{}-
\setlength{\parskip}{1ex}
\textbf{Note:} 
e.g. pform = fl\_get\_fselector\_form()


\textbf{Status:} 
Tested + Doc + NoDemo = OK


    \end{boxedminipage}

    \label{xformslib:flgoodies:fl_get_fselector_fdstruct}
    \index{xformslib \textit{(package)}!xformslib.flgoodies \textit{(module)}!xformslib.flgoodies.fl\_get\_fselector\_fdstruct \textit{(function)}}

    \vspace{0.5ex}

\hspace{.8\funcindent}\begin{boxedminipage}{\funcwidth}

    \raggedright \textbf{fl\_get\_fselector\_fdstruct}()

    \vspace{-1.5ex}

    \rule{\textwidth}{0.5\fboxrule}
\setlength{\parskip}{2ex}

Obtains a FD\_FSELECTOR class instance, allowing direct access to the
individual objects of a file selector.

-{}-
\setlength{\parskip}{1ex}
      \textbf{Return Value}
    \vspace{-1ex}

      \begin{quote}

file selector class instance
      {\it (type=pointer to xfdata.FD\_FSELECTOR)}

      \end{quote}

\textbf{Note:} 
e.g. fdfsel = fl\_get\_fselector\_fdstruct()


\textbf{Status:} 
Tested + Doc + NoDemo = OK


    \end{boxedminipage}

    \label{xformslib:flgoodies:fl_hide_fselector}
    \index{xformslib \textit{(package)}!xformslib.flgoodies \textit{(module)}!xformslib.flgoodies.fl\_hide\_fselector \textit{(function)}}

    \vspace{0.5ex}

\hspace{.8\funcindent}\begin{boxedminipage}{\funcwidth}

    \raggedright \textbf{fl\_hide\_fselector}()

    \vspace{-1.5ex}

    \rule{\textwidth}{0.5\fboxrule}
\setlength{\parskip}{2ex}

Hides a file selector.

-{}-
\setlength{\parskip}{1ex}
\textbf{Note:} 
e.g. fl\_hide\_fselector()


\textbf{Status:} 
Tested + Doc + NoDemo = OK


    \end{boxedminipage}

    \label{xformslib:flgoodies:fl_set_fselector_filetype_marker}
    \index{xformslib \textit{(package)}!xformslib.flgoodies \textit{(module)}!xformslib.flgoodies.fl\_set\_fselector\_filetype\_marker \textit{(function)}}

    \vspace{0.5ex}

\hspace{.8\funcindent}\begin{boxedminipage}{\funcwidth}

    \raggedright \textbf{fl\_set\_fselector\_filetype\_marker}(\textit{dirmk}, \textit{fifomk}, \textit{sockmk}, \textit{cdevmk}, \textit{bdevmk})

    \vspace{-1.5ex}

    \rule{\textwidth}{0.5\fboxrule}
\setlength{\parskip}{2ex}

Changes the prefix by which the listing of files in a directory special
files are marked with in browser. By default D is used for directories, p
for pipes etc.)

-{}-
\setlength{\parskip}{1ex}
      \textbf{Parameters}
      \vspace{-1ex}

      \begin{quote}
        \begin{Ventry}{xxxxxx}

          \item[dirmk]


marker character for directories
            {\it (type=int or char)}

          \item[fifomk]


marker for pipes and FIFOs
            {\it (type=int or char)}

          \item[sockmk]


marker for sockets
            {\it (type=int or char)}

          \item[cdevmk]


marker for character device files
            {\it (type=int or char)}

          \item[bdevmk]


marker character for block device files
            {\it (type=int or char)}

        \end{Ventry}

      \end{quote}

\textbf{Note:} 
e.g. fl\_set\_fselector\_filetype\_marker('d', 'P', 'S', 'V', 'b')


\textbf{Status:} 
Tested + Doc + NoDemo = OK


    \end{boxedminipage}

    \label{xformslib:flgoodies:fl_set_fselector_title}
    \index{xformslib \textit{(package)}!xformslib.flgoodies \textit{(module)}!xformslib.flgoodies.fl\_set\_fselector\_title \textit{(function)}}

    \vspace{0.5ex}

\hspace{.8\funcindent}\begin{boxedminipage}{\funcwidth}

    \raggedright \textbf{fl\_set\_fselector\_title}(\textit{title})

    \vspace{-1.5ex}

    \rule{\textwidth}{0.5\fboxrule}
\setlength{\parskip}{2ex}

Sets the title of a file selector.

-{}-
\setlength{\parskip}{1ex}
      \textbf{Parameters}
      \vspace{-1ex}

      \begin{quote}
        \begin{Ventry}{xxxxx}

          \item[title]


title to be set
            {\it (type=str)}

        \end{Ventry}

      \end{quote}

\textbf{Note:} 
e.g. fl\_set\_fselector\_title(``My own title of F.S.)


\textbf{Status:} 
Tested + Doc + NoDemo = OK


    \end{boxedminipage}

    \label{xformslib:flgoodies:fl_goodies_atclose}
    \index{xformslib \textit{(package)}!xformslib.flgoodies \textit{(module)}!xformslib.flgoodies.fl\_goodies\_atclose \textit{(function)}}

    \vspace{0.5ex}

\hspace{.8\funcindent}\begin{boxedminipage}{\funcwidth}

    \raggedright \textbf{fl\_goodies\_atclose}(\textit{pFlForm}, \textit{vdata})

    \vspace{-1.5ex}

    \rule{\textwidth}{0.5\fboxrule}
\setlength{\parskip}{2ex}

\emph{todo}

-{}-
\setlength{\parskip}{1ex}
      \textbf{Parameters}
      \vspace{-1ex}

      \begin{quote}
        \begin{Ventry}{xxxxxxx}

          \item[pFlForm]


form
            {\it (type=pointer to xfdata.FL\_FORM)}

          \item[vdata]


user data to be passed to function; callback has to take care of
type check
            {\it (type=any type (e.g. 'None', int, str, etc..))}

        \end{Ventry}

      \end{quote}

      \textbf{Return Value}
    \vspace{-1ex}

      \begin{quote}

unused value (xfdata.FL\_IGNORE)
      {\it (type=int)}

      \end{quote}

\textbf{Note:} 
e.g. \emph{todo}


\textbf{Status:} 
Tested + NoDoc + NoDemo = OK


    \end{boxedminipage}


%%%%%%%%%%%%%%%%%%%%%%%%%%%%%%%%%%%%%%%%%%%%%%%%%%%%%%%%%%%%%%%%%%%%%%%%%%%
%%                               Variables                               %%
%%%%%%%%%%%%%%%%%%%%%%%%%%%%%%%%%%%%%%%%%%%%%%%%%%%%%%%%%%%%%%%%%%%%%%%%%%%

  \subsection{Variables}

    \vspace{-1cm}
\hspace{\varindent}\begin{longtable}{|p{\varnamewidth}|p{\vardescrwidth}|l}
\cline{1-2}
\cline{1-2} \centering \textbf{Name} & \centering \textbf{Description}& \\
\cline{1-2}
\endhead\cline{1-2}\multicolumn{3}{r}{\small\textit{continued on next page}}\\\endfoot\cline{1-2}
\endlastfoot\raggedright \_\-\_\-p\-a\-c\-k\-a\-g\-e\-\_\-\_\- & \raggedright \textbf{Value:} 
{\tt \texttt{'}\texttt{xformslib}\texttt{'}}&\\
\cline{1-2}
\end{longtable}

    \index{xformslib \textit{(package)}!xformslib.flgoodies \textit{(module)}|)}
